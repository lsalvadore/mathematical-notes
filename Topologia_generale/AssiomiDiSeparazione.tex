\section{Assiomi di separazione.}\label{AssiomiDiSeparazione}
\begin{Axiom}
	\TheoremName{Assiomi di separazione}
	Sia $(\TopologicalSpace,\Topology)$ uno spazio topologico.
	Si dice che $\TopologicalSpace$ \`e
	\begin{itemize}
		\item \Define{$T_0$}[$T_0$][spazio]
		quando per ogni coppia $(x_1,x_2) \in
		\TopologicalSpace$ esiste un aperto $\Open$ tale che
		$\And{x_1 \in \Open}{x_2 \notin \Open}$ o
		$\And{x_2 \in \Open}{x_1 \notin \Open}$; in questo
		caso $\TopologicalSpace$ si chiama
		\Define{spazio di Kolmogorov}[di Kolmogorov][spazio];
		\item \Define{$T_1$}[$T_1$][spazio]
		quando per ogni coppia $(x_1,x_2) \in
		\TopologicalSpace$ esiste due aperti $\Open_1$ e $\Open_2$
		tali che
		$\And{x_1 \in \Open_1}{x_2 \notin \Open_1}$ e
		$\And{x_2 \in \Open_2}{x_1 \notin \Open_2}$;
		\item \Define{$T_2$}[$T_2$][spazio]
		quando per ogni coppia $(x_1,x_2) \in
		\TopologicalSpace$ esiste due aperti disgiunti $\Open_1$ e
		$\Open_2$ tali che
		$\And{x_1 \in \Open_1}{x_2 \notin \Open_1}$ e
		$\And{x_2 \in \Open_2}{x_1 \notin \Open_2}$; in questo
		caso $\TopologicalSpace$ si chiama
		\Define{spazio di Hausdorff}[di Hausdorff][spazio];
		\item \Define{$T_3$}[$T_3$][spazio] quando per ogni
		$x \in \TopologicalSpace$ e chiuso $\Closed$ tale che
		$x \notin \Closed$ esistono due aperti disgiunti $\Open_1$
		e $\Open_2$ tali che $x \in \Open_1$ e $\Closed
		\subseteq \Open_2$;
		\item \Define{$T_4$}[$T_4$][spazio] quando per ogni
		dati due chiusi disgiunti $\Closed_1$ e $\Closed_2$ tale
		che $x \notin \Closed$ esistono due aperti disgiunti
		$\Open_1$ e $\Open_2$ tali che $\Closed_1 \subseteq
		\Open_1$ e $\Closed_2 \subseteq \Open_2$.
	\end{itemize}
\end{Axiom}
\begin{Definition}
	Sia $(\TopologicalSpace,\Topology)$ uno spazio topologico.
	$\TopologicalSpace$ si dice \Define{regolare}[regolare][spazio] se
	\`e $T_1$ e $T_3$.
\end{Definition}
