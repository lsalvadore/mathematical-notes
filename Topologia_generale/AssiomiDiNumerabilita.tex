\section{Assiomi di numerabilita.}\label{AssiomiDiNumerabilita}
\begin{Axiom}
	\TheoremName{Primo assioma di numerabilit\`a}
	Sia $(\TopologicalSpace,\Topology)$ uno spazio topologico.
	$\TopologicalSpace$ verifica il primo assioma di numerabilit\`a
	se ogni punto di $\TopologicalSpace$ ammette una base locale
	numerabile.
\end{Axiom}
\begin{Axiom}
	\TheoremName{Secondo assioma di numerabilit\`a}
	Sia $(\TopologicalSpace,\Topology)$ uno spazio topologico.
	$\TopologicalSpace$ verifica il secondo assioma di numerabilit\`a
	se ammette una base numerabile.
\end{Axiom}
\begin{Definition}
	Sia $(\TopologicalSpace,\Topology)$ uno spazio topologico.
	$\Class \subseteq \TopologicalSpace$ si dice
	\Define{denso}[densa][classe] quando interseca ogni aperto non
	vuoto.
\end{Definition}
\begin{Definition}
	Sia $(\TopologicalSpace,\Topology)$ uno spazio topologico.
	$\TopologicalSpace$ si dice
	\Define{separabile}[separabile][spazio] se esiste
	$\Class \subseteq \TopologicalBase$ denso e numerabile.
\end{Definition}
\begin{Theorem}
	Sia $(\TopologicalSpace,\Topology)$ uno spazio topologico e sia $x \in \TopologicalSpace$. Se $x$ ammette una base locale numerabile $(\LocalBase_n)_{n \in \mathbb{N}}$, allora $x$ ammette anche una base locale numerabile non crescente.
\end{Theorem}
\Proof Una base locale non crescente \`e data da $(\VarLocalBase_n)_{n \in \mathbb{N}}$, con $\VarLocalBase_n = \bigcap_{i \in n} \LocalBase_i$. \EndProof
