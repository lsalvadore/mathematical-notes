\section{Basi e ricoprimenti.}\label{BasiERicoprimenti}
\begin{Definition}
	Sia $(\TopologicalSpace,\Topology)$ uno spazio e sia, per
	un'opportuna classe $I$, $(\TopologicalBase_i)_{i \in I}$ una
	famiglia di sottoclassi di $\TopologicalSpace$.
	$(\TopologicalBase_i)_{i \in I}$ si dice
	\begin{itemize}
		\item \Define{ricoprimento} di $\TopologicalSpace$ quando
		$\bigcup_{i \in I} \TopologicalBase_i =
		\TopologicalSpace$;
		\item \Define{ricoprimento aperto}[aperto][ricoprimento]
		quando $(\TopologicalBase_i)_{i \in I}$ \`e un
		ricoprimento e, per ogni $i \in I$, $\TopologicalBase_i$
		\`e aperto;
		\item \Define{ricoprimento chiuso}[chiuso][ricoprimento]
		quando $(\TopologicalBase_i)_{i \in I}$ \`e un
		ricoprimento e, per ogni $i \in I$, $\TopologicalBase_i$
		\`e chiuso;
		\item \Define{base}[di uno spazio topologico][base] quando
		quando $(\TopologicalBase_i)_{i \in I}$ \`e un
		ricoprimento aperto e ogni aperto di $\Topology$ \`e
		unione di elementi di $(\TopologicalBase_i)_{i \in I}$.
	\end{itemize} 
\end{Definition}
\begin{Theorem}\label{th_BaseDiUnaTopologia}
	Sia $\TopologicalSpace$ una classe e sia
	$(\TopologicalBase_i)_{i \in I}$ una famiglia di parti di
	$\TopologicalSpace$ tali che
	\begin{itemize}
		\item $\bigcup_{i \in I} \TopologicalBase_i =
		\TopologicalSpace$;
		\item fissati $i, j \in I$, $\Implies{x \in
		\TopologicalBase_i \ cap \TopologicalBase_j}{\Exists{k \in
		I}{\And{x \in \TopologicalBase_k}{\TopologicalBase_k
		\subseteq \TopologicalBase_i \cap \TopologicalBase_j}}}$.
	\end{itemize}
	Allora $\TopologicalBase_i$ \`e base di una topologia per
	$\TopologicalSpace$.
\end{Theorem}
\Proof La topologia $\Topology$ in questione non pu\`o che essere quella i
cui elementi sono tutte e sole le unioni di elementi di
$(\TopologicalBase_i)_{i \in I}$: proviamo che si tratta effettivamente di
una topologia.
\par A tal fine basta provare che l'intersezione di due unioni di elementi
 di $(\TopologicalBase_i)_{i \in I}$ \`e ancora unione di elementi di
$(\TopologicalBase_i)_{i \in I}$. Abbiamo
$\left (\bigcup_{i \in I_1} \TopologicalBase_i \right ) \cap
\left (\bigcup_{i \in I_2} \TopologicalBase_i \right ) =
\bigcup_{(i,j) \in I_1 \times I_2}
(\TopologicalBase_i \cap \TopologicalBase_j) =
\bigcup_{(i,j) \in I_1 \times I_2}
\bigcup_{x \in (\TopologicalBase_i \cap \TopologicalBase_j)}
\TopologicalBase_{x,i,j}$, dove $\TopologicalBase_{x,i,j}$ \`e un elemento
di $(\TopologicalBase_i)_{i \in I}$ tale che
$x \in \TopologicalBase_{x,i,j}$ e $\TopologicalBase_{x,i,j} \subseteq
(\TopologicalBase_i \cap \TopologicalBase_j)$. \EndProof
\begin{Definition}
	Sia $(\TopologicalSpace,\Topology)$ uno spazio e sia
	$x \in \TopologicalSpace$. Una famiglia di intorni
	$(\Neighborhood_{x,i})_{i \in I} \in \Neighborhoods{x}^{I}$ si chiama
	\Define{base locale}[locale][base] o
	\Define{sistema fondamentale di intorni}[fondamentale di intorni][sistema]
	quando per ogni intorno $\Neighborhood_x \in \Neighborhoods{x}$ di $x$ esiste $i \in I$
	tale che $\Neighborhood_{x,i} \subseteq
	\Neighborhood_x$.
\end{Definition}
