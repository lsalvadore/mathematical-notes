\section{Definizioni e teoremi di base.}\label{DefinizioniETeoremiDiBase}
\begin{Definition}
	Sia $\TopologicalSpace$ una classe. Si dice che $\Topology$ \`e
	una \Define{topologia} di $\TopologicalSpace$ quando
	\begin{itemize}
		\item $\ForAll{\Open \in \Topology}{\Open \subseteq
		\TopologicalSpace}$;
		\item $\ForAll{\Part \subseteq \Topology}{\bigcup_{\Open
		\in \Part} \Open \in \Topology}$;
		\item $\ForAll{\Part \subseteq \Topology}{\Implies{\Cardinality{\Part}
		\leq \aleph_0}{{\bigcap_{\Open \in \Part} \Open \in
		\Topology}}}$;
	\end{itemize}
	La coppia $(\TopologicalSpace,\Topology)$ si chiama
	\Define{spazio topologico}[topologico][spazio] e ogni elemento di
	$\Topology$ si chiama \Define{aperto}.
\end{Definition}
\begin{Definition}
	Siano $(\TopologicalSpace_1,\Topology_1)$ e
	 $(\TopologicalSpace_2, \Topology_2)$ spazi topologici.
	 Un'applicazione
	 $\Continuous:  \TopologicalSpace_1 \rightarrow
	 \TopologicalSpace_2$ tale che
	$\Implies{\Open \in \Topology_2}{\Continuous^{-1}(\Open) \in
	 \Topology_1)}$ si dice \Define{continua}[continua][applicazione].
\end{Definition}
\begin{Theorem}
	Le applicazioni continue costituiscono una categoria.
\end{Theorem}
\Proof \`E sufficiente provare che la composizione di applicazioni
 continue \`e continua. Siano
\begin{itemize}
	\item $\Continuous_1: \TopologicalSpace_1 \rightarrow
	\TopologicalSpace_2$;
	\item $\Continuous_2: \TopologicalSpace_2 \rightarrow
	\TopologicalSpace_3$;
\end{itemize}
applicazioni continue tra gli spazi topologici
$(\TopologicalSpace_1,\Topology_1)$,
$(\TopologicalSpace_2,\Topology_2)$ e
$(\TopologicalSpace_3,\Topology_3)$.
Sia $\Open \in \Topology_3$. Abbiamo
$(\Continuous_2 \circ \Continuous_1)^{-1}(\Open) =
\Continuous_1^{-1}(\Continuous_2^{-1}(\Open)) \in \Topology_1$.
Dunque $\Continuous_2 \circ \Continuous_1$ \`e continua. \EndProof
\begin{Definition}
	Chiamiamo \Define{categoria degli spazi topologici}[degli spazi topologici][categoria]
	la categoria costituita da tutte le applicazioni continue, d'ora
	in poi denotata $\CatTopologicalSpace$. \`E inoltre definito il
	funtore dimenticante canonico $\Forgetful: \CatTopologicalSpace
	\rightarrow \CatClass$ che ad ogni spazio topologico
	$(\TopologicalSpace,\Topology)$ associa la classe
	$\TopologicalSpace$. 
\end{Definition}
\begin{Definition}
	Chiamiamo \Define{omeomorfismi} gli isomorfismi di
	$\CatTopologicalSpace$.
\end{Definition}
\begin{Definition}
	Sia $(\TopologicalSpace,\Topology)$ uno spazio topologico.
	Definiamo \Define{chiuso} una classe $\Closed \subseteq
	\TopologicalSpace$ tale che $\Compl{\Closed} \in \Topology$.
\end{Definition}
\begin{Theorem}
	Sia $(\TopologicalSpace,\Topology)$, uno spazio topologico.
	$\emptyset$ e $\TopologicalSpace$ sono aperti e chiusi.
\end{Theorem}
\Proof Basta provare che $\emptyset, \TopologicalSpace \in \Topology$.
Abbiamo $\bigcup_{\Part \in \emptyset} \Part = \emptyset$ e
$\bigcap_{\Part \in \emptyset} \Part = \TopologicalSpace$. \EndProof
\begin{Definition}
	Siano $(\TopologicalSpace_1,\Topology_1)$ e
	$(\TopologicalSpace_2,\Topology_2)$ spazi topologici. Sia inoltre
	$\Continuous: \TopologicalSpace_1 \rightarrow \TopologicalSpace_2$
	un'applicazione qualsiasi.
	Dato un punto $x \in \TopologicalSpace_1$ definiamo
	\Define{intorno} di $x$ una classe $\Neighborhood$ tale che
	\begin{itemize}
		\item $\Neighborhood \subseteq \TopologicalSpace_1$;
		\item $\Exists{\Open \in \Topology}{\And{x \in \Open}{\Open \subseteq
		\Neighborhood}}$.
	\end{itemize}
	Denoteremo $\Neighborhoods{x}[\Topology_1]$ la classe di tutti gli intorni di $x$ rispetto alla topologia $\tau_1$, omettendo la spcifica della topologia ogni qual volta essa sia chiara dal contesto.
	Inoltre, diciamo che $\Continuous$ \`e
	\Define{continua}[continua in un punto][applicazione] quando
	per ogni $\Neighborhood_{\Continuous(x)} \in \Neighborhoods{\Continuous(x)}$ intorno di
	$\Continuous(x)$ esiste $\Neighborhood_x \in \Neighborhoods{x}$ intorno di $x$ tale che
	$\Continuous(\Neighborhood_x) \subseteq \Neighborhood_{\Continuous{x}}$.
\end{Definition}
\begin{Definition}
	Siano
	\begin{itemize}
		\item $(\TopologicalSpace,\Topology)$ uno spazio
		 topologico;
		\item $\Part \subseteq \TopologicalSpace$;
		\item $x \in \TopologicalSpace$.
	\end{itemize}
	Il punto $x$ si dice
	\begin{itemize}
		\item \Define{interno}[interno][punto] a $\Part$ quando
		$\Part$ \`e un intorno di $x$;
		\item \Define{esterno}[esterno][punto] a $\Part$ quando
		$\Compl{\Part}$ \`e un intorno di $x$;
		\item \Define{aderente}[aderente][punto] o
		\Define{di frontiera}[di frontiera][punto] quando
		per ogni intorno $\Neighborhood \in \Neighborhoods{x}$ si verificano
		simultaneamente $\Neighborhood \cap \Part \neq \emptyset$
		e $\Neighborhood \cap \Compl{\Part} \neq \emptyset$.
	\end{itemize}
	Inoltre definiamo
	\begin{itemize}
		\item l'\Define{interno} di $\Part$, denotato
		 $\Interior{\Part}$, la classe di tutti i punti interni
		 di $\Part$;
		\item l'\Define{esterno} di $\Part$, denotato
		 $\Exterior{\Part}$, la classe di tutti i punti esterni
		 di $\Part$;
		\item la \Define{frontiera} o \Define{bordo} o ancora
		\Define{contorno} di $\Part$, denotato
		$\Boundary{\Part}$, la classe di tutti i punti di
		 frontiera di $\Part$.
	\end{itemize}
\end{Definition}
\begin{Theorem}
	Siano
	\begin{itemize}
		\item $(\TopologicalSpace,\Topology)$ uno spazio
		 topologico;
		\item $\Part \subseteq \TopologicalSpace$.
	\end{itemize}
	Le classi non vuote tra $\Interior{\Part}$, $\Exterior{\Part}$ e
	$\Boundary{\Part}$ costituiscono una partizione.
\end{Theorem}
\Proof Sia $x \in \TopologicalSpace$.
\begin{itemize}
	\item se $x \in \Interior{\Part}$, allora $x \in \Part$ per cui
	$\Compl{\Part}$ non pu\`o essere intorno di $x$. Inoltre
	$\Interior{\Part}$ \`e un intorno di $x$ disgiunto da
	$\Compl{\Part}$, dunque $x \notin \Boundary{\Part}$;
	\item se $x \in \Exterior{\Part}$, allora
	 $x \in \Compl{\Part}$ per cui
	$\Part$ non pu\`o essere intorno di $x$. Inoltre
	$\Exterior{\Part}$ \`e un intorno di $x$ disgiunto da
	$\Part$, dunque $x \notin \Boundary{\Part}$;
	\item se $x \in \Boundary{\Part}$, allora n\'e $\Part$ n\'e
	 $\Compl{\Part}$ possono essere intorni di $x$. \EndProof
\end{itemize}
\begin{Theorem}
	Siano
	\begin{itemize}
		\item $(\TopologicalSpace,\Topology)$ uno spazio
		 topologico;
		\item $\Part \subseteq \TopologicalSpace$.
	\end{itemize}
	$\Interior{\Part}$ \`e il pi\`u grande aperto sottoclasse di
	$\Part$.	
\end{Theorem}
\Proof Sia $\Open \in \Topology$ tale che $\Open \subseteq \Part$. Sia
$x \in \Open$. $\Part$ \`e intorno di $x$, dunque $x \in
\Interior{\Part}$. \EndProof
\begin{Definition}
	Siano
	\begin{itemize}
		\item $(\TopologicalSpace,\Topology)$ uno spazio
		 topologico;
		\item $\Part \subseteq \TopologicalSpace$.
	\end{itemize}
	Si definisce \Define{chiusura}	di $\Part$, denotata
	$\Closure{\Part}$, il pi\`u piccolo chiuso tale che
	$\Part \subseteq \Closure{\Part}$.
\end{Definition}
\begin{Theorem}
	Siano $(\TopologicalSpace,\Topology)$ uno spazio topologico e $\Part \subseteq \TopologicalSpace$.
	Abbiamo $\Closure{\Part} = \Part \cup \Boundary{\Part}$;
\end{Theorem}
\Proof Abbiamo $\Compl{\Part \cup \Boundary{\Part}} = \Exterior{\Part} =
\Interior{\Compl{\Part}}$, dunque $\Part \cup \Boundary{\Part}$ \`e un chiuso.
\par Sia $\Closed$ un chiuso tale che $\Part \subseteq \Closed$. Dunque
$\Compl{\Closed}$ \`e un aperto tale che $\Compl{\Closed}
\subseteq \Compl{\Part}$, dunque $\Compl{\Closed} \subseteq
\Interior{\Compl{\Part}} = \Compl{\Part \cup \Boundary{\Part}}$, da cui
$\Part \cup \Boundary{\Part} \subseteq \Closed$. \EndProof
