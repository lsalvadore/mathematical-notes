\section{Spazi vettoriali topologici}
\label{Topologia_SpaziVettorialiTopologici}
\begin{Definition}
	Sia $\LinearSpace$ un $\Field$-spazio vettoriale e $\Topology$ una topologia su $\LinearSpace$ tale che
	\begin{itemize}
		\item la somma di vettori $+: \LinearSpace^2 \rightarrow \LinearSpace$ sia continua;
		\item il prodotto per scalare $\cdot: \Field \times \LinearSpace \rightarrow \LinearSpace$ sia continuo.
	\end{itemize}
	Chiamiamo $(\LinearSpace,\Topology)$ \Define{spazio vettoriale topologico}[vettoriale topologico][spazio].
\end{Definition}
\begin{Definition}
	Sia $(\LinearSpace,\Topology)$ uno spazio vettoriale topologico.
	Una parte $\Part \subseteq \LinearSpace$ si dice \Define{limitata}[limitata][parte] se per ogni intorno $\Neighborhood \in \Neighborhoods{0}$ esiste $\Scalar \in \Field$ tale che $\Part \subseteq \Scalar \Neighborhood$. Un'applicazione a valori in $\LinearSpace$ si dice \Define{limitata}[limitata][applicazione] se \`e limitata la sua immagine.
\end{Definition}
