\section{Distribuzione binomiale.}
\label{VariabiliAleatorieNotevoli_DistribuzioneBinomiale}
\begin{Definition}
  Siano
  \begin{itemize}
    \item $(\SamplesSpace,\Events,\Probability)$ uno spazio di probabilit\`a;
    \item $\BernoulliProbability \in [0,1]$;
    \item $\BernoulliCoprobability = 1 - \BernoulliProbability$;
    \item $\BernoulliTrials \in \mathbb{N}$;
    \item $\RandomVariable: \SamplesSpace \rightarrow \mathbb{R}$
      una variabile aleatoria.
  \end{itemize}
  Diciamo che $\RandomVariable$ ha distribuzione di probabilit\`a
  \Define{binomiale}[di probabilit\`a binomiale][distribuzione]
  di parametri $\BernoulliTrials$ e $\BernoulliProbability$
  quando la sua distribuzione \`e
  \[
    \Probability(X = k) =
    \begin{cases}
      \binom{n}{k}
      \BernoulliProbability^k \BernoulliCoprobability^{n - k}\text{ se }
      k \in n + 1,\\
      0\text{ altrimenti}.
    \end{cases}
  \]
  Una variabile aleatoria con distribuzione di probabilit\`a binomiale si chiama
  \Define{variabile aleatoria binomiale}[aleatoria binomiale][variabile].
  \par Inoltre, se $n = 1$, allora chiamiamo la distribuzione
  \Define{bernoulliana}[di probabilit\`a bernoulliana][distribuzione]
  di parametro $\BernoulliProbability$ e analogamente la variabile aleatoria
  \Define{bernoulliana}[aleatoria bernoulliana][variabile].
\end{Definition}
\begin{Theorem}
  Sia $\RandomVariable$ una variabile aleatoria bernoulliana di parametro
  $\BernoulliProbability$. Abbiamo
  \[
    \ExpectedValue{\RandomVariable} = \BernoulliProbability.
  \]
\end{Theorem}
\Proof Denotate
\begin{itemize}
  \item $\ProbabilityDistribution{\RandomVariable}$ la distribuzione di
probabilit\`a di $\RandomVariable$;
  \item $\BernoulliCoprobability = 1 - \BernoulliProbability$;
\end{itemize}
abbiamo
\begin{align*}
  \ExpectedValue{\RandomVariable}
  &= \int \RandomVariable d\ProbabilityDistribution{\RandomVariable},\\
  &= 0 \cdot \BernoulliProbability^0 \BernoulliCoprobability^{1 - 0}
    + 1 \cdot \BernoulliProbability^1 \BernoulliCoprobability^{1 - 1},\\
  &= \BernoulliProbability.\text{ \EndProof}
\end{align*} 
\begin{Theorem}
  Una variabile aleatoria $\RandomVariable$ ha distribuzione di probabilit\`a
  binomiale di parametri
  $\BernoulliTrials$ e
  $\BernoulliProbability$
  se e solo se esistono variabili aleatorie bernoulliane indipendenti
  $(\RandomVariable_i)_{i \in \BernoulliTrials}$
  di parametro
  $\BernoulliProbability$
  definite sullo stesso spazio di probabilit\`a tali che
  $\RandomVariable = \sum_{i \in BernoulliTrials} \RandomVariable_i$
\end{Theorem}
\begin{Corollary}
  Sia $\RandomVariable$ una variabile aleatoria binomiale di parametri
  $\BernoulliTrials$ e $\BernoulliProbability$. Abbiamo
  \[
    \ExpectedValue{\RandomVariable} = \BernoulliTrials\BernoulliProbability.
  \]
\end{Corollary}
\Proof Con le notazioni del teorema precedente, abbiamo
$\ExpectedValue{\RandomVariable}
= \sum_{i \in \BernoulliTrials} \ExpectedValue{\RandomVariable_i}
= \BernoulliTrials\BernoulliProbability$. \EndProof
