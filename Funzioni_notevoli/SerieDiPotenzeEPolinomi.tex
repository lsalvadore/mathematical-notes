\section{Serie di potenze e polinomi}
\label{FunzioniNotevoli_SerieDiPotenzePolinomi}
\begin{Theorem}
	Sia $n \in \mathbb{N}$, sia $\Monomial{m}(x) = x^n \in \RingAdjunction{\mathbb{C}}{x}$. $\Monomial{m}$ \`e derivabile ovunque in $\mathbb{C}$ e abbiamo
	\[
		\Derivative{\Monomial}(x) =
		\begin{cases}
			x^{n - 1}&\text{ se }n > 0,\\
			0&\text{ altrimenti}.
		\end{cases}
	\]
\end{Theorem}
\Proof Per $n > 1$, abbiamo
\begin{align*}
	\Derivative{\Polynomial}
	&= \lim_{h \rightarrow 0} \frac{(x + h)^n - x^n}{h},\\
	&= \lim_{h \rightarrow 0} \frac{\sum_{k = 1}^n \binom{n}{k} x^{n - k}h^k}{h},\\
	&= \lim_{h \rightarrow 0} \left ( \sum_{k = 1}^n \binom{n}{k} x^{n - k}h^{k - 1} \right ),\\
	&= n x^{n - 1}.
\end{align*}
\par Se invece $n = 0$, allora $\Monomial{m}$ \`e costante e dunque $\Derivative{\Monomial{m}} = 0$. \EndProof
\begin{Corollary}
	Fissato $n \in \mathbb{N}$, sia $\Polynomial(x) = \sum_{k \in n} a_k x^k \in \RingAdjunction{C}{x}$. $\Polynomial$ \`e derivabile ovunque in $\mathbb{C}$ e abbiamo
	\[
		\Derivative{\Polynomial}(x) =
		\begin{cases}
			\sum_{k \in \NotZero{n - 1}} k a_k x^{n - 1}&\text{ se }n > 0,\\
			0&\text{ altrimenti}.
		\end{cases}
	\]
\end{Corollary}
\Proof Segue direttamente dal teorema precedente per la linearit\`a dell'operatore di derivazione. \EndProof
\begin{Corollary}
	Sia $s(X) = \sum_{k \in \mathbb{N}} a_k X^k \in \PowerSeries{\mathbb{C}}{X}$. Se $\xi \in \mathbb{C}$ appartiene al cerchio di convergenza, allora $s(X)$ \`e derivabile in $\xi$ e
	\[
	\Derivative{s}(\xi) = \sum_{k \in \NotZero{\mathbb{N}}} n a_k \xi^{k - 1},
	\]
	vale a dire, il valore della derivata di $s$ in $\xi$ coincide col valore della derivata formale di $s$ in $\xi$.
\end{Corollary}
\Proof Segue direttamente dal teorema \ref{SuccessioniESerie_ConvergenzaDerivataFormale} e dal teorema precedente. \EndProof
\begin{Theorem}
	Siano
  \begin{itemize}
    \item $n \in \NotZero{\mathbb{N}}$;
    \item $s(X) = \sum_{k \in \mathbb{N}} a_k X^k
                \in \PowerSeries{\mathbb{C}}{X}$;
    \item $\RadiusOfConvergence \in \RealNonNegative$ il raggio di convergenza
      di $s(X)$;
    \item $\Matrix \in \mathbb{C}^{n \times n}$.
  \end{itemize}
  Se $\SpectralRadius{\Matrix} < \RadiusOfConvergence$, allora la serie
  $\sum_{k \in \mathbb{N}} a_k \Matrix^k$ \`e convergente.
\end{Theorem}
