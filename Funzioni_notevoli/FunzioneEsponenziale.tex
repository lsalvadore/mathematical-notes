\section{Funzione esponenziale.}
\label{FunzioniNotevoli_FunzioneEsponenziale}
\begin{Theorem}
	Sia
  $\Exponential(X) =
  \sum_{n \in \mathbb{N}} \frac{X^n}{n!}
  \in \PowerSeries{\mathbb{C}}{X}$. 
  $\Exponential(X)$ converge ovunque in $\mathbb{C}$.
\end{Theorem}
\begin{Definition}
	La funzione $\Exponential: \mathbb{C} \rightarrow \mathbb{C}$ che a
  $z \in \mathbb{C}$ associa la somma della serie
  $\Exponential(z) \in \mathbb{C}$ si chiama
  \Define{esponenziale}.
  Chiamiamo $\Exponential(1)$
  \Define{numero di Nepero}[di Nepero][numero],
  denotato $e$.
  Denoteremo il valore della funzione esponenziale in un punto
  $z \in \mathbb{C}$ come $\Exponential(z)$ o anche $e^z$.
\end{Definition}
\begin{Theorem}
	L'esponziale verifica la relazione
	\[
		\ForAll{(z,w) \in \mathbb{C}^2}{\Exponential(z + w) = \Exponential(z)\Exponential(w)}.
	\]
\end{Theorem}
\Proof Sia $(z,w) \in \mathbb{C}^2$. Abbiamo
\begin{align*}
	\Exponential(z + w)
	&= \sum_{n \in \mathbb{N}} \frac{(z + w)^n}{n!},\\
	&= \sum_{n \in \mathbb{N}} \sum_{k = 0}^n \frac{\binom{n}{k} z^k w^{n - k}}{n!},\\
	&= \sum_{n \in \mathbb{N}} \sum_{k = 0}^n \frac{n! z^k w^{n - k}}{)n - k)!k!n!},\\
	&= \sum_{n \in \mathbb{N}} \sum_{k = 0}^n \frac{z^k}{k!} \frac{w^(n - k)}{n - k!},\\
	&= \left ( \sum_{n \in \mathbb{N}} \frac{z^n}{n!} \right ) \left ( \sum_{n \in \mathbb{N}} \frac{w^n}{n!} \right ),\\
	&= \Exponential(z)\Exponential(w).\text{ \EndProof}
\end{align*}
\begin{Theorem}
  Abbiamo
  \[
    \frac{d\Exponential}{dz}(z) = \Exponential(z).
  \]
\end{Theorem}
\Proof Derivando la serie formale che definisce l'esponenziale otteniamo
\begin{align*}
  \frac{d\Exponential}{dz}(z)
    &= \sum_{n \in \NotZero{\mathbb{N}}} n \frac{z^{n - 1}}{n!},\\
    &= \sum_{n \in \NotZero{\mathbb{N}}} \frac{z^{n - 1}}{(n - 1)!},\\
    &= \sum_{n \in \mathbb{N}} \frac{z^n}{n!},\\
    &= \Exponential(z).\text{ \EndProof}
\end{align*}
\begin{Theorem}
	Sia
  $\Exponential(X) =
  \sum_{n \in \mathbb{N}} \frac{X^n}{n!}
  \in \PowerSeries{\mathbb{C}}{X}$. 
  Per ogni $n \in \NotZero{\mathbb{N}}$ e per ogni
  $\Matrix \in \mathbb{C}^{n \times n}$,
  $\Exponential(\Matrix)$ converge.
\end{Theorem}
\Proof Segue direttamente dal fatto il raggio spettrale di una matrice \`e
finito, mentre il raggio di convergenza di $\Exponential(X)$ \`e $+\infty$.
\EndProof
\begin{Definition}
	Fissato $n \in \NotZero{\mathbb{N}}$, la funzione
  $\Exponential: \mathbb{C}^{n \times n} \rightarrow \mathbb{C}^{n \times n}$
  che a
  $\Matrix \in \mathbb{C}^{n \times n}$ associa la somma della serie
  $\Exponential(\Matrix) \in \mathbb{C}^{n \times n}$
  si chiama
  \Define{esponenziale di matrice}[di matrice][esponenziale].
  Denoteremo il valore della funzione esponenziale di matrice in un punto
  $\Matrix \in \mathbb{C}^{n \times n}$ come $\Exponential(\Matrix)$ o
  anche $e^\Matrix$.
\end{Definition}
\begin{Theorem}
  Siano $A, B \in \mathbb{C}^{n \times n}$ ($n \in \NotZero{\mathbb{N}}$) tali che $AB = BA$.
  Allora $\Exponential(A + B) = \Exponential(B) \Exponential(C)$.
\end{Theorem}
\Proof Abbiamo, per il teorema del binomio di Newton,
\begin{align*}
  \Exponential(A + B)
  &= \sum_{k \in \mathbb{N}} \frac{(A + B)^k}{k!},\\
  &= \sum_{k \in \mathbb{N}} \frac{\sum_{l = 0}^k \binom{k}{l} A^lB^{k - l}}{k!},\\
  &= \sum_{k \in \mathbb{N}} \frac{\sum_{l = 0}^k \frac{k!}{l!(k - l)!} A^lB^{k - l}}{k!},\\
  &= \sum_{k \in \mathbb{N}} \sum_{l = 0}^k \frac{A^l}{l!}\frac{B^{k - l}}{(k - l)!},\\
  &= \left ( \sum_{l \in \mathbb{N}} \frac{A^l}{l!} \right )
      \left ( \sum_{k \in \mathbb{N}} \frac{B^k}{k!} \right ),\\
  &= \Exponential(A) \Exponential(B).\text{ \EndProof}
\end{align*}
\begin{Theorem}
  Siano
  \begin{itemize}
    \item $n \in \NotZero{\mathbb{C}}$;
    \item $\Matrix \in \mathbb{C}^{n \times n}$;
    \item $\Norm{\cdot}$ una norma di matrice su $\mathbb{C}^{n \times n}$.
  \end{itemize}
  Vale la disuguaglianza
  \[
    \Norm{\Exponential{\Matrix}} \leq \Exponential{\Norm{\Matrix}}.
  \]
\end{Theorem}
\Proof Abbiamo
\begin{align*}
  \Norm{\Exponential{\Matrix}}
  &= \Norm{\sum_{k \in \mathbb{N}} \frac{\Matrix^k}{k!}},\\
  &\leq \sum_{k \in \mathbb{N}} \frac{\Norm{\Matrix}^k}{k!},\\
  &= \Exponential{\Norm{\Matrix}}.\text{ \EndProof}
\end{align*}
\begin{Theorem}
  Siano
  \begin{itemize}
    \item $n \in \NotZero{\mathbb{N}}$;
    \item $\Matrix \in \mathbb{C}^{n \times n}$;
    \item $f: \mathbb{C} \rightarrow \mathbb{C}^{n \times n}$ tale che
      $f(z) = \Exponential(z\Matrix)$.
  \end{itemize}
  Abbiamo $f'(z) = \Matrix \Exponential(z \Matrix)$.
\end{Theorem}
\Proof Abbiamo
\begin{align*}
  f'(z)
    &= \sum_{k \in \NotZero{\mathbb{N}}} k \frac{z^{k - 1} \Matrix^k}{k!},\\
    &= \Matrix \sum_{k \in \NotZero{\mathbb{N}}} \frac{z^{k - 1}\Matrix^{k - 1}}{(k - 1)!},\\
    &= \Matrix \sum_{k \in \mathbb{N}} \frac{z^k \Matrix^k}{k!},\\
    &= \Matrix \Exponential(z \Matrix).\text{ \EndProof}
\end{align*}
