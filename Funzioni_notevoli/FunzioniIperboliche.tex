\section{Funzioni iperboliche.}
\label{FunzioniNotevoli_FunzioniIperboliche}
\begin{Theorem}
	Siano
	\begin{itemize}
		\item $\cosh(X)
          = \sum_{n \in \mathbb{N}} \frac{X^{2n}}{(2n)!}
          \in \PowerSeries{\mathbb{C}}{X}$;
		\item $\sinh(X)
          = \sum_{n \in \mathbb{N}} \frac{X^{2n + 1}}{(2n + 1)!}
          \in \PowerSeries{\mathbb{C}}{X}$.
	\end{itemize}
	Abbiamo
	\begin{itemize}
		\item $\cosh(X)$ converge ovunque in $\mathbb{C}$;
		\item $\sinh(X)$ converge ovunque in $\mathbb{C}$.
	\end{itemize}
\end{Theorem}
\begin{Definition}
	Le funzioni $\cosh$ e $\sinh$ del teorema precedente si chiamano
  \Define{coseno iperbolico}[iperbolico][coseno] e
  \Define{seno}[iperbolico][seno] rispettivamente: le denoteremo sempre
  $\cosh$ e $\sinh$.
\end{Definition}
\par \`E molto comune l'abuso di notazione secondo cui, per
$z, \alpha \in \mathbb{C}$, denotiamo $\cosh^\alpha{z}$ e
$\sinh^\alpha{z}$ le quantit\`a $(\cosh{z})^\alpha$ e $(\sinh z)^\alpha$
rispettivamente. Analoghe considerazioni varranno per le ulteriori
funzioni iperboliche che introdurremo.
\par Noi adotteremo sempre questo abuso di linguaggio, salvo esplicite
indicazioni contrarie.
\begin{Theorem}
  Sia $z \in \mathbb{C}$. Abbiamo
  \begin{itemize}
    \item $\cosh z
            = \cos iz
            = \frac{e^z + e^{-z}}{2}$;
    \item $\sinh z
            = - i \sin iz
            = \frac{e^z - e^{-z}}{2}$.
  \end{itemize}
\end{Theorem}
\Proof Per il coseno iperbolico abbiamo
\begin{align*}
  \cosh z
  &= \sum_{n \in \mathbb{N}} \frac{z^{2n}}{(2n)!},\\
  &= \sum_{n \in \mathbb{N}} (-1)^{n} \frac{(iz)^{2n}}{(2n)!},\\
  &= \cos iz.
\end{align*}
\par Inoltre
\begin{align*}
  \frac{e^z + e^{-z}}{2}
  &= \frac{1}{2}
      \sum_{n \in \mathbb{N}} \frac{z^n + (-z)^n}{n!},\\
  &= \frac{1}{2}
      \sum_{n \in \mathbb{N}} \frac{2z^{2n}}{(2n)!},\\
  &= \sum_{n \in \mathbb{N}} \frac{z^{2n}}{(2n)!},\\
  &= \cosh z.
\end{align*}
\par Per il seno iperbolico abbiamo
\begin{align*}
  \sinh z
  &= \sum_{n \in \mathbb{N}} \frac{z^{2n + 1}}{(2n + 1)!},\\
  &= \sum_{n \in \mathbb{N}} (-1)^{n} i \frac{(iz)^{2n + 1}}{(2n + 1)!},\\
  &= - i \sin iz.
\end{align*}
\par Inoltre
\begin{align*}
  \frac{e^z - e^{-z}}{2}
  &= \frac{1}{2}
      \sum_{n \in \mathbb{N}} \frac{z^n - (-z)^n}{n!},\\
  &= \frac{1}{2}
      \sum_{n \in \mathbb{N}} \frac{2z^{2n + 1}}{(2n + 1)!},\\
  &= \sum_{n \in \mathbb{N}} \frac{z^{2n + 1}}{(2n + 1)!},\\
  &= \sinh z.\text{ \EndProof}
\end{align*}
\begin{Corollary}
	Sia $z \in \mathbb{C}$. Abbiamo $\cosh^2 z - \sinh^2 z = 1$.
\end{Corollary}
\Proof Abbiamo
\begin{align*}
\cosh^2 z - \sinh^2 z
&= \left ( \frac{e^z + e^{-z}}{2} \right)^2
  - \left ( \frac{e^z - e^{-z}}{2} \right)^2,\\
&= \frac{e^{2z} + 2 e^{z-z} + e^{-2z} - e^{2z} + 2 e^{z-z} - e^{-2z}}
    {4}
&= e^0,\\
&= 1.
\end{align*}
\begin{Definition}
  Definiamo \Define{tangente iperbolica}[iperbolica][tangente] la
  funzione
  $\tanh: \mathbb{C}
    \rightarrow \mathbb{C}$
  tale che $\tanh{z} = \frac{\sinh{z}}{\cosh{z}}$.
CONTROLLARE DOMINIO: quali sono gli zeri di cosh?
\end{Definition}
\begin{Theorem}
  Abbiamo
  \begin{itemize}
    \item $\frac{d\cosh}{dz}(z) = \sinh z$;
    \item $\frac{d\sinh}{dz}(z) = \cosh z$;
    \item $\frac{d\tanh}{dz}(z) = 1 - \tanh^2 z = \frac{1}{\cosh^2 z}$.
  \end{itemize}
\end{Theorem}
\Proof Derivando le serie formali che definiscono coseno iperbolico e
seno iperbolico, otteniamo
\begin{align*}
  \frac{d\cosh}{dz}(z)
    &= \sum_{n \in \NotZero{\mathbb{N}}} 2n \frac{z^{2n - 1}}{(2n)!},\\
    &= \sum_{n \in \NotZero{\mathbb{N}}} \frac{z^{2n - 1}}{(2n - 1)!},\\
    &= \sum_{n \in \mathbb{N}} \frac{z^{2n + 1}}{(2n + 1)!},\\
    &= \sinh z.
\end{align*}
e
\begin{align*}
  \frac{d\sinh}{dz}(z)
    &= \sum_{n \in \mathbb{N}} (2n + 1)\frac{z^{2n}}{(2n + 1)!},\\
    &= \sum_{n \in \mathbb{N}} \frac{z^{2n}}{(2n)!},\\
    &= \cosh z.
\end{align*}
\par Usando infine la formula della derivata di un rapporto di funzioni,
\begin{align*}
  \frac{d\tanh}{dz}(z)
    &= \frac{\cosh z \cosh z - \sinh z \sinh z}{\cosh^2 z},\\
    &= 1 - \tanh z,\\
    &= \frac{1}{\cosh^2 z}.\text{ \EndProof}
\end{align*}
