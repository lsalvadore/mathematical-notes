\section{Funzioni trigonometriche.}
\label{FunzioniNotevoli_FunzioniTrigonometriche}
\begin{Theorem}
	Siano
	\begin{itemize}
		\item $\cos(X)
          = \sum_{n \in \mathbb{N}} (-1)^n \frac{X^{2n}}{(2n)!}
          \in \PowerSeries{\mathbb{C}}{X}$;
		\item $\sin(X)
          = \sum_{n \in \mathbb{N}} (-1)^n \frac{X^{2n + 1}}{(2n + 1)!}
          \in \PowerSeries{\mathbb{C}}{X}$.
	\end{itemize}
	Abbiamo
	\begin{itemize}
		\item $\cos(X)$ converge ovunque in $\mathbb{C}$;
		\item $\sin(X)$ converge ovunque in $\mathbb{C}$.
	\end{itemize}
\end{Theorem}
\begin{Definition}
	Le funzioni $\cos$ e $\sin$ del teorema precedente si chiamano
  \Define{coseno} e \Define{seno} rispettivamente: le denoteremo sempre
  $\cos$ e $\sin$.
\end{Definition}
\par \`E molto comune l'abuso di notazione secondo cui, per
$z, \alpha \in \mathbb{C}$, denotiamo $\cos^\alpha{z}$ e
$\sin^\alpha{z}$ le quantit\`a $(\cos{z})^\alpha$ e $(\sin{z})^\alpha$
rispettivamente. Analoghe considerazioni varranno per le ulteriori
funzioni trigonometriche che introdurremo.
\par Noi adotteremo sempre questo abuso di linguaggio, salvo esplicite
indicazioni contrarie.
\begin{Theorem}
	Sia $z \in \mathbb{C}$. Abbiamo
  $e^z
  = e^\RealPart{z} (\cos \ImaginaryPart{z} + i \sin \ImaginaryPart{z})$.
\end{Theorem}
\Proof Abbiamo
$e^z
= e^{\RealPart{z} + i \ImaginaryPart{z}}
= e^{\RealPart{z}} e^{i \ImaginaryPart{z}}$.
\par Esaminiamo il fattore $e^{i \ImaginaryPart{z}}$: abbiamo
\begin{align*}
	e^{i \ImaginaryPart{z}}
	&= \sum_{n \in \mathbb{N}} \frac{i^n \ImaginaryPart{z}^n}{n!},\\
	&= \sum_{n \in \mathbb{N}}
	\left ( \frac{i^{4n} \ImaginaryPart{z}^{4n}}{(4n)!}
	+ \frac{i^{4n + 1} \ImaginaryPart{z}^{4n + 1}}{(4n + 1)!}
	+ \frac{i^{4n + 2} \ImaginaryPart{z}^{4n + 2}}{(4n + 2)!}
	+ \frac{i^{4n + 3} \ImaginaryPart{z}^{4n + 3}}{(4n + 3)!} \right ),\\
	&= \sum_{n \in \mathbb{N}}
	\left ( \frac{\ImaginaryPart{z}^{4n}}{(4n)!}
	+ \frac{i \ImaginaryPart{z}^{4n + 1}}{(4n + 1)!}
	+ \frac{- \ImaginaryPart{z}^{4n + 2}}{(4n + 2)!}
	+ \frac{- i \ImaginaryPart{z}^{4n + 3}}{(4n + 3)!} \right ),\\
	&= \left (
      \sum_{n \in \mathbb{N}} (-1)^n \frac{\ImaginaryPart{z}^{2n}}{(2n)!}
      \right )
      + i
      \left (
      \sum_{n \in \mathbb{N}}
        (-1)^n \frac{\ImaginaryPart{z}^{2n + 1}}{(2n + 1)!}
      \right ),\\
	&= \cos{\ImaginaryPart{z}} + i \sin{\ImaginaryPart{z}}.
  \text{ \EndProof}
\end{align*}
\begin{Corollary}
	Sia $z \in \mathbb{C}$. Abbiamo $\cos^2{z} + \sin^2{z} = 1$.
\end{Corollary}
\Proof Abbiamo $e^0 = 1$, ma anche
\begin{align*}
	e^0
	&= e^{z - z},\\
	&= e^z e^{-z},\\
	&= e^{\RealPart{z}}(\cos \ImaginaryPart{z} + i \sin \ImaginaryPart{z})
	  e^{- \RealPart{z}}(\cos \ImaginaryPart{z} - i \sin \ImaginaryPart{z})
	&= \cos^2{x} - i^2 \sin^2{x},\\
	&= \cos^2{x} + \sin^2{x}.
\end{align*}
\par Ne consegue $\cos^2{x} + \sin^2{x} = 1$. \EndProof
\begin{Definition}
  Definiamo \Define{tangente} la funzione
  $\tan: \mathbb{C}
    \SetMin \bigcup_{k \in \mathbb{Z}} \lbrace k\pi \rbrace
    \rightarrow \mathbb{C}$
  tale che $\tan{z} = \frac{\sin{z}}{\cos{z}}$.
CONTROLLARE DOMINIO: gli zeri di cos sono tutti reali?
\end{Definition}
\begin{Theorem}
  Siano $x, y \in \mathbb{R}$. Abbiamo
  \begin{itemize}
    \item $\cos{x + y} = \cos{x}\cos{y} - \sin{x}\sin{y}$;
    \item $\sin{x + y} = \sin{x}\cos{y} + \cos{x}\sin{y}$;
    \item $\tan{x + y} = \frac{\tan{x} + \tan{y}}{1 - \tan{x}\tan{y}}$.
  \end{itemize}
\end{Theorem}
\begin{Corollary}
  \TheoremName{Formule di duplicazione}[di duplicazione][formule]
  Sia $x \in \mathbb{R}$. Abbiamo
  \begin{itemize}
    \item $\cos{2x}
            = \cos^2{x} - \sin^2{x}
            = 1 - 2 \sin^2{x}
            = \cos^2{x} - 1$;
    \item $\sin{2x} = 2 \sin{x} \cos{x}$;
    \item $\tan{2x} = \frac{2 \tan{x}}{1 - \tan^2{x}}$.
  \end{itemize}
\end{Corollary}
\begin{Corollary}
  \TheoremName{Formule di bisezione}[di bisezione][formule]
  \begin{itemize}
    \item $\cos{\frac{x}{2}}
            = \pm \sqrt{\frac{1 + \cos{x}}{2}}$;
    \item $\sin{\frac{x}{2}}
            = \pm \sqrt{\frac{1 - \cos{x}}{2}}$;
    \item $\tan{\frac{x}{2}}
            = \pm \sqrt
                {\frac{1 - \cos{x}}{1 + \cos{x}}}
            = \frac{\sin{x}}{1 + \cos{x}}
            = \frac{1 - \cos{x}}{\sin{x}}$.
  \end{itemize}
\end{Corollary}
\begin{Theorem}
  \TheoremName{Formule di prostaferesi}[di prostaferesi][formule]
  Siano $x, y \in \mathbb{R}$, abbiamo
  \begin{itemize}
    \item $\sin x + \sin y
            = 2 \sin \frac{x + y}{2} \cos \frac{x - y}{2}$;
    \item $\sin x - \sin y
            = 2 \sin \frac{x - y}{2} \cos \frac{x + y}{2}$;
    \item $\cos x + \cos y
            = 2 \cos \frac{x + y}{2} \cos \frac{x - y}{2}$;
    \item $\cos x - \cos y
            = - 2 \sin \frac{x + y}{2} \sin \frac{x - y}{2}$;
  \end{itemize}
\end{Theorem}
\begin{Theorem}
  \TheoremName{Formule di Werner}[di Werner][formule]
  Siano $x, y \in \mathbb{R}$, abbiamo
  \begin{itemize}
    \item $\sin{x}\cos{y} = \frac{\sin(x + y) + \sin(x - y)}{2}$;
    \item $\cos{x}\cos{y} = \frac{\cos(x + y) + \cos(x - y)}{2}$;
    \item $\sin{x}\sin{y} = \frac{\cos(x - y) - \cos(x + y)}{2}$.
  \end{itemize}
\end{Theorem}
\begin{Theorem}
  Abbiamo
  \begin{itemize}
    \item $\frac{d\cos}{dz}(z) = - \sin z$;
    \item $\frac{d\sin}{dz}(z) = \cos z$;
    \item $\frac{d\tan}{dz}(z) = 1 - \tan^2 z = \frac{1}{\cos^2 z}$.
  \end{itemize}
\end{Theorem}
\Proof Derivando le serie formali che definiscono coseno e
seno otteniamo
\begin{align*}
  \frac{d\cos}{dz}(z)
    &= \sum_{n \in \NotZero{\mathbb{N}}} (-1)^n 2n \frac{z^{2n - 1}}{(2n)!},\\
    &= \sum_{n \in \NotZero{\mathbb{N}}} (-1)^n \frac{z^{2n - 1}}{(2n - 1)!},\\
    &= \sum_{n \in \mathbb{N}} (-1)^{n + 1} \frac{z^{2n + 1}}{(2n + 1)!},\\
    &= - \sum_{n \in \mathbb{N}} (-1)^n \frac{z^{2n + 1}}{(2n + 1)!},\\
    &= - \sin z.
\end{align*}
e
\begin{align*}
  \frac{d\sin}{dz}(z)
    &= \sum_{n \in \mathbb{N}} (-1)^n (2n + 1)\frac{z^{2n}}{(2n + 1)!},\\
    &= \sum_{n \in \mathbb{N}} (-1)^n \frac{z^{2n}}{(2n)!},\\
    &= \cos z.
\end{align*}
\par Usando infine la formula della derivata di un rapporto di funzioni,
\begin{align*}
  \frac{d\tan}{dz}(z)
    &= \frac{\cos z \cos z + \sin z \sin z}{\cosh^2 z},\\
    &= 1 - \tan z,\\
    &= \frac{1}{\cos^2 z}.\text{ \EndProof}
\end{align*}
