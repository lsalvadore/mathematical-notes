\section{Variabili aleatorie.}
\label{Probabilita_VariabiliAleatorie}
\begin{Definition}
  Siano
  \begin{itemize}
    \item $(\SamplesSpace,\Events,\Probability)$ uno spazio di
      probabilit\`a;
    \item $(\MeasureSpace,\SigmaAlgebra)$ uno spazio di misura.
  \end{itemize}
  Chiamiamo
  \Define{variabile $(\MeasureSpace,\SigmaAlgebra)$-aleatoria}[aleatoria][variabile] o
  \Define{$(\MeasureSpace,\SigmaAlgebra)$-casuale}[casuale][variabile] o ancora
  \Define{$(\MeasureSpace,\SigmaAlgebra)$-stocastica}[stocastica][variabile]
  un'applicazione misurabile
  $\RandomVariable: \SamplesSpace \rightarrow \MeasureSpace$.
\end{Definition}
\par  Ometteremo la specifica di $\MeasureSpace$ o $\SigmaAlgebra$ ogni
qualvolta essi saranno chiari dal contesto: in particolare, assumeremo molto
spesso
$\MeasureSpace = \mathbb{R}$
e $\SigmaAlgebra$ uguale alla $\sigma$-algebra di Borel di $\mathbb{R}$ munito
della topologia euclidea.
\begin{Definition}
  Siano
  \begin{itemize}
    \item $(\SamplesSpace,\Events,\Probability)$ uno spazio di
      probabilit\`a;
    \item $(\MeasureSpace,\SigmaAlgebra)$ uno spazio di misura;
    \item $\RandomVariable: \SamplesSpace \rightarrow \MeasureSpace$ una
      variabile aleatoria.
  \end{itemize}
  Chiamiamo
  \Define{distribuzione di probabilit\`a}[di probabilit\`a][distribuzione]
  o
  \Define{legge di probabilit\`a}[di probabilit\`a][legge]
  la misura immagine di $\Measure$ tramite $\RandomVariable$.
\end{Definition}
