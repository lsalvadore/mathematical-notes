\section{Probabilit\`a condizionata.}
\label{Probabilita_ProbabilitaCondizionata}
\begin{Theorem}
  Sia $(\ProbabilitySpace,\Events,\Probability)$ uno spazio di probabilit\`a e
  sia $\Event \in \Events$.
  Denotiamo
  L'applicazione $\VarProbability: \Events \rightarrow [0,1]$ che a
  $\VarEvent \in \Events$ associa
  $\frac{\Probability(\VarEvent \cap \Event)}{\Probability(\Event)}$ \`e una
  probabilit\`a sullo spazio misurabile $(\ProbabilitySpace,\Events)$.
\end{Theorem}
\Proof Abbiamo
$\VarProbability(\ProbabilitySpace)
= \frac{\Probability(\ProbabilitySpace \cap \Event)}{\Probability(\Event)}
= \frac{\Probability(\Event)}{\Probability(\Event)} = 1$.
\par Sia $(\Event_n)_{n \in \mathbb{N}}$ una famiglia numerabile di eventi due
a due disgiunti.
Abbiamo
\begin{align*}
  \VarProbability \left ( \bigcup_{n \in \mathbb{N}} \Event_n \right )
  &= \frac{\Probability
    \left ( \left ( \bigcup_{n \in \mathbb{N}} \Event_n \right )
      \cap \Event \right )}{
    \Probability(\Event)},\\
  &= \frac{\Probability
    \left ( \bigcup_{n \in \mathbb{N}} (\Event_n \cap \Event ) \right )}{
    \Probability(\Event)},\\
  &= \frac{\sum_{n \in \mathbb{N}}\Probability(\Event_n \cap \Event)}{
    \Probability(\Event)},\\
  &= \sum_{n \in \mathbb{N}}\VarProbability(\Event_n).\text{ \EndProof}
\end{align*}
\begin{Definition}
  Nelle ipotesi del teorema precedente, chiamiamo
  \Define{probabilit\`a condizionata}[condizionata][probabilit\`a]
  a $\Event$ la probabilit\`a $\VarProbability$.
  Denoteremo
  $\Probability(\VarEvent|\Event)$ la probabilit\`a condizionata a
  $\Event$ di $\VarEvent \in \Events$.
\end{Definition}
\begin{Theorem}
  \TheoremName{Teorema di Bayes}[di Bayes][teorema]
  Sia $(\ProbabilitySpace,\Events,\Probability)$ uno spazio di probabilit\`a e
  siano $\Event, \VarEvent \in \Events$ con
  $\Probability(\VarEvent) \neq 0$.
  Abbiamo
  \[
    \Probability(\Event|\VarEvent)
    = \frac{\Probability(\VarEvent|\Event) \Probability(\Event)}
      {\Probability(\VarEvent)}.
  \]
\end{Theorem}
\Proof Supponiamo $\Probability(\Event) \neq 0$. Abbiamo
\begin{align*}
  \Probability(\Event|\VarEvent)
  &= \frac{\Probability(\Event \cap \VarEvent)}{\Probability(\VarEvent)},\\
  &= \frac{\Probability(\VarEvent|\Event)\Probability(\Event)}
        {\Probability(\VarEvent)}.
\end{align*}
\par Supponiamo invece $\Probability(\Event) = 0$. Allora
$\Probability(\Event|\VarEvent)
= \frac{\Probability(\Event \cap \VarEvent)}{\Probability(\VarEvent)}
= 0$. \EndProof
