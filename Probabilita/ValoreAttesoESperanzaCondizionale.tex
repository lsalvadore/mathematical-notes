\section{Valore atteso e speranza condizionale.}\label{ValoreAttesoESperanzaCondizionale}
\begin{Definition}
	Si definisce
  \Define{valore atteso}[atteso][valore]
  della variabile aleatoria integrabile $\RandomVariable$ la quantit\`a
  $\ExpectationValue{X} = \int_\SamplesSpace X(\omega) d\Probability{\omega}$.
\end{Definition}
\begin{Definition}
	Due variabili aleatorie $\RandomVariable$ e $\VarRandomVariable$ su uno stesso
  spazio di probabilit\`a $(\ProbabilitySpace,\Events,\Probability)$ si dicono
  \Define{equivalenti}[aleatorie equivalenti][variabili]
  quando sono quasi certamente uguali:
  $\Probability(X = Y) = 1$.
\end{Definition}
\begin{Definition}
	Sia $X$ una variabile aleatoria integrabile per lo spazio $(\SamplesSpace,\SigmaAlgebra_1,\Probability)$ e sia $\SigmaAlgebra_2$ un'altra $\sigma$-algebra tale che $\SigmaAlgebra_2 \subseteq \SigmaAlgebra_1$. Qualsiasi variabile aleatoria $Y$ $\SigmaAlgebra_2$-misurabile tale che per ogni $A \in \SigmaAlgebra_2$ si abbia $\int_A Xd\Probability = \int_A Yd\Probability$ si chiama \Define{speranza condizionale di $X$ rispetto a $\SigmaAlgebra_2$} e si denota $Y = \ConditionalExpectationValue{X}{\SigmaAlgebra_2}$.
\end{Definition}
\begin{Theorem}
	Date una variabile aleatoria integrablie $X$ per lo spazio $(\SamplesSpace,\SigmaAlgebra_1,\Probability)$ e una $\sigma$-algebra $\SigmaAlgebra_2 \subseteq \SigmaAlgebra_1$ esiste una speranza condizionale di $X$ rispetto a $\SigmaAlgebra_2$ e tutte le speranze condizionali sono equivalenti.
\end{Theorem}
\begin{Definition}
	Siano $X$ una variabile aleatoria integrabile da $(\SamplesSpace_1,\SigmaAlgebra_1,\Probability)$ e $Y: (\SamplesSpace_1,\SigmaAlgebra_1) \rightarrow (\SamplesSpace_2,\SigmaAlgebra_2)$ una variabile aleatoria. Si chiama \Define{speranza condizionale di $X$ rispetto ad $Y$} la quantit\`a $\ConditionalExpectationValue{X}{Y} = \ConditionalExpectationValue{X}{Y^{-1}(\SigmaAlgebra_2)}$.
\end{Definition}
\begin{Theorem}
	La speranza condizionale gode delle seguenti propriet\`a:
	\begin{itemize}
		\item $\ExpectationValue{X} = \ExpectationValue{\ConditionalExpectationValue{X}{\SigmaAlgebra_2}}$;
		\item $\ConditionalExpectationValue{aX+Y}{\SigmaAlgebra_2} = a\ConditionalExpectationValue{X}{\SigmaAlgebra_2} + \ConditionalExpectationValue{Y}{\SigmaAlgebra_2}$;
		\item se $X \leq Y$ quasi certamente, allora $\ConditionalExpectationValue{X}{\SigmaAlgebra_2} \leq \ConditionalExpectationValue{Y}{\SigmaAlgebra_2}$;
		\item se $X$ \`e $\SigmaAlgebra_2$-misurabile, allora $\ConditionalExpectationValue{X}{\SigmaAlgebra_2} = X$;
		\item se $X$ \`e indipendente da $\SigmaAlgebra_2$, allora $\ConditionalExpectationValue{X}{\SigmaAlgebra_2} = \ExpectationValue{X}$;
		\item se $\SigmaAlgebra_3 \subseteq \SigmaAlgebra_2$, allora $\ConditionalExpectationValue{\ConditionalExpectationValue{X}{\SigmaAlgebra_2}}{\SigmaAlgebra_3} = \ConditionalExpectationValue{X}{\SigmaAlgebra_3}$;
		\item se $Y$ \`e $\SigmaAlgebra_2$-misurabile e limitata, allora $\ConditionalExpectationValue{XY}{\SigmaAlgebra_2} = Y\ConditionalExpectationValue{X}{\SigmaAlgebra_2}$;
		\item se $(X_n)_{n \in \mathbb{N}}$ \`e una successione crescente di variabili aleatorie convergente quasi certamente ad $X$, allora $(\ConditionalExpectationValue{X_n}{\SigmaAlgebra_2})_{n \in \mathbb{N}}$ converge quasi certamente a $\ConditionalExpectationValue{X}{\SigmaAlgebra_2}$.
	\end{itemize}
\end{Theorem}
