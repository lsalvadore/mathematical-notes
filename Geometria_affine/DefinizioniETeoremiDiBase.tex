\section{Definizioni e teoremi di base.}\label{DefinizioniETeoremiDiBase}
\begin{Definition}
	Siano $\AffineSpace$ una classe munita di un'azione destra $+$ di uno spazio vettoriale $\LinearSpace$ libera e transitiva: $(\AffineSpace,+)$ si chiama \Define{spazio affine}[affine][spazio] su $\LinearSpace$. Diremo anche che $\LinearSpace$ \`e \Define{associato}[vettoriale associato ad uno spazio affine][spazio] a $\AffineSpace$ e che $\AffineSpace$ \`e un $\LinearSpace$-spazio affine.
\end{Definition}
\begin{Theorem}
	Sia $\AffineSpace$ un $\LinearSpace$-spazio affine. Sia $P \in \AffineSpace$: l'applicazione $f: \LinearSpace \rightarrow \AffineSpace$ che a $\Vector \in \LinearSpace$ associa $P + \Vector$ \`e biettiva.
\end{Theorem}
\Proof L'iniettivit\`a di $f$ \`e conseguenza della libert\`a dell'azione di $\LinearSpace$, la suriettivit\`a della transitivit\`a. \EndProof
