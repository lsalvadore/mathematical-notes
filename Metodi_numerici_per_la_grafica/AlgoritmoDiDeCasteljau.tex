\section{Algoritmo di de Casteljau.}
\label{MetodiNumericiPerLaGrafica_AlgoritmoDiDeCasteljau}
\begin{Definition}
  \label{Definition_DeCasteljau}
  Sia $\AffineSpace$ uno spazio $\mathbb{R}$-affine e sia,
  fissato $n \in \mathbb{N}$,
  $(P_k)_{k \in n} \in \AffineSpace^n$.
  Chiamiamo
  \Define{algoritmo di de Casteljau}[di de Casteljau][algoritmo]
  l'algoritmo seguente
  \begin{enumerate}
    \item\label{DeCasteljau_1} si ponga $r = 1$ e, per ogni $k \in n$,
      \[
        \DeCasteljau{k}{0} = P_k;
      \]
    \item\label{DeCasteljau_2} per ogni $k \in n - r$ definiamo
      \[
        \DeCasteljau{k}{r}(t)
          = (1 - t)\DeCasteljau{k}{r - 1} + t\DeCasteljau{k + 1}{r - 1};
      \]
    \item\label{DeCasteljau_3} incrementiamo $r$ di $1$;
    \item\label{DeCasteljau_4} se $r \leq n - 1$ torniamo al punto
      \ref{DeCasteljau_2}, altrimenti l'algoritmo termina.
  \end{enumerate}
  Chiamiamo inoltre
  \begin{itemize}
    \item \Define{punti di controllo}[di controllo][punto]
      i punti $P_k$ al variare di $k \in n$;
    \item \Define{poligono di controllo}[di controllo][poligono]
      il poligono definito dai punti di controllo o, pi\`u frequentemente,
      la spezzata definita da $(P_k)_{k \in n}$;
    \item \Define{curva di B\'ezier}[di B\'ezier][curva]
      la curva $\DeCasteljau{0}{n - 1}$.
  \end{itemize}
\end{Definition}
\begin{listing}
	\insertcode{octave}{"Metodi_numerici_per_la_grafica/DeCasteljau.m"}
	\caption{Implementazione dell'algoritmo di De Casteljau in
  \LanguageName{octave}.}
\end{listing}
\begin{Theorem}
  Con le notazioni della definizione precedente, l'algoritmo di Casteljau costa
  \begin{itemize}
    \item $\frac{n(n + 1)}{2}$ combinazioni lineari;
    \item $\BigO{n}$ spazio.
  \end{itemize}
\end{Theorem}
\Proof Per ogni valore di $r$ vengono calcolate $n - r$ combinazioni lineari:
in totale abbiamo dunque
$\sum_{r = 1}^{n - 1} (n - r)
= n^2 - \sum_{r = 1}^{n - 1} r
= n^2 - \frac{n(n - 1)}{2}
= \frac{n^2 + n}{2}
= \frac{n(n + 1)}{2}$.
\par Per quanto riguarda lo spazio,
\begin{Theorem}
  L'algoritmo di Casteljau \`e numericamente stabile.
\end{Theorem}
\Proof Segue direttamente dalla stabilit\`a numerica delle combinazioni lineari.
\EndProof
\begin{Theorem}
  Sia $\BezierCurve$ una curva di B\'ezier generata dai punti di controllo
  $(P_k)_{k \in n}$ ($n \in \mathbb{N}$). $\BezierCurve$ passa da
  $P_0$ e $P_{n - 1}$.
\end{Theorem}
\Proof Costruiamo $\BezierCurve$ tramite l'algoritmo di Casteljau. Con le
notazioni della definizione \ref{Definition_DeCasteljau}, si dimostra
immediatamente per induzione che
\[
  \ForAll{r \in n}{\DeCasteljau{0}{r}(0) = P_0},
\]
e
\[
  \ForAll{r \in n}{\DeCasteljau{n - r - 1}{r}(1) = P_{n - 1}}.\text{ \EndProof}
\]
\begin{figure}
	\includegraphics[width=0.8\textwidth]{Metodi_numerici_per_la_grafica/Immagine_DeCasteljau.png}
	\centering
	\caption[Esempio di curva di B\'ezier]
  {In blu una curva di B\'ezier, corrispondente al poligono di controllo
  tracciato in rosso.}
\end{figure}
