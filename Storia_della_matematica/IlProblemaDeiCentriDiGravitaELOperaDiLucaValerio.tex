\subsubsection{Il problema dei centri di gravit\`a e l'opera di Luca Valerio.}\label{IlProblemaDeiCentriDiGravitaELOperaDiLucaValerio}
\par La \textit{traditio} archimedea ha portato con s\'e, fra gli altri, il tema della ricerca dei centri di gravit\`a: questo \`e uno dei primi temi affrontati dai matematici del Cinquecento.
\par Una delle figure pi\`u importanti in questo ambito \`e Luca Valerio (1553 - 1618), allievo di Clavio e dunque, indirettamente, di Maurolico. Egli affronta nel suo \textit{De centro gravitatis solidorum libri tres} il problema della ricerca dei centri di gravit\`a di tutti i solidi tradizionali (cubo, piramide, sfera, paraboloide...) e delle loro parti, che era stato risolto solo parzialmente da Archimede; Valerio per\`o guarda al problema in maniera completamente nuova: egli inventa un metodo generale.
\par Valerio parte dai lavori di Archimede e da quelli di Commandino e coglie la generalit\`a dei loro metodi: egli pertanto, per la prima volta nella Storia della matematica compie il passo di introdurre oggetti generali, definiti da una propriet\`a, che consenta di enunciare teoremi generali dai quali conseguono automaticamente i teoremi di Archimede e Commandino riguardo alle figure particolari da essi studiate. Valerio considera una classe di figure piane e una classe di solidi che definisce digradanti rispettivamente
\begin{itemize}
	\item \textit{circa diametrum}: sono le figure piane che presentano un diametro, cio\`e una retta che biseca qualsiasi corda della figura;
	\item \textit{circa axim}: si tratta dei solidi che presentano un asse di simmetria (la retta che congiunge i centri di due basi opposte o un vertice e il centro della base corrispondente) e che, quando vengono tagliati da un piano che scorre lungo quest'asse, presentano sezioni simili ma di area sempre minore.
\end{itemize}
\par Il passo successivo consiste nell'enunciare e dimostrare teoremi sulle figure digradanti: per esempio, un teorema afferma che due figure avente stesso asse e sezioni proporzionali hanno lo stesso centro di gravit\`a.
\par Chiaramente, queste sezioni anticipano il principio degli indivisibili di Cavalieri, tuttavia Valerio rimane ancorato alla tradizione e cerca nella geometria classica la giustificazione dei suoi metodi e inoltre rimane comunque legato all'antico problema dei centri di gravit\`a, non va oltre ad esso.
\par Valerio \`e anche considerato da alcuni il vero inventore del metodo di esaustione (non \`e forse un caso se ad inventare questo termine sar\`a Gr\'egoire de Saint-Vincent, anch'egli della scuola di Clavio). Come sappiamo, il metodo di esaustione \`e gi\`a stato usato spesso dagli antichi, e in particolare da Archimede, ma si tratta pi\`u di uno stile dimostrativo, da riadattare caso per caso, che un metodo generale. Questo invece \`e il metodo di esaustione di Valerio: per conoscere il rapporto tra due grandezze $A$ e $B$, basta approssimarle con due grandezze $E$ ed $F$ -- cio\`e scegliere due grandezze $E$ ed $F$ tali che $A - E$ e $B - F$ possano essere resi piccoli a piacere -- e si avr\`a che il rapporto cercato \`e lo stesso che intercorre tra $E$ ed $F$.
\par Accenniamo infine che, nonostante il problema dei centri di gravit\`a sia legato alla forma dei solidi coinvolti, Valerio inizia a staccarsi dal concetto di forma per concentrarsi sul concetto di quantit\`a, prendendosi libert\`a di comporre e scomporre le figure secondo necessit\`a, sebbene egli ritenga che questo metodo, essendo lontano dallo stile archimedeo che costituisce il suo riferimento, non sia legittimo, un po' come il \textit{Metodo meccanico} di Archimede (che Valerio non conosceva) era considerato solo un mezzo matematico euristico dal suo stesso autore.
\par Per il suo stile, vicino a quello degli antichi e in particolare di Archimede, Valerio pu\`o essere considerato l'ultimo geometra della misura, ma per i suoi contenuti innovativi egli \`e anche il primo dei moderni analisti: dai suoi lavori infatti prender\`a le mosse Bonaventura Cavalieri.
