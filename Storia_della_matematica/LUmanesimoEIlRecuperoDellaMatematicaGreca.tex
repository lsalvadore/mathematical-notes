\subsubsection{L'umanesimo e il recupero della matematica greca.}\label{LUmanesimoEIlRecuperoDellaMatematicaGreca}
\paragraph{Principi generali dell'umanesimo relativi alla matematica.} L'umanesimo matematico, che possiamo collocare grosso modo nell'Italia del XVI secolo, trae senz'altro le sue origini dal \textit{Liber Abaci} e dalle scuole d'abaco: \`e attraverso questi due elementi che nasce una nuova generazione di matematici, i quali, come gli artisti e letterati loro contemporanei (quando non sono artisti o letterati essi stessi) si sentono inevitabilmente attratti dal mondo classico. Si pongono dunque, per quanto riguarda l'ambito matematico, i seguenti problemi:
\begin{itemize}
	\item il reperimento di manoscritti greci;
	\item la loro traduzione;
	\item la loro comprensione.
\end{itemize}
\par Nella sottosezione \ref{LaTraditioDellOperaDiArchimedeDiApollonioEDiPappo} vedremo esempi concreti di come sono state affrontate queste problematiche; qui diamo delle linee generali.
\par Com'\`e noto il sapere dei greci non viene perduto con l'ascesa n\'e con la caduta della civilt\`a romana, bens\`i viene conservato dalla civilt\`a araba, che ne promuove lo studio. Gli arabi traducono molti lavori, ma il loro spirito \`e pragmatico, anche quando rivolto alla teoria, non certo umanistico: essi sono interessati esclusivamente a capire i testi, quindi a migliorarli aggiungendovi il proprio contributo, non certo a condurre ricerche filologiche. Questo spirito \`e comunque sufficiente perch\'e i califfi impieghino grandi sforzi per recuperare numerosi manoscritti, per esempio dall'impero bizantino, ne promuovano la traduzione, tipicamente in due fasi, affidandoli prima a traduttori siriaci (che conoscevano il greco), poi a traduttori arabi, e infine li raccolgano per lo studio in grandi biblioteche (famosa quella di Baghdad tra IX e XII secolo, sotto il califfato abbaside).
\par Accanto alle corti di Federico II di Svevia e poi di Carlo di Angi\`o che diedero grande impulso alla diffusione della cultura grecoaraba (abbiamo gi\`a citato i contatti di Fibonacci con Federico II), il principale centro di traduzione dei testi greci ed arabi \`e la penisola iberica: i signori arabi incoraggiano il lavoro scientifico e inevitabilmente il mondo cristiano ne viene attratto; emblematico in questo senso il caso di Gerbert d'Aurillac, il futuro papa Silvestro II, che cerca di diffondere gi\`a prima di Fibonacci il sistema numerico posizionale. Oltre a Gerbert d'Aurillac ricordiamo Adelardo di Bath, il gi\`a citato traduttore di Euclide. E vengono tradotti in latino moltissimi testi greci: gli \textit{Elementi} di Euclide, il \textit{De mensura circuli} di Archimede, l'\textit{Almagesto} di Tolomeo; accanto a questi si aggiungono opere originali arabe come i lavori di al-Kh\=arizm\={\i}.
\par Quindi, come si vede, le opere della matematica greca (e non solo) arrivano in Europa tradotti ben prima dell'umanesimo, ma come dovrebbero far capire il confronto tra Gerbert d'Aurillac e Fibonacci e la \textit{traditio} euclidea gi\`a esaminata, questo non \`e sufficiente affinch\'e la civilt\`a occidentale le capisca e le assimili: le scuole d'abaco costituiscono la premessa necessaria a questo particolare passaggio. Gi\`a nel caso degli \textit{Elementi} di Euclide abbiamo visto che \`e il maestro d'abaco Tartaglia a trovare per primo un equilibrio tra filologia e matematica dando cos\`i un contributo decisivo alla restituzione dell'opera. Nel seguito, descriviamo due personalit\`a protagoniste dell'umanesimo matematico: Commandino e Maurolico.
\paragraph{Federico Commandino (1509 - 1575).} Commandino \`e un umanista che frequenta le corti rinascimentali: viene contatto dal cardinale Cervini della biblioteca Vaticana affinch\'e egli spieghi il senso di due manoscritti compresi in un codice di Moerbeke\footnote{\CFR\ sottosezione \ref{LaTraditioDellOperaDiArchimedeDiApollonioEDiPappo}.} -- i \textit{Galleggianti} di Archimede e \textit{De analemmate}, un'opera astronomica di Tolomeo. Egli riesce a pubblicare la sua edizione dei \textit{Galleggianti} circa quindici anni pi\`u tardi nel 1565: in effetti Commandino si rende rapidamente conto di non avere le conoscenze necessarie per capire e quindi restituire correttamente il testo; egli si dedica dunque ad un lungo lavoro preliminare, prima di ricerca di manoscritti, poi di traduzione e comprensione, che porta alla pubblicazione dei seguenti lavori:
\begin{itemize}
	\item la \textit{Archimedis operis non nulla} (1558), che comprende la traduzione di numerosi lavori di Archimede;
	\item il \textit{Liber de centro gravitati solidorum} (1565), opera originale di Commandino, edita insieme ai \textit{Galleggianti}, volta a spiegare il risultato non dimostrato sul centro di gravit\`a del paraboloide;
	\item un'edizione dei primi quattro libri delle \textit{Coniche} di Apollonio (1566), comprendente una traduzione, un commento, lemmi di Pappo e due testi di Sereno (\textit{De sectione coni} e \textit{De sectione cylindri})\footnote{Non \`e per\`o la prima edizione a stampa: la prima \`e quella stampata a Venezia nel 1537, tradotta dal latino da Giovanni Battista Memmo. Anche Maurolico compie una traduzione dei primi quattro libri a cui aggiunge una ricostruzione dei due successivi: ci\`o avviene nell'anno 1547, ma la pubblicazione \`e del 1654.}.
\end{itemize}
\par Commandino usa la stessa metodologia per l'edizione del \textit{De analemmate} (1562), preceduto del \textit{Planisphaerium} di Tolomeo (1558)\footnote{La prima edizione latina \`e del 1507.}; egli sottolinea inoltre i legami dei problemi affrontati da Tolomeo con quelli della contemporanea prospettiva pittorica.
\par Commadino traduce anche l'opera di Pappo, che viene pubblicata postuma nel 1588 a spese di Francesco II duca di Urbino.
\par L'impostazione di Commandino \`e chiaramente filologica: nel \textit{Liber de centro gravitati solidorum}, di cui egli stesso \`e autore, si nota la volont\`a di rimanere molto vicino al modello archimedeo, modello che contribuisce dunque a imporre nella matematica rinascimentale.
\paragraph{Francesco Maurolico (1494 - 1575).} Maurolico \`e un matematico messinese con un approccio sensibilmente diverso da quello di Commandino o di Tartaglia, pi\`u vicino a quello del mondo arabo altomedievale: egli diffonde s\`i gli scritti dei matematici greci, ma in una forma nuova e riattualizzata. Lavora tutta la vita in Sicilia, effettuando solo pochi viaggi, rimanendo dunque estraneo ai principali centri umanistici, ci\`o che spiega la sua impostazione particolarmente originale.
\par Maurolico adotta un'impostazione enciclopedica: egli studia tutto il sapere matematico (e non solo) a sua disposizione. Emblematico del suo approccio \`e il suo lavoro su Archimede: egli comincia a studiare l'\textit{Equilibrio dei piani} gi\`a prima del 1529, ma solo nel 1544 ne pubblica una prima edizione che diventa il postumo \textit{Archimedis de momentis aequalibus ex traditione Francisci Maurolyci} (1628): si tratta di un'opera in quattro libri molto lontana dal lavoro originale di Archimede. Esso \`e cos\`i strutturato:
\begin{itemize}
	\item libro I: equilibrio, momento, legge della leva;
	\item libro II: centro di gravit\`a dei poligoni;
	\item libro III: centro di gravit\`a della parabola;
	\item libro IV: determinazione del centro di gravit\`a di alcuni solidi (nell'edizione postuma l'autore aggiunge il caso del paraboloide).
\end{itemize}
\par Il \textit{De momentis} mostra chiaramente che Maurolico individua nell'originale i principi teorici fondamentali (simmetrie, momenti) e li rifonde in maniera pi\`u unitaria e in un certo senso moderna.
\par Le opere di Maurolico purtroppo non sono stampate per molto tempo, ma esercitano comunque una certa influenza dopo la morte del loro autore grazie all'amico Cristoforo Clavio (1537 - 1612) che basa su di esse l'insegnamento matematico rivolto alla Compagnia di Ges\`u.
