\subsection{Archimede e la geometria della misura.}\label{ArchimedeELaGeometriaDiMisura}
\paragraph{Biografia di Archimede.} Poco si sa della vita di Archimede: la sua figura ci \`e stata tramandata in una forma pi\`u leggendaria che storica, ma sia i dati storici che quelli leggendari contribuiscono a descrivere il famoso matematico siracusano.
\par Archimede nasce nel 287 a.C. a Siracusa e muore nel 212 a.C nella stessa citt\`a: la data di morte, corrispondente alla presa di Siracusa da parte dei romani, \`e certa. Non \`e noto se Archimede fosse di umili o di nobili natali, ma una teoria interessante lo vede figlio di un astronomo di nome Fidia, il quale sarebbe stato verosimilmente il primo ad indirizzarlo verso gli studi matematici.
\par La giovent\`u di Archimede vede l'ascesa di Gerone come re di Siracusa, eletto alla carica nel 270 a.C. Si pu\`o pensare che il giovane Archimede fu ulteriormente sollecitato verso gli studi meccanici osservando macchine da guerra, per esempio nello scontro di Gerone coi mercenari Mamertini che scatur\`i poi nella prima guerra punica (264 a.C. - 241 a.C.), o i numerosi cantieri aperti nella citt\`a.
\par Alcuni elementi suggeriscono un viaggio di Archimede ad Alessandria d'Egitto. Innanzittutto, \`e evidente l'attrazione che l'importante centro culturale esercita su qualsiasi intellettuale dell'epoca. L'amicizia di Archimede con Conone di Samo, verosimilmente stretta in Sicilia quando Conone vi si reca per compiere alcune misure astronomiche, potrebbe aver spinto Archimede ad un viaggio in Egitto, dove Conone lavora alla corte di Tolomeo III Evergete intorno al 246-245 a.C. allo scopo di rintracciare per il sovrano il ``ricciolo'' o la ``chioma'' di Berenice, un insieme di stelle che la regina Berenice aveva dedicato agli d\`ei; dopo la morte di Conone, Archimede prosegue i contatti con Dositeo, anch'egli attivo ad Alessandria, ma sempre in forma epistolare dalla sua Siracusa. Da segnalare la notizia, riferita da Diodoro Siculo (I secolo a.C.), secondo cui Archimede avrebbe inventato (o perfezionato) proprio in Egitto la coclea, nota anche come pompa a spirale o vite di Archimede. Infine, altro indizio significativo in favore di un eventuale viaggio del giovane Archimede nella terra del Nilo \`e il destinatario del \textit{Metodo meccanico}, il suo testamento scientifico: Eratostene di Cirene (276 a.C. - 194 a.C.), direttore della biblioteca di Alessandria dal 245 a.C. al 204 a.C.
\par Dal 240 a.C, anno in cui Archimede dedica l'\textit{Arenario} a Gelone, figlio di re Gerone, appena associato al regno, vive verosimilmente a Siracusa, dove si svolgono alcuni anedotti pi\`u o meno leggendari. Il primo riguarda la \textit{Syrakosia}: una nave colossale, descritta da Ateneo di Naucrati (II secolo d.C - III secolo d.C.) come una citt\`a in movimento, che viene utilizzata nel 238 a.C, ribattezzata \textit{Alexandris}, per inviare ad Alessandria le risorse necessarie per combattere una grave carestia. La leggenda, riferita da Plutarco (46-48 d.C - 125-127 d.C.), narra che Gerone, dopo aver sentito Archimede affermare di essere capace di muovere qualsiasi peso, lo sfida a muovere una delle sue navi: Archimede vara la \textit{Syrakosia}, con l'equipaggio a bordo, senza sforzo con un suo marchingegno. \`E in quest'occasione che viene pronunciata la famosa frase ``Datemi un punto d'appoggio e vi sollever\`o il mondo!''
\par Altro famoso aneddoto \`e quello della corona, raccontato da Vitruvio (I secolo d.C.): Gerone, sospettando che un artigiano a cui ha commissionato una corona in oro lo abbia truffato tenendo per s\'e dell'oro sostituendolo con argento, chiede ad Archimede di verificare. Archimede non pu\`o rompere o fondere la corona perch\'e ormai \`e stata consacrata: si narra che l'idea risolutiva viene in mente al matematico siracusano mentre si trova ai bagni e che egli sia investito da un tale entusiasmo da correre a casa ancora nudo gridando ``Eureka!'' Quale metodo egli abbia effettivamente usato non \`e chiaro: il racconto di Vitruvio prevede l'immersione della corona in acqua e il calcolo del volume da essa spostata, confrontata con analoghe misure per una certa quantit\`a di oro e una certa quantit\`a d'argento. Il principio usato \`e verosimilmente il famoso principio che porta il nome di Archimede unito al calcolo di pesi specifici.
\par L'ultimo aneddoto riguarda la morte di Archimede avvenuta nel 212 a.C, nel corso della seconda guerra punica (218 a.C. - 202 a.C.). Siracusa, inizialmente schierata con Roma, vede affermarsi al suo interno un forte partito filocartaginese, di cui, secondo Livio, fa parte anche Gelone. Gelone muore nel 216 a.C, cosicch\'e alla morte di Gerone, nel 215 a.C, \`e Geronimo, figlio di Gelone, a prendere il potere: essendo Geronimo solo quindicenne gli vengono affiancati numerosi tutori. Geronimo, filocartaginese, viene presto ucciso (214 a.C.) dal partito filoromano e nel 212 a.C Roma, davanti all'instabilit\`a di Siracusa, decide di attaccarla: dell'attacco \`e incaricato Marco Claudio Marcello (ante 268 a.C. - 208 a.C.), mentre la difesa della citt\`a \`e affidata ad Archimede.
\par La famosa storia degli specchi che bruciano le navi dei romani \`e poco verosimile: gli specchi parabolici infatti non possono concentrare i raggi luminosi su grandi distanze ed \`e assurdo pensare che le navi romane rimangano ferme ad attendere di prendere fuoco; pi\`u probabile \`e l'uso di specchi come ``accendini'' da usare per proiettili incendiari o per scaldare l'acqua di ``cannoni a vapore''. Altre macchine sfruttate da Archimede per l'occasione sono potenti gru che tirano massi sulle navi romane, la \textit{manus ferrea} che afferra le navi per poi ributtarle violentemente in acqua.
\par Archimede muore al termine della difficile conquista della citt\`a da parte dei romani. Secondo Livio (59 a.C. - 17 d.C.), Archimede \`e ucciso da un soldato romano intento al saccheggio. Plutarco d\`a diverse versioni che prevedono tutte l'uccisione da parte di un soldato romano: in un caso il soldato scambierebbe le attrezzature matematiche con cui Archimede si starebbe recando da Marcello per oggeti di valore; in un altro, dopo aver intimato ad Archimede di seguirlo per portarlo da Marcello, lo ucciderebbe in uno scatto d'ira sentendosi rispondere che avrebbe dovuto attendere che il matematico finisse di lavorare al problema che lo stava occupando; nell'ultimo il soldato lo minaccerebbe subito e lo ucciderebbe ignorando la richiesta del grande scienziato di terminare la sua dimostrazione. Le fonti riferiscono che Marcello non vuol che Archimede venga ucciso, tuttavia una teoria moderna vede la morte di Archimede come la conseguenza naturale della sua guida a capo della difesa della citt\`a, verosimilmente dovuta, oltre che alle sue evidenti capacit\`a, anche ad una presa di posizione politica a favore di Cartagine.
\par Sulla tomba di Archimede viene posta una sfera inscritta in un cilindro col rapporto tra il volume dei due solidi inciso -- $\frac{2}{3}$ -- che \`e lo stesso rapporto fra le aree laterali, secondo le volont\`a dello stesso Archimede, che aveva considerato il teorema su questi due rapporti il suo risultato pi\`u importante.
\paragraph{Opere.} L'elenco che segue riporta le opere di Archimede che ci sono pervenute, nello stesso ordine proposto dal filologo danese Heiberg:
\begin{itemize}
	\item \textit{Sulla sfera e il cilindro}: opera in due libri indirizzata a Dositeo. Nel primo libro Archimede stabilisce che il volume della sfera \`e due terzi del volume di un cilindro ad essa circostritto che che la superficie della sfera \`e quattro cerchi massimi, nel secondo libro sfrutta i risultati del primo per risolvere problemi correlati, come la divisione di una sfera per mezzo di un piano in due parti aventi un rapporto dato;
	\item \textit{Misura del cerchio}: comprende tre proposizione. La prima prova l'equivalenza tra il cerchio e un triangolo rettangolo avente per cateti raggio e circonferenza rettificata, la terza fornisce un'approssimazione di $\pi$: $3 + \frac{10}{71} < \pi < 3 + \frac{1}{7}$;
	\item \textit{Sui conoidi e sferoidi}: indirizzata a Dositeo, usa risultati della geometria di misura su solidi di rivoluzione: paraboloide e iperbolide (conoidi), ellissoidi (sferoidi);
	\item \textit{Sulle spirali}: altra opera indirizzata a Dositeo, studia la spirale di Archimede;
	\item \textit{Sull'equilibrio dei piani}: opera in due libri. Nel primo libro si studiano le leve e si ricercano i centri di gravit\`a di alcune figure piane, mentre il secondo si occupa della ricerca di centri di gravit\`a del segmento di parabola;
	\item \textit{Arenario}: dedicata a Gelone di Siracusa, descrive un sistema di numerazione per contare quantit\`a molto grandi, per esempio di granelli di sabbia;
	\item \textit{Quadratura della parabola}: indirizzata a Dositeo, dimostra che il segmento di parabola equivale ai $\frac{4}{3}$ del triangolo avente stessa base e stessa altezza del segmento di parabola. La dimostrazione viene svolta in due modi, uno meccanico e un altro geometrico;
	\item \textit{Sui galleggianti}: opera in due libri. Nel primo libro Archimede enuncia il famoso principio che porta il suo nome e descrive la condizione di galleggiamento di un segmento di sfera, nel secondo studia il galleggiamento di un segmento di paraboloide;
	\item \textit{Stomachion}: affronta il problema di suddivedere un quadrato o un rettangolo in quattordici pezzi commensurabili;
	\item \textit{Sul metodo meccanico}: indirizzata a Eratostene. L'opera descrive, con esempi, i metodi euristici adottati da Archimede e studia i problemi dell'``unghia'' cilindrica e del solido che si ottiene da due cilindri inscritti in un cubo con le basi sulle facce;
	\item \textit{Libro dei lemmi}: studia figure ottenute intersecando cerchi. Ci \`e pervenuto solo attraverso una parafrasi in arabo;
	\item \textit{Il problema dei buoi}: indovinello sul conteggio di buoi suddivisi in quattro colori e sui quali sono date certe condizioni numeriche; comporta numeri decisamente elevati. Non conosciamo la soluzione di Archimede.
\end{itemize}
\paragraph{La geometria di misura.} Archimede, come si \`e visto, lavora in numerosi settori: uno di quelli pi\`u importanti \`e la cosidetta geometria di misura. La geometria di misura si occupa del calcolo di aree e volumi, che all'epoca di Archimede non siginifica esprimere aree e volumi in termini numerici, bens\`i esprimere aree e volumi in termini di rapporti con aree e volumi pi\`u semplici (e per questo considerati noti).
\par Il metodo di Archimede \`e molto sofisticato, ma vago: le idee di fondo sono sempre le stesse, ma l'applicazione \`e diversa a seconda del problema particolare.
\par Il punto di partenza \`e l'individuazione del rapporto che si desidera dimostrare. Per questo scopo Archimede compie un ragionamento informale che descrive nel \textit{Metodo}, di cui riportiamo l'esempio relativo al confronto tra paraboloide e cilindro, appoggiandoci alla figura \ref{Archimede_MetodoMeccanico}.
\begin{figure}
	\includegraphics[width=0.8\textwidth]{Storia_della_matematica/Immagine_Archimede_MetodoMeccanico.png}
	\centering
	\caption{Confronto tra paraboloide e cilindro secondo il \textit{Metodo meccanico}.}
	\label{Archimede_MetodoMeccanico}
\end{figure}
\par L'idea di Archimede \`e la seguente: immaginiamo un paraboloide (in rosso) e il cilindro ad esso circoscritto (in blu). Supponiamo di tagliare il cilindro e il paraboloide secondo un piano parallelo alle basi del cilindro: otteniamo due sezioni circolari. \`E facile dimostrare che il rapporto tra la sezione del paraboloide $P$ e quella del cilindro $C$ \`e lo stesso rapporto che intercorre tra il segmento $s$ di estremi il vertice del paraboloide e il centro delle sezioni concentriche e l'altezza $h$ del cilindro: $P:C = s: h$. Immaginiamo ora di traslare la sezione $P$ lungo l'asse del paraboloide per una distanza oltre il vertice pari all'altezza $h$, ottenendo il cerchio con circonferenza magenta. Se immaginiamo inoltre che l'asse del paraboloide sia una bilancia con fulcro nel vertice del paraboloide (il triangolo nero), sulla quale poniamo solo la sezione magenta e quella blu, avremo, per la legge delle leva (che Archimede aveva gi\`a dimostrato nell'\textit{Equilibrio dei piani}: la leva \`e in equilibrio quando il braccio della forza sta al braccio della resistenza come la resistenza sta alla forza), la bilancia in equilibrio. L'argomentazione si pu\`o ripetere per tutte le sezioni analoghe del paraboloide e del cilindro.
\par Ora, Archimede conosce anche il centro di massa del paraboloide, situato nel punto medio del suo asse finito: egli immagina dunque di traslare adesso l'intero paraboloide, portandone il centro di massa nello stesso punto in cui portava il centro delle sezioni circolari (paraboloide magenta). Per quanto dimostrato in precedenza, la bilancia, caricata col paraboloide traslato e con il cilindro, sar\`a in equilibrio (per le propriet\`a del baricentro, \`e come se avessimo caricato tutte le sezioni del paraboloide nell'estremo sinistro della bilancia, ciascuna delle quali controbilancia una sezione del cilindro), e dunque il rapporto tra i volumi deve essere lo stesso che intercorre tra i bracci della leva: il volume del paraboloide, il cui centro di massa si trova a distanza $h$ dal fulcro, \`e met\`a del volume del cilindro, il cui centro di massa si trova a distanza $\frac{h}{2}$ dal fulcro.
\par Il motivo per cui Archimede si serva di questo metodo solo come di un ``trucco'' per individuare il rapporto da dimostrare poi pi\`u rigorosamente e non consideri gi\`a quest'argomentazione come una dimostrazione risiede probabilmente nel passaggio, da noi posto tra parentesi, in cui egli considera i due solidi come somma delle infinite sezioni che li compongono: questo passaggio \`e inaccettabile per la matematica greca che non riesce a concepire il continuo. Un'altra motivazione pu\`o risiedere nello status inferiore della meccanica rispetto alla geometria.
\par Archimede procede dunque con la dimostrazione geometrica vera e propria (l'unica che compare nei suoi scritti: il metodo euristico \`e esposto solo nel \textit{Metodo meccanico}). L'idea generale, che illustriamo in linguaggio moderno, consiste nel costruire due successioni di figure $(I_n)_{n \in \mathbb{N}}$ e $(C_n)_{n \in \mathbb{N}}$, le prime inscritte alla figura $X$ che si desidera ``calcolare'', le seconde circoscritte, con le ulteriori propriet\`a che, per ogni $n \in \mathbb{N}$, si abbia $I_n < R < C_n$, dove $R$ \`e la figura nota di riferimento, e la succesione $(C_n - I_n)_{n \in \mathbb{N}}$ sia infinitesima. Si procede ora alla dimostrazione che $X = R$ per un doppio assurdo:
\begin{itemize}
	\item supponiamo $X < R$, allora per $n$ sufficientemente grande abbiamo $C_n - I_n < R - X$, dunque $C_n - I_n < C_n - X$ da cui $X < I_n$, che contraddice il fatto he $I_n$ sia inscritto in $X$;
	\item supponiamo $X > R$, allora per $n$ sufficientemente grande abbiamo $C_n - I_n < X - R$, dunque $C_n - I_n < X - I_n$ da cui $X > C_n$, che contraddice il fatto he $C_n$ sia circoscritto ad $X$.
\end{itemize}
\par Dunque non pu\`o che essere $X = R$. Questo metodo, chiamato impropriamente \Define{metodo di esaustione} (impropriamente perch\'e non \`e mai stato formalizzato come metodo), era gi\`a stato utilizzato da Eudosso di Cnido (fine V secolo a.C. - IV secolo a.C.) per ``calcolare'' l'area del cerchio, come riportato anche nel libro XII degli \textit{Elementi} di Euclide. Eudosso non segue esattamente la procedura sopradescritta e spesso non la segue neanche Archimede: quest'ultimo la applica per dimostrare che un paraboloide ha volume dato dalla met\`a del volume del cilindro circoscritto (lo stesso esempio che abbiamo trattato per l'illustrazione del metodo euristico), ma la modifca secondo le necessit\`a del caso, ci\`o che prova che il metodo di esaustione, pi\`u che un metodo, \`e un'``impostazione mentale''. Vediamo l'esempio della quadratura di un segmento di parabola $P$, basandoci sulla figura \ref{Archimede_QuadraturaParabola}.
\begin{figure}
	\includegraphics[width=0.8\textwidth]{Storia_della_matematica/Immagine_Archimede_QuadraturaParabola.png}
	\centering
	\caption{Quadratura della parabola.}
	\label{Archimede_QuadraturaParabola}
\end{figure}
\par Prima di affrontare la dimostrazione vera e propria, Archimede enuncia, senza dimostrazioni, alcuni risultati della teoria delle coniche preapolloniana:
\begin{itemize}
	\item i \Define{diametri} (le rette che bisecano un fascio parallelo di corde; i punti di intersezione tra i diametri e la parabola sono detti \Define{vertici}) sono tutti paralleli all'asse della parabola;
	\item fissato un diametro, i quadrati delle \Define{ordinate} (le semicorde individuate dal diametro) sono proporizionali ai segmenti di diametro aventi per estremi il vertice e il punto medio della corda in questione;
	\item fissato un diametro, le corde sono parallele alla tangente nel vertice.
\end{itemize}
\par Definiamo \Define{base} di un segmento di parabola la corda che lo delimita e \Define{altezza} il segmento della perpendicolare alla base passante per il vertice. La tesi, che Archimede intuisce grazie al suo metodo euristico\footnote{La \Define{Quadratura della parabola} \`e in effetti diviso in due parti: la dimostrazione geometrica che qui riportiamo \`e la seconda parte, mentre la prima parte era un dimostrazione meccanica, piuttosto oscura, probabilmente da collegare a quanto descritto nel \Define{Metodo meccanico}.}, \`e che un segmento di parabola \`e equivalente ai $\frac{4}{3}$ del triangolo con stessa base e stessa altezza.
\par Archimede procede inscrivendo nella parabola una successione di poligoni ottenuta tramite triangoli: il primo poligono $P_0$ (in nero) \`e il triangolo che ha base coincidente con quella della parabola e per terzo vertice il vertice del parabola, il secondo poligono $P_1$ (in rosso, pi\`u la base del triangolo nero) si ottiene dal primo costruendo nei due segmenti di parabola staccati da $P_0$ due nuovi triangoli con la stessa tecnica descritta in precedenza e ``saldandoli'' a $P_0$ e i poligoni successivi si costruiscono analogamente, inscrivendo triangoli nei nuovi segmenti di parabola e ``saldandoli''.
\par Ora, Archimede afferma che il triangolo $P_0$ \`e maggiore della met\`a del segmento parabolico: ci\`o si vede facilmente considerando il parallelogramma circoscritto al segmento di parabola, che \`e doppio di $P_0$ e maggiore del segmento di parabola. Quindi, con un'argomentazione un po' oscura, Archimede ne deduce che i poligoni $P_n$ approssimano il segmento di parabola $P$, cio\`e che la successione $P_n - P$ \`e infinitesima.
\par Ora, sfruttando le propriet\`a sopraelencate della parabola, si dimostra che, per $n > 0$,  la superficie $T_n = P_n - P_{n - 1}$ \`e $\frac{T_{n - 1}}{4}$. Ne risulta, posto $T_0 = P_0$ che $P_n = T_0 + \sum_{i = 0}^n T_i = \frac{4}{3}T_0 - \frac{T_0}{3 \cdot 4^n}$. Notiamo che $(T_n)_{n \in \mathbb{N}}$ \`e una successione infinitesima.
\par Segue la doppia riduzione all'assurdo: si noti che non abbiamo costruito nessuna successione di poligoni circoscritti e che utlizziamo ora due successioni infinitesime. Abbiamo
\begin{itemize}
	\item supponiamo $P > \frac{4}{3}T_0$, allora per $n$ sufficientemente grande abbiamo $P - P_n < P - \frac{4}{3}T_0$, da cui $P_n > \frac{4}{3}T_0$, ma questo \`e assurdo perch\'e abbiamo anche $P_n = \frac{4}{3}T_0 - \frac{T_0}{3 \cdot {1}{4^n}}$;
	\item supponiamo $P < \frac{4}{3}T_0$, allora per $n$ sufficientemente grande abbiamo $T_n < \frac{4}{3}T_0 - P$, dunque $\frac{4}{3}T_0 - P_n = \frac{T_n}{3} < T_n < \frac{4}{3}T_0 - P$, da cui $P < P_n$ che \`e assurdo essendo $P_n$ inscritto in $P$.
\end{itemize}
\par Dunque deve essere $P = \frac{4}{3}T_0$.
