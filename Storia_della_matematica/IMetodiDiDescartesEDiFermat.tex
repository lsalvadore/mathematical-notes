\subsection{Il problema delle tangenti: i metodi di Descartes e di Fermat.}\label{IMetodiDiDescartesEDiFermat}
\par Uno dei problemi pi\`u antichi \`e quello delle tangenti\footnote{\CFR\ sottosezione \ref{LaGeometriaDiPosizione}.}: nel corso del Seicento molti matematici trovano nuovi metodi per calcolare tangenti, ma i due principali, dai quali prendono le mosse quelli successivi sono quelli di Descartes e di Fermat.
\paragraph{Il metodo di Descartes.} Il metodo di Descartes consiste nel trovare una circonferenza tangente alla curva nel punto dato, cio\`e che abbia con la curva un'intersezione doppia; di solito la circonferenza viene cercata col centro sull'asse delle ascisse. Trovata la circonferenza, la tangente sar\`a la retta perpendicolare al raggio passante per il punto voluto.
\par Supponiamo di voler cercare la tangente nel punto $(x_0,y_0)$ alla curva $P(x,y) = 0$. La circonferenza ha equazione $(x - v)^2 + y^2 = r^2$, con $v$ e $r$ da determinare. Si mettono a sistema le due equazioni eliminando la $y$; se $P(x,y)$ \`e di grado $n$, allora l'equazione $Q(x) = 0$ che ne risulta \`e di grado $2n$ (e questo gi\`a suggerisce la complessit\`a dei calcoli previsti dal metodo). Si impone allora che $Q(x)$ sia della forma $(x - x_0)^2 R(x)$, con $R(x)$ polinomio di grado $2n - 2$ e eguagliando i coefficienti delle potenze di $x$ si trovano $v$ ed $r$.
\par Diamo un esempio di applicazione nel caso della parabola $y = mx^2$, mostrando quanto complicati possano essere i calcoli anche nei casi pi\`u semplici. Il sistema tra la parabola e la circonferenza ci conduce all'equazione $x^2 - 2vx + v^2 + m^2x^4 - r^2 = 0$. Il primo membro deve essere uguale a $(x - x_0)^2(ax^2 + bx + c)$, che \`e uguale a $ax^4 + bx^3 + cx^2 - 2x_0ax^3 - 2x_0bx^2 -2x_0cx + x_0^2ax^2 + x_0^2bx + x_0^2c = ax^4 + (b - 2x_0a)x^3 + (c - 2x_0b + x_0^2a)x^2 + (x_0^2b - 2x_0c)x + x_0^2c$.
\par Eguagliamo i coefficienti e risolviamo il sistema (per sostituzione ovviamente: Descartes non conosceva certo le matrici; in questo caso comunque il sistema \`e abbastanza facile).
\begin{align}
	&\begin{cases}
		a = m^2,\\
		b - 2 x_0a = 0,\\
		c - 2 x_0b + x_0^2a = 1,\\
		x_0^2b - 2 x_0c = - 2v,\\
		x_0^2c = v^2 - r^2,
	\end{cases}\nonumber\\
	&\begin{cases}
		a = m^2,\\
		b = 2x_0m^2,\\
		c = 1 + 3x_0^2m^2,\\
		v = - \frac{x_0^2b - 2 x_0c}{2} = - \frac{2x_0^3m^2 - 2x_0 - 6x_0^3m^2}{2} = x_0 + 2x_0^3m^2,\\
		r^2 = v^2 - x_0^2c = x_0^2 + 4x_0^4m^2 + 4x_0^6m^2 - x_0^2 - 3 x_0^4m^2 = x_0^4m^2 + 4x_0^6m^2.
	\end{cases}\nonumber
\end{align}
\par Il problema a questo punto pu\`o considerarsi risolto (il calcolo della perpendicolare al raggio \`e banale), ma con un po' di geometria (\Cfr\ figura \ref{Tangenti_Descartes}) possiamo ricavare il risultato classico, secondo cui per costruire la tangente \`e sufficiente riportare l'ordinata del punto di tangenza sull'asse, dalla parte opposta rispetto al vertice, e tracciare la retta che unisce il punto cos\`i individuato col punto di tangenza.
\begin{wrapfigure}{R}{0.5\textwidth}
	\includegraphics[width=0.48\textwidth]{Storia_della_matematica/Immagine_Tangenti_Descartes.png}
	\caption{Dimostrazione della propriet\`a classica della tangente ad una parabola secondo Descartes.}
	\label{Tangenti_Descartes}
\end{wrapfigure}
\par I triangoli $ABT$ e $TBC$ sono entrambi rettangoli e gli angoli $ATB$ e $TCB$ sono entrambi complementari dell'angolo $BTC$ e dunque uguali: i triangoli $ABT$ e $TBC$ sono dunque simili. Per similitudine, $BT:BC=AB:BT$, da cui $AB = \frac{BT^2}{BC} = \frac{y_0^2}{v - x_0} = \frac{m^2x_0^4}{2x_0^3m^2} = \frac{x_0}{2}$; quindi $A$ \`e il punto medio di $OB$. Ne consegue che i triangoli rettangoli $ODA$ e $ABT$, i cui angoli $OAD$ e $TAB$ sono uguali, sono congruenti. Quindi $OD = BT$, che \`e il risultato classico.
\paragraph{Il metodo di Fermat.} Fermat concepisce un proprio metodo per il calcolo delle tangenti, indipendentemente da Descartes, basandosi sul proprio metodo per il calcolo dei massimi e dei minimi di una grandezza variabile\footnote{Il concetto di funzione non \`e ancora stato inventato: siamo ancora nell'ambito dell'algebra simbolica di Fran\c{c}ois Vi\`ete, \Cfr\ sottosezione \ref{FrancoisVieteELInvenzioneDellAlgebraSimbolica} e, come Vi\`ete e Fermat, useremo vocali per denotare quantit\`a ignote e consonanti per quantit\`a note}: cominciamo proprio da quest'ultimo.
\par Sia $F$ una una grandezza variabile e cerchiamone il massimo $M$; Fermat sottointende di cercare un massimo locale, che $F$ sia continua e di lavorare in un intorno in cui il massimo locale \`e unico. Per considerazioni che risalgono a Pappo, scelto $Z < M$ a destra e a sinistra del massimo esistono due punti $A$ ed $E$ per cui $F(A) = F(E) = Z$; inoltre, avvicinando $Z$ a $M$, i punti $A$ ed $E$ si avvicineranno sempre di pi\`u, pertanto, l'equazione $\frac{F(A) - F(E)}{A - E} = 0$ che vale per $Z < M$ fonir\`a una nuova equazione, valida quando $Z = M$, ponendo $E = A$ dopo aver effettuato le dovute semplificazioni (altrimenti avremmo una divisione per zero): risolvendo l'equazione in $A$, si trover\`a il punto in cui $F$ assume il suo massimo $M$.
\par Fermat osserva per\`o che l'operazione di divisione per il binomio $A - E$ risulta nei casi pratici piuttosto complicata, egli modifica dunque il proprio metodo prescrivendo di sostituire alla variabile $E$ il binomio $A + E$. L'equazione $\frac{F(A) - F(E)}{A - E} = 0$ diventa dunque $\frac{F(A) - F(A + E)}{E} = 0$ e la divisione per $E$ \`e molto pi\`u facile: a questo punto si pone $A + E = A$, cio\`e $E = 0$. Tuttavia il passaggio non \`e un semplice cambio di notazioni come pu\`o sembrare a prima vista, vi sono anche importanti differenze concettuali implicite: nel metodo originale $A$ ed $E$ erano entrambe grandezze indipendenti, ora invece $A$ \`e una grandezza indipendente ed $E$ \`e una variazione della grandezza $A$; nel metodo originale $A$ aveva inizialmente un valore a cui non corrispondeva il massimo di $F$ e poi il valore corrispondente al massimo, ora invece $A$ ha da subito il valore corrispondente al massimo ed $A + E$ viene a coincidere con $A$ azzerando $E$, cio\`e spostando il solo punto $A + E$; nel metodo originale, prima del ``passaggio al limite'', l'equazione $\frac{F(A) - F(E)}{A - E}$ era vera, ora invece la relazione $\frac{F(A) - F(A + E)}{E} = 0$ \`e vera solo ``al limite'' e in effetti Fermat si riferisce ad essa chiamandola non \textit{\'egalit\'e} o \textit{\'equation}, bens\'i \textit{ad\'egalit\'e}.
\par Alcuni storici hanno visto nel metodo di Fermat un'anticipazione dell'invenzione della derivata, ma ci\`o \`e vero solo sino ad un certo punto: la forza del concetto di derivata risiede nelle regole generali che consentono di ottenere la derivata di una funzione combinando opportunamente derivate di funzioni pi\`u semplici; Fermat invece intuisce solo il concetto di derivata in un punto come limite del rapporto incrementale e cos\`i, per cercare un massimo, si vede costretto ad eguagliare a $0$ un complicato limite anzich\'e una semplice derivata.
\par Tornando al problema delle tangenti, Fermat sostiene che le adeguaglianze possano essere applicate con successo anche in questo caso. Vediamo come con un esempio: il calcolo della tangente ad una parabola (\Cfr\ figura \ref{Tangenti_Fermat}).
\begin{wrapfigure}{R}{0.5\textwidth}
	\includegraphics[width=0.48\textwidth]{Storia_della_matematica/Immagine_Tangenti_Fermat.png}
	\caption{Dimostrazione della propriet\`a classica della tangente ad una parabola secondo Fermat.}
	\label{Tangenti_Fermat}
\end{wrapfigure}
\par Calcolare la tangente in un punto $T$ in questo caso significa calcolare la lunghezza del segmento $VP$. Per le propriet\`a della parabola, $VB:VC=RB^2:TC^2$. Ora, sostituiamo nella proporzione $SB$ a $RB$, ottenendo $VB:VC < SB^2:TC^2$. Per la similitudine tra triangoli, abbiamo $SB:TC=PB:PC$ e dunque $VB:VC < PB^2:PC^2$. Ora, Fermat vede che spostando il punto $B$ verso $C$ la disuguaglianza appena trovata diventa un'uguaglianza, dunque pone l'\textit{ad\'egalit\'e} $VB:VC = PB^2:PC^2$.
\par Poniamo l'\textit{ad\'egalit\'e} in termini algebrici: poniamo $PV = A$, $CB = E$, $CV = K$. Abbiamo allora
\begin{align}
	&(K - E):K=(A + K - E)^2:(A + K)^2,\nonumber\\
	&(K - E)(A + K)^2=(A + K - E)^2 K,\nonumber\\
	&(K - E)(A^2 + 2AK + K^2)=A^2K + K^3 - E^2K + 2AK^2 - 2 AEK - 2EK^2,\nonumber\\
	&A^2K + 2AK^2 + K^3 - A^2E - 2AEK - EK^2=A^2K + K^3 - E^2K + 2AK^2 - 2 AEK - 2EK^2,\nonumber\\
	&- A^2E = - E^2K - EK^2,\nonumber\\
\end{align}
\par Dividiamo per $E$ e otteniamo $A^2 = EK + K^2$; poniamo $E = 0$ ed arriviamo a $A^2 = K^2$ e dunque $A = K$, che il risultato classico ben noto.
\par Naturalmente, se lo si desidera, \`e possibile risolvere il problema tramite il metodo di massimi e minimi vero e proprio (la grandezza $\frac{VB}{SB^2}$ ha un massimo per $B = C$), ma Fermat non compie questo passaggio: esso appesantirebbe solamente la dimostrazione e probabilmente egli non lo concepisce nemmeno ma pensa direttamente ad applicare in ambito geometrico il concetto di adeguaglianza.
