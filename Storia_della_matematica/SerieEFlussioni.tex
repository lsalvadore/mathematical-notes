\subsection{Serie e flussioni}\label{SerieEFlussioni}
\par Isaac Newton (1642 - 1727) \`e l'altro grande inventore del calcolo infinitesimale oltre a Leibniz. Il metodo di Newton prevede due concetti fondamentali: quello delle \Define{flussioni}, analogo delle nostre derivate, e quello degli sviluppi in serie. Newton considera questi due metodi l'uno parte integrante dell'altro, a tal punto che, nella famosa disputa con Leibniz, egli accusa il filosofo tedesco di voler spezzare il suo lavoro in due per potersi appropriare di una parte di esso. La Storia, tuttavia, dimostra che l'atteggiamento pi\`u fecondo \`e quello di Leibniz: mentre i matematici inglesi riterranno la nuova teoria del calcolo infinitesimale completa grazie agli sviluppi in serie, sul continente invece il nuovo strumento dato dal differenziale diventa il protagonista di nuovi progressi, suggerendo lo studio delle equazioni differenziali e manifestando progressivamente la relazione tra derivazione e integrazione.
\paragraph{Flussioni.} Newton ragiona in termini cartesiani e meccanici. Egli vede un'equazione nelle incognite $x$ ed $y$ non come una curva nel piano, ma come due grandezze variabili (\Define{fluenti}) nel tempo: le loro \Define{flussioni}, cio\`e quelle che noi chiamiamo derivate, si indicano $\dot{x}$ e $\dot{y}$; la notazione \`e in uso ancora oggi, soprattutto nella derivazione rispetto al tempo di funzioni meccaniche. Le flussioni risolvono anche il problema della tangente ad una curva, per il noto risultato secondo cui la velocit\`a \`e appunto tangente alla traiettoria; se si desidera conoscere la direzione della tangente, \`e sufficiente calcolare il rapporto delle flussioni\footnote{In notazioni Leibniziane, $\frac{\frac{dx}{dt}}{\frac{dy}{dt}} = \frac{dx}{dt} \frac{dt}{dy} = \frac{dx}{dy}$.}.
\par A differenza di Leibniz, Newton, facendo appello al concetto di velocit\`a, vago ma intuitivo all'epoca, riesce a non introdurre mai grandezze infinitesime nei suoi enunciati, sebbene esse compaiano nelle sue dimostrazioni. Diamo un esempio con la derivazione della regola per il calcolo della flussione di un prodotto, che naturalmente \`e analoga per Leibniz: dato un tempo infinitesimo $o$, le grandezze $x$ ed $y$ aumentano nel tempo $o$ di $o\dot{x}$ e $o\dot{y}$ (Newton chiama quest'ultime grandezze \Define{momenti} di $x$ ed $y$), abbiamo che il momento di $xy$ \`e $(x + o\dot{x})(y + o\dot{y}) - xy = x \cdot o\dot{y} + y \cdot o\dot{x} + ox \cdot oy = o(x\dot{y} + \dot{x}y) + o^2 xy = o(x\dot{y} + \dot{x}y)$ visto che $o^2 xy$ \`e infinitesimo\footnote{Diremmo noi infinitesimo di ordine superiore rispetto ad $o$.}; la flussione \`e allora il rapporto del momento col tempo, cio\`e $x\dot{y} + \dot{x}y$.
\paragraph{Serie.} Alcuni matematici, come Mercatore (1512 - 1594) e Gregory (1638 - 1675), avevano dato alcuni sviluppi in serie delle funzioni del tempo, ma Newton riesce a darne molti di pi\`u: sfruttando svariate tecniche che vanno dall'integrazione termine a termine all'interpolazione esaurisce l'insieme delle funzioni considerate all'epoca.
\par Prima di Newton, salvo eccezioni, le serie venivano viste come utili mezzi di approssimazione; egli invece le usa come base per la sua nuova teoria:
\begin{itemize}
	\item per la ricerca della tangente ad una curva, basta ricavare la serie corrispondente ed applicarvi uno dei tanto metodi disponibili per i polinomi, come quelli di Hudde (1628 - 1704) o Sluse (1622 - 1685);
	\item ogni quadratura di una funzione viene sistematicamente ricondotta alla quadratura termine a termine della serie corrispondente alla funzione stessa;
	\item si possono anche risolvere equazioni differenziali come $\dot{y} = 1 - 3x + y + x^2 + xy$, dove si suppone $y(0) = 0$ (esempio trattato nel \textit{Methodus fluxionum et serierum infinitarum}): il metodo procede prendendo nel secondo membro tutti i termini di grado $0$; otteniamo $\dot{y} = 1$, da cui $y = x$; sostituiamo quanto trovato e consideriamo solo i termini di grado $1$; otteniamo $\dot{y} = 1 - 2x$, da cui $y = x - x^2$; procedendo grado per grado, sostituendo ogni volta, otteniamo uno sviluppo in serie di $y$.
\end{itemize}
