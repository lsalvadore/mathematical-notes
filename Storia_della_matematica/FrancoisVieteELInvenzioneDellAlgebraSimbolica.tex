\subsubsection{Fran\c{c}ois Vi\`ete e l'invenzione dell'algebra simbolica.}\label{FrancoisVieteELInvenzioneDellAlgebraSimbolica}
\paragraph{Prima di Vi\`ete.} Come abbiamo visto, la matematica greca \`e molto pi\`u interessata alla geometria che all'aritmetica ed \`e dunque naturale che essa non abbia sviluppato quasi nessun simbolismo, se si escludono gli enti geometrici rappresentati da lettere proprio come oggi. Bisogna comunque notare l'eccezione di Diofanto (III - IV secolo a.c.) che nel suo \textit{Arithmetica} fa uso sistematico di simboli per rappresentare numeri ed operazioni: sebbene i simboli da lui scelti non siano i pi\`u efficienti (manca per esempio la notazione posizionale), essi si rivelano comunque efficaci. \`E infatti proprio da Diofanto che partir\`a Vi\`ete per la sua proposta.
\par La matematica medievale dimostra interessi diversi rispetto a quelli greci, e cos\`i, come abbiamo visto, a seguito della pubblicazione del \textit{Liber abaci} si impone rapidamente l'uso del sistema numerico decimale posizionale. Inoltre, sempre grazie agli arabi, inizia a svilupparsi tra gli abacisti una corrente algebrica, che richiede un linguaggio meno retorico e pi\`u efficiente: non si arriva ancora ad un linguaggio simbolico, ma si iniziano ad abbrevviare le operazioni pi\`u comuni ($p$ per ``pi\`u'', $m$ per ``meno''), si introducono parole chiave come \textit{res} e \textit{census}\footnote{Il loro uso definisce la \Define{notazione cossica}, dalla traduzione della parola latina \Define{res} in lingua tedesca.} -- per indicare rispettivamente l'incognita e il suo quadrato --, simboli per abbreviarle e si cominciano anche a denotare con esponenti le potenze dell'incognita. \`E in questo contesto che si muove Fran\c{c}ois Vi\`ete.
\paragraph{Fran\c{c}ois Vi\`ete (1540 - 1603)} Vi\`ete desidera restituire il sapere degli antichi riattualizzandolo e cerca di ritrovare quei risultati a cui Diofanto fa spesso riferimento senza per\`o dimostrarli: \`e questo lo spirito con cui scrive \textit{In artem analyticam isagoge} (1591). \`E cos\`i che nasce la sua idea pi\`u innovativa: utilizzare simboli non solo per incognite e operazioni, ma anche per i coefficienti stessi delle equazioni. Vi\`ete sceglie di indicare con consonanti le quantit\`a note (con segno positivo: egli non introduce mai parametri negativi) e con vocali quelle ignote\footnote{A Descartes \`e dovuto l'uso attuale di indicare le quantit\`a note con le prime lettere dell'alfabeto e le incognite con le ultime.} e distingue tra
\begin{itemize}
	\item \textit{logistica numerosa}, che studia equazioni i cui coefficienti sono tutti numeri ben fissati, come \`e sempre stato fatto sino ad allora;
	\item \textit{logistica speciosa}, che studia equazioni i cui coefficienti sono simboli.
\end{itemize}
\par Vi\`ete non considera i parametri come numeri fissati qualunque, ma come veri e propri parametri: egli \`e perfettamente consapevole che non sta pi\`u studiando equazioni particolari, ma vere e proprie classi di equazioni. \`E il passaggio dall'aritmetica e da una forma primitiva di algebra, all'algebra simbolica.
\par Grazie a questo nuovo linguaggio, Vi\`ete riesce a dimostrare in piena generalit\`a identit\`a come quelle date dai prodotti notevoli.
