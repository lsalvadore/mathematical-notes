\subsection{Euclide e gli \textit{Elementi}.}\label{EuclideEGliElementi}
\paragraph{Biografia di Euclide.} Ben poco \`e noto della vita di Euclide. Secondo il \textit{Commento al primo libro degli Elementi} di Proclo (412-485), Euclide sarebbe vissuto ad Alessandria d'Egitto nel III secolo a.c. La sua collocazione cronologica si basa su alcune citazioni di Euclide ad opera di Archimede che dimostrebbero che Euclide \`e antecedente ad Archimede, ma queste citazioni sono imprecise ed Archimede potrebbe anche citare invece Eudosso; altra prova addotta \`e un anedotto che coinvolge Euclide e Tolomeo I (il sovrano chiese se esistesse un modo pi\`u semplice per imparare la geometria, Euclide rispose che ``non vi sono vie regie per la geometria''), ma lo stesso anedotto viene raccontato anche con Menecmo e Alessandro Magno come protagonisti. 
\par Ad di l\`a dell'attendibilit\`a del discorso di Proclo, si ritiene comunque che Euclide sia vissuto all'incirca nel III secolo, in ragione della fondazione della biblioteca di Alessandria all'inizio del secolo che potrebbe giustificare la chiamata di un uomo come Euclide nell'importante centro culturale e dello stile del suo lavoro: geometrico, sintetico e deduttivo, in perfetto accordo con la filosofia platonica; enciclopedico, come Aristotele. Inoltre alcuni degli argomenti da lui studiati sono tipici dell'epoca ellenistica, come le sezioni coniche o l'ottica.
\paragraph{Le opere.} L'opera principale di Euclide consiste naturalmente nei suoi famosi \textit{Elementi}, ma egli scrisse anche molte altre opere che trattavano di sezioni coniche, statica, musica, ottica, astronomia e altri argomenti di geometria pi\`u avanzati. Le opere che ci sono pervenute, oltre agli \textit{Elementi}, sono
\begin{itemize}
	\item i \textit{Dati}: una raccolta di proposizioni che permette di determinare quando, dati certi elementi, \`e possibile dedurne (costruirne) altri;
	\item l'\textit{Ottica}: un trattato sulla visione dell'occhio;
	\item la \textit{Catottrica}\footnote{Vi sono dubbi sull'autenticit\`a.}: un trattato sugli specchi;
	\item i \textit{Fenomeni}: un trattato di astronomia;
	\item la \textit{Section canonis}: un trattato sulla musica (teoria armonica).
\end{itemize}
\par Analizziamo i suoi \textit{Elementi}.
\paragraph{Gli \textit{Elementi}: il genere.} Il genere degli \textit{Elementi}, vale a dire una raccolta di risultati ordinata, sistematica e completa del sapere matematico sino ad allora nota, non \`e una novit\`a, ma essi si impongono subito come modello di esposizione e come manuale. Il metodo \`e \Define{sintetico}: i teoremi presuppongono ipotesi da cui si ricavano deduttivamente teoremi. Solo i risultati vengono esposti: non viene mai spiegato come il teorema e, soprattutto, la sua dimostrazione sono stati trovati.
\paragraph{Gli \textit{Elementi}: contenuti.} Gli \textit{Elementi} trattano argomenti di geometria e di teoria dei numeri. Pi\`u precisamente:
\begin{itemize}
	\item i libri I -- VI trattano di geometria piana:
	\begin{itemize}
		\item libro I: geometria del triangolo e del parallelogramma;
		\item libro II: sezioni di segmenti, aree associate ad esse e quadratura di poligoni;
		\item libro III: geometria del cerchio, tangenti;
		\item libro IV: poligoni regolari;
		\item libro V: grandezze e proporzioni;
		\item libro VI: similitudini di figure piane;
	\end{itemize}
	\item i libri VII -- X trattano di teoria dei numeri:
	\begin{itemize}
		\item libro VII: frazioni, massimo comun divisore, minimo comune multiplo;
		\item libro VIII -- IX: progressioni geometriche, numeri primi, numeri perfetti;
		\item libro X: classificazione dei numeri irrazionali;
	\end{itemize}
	\item i libri XI -- XII trattano di geometria solida:
	\begin{itemize}
		\item libro XI: figure solide fondamentali, parallelepipedi;
		\item libro XII: piramidi, prismi, coni, cilindri, sfere;
		\item libro XIII: sezione aurea e solidi platonici;
	\end{itemize}
	\item i libri XIV -- XV sono considerati apocrifi e trattano di solidi platonici inscritti in sfere o in altri solidi platonici.
\end{itemize}
\paragraph{Gli \textit{Elementi}: la forma.}
\par Come vedremo pi\`u avanti, ci sono diverse difficolt\`a nello stabili quale sia il testo originale di Euclide, ma, secondo Proclo i teoremi erano tutti esposti in una forma canonica che presentava, in ordine, i seguenti elementi:
\begin{itemize}
	\item la \Define{protasi}: l'enunciato;
	\item l'\Define{ectesi}: la descrizione della figura geometrica su cui si lavora con la denominazione delle parti coinvolte nella protasi;
	\item il \Define{diorismo}: la denominazione di altre parti della figura che non erano esplicitamente coinvolte dalla protasi;
	\item la \Define{costruzione}: la costruzione di nuovi elementi all'interno della figura, con eventuali nuove denominazioni;
	\item la \Define{dimostrazione}: il ragionamento logico deduttivo, basato sulla figura, che permette di mostrare la verit\`a della protasi;
	\item il \Define{sumperasma}: la conclusione della dimostrazione, che ribadisce la protasi stavolta con un valore di dimostrata verit\`a.
\end{itemize}
\par Nonostante la forma sistematica dell'opera, non mancano difficolt\`a logiche e interpretative: per esempio, nella prima proposizione che mostra la procedura di costruzione di un triangolo equilatero Euclide usa il fatto non dimostrato che due circonferenze tracciate agli estremi di un segmento con lo stesso segmento per raggio si intersecano in due punti; nella quarta proposizione si fa riferimento al concetto non definito di movimento rigido.
\paragraph{Gli \textit{Elementi}: le difficolt\`a nello stabilire il testo originale di Euclide.} \`E sempre difficile stabilire il testo originale di un documento antico: esso \`e facilmente soggetto a errori di copiatura, perdite di parti intere, attribuzioni sbagliate ecc. Nel caso di un testo matematico come gli \textit{Elementi} si verifica anche la presenza di numerose alterazioni volontarie:
\begin{itemize}
	\item aggiunta o rimozione di materiale (soprattutto aggiunta): esse sono giustificate dal desiderio di rendere il testo pi\`u chiaro e completo o meno prolisso e pi\`u essenziale;
	\item riordinamento delle proposizione: lo scopo \`e quello di migliorare il ragionamento deduttivo;
	\item sostituzione di dimostrazione: pu\`o essere l'effetto della volont\`a di una maggior generalit\`a o di evitare particolari metodi di dimostrazione, in ogni caso di migliorare la dimostrazione.
\end{itemize}
\paragraph{Gli \textit{Elementi}: la Storia del testo.} Si ritiene che esistano due edizioni fondamentali nel Mondo Antico: l'edizione euclidea di et\`a ellenistica e un'edizione del IV secolo d.c. ad opera di Teone d'Alessandria. Circolavano inoltre gli importanti commenti di Erone (I secolo d.c.) e Pappo (III-IV secolo d.c.), che possono aver avuto una certa influenza sull'edizione teonina. Si pensa che il testo che ci \`e pervenuto sia principalmente quello di Teone.
\par A partire dal IX secolo gli arabi traducono gli \textit{Elementi} contribuendo alla loro diffusione e conservazione, ma la tradizione araba presenta testi molto pi\`u essenziali, in cui mancano numerose definizioni e diversi lemmi.
\par Nell'Europa medievale incontriamo sia traduzioni dal greco al latino che traduzioni dall'arabo al latino. Le pi\`u importanti di queste sono quelle di Adelardo di Bath (1075-1160) basate sia su fonti greche che su fonti arabe: su di esse (ma non solo) lavora Campano di Novara (m. 1296) per scrivere la propria traduzione degli \textit{Elementi} (1259), che costituisce un modello di riferimento sino al Cinquecento. Campano non mira a una ricostruzione del contenuto originale, bens\`i alla stesura di un testo matematicamente coerente, chiaro e corretto: non esita dunque ad allontanarsi dal testo di Euclide ogni qual volta egli lo ritenga necessario.
\par Nella seconda met\`a del Quattrocento viene inventata la stampa. Venezia era particolarmente interessata al reperimento, alla traduzione e alla diffusione dei testi antichi e si trova allora in una posizione geografica e storica ideale per queste attivit\`a; diventa dunque rapidamente uno dei principali centri di stampa di Europa. \`E proprio a Venezia che Erhardt Ratdolt (1142 - 1528) pubblica nel 1482 la prima edizione a stampa (\textit{editio princeps}) degli \textit{Elementi}: Ratdolt non \`e un matematico, il suo interesse per i testi scientifici nasce da una collaborazione col matematico Regiomontano (1436 - 1476), e dunque sceglie l'edizione di Campano perch\'e la pi\`u diffusa, senza porsi problemi filologici n\'e matematici. Notevole la presenza delle figure.
\par Nel 1505 viene stampata da Zamberti (1473 - 1543) una nuova edizione degli \textit{Elementi} in spirito filologico: Zamberti tenta di restare quanto pi\`u vicino possibile al suo codice greco. Il risultato \`e un testo vicino al testo originale (anche se bisogna tener conto che il codice greco in questione era pessimo), ma privo di qualsiasi sensibilit\`a geometrica. Ne scaturisce un dualismo tra l'edizione campiana e quella zambertiana che perdura per tutto il Cinquecento, con tentativi di sintesi o addirittura di accostamento delle due versioni.
\par Nel 1543 Tartaglia (1499 - 1557) pubblica la prima edizione a stampa in lingua volgare: \`e la prima edizione integrale pervenutaci, anche se almeno frammenti erano gi\`a stati pubblicati, per esempio nel \textit{Summa de arithmetica, geometria, proportioni et proportionalita} (1494) di Luca Pacioli (1445 circa - 1517), il quale si dice abbia pubblicato una traduzione integrale degli \textit{Elementi} che per\`o non ci \`e pervenuta. Tartaglia, sebbene sia solo un maestro di abaco, \`e un matematico e in quanto tale compie un'operazione di sintesi consapevole delle due edizioni campana e zambertiana: cerca dunque di modificare il meno possibile il testo, ma lo modifica comunque quando questo \`e necessario per ottenere un testo matematicamente coerente, in particolare con l'aggiunta di commenti propri e suggerimenti di applicazioni pratiche; ad esempio Tartaglia giustifica la sua scelta tra il termine campiano ``colonna rotonda'' e quello zambertiano ``cilindro'' (sceglie ``cilindro'' in quanto termine originale, mentre il termine ``colonna rotonda'' si riferische alle colonne architettoniche le quali per\`o non sono cilindriche bens\'i hanno ``un puoco di panzetta'').
\par La miglior sintesi tra matematica e filologia \`e dovuta a Commandino (1509 - 1575) che traduce gli \textit{Elementi} sia in latino (1572) che in volgare (1575). In particolare, Commandino esamina la questione della paternit\`a del testo, cio\`e se esso sia opera di Euclide o di Teone e in quale misura: la sua conclusione \`e che Euclide diede sia le proposizione che le dimostrazioni, ma che Teone riscrisse le dimostrazioni nel modo in cui egli le insegnava ai suoi discepoli (prima di Commandino c'era chi si era a spinto a sostenere che le dimostrazioni fossero tutte originali di Teone). Notevole \`e anche la scelta grafica nuova di Commandino: egli non rappresenta pi\`u i solidi come schiacciati nel piano, come veniva fatto nei codici greci, arabi, nonch\'e nelle edizioni campane e zambertiane, bens\`i egli li disegna secondo le contemporanee tecniche prospettiche; la scelta \`e accettabile anche da un punto di vista filologico in virt\`u della \textit{scaenographica}, citata da Gemino (I secolo a.c.), una tecnica che rappresenta gli oggetti come appaiono alla vista, ma di cui non si conoscono i dettagli.
\par L'edizione di Commandino rimane quella di riferimento sino al 1818 circa quando viene individuato nel manoscritto Vat.Gr.190 un testo corrispondente all'edizione euclidea con aggiunte dall'edizione teonina e per questo verosimilmente il pi\`u antico testimone che abbiamo degli \textit{Elementi}. Tale manoscritto, databile circa al IX secolo, \`e uno dei sei (tutti greci) utilizzati da Johann Ludvig Heiberg per l'edizione di riferimento di oggi. Interessante notare che le figure dell'edizione di Heiberg non sono quelle originali: quest'ultime infatti presentavano spesso casi particolari, mentre Heiberg preferisce offrire casi generali.
