\subsubsection{La \textit{traditio} dell'opera di Archimede, di Apollonio e di Pappo.}\label{LaTraditioDellOperaDiArchimedeDiApollonioEDiPappo}
\paragraph{Il codice Moerbeke.} La \textit{traditio}, cio\`e la tradizione, delle opere degli antichi matematici greci ruota attorno a quella di coloro che furono forse i pi\`u grandi matematici dell'antico mondo greco: quella degli \textit{Elementi} di Euclide, che abbiamo gi\`a esaminato ampiamente nella sottosezione \ref{EuclideEGliElementi}, e quella dei lavori di Archimede, a cui abbiamo gi\`a dato alcuni accenni e che esaminiamo pi\`u approfonditamente adesso.
\par Nel corso del basso medioevo, come si \`e visto, la scena matematica \`e dominata dalla scuole d'abaco e dunque da un approccio pragmatico e dunque pragmatico \`e anche l'interesse manifestato nei confronti dei pochi scritti di Archimede che circolano, dai quali gli abacisti traggono, essenzialmente, le formule per il calcolo della circonferenza, del cerchio, della sfera e della palla, senza curarsi troppo di comprenderne la provenienza.
\par Uno dei pochi personaggi che si dedica ad uno studio completo dell'opera di Archimede \`e il domenicano Guglielmo di Moerbeke (circa 1215 - 1286). Egli, lavorando presso la corte papale di Viterbo, prepara un codice che contiene la traduzione di due manoscritti greci, oggi perduti, contenenti alcuni testi di Archimede e alcuni commenti. L'opera \`e particolarmente importante perch\'e lo stile di Gugliemo \`e molto vicino a quello degli originali, cosa che compensa la perdita degli originali stessi e perch\'e i \textit{Galleggianti} ci erano pervenuti solo tramite uno dei due manoscritti greci in questione e sarebbero dunque andati perduti senza il codice Moerbeke. Grazie a Guglielmo, alla fine del 1269 tutta l'opera di Archimede oggi pervenutaci, all'eccezione dell'\textit{Arenario} e del \textit{Metodo}, \`e disponibile ai matematici tardomedievali, che per\`o abbiamo visto essere maestri d'abaco tesi all'utilit\`a pratica e dunque ancora impreparati ad accogliere l'impostazione teorica degli antichi. Inoltre, l'ambiente propizio allo sviluppo matematico della corte di Viterbo si interrompe presto a causa della cattivit\`a avignonese (1309 - 1377). Il contesto europeo \`e ulteriormente turbato dalla Guerra dei Cent'anni (1337 - 1453), nonch\'e dalla peste nera (1348). Tutto ci\`o ha per risultato che il codice di Moerbeke rimane pressoch\'e ignorato sino al XIV secolo, eccezion fatto per una copia della traduzione delle \textit{Spirali}, chiaramente di scarso interesse per l'epoca.
\paragraph{Da Iacopo di San Cassiano (circa 1415 - circa 1452) a Tartaglia.} Il papa Niccol\`o V (1397 - 1455) \`e un grande sostenitore dell'umanesimo ed \`e infatti il fondatore della Biblioteca Vaticana. \`E proprio per la Biblioteca Vaticana che Niccol\`o V commissiona a Iacopo di San Cassiano una traduzione dell'opera di Archimede: la traduzione che ne risulta viene rivista e corretta attorno al 1462 da Regiomontano, venuto in Italia insieme al gi\`a citato cardinale Bessarione; Regiomontano si serve, tra altri documenti, di uno dei manoscritti che aveva utilizzato anche Gugliemo da Moerbeke, e forse della traduzione di Guglielmo stesso. Regiomontano vorrebbe sfruttare la stampa per pubblicare un'edizione delle opere di Archimede, ma purtroppo muore nel 1476; il suo lavoro viene comunque utilizzato per l'\textit{editio princeps} grecolatina di Basilea (1543 - 1544).
\par A partire dalla traduzione di Iacopo da Cassiano, gli abacisti, sia pure sempre orientati ad interessi pratici, cominciano ad interessarsi maggiormente ai lavori di Archimede. In particolare,
\begin{itemize}
	\item Piero della Francesca dimostra di aver studiato Archimede nel suo \textit{Libellus de quinque corporibus regularibus} e si discute anche di un codice della biblioteca riccardiana di Firenze di cui egli potrebbe essere l'autore e che attesterebbe anch'esso gli interessi archimedei di Piero;
	\item Leonardo da Vinci propone una dimostrazione riguardo alla posizione del centro di gravit\`a del triangolo nel \textit{Codice Arundel}, anche se ingenua rispetto allo stile di Archimede.
\end{itemize}
\par Gli abacisti non sono dunque ancora in grado di apprezzare a pieno gli scritti di Archimede, ma l'interesse \`e ormai nato. Notevole il codice M, citato anche da Leonardo, usato da Luca Gaurico (1475 - 1558) per la pubblicazione del \textit{Tetragonismus} (Venezia, 1503), contenente la \textit{Quadratura della parabola} e la \textit{Misura del cerchio}: sebbene il \textit{Tetragonismus} sommi gli errori del codice M, di Gaurico e dello stampatore, esso resta comunque la prima stampa di testi archimedei.
\par Ma i lavori di Archimede si diffondono ancora di pi\`u con l'opera di Tartaglia, che dedica molto impegno ai testi del grande matematico di Siracusa:
\begin{itemize}
	\item reperimento nel 1531 di una copia latina del primo libro de \textit{La sfera e il cilindro};
	\item pubblicazione dei lavori di Archimede nel 1543 - 1544, basata sul codice M -- unica edizione sino al 1565 ad includere i \textit{Galleggianti};
	\item pubblicazione di un'edizione in volgare dei \textit{Galleggianti} nel 1551 in \textit{Ragionamenti intorno alla sua travagliata inventione};
	\item pubblicazione postuma (1557) di una parafrasi in volgare de \textit{La sfera e il cilindro} in \textit{General trattato di numeri e misure}.
\end{itemize}
\par Abbiamo gi\`a visto che la trasmissione dei testi archimedei porta alla trasmissione anche di quelli apolloniani e pappiani. Per ulteriori dettagli sulla \textit{traditio} di questi autori, rimandiamo alla sottosezione \ref{LUmanesimoEIlRecuperoDellaMatematicaGreca}.
