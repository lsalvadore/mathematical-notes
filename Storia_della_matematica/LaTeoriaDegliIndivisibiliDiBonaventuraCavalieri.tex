\subsection{La teoria degli indivisibili di Bonaventura Cavalieri}\label{LaTeoriaDegliIndivisibiliDiBonaventuraCavalieri}
\paragraph{Bonaventura Cavalieri (1598 - 1647).} Bonaventura Cavalieri nasce presumibilmente a Milano nel 1598. Nel 1615 entra a far parte dell'ordine religioso dei Gesuati di San Girolamo. L'anno successivo l'ordine lo trasferisce a Pisa, dove Cavalieri impara la matematica sotto la guida di Benedetto Castelli, discepolo di Galileo, che gli presenta Galileo stesso.
\par Nel 1620 Cavalieri viene ritrasferito a Milano per studiare teologia. Gi\`a dal 1621 egli comincia per\`o a lavorare alla sua teoria degli indivisibili: nel 1627 viene pubblicata una prima versione della \textit{Geometria indivisibilibus}, con la quale si guadagna la stima di Galileo e l'amicizia di Torricelli.
\par Nel 1629 Cavalieri, con l'appoggio di Galileo, riesce ad aggiudicarsi una delle due letture di matematica a Bologna, dove rimane per il resto della sua vita, pubblicando, accanto ad i suoi importanti lavori di matematica, anche testi di astronomia e astrologia per volont\`a del senato bolognese. Nel 1641 lo svizzero Paolo Guldino (1577 - 1643) attacca ferocemente la teoria degli indivisibili nel quarto libro della \textit{Centrobaryca} (1641), con accuse sia di plagio nei confronti di Keplero e Sovero che di errori metodologici, spingendo Cavalieri a rispondere prima con un \textit{Dialogo}, ritirato su consiglio degli amici per la recente morte di Guldino, poi con le \textit{Exercitationes geometriae} (1647).
\paragraph{La teoria degli indivisibili -- Prima versione.} Per la sua teoria degli indivisibili Cavalieri si isipira al \textit{De centro} di Luca Valerio. Egli concepisce in momenti diversi due teorie diverse.
\par In un primo momento il tentativo \`e quello di introdurre una nuova grandezza geometrica: ``tutte le linee'' (\textit{omnes linae}) di una figura piana. L'idea \`e quella di tagliare il piano della figura con un altro piano libero di scorrere sul piano originale ma rimanendo sempre parallelo alla stessa direzione: i segmenti della figura cos\`i individuati sono ``tutte le linee'' della figura; analogamente sono definibili ``tutti i piani'' di un solido. I cosiddetti ``indivisibili'' sono dunque le suddette linee e i suddetti piani. Perch\'e questo oggetto venga effettivamente riconosciuto come una grandezza, \`e necessario che sia inquadrabile dalla teoria delle proporzioni descritta nel libro V degli \textit{Elementi} di Euclide: Cavalieri si cimenta dunque in una serie di dimostrazioni per provare questa inquadrabilit\`a, tuttavia le sue argomentazioni sono vaghe e finiscono generalmente per identificare ``tutte le linee'' di una figura con la figura stessa.
\par Nonostante i problemi fondazionali, i nuovi enti si dimostrano particolamente validi per affrontare il calcolo di aree e volumi, per esempio per calcolare il volume di nuovi solidi di rotazione introdotti da Keplero.
\par Vale la pena di segnalare un teorema importante sui rapporti tra solidi. Cavalieri definisce un solido \Define{simile} quando le sue sezioni verticali da una parte e le sue sezioni orizzontali dall'altra sono tutte simili tra loro: per esempio una piramide a base quadrata \`e un solido simile perch\'e tutte le sue sezioni verticali sono triangoli simili tra loro e tutte le sue sezioni orizzontali sono quadrati. Vi sono inoltre i solidi \Define{mutuamente simili}: si tratta di solidi simili con la stessa base, ma con sezioni verticali diverse. Cavalieri dimostra che il rapporto di solidi mutuamente simili non dipende dalla base, ma solo dai ``profili'' dei solidi, cio\`e dalle rispettive sezioni verticali massime.
\par Diamo un cenno della dimostrazione. Consideriamo due solidi mutuamente simili $A$ e $B$. Si consideri un ulteriore solido simile $A'$ con sezione orizzontale quadrata e ``profilo'' uguale a quello di $A$: se tagliamo $A$ e $A'$ con un piano orizzontale otteniamo figure piane simili alle rispettive basi. Il rapporto tra ciascuna sezione orizzontale di $A$ e la sezione orizzontale di $A'$ corrispondente \`e indipendente dal particolare piano orizzontale considerato: in effetti la sezione orizzontale di $A$ sta alla base di $A$  come la sezione orizzontale di $A'$ sta alla base di $A'$, e dunque le due sezioni stanno tra loro come le rispettive basi.
\par Ora, Cavalieri fa uso della propriet\`a del comporre delle proporzioni per dedurre che la somma di tutte le sezioni di $A$ sta alla somma di tutte le sezioni di $A'$ come le rispettive basi, e cio\`e che i solidi $A$ e $A'$ stanno tra loro nel medesimo rapporto. Ripetendo il ragionamento sul solido $B$ e utilizzando la stessa base usata per $A'$ per costruire $B'$, si deduce che ancora nello stesso rapporto stanno $B$ e $B'$. Quindi $A$ sta a $B$ come $A'$ sta a $B'$, cio\`e i due solidi di partenza stanno tra loro come i quadrati costruiti sulle basi delle rispettive sezioni massime.
\par Il teorema \`e molto importante perch\'e permette di generalizzare il risultato che stabilisce che il volume di una piramide \`e un terzo di quello del prisma con stessa base e stessa altezza, non solo al caso di cono e cilindro, ma a copiramidi con basi qualsiasi (per esempio ellitiche) e dei corrispondenti cilindprismi, risultato tutt'altro che banale. Tuttavia la dimostrazione evidenzia alcune difficolt\`a: la propriet\`a del comporre \`e applicata sommando un numero infinito di termini, senza nessuna legittimizzazione dell'operazione, e inoltre si afferma che la somma di ``tutti i piani'' sia il solido intero, di nuovo senza giustificazione.
\paragraph{La teoria degli indivisibili -- Seconda versione.} Cavalieri \`e consapevole dei difetti della sua teoria, che come abbiamo accennato gli vengono anche sottolineati pi\`u tardi da critici quali Guldino, \`e gi\`a mentre viene stampato il sesto libro della \textit{Geometria} ne prepara un settimo per superarle: ivi, egli evita discorsi sull'infinito, abbandonando anche l'introduzione degli indivisibili come nuovi enti e proponendo invece un metodo, detto \Define{metodo distributivo}, il quale consiste essenzialmente nell'applicazione di un teorema che afferma quanto segue: se due figure piane sono tagliate da un fascio di rette parallele in modo che i segmenti staccati sulle due figure stiano sempre tra loro nello stesso rapporto, allora il rapporto tra le due figure \`e il medesimo rapporto. Naturalmente vale un analogo enunciato per i solidi (quello che ancora oggi chiamiamo ``principio di Cavalieri''). Qui Cavalieri si muove in ampia generalit\`a, una generalit\`a mai vista prima, considerando solidi con buchi, convessit\`a, concavit\`a -- figure che non hanno pi\`u nulla a che vedere con l'ideale perfezione, cio\`e in questo caso la semplicit\`a, del cerchio o del quadrato degli antichi: egli tenta di ricondurre le sue dimostrazioni alla teoria delle figure digradanti di Valerio, si affida cio\`e all'autorit\`a di Valerio come Valerio si \`e affidato all'autorit\`a di Archimede.
\par Sia Valerio che Cavalieri dimostrano un gran coraggio nell'affrontare pionieristicamente problemi generali, studiando metodi generali, ma entrambi, come anche Galileo in campo fisico, si abbattono sullo stesso scoglio: la mancanza di un linguaggio adeguato. Essi sono tutti ancora troppo legati ad Archimede e al suo linguaggio basato sulla teoria delle proporzioni; per compiere il passo successivo, l'invenzione del calcolo infinitesimale, manca ancora il linguaggio giusto: la geometria analitica di Cartesio.
