\subsection{La \textit{G\'eom\'etrie} di Ren\'e Descartes.}\label{LaGeometrieDiReneDescartes}
\paragraph{Il \textit{Discours de la m\'ethode} (1637).} La \textit{G\'eom\'etrie}, che descriveremo nei suoi caratteri principali nei paragrafi seguenti, costituisce un saggio facente parte del \textit{Discours de la m\'ethode} di Ren\'e Descartes (1596 - 1650), atto a dimostrare l'efficacia della \textit{m\'ethode}. Il \textit{Discours}  \`e un trattato sulla conoscenza secondo Descartes, la quale proviene per lui da idee chiare e distinte, che vengono utilizzate per costruire idee progressivamente pi\`u complesse. Nella \textit{G\'eom\'etrie}, le idee chiare e distinte sono sostanzialmente gli assiomi e le definizioni di partenza e il processo di costruzione di idee pi\`u complesse corrisponde al processo di costruzione di linee -- cio\`e curve -- pi\`u complesse, complessit\`a misurata dal \Define{genere} (\Cfr\ pi\`u avanti) della linea: il Cartesio geometra \`e dunque anche un po' inventore come Archimede. Abbiamo appena fatto riferimento al concetto di costruzione: in effetti, bench\'e sia proprio a partire dalla \textit{G\'eom\'etrie}, oltre che dai lavori di Fermat\footnote{Entrambi in questo allievi di Vi\`ete.}, che i matematici smetteranno di studiare curve date dal processo di costruzione e cominceranno a studiare curve date da un'equazione, Cartesio resta attaccato al concetto antico che identifica costruibilit\`a con esistenza. Come vedremo nel Libro I, egli calcola la lunghezza di segmenti, ma, alla luce delle sue idee chiare e distinte, la formula che ottiene per il calcolo equivale ad un processo di costruzione.
\par Il metodo che Descartes intende illustare diventa per\`o fonte di molteplici contraddizioni e difficolt\`a. Per prima cosa, come vedremo, le ``idee chiare e distinte'' su cui basa il suo trattato non hanno proprio nulla di chiaro: non \`e affatto chiaro perch\'e il prodotto di due segmenti dovrebbe essere definito in un modo piuttosto che in un altro; l'unica vera legittimizzazione della scelta delle definizioni del Libro I \`e costituita pragmaticamente dal loro successo. Il metodo cartesiano diventa dunque un involontario stratagemma per evitare di fornire scomode giustificazioni.
\par I concetti non vengono presentati n\'e sono presentabili coerentemente con l'ordine di complessit\`a previsto: l'intenzione \`e quello di introdurre per primi i segmenti come oggetto geometrico pi\`u semplice, tuttavia il calcolo della radice quadrata di un segmento richiede l'uso di un compasso, cio\`e della circonferenza, e le operazioni sugli oggetti pi\`u semplici si trovano ad essere definite in termini di oggetti pi\`u complessi.
\par Infine, Descartes si trova ad essere forse pi\`u schiavo del proprio programma che padrone: il suo metodo gli impone delle limitazioni oltre le quali egli non tenta neppure di andare. Per esempio, egli si occupa solo di linee \Define{geometriche} rinunciando a prescindere allo studio delle linee \Define{meccaniche} (\Cfr\ pi\`u avanti).
\paragraph{La \textit{G\'eom\'etrie} -- Libro I.} Il primo libro, come anticipato, si occupa di dare le definizioni di base sui segmenti e fornisce la prima prova dell'efficacia del metodo attraverso il problema di Pappo. Descartes vi fa ampio riferimento all'algebra simbolica di Vi\`ete, rinnovandola: \`e Descartes che stabilisce la convenzione ancora oggi in uso di utilizzare le prime lettere dell'alfabeto per le quantit\`a note e le ultime per le incognite; ma, soprattutto, \`e Descartes che rompe, una volta per tutte, con la \textit{regola di omogeneit\`a}.
\par Dagli antichi greci sino ai matematici del Cinquecento passando per il mondo arabo, \`e sempre stato assunto da tutti che il prodotto di due lunghezze fosse un'area e il prodotto di tre un volume; inoltre, lunghezze si sommano con lunghezze, aree con aree, volumi con volumi: \`e il contenuto della regola di omogeneit\`a, ribadita anche da Fran\c{c}ois Vi\`ete. Descartes invece, per la prima volta, costruisce un'algebra chiusa delle lunghezze: il prodotto di due segmenti, e dunque di un numero arbitrario di segmenti, \`e ancora un segmento e diventa dunque possibile sommare un segmento col prodotto di tre segmenti contro la regola di omogeneit\`a.
\par Le operazioni sui segmenti sono definite, dopo aver fissato un segmento unit\`a, in modo che la lunghezza del segmento risultante coincida col risultato che si otterrebbe applicando la stessa operazione alle lunghezze dei segmenti anzich\'e sui segmenti stessi. Diamo rapidamente tre esempi.
\par Definiamo il prodotto di due segmenti. Si dispongono i due segmenti in maniera che abbiano un estremo comune: per esempio nella figura \ref{Geometrie_Prodotto} abbiamo disposto i due fattori $OA$ e $OB$ -- i segmenti blu -- in modo tale che abbiano in comune l'estremo $O$. Si riporta su uno dei due fattori, per esempio $OA$, il segmento unit\`a con un estremo in $O$: otteniamo nel nostro caso il segmento $OU$. Tracciamo dunque la parallela ad $UB$ passante per $A$: essa incontra $OB$ nel punto $C$. Il segmento $OC$ \`e il prodotto di $OA$ per $OB$.
\begin{wrapfigure}{R}{0.5\textwidth}
	\includegraphics[width=0.48\textwidth]{Storia_della_matematica/Immagine_Geometrie_Prodotto.png}
	\caption{Definizione del prodotto di due segmenti.}
	\label{Geometrie_Prodotto}
\end{wrapfigure}
\par In effetti, i triangoli $OUB$ e $OAC$ sono simili ed abbiamo $OC:OA=OB:OU$, da cui $OC = \frac{OA \cdot OB}{OU} = OA \cdot OB$.
\par Analogamente possiamo definire la divisione di segmenti. Basandoci sulla medesima figura, la divisione di due segmenti si ottiene facendo coincidere due degli estremi -- ottenendo per esempio i segmenti $OA$ e $OC$ --, riportando su uno dei due segmenti il segmento unit\`a -- costruiamo $OU$ --, intersecando la parallela ad $AC$ passante per $U$ con $OC$ -- individuiamo $B$: il segmento $OB$, di lunghezza $OB = \frac{OC \cdot OU}{OA} = \frac{OC}{OA}$, \`e per definizione il rapporto dei segmenti $OC$ e $OA$.
\begin{wrapfigure}{R}{0.5\textwidth}
	\includegraphics[width=0.48\textwidth]{Storia_della_matematica/Immagine_Geometrie_Radice.png}
	\caption{Definizione della radice di un segmento.}
	\label{Geometrie_Radice}
\end{wrapfigure}
\par Diamo infine la definizione della radice di un segmento. Supponiamo di voler costruire la radice del segmento $AB$ (\Cfr\ figura \ref{Geometrie_Radice}): dobbiamo prolungare la retta $AB$ da una parte di un'unit\`a, ottenendo per esempio il segmento $UA$; tracceremo dunque una semicirconferenza di diametro $UB$ e intersecheremo la perpendicolare al diametro passante per $A$ con la semicirconferenza individuando il punto $R$. Il segmento $AR$ \`e per definizione la radice di $AB$. Abbiamo, per quanto riguarda le lunghezze, $AR = \sqrt{ \left ( \frac{UB}{2} \right ) ^2 - \left ( \frac{UB}{2} - UA \right )^2 } = \sqrt{ 2 \frac{UA \cdot UB}{2} - UA^2} = \sqrt{UB - 1} = \sqrt{AB}$.
\par Con l'introduzione dell'algebra chiusa dei segmenti (di cui ovviamente Descartes non scrive in termini cos\`i moderni), Descartes \`e capace di risolvere qualsiasi problema di costruzione con riga e compasso: l'idea \`e quella di utilizzare un metodo di analisi e sintesi (secondo la terminologia pappiana). Si suppone che la costruzione sia stata eseguita, si esprime la grandezza ignota che si desidera costruire in termini delle grandezze note, ottenendo un'equazione di grado al pi\`u $2$, la cui risoluzione \`e possibile, in termini delle grandezze note, con l'uso solo di somme, sottrazioni, prodotti, divisioni ed estrazioni di radici: la formula trovata fornisce dunque anche il metodo di costruzione.
\par Il trattato passa alla risoluzione del problema di Pappo. Sono date nel piano $n$ rette $l_i$ e altrettanti angoli $\phi_i$; dato un generico punto $P$, si denota $d_i$ la distanza di $P$ da $l_i$ calcolata lungo la retta passante per $P$ che compie con $l_i$ l'angolo $\phi_i$: si chiede di trovare tutti i punti $P$ tali che il rapporto del prodotto delle prime $\left \lfloor \frac{n}{2} \right \rfloor$ distanze $d_i$ con le restanti distanze $d_i$ sia uguale a un certo rapporto dato $\frac{\alpha}{\beta}$\footnote{Descartes, nel caso $n$ sia dispari, aggiunge a denominatore un'ulteriore distanza pari ad un'unit\`a, presumibilmente per calcolare un rapporto tra grandezze omogenee, a dispetto della rinuncia alla legge di omogeneit\`a vista prima: pu\`o darsi che egli non voglia estendere il suo nuovo punto di vista ai rapporti o che desideri riportare il problema di Pappo secondo lo stile antico.}. Prima della \textit{M\'ethode}, il problema era stato risolto solo in parte da Apollonio, per casi molto particolari: Apollonio era riuscito a dimostrare che il luogo descritto da $P$ nel caso in cui $n = 4$ \`e una sezione conica, senza saper descrivere quale precisamente; ora Descartes riesce a fornire una soluzione generale al problema, sebbene con alcune limitazioni.
\par La soluzione cartesiana prevede di fissare due rette principali: la prima retta \`e una delle rette date, per esempio $l_1$, e la seconda \`e la retta passante per $P$ che interseca $l_1$ con angolo $\phi_1$ -- chiamiamo $B$ il punto di intersezione e $r$ la retta. Descartes denota con $x$ la distanza $AB$, dove $A$ \`e un'intersezione di $l_1$ con un'altra delle rette date, e denota con $y$ la distanza $d_1$. Il processo \`e analogo ad aver fissato un sistema di assi cartesiani nel piano con origine in $A$ (gli assi non sono per\`o ortogonali).
\par A questo punto Descartes ricava, servendosi di considerazioni geometriche sui triangoli di cui conosce gli angoli per ipotesi o i lati per calcoli precedenti, una per una le distanze $d_i$ in termini di $x$ ed $y$: egli determina che ogni $d_i$ \`e somma di un termine lineare in $x$, di un termine lineare in $y$ e di un termine noto (il termine in $x$ o quello in $y$ pu\`o mancare se $l_i$ \`e parallela a $l_1$ o $r$ rispettivamente). Egli fissa ora un valore arbitrario per la $y$: la proporzione che definisce il problema diventa dunque un'equazione in $x$ di grado $\left \lfloor \frac{n}{2} \right \rfloor$, la cui risoluzione fornisce i punti $P(x,y)$ per il valore fissato di $y$.
\par Il problema di Pappo \`e dunque stato ricondotto al problema della soluzione di equazioni polinomiali, che Descartes non approfondisce ulteriormente, se non attraverso i metodi di costruzione del libro II, che per\`o hanno poco di algebrico o di geometrico e sono solamente metodi meccanici, in verit\`a di poca utilit\`a.
\par Concludiamo sul problema di Pappo che, se si sceglie di studiare il caso in cui vengono date due rette perpendicolari e gli angoli fissati sono retti, ritroviamo il ben noto piano cartesiano (e aggiungendo una terza retta arriviamo al classico spazio cartesiano).
\par Lo studio del problema di Pappo \`e coerente col metodo cartesiano: Descartes ha prima studiato i segmenti, che egli considera gli oggetti pi\`u semplici, ed ora passa al livello successivo di difficolt\`a, al luogo geometrico, definito tramite rette; in altre parole, Descartes ha inizialmente studiato problemi in un'incognita ed ora passa a problemi in due incognite. Il problema di Pappo suggerisce inoltre un metodo di classificazione dei problemi: noi moderni tendiamo a classificare le curve in funzione del loro grado, Descartes invece classifica i problemi (e non le curve) secondo le curve che sono necessarie per costruire le soluzioni delle equazioni che ne descrivono la soluzione. Ed \`e proprio a problemi di costruzioni e classificazione che \`e dedicato il secondo libro.
\paragraph{La \textit{G\'eom\'etrie} -- Libro II.} Il secondo libro \`e dedicato alla classificazione dei luoghi geometrici e alla loro costruzione. Le costruzioni, in linea col metodo cartesiano, prevedono l'utilizzo di macchinari tanto pi\`u complessi quanto pi\`u complessa \`e la linea da disegnare. Le macchine descritte da Descartes sono piuttosto complicate e probabilmente anche piuttosto scomode da usare: si tratta di macchine ideali, atte a dimostrare l'efficacia della \textit{m\'ethode}, piuttosto che di macchine veramente utili a disegnare linee che nessuno sente veramente la necessit\`a di costruire. La loro necessit\`a deriva dal fatto che il problema di Pappo descritto nel primo libro ha permesso di definire nuove linee, ma la loro costruzione punto per punto non si presta ad uno studio matematico efficace: occorre costruirle praticamente. Tuttavia il legame tra le macchine escogitate da Descartes e il problema di Pappo, non viene mai esplicitato, solo accennato, coerentemente con lo stile cartesiano.
\par Vediamo le due macchine descritte da Descartes: nelle descrizioni che seguono supponiamo tutti gli assi allungabili secondo necessit\`a (stando alle figure della \textit{G\'eom\'etrie}, Descartes invece li supponeva di lunghezza fissa arbitaria e che essi potessero ``uscire'' dallo strumento a volont\`a). La prima, esemplificata dalla figura \ref{Geometrie_Macchina1}, \`e una sorta di generalizzazione del comune compasso: essa \`e munita di due bracci, tracciati in colore blu, che si possono aprire; vi sono poi altri assi organizzati in modo tale
\begin{itemize}
	\item gli assi rossi e gli assi neri mantengono sempre le loro giunture, le quali sono libere di scorrere sui bracci;
	\item gli assi rossi si mantengono sempre perpendicolari al braccio superiore;
	\item gli assi neri si mantengono sempre perpendicolari al braccio inferiore.
\end{itemize}
\par Se manteniamo fisso il braccio inferiore e la prima giuntura sul braccio superiore, le giunture degli assi rossi col braccio superiore descrivono, partendo dalla circonferenza, curve sempre pi\`u complesse (cos\`i si esprime Descartes, senza precisare altro sulla crescente complessit\`a, soddisfatto della coerenza del macchinario col suo metodo). Sfruttando i triangoli simili nella macchina, \`e possibile costruire segmenti in particolari relazioni di medioproporzionalit\`a.
\begin{wrapfigure}{R}{0.5\textwidth}
	\includegraphics[width=0.48\textwidth]{Storia_della_matematica/Immagine_Geometrie_Macchina1.png}
	\caption{Una macchina descritta da Descartes.}
	\label{Geometrie_Macchina1}
\end{wrapfigure}
\par Passiamo alla seconda macchina: la descriviamo con l'aiuto della figura \ref{Geometrie_Macchina2}. Essa prevede
\begin{itemize}
	\item due assi perpendicolari neri;
	\item un asse blu legato agli assi neri in modo tale che la giuntura sull'asse orizzontale sia fissa e quella sull'asse verticale mobile;
	\item una curva rossa fissa che si abbassa o si alza in maniera solidale all'asse blu in modo che la loro distanza misurata sull'asse verticale resti costante.
\end{itemize}
\par La linea viene tracciata considerando il luogo dell'intersezione dell'asse blu con la linea rossa al ruotare dell'asse blu attorno al suo perno. Se la linea rossa \`e un'iperbole otteniamo la \Define{parabola di Cartesio}, che non \`e una parabola bens\`i la parte sinistra della curva di equazione $xy + x^3 + x^2 + x = 1$, che riportiamo in figura \ref{Geometrie_ParabolaCartesiana}.
\begin{wrapfigure}{R}{0.5\textwidth}
	\includegraphics[width=0.48\textwidth]{Storia_della_matematica/Immagine_Geometrie_Macchina2.png}
	\caption{Un'altra macchina descritta da Descartes.}
	\label{Geometrie_Macchina2}
\end{wrapfigure}
\par Sono presenti due nuove classificazioni dei problemi di costruzione geometrica (e quindi, indirettamente, delle curve), entrambe riconducibili alla prima illustrata nel Libro I. La prima classificazione riprende quella greca, la quale distingue tra problemi
\begin{itemize}
	\item \Define{piani}, che richiedono l'uso della riga e del compasso;
	\item \Define{solidi}, che richiedono inoltre l'uso di almeno una sezione conica;
	\item \Define{lineari}, che richiedono inoltre l'uso di qualche altra linea, detta linea \Define{meccanica}.
\end{itemize}
\par Descartes spezza per\`o la classe delle line meccaniche in due classi:
\begin{itemize}
	\item le linee \Define{geometriche}, le uniche accettabili in geometria, che si costruiscono per mezzo di meccanismi del tipo da lui descritto, in cui si collegano rette o curve create direttamente o indirettamente da rette, le quali si muovono continuamente e sincronicamente. Descartes ritiene, a ragione, che tutti i luoghi pappiani siano costruibili per mezzo di linee geometriche e, a torto, che tutte le linee geometriche siano luoghi pappiani;
	\item le linee \Define{meccaniche} (sottoinsieme dunque delle meccaniche dei greci), che si ottengono tramite la combinazione di movimenti indipendenti, come la spirale o la quadratrice.
\end{itemize}
\par Come si vede, la classificazione cartesiana appena descritta \`e puramente geometrica. Descartes propone una seconda classificazione, questa volta algebrica, basandosi sulla descrizione di una linea per mezzo di un'equazione in due incognite raggiunta tramite il problema di Pappo; si tratta dunque di una classificazione delle linee geometriche. Egli definisce il concetto di \Define{genere}: in termini moderni, possiamo dire che il genere di una linea \`e la parte intera della met\`a del suo grado aumentato di $1$. Ma al solito, Descartes, pur fornendo effettivamente, nel linguaggio che gli \`e proprio, la suddetta definizione, ragiona comunque in termini di costruibilit\`a e di complessit\`a progressiva e stabilisce dunque che le linee di genere $1$ sono quelle costrubili con riga, compasso e sezioni coniche, quelle di genere $2$ necessitano di qualche linea pi\`u complicata, quella di genere $3$ di altre linee ancora e cos\`i via, senza maggiori precisazioni.
\par Vi sono diversi motivi per cui Descartes considera il genere di una curva anzich\'e il suo grado:
\begin{wrapfigure}{R}{0.5\textwidth}
	\includegraphics[width=0.48\textwidth]{Storia_della_matematica/Immagine_Geometrie_ParabolaCartesiana.png}
	\caption{La parabola cartesiana.}
	\label{Geometrie_ParabolaCartesiana}
\end{wrapfigure}
\begin{itemize}
	\item la seconda macchina descritta gli ha suggerito, erroneamente, che fosse vero in generale che partendo da una curva di genere $a$ essa costruisse una curva di genere $a + 1$;
	\item egli conosceva una regola, probabilmente la regola oggi attribuita a Ferrari, per abbassare il grado di un'equazione di grado $4$ a quella di un'equazione di grado $3$;
	\item egli ha osservato che aggiungendo la parabola all'insieme delle linee costrubili erano risolubili tutti i problemi di genere $1$ e aggiungendovi la parabola cartesiana tutti i problemi di genere $2$.
\end{itemize}
\paragraph{La \textit{G\'eom\'etrie} -- Libro III.} Descartes non ignora per\`o neanche la classificazione della complessit\`a di una curva secondo il grado della sua equazione: espone questa possibilit\`a nel Libro III. Il terzo libro \`e dedicato alla descrizione dei metodi di risoluzione teorica dei problemi. Si insegna che per attaccare un problema bisogna riuscire ad esprimerlo in termini algebrici secondo quanto visto nel libro I, quindi bisogna abbassare il grado dell'equazione che ne risulta il pi\`u possibile e servirsi di strategie canoniche di risoluzione di queste equazioni. Non mancano considezioni geometriche tuttavia e vi si spiega per esempio come calcolare la normale e dunque la tangente ad una curva (\Cfr\ sottosezione \ref{IMetodiDiDescartesEDiFermat}).
\paragraph{La \textit{G\'eom\'etrie} -- Influenza.} Nel corso della descrizione della \textit{G\'eom\'etrie} abbiamo accennato ad alcuni dei suoi limiti che ora esplicitiamo:
\begin{itemize}
	\item Descartes vede sempre l'algebra come ausiliare alla geometria: le curve restano sempre le uniche protagoniste della teoria e lo scopo \`e la loro costruzione. Egli non compie, al contrario di Fermat, il passo di identificare la curva con la sua equazione;
	\item nonostante l'evidente ricerca di metodi generali, efficaci ma anche semplici, egli non pensa mai a fissare la convenzione degli assi ortogonali;
	\item il suo metodo della tangente non si avvicina minimamente ai concetti del calcolo infinitesimale, e questo di nuovo al contrario di quello di Fermat.
\end{itemize}
\par La \textit{G\'eom\'etrie} \`e vista inizialmente come un testo molto complesso, ma si impone comunque rapidamente grazie soprattutto alle numerose edizioni tradotte in lingua latina ampiamente commentate di van Schoten (1615 - 1660). Molti matematici, principalmente francesi, olandesi ed inglesi, iniziano a studiare da un punto di vista geometrico sistematicamente equazioni in una o due incognite di gradi diversi. Vale la pena inoltre citare De Witt (1625 - 1672) che negli \textit{Elementa curvarum linearum} (1646) per la prima volta dichiara esplicitamente che un'equazione di primo grado in $x$ ed $y$ \Define{\`e} una retta.
