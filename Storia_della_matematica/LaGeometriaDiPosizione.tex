\subsection{La geometria di posizione: da Apollonio a Pappo - Il metodo dell'analisi e sintesi.}\label{LaGeometriaDiPosizione}
\paragraph{Biografia di Apollonio di Perga (circa 262 a.C. - 190 a.C.).} Apollonio nasce a Perga nel regno di Pergamo e si trasferisce giovane ad Alessandria di Egitto dove studia geometria euclidea. Eccelle in geometria, tanto da essere chiamato il ``grande geometra'', e in astronomia. Il suo capolavoro sono le \textit{Coniche}.
\paragraph{La geometria di posizione: le \textit{Coniche}.} La geometria di posizione, a differenza della geometria di misura che abbiamo visto con Archimede, anzich\'e confrontare grandezze (che come sappiamo si identificano con le figure a cui si riferiscono) studia in quali relazioni posizionali si trovino i vari elementi geometrici: tra tali relazioni annoveriamo l'incidenza, il parallelismo, la tangenza, la giacenza. Vediamo alcuni esempi di studio di geometria di posizione nello studio delle coniche.
\par Esponiamo brevemente come venivano definite le coniche prima di Apollonio, per esempio nei lavori di Euclide. Euclide definisce il cono nei suoi \textit{Elementi}\footnote{Ricordiamo che le sezioni coniche, invece, non sono studiate dagli \textit{Elementi}.} come la figura che si ottiene dalla rotazione di un triangolo rettangolo attorno ad uno dei suoi cateti, distinguendo dunque tra
\begin{itemize}
	\item \Define{cono acutangolo}, quando la rotazione avviene attorno al cateto maggiore;
	\item \Define{cono ottusangolo}, quando avviene attorno al cateto minore;
	\item \Define{cono rettangolo}, quando il triangolo rettangolo \`e isoscele.
\end{itemize}
Le sezione coniche si ottengono allora sezionando il cono con un piano perpendicolare al'apotema del cono (l'ipotenusa del triangolo in rotazione): otteniamo
\begin{itemize}
	\item un'\Define{ellise}, quando il cono \`e acutangolo;
	\item un'\Define{iperbole}, quando il cono \`e ottusangolo;
	\item una \Define{parabola}, quando il cono \`e rettangolo.
\end{itemize}
\par Quest'impostazione ha alcuni importanti difetti:
\begin{itemize}
	\item tutti i coni sono retti;
	\item tutti i coni hanno una sola falda e dunque tutte le iperboli hanno un solo ramo;
	\item tutti i coni non solo sono finiti, ma non sono neppure prolungabili\footnote{La matematica greca non concepisce l'infinito attuale, tutt'al pi\`u intuisce quello potenziale. Ci\`o che chiamiamo impropriamente \Define{rette} riferendoci per esempio agli scritti di Euclide sono in realt\`a segmenti, con la caratteristica per\`o di essere prolungabili, ci\`o che giustifica l'uso del nostro linguaggio moderno.}: dato il processo di costruzione, non \`e concepibile nessun concetto di prolungamento;
	\item ad ogni cono corrisponde una ed una sola sezione conica\footnote{Per tipologia e per grandezza. Chiaramente sussiste la possibilit\`a di tagliare un apotema diverso del cono e di tagliare lo stesso apotema in un punto diverso; in particolare, osserviamo che se l'apotema \`e tagliato troppo lontano dal vertice l'ellisse pu\`o risultare ``tagliata'' rispetto a ci\`o che intendiamo solitamente.}.
\end{itemize}
\par Apollonio rivoluziona tutto ci\`o. Dati una circonferenza ed un punto esterno al piano della circonferenza, egli definisce \Define{cono} la figura generata da un retta passante per il punto -- detto \Define{vertice} -- e da qualsiasi altro punto appartenente alla circonferenza facendo scorrere questo secondo punto lungo tutta la circonferenza. Una \Define{sezione conica} \`e la figura individuata tagliando il cono con un piano qualsiasi.
\par La classificazione delle sezione coniche comincia da quelle banali: se il cono viene sezionato da un piano su cui giace l'asse si ottiene un (doppio) triangolo, tutti i piani paralleli al piano della base individuano sezioni circolari. Sono circolari anche le sezioni \Define{subcontrarie}, cio\`e le sezioni che si ottengono da una situazione come quella di figura \ref{Apollonio_subcontraria}, dove il triangolo blu giace su un piano passante per l'asse e perpendicolare alla base del cono ed \`e simile al triangolo rosso.
\begin{wrapfigure}{R}{0.5\textwidth}
	\includegraphics[width=0.48\textwidth]{Storia_della_matematica/Immagine_Apollonio_subcontraria.png}
	\caption{Sezione subcontraria.}
	\label{Apollonio_subcontraria}
\end{wrapfigure}
\par Apollonio passa dunque a classificare le coniche restanti. Egli definisce
\begin{itemize}
	\item \Define{ellisse} la sezione ottenuta da un piano secante due lati del triangolo per l'asse sulla stessa falda;
	\item \Define{iperbole} la sezione ottenuta da un piano secante due lati del triangolo per l'asse su falde diverse;
	\item \Define{parabola} la sezione ottenuta da un piano parallelo ad uno dei lati del triangolo per l'asse (e secante dunque l'altro lato).
\end{itemize}
\par Un altro problema molto interessante di geometria di posizione affrontato da Apollonio \`e quello delle tangenti: Euclide definisce la tangente ad una circonferenza come una retta che tocca la circonferenza senza tagliarla. Si noti che \`e una definizione topologica, cio\`e appunto di geometria di posizione. Inoltre, solo in un secondo momento Euclide dimostra che la tangente alla circonferenza in un punto \`e la perpendicolare al diametro per quel punto e che esiste, per ogni punto della circonferenza, una sola tangente. Allo stesso modo, Apollonio definisce la tangente per le coniche come la retta che tocca la conica lasciandola tutta da una parte, dimostra che \`e perpendicolare al diametro nel punto di tangenza e che per ogni punto della conica esiste una e una sola tangente.
\paragraph{Analisi e sintesi nelle \textit{Coniche}.} Apollonio \`e uno dei primi matematici ad usare il metodo dell'analisi e sintesi: prima di lui il metodo tipico era quello esemplificato da Euclide, esclusivamente sintetico -- si parte dalle premesse e si arriva alle conclusioni, in quest'ordine. Il metodo dell'analisi e della sintesi parte invece dalle conclusioni per arrivare a qualcosa di noto, dunque ritorna alle conclusioni. Apollonio, per quanto a noi noto, non codifica esplicitamente il metodo, neanche approssimativamente come fece Archimede col suo \textit{Metodo meccanico}, ma nelle \textit{Coniche} appaiono diversi esempi di applicazioni. Vediamone uno appoggiandoci alla figura \ref{Apollonio_diametro}.
\par Apollonio desidera costruire un diametro di una conica (in blu), cio\`e una retta che biseca un fascio di corde parallele. Egli suppone di aver gi\`a costruito il diametro (in magenta) e osserva allora che, date le corde rosse corripondenti al diametro magenta, esse sono bisecate dal diametro. Dunque, prese due corde parallele qualsiasi e individuato il loro punto medio, la retta passante per i punti medi \`e un diametro.
\begin{wrapfigure}{R}{0.5\textwidth}
	\includegraphics[width=0.48\textwidth]{Storia_della_matematica/Immagine_Apollonio_diametro.png}
	\caption{Costruzione di un diametro.}
	\label{Apollonio_diametro}
\end{wrapfigure}
\par Il metodo di analisi e sintesi sar\`a successivamente formalizzato da Pappo.
\paragraph{Biografia di Pappo.} Pappo di Alessandria (circa 290 d.C. - circa 350 d.C.) lavora alcuni secoli pi\`u tardi di Apollonio, in un'epoca dominata dalla civilt\`a romana, poco interessata alla matematica che essa limitava ad usare per scopi pratici senza per\`o svilupparla: Pappo \`e uno dei pochi matematici dell'epoca di cui abbiamo notizia che scopre teoremi nuovi (e non \`e certo un caso che lavori ad Alessandria piuttosto che a Roma). \`E ricordato soprattutto per alcuni teoremi e per le \textit{Colletiones mathematicae}, un compendio di matematica in otto libri, di cui purtroppo sono andati perduti il primo libro e parte del secondo.
\paragraph{Analisi e sintesi nelle \textit{Colletiones mathematicae}.} Nelle \textit{Colletiones}, Pappo cerca di descrivere in cosa consiste il metodo dell'analisi e della sintesi. La spiegazione \`e piuttosto oscura, ma alcuni fatti sembrano chiari:
\begin{itemize}
	\item l'analisi parte da ci\`o che si desidera dimostrare o costruire e arriva a qualcosa di noto;
	\item la sintesi \`e il processo reciproco dell'analisi: parte dal qualcosa di noto raggiunto dall'analisi e torna a ci\`o che si desidera dimostrare o costruire;
	\item la sintesi consiste nel ribaltamento solamente delle premesse e delle conclusioni, non di ogni singolo passaggio: se cos\`i non fosse non ci sarebbe alcun bisogno di ribaltare esplicitamente un'analisi nel caso di una dimostrazione (persisterebbe comunque la necessit\`a di ribaltamento nel caso di una costruzione);
	\item l'analisi \`e metodologicamente prioritaria rispetto alla sintesi, ma, essendo il punto di partenza dell'analisi lo scopo della sintesi, la sintesi \`e logicamente o ontologicamente prioritaria;
	\item l'analisi, secondo Pappo, porta a un fatto noto vero quando l'ipotesi \`e vera e falso quando l'ipotesi \`e falsa: questo mostra come l'analisi comprenda la tecnica di dimostrazione per assurdo, ma anche che Pappo \`e inconsapevole della possibilit\`a di ottenere conclusioni giuste da premesse sbagliate.
\end{itemize}
