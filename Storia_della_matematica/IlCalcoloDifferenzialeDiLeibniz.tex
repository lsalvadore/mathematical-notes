\subsection{Il problema delle tangenti e il calcolo differenziale di Leibniz.}\label{IlCalcoloDifferenzialeDiLeibniz}
\par Come abbiamo visto nella sottosezione \ref{IMetodiDiDescartesEDiFermat}, nel Seicento cominciano a farsi strada diversi metodi per il calcolo delle tangenti, in particolare il metodo di Descartes e quello di Fermat, dai quali derivano varie regole per trovare le tangenti ai polinomi: questi ultimi si ispirano prevalentemente all'impostazione di Fermat, ma la loro preferenza per i polinomi \`e chiaramente di origine cartesiana. Tutti questi metodi hanno per\`o l'inconveniente di diventare rapidamente molto complessi, troppo complessi per essere portati fino in fondo, a meno di munirsi di molto tempo e di molta pazienza: basta introdurre un radicale o considerare una funzione fratta per andare incontro a complicate operazioni, consistenti appunto nella rimozioni dei radicali e delle frazioni. Il metodo di Leibniz invece intoduce i \Define{differenziali} (che lui chiama \Define{differenze}, ma noi usiamo il termine moderno), i quali consentono di spezzare il problema complesso in tanti piccoli problemi pi\`u semplici.
\par Il calcolo differenziale di Gottfried Wilhelm Leibniz (1646 - 1716) nasce nel 1684 con la memoria \textit{Nova methodus pro maximis et minimis, itemque tangentibus, quae nec fractas nec irractionalas quantites moratur, et singulare pro illis calculi genus}, pubblicata sugli \textit{Acta Eruditorum}. Leibniz fissa un riferimento cartesiano con assi ortogonali e vi considera una generica curva $\gamma$: l'asse delle ordinate viene scelto comodamente passante per il punto di $\gamma$ in cui si desidera definire i differenziali. Egli sceglie il differenziale $dx$ come un segmento orizzontale arbitrario, quindi stabilisce che il differenziale $dy$ corrispondente \`e il segmento che verifica la proporzione $dy : dx = OP : OT$ (\Cfr\ figura \ref{Leibniz_Differenziale}).
\begin{wrapfigure}{R}{0.5\textwidth}
	\includegraphics[width=0.48\textwidth]{Storia_della_matematica/Immagine_Leibniz_Differenziale.png}
	\caption{Definizione del differenziale di Leibniz.}
	\label{Leibniz_Differenziale}
\end{wrapfigure}
\par \`E importante notare che i differenziali, sebbene nella teoria di Leibniz abbiano lo stesso carattere infinitesimo che gli attribuiamo oggi, sono grandezze finite. Tuttavia, questo punto importante viene contraddetto indirettamente pi\`u avanti, quando Leibniz definisce la tangente ad una curva, retta che come abbiamo visto interviene nella definizione dei differenziali: la tangente \`e quella retta che congiunge due punti infinitamente vicini della curva, ovvero il lato del poligono ``ad infiniti lati'' che equivale alla curva. Chiaramente, Leibniz ha intuito un concetto di limite, ma lo considera solo da un punto di vista geometrico, mentre mantiene l'algebra in un ambito finito.
\par E proprio con l'algebra prosegue la memoria, dopo aver illustrato brevemente alcuni usi del differenziale (studio della crescenza di una curva, dei suoi massimi e minimi relativi, della concavit\`a): Leibniz compie il passo rivoluzionario di ricavare le regole a noi ben familiari di calcolo per il differenziale della somma di grandezze, del prodotto, del rapporto, della radice, delle potenze. Tutto ci\`o gli consente di esibirsi nella ricerca di quei massimi e minimi che prima erano inaccessibili, esibizione che egli ripete pi\`u volte anche nei suoi scambi epistolari per dimostrare l'efficacia dei suoi differenziali.
\par L'introduzione dei differenziali ha per naturale conseguenza la ricerca di grandezze i cui differenziali sono in una data relazione o, per dire la stessa cosa in termini pi\`u familiari, lo studio di equazioni differenziali. Ci vuole del tempo perch\'e si imponga il teorema fondamentale del calcolo integrale, ma Evangelista Torricelli (1608 - 1647) e Isaac Barrow (1630 - 1677) riescono ad anticiparlo abbastanza presto (tanto presto da precedere entrambi la memoria leibniziana: ambedue morirono prima della sua pubblicazione), motivo per cui il teorema \`e intitolato in loro onore.
\par Segnaliamo infine l'importanza delle notazioni Leibniziane: la $d$ del differenziale si presta bene ad essere interpretata come un'operazione, anche se la definizione del differenziale stesso non lo rende subito evidente. Questa ed altre notazioni, come il simbolo $\int$ di integrale, sono dovuti a Leibniz ed in uso ancora oggi.
