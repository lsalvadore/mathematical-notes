\subsection{Caratteri generali della matematica greca.}\label{CaratteriGeneraliDellaMatematicaGreca}
\par In questa sottosezione discutiamo la matematica greca nel periodo che va dal VI secolo a.c. al VI secolo d.c. In particolare distinguiamo due periodi:
\begin{itemize}
	\item il \Define{periodo classico}: dal VI secolo a.c. al 323 a.c. (morte di Alessandro Magno):
	\item il \Define{periodo ellenistico}: dal 323 a.c. fino al VI secolo d.c.
\end{itemize}
\par Nel periodo classico si distinguono nettamente due tipi di approccio alla matematica: un approccio \`e pragmatico, sorge dai problemi della vita quotidiana di commercianti, artigiani e funzionari; l'altro \`e filosofico e nasce da speculazioni filosofiche.
\par Ben poco \`e noto dell'approccio pragmatico. Sembrano distinguirsi in particolare due classi di matematici pragmatici:
\begin{itemize}
	\item i \Define{``calcolatori''}: si tratta di figure che svolgono calcoli aritmetici per professione, servendosi di una tavola con funzioni simili a quelle del pi\`u moderno abaco. Si pensa che essi abbiano tratto le loro conoscenze dalla Mesopotamia (erano in uso da tempi pi\`u antichi le stesse tavole) e dall'Egitto (il loro modo di trattare le frazioni ha somiglianze col metodo egizio). Hanno verosimilmente influenzato i pitagorici;
	\item i \Define{``misuratori''}: si occupano di calcolare aree e volumi. Essi hanno una tradizione scritta, le cui tracce si ritrovano per\`o pi\`u nel Vicino Oriente che in Grecia: i loro testi somigliano molto a ricette di cucina, indicando istruzioni con una sintassi precisa, imperativi e sempre tramite esempi numerici concreti, mai attraverso formule generali; le istruzioni sono generalmente accompagnate da diagrammi esplicativi. Si pensa che a svolgere la funzione di ``misuratore'' siano inizialmente stranieri arrivati in Grecia portando con loro il proprio sapere matematico: successivamente da essi i greci imparano a loro volta queste competenze. \`E ragionevole pensare che i ``misuratori'' si organizzino in corporazioni che cercano di mantenere segrete le loro tecniche e che essi siano di estrazione sociale piuttosto bassa.
\end{itemize}
\par Sia ``calcolatori'' che ``misuratori'' sviluppano necessariamente una certa attitudine teorica, quasi certamente inconsapevole, i primi notando nei loro calcoli alcuni schemi ricorrenti, i secondi riadattanto ogni ``ricetta'' ai particolari casi numerici che sono chiamati a risolvere. Non sviluppano mai tuttavia concetti fondamentali della matematica quali il concetto di definizione, di teorema o di dimostrazione.
\par Per quanto riguarda l'approccio filosofico invece distinguiamo diverse scuole; in ciascuna di esse la materia di studio \`e la filosofia (all'epoca concentrata sulla comprensione dell'Essere) e la matematica si configura come un ramo della filosofia (da usare come strumento per la comprensione dell'Essere):
\begin{itemize}
	\item la \Define{scuola ionica}: il principale esponente \`e Talete di Mileto\footnote{Una citt\`a della Ionia, regione dell'Anatolia da cui prende il nome la scuola.} che vive verosimilmente nel VI secolo a.c. Talete \`e pi\`u importante come iniziatore della filosofia che della matematica, tuttavia gli si attribuiscono teoremi sui triangoli simili, che si ritiene egli usi per calcolare l'altezza delle piramidi;
	\item la \Define{scuola pitagorica}: fondata a Crotone da Pitagora (VI secolo a.c.), probabile allievo di Talete. La scuola \`e attiva in ambito matematico per il VI e V secolo, occupandosi prevalentamente di aritmetica, sia pure arrangiando ciottoli in forme geometriche (da qui espressioni ancora in uso oggi come ``al quadrato'' e ``al cubo''). La scuola pitagorica \`e la prima ad iniziare lo studio matematico della musica. La matematica pitagorica ha una forte valenza mistica-religiosa: i seguaci di Pitagora credono che ogni cosa in Natura sia numero (nel senso di numero naturale) e che la Natura si possa dunque spiegare in termini di numeri e rapporti fra numeri: \`e dalla scoperta dell'incommensurabilit\`a della diagonale di un quadrato col lato, ad opera di Ippaso da Metaponto (V secolo a.c.), che si pone per la prima volta il problema della legittimit\`a dei numeri -- per i pitagorici, e per tutta la matematica classica (dell'approccio filosofico), solo i numeri razionali positivi sono legittimi;
	\item la \Define{scuola eleatica}: il principale esponente di questa scuola \`e Zenone di Elea (V secolo a.c.), probabilmente originalmente un pitagorico come il suo maestro Parmenide. Egli esamina per la prima volta il concetto di ``continuo'', principalmente attraverso paradossi (il pi\`u famoso dei quali \`e probabilmente quello di Achille e la tartaruga), cio\`e ragionamenti che sembrano convincenti ma si scontrano con la realt\`a del mondo sensible: non avendo sviluppato il concetto di limite matematico, Zenone ne conclude che \`e il mondo sensibile ad essere fallace, rafforzando le gi\`a solide basi del pensiero che sfocia nella scuola platonica;
	\item la \Define{scuola sofista}: si sviluppa ad Atene nel V secolo a.c. Dal punto di vista matematico, i sofisti sono i primi a formulare problemi di costruzione geometrica, ponendo di fatto il requisito che queste costruzioni siano sempre eseguite con solamente riga e compasso. Lavorano principalmente al problema della duplicazione del cubo, della quadratura del cerchio e della trisezione dell'angolo;
	\item la \Define{scuola platonica}: fa riferimento a Platone (IV - III secolo a.c.) e succede ai sofisti come scuola dominante ad Atene. Ha una forte impronta pitagorica e il platonismo \`e la filosofia che pi\`u nettamente distingue mondo sensibile e mondo ideale: stabilisce che l'uomo intellettualmente elevato deve occuparsi di comprendere le idee, in quanto vere ed immmutabili, mentre le cose sono materia da lasciare all'uomo volgare; \`e all'Accademia platonica che si fa pi\`u drastica la distinzione tra approccio filosofico -- riservato ai cittadini nel senso ateniese del termine, cio\`e alla classe dirigente o aristocratica -- e l'approccio pragmatico -- tipico delle attiviti\`a considerate volgari e indegne di un uomo elevato, come ad esempio il commercio. La matematica assume il ruolo di attivit\`a iniziatica allo studio delle idee ed \`e proprio cos\`i che si fa spazio per la prima volta il concetto di dimostrazione in forma consapevole: il filosofo-matematico platonico fissa delle ipotesi che considera evidenti (cio\`e non possono essere un inganno del mondo sensibile) e attraverso un ragionamento deduttivo arriva ad una nuova verit\`a, non necessariamente evidente. Questa operazione di deduzione riesce pi\`u naturale in ambito geometrico, sia per le influenze pitagoriche che per la presa di distanze dall'approccio pragmatico basato sui numeri: \`e l'inizio del primato matematico della geometria sull'aritmetica, che rimarr\`a tale per secoli, sino alla nascita dell'algebra e la sua successiva integrazione con la geometria attraverso i lavori di Descartes e Fermat. I platonici sono i primi a studiare le sezioni coniche e aprono il loro insieme di numeri legittimi a quei numeri irrazionali che possono essere costruiti geometricamente;
	\item la \Define{scuola di Eudosso}: Eudosso (IV secolo a.c.) \`e considerato uno dei pi\`u grandi matematici dell'Antichit\`a. Discepolo di Platone, egli introduce una nuova teoria delle proporzioni e delle grandezze per legittimizzare gli irrazionali incontrati nel corso di costruzioni geometriche: la sua idea \`e quella di aggirare il concetto di numero -- non si assegna dunque pi\`u un numero al lato di un quadrato calcolando poi un rapporto irrazionale della diagonale sul lato; si considera invece direttamente il rapporto, che \`e uguagliabile ad altri rapporti, senza per\`o mai passare dal concetto di numero. \`E il passo decisivo che cancella i numeri dalla geometria, sino come gi\`a detto all'avvento della geometria cartesiana; si continuer\`a tuttavia a studiare la teoria dei numeri (naturali e razionali positivi), su base deduttiva come per la geometria, ma separatamente da questa (si vedano gli \textit{Elementi} di Euclide);
	\item la \Define{scuola aristotelica}: Aristotele (IV secolo a.c.) \`e un filosofo noto per il suo approccio enciclopedico, ed anche in matematica notiamo questa sua caratteristica. \`E il primo filosofo a precisare il concetto di definizione (anche in senso matematico) e a distinguire il concetto di definizione dalla propriet\`a di esistenza: non basta definire una cosa per dimostrare che essa esiste. Egli stabilisce come criterio di esistenza di una figura (ricordiamo il primato della geometria: esso \`e il motivo per cui la questione \`e affrontata solo in ambito geometrico) la sua costruibilit\`a con riga e compasso (eredit\`a sofista): ad esempio, trovare una procedura di costruzione del quadrato implica l'esistenza del quadrato. Aristotele \`e anche il primo a definire il concetto di assioma: egli distingue tra \Define{assiomi} -- verit\`a generali evidenti come le leggi logiche -- e \Define{postulati} -- affermazioni non necessariamente evidenti e meno generali, adatte come punto di partenze per una particolare scienza, e la cui verit\`a pu\`o apparire successivamente nello sviluppo della teoria dalle conseguenze che se ne deducono. Euclide riprende la distinzione aristotelica tra assioma e postulato, ma in seguito i due concetti verranno trattati come identici perdendo la distinzione.
\end{itemize}
\par Euclide (III secolo a.C.) e Apollonio (III secolo a.C. - II secolo a.C.) cronologicamente appartengono al periodo ellenistico, tuttavia per il contenuto delle loro opere, che approfondiremo pi\`u tardi, sono considerati matematici classici.
\par Il periodo ellenistico si caratterizza per una sintesi dell'approccio pragmatico e di quello filosofico, che sfocia in ci\`o che potremmo chiamare la matematica applicata: \`e in effetti soprattutto in periodo ellenistico che si sviluppano scienze come l'astronomia, l'ottica o la meccanica, sebbene questi argomenti fossero gi\`a stati esaminati anche in periodo classico. I matematici ellenistici si curano meno delle giustificazioni teoriche del loro lavoro e usano per esempio i numeri irrazionali senza esitazioni; inoltre rivalutano il primato della geometria, dando pi\`u spazio ai numeri. Di tutto ci\`o sono emblematici i lavori di Archimede (III secolo a.C.) -- che si occupa, fra l'altro, di calcolare aree e volumi, di leve, di ottica (problemi pragmatici) sfruttando i metodi della matematica degli \textit{Elementi} di Euclide -- di Tolomeo (II secolo d.C.), il quale pu\`o essere considerato il padre della trigonometria (una materia geometrica) e non manca di costruire le prime tavole trigonometriche (con valori numerici), e di Diofanto (IV secolo a.C. - III secolo a.C.), che introduce per la prima volta un'algebra simbolica e studia sistematicamente equazioni che chiamiamo oggi appunto diofantee (con metodi deduttivi).
