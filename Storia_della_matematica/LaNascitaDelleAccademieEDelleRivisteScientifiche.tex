\subsection{La nascita delle accademie e delle riviste scientifiche.}\label{LaNascitaDelleAccademieEDelleRivisteScientifiche}
\paragraph{La nascita delle accademie.} Le accadamie sorgono in Europa nella seconda met\`a del XVII secolo. Si tratta di associazioni, nel caso di nostro interesse di scienziati, che lavorono insieme per promuovere la ricerca. Sino al XV secolo gli scienziati erano individui, lavoravano essenzialmente da soli, eccezion fatta per qualche scambio epistolare, e ognuno aveva i propri metodi, le proprie credenze, i propri interessi: tutto questo cambia con le accademie. Queste nuove comunit\`a scientifiche vengono a costituire una sorta di ``scienziato collettivo'': ci\`o che conta non sono pi\`u le teorie dei singoli scienziati, bens\`i quello che la comunit\`a scientifica accetta o respinge.
\par Accademie significative nate nel XVII secolo sono
\begin{itemize}
	\item l'\textit{Accademia dei Nazionale dei Lincei} italiana nata (in Umbria) nel 1603;
	\item la \textit{Royal Society} inglese nata nel 1661;
	\item l'\textit{Acad\'emie des Sciences} francese nata nel 1666.
\end{itemize}
\par Alcune accademie sono pubbliche, altre private, altre ancora nascono private e diventano pubbliche: quest'ultimo \`e il caso della \textit{Royal Society} nata da un gruppo di studiosi nel 1645 presso il Gresham College.
\paragraph{La nascita delle riviste scientifiche.} Simultaneamente alle accademie nascono le prime riviste scientifiche. Abbiamo inizialmente
\begin{itemize}
	\item resoconti dell'attivit\`a delle accademie: \`e il caso delle \textit{Philosophical Transactions of the Royal Society} (1665) e dei \textit{M\'emoires de l'Acad\'emie Royale des Sciences};
	\item pubblicazioni private: per esempio il \textit{Journal des S\c{c}avans} (1666);
	\item raccolte di recensioni: prima il \textit{Giornale de' Letterati} (1668 - 1680 a Roma, ricomparso il secolo successivo in varie forme e varie citt\`a), ma poi, soprattutto gli \textit{Acta Eruditorum} (Lipsia, 1682).
\end{itemize}
\par Tuttavia la comunit\`a scientifica si accorge rapidamente delle nuove possibilit\`a offerte dalla pubblicazione delle riviste e inizia a utilizzarle per comunicare risultati nuovi e originali, in maniera pi\`u semplice di quella che avevano consentito fino ad allora i volumi comuni: un esempio molto significativo \`e quello della memoria di Leibniz fondatrice del calcolo infinitesimale.
\par Le riviste scientifiche contribuiscono in maniera decisiva all'affermazione del cosiddetto scienziato collettivo.
