\subsubsection{La meccanica: da Giordano a Galileo.}\label{LaMeccanicaDaGiordanoAGalileo}
\paragraph{Giordano Nemorario (XIII secolo).} Gli antichi, sia pure con notevoli eccezioni come quella di Archimede, si sono dedicati poco allo studio della meccanica: Aristotele ha illustrato una meccanica che spiega le cause del moto e nulla di pi\`u viene ricercato: gli oggetti hanno delle qualit\`a, come la pesantezza, e vanno laddove le loro qualit\`a li spingono; l'aria e il fuoco sono dotati di leggerezza e vanno verso l'alto, l'acqua e la terra di pesantezza e vanno verso il basso. Le teorie fisiche rimangono pressoch\'e immutate sino al XIII secolo. Giordano Nemorario \`e uno dei primi a cambiare le cose.
\par Poco si sa della vita di Giordano Nemorario, ma i suoi scritti sono senza dubbio molto influenti: essi sono citati da Ruggero Bacone (1214 - 1294) e sono fra i primi testi dati alle stampe nel Cinquecento. Giordano, studiando le leve, inventa il concetto di \textit{gravitas secundum situm}, che possiamo confrontare con la \textit{gravitas secundum speciem} esposta nel \textit{Liber archimenedis de ponderibus}, di autore ignoto:
\begin{itemize}
	\item secondo la \textit{gravitas secundum situm} un corpo \`e tanto pi\`u pesante quanto pi\`u la sua ``linea di moto'' \`e prossima alla verticale; per esempio \`e molto faticoso sollevare un masso lungo la verticale, ma lo \`e molto meno spingerlo orizzontalmente o lungo un piano inclinato; tanto pi\`u \`e inclinato il piano, cio\`e la sua linea di moto \`e vicina alla verticale, tanto pi\`u il corpo \`e pesante. Sulla base di questo principio, che Giordano adopera come un assioma, egli riesce a dimostrare alcuni risulati sui piani inclinati;
	\item la \textit{gravitas secundum speciem} invece anticipa il concetto di peso specifico: questo concetto stabilisce che per confrontare il peso di due materie si devono paragonarne quantit\`a uguali in volume. Manca la definizione esplicita di peso specifico come rapporto, ma l'idea \`e introdotta.
\end{itemize}
\paragraph{Il Cinquecento.} Nel Cinquecento ricompare un trattato pseudoaristotelico intitolato \textit{Quaestiones Mechanicae}: la prima pubblicazione, in greco, avviene a Venezia nel 1497; seguono una traduzione in latino a Parigi di Vittore Fausto nel 1517 e vari commenti e parafrasi. Notevoli sono la parafrasi di Alesandro Piccolomini (Roma, 1547) -- che attribuisce in introduzione alla meccanica, sino ad allora considerata un'arte manuale, la dignit\`a di scienza che consente all'Uomo di superare con le sue forze le forze della Natura -- e la parafrasi di Tartaglia in \textit{Quesiti et inventioni diverse} (Venezia, 1546 e 1554), che riprende anche le teorie di Giordano sui pesi.
\par Tutto ci\`o che abbiamo descritto sin'ora, da Giordano Nemorario ai \textit{Quesiti et inventioni diverse}, ha poco in comune con la filosofia naturale degli antichi: il primo a tentare una sintesi tra l'impostazione aristotelica e quella archimedea (e neoarchimedea) \`e Guidobaldo dal Monte (1545 - 1607), allievo di Commandino, amico e mentore di Galileo. Guidobaldo tenta, in chiave moderna, di ricondurre le macchine alla geometria nel suo \textit{Mechanicorum liber} (Pesaro, 1547; e poi, in volgare, Venezia, 1581), riconducendole tutte alla leva. Tuttavia, egli si dimostra troppo aderente ai classici, tanto da riportare il risultato sbagliato di Pappo sul piano inclinato anzich\'e quello corretto di Giordano.
\par Pi\`u tardi, nel 1588, Guidobaldo pubblica a Pesaro, proseguendo l'opera archimedea di Commandino, il \textit{Paraphrasis in duos Archimedis aequiponderatium libros}, col quale egli stabilisce la concordanza di Aristotele e Archimede: essi affermano, secondo lui, le stesse cose; laddove il filosofo descrive la macchina attraverso le cause, il matematico la descrive ponendo gli assiomi, in un quadro di ``\textit{maximus consensus}''. Cos\`i Guidobaldo riporta la meccanica nell'ambito della geometria tradizionale, non senza scontrarsi tuttavia col problema delle grandezze derivate che i nuovi lavori hanno cominciato a proporre, come la \textit{gravitas secundum situm} e quella \textit{secundum speciem}: il passo successivo sar\`a compiuto da Galileo.
\paragraph{Galileo Galilei (1564 - 1642).} In un primo momento, col suo \textit{De motu antiquiora}, Galileo rompe con la tradizione aristotelica: egli \`e intento e studiare il moto dei gravi e dei leggeri e considera la possibilit\`a di abbandonare l'idea aristotelica dei leggeri, lasciando solo i gravi, da trattare archimedicamente. \`E una delle prime volte che il matematico non solo descrive il moto, ma ne considera anche le cause; per\`o nei tentativi precedenti si sono presi come punti di partenza i teoremi archimedei: Galileo invece tenta di dimostrare anche quelli.
\par La sua idea \`e quella di partire dai \textit{Galleggianti}: per il giovane Galileo, i corpi leggeri salgono perch\'e scacciati da quelli pi\`u pesanti che premono su di loro. Egli tenta di dimostrare la legge del galleggiamento riconducendola a quella della leva (e dunque l'Archimede dei \textit{Galleggianti} \`e s\'i messo in discussione, ma per essere recuperato attraverso l'Archimede dell'\textit{Equilibrio dei piani}), ma si accorger\`a pi\`u tardi egli stesso di essersi sbagliato. I suoi studi gli permettono comunque di ristabilire la legge corretta per il piano inclinato e costituiscono il primo passo per un modello matematico della Natura completo.
\par La carriera di Galileo prosegue con le lezioni di meccanica a Padova, dove egli insegna con un tale successo che gli appunti sui suoi corsi vengono ampiamente diffusi, bench\'e egli non li publicasse. Nel suo \textit{Mecaniche}, di impostazione simile al guidobaldiano \textit{Mechanicorum liber}, egli rompe definitivamente con la filosofia della Natura: egli si rivolge ad architetti ed ingegneri, non pi\`u a filosofi, e nega che gli effetti delle macchine siano inganni nei confronti della natura. Galileo inoltre introduce per la prima volta, accanto al concetto di gravit\`a, quello di momento, definito come propensione di un corpo ad andare verso il basso: il linguaggio non \`e ancora quello moderno del prodotto vettoriale di oggi, ma egli ha comunque piena consapevolezza del concetto, superando la \textit{gravitas secundum situm} di Giordano.
\par Galileo scriver\`a successivamente i famosi lavori che hanno segnato la Storia della Scienza nello stesso spirito. Va sottolineato per\`o che egli non supera mai l'impostazione archimedea: il suo linguaggio, anche mentre introduce grandezze derivate nuove come appunto il momento, o anche la velocit\`a, l'accelerazione, la forza, resta sempre quello delle proporzioni. Come Valerio, anche Galileo si ferma laddove serve un linguaggio nuovo: il calcolo infinitesimale.
