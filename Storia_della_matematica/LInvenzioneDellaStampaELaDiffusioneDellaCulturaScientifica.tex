\subsubsection{L'invenzione della stampa e la diffusione della cultura scientifica.}\label{LInvenzioneDellaStampaELaDiffusioneDellaCulturaScientifica}
\paragraph{Primato veneziano.} Intorno all'anno 1450 viene inventata la stampa e in breve tempo Venezia si afferma come una delle pi\`u importanti capitali europee dell'editoria umanistica: tra il 1470 e il 1500 si calcola che a Venezia lavorassero circa 200 tipografie, contro le 60 di Milano, le 30 di Firenze; anche oltr'alpe il confronto \`e impressionante considerando che circa 200 erano le tipografie nell'intera Germania e a 150 ammontavano quelle di Lione e Parigi insieme. Questo \`e dovuto a numerosi fattori:
\begin{itemize}
	\item l'umanesimo ha coinvolto molte citt\`a, tra le quali Firenze, Roma, Urbino e appunto Venezia. A Venezia il cardinale Basilio Bessarione (1403 - 1472), metropolita di Nicea prima del suo trasferimento in Italia nel 1440, ha raccolto un gran quantit\`a di manoscritti greci e li dona alla sua morte a San Marco, accessibile al pubblico dalla met\`a del Cinquecento e di cui si occupa anche Pietro Bembo (1470 - 1547);
	\item la posizione geografica \`e ideale per il reperimento dei manoscritti e per la diffusione delle loro edizioni;
	\item la Repubblica di Venezia incoraggia e promuove la stampa;
	\item il Veneto \`e ricco di cartiere;
	\item il vicino Studio di Padova contribuiva alla domanda di testi, sia per gusto che per l'insegnamento;
	\item un alto livello di istruzione dovuto alla presenza di
	\begin{itemize}
		\item scuole d'abaco;
		\item la Scuola di Rialto, che diventa pubblica intorno alla met\`a del Quattrocento, dedita alla filosofia, alla logica, alla geometria;
		\item la Scuola di San Marco, dedita alla grammatica, alla retorica, alla filologia.
	\end{itemize}
	\item la stampa a caratteri mobili arriva a Venezia nel 1469, ma gi\`a prima si stampavano carte da gioco, immagini religiose e anche libretti scolastici.
\end{itemize}
\paragraph{Regiomontano e Ratdolt.} Johannes M\"uller (1436 - 1476), meglio noto come Regiomontano, \`e il primo dei grandi editori scientifici. Egli fa parte del circolo di Bessarione e si convince rapidamente dell'importanza di stampare gli antichi manoscritti matematici greci e non solo: egli apre una tipografia a Norimberga nel 1472 e stampa un programma che comprende numerosi autori greci, tra cui Euclide e Archimede, medioevali, come Giordano Nemorario, e scritti nuovi e originali. Ma la morte prematura di Regiomontano gli impedisce di concretizzare il progetto: \`e Erhardt Ratdolt (1442 - 1528) a raccoglierne l'eredit\`a, tipografo tedesco che lavora a Venezia dal 1475.
\par Purtroppo per\`o, a differenza di Regiomontano, Ratdolt non ha le conoscenze matematiche necessarie per realizzare lo scopo consapevolmente ed egli produce dunque testi come i gi\`a commentati \textit{Elementi} di Euclide (\Cfr\ sottosezione \ref{EuclideEGliElementi}). Le sue tecniche tipografiche sono per\`o eccellenti, soprattutto per le sue stampe iconografiche: egli stampa immagini astronomiche, immagini policrome, impiega l'oro.
\par Dopo Ratdolt, altri editori intraprendono progetti di edizioni enciclopediche:
\begin{itemize}
	\item Giorgio Valla (1447 - 1500) pubblica \textit{De expedentis et fugiendi rebus opus}, ricco di molte nuove traduzioni dal greco in latino anche di testi matematici;
	\item Paganino de' Paganini pubblica la \textit{Summa de arithmetica, geometria, proportioni et proportionalita} di Luca Pacioli (1445 - 1517), in lingua volgare, con cui l'autore mira a superare la matematica pragmatica delle scuole d'abaco reintegrandola col sapere teorico. La \textit{Summa} include molti passi dell'edizione campana degli \textit{Elementi} di Euclide; oggi Pacioli viene ``accusato'' di aver plagiato i risultati dei maestri d'abaco, tra i quali quelli del suo maestro Piero della Francesca, bench\'e si debba tenere a mente che \`e proprio con l'invenzione della stampa che nascono i primi concetti legati al diritto d'autore\footnote{A Tartaglia per esempio vengono riconosciuti a Venezia diritti esclusivi di stampa della propria opera.}.
\end{itemize}
