\section{Definizioni e teoremi di base.}\label{DefinizioniETeoremiDiBase}
\begin{Definition}
	Chiamiamo
	\Define{problema decisionale}[decisionale][problema] o
	\Define{problema di esistenza}[di esistenza][problema]
	il dato di un insieme $\AdmissibleSet$ detto
  \Define{insieme ammissibile}[ammissibile di un problema di ottimizzazione][insieme],
  che dipende da certi dati detti
  \Define{parametri}[di un problema di ottimizzazione][parametro] o
  \Define{varabili}[di un problema di ottimizzazione][variabile]:
  la specifica di tali parametri produce una particolare
  \Define{istanza}[di un problema di ottimizzazione][istanza].
  Gli elementi di $\AdmissibleSet$ si chiamano
  \Define{soluzioni ammissibili}[ammissible di un problema di ottimizzazione][soluzione];
  se inoltre $\AdmissibleSet$ \`e espresso come sottoinsieme
  di un insieme $\AdmissibleSet'$, allora gli elementi di
  $\AdmissibleSet' - \AdmissibleSet$ vengono talvolta chiamate
  \Define{soluzioni non ammissibili}[non ammissible di un problema di ottimizzazione][soluzione].
  Risolvere un problema decisionale significa determinare se esistono soluzioni ammissibili.
\end{Definition}
\begin{Definition}
	Un problema decisionale $\DecisionalProblem$ si dice \Define{vuoto}[decisionale][problema] quando $\AdmissibleSet = \emptyset$.
\end{Definition}
\begin{Definition}
	Chiamiamo
	\Define{problema di certificato}[di certificato][problema]
	il dato di
	\begin{itemize}
		\item un insieme $\AdmissibleSet$ detto
		\Define{insieme ammissibile}[ammissibile di un problema di ottimizzazione][insieme],
		che dipende da certi dati detti
		\Define{parametri}[di un problema di ottimizzazione][parametro] o
		\Define{varabili}[di un problema di ottimizzazione][variabile]:
		la specifica di tali parametri produce una particolare
		\Define{istanza}[di un problema di ottimizzazione][istanza].
		Gli elementi di $\AdmissibleSet$ si chiamano
		\Define{soluzioni ammissibili}[ammissible di un problema di ottimizzazione][soluzione];
		$\AdmissibleSet$ \`e espresso come sottoinsieme
		di un insieme $\AdmissibleSet'$ e gli elementi di
		$\AdmissibleSet' - \AdmissibleSet$ vengono talvolta chiamate
		\Define{soluzioni non ammissibili}[non ammissible di un problema di ottimizzazione][soluzione];
		\item un elemento $x \in \AdmissibleSet'$
	\end{itemize}
	Risolvere un problema di certificato significa determinare se $x$ \`e o meno una soluzione ammissibile.
\end{Definition}
\begin{Definition}
	Chiamiamo
	\Define{problema di ottimizzazione}[di ottimizzazione][problema]
	il dato di
	\begin{itemize}
		\item un insieme $\AdmissibleSet$ detto
		\Define{insieme ammissibile}[ammissibile di un problema di ottimizzazione][insieme],
		che dipende da certi dati detti
		\Define{parametri}[di un problema di ottimizzazione][parametro] o
		\Define{varabili}[di un problema di ottimizzazione][variabile]:
		la specifica di tali parametri produce una particolare
		\Define{istanza}[di un problema di ottimizzazione][istanza].
		Gli elementi di $\AdmissibleSet$ si chiamano
		\Define{soluzioni ammissibili}[ammissible di un problema di ottimizzazione][soluzione];
		se inoltre $\AdmissibleSet$ \`e espresso come sottoinsieme
		di un insieme $\AdmissibleSet'$, allora gli elementi di
		$\AdmissibleSet' - \AdmissibleSet$ vengono talvolta chiamate
		\Define{soluzioni non ammissibili}[non ammissible di un problema di ottimizzazione][soluzione].
		\item una \Define{funzione obiettivo}[obiettivo][funzione]
		o \Define{funzione di costo}[di costo]{funzione}
		$\ObjectiveFunction: \AdmissibleSet \rightarrow \mathbb{R}$;
		\item l'indicazione di se il problema \`e
		\Define{di minimo}[di minimo][problema] o
		\Define{di massimo}[di massimo][problema]: chiameremo
		\Define{soluzione ottimale}[ottimale di un problema di ottimizzazione][soluzione]
		una soluzione ammissibile $\OptimalSolution$ tale che
		\begin{itemize}
			\item $\ObjectiveFunction(\OptimalSolution) =
			\min_{\AdmissibleSolution \in \AdmissibleSet}
			\ObjectiveFunction(\AdmissibleSolution)$ se il
			problema \`e definito di minimo;
			\item $\ObjectiveFunction(\OptimalSolution) =
			\max_{\AdmissibleSolution \in \AdmissibleSet}
			\ObjectiveFunction(\AdmissibleSolution)$ se il
			problema \`e definito di massimo.
		\end{itemize}
	\end{itemize}
	Denoteremo i dati di un problema di minimo
	$\MinimumProblem$ con la notazione
	$$(\MinimumProblem)\ \min \lbrace \ObjectiveFunction(\AdmissibleSolution): \AdmissibleSolution \in \AdmissibleSet \rbrace$$
	e i dati di un problema di massimo $\MaximumProblem$ con
	$$(\MaximumProblem)\ \max \lbrace \ObjectiveFunction(\AdmissibleSolution): \AdmissibleSolution \in \AdmissibleSet \rbrace.$$
	\par In alternativa, quando l'insieme ammissibile \`e descritto tramite vincoli su variabili, denoteremo il problema scrivendo su una riga $\min$ o $\max$ e la funzione da minimizzare o massimizzare, e nelle righe successive i vincoli posti (forniremo nel seguito molti esempi).
	\par Risolvere un problema di ottimizzazione significa individuarne una soluzione ottimale.
\end{Definition}
\begin{Definition}
	Supponiamo l'insieme ammissibile di un problema (decisionale, di certificato o di ottimizzazione indifferentemente) sia sottoinsieme di uno spazio vettoriale di dimensione $n \in \mathbb{N}$: $n$ si chiama \Define{dimensione del problema}[di un problema][dimensione].
\end{Definition}
\begin{Definition}
	Un problema di ottimizzazione $\OptimizationProblem$ si dice \Define{vuoto}[di ottimizzazione][problema] quando $\AdmissibleSet = \emptyset$.
\end{Definition}
\begin{Example}
	Non tutti i problemi con valore ottimo finito ammettono soluzioni ottime: \`e il caso dei problemi $\min \lbrace x: x > 0 \rbrace$ e $\min \lbrace \frac{1}{x}: x \geq 0 \rbrace$.
\end{Example}
\begin{Example}
	\TheoremName{Problema di equipartizione} Il problema di
	equipartizione \`e determinato da un primo parametro
	$n \in \mathbb{N}$, che determina il numero dei
	restanti parametri, e appunto dai restanti parametri
	$(a_i)_{i \in n} \in \mathbb{N}^n$. Il problema pu\`o
	essere formulato come segue:
	$$(\EquipartitionProblem)\ \min \lbrace \ObjectiveFunction(\Set): \Set \in \PowerSet{n} \rbrace$$
	dove $\ObjectiveFunction(\Set) = \AbsoluteValue{\sum_{i \in S} a_i - \sum_{n \SetMin S} a_i}$.
\end{Example}
\begin{Example}
	\TheoremName{Problema di impaccamento di cerchi (Circle Packing)}
	Il problema di impaccamento di cerchi prevede un unico parametro
	$n \in \mathbb{N}$ e risponde alla seguente domanda: qual \`e il
	massimo raggio che possono avere $n$ cerchi identici posti
	all'interno di un quadrato di lato unitario che non si
	sovrappongono?
	\par Consideriamo l'insieme $\AdmissibleSet'$ di tutte le coppie
	$((x_i,y_i)_{i \in n}, r)$, con, per ogni $i \in n$, $x_i, y_i \in
	[0,1]$ e $r \in \mathbb{R}$: interpretiamo $(x_i, y_i)$ come le
	coordinate del centro del cerchio di indice $i$ e $r$ come il
	raggio comune di tutti gli $n$ cerchi. Restringiamo
	$\AdmissibleSet'$ ad un insieme ammissible $\AdmissibleSet$
	aggiungendo le condizioni seguenti:
	\begin{itemize}
		\item affinch\'e il cerchio di centro $(x_i,y_i)$ sia
		interamente contenuto nel quadrato $[0,1]^2$ richiediamo
		$r \leq x_i \leq 1 - r$ e $r \leq y_i \leq 1 - r$;
		\item affinch\'e i cerchi di centro $(x_i,y_i)$ e
		$(x_j,y_j)$, con $i \neq j$, non si sovrappongano
		richiediamo
		$\Distance[2]((x_i,y_i),(x_j,y_j)) \leq 2r$.
	\end{itemize}
	Il problema pu\`o allora essere descritto come segue:
	$$(\CirclePacking)\ \max \lbrace r: ((x_i,y_i)_{i \in n}, r) \in [0,1]^n \times \mathbb{R} \rbrace.$$
\end{Example}
\begin{Definition}
	Un problema di ottimizzazione si dice
	\begin{itemize}
		\item \Define{combinatorio}[di ottimizzazione combinatorio][problema] quando il suo insieme ammissible \`e finito o numerabile;
		\item \Define{continuo}[di ottimizzazione continuo][problema] quando il suo insieme ammissible \`e pi\`u che numerabile.
	\end{itemize}
\end{Definition}
\begin{Theorem}
	Problemi di minimo e problemi di massimo sono equivalenti, nel senso che
	\begin{itemize}
		\item per ogni problema di minimo esiste un problema di massimo le cui soluzioni ottimali coincidono con le soluzioni ottimali del problema di minimo;
		\item per ogni problema di massimo esiste un problema di minimo le cui soluzioni ottimali coincidono con le soluzioni ottimali del problema di massimo.
	\end{itemize}
\end{Theorem}
\Proof Per trasformare un problema di minimo o di massimo in un problema dell'altro tipo \`e sufficiente moltiplicare la funzione obiettivo per $-1$. \EndProof
\par Nel seguito, per il teorema appena enunciato, enunceremo alcune definizioni o dimostreremo alcuni teoremi validi sia per problemi di minimo che di massimo solo per problemi di minimo: il lettore ricaver\`a facilmente i risultati equivalenti per i problemi di massimo.
\begin{Theorem}
	Problemi decisionali e problemi di ottimizzazione sono equivalenti, nel senso che
	\begin{itemize}
		\item per ogni problema decisionale esiste un problema di ottimizzazione le cui soluzioni ottimali coincidono con le soluzioni ammissibili del problema decisionale;
		\item per ogni problema di ottimizzazione esiste un problema decisionale le cui soluzioni ammissibili coincidono con le soluzioni ottimali del prolbema di ottimizzazione.
	\end{itemize}
\end{Theorem}
\Proof Sia $\DecisionalProblem$ un problema decisionale di insieme ammissibile $\AdmissibleSet$ definito come sottoinsieme di $\AdmissibleSet'$. Il problema di massimo avente $\AdmissibleSet'$ come insieme ammissibile e funzione obiettivo coincidente con la funzione caratteristica di $\AdmissibleSet$ ha la propriet\`a richiesta.
\par Sia $\OptimizationProblem$ un problema di ottimizzazione di insieme ammissible $\AdmissibleSet$. Il problema decisionale il cui insieme ammissibile \`e il sottoinsieme di $\AdmissibleSet$ costituito da tutte le soluzioni ottimali ha la propriet\`a richiesta. \EndProof
\par Nel seguito, per il teorema appena enunciato, enunceremo alcune definizioni o dimostreremo alcuni teoremi validi sia per i problemi di ottimizzazione che per i problemi decisionali solo per problemi di ottimizzazione: il lettore ricaver\`a facilmente i risultati equivalenti per i problemi decisionali.
\begin{Definition}
	Sia $\MinimumProblem$ un problema di minimo con insieme ammissibile $\AdmissibleSet$ e funzione obiettivo $\ObjectiveFunction$. Fissato $k \in \mathbb{R}$, definiamo
	\Define{problema decisionale associato}[decisionale associato][problema] o
	\Define{versione decisionale}[di un problema di ottimizzazione][versione decisionale]
	il problema decisionale consistente nell'individuare l'esistenza di elementi nell'insieme
	$\AdmissibleSet_k = \lbrace \AdmissibleSolution \in \AdmissibleSet | \ObjectiveFunction(\AdmissibleSolution) \leq k$.
\end{Definition}
\begin{Definition}
	Sia $\MinimumProblem$ un problema di minimo. Un
	\Define{algoritmo di risoluzione}[di risoluzione di un problema][algoritmo] \`e
	un algoritmo che associa ad un'istanza di $\MinimumProblem$ il suo valore ottimo (da cui \`e poi generalmente possibile individuare una soluzione ottima). Un algoritmo capace di individuare il valore ottimo di qualsiasi istanza di un determinato problema si dice
	\Define{esatto}[algoritmo esatto][algoritmo].
\end{Definition}
\begin{Definition}
	Siano dati i seguenti problemi di minimo:
	$$(\MinimumProblem_1) \min \lbrace \ObjectiveFunction_1(\AdmissibleSolution): \AdmissibleSolution \in \AdmissibleSet_1 \rbrace,$$
	$$(\MinimumProblem_2) \min \lbrace \ObjectiveFunction_2(\AdmissibleSolution): \AdmissibleSolution \in \AdmissibleSet_2 \rbrace.$$
	Se abbiamo
	\begin{itemize}
		\item $\AdmissibleSet_1 \subseteq \AdmissibleSet_2$;
		\item $\ForAll{\AdmissibleSolution \in \AdmissibleSet_1}{\ObjectiveFunction_2(\AdmissibleSolution) \leq \ObjectiveFunction_1(\AdmissibleSolution)}$;
	\end{itemize}
	Allora diciamo che $\MinimumProblem_2$ \`e un
	\Define{rilassamento}[di un problema di ottimizzazione][rilassamento]
	di $\MinimumProblem_1$.
\end{Definition}
\begin{Definition}
	Dato un problema di ottimizzazione, definiamo
	\begin{itemize}
		\item \Define{variabile quantitativa}[quantitativa][variabile] una variabile che esprime una quantit\`a del problema;
		\item \Define{variabile decisionale}[decisionale][variabile] una variabile che esprime una decisione da prendere nella soluzione del problema;
		\item \Define{variabile booleana}[booleana][variabile] o \Define{variabile binaria}[binaria][variabile] o ancora \Define{variabile logica}[logica][variabile] una variabile decisionale che pu\`o valere $0$ -- equivalente al significato di falso -- o $1$ -- equivalente a vero --;
		\item \Define{variabile strutturale}[strutturale][variabile] una variabile decisionale essenziale del problema, togliendo la quale non \`e pi\`u possibile descriverne una soluzione (a meno di introdurne altre equivalenti);
		\item \Define{variabile ausiliaria}[ausiliaria][variabile] una variabile decisionale non indispensabile alla descrizione di una soluzione del problema, ma ci\`o non di meno utile per esempio ad esprimere vincoli.
	\end{itemize}
\end{Definition}
