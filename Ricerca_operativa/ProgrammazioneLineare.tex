\section{Programmazione lineare.}\label{ProgrammazioneLineare}
\begin{Definition}
	Si chiama
	\Define{problema di programmazione lineare}[di programmazione lineare][problema]
	un problema di ottimizzazione tale che
	\begin{itemize}
		\item esistono $\ConstraintsNumber \in \mathbb{N}$ condizioni del tipo $a \cdot \AdmissibleSolution \left \lbrace \begin{array}{c} \leq\\=\\\geq \end{array} \right \rbrace b$, con $a \in \mathbb{R}^\ProblemDimension$ e $b \in \mathbb{R}$ e dove $\ProblemDimension$ \`e la dimensione del problema, tale che $\AdmissibleSolution \in \AdmissibleSet$ se e solo se $\AdmissibleSolution$ verifica tutte le condizioni: tali condizioni si chiamano \Define{vincoli lineari}[lineare][vincolo];
		\item la sua funzione obiettivo $\ObjectiveFunction$ sia della forma
		$\ObjectiveFunction(\AdmissibleSolution) = \CostsVector \cdot \AdmissibleSolution$, dove $\CostsVector \in \mathbb{R}^\ProblemDimension$ \`e un vettore caratteristico della particolare istanza del problema in esame detto \Define{vettore dei costi}[dei costi][vettore]: $\ObjectiveFunction$ \`e dunque una funzione lineare, pi\`u precisamente un funzionale.
	\end{itemize}
	Il problema si dice di \Define{programmazione lineare intera}[di programmazione lineare intera][problema] quando $\AdmissibleSet \subseteq \mathbb{Z}^\ProblemDimension$; si dice di \Define{programmazione lineare mista}[di programmazione lineare mista][problema] quando si richiede che solo alcune delle componenti di una soluzione ammissibile siano intere.
\end{Definition}
\begin{Definition}
	Dato un problema di ottimizzazione e una sua variabile $x$, il vincolo $x \in \mathbb{Z}$ si chiama \Define{vincolo di integrit\`a}[di integrit\`a][vincolo].
\end{Definition}
\begin{Theorem}
	Ogni problema di programmazione lineare di dimensione $\ProblemDimension$ e con $\ConstraintsNumber$ vincoli lineari pu\`o essere messo nella forma $\max\ \lbrace \CostsVector \cdot x : A\AdmissibleSolution \leq b \rbrace$, dove $\CostsVector,\AdmissibleSolution \in \mathbb{R}^\ProblemDimension$, $A \in \mathbb{R}^{\ConstraintsNumber \times \ProblemDimension}$, $b \in \mathbb{R}^\ConstraintsNumber$ e su $\ConstraintsNumber$ consideriamo l'ordinamento parziale definito da $\Coimplies{(x_i)_{i \in \ConstraintsNumber} \leq (y_i)_{i \in \ConstraintsNumber}}{\ForAll{i \in \ConstraintsNumber}{x_i \leq y_i}}$.
\end{Theorem}
\Proof Abbiamo gi\`a provato l'equivalenza tra problemi di massimo e di minimo tramite la moltiplifcazione per $-1$ della funzione obiettivo. Basta dunque provare che tutti i vincoli lineari possono essere riscritti nella forma $a \cdot \AdmissibleSolution \leq b$ per opportuni $a \in \mathbb{R}^\ProblemDimension$ e $b \in \mathbb{R}$.
\par Supponiamo dato il vincolo lineare $a \cdot \AdmissibleSolution \geq b$ ($a \in \mathbb{R}^\ProblemDimension$, $b \in \mathbb{R}$): esso \`e equivalente al vincolo $-a \cdot \AdmissibleSolution \leq -b$. Supponiamo dato il vincolo $a \cdot \AdmissibleSolution = b$ ($a \in \mathbb{R}^\ProblemDimension$, $b \in \mathbb{R}$): esso \`e equivalente alla congiunzione dei vincoli $a \cdot \AdmissibleSolution \leq b$ e $a \cdot \AdmissibleSolution \geq b$. \EndProof
\begin{Definition}
	Dati $a \in \mathbb{R}^\ProblemDimension$ e $b \in \mathbb{R}$, sia un vincolo lineare del tipo $a \cdot \AdmissibleSolution \leq b$ o $a \cdot \AdmissibleSolution \geq b$. La variabile $b - a \cdot \AdmissibleSolution$ nel primo caso e $a \cdot \AdmissibleSolution - b$ nel secondo caso si chiamano \Define{variabili di scarto}[di scarto][variabile].
\end{Definition}
\subsection{Esempi di programmazione lineare.}\label{EsempiDiProgrammazioneLineare}
\begin{Example}
	\TheoremName{Il problema dello zaino (\textit{Knapsack Problem})} Siano
	\begin{itemize}
		\item $I$ un insieme finito, detti \NIDefine{oggetti};
		\item $(a_i)_{i \in I} \in \mathbb{N}^I$ una famiglia di elementi, detti \NIDefine{pesi};
		\item $(c_i)_{i \in I} \in \mathbb{N}^I$ una famiglia di elementi, detti \NIDefine{costi};
		\item $b \in \mathbb{N}$ una costante, detta \NIDefine{massimo peso dello zaino}.
	\end{itemize}
	Lo scopo del problema \`e individuare un sottoinsieme $Z \subseteq I$ di oggetti (lo \NIDefine{zaino}) il cui peso non supera il peso $b$ e di massimo costo possibile.
	\par Il problema pu\`o dunque essere formulato come segue:
	$$(\KnapsackProblem)\ \max \lbrace \sum_{i \in Z} c_i: Z \in \AdmissibleSet \rbrace,$$
	dove $\AdmissibleSet = \lbrace X \subseteq I: \sum_{i \in X} a_i \leq b \rbrace$.
	\par Questo problema pu\`o essere riformulato come problema di programmazione lineare intero definendo, per ogni soluzione ammissible $Z$, la famiglia di variabili booleane $(x_i)_{i \in Z}$, coincidente con la funzione caratteristica di $Z$: allora la funzione obiettivo pu\`o essere scritta come $\CostsVector \cdot x$, dove $\CostsVector = (c_i)_{i \in I}$ e $x = (x_i)_{i \in Z}$, e l'insieme ammissibile pu\`o essere descritto come quel sottoinsieme di $\mathbb{Z}^{\Cardinality{I}}$ tale che $a \cdot x \leq b$, dove $a = (a_i)_{i \in I}$.
\end{Example}
\begin{Example}
	\TheoremName{Problema dell'albero di copertura di costo minimo (\textit{Minimal Spanning Tree})} Sia $\Graph = (\Nodes, \Edges)$ un grafo di nodi $\Nodes$ e archi $\Edges \subseteq \Nodes^2$; sia $(c_{i,j})_{(i,j) \in \Edges}$ una famiglia di numeri reali positivi che chiameremo \NIDefine{costo} dell'arco $(i,j)$. Vogliamo determinare un grafo parziale di $\Graph$ connesso e di costo minimo, definendo il costo del grafo come la somma dei costi dei suoi archi: si dimostra facilmente che un grafo con queste caratteristiche \`e un albero di copertura di $\Graph$, i.e. un grafo connesso e privo di cicli.
	\par Possiamo dunque definire il problema con
	$$(\MinimalSpanningTree)\ \min \lbrace \sum_{(i,j) \in \Edges'} c_{i,j} : (\Nodes,\Edges')\text{ \`e un albero di copertura di }\Graph \rbrace.$$
	\par Determinare un sottoinsieme $\Edges' \subseteq \Edges$ equivale a definire una famiglia $(x_{(i,j)})_{(i,j) \in \Edges}$ coincidente con la funzione caratteristica di $\Edges'$. A questo punto l'insieme ammissibile del problema pu`\o essere ridefinito come l'insieme di tutte le possibile famiglie $(x_{(i,j)})_{(i,j) \in \Edges}$ e la condizione di connessione come l'insieme (finito) di condizioni $\sum_{i \in S\\j \in \Nodes \SetMin S} x_{(i,j)} \geq 1$ per ogni sottoinsieme proprio $S$ non vuoto di $\Nodes$. Il problema diventa cos\`i un problema di programmazione lineare intera.
\end{Example}
\begin{Example}
	\TheoremName{Il problema del commesso viaggiatore (\textit{Travelling Salesman Problem})}
	Dato un grafo completo $\Graph = (\Nodes, \Nodes^2)$ e una famiglia di numeri reali positivi $(c_{i,j})_{(i,j) \in \Nodes^2}$, detti \NIDefine{costi}, si ricerca un ciclo hamiltoniano di costo minimo. Ci\`o pu\`o essere modellizzato introducendo le stesse variabili $(x_{(i,j)})_{(i,j) \in \Nodes^2}$ utilizzate nell'esempio del problema dell'albero di copertura di costo minimo, imponendo la stessa restrizione $\sum_{i \in S\\j \in \Nodes \SetMin S} x_{(i,j)} \geq 1$ per garantire la connessione del ciclo hamiltoniano e aggiungendo, per ogni nodo $i \in \Nodes$, la condizione $\sum_{j \in \Nodes} x_{i,j} = 2$ per garantire che ogni nodo sia estremo di due e solo due archi; per funzione obiettivo si pu`\o usare la stessa usata per il problema dell'albero di copertura di costo minimo.
	\par Osserviamo che con le formulazioni fornite il problema dell'albero di copertura di costo minimo costituisce un rilassamento del problema del commesso viaggiatore.
\end{Example}
\par Alcuni problemi, per essere espressi in termini di programmazione lineare, richiedono di combinare valori booleani diversi secondo le classiche regole della logica booleana. A tal fine \`e utile tenere a mente le relazioni seguenti:
\begin{itemize}
	\item se $a$ \`e una variabile booleana, $\Not{a}$ \`e rappresentata da $1 - a$;
	\item se $a$ e $b$ sono variabili booleane, $a \rightarrow b$ equivale alla relazione $a \leq b$;
	\item se $a$ e $b$ sono variaibli booleane, $c = \Or{a}{b}$ equivale alle relazioni
	\begin{itemize}
		\item $a \leq c$;
		\item $b \leq c$;
		\item $c \leq a + b$ (si rammenta che $c \in \lbrace 0, 1 \rbrace$, per cui non \`e possibile $c = 2$ nel caso $a = b = 1$);
	\end{itemize}
	\item se $a$ e $b$ sono variaibli booleane, $c = \Xor{a}{b}$ equivale alle relazioni
	\begin{itemize}
		\item $a - b \leq c$;
		\item $b - a \leq c$;
		\item $c \leq a + b$;
		\item $c \leq 2 - a - b$;
	\end{itemize}
	\item se $a$ e $b$ sono variaibli booleane, $c = \And{a}{b}$ equivale alle relazioni
	\begin{itemize}
		\item $c \leq a$;
		\item $c \leq b$;
		\item $c \geq a + b - 1$.
	\end{itemize}
\end{itemize}
\begin{Example}
	\TheoremName{Il problema della soddisfattibilit\`a proposizione (\textit{Satisfatibility problem})} Sia $(\Litteral_j)_{j \in n}$ ($n \in \mathbb{N}$) una famiglie di letterali. Sia $(C_i)_{i \in m}$ ($m \in \mathbb{N}$) una famiglia di proposizioni che sono la disgiunzione di letterali di $(\Litteral_j)_{j \in n}$ o di loro negazioni. Sia infine $\Formula$ la congiunzione dei $(C_i)_{i \in m}$. Il problema richiede di determinare se esistono valori dei letterali $\Litteral_j$ tali da rendere vera $\Formula$: la sua importanza risiede nella possibilit\`a di mettere qualsiasi formula del calcolo proposizionale in forma normale congiuntiva.
	\par Introduciamo la matrice $\Matrix = (\Matrix_{i,j})_{(i,j) \in m \times n}$: $\Matrix_{i,j} = \begin{cases} 1 \text{ se }\Litteral_j\text{ \`e presente in } C_i; -1 \text{ se }\Litteral_j\text{ \`e presente negato in } C_i;0 \text{altrimenti}.\end{cases}$
	\par Ora, se $X \in \mathbb{R}^n$ \`e un vettore booleano le cui componenti sono i valori di verit\`a dei letterali $\Litteral_j$, $C_i$ equivale a $\Transposed{\Matrix_i} X \geq 1 - n(i)$, dove $n(i)$ \`e il numero di letterali negati in $C_i$.
	\par Il problema di programmazione lineare pu`\o dunque essere definito usando come insieme ammissibile l'insieme di tutti gli $X$ che soddisfano la disugualgianza $\Transposed{\Matrix_i} X \geq 1 - n(i)$, per ogni $i \in n$. Per funzione obiettivo si pu\`o scegliere una qualsiasi funzione costante (in effetti qui siamo di fronte ad un problema di decisione pi\`u che di ottimizzazione).
\end{Example}
\begin{Definition}
	Siano $A$ e $B$ due insiemi finiti e sia $(x_{i,j})_{(i,j) \in A \times B}$ una famiglia di variabili booleane. Chiamiamo
	\begin{itemize}
	\item $\Define{vincoli di semiassegnamento}[di semiassegnamento][vincoli]$ le equazioni $\sum_{j \in B} x_{i,j} = 1$ al variare di $i \in A$;
	\item $\Define{vincoli di assegnamento}[di assegnamento][vincoli]$ le equazioni $\sum_{j \in B} x_{i,j} = 1$ al variare di $i \in A$ e $\sum_{i \in A} x^{i,j} = 1$ al variare di $j \in B$: $(x_{i,j})_{(i,j) \in A \times B}$ risulta cos\`i essere una matrice di permutazione.
	\end{itemize}
\end{Definition}
\begin{Example}
	\TheoremName{Il problema di minimizzazione del numero delle macchine (\textit{Minimal Cardinality Machine Scheduling})} Siano
	\begin{itemize}
		\item $N$ un insieme finito di lavori;
		\item $(d_i)_{i \in N}$ la durata del lavoro $i \in N$;
		\item $(t_i)_{i \in N}$ il momento in cui deve iniziare il lavoro $i \in N$.
	\end{itemize}
	Sapendo che ogni lavoro deve essere svolto da una macchina e che ogni macchina pu\`o svolgere al massimo un lavoro per volta, si chiede quale sia il numero minimo di macchine necessario per compiere i lavori previsti nei termini previsti.
	\par Osserviamo per prima cosa che se mettiamo a disposizione tante macchine quanti sono i lavori possiamo senz'altro eseguire tutti i lavori: introduciamo dunque un insieme di macchine $M$ equicardinale ad $N$ e consideriamo il problema equivalente di determinare quale sia il numero minimo di macchine di $M$ necessario per compiere tutti i lavori. Introduciamo le variabili booleane $(x_{i,j})_{(i,j) \in N \times M}$: $x_{i,j} = \begin{cases} 1\text{ se il lavoro }i\text{ \`e assegnato alla macchina }j;\\0\text{ altrimenti.}\end{cases}$
	\par Il fatto che un lavoro venga eseguito una sola volta, e dunque da una sola macchina, si esprime coi vincoli di semiassegnamento $\sum_{j \in M}x_{i,j} = 1$ per ogni $i \in N$.
	\par Per ogni lavoro $i \in N$ definiamo l'insieme $S_i$ dei lavori che non possono essere eseguiti dalla stessa macchina che esegue $i$: $S_i = \lbrace h \in N: [t_i;t_i + d_i] \cap [t_h;t_h + d_h] \neq \emptyset \rbrace$. Poniamo allora ulteriori vincoli per garantire che ogni macchina esegua solo un lavoro per volta: per ogni $(i,j) \in N \times M$ e per ogni $h \in S_i$ poniamo $x_{i,j} + x_{h,j} \leq 1$. Osserviamo che per ogni coppia di lavori $(i,h) \in N^2$ abbiamo introdotto lo stesso vincolo due volte (la cosa sarebbe facilmente evitabile ordinando $N$ e introducendo il vincolo solo quando $i < h$).
	\par L'insieme ammissible $\AdmissibleSet$ \`e dunque il sottoinsieme di tutte le applicazioni $x_{i,j}: N \times M \rightarrow \lbrace 0, 1\rbrace$ che rispetta tutti i vincoli dati.
	\par Introduciamo ora ulteriori variabili booleane $(y_j)_{j \in M}$, vere se la macchina $j \in M$ risulta usata. Possiamo dunque esprimere la funzione obiettivo come $\ObjectiveFunction = \sum_{j \in M} y_j$.
	\par Per esprimere che la macchina $j \in M$ \`e usata, imponiamo $ny_j \geq \sum_{i \in N} x_{i,j}$: non si tratta di ulteriori vincoli per definire l'insieme ammissibile, ma solo di una disequazione che determina i valori $(y_j)_{j \in M}$ in maniera univoca. Le variabili $(y_j)_{j \in M}$ sono variabili ausiliari, in opposizione alle variabili $(x_{i,j})_{(i,j) \in N \times M}$ invece strutturali.
	Il problema si definisce dunque come
	$$(\MinimalCardinalityMachineScheduling)\ \min \lbrace \sum_{j \in M} y_j: (x_{i,j})_{(i,j) \in N \times M} \in \AdmissibleSet \rbrace.$$
\end{Example}
\begin{Example}
	\TheoremName{Il problema della colorazione di grafo (\textit{Graph Coloring})} Sia $\Graph = (\Nodes,\Edges)$ un grafo con $n = \Cardinality{\Nodes} \in \NotZero{\mathbb{N}}$ nodi: vogliamo associare ad ogni nodo un elemento di un insieme $C$, per esempio un insieme di colori (da qui il nome del problema) in modo tale che a due nodi adiacenti non sia associato lo stesso elemento e usando il minor numero di elementi di $C$ possibile.
	\par Come per il problema $\MinimalCardinalityMachineScheduling$, possiamo assumere $\Cardinality{C} = \Cardinality{\Nodes}$. Inoltre, introduciamo le variabili booleane $(x_{i,c})_{(i,c) \in \Nodes \times C}$ tali che $x_{i,j} = \begin{cases} 1\text{ se }j\text{ \`e associato a }i,\\0\text{ altrimenti.}\end{cases}$.
	\par Ad ogni nodo deve essere associato un solo elemento di $C$, per cui imponiamo i vincoli di semiassegnamento $\ForAll{i \in \Nodes}{\sum_{c \in C} x_{i,c} = 1}$.
	\par Inoltre a due nodi adiacenti non deve essere associato lo stesso elemento di $C$: imponiamo dunque i vincoli $\ForAll{(i,j) \in \Nodes^2}{\ForAll{c \in C}{x_{i,c} + x_{j,c} \leq 1}}$.
	\par Scegliamo dunque come insieme ammissibile $\AdmissibleSet$ l'insieme $\lbrace 0, 1\rbrace^{\Nodes \times C}$ con i vincoli sopraindicati.
	\par Di nuovo in analogia col problema $\MinimalCardinalityMachineScheduling$, introduciamo le variabili booleane $(y_c)_{c \in C}$: vogliamo che esse risultino vere quando $c \in C$ \`e associato a qualche $i \in \Nodes$; otteniamo ci\`o imponendo $\ForAll{c \in C}{\ForAll{i \in \Nodes}{y_c \geq x_{i,c}}}$.
	\par Il problema si definisce dunque come
	$$(\GraphColoring)\ \min \lbrace \sum_{j \in M} y_j: (x_{i,j})_{(i,j) \in \Nodes \times C} \in \AdmissibleSet \rbrace.$$
\end{Example}
\begin{Definition}
	Siano $\Omega$ un insieme finito e $P \subseteq \PowerSet{\Omega}$. Siano le variabili booleane $(x_p)_{p \in P}$. Chiamiamo
	\begin{itemize}
		\item \Define{vincoli di copertura}[di copertura][vincoli]: $\ForAll{\omega \in \Omega}{\sum_{p \in P\\\omega \in p} x_p \geq 1}$;
		\item \Define{vincoli di partizione}[di partizione][vincoli]: $\ForAll{\omega \in \Omega}{\sum_{p \in P\\\omega \in p} x_p = 1}$;
		\item \Define{vincoli di riempimento}[di riempimento][vincoli]: $\ForAll{\omega \in \Omega}{\sum_{p \in P\\\omega \in p} x_p \leq 1}$.
	\end{itemize}
\end{Definition}
\begin{Example}
	\TheoremName{Problemi di copertura, partizione e riempimento} Siano $\Omega$ un insieme finito e $P \subseteq \PowerSet{\Omega}$. Siano inoltre definiti i numeri $(c_p)_{p \in \Omega}$, detti \emph{costi}. Vogliamo trovare, se esiste, $S \subseteq P$ tale che $\bigcup{s \in S}s = \Omega$ e tale che il suo costo, definito come $\sum_{s \in S} c_s$ sia minimo: \`e il problema di copertura.
	\par Introduciamo le variabili booleane $(x_p)_{p \in P}$, da considerarsi vere se e solo se selezioniamo $p \in P$ per entrare in $S$. Possiamo allora definire il problema come
$$(\CoveringProblem)\ \min \lbrace \sum_{p \in P} c_p x_p: (x_p)_{p \in P} \in \AdmissibleSet \rbrace,$$
	dove $\AdmissibleSet$ \`e l'insieme di tutte le famiglie $(x_p)_{p \in P}$ che rispettano i vincoli di copertura.
	\par Aggiungendo alla definizione del problema di copertura l'ulteriore condizione che elementi distinti di $S$ siano disgiunti otteniamo il problema di partizione $\PartitioningProblem$, che risolviamo allo stesso modo sostituendo ai vincoli di copertura quelli di partizionamento.
	\par Infine, se trasformiamo il problema di partizione togliendo la condizione $\bigcup{s \in S}s = \Omega$, otteniamo il problema di riempimento $\FillingProblem$, che risolviamo sostituendo ai vincoli di partizionamento quelli di di riempimento.
\end{Example}
\begin{Definition}
	Data una famiglia di variabili quantitative $(x_i)_{i \in I}$ e un'ulteriore variabile $V$, chiamiamo \Define{vincoli soglia}[di soglia][vincoli] disuguaglianze del tipo $V \leq x_i$ o $x_i \leq V$, con $i \in I$
\end{Definition}
\begin{Example}
	\TheoremName{Il problema del minor tempo di produzione su macchine (\textit{Minimal Makespan Machine Scheduling})} Siano dati un insieme di lavori $N$ e un insieme di macchine $M$: ogni macchina pu\`o eseguire un solo lavoro per volta. Ad ogni lavoro \`e associata una durata $(d_i)_{i \in N}$. A differenza di quanto supporto per il problema $\MinimalCardinalityMachineScheduling$, in questo caso supponiamo che tutti i lavori possano essere svolti in qualsiasi momento. Si chiede di distribuire i lavori in maniera tale da vederli tutti completati il prima possibile.
	\par Introduciamo le variabili booleane $(x_{i,j})_{(i,j) \in N \times M}$: $x_{i,j}$ \`e vera se e solo se il lavoro $i$ \`e associato alla macchina $j$. Come insieme ammissibile $\AdmissibleSet$ scegliamo l'insieme di tutte le famiglia $(x_{i,j})_{(i,j) \in N \times M}$ che rispettano i vincoli di semiassegnamento $\ForAll{i \in N}{\sum_{j \in M} x_{i,j} = 1}$ (ogni lavoro deve essere svolto da una sola macchina).
	\par Introduciamo la variabile ausiliaria $t$ e i vincoli di soglia $\ForAll{j \in M}{\sum_{i \in N} d_ix_{i,j} \leq t}$: imponiamo cio\`e che $t$ sia grande almeno quanto la durata di svolgimento di tutti i lavori. Minimizzando $t$, minimizziamo dunque anche la durata dei lavori:
$$(\MinimalMakespanMachineScheduling)
\begin{array}{ll}
	\min &t,\\
	&\ForAll{i \in N}{\sum_{j \in M} x_{i,j} = 1},\\
	&\ForAll{j \in M}{\sum_{i \in N} d_ix_{i,j} \leq t},\\
	&\ForAll{(i,j) \in N \times M}{x_{i,j} \in \lbrace 0, 1 \rbrace}.
\end{array}$$
\end{Example}
\begin{Example}
	In alcuni problemi pu\`o capitare di trovarsi di fronte ad una funzione lineare a tratti
	$$f(x) = \begin{cases}
	b_0 + c_0x\text{ se }x \in [a_0,a_1],\\
	b_1 + c_1x\text{ se }x \in (a_1,a_2],\\
	...\\
	b_{n - 1} + c_{n - 1}x\text{ se }x \in [a_{n - 1},a_n],
	\end{cases}$$
	con $\ForAll{i \in n}{a_i \leq a_{i + 1}}$.
	\`E possibile trasformare $f(x)$ in una funzione lineare. Introduciamo le variabili booleane $(y_i)_{i \in n}$ e le variabili quantitative $(z_i)_{i \in n}$ e imponiamo i seguenti vincoli:
	\begin{itemize}
		\item $\Implies{x = a_0}{y_0 = 1}$ e $\ForAll{i \in n}{\Implies{x \in (a_i,a_{i + 1}]}{y_i = 1}}$: questi vincoli garantiscono che $y_i$ sia vero se e solo se $x$ appartiene all'$i$-esimo intervallo di definizione;
		\item $\ForAll{i \in n}{0 \leq z_i \leq (a_{i + 1} - a_i)y_i}$: gli $z_i$ sono dunuqe grandezze reali positive grandi al pi\`u quando la lunghezza dell'$i$-esimo intervallo e certamente nulle se $x$ non appartiene all'$i$-esimo intervallo.
	\end{itemize}
	\par Abbiamo dunque $x = \sum_{i \in n}{a_iy_i + z_i} = \sum_{i \in n}{a_iy_i} + \sum_{i \in n} z_i$: $x$ \`e funzione lineare degli $(y_i)_{i \in n}$ e dei $(z_i)_{i \in n}$.
	\par Inoltre $f(x) = g(z_0,...,z_n,y_0,...,y_n) = \sum_{i \in n}b_iy_i + c_i(a_iy_i + z_i) = \sum_{i \in n}(b_i + c_ia_i)y_i + \sum_{i \in n}{c_iz_i}$: come si vede, il secondo membro \`e una funzione lineare dei $(y_i)_{i \in n}$ e dei $(z_i)_{i \in n}$.
	\par Nel calcolare massimi o minimi di $f(x)$ occorre essere particolarmente prudenti nei punti di discontinuit\`a: ciascuno di essi pu\`o essere rappresentato attravero due diversi assegnazioni degli argomenti di $g$, ma solo una di queste assegnazioni corrisponde ad un effettivo valore di $f(x)$.
\end{Example}
\begin{Example}
	Riprendiamo la funzione $f$ dell'esempio precedente e supponiamo stavolta che essa sia continua e che la successione dei $(c_i)_{i \in n}$ sia non decresente.
	\par Allora \`e possibile sostituire $f$ con una funzione lineare $g$ introducendo stavolta solo una famiglia di variabili $(z_i)_{i \in n}$. In tal caso, i vincoli sono dati da $\ForAll{i \in n}{0 \leq z_i \leq a_{i + 1} - a_i}$ e poniamo $g(z_0,...,z_{n - 1}) = b_0 + \sum_{i \in n}c_iz_i$: se $g$ ha minimo in $(z_i)_{i \in n}$, allora abbiamo
	\begin{itemize}
		\item per opportuno $h \in n$, $z_i = \begin{cases}a_{i + 1} - a_i\text{ se }i < h,\\0\text{ se }i > h;\end{cases}$
		\item $f$ ha minimo in $x = a_0 + \sum_{i \in n} z_i$.
	\end{itemize}
	\par La prima affermazione discende dal fatto che, per la crescenza della successione $c_i$, se sostituiamo in $g$ una successione $(z_i)_{i \in n}$ arbitraria con $(z_i')_{i \in n}$ nella forma descritta della prima affermazione e tale che $\sum_{i \in n} z_i = \sum_{i \in n} = z_i'$, allora il valore di $g$ cala.
	\par La seconda affermazione
\end{Example}
\begin{Definition}
	Fissato $n \in \mathbb{N}$, siano dati numeri reali $(A_i)_{i \in n}$ e $(b_i)_{i \in n}$ e supponiamo che un problema sia sottoposto ai vincoli $\ForAll{i \in n}{A_ix \leq b}$
\end{Definition}

