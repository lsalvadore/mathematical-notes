\section{Definizioni e teoremi di base.}
\label{Integrali_DefinizioniETeoremiDiBase}
\begin{Theorem}
\label{Misura_th_ApplicazioneCaratteristica}
  Siano
  \begin{itemize}
    \item $(\MeasureSpace,\SigmaAlgebra,\Measure)$ uno spazio di misura;
    \item $\MeasurablePart \in \SigmaAlgebra$;
    \item $\CharacteristicApplication{\MeasurablePart}:
      \MeasurableSpace \rightarrow \mathbb{R}$
      l'applicazione caratteristica
      di $\MeasurablePart$;
    \item $c \in \mathbb{R}$.
  \end{itemize}
  L'applicazione $c \CharacteristicApplication{\MeasurablePart}$ \`e misurabile.
\end{Theorem}
\Proof Se $c = 0$, allora $c \CharacteristicApplication{\MeasurablePart}$ \`e
costante e dunque misurabile.
\par Assumiamo $c \neq 0$ e sia $\VarMeasurablePart \subseteq \mathbb{R}$
misurabile.
\par Ora,
\begin{itemize}
  \item se $0, c \notin \VarMeasurablePart$, allora
    $\CharacteristicApplication{\MeasurablePart}^{-1} (\VarMeasurablePart)
    = \emptyset \in \SigmaAlgebra$;
  \item se $0 \notin \VarMeasurablePart$ e $c \in \VarMeasurablePart$, allora
    $\CharacteristicApplication{\MeasurablePart}^{-1} (\VarMeasurablePart)
    = \MeasurablePart \in \SigmaAlgebra$;
  \item se $0 \in \VarMeasurablePart$ e $c \notin \VarMeasurablePart$, allora
    $\CharacteristicApplication{\MeasurablePart}^{-1} (\VarMeasurablePart)
    = \Complement{\MeasurablePart} \in \SigmaAlgebra$;
  \item se $0, c \in \VarMeasurablePart$, allora
    $\CharacteristicApplication{\MeasurablePart}^{-1} (\VarMeasurablePart)
    = \MeasureSpace \in \SigmaAlgebra$. \EndProof
\end{itemize}
\begin{Definition}
  Siano
  \begin{itemize}
    \item $(\MeasureSpace,\SigmaAlgebra,\Measure)$ uno spazio di misura;
    \item $\MeasurablePart \in \SigmaAlgebra$.
  \end{itemize}
  L'\Define{integrale}[di un'applicazione caratteristica][integrale]
  dell'applicazione caratteristica
  $\CharacteristicApplication{\MeasurablePart}:
  \MeasurableSpace \rightarrow \mathbb{R}$
  di $\MeasurablePart$, denotato
  $\int \CharacteristicApplication{\MeasurablePart} d\Measure$, \`e 
  $\int \CharacteristicApplication{\MeasurablePart} d\Measure
  = \Measure(\MeasurablePart)$.
\end{Definition}
\begin{Definition}
  Siano
  \begin{itemize}
    \item $(\MeasureSpace,\SigmaAlgebra,\Measure)$ uno spazio di misura;
    \item $N \in \mathbb{N}$;
    \item $(\MeasurablePart_n)_{n \in N} \in \SigmaAlgebra^N$;
    \item $(\Scalar_n)_{n \in N} \in (\NotZero{\mathbb{C}})^N$
    \item $\MeasurableApplication
      = \sum_{n \in N} \Scalar_n\CharacteristicApplication{\MeasurablePart_n}$.
  \end{itemize}
  Diciamo che $\MeasurableApplication$ \`e
  un'\Define{applicazione semplice}[semplice][applicazione]
  o
  una \Define{funzione semplice}[semplice][funzione].
\end{Definition}
\begin{Theorem}
  Con le notazioni della definizione precedente,
  assumendo $\MeasurableApplication$ reale,
  l'applicazione semplice
  $\MeasurableApplication$ \`e misurabile.
\end{Theorem}
\Proof Per ogni $c \in \Image{\MeasurableApplication}$, definiamo
$S_c
= \lbrace X \in \PowerSet{N} | \sum_{n \in N} \Scalar_n = c \rbrace$.
\par Sia $\VarMeasurablePart \in \LebesgueSigmaAlgebra$.
\par Abbiamo
\begin{align*}
  \MeasurableApplication^{-1}(\VarMeasurablePart)
  &= \MeasurableApplication^{-1}
    \left ( \VarMeasurablePart \cap \Image{\MeasurableApplication} \right ),\\
  &= \MeasurableApplication^{-1}
    \left ( \bigcup_{c \in \VarMeasurablePart \cap \Image{\MeasurableApplication}}
    \lbrace c \rbrace \right ),\\
  &= \bigcup_{c \in \VarMeasurablePart \cap \Image{\MeasurableApplication}}
    \MeasurableApplication^{-1}(\lbrace c \rbrace).
\end{align*}
\par Basta dunque provare
\[
  \ForAll{c \in \Image{\MeasurableApplication}}
  {\MeasurableApplication^{-1}(c) \in \SigmaAlgebra}.
\]
\par Abbiamo
$\MeasurableApplication^{-1}(\lbrace c \rbrace)
= \bigcup_{X \in S_C} \bigcap_{n \in X} \MeasurablePart_n \in \SigmaAlgebra$.
\EndProof
\begin{Theorem}
  Con le notazioni della definizione precedente,
  assumendo $\MeasurableApplication$ reale,
  se
  \begin{itemize}
    \item $M \in \mathbb{N}$;
    \item $(\VarMeasurablePart_m)_{m \in M} \in \SigmaAlgebra^M$;
    \item $(\VarScalar_m)_{m \in M} \in (\NotZero{\mathbb{R}})^M$;
    \item $\MeasurableApplication
      = \sum_{m \in M}
        \VarScalar_m\CharacteristicApplication{\VarMeasurablePart_m}$,
  \end{itemize}
  allora
  $\sum_{n \in N} \Scalar_n \Measure{\MeasurablePart_n}
    = \sum_{m \in M} \VarScalar_m \Measure{\VarMeasurablePart_m}$.
\end{Theorem}
\Proof Procediamo per doppia induzione su $N$ e $M$.
\par Siano $N = M = 1$.
\par Abbiamo 
\begin{align*}
  \MeasurableApplication
  &= \Scalar_0 \CharacteristicApplication{\MeasurablePart_0},
  &= \VarScalar_0 \CharacteristicApplication{\VarMeasurablePart_0},
\end{align*}
da cui
\begin{align*}
  \Image{\MeasurableApplication}
  &= \lbrace \Scalar_0 \rbrace,
  &= \lbrace \VarScalar_0 \rbrace,
\end{align*}
e quindi $\Scalar_0 = \VarScalar_0$.
\par Abbiamo inoltre
\begin{align*}
  \MeasurableApplication^{-1}(\Scalar_0)
  &= \MeasurablePart_0,\\
  &= \VarMeasurablePart_0.
\end{align*}
Dunque
$\MeasurablePart_0 = \VarMeasurablePart_0$
e infine
$\Scalar_0 \Measure{\MeasurablePart_0}
= \VarScalar_0 \Measure{\VarMeasurablePart_0}$.
%INCOMPLETA
\begin{Definition}
  Con le notazioni della definizione precedente,
  assumendo $\MeasurableApplication$ reale,
  l'\Define{integrale}[di un'applicazione semplice][integrale]
  dell'applicazione semplice 
  $\MeasurableApplication$
  $\int \MeasurableApplication d\Measure$, \`e 
  $\int \MeasurableApplication d\Measure
    = \sum_{n \in N} \Scalar_n
      \int \CharacteristicApplication{\MeasurablePart_n}d\Measure
    = \sum_{n \in N} \Scalar_n \MeasurablePart_n$.
\end{Definition}
\begin{Theorem}
  Siano
  \begin{itemize}
    \item $(\MeasureSpace,\SigmaAlgebra,\Measure)$ uno spazio di misura;
    \item $(\VarMeasureSpace,\VarSigmaAlgebra)$ uno spazio misurabile;
    \item $\MeasurableApplication: \MeasureSpace \rightarrow \VarMeasureSpace$
      un'applicazione misurabile;
    \item $\VarMeasure$ la misura immagine di $\Measure$ tramite
      $\MeasurableApplication$;
    \item $\VarMeasurableApplication:
      \VarMeasureSpace \rightarrow \mathbb{R}$
      un'applicazione misurabile.
  \end{itemize}
  $\VarMeasurableApplication$ \`e integrabile rispetto a
  $\VarMeasure$ se e solo se
  $\VarMeasurableApplication \circ \MeasurableApplication$
  \`e integrabile rispetto a
  $\Measure$ e in tal caso vale
  $\int \VarMeasurableApplication d\VarMeasure
    = \int (\VarMeasurableApplication \circ \MeasurableApplication)  d\Measure$.
\end{Theorem}
