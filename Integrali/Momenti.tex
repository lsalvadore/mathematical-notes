\section{Momenti.}
\label{Integrali_Momenti}
\begin{Definition}
	Sia $f: \mathbb{R} \rightarrow \mathbb{R}$ una funzione continua e $c \in \mathbb{R}$. Per ogni $n \in \mathbb{N}$ definiamo \Define{momento semplice}[di una funzione reale continua][momento] o anche solamente \Define{momento}[di una funzione reale continua][momento] di origine $c$ e ordine $n$ di $f$ la quantit\`a
\[
	\int_{-\infty}^{+\infty} (x - c)^n f(x) dx.
\]
	Qualora non venga specificata l'origine $c$, si sottointende $c = 0$.
	Chiamiamo inoltre
	\begin{itemize}
		\item \Define{media}[di una funzione reale continua][media] il primo momento di $f$;
		\item \Define{momento centrale}[centrale][momento] di $f$ il momento semplice di $f$ di ordine $n$ e di origine la media di $f$;
		\item \Define{varianza} il secondo momento di $f$ di origine la media di $f$.
	\end{itemize}
\end{Definition}
