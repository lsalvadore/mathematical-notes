\section{Matrici triangolari.}
\label{Matrici_MatriciTriangolari}
\begin{Definition}
	Siano $\Ring$ un anello, $n, m \in \mathbb{N}$ e $\Matrix \in \Ring^{n \times m}$.
  Diciamo che $\Matrix$ \`e
	\begin{itemize}
		\item \Define{triangolare superiore}[triangolare superiore][matrice] quando
      $\ForAll{(i,j) \in n \times m}{\Implies{i > j}{\Matrix_i^j = 0}}$:
      nel caso quadrato abbiamo
\[
\lmatrix
\begin{array}{ccc}
\Matrix_0^0 & \cdots & \Matrix_{n - 1}^{n - 1}\\
& \ddots & \vdots\\
& & \Matrix_{n - 1}^{n - 1}
\end{array}
\rmatrix;
\]
		\item \Define{triangolare strettamente superiore}[triangolare strettamente superiore][matrice] quando
      $\ForAll{(i,j) \in n \times m}{\Implies{i \geq j}{\Matrix_i^j = 0}}$:
      nel caso quadrato abbiamo
\[
\lmatrix
\begin{array}{cccc}
0 & \Matrix_0^1 & \cdots & \Matrix_0^{n - 1}\\
& 0 & \ddots & \vdots\\
& & 0 & \Matrix_{n - 2}^{n - 1}\\
& & & 0
\end{array}
\rmatrix;
\]
		\item \Define{triangolare inferiore}[triangolare inferiore][matrice] quando
      $\ForAll{(i,j) \in n \times m}{\Implies{i < j}{\Matrix_i^j = 0}}$:
      nel caso quadrato abbiamo
\[
\lmatrix
\begin{array}{ccc}
\Matrix_0^0 & &\\
\vdots & \ddots &\\
\Matrix_{n - 1}^0 & \cdots & \Matrix_{n - 1}^{n - 1}\\
\end{array}
\rmatrix;
\]
		\item \Define{triangolare strettamente inferiore}[triangolare strettamente inferiore][matrice] quando
    $\ForAll{(i,j) \in n \times m}{\Implies{i \leq j}{\Matrix_i^j = 0}}$:
      nel caso quadrato abbiamo
\[
\lmatrix
\begin{array}{cccc}
0 & & &\\
\Matrix_1^0 & \ddots & &\\
\vdots & \ddots & \ddots &\\
\Matrix_{n - 1}^0 & \cdots & \Matrix_{n - 1}^{n - 2} & 0\\
\end{array}
\rmatrix;
\]
	\end{itemize}
\end{Definition}
\begin{Theorem}
  Siano
  \begin{itemize}
    \item $m, n, p \in \NotZero{\mathbb{N}}$;
    \item $\Ring$ un anello;
    \item $\TriangularMatrix \in \Ring^{m \times n}$;
    \item $\VarTriangularMatrix \in \Ring^{n \times p}$;
    \item $\Matrix = \TriangularMatrix\VarTriangularMatrix$.
  \end{itemize}
  Allora
  \begin{itemize}
    \item se $\TriangularMatrix$ e $\VarTriangularMatrix$ sono triangolari
      inferiori, $\Matrix$ \`e triangolare inferiore;
    \item se $\TriangularMatrix$ e $\VarTriangularMatrix$ sono triangolari
      strettamente inferiori, $\Matrix$ \`e triangolare strettamente inferiore.
  \end{itemize}
\end{Theorem}
\Proof Sia $(k,j) \in m \times p$. Abbiamo
$\Matrix_k^j
= \sum_{l \in n} \TriangularMatrix_k^l \VarTriangularMatrix_l^j$.
\par Ora,
\begin{itemize}
  \item se $\TriangularMatrix$ e $\VarTriangularMatrix$ sono triangolari
    inferiori,
    $\ForAll{(k,l) \in m \times n}{
      \Implies{k < l}{\TriangularMatrix_k^l = 0}}$;
  \item se $\TriangularMatrix$ e $\VarTriangularMatrix$ sono triangolari
    strettamente inferiori,
    $\ForAll{(k,l) \in m \times n}{
      \Implies{k \leq l}{\TriangularMatrix_k^l = 0}}$.
\end{itemize}
\par Ne segue immediatamente la tesi. \EndProof
\begin{Definition}
	Siano $\Ring$ un anello, $n, m \in \mathbb{N}$ e $\Matrix \in \Ring^{n \times m}$.
  Sia $(\tau_k)_{k \in n}$ una successione tale che, per ogni $k \in n$,
  \begin{itemize}
    \item $\Implies{j < \tau_k}{\Matrix_k^j = 0}$;
    \item $\Matrix_k^{\tau_k} \neq 0$.
  \end{itemize}
  Diciamo che $\Matrix$ \`e una
  \Define{matrice a gradini}[a gradini][matrice]
  quando la successione $(\tau_k)_{k \in n}$ \`e strettamente crescente.
\end{Definition}
\begin{Theorem}
  Con le notazioni della definizione precedente, $\Matrix$ \`e una matrice
  triangolare superiore.
\end{Theorem}
\Proof Procediamo per induzione su $n$. Il teorema \`e immediato se $n = 1$.
\par Supponiamo il teorema dimostrato per $n = N \in \mathbb{N}$ e proviamolo
per $n = N + 1$. La sottomatrice $\Matrix'$ ottenuta da $\Matrix$
eliminando la prima riga e la prima colonna \`e ancora una matrice a gradini
e dunque vi si pu\`o applicare l'ipotesi induttiva: $\Matrix'$ \`e
triangolare superiore.
\par Poich\'e $\Matrix$ \`e a gradini, tutti gli elementi
$(\Matrix_k^0)_{k = 1}^{n - 1}$ devono essere nulli. Ne consegue che
$\Matrix$ \`e triangolare superiore. \EndProof
