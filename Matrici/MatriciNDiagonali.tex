\section{Matrici $N$-diagonali.}
\label{Matrici_MatriciNDiagonali}
\begin{Definition}
	Siano $\Ring$ un anello, $n \in \mathbb{N}$ e $\Matrix \in \Ring^{n \times n}$. Fissato $N \in \mathbb{N}$, chiamiamo $\Matrix$ \Define{$2N + 1$-diagonale}[$N$-diagonale][matrice] quando $\ForAll{(i,j) \in n \times n}{\Implies{\AbsoluteValue{i - j} > N}{\Matrix_i^j = 0}}$. In particolare, chiamiamo $\Matrix$
\begin{itemize}
	\item \Define{diagonale}[diagonale][matrice] se \`e $1$-diagonale:
\[
\lmatrix
\begin{array}{ccc}
\Matrix_0^0 & &\\
& \ddots &\\
& & \Matrix_{n - 1}^{n - 1}
\end{array}
\rmatrix;
\]
	\item \Define{tridiagonale}[tridiagonale][matrice] se \`e $3$-diagonale:
\[
\lmatrix
\begin{array}{ccccc}
\Matrix_0^0 & \Matrix_0^1 & & &\\
\Matrix_1^0 & \ddots & \ddots & &\\
& \ddots  & \ddots & \ddots &\\
& & \ddots & \ddots & \Matrix_{n - 2}^{n - 1}\\
& & & \Matrix_{n - 1}^{n - 2} & \Matrix_{n - 1}^{n - 1}
\end{array}
\rmatrix;
\]
\end{itemize}
e cos\`i via.
\end{Definition}
\begin{Theorem}
	Siano $\Ring$ un anello, $n \in \mathbb{N}$, $\Matrix, \DiagonalMatrix \in \Ring^{n \times n}$, con $\DiagonalMatrix$ diagonale. Abbiamo, posto $A = \DiagonalMatrix\Matrix$ e $B = \Matrix\DiagonalMatrix$,
	\begin{itemize}
		\item $A_i^j = \DiagonalMatrix_i^i\Matrix_i^j$;
		\item $B_i^j = \Matrix_i^j\DiagonalMatrix_j^j$.
	\end{itemize}
\end{Theorem}
\Proof Segue per calcolo diretto. \EndProof
