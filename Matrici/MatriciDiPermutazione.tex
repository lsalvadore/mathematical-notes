\section{Matrici di permutazione.}
\label{Matrici_MatriciDiPermutazione}
\begin{Definition}
	Dato una anello con identit\`a $\RingWithIdentity$, chiamiamo \Define{matrice di permutazione}[di permutazione][matrice] una matrice quadrata $\PermutationMatrix = (\PermutationMatrixElement_{i,j})_{(i,j) \in n \times n} \in \RingWithIdentity^{n \times n}$ ($n \in \NotZero{\mathbb{N}}$) tale che
\begin{itemize}
	\item $\ForAll{(i,j) \in n \times n}{\PermutationMatrixElement_{i,j} \in \lbrace 0, 1 \rbrace}$;
	\item $\ForAll{i \in n}{\ForAll{(j,j') \in n \times n}{\Implies{\And{\PermutationMatrixElement_{i,j} = 1}{\PermutationMatrixElement_{i,j'} = 1}}{j = j'}}}$;
	\item $\ForAll{j \in n}{\ForAll{(i,i') \in n \times n}{\Implies{\And{\PermutationMatrixElement_{i,j} = 1}{\PermutationMatrixElement_{i',j} = 1}}{i = i'}}}$.
\end{itemize}
\end{Definition}
\begin{Theorem}
	Se $\PermutationMatrix$ \`e una matrice di permutazione, lo \`e anche $\Transposed{\PermutationMatrix}$.
\end{Theorem}
\Proof Segue direttamente dalle definizioni. \EndProof
\begin{Theorem}
	Sia $\PermutationMatrix$ una matrice quadrata di ordine $n$. $\PermutationMatrix$ \`e di permutazione se e solo se esistono uniche permutazioni $\Permutation, \VarPermutation \in \SymmetricGroup{n}$ tali che
	\begin{itemize}
		\item $\ForAll{i \in n}{\PermutationMatrix_{\Permutation(i)} = \Identity_i}$;
		\item $\ForAll{j \in n}{\PermutationMatrix^{\VarPermutation(j)} = \Identity^j}$.
	\end{itemize}
\end{Theorem}
\Proof $\Identity$ \`e una matrice di permutazione e modificandone l'ordine delle righe mantiene la propriet\`a di avere un solo $1$ per ogni riga e per ogni colonna. Dunque, se $\ForAll{i \in n}{\PermutationMatrix_i = \Identity_{\Permutation{i}}}$, allora $\PermutationMatrix$ \`e di permutazione.
\par Supponiamo $\PermutationMatrix$ di permutazione e definiamo la permutazione $\Permutation \in \SymmetricGroup{n}$ che a $i \in n$ associa l'indice dell'unica riga di $\PermutationMatrix$ che contiene un $1$ in posizione $i$. Abbiamo allora $\PermutationMatrix_{\Permutation(i)} = \Identity_i$.
\par L'unicit\`a segue dal fatto che tutte le righe di $\PermutationMatrix$
da una parte e di $\Identity$ dall'altra sono distinte.
\par Il resto del teorema segue direttamente per trasposizione. \EndProof
\begin{Definition}
  Con le notazioni del teorema precedente, chiamiamo
  \begin{itemize}
    \item $\Permutation^{-1}$
      \Define{permutazione associata alle righe}[associata alle righe di una matrice di permutazione][permutazione];
    \item $\VarPermutation^{-1}$
      \Define{permutazione associata alle colonne}[associata alle colonne di una matrice di permutazione][permutazione].
  \end{itemize}
\end{Definition}
\begin{Theorem}
	Siano
	\begin{itemize}
		\item $\RingWithIdentity$ un anello con identit\`a;
		\item $m, n \in \mathbb{N}$;
		\item $\Matrix \in \RingWithIdentity^{m \times n}$;
		\item $\PermutationMatrix \in \RingWithIdentity^{m \times m}$ matrice di
      permutazione;
		\item $\VarPermutationMatrix \in \RingWithIdentity^{n \times n}$ matrice di
      permutazione;
    \item $\Permutation \in \SymmetricGroup{m}$ permutazione associata alle
      righe di $\PermutationMatrix$;
    \item $\VarPermutation \in \SymmetricGroup{n}$ permutazione associata alle
      colonne di $\VarPermutationMatrix$.
	\end{itemize}
	Abbiamo
	\begin{itemize}
		\item $\ForAll{i \in m}{(\PermutationMatrix\Matrix)_i = \Matrix_{\Permutation(i)}}$;
		\item $\ForAll{j \in n}{(\Matrix\VarPermutationMatrix)^{j} = \Matrix^{\VarPermutation(j)}}$.
	\end{itemize}
\end{Theorem}
\Proof Abbiamo
\begin{itemize}
	\item per ogni $j \in n$, $(\PermutationMatrix\Matrix)_{\Permutation^{-1}(i)}^j = \sum_{k \in m} \PermutationMatrix_{\Permutation^{-1}(i)}^k \Matrix_k^j = \sum_{k \in m} \Identity_i^k \Matrix_k^j = \Matrix_i^j$ e quindi $\ForAll{i \in m}{(\PermutationMatrix\Matrix)_{\Permutation^{-1}(i)} = \Matrix_i}$;
	\item per ogni $i \in m$, $(\Matrix\VarPermutationMatrix)_i^{\VarPermutation^{-1}(j)} = \sum_{k \in n} \Matrix_i^k\VarPermutationMatrix_k^{\VarPermutation^{-1}(j)} = \sum_{k \in n} \Matrix_i^k \Identity_k^j = \Matrix_i^j$ e quindi $\ForAll{j \in n}{(\Matrix\VarPermutationMatrix)^{\VarPermutation^{-1}(j)} = \Matrix^j}$. \EndProof
\end{itemize}
\begin{Theorem}
	Siano $\RingWithIdentity$ un anello con identit\`a, $n \in \mathbb{N}$ e $\PermutationMatrix \in \RingWithIdentity^{n \times n}$ una matrice di permutazione. $\PermutationMatrix$ \`e una matrice ortogonale (unitaria nel caso $\RingWithIdentity = \mathbb{C}$).
\end{Theorem}
\Proof Segue per calcolo diretto. \EndProof
