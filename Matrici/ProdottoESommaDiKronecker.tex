\section{Prodotto e somma di Kronecker}
\label{Matrici_ProdottoESommaDiKronecker}
\begin{Definition}
  Fissato un anello $\Ring$, siano
  \begin{itemize}
    \item siano $m, n, p, q \in \NotZero{\mathbb{N}}$;
    \item $A \in \Ring^{m \times n}$;
    \item $B \in \Ring^{p \times q}$.
  \end{itemize}
  Chiamiamo
  \Define{prodotto di Kronecker}[di Kronecker][prodotto]
  la matrice $A \KroneckerProduct B \in \Ring^{mp \times nq}$ definita da
  $(A \KroneckerProduct B)_k^j = A_{q_1}^{q_2} B_{r_1}^{r_2}$
  al variare di $(k,j) \in mp \times nq$, dove
  \begin{itemize}
    \item $q_1$ \`e il quoziente della divisione euclidea di $k$ per $p$;
    \item $r_1$ \`e il resto della divisione euclidea di $k$ per $p$;
    \item $q_2$ \`e il quoziente della divisione euclidea di $j$ per $q$;
    \item $r_2$ \`e il resto della divisione euclidea di $j$ per $q$;
  \end{itemize}
  vale a dire
  \[
    A \KroneckerProduct B =
    \lmatrix
    \begin{array}{ccc}
      A_0^0 B & \cdots & A_0^{n - 1} B,\\
      \vdots & \ddots \vdots,\\
      A_{m - 1}^0 B & \cdots & A_{m - 1}^{n - 1} B,\\
    \end{array}
    \rmatrix.
  \]
  Se $\Ring$ \`e un anello unitario, chiamiamo inoltre
  \Define{somma di Kronecker}[di Kronecker][somma]
  la matrice
  $A \KroneckerSum B
  = \Identity \KroneckerProduct A + B \KroneckerProduct \Identity$.
\end{Definition}
\begin{Theorem}
  Fissato un anello $\Ring$, siano
  \begin{itemize}
    \item siano $m, n, p, q, s, t \in \NotZero{\mathbb{N}}$;
    \item $A \in \Ring^{m \times n}$;
    \item $B \in \Ring^{p \times q}$;
    \item $C \in \Ring^{n \times s}$;
    \item $D \in \Ring^{q \times t}$.
  \end{itemize}
  Abbiamo
  \[
    (A \KroneckerProduct B)(C \KroneckerProduct D)
    = AC \KroneckerProduct BD.
  \]
\end{Theorem}
\Proof Abbiamo, per $(k,j) \in mq \times pt$,
posto
\begin{itemize}
    \item $q_1$ \`e il quoziente della divisione euclidea di $k$ per $p$;
    \item $r_1$ \`e il resto della divisione euclidea di $k$ per $p$;
    \item $q_2$ \`e il quoziente della divisione euclidea di $j$ per $q$;
    \item $r_2$ \`e il resto della divisione euclidea di $j$ per $q$;
    \item $q_3$ \`e il quoziente della divisione euclidea di $l$ per $q$;
    \item $r_3$ \`e il resto della divisione euclidea di $l$ per $q$;
\end{itemize}
\begin{align*}
  ((A \KroneckerProduct B)(C \KroneckerProduct D))_k^j
  &= (A \KroneckerProduct B)_k^l (C \KroneckerProduct D)_l^j,\\
  &= A_{q_1}^{q_3} B_{r_1}^{r_3} C_{q_3}^{q_4} D_{r_3}^{r_4},\\
  &= A_{q_1}^{q_3} C_{q_3}^{q_4} B_{r_1}^{r_3} D_{r_3}^{r_4},\\
  &= AC \KroneckerProduct BD.\text{ \EndProof}
\end{align*}
\begin{Corollary}
  Fissato un anello $\Ring$, siano
  \begin{itemize}
    \item siano $n, p \in \NotZero{\mathbb{N}}$;
    \item $A \in \Ring^{n \times n}$ invertibile;
    \item $B \in \Ring^{p \times p}$ invertibile.
  \end{itemize}
  Abbiamo $(A \KroneckerProduct B)^{-1} = A^{-1} \KroneckerProduct B^{-1}$.
\end{Corollary}
\Proof Abbiamo
\begin{align*}
  (A \KroneckerProduct B)(A^{-1} \KroneckerProduct B^{-1})
  &= AA^{-1} \KroneckerProduct BB^{-1},\\
  &= \Identity_n \KroneckerProduct \Identity_p,\\
  &= \Identity_{np},\\
  &= A^{-1}A \KroneckerProduct B^{-1}B,\\
  &= (A^{-1} \KroneckerProduct B^{-1})(A \KroneckerProduct B).\text{ \EndProof}
\end{align*}
