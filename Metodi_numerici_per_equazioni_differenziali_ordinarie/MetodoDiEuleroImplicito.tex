\subsubsection{Metodo di Eulero implicito.}
\label{MetodiNumericiPerEquazioniDifferenzialiOrdinarie_MetodoDiEuleroImplicito}
\begin{Definition}
	Dati un problema di Cauchy su un intervallo $I$
	\[
	\begin{cases}
		y' = f(t,y),\\
		y(t_0) = y_0,
	\end{cases}
	\]
	e una discretizzazione uniforme di $I$ di passo $\Step \in \RealPositive$ e $N$ nodi, chiamiamo \Define{metodo di Eulero implicito}[di Eulero implicito][metodo] o \Define{metodo di Eulero all'indietro}[di Eulero all'indietro][metodo] il metodo a un passo definito da
	\[
	\begin{cases}
		y_0 = y_0,\\
		y_{j + 1} = y_j + \Step \phi(t_j,y_j)\text{ per }i \in \mathbb{N},
	\end{cases}
	\]
	con
	\begin{itemize}
		\item $y_{j + 1} = \xi$ tale che $y_j + \Step f(t_{j + 1},\xi) = \xi$;
		\item $\phi(t_j,y_j) = \eta$ tale che $y_j + \Step \eta = y_{j + 1}$.
	\end{itemize}
\end{Definition}
\begin{listing}
	\insertcode{octave}{"Metodi_numerici_per_equazioni_differenziali_ordinarie/EuleroImplicito.m"}
	\caption{Metodo di Eulero implicito implementato in \LanguageName{octave}.}
\end{listing}
\begin{Theorem}
	Con le notazioni del teorema precedente, abbiamo
	\[
	\begin{cases}
		y_0 = y_0,\\
		y_{j + 1} = y_j + \Step f(t_{j + 1},y_{j + 1})\text{ per }i > 0.
	\end{cases}
	\]
\end{Theorem}
\Proof Il metodo descritto nell'enunciato si ottiene dalla definizione semplicemente sostituendo $y_{j + 1}$ a $\xi$. \EndProof
\begin{Theorem}
	Il metodo di Eulero implicito \`e un metodo di Runge-Kutta a $1$ stadio con \tableau
	\[
	\begin{array}{c|c}
	1	&	1\\
	\hline
		&	1
	\end{array}.
	\]
\end{Theorem}
\Proof Segue per verifica diretta. \EndProof
\begin{Theorem}
	Con le notazioni della definizione precedente, assumendo $y \in \CClass{2}$, il metodo di Eulero implicito \`e consistente di ordine $1$.
\end{Theorem}
\Proof Abbiamo evidentemente $\LocalTruncationError_j = 0$. Inoltre, per ogni $i \in N - 1$ abbiamo $y(t_{j + 1}) - y(t_j) - \Step f(t_{j + 1},y_{j + 1}) = y(t_{j + 1}) - y(t_j) - \Step y'(t_{j + 1}) = \Step y'(t_{j + 1}) + \frac{\Step^2}{2} y''(\xi) - \Step y'(t_{j + 1}) = \BigO{\Step^2}$, dove $\xi \in (t_j,t_{j + 1})$. \EndProof
\begin{Theorem}
	Il metodo di Eulero implicito
	\begin{itemize}
		\item ha funzione di stabilit\`a $\StabilityFunction(z) = (1 - z)^{-1}$;
		\item ha regione di stabilit\`a \`e $\StabilityRegion = \lbrace z | \AbsoluteValue{1 + z} > 1 \rbrace$ (e dunque \`e $A$-stabile).
	\end{itemize}
\end{Theorem}
\Proof Applichiamo il metodo di Eulero esplicito con passo $\Step$ al problema test di Dahlquist di parametro $\DahlquistParameter$: 
\[
\begin{cases}
	y_0 = y_0,\\
	y_{j + 1} = y_j + \Step \DahlquistParameter y_{j + 1}\text{ per }j > 0.
\end{cases}
\]
\par Abbiamo dunque, per $j \in \mathbb{N}$, $y_{j + 1} = (1 - \Step \DahlquistParameter)^{-1}y_j$ e dunque $y_j = (1 - \Step \DahlquistParameter)^{-j}$: la successione \`e infinitesima se e solo se $\AbsoluteValue{1 - \Step \DahlquistParameter} > 1$, equivalente a $\AbsoluteValue{1 - z} > 1$. \EndProof
