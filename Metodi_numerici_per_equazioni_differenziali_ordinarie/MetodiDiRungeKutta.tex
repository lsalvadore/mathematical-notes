\subsection{Metodi di Runge-Kutta.}
\label{MetodiNumericiPerEquazioniDifferenzialiOrdinarie_MetodiDiRungeKutta}
\begin{Definition}
\label{MetodiNumericiPerEquazioniDifferenzialiOrdinarie_Definizione_MetodoDiRungeKutta}
	Siano dati un problema di Cauchy su un intervallo $I$
	\[
	\begin{cases}
		y' = f(t,y),\\
		y(t_0) = y_0,
	\end{cases}
	\]
	e una discretizzazione uniforme di $I$ di passo $\Step \in \RealPositive$ ed $N \in \NaturalExtended$ nodi. Siano inoltre
	\begin{itemize}
		\item $s \in \mathbb{N}$;
		\item $A = (a_{i,j})_{(i,j) \in \NaturalShift{s} \times \NaturalShift{s}} \in \mathbb{R}^{\NaturalShift{s} \times \NaturalShift{s}}$;
		\item $b, c \in \mathbb{R}^s$.
	\end{itemize}
	Chiamiamo \Define{metodo di Runge-Kutta ad $s$ stadi}[di Runge-Kutta ad $s$-stadi][metodo] il metodo numerico definito da
	\[
	\begin{cases}
		y_0 = y_0,\\
		Y_i^{(j + 1)} = y_j + \Step\sum_{k \in \NaturalShift{s}} a_{i,k}f(t_j + c_k\Step,Y_k^{(j + 1)})\text{ per }i \in \NaturalShift{s}\text{ e }j \in N,\\
		y_{j + 1} = y_j + \Step\sum_{k \in \NaturalShift{s}} b_k f(t_j + c_k\Step,Y_k^{(j + 1)})\text{ per }j \in N.
	\end{cases}
	\]
	Rappresenteremo spesso un metodo di Runge-Kutta tramite una tabella del tipo
	\[
	\begin{array}{c|c}
		c	&	A\\
		\hline
			&	b
	\end{array},
	\]
	detta \Define{\tableau\ di Butcher}[di Butcher][tableau].
\end{Definition}
\begin{Definition}
	Diciamo che un metodo di Runge-Kutta \`e \Define{banale}[di Runge-Kutta banale][metodo] quando ha $0$ stadi.
\end{Definition}
\begin{Theorem}
	Con le notazioni della definizione precendente, se $A$ \`e strettamente triangolare inferiore, allora il metodo di Runge-Kutta \`e esplicito.
\end{Theorem}
\Proof Se $A$ \`e strettamente triangolare inferiore, abbiamo allora, per ogni $(i,j) \in \NaturalShift{s} \times N$,
$Y_i^{(j + 1)} = y_j + \Step\sum_{k \in \NaturalShift{s}} a_{i,k}f(t_j + c_k\Step,Y_k^{(j + 1)}) = y_j + \Step\sum_{k \in \NaturalShift{i - 1}} a_{i,k}f(t_j + c_k\Step,Y_k^{(j + 1)})$. Quindi tutti gli elementi della famiglia $(Y_i^{(j + 1)})_{i \in \NaturalShift{s}}$ sono calcolabili in ordine direttamente, conoscendo solo gli elementi precedenti. \EndProof
\begin{Corollary}
	Con le notazioni del teorema precedente, se $A$ \`e simile a una matrice triangolare inferiore per mezzo di una matrice di permutazione, allora il metodo di Runge-Kutta \`e esplicito.
\end{Corollary}
\Proof Segue immediatamente dal teorema precedente, dopo riordino delle componenti di $(Y_i^{(j + 1)})_{i \in \NaturalShift{s}}$ ($(i,j) \in \NaturalShift{s} \times N$). \EndProof
\par Nel seguito, assumeremo sempre, con le notazioni della definizione \ref{MetodiNumericiPerEquazioniDifferenzialiOrdinarie_Definizione_MetodoDiRungeKutta}:
\begin{itemize}
	\item $\ForAll{i \in \NaturalShift{s}}{\sum_{k \in \NaturalShift{s}} a_{i,k} = c_k}$;
	\item $\sum_{k \in \NaturalShift{s}} b_k = 1$.
\end{itemize}
\par Quest'assunto \`e giustificato dal teorema seguente, che ci consente di applicare un metodo numerico ad un'equazione differenziale autonoma ottenuta aggiungendo la variabile indipendente dalle funzioni incognite integrando esattamente il tempo.
\begin{Theorem}
	Con le notazioni della definizione \ref{MetodiNumericiPerEquazioniDifferenzialiOrdinarie_Definizione_MetodoDiRungeKutta}, se abbiamo
	\begin{itemize}
		\item $\ForAll{i \in \NaturalShift{s}}{\sum_{k \in \NaturalShift{s}} a_{i,k} = c_k}$;
		\item $\sum_{k \in \NaturalShift{s}} b_k = 1$;
	\end{itemize}
	allora il problema di Cauchy
	\[
	\begin{cases}
		y' = 1,\\
		y(0) = 0,
	\end{cases}
	\]
	\`e risolto esattamente nei nodi $(t_j)_{j \in N}$ negli stadi intermedi.
\end{Theorem}
\Proof Abbiamo
\[
\begin{cases}
	Y_i^{(j + 1)} = y_j + \Step\sum_{k \in \NaturalShift{s}} a_{i,k}\text{ per }i \in \NaturalShift{s}\text{ e }j \in N,\\
	y_{j + 1} = y_j + \Step\sum_{k \in \NaturalShift{s}} b_k\text{ per }j \in N,
\end{cases}
\]
da cui
\[
\begin{cases}
	t_j + c_i \Step = t_j + \Step\sum_{k \in \NaturalShift{s}} a_{i,k}\text{ per }i \in \NaturalShift{s}\text{ e }j \in N,\\
	t_j + \Step = t_j + \Step\sum_{k \in \NaturalShift{s}} b_k\text{ per }j \in N,
\end{cases}
\]
da cui la tesi. \EndProof
\begin{Theorem}
	Un metodo di Runge-Kutta
	\[
	\begin{cases}
		y_0 = y_0,\\
		y_{j + 1} = y_j + \Step\phi_\Step(t_j,y_j)\text{ per }i > 0,
	\end{cases}
	\]
	per un problema di Cauchy
	\[
	\begin{cases}
		y' = f(t,y),\\
		y(t_0) = y_0,
	\end{cases}
	\]
	con $f$ lipschitziana \`e convergente se e solo se \`e consistente.
\end{Theorem}
\Proof Segue dal teorema di Lax-Richmyer, osservando che la lipschitzianit\`a di $f$ implica quella di $\phi$. \EndProof
\begin{Definition}
	Chiamiamo \Define{serie di Neumann}[di Neumann][serie] una serie $\sum_{n \in \mathbb{N}} A^n$, dove $A$ \`e un operatore qualsiasi per cui abbiano un senso le operazioni di somma e composizione.
\end{Definition}
\begin{Lemma}
	Sia $\Matrix$ una matrice quadrata tale che $\SpectralRadius{\Matrix} < 1$. Allora
	\begin{itemize}
		\item $\Identity - \Matrix$ \`e invertibile;
		\item la serie di Neumann $\sum_{n \in \mathbb{N}} \Matrix^n$ converge;
		\item $(\Identity - \Matrix)^{-1} = \sum_{n \in \mathbb{N}} \Matrix^n$.
	\end{itemize}
\end{Lemma}
\Proof Poich\'e $\SpectralRadius{\Matrix} < 1$, $1$ non \`e autovalore di $\Matrix$ e dunque $\Identity - \Matrix$ \`e invertibile.
\par Esiste una norma matriciale $\Norm{\cdot}$ tale che, per opportuna costante $a \in (\SpectralRadius{\Matrix},1)$, $\SpectralRadius{\Matrix} \leq \Norm{\Matrix} \leq a$. Dunque la serie $\sum_{n \in \mathbb{N}} \Matrix^n$ \`e assolutamente convergente per confronto con la serie geometrica di ragione $a$.
\par Abbiamo infine $\lim_{N \rightarrow \infty} \Norm{\Identity - (\Identity - \Matrix)\sum_{n \in N} \Matrix^n} = \lim_{N \rightarrow \infty} \Norm{\Matrix^N} \leq \lim_{N \rightarrow \infty} \Norm{\Matrix}^N \leq \lim_{N \rightarrow \infty} a^N = 0$. \EndProof
\begin{Theorem}
	Sia un metodo di Runge-Kutta a $\StagesNumber$ stadi e consistente di ordine $\ConsistencyOrder$ definito dal \tableau\ di Butcher
	\[
	\begin{array}{c|c}
		c	&	A\\
		\hline
			&	b
	\end{array}.
	\]
	e siano inoltre
	\begin{itemize}
		\item $e \in \mathbb{R}^\StagesNumber$ tale che $\ForAll{i \in \NaturalShift{\StagesNumber}}{e_i = 1}$;
		\item $C = (c_i\KroneckerDelta{i}{j})_{(i,j) \in \NaturalShift{\StagesNumber}\times\NaturalShift{\StagesNumber}}$;
		\item $(l,k) \in \mathbb{N}^2$ tali che $l > 0$, $k \geq 0$ e $1 \leq l + k \leq \ConsistencyOrder$.
	\end{itemize}
	Allora
	\begin{itemize}
		\item vale l'identit\`a $\Transposed{b}A^kC^{l - 1} e = \frac{(l - 1)!}{(l + k)!}$;
		\item se il metodo di Runge-Kutta \`e esplicito, deve essere $\ConsistencyOrder \leq \StagesNumber$.
	\end{itemize}
\end{Theorem}
\Proof Fissiamo $l \in \mathbb{N}$ tale che $1 \leq l \leq \ConsistencyOrder$ e consideriamo il problema di Cauchy definito su $\RealPositive$
\[
\begin{cases}
	y' = y + t^{l - 1},\\
	y(0) = 0.
\end{cases}
\]
\par Esso ha soluzione $\int_0^t e^{t -\tau}\tau^{l - 1}d\tau \in \CClass{\infty}$.
\par Applichiamo un passo del metodo di Runge-Kutta:
\[
\begin{cases}
	Y_i = \Step\sum_{k \in \NaturalShift{\StagesNumber}} A_{i,k}(Y_k + (\Step c_k)^{l - 1})\text{ per }i \in \NaturalShift{s},\\
	y_1 = \Step\sum_{k \in \NaturalShift{\StagesNumber}} b_k(Y_k + (\Step c_k)^{l - 1}).
\end{cases}
\]
\par Posto $Y = (Y_i)_{i \in \NaturalShift{s}}$, la prima equazione \`e equivalente a $Y = \Step A Y + \Step^l A C^{l - 1} e$ cio\`e a $(\Identity - \Step A) Y = \Step^l A C^{l - 1} e$ e dunque, per il lemma precedente assumendo $h$ sufficientemente piccolo da garantire $\SpectralRadius{A} < 1$, a $Y = \Step^{l} \left( \sum_{k \in \mathbb{N}} \Step^k A^k \right ) A C^{l - 1} e$.
\par Per l'errore locale di troncamento $\LocalTruncationError$ abbiamo, utilizzando la seconda equazione,
\begin{align*}
	\Step\LocalTruncationError = y(h) - y_1 &=y(h) - \Step \Transposed{b} Y - \Transposed{b} \Step^l C^{l - 1} e,\\
	&= y(h) - \Step \Transposed{b}\Step^l \left ( \sum_{k \in \mathbb{N}} \Step^k A^k \right ) A C^{l - 1} e - \Transposed{b} \Step^l C^{l - 1}e,\\
	&= y(h) - \Transposed{b}\Step^l \left ( \left ( \sum_{k \in \mathbb{N}} \Step^k A^k \right ) \Step A + \Identity \right ) C^{l - 1}e,\\
	&= y(h) - \Transposed{b}\Step^l \left ( \left ( \sum_{k \in \mathbb{N}} \Step^{n + 1} A^{n + 1} \right ) + \Identity \right ) C^{l - 1}e,\\
	&= y(h) - \Transposed{b}\Step^l \left ( \sum_{k \in \mathbb{N}} \Step^k A^k \right ) C^{l - 1}e.
\end{align*}
\par Consideriamo lo sviluppo di Taylor di $y(h) = \sum_{k = 0}^\infty \frac{h^k y^{(k)}(0)}{n!}$. Al variare di $k$ in $\mathbb{N}$ abbiamo
\[
y^{(k)} = \begin{cases}
y^{(k - 1)} + \frac{(l - 1)!}{(l - k)!}t^{l - k}&\text{ se }k \in l + 1,\\
y^{(k - 1)}&\text{ altrimenti},
\end{cases}
\]
da cui
\[
y^{(k)}(0) = \begin{cases}
0&\text{ se }k \in l,\\
(l - 1)!&\text{ altrimenti}.
\end{cases}
\]
\par Abbiamo dunque $y(h) = \sum_{k \in \mathbb{N}} \frac{h^{l + k} (l - 1)!}{(l + k)!}$. Ne deduciamo, per l'errore locale di troncamento,
\begin{align*}
	\Step\LocalTruncationError &= y(h) - \Transposed{b}\Step^l \left ( \sum_{k \in \mathbb{N}} \Step^k A^k \right ) C^{l - 1}e,\\
	&= \sum_{k \in \mathbb{N}} \frac{h^{l + k} (l - 1)!}{(l + k)!} - \Transposed{b}\Step^l \left ( \sum_{k \in \mathbb{N}} \Step^k A^k \right ) C^{l - 1}e,\\
	&= \sum_{k \in \mathbb{N}} h^{l + k} \left ( \frac{(l - 1)!}{(l + k)!} - \Transposed{b} A^k C^{l - 1}e \right ).
\end{align*}
\par L'identit\`a $\Transposed{b}A^kC^{l - 1} e = \frac{(l - 1)!}{(l + k)!}$ per ogni $(l,k) \in \mathbb{N}^2$ tale che $l > 0$ e $1 \leq l + k \leq \ConsistencyOrder$ segue dalla consistenza di ordine $\ConsistencyOrder$ del metodo.
\par Infine, poniamo $l = 1$ e consideriamo il coefficiente $\gamma$ di $\Step^\StagesNumber$ nello sviluppo di Taylor di $\LocalTruncationError$: esso \`e $\gamma = \frac{1}{(\StagesNumber + 1)!} - \Transposed{b}A^{\StagesNumber}e$. Se il metodo \`e esplicito, allora $A$ \`e simile a una matrice strettamente triangolare inferiore per mezzo di una matrice di permutazione e dunque $A^{\StagesNumber} = 0$, da cui $\gamma = \frac{1}{(\StagesNumber + 1)!} \neq 0$ e quindi $\StagesNumber \geq \ConsistencyOrder$. \EndProof
\par Per un metodo di Runge-Kutta di $\StagesNumber \in \NaturalShift{10}$ stadi esplicito, si dimostra che il massimo ordine di convergenza possibile \`e dato dalla seguente tabella \ref{RungeKutta_OrdiniDiConvergenza}.
\begin{table}[h]
	\centering
	\begin{tabular}{|c|cccccccccc|}
		\hline
		Numero di stadi&1&2&3&4&5&6&7&8&9&10\\
		\hline
		Ordine massimo di convergenza&1&2&3&4&4&5&6&6&7&7\\
		\hline
	\end{tabular}
	\caption{Massimi ordini di convergenza di un metodo di Runge-Kutta esplicito al variare del numero di stadi in $\NaturalShift{10}$.}
	\label{RungeKutta_OrdiniDiConvergenza}
\end{table}
\par Nella pratice si tende a preferire metodi di Runge-Kutta a $4$ stadi, dato che essi possono arrivare ad ordine di convergenza $4$ mentre i metodi con un numero superiore di stadi aumentano s\'i l'ordine di convergenza, ma troppo lentamente rispetto alla complessit\`a del calcolo di stadi ulteriori. Si usano anche metodi di Runge-Kutta con un numero minore di $4$ stadi in quei casi in cui la soluzione del problema di Cauchy non \`e sufficientemente regolare per il raggiungimento dell'ordine di convergenza $4$.
\begin{Definition}
	Un metodo di Runge-Kutta esplicito che ha numero di stadi $\StagesNumber$ uguale al suo ordine di consistenza $\ConsistencyOrder$ si dice \Define{ottimale}[di Runge-Kutta ottimale][metodo].
\end{Definition}
\begin{Theorem}
	Sia un metodo di Runge-Kutta a $\StagesNumber$ stadi definito dal \tableau\ di Butcher
	\[
	\begin{array}{c|c}
		c	&	A\\
		\hline
			&	b
	\end{array}.
	\]
	Il metodo ha funzione di stabilit\`a $\StabilityFunction(z) = 1 + z \Transposed{b} (\Identity - z A)^{-1} e$, con $e \in \mathbb{R}^\StagesNumber$ tale che $\ForAll{i \in \NaturalShift{\StagesNumber}}{e_i = 1}$.
\end{Theorem}
\Proof Applichiamo il metodo di Runge-Kutta al problema test di Dahlquist di parametro $\DahlquistParameter$: fissato $j \in \mathbb{N}$ abbiamo
\[
\begin{cases}
	Y_i = y_j + \Step\sum_{k \in \NaturalShift{s}} a_{i,k}\DahlquistParameter Y_k)\text{ per }i \in \NaturalShift{s},\\
	y_{j + 1} = y_j + \Step\sum_{k \in \NaturalShift{s}} b_k \DahlquistParameter Y_k.
\end{cases}
\]
Definendo $Y = (Y_i)_{i \in \NaturalShift{s}}$ abbiamo
\[
\begin{cases}
	Y = y_j e + \Step \DahlquistParameter A Y\text{ per }i \in \NaturalShift{s},\\
	y_{j + 1} = y_j + \Step \DahlquistParameter \Transposed{b_k} Y.
\end{cases}
\]
\par Ponendo $z = \Step \DahlquistParameter$, dalla prima equazione deduciamo, assumendo che $z^{-1}$ non sia autovalore di $A$ (\`e sempre possibile scegliere $\Step$ in modo da evitarlo\footnote{Se $z^{-1}$ fosse autovalore di $A$ il sistema lineare che definisce $Y$ sarebbe singolare e il metodo di Runge-Kutta non sarebbe applicabile.}) $Y = y_j (\Identity - zA)^{-1}e$. Dunque $y_{j + 1} = y_j + z \Transposed{b} y_j (\Identity - zA)^{-1}e = (1 + z \Transposed{b}(\Identity - zA)^{-1}) y_j$. \EndProof
\begin{Corollary}
	Con le notazioni del teorema precedente, abbiamo $\StabilityRegion = \lbrace z \in \mathbb{C} | \AbsoluteValue{1 + z\Transposed{b}(\Identity - z A)^{-1}e} < 1 \rbrace$.
\end{Corollary}
\Proof Segue direttmante dal teorema precedente. \EndProof
\begin{Theorem}
	Sia un metodo di Runge-Kutta esplicito a $\StagesNumber$ stadi e consistente di ordine $\ConsistencyOrder$ definito dal \tableau\ di Butcher
	\[
	\begin{array}{c|c}
		c	&	A\\
		\hline
			&	b
	\end{array}.
	\]
	e sia inoltre $e \in \mathbb{R}^\StagesNumber$ tale che $\ForAll{i \in \NaturalShift{\StagesNumber}}{e_i = 1}$.
	Il metodo ha funzione di stabilit\`a $\StabilityFunction(z) = \sum_{j = 0}^\ConsistencyOrder \frac{z^\ConsistencyOrder}{\ConsistencyOrder !} + \sum_{j = \ConsistencyOrder + 1}^\StagesNumber z^j \Transposed{b} A^{j - 1} e$.
\end{Theorem}
\Proof Per l'esplicit\`a del metodo, $A$ \`e nilpotente di ordine $\StagesNumber$, dunque $\ForAll{z \in \mathbb{C}}{\SpectralRadius{zA} = 0}$. Abbiamo allora $(\Identity - z A)^{-1} = \sum_{j = 0}^{\StagesNumber - 1} z^jA^j$.
\par Il metodo ha funzione di stabilit\`a
\begin{align*}
	\StabilityFunction(z) &= 1 + z \Transposed{b} \left ( \sum_{j = 0}^{\StagesNumber - 1} z^jA^j \right ) e,\\
	&= 1 + \left ( \sum_{j = 0}^{\StagesNumber - 1} z^{j + 1} \Transposed{b} A^j e \right ),\\
	&= \sum_{j = 0}^\ConsistencyOrder \frac{z^\ConsistencyOrder}{\ConsistencyOrder !} + \sum_{j = \ConsistencyOrder + 1}^\StagesNumber z^j \Transposed{b} A^{j - 1} e.\text{ \EndProof}
\end{align*}
\begin{Corollary}
	Tutti i metodi di Runge-Kutta espliciti ottimali di ordine di consistenza $\ConsistencyOrder$ hanno la stessa funzione di stabilit\`a.
\end{Corollary}
\Proof Qualsiasi metodo di Runge-Kutta esplicito ottimale di ordine di consistenza $\ConsistencyOrder$ ha funzione di stabilit\`a $\StabilityFunction(z) = \sum_{j = 0}^\ConsistencyOrder \frac{z^\ConsistencyOrder}{\ConsistencyOrder !}$. \EndProof
\begin{figure}
	\includegraphics[width=0.8\textwidth]{Metodi_numerici_per_equazioni_differenziali_ordinarie/Immagine_RegioneDiStabilitaMetodiEsplicitiOttimali.png}
	\centering
	\caption{Regione di stabilit\`a dei metodi di Runge-Kutta espliciti ottimali.}
\end{figure}
\begin{Corollary}
	Sia un metodo di Runge-Kutta esplicito a $\StagesNumber > 0$ stadi e consistente di ordine $\ConsistencyOrder$ definito dal \tableau\ di Butcher
	\[
	\begin{array}{c|c}
		c	&	A\\
		\hline
			&	b
	\end{array}.
	\]
	Il metodo non \`e
	\begin{itemize}
		\item $A$-stabile;
		\item $A(\alpha)$-stabile ($\alpha \in (-\pi,\pi]$);
		\item $L$-stabile.
	\end{itemize}
\end{Corollary}
\Proof Per la non banalit\`a del metodo di Runge-Kutta, deve essere $\ConsistencyOrder > 0$, quindi $\StabilityFunction$ \`e un polinomio di grado almento $1$. Pertanto $\lim_{\AbsoluteValue{z} \rightarrow + \infty} \AbsoluteValue{\StabilityFunction(z)} = + \infty$.
\par Ora,
\begin{itemize}
	\item il metodo non \`e $A$-stabile o $A(\alpha)$-stabile ($\alpha \in (-\pi,\pi]$)  perch\'e la regione di stabilit\`a \`e limitata;
	\item il metodo non \`e $L$-stabile perch\'e $\lim_{z \rightarrow - \infty} \AbsoluteValue{\StabilityFunction(z)} \neq 0$. \EndProof
\end{itemize}
\begin{Definition}
\label{MetodiNumericiPerEquazioniDifferenzialiOrdinarie_Definizione_MetodoDiRungeKutta_Embedded}
	Siano dati un problema di Cauchy su un intervallo $I$
	\[
	\begin{cases}
		y' = f(t,y),\\
		y(t_0) = y_0,
	\end{cases}
	\]
	e una discretizzazione uniforme di $I$ di passo $\Step \in \RealPositive$ ed $N \in \NaturalExtended$ nodi. Siano inoltre due metodi Runge-Kutta di ordine $p$ e $q < p$ (tipicamente $q = p - 1$) definiti rispettivamente dai \tableau
	\[
		\begin{array}{c|c}
		c	&	A\\
		\hline
			&	b
		\end{array},
		\begin{array}{c|c}
		c	&	A,\\
		\hline
			&	b'
		\end{array},
	\]
	che riscriviamo pi\`u compattamente
	\[
		\begin{array}{c|c}
		c	&	A\\
		\hline
			&	b\\
		\hline
			&	b'
		\end{array},
	\]
	dove
	\begin{itemize}
		\item $s \in \mathbb{N}$;
		\item $A = (a_{i,j})_{(i,j) \in \NaturalShift{s} \times \NaturalShift{s}} \in \mathbb{R}^{\NaturalShift{s} \times \NaturalShift{s}}$;
		\item $b, b', c \in \mathbb{R}^s$.
	\end{itemize}
	Chiamiamo \Define{metodo di Runge-Kutta \embedded}[di Runge-Kutta \embedded][metodo] il metodo numerico a passo adattivo definito dallo schema seguente
	\begin{itemize}
		\item si fissa una tolleranza $\epsilon > 0$;
		\item si fissa un passo iniziale $\Step$;
		\item si effettuano due passi del metodo di ordine $p$ e si stima l'errore utilizzando il metodo di ordine $q$ ripercorrendo i passaggi della dimostrazione del teorema di estrapolazione di Richardson;
		\item se l'errore \`e minore della tolleranza $\epsilon$, si raddoppia il passo $\Step$ finch\'e la stima dell'errore rimane sotto la tolleranza; se invece \`e maggiore di $\epsilon$ si dimezza $\Step$ sino a quando la stima dell'errore scende sotto $\epsilon$;
		\item si applica il metodo numerico di ordine $p$ col nuovo passo $\Step$, ripetendo ogni tanto la stima dell'errore e aggiustando il passo $\Step$ secondo le istruzioni del punto precedente.
	\end{itemize}
\end{Definition}
\subsubsection{Metodo di Eulero esplicito.}
\label{MetodiNumericiPerEquazioniDifferenzialiOrdinarie_MetodoDiEuleroEsplicito}
\begin{Definition}
	Dati un problema di Cauchy su un intervallo $I$
	\[
	\begin{cases}
		y' = f(t,y),\\
		y(t_0) = y_0,
	\end{cases}
	\]
	e una discretizzazione uniforme di $I$ di passo $\Step \in \RealPositive$ e $N$ nodi, chiamiamo \Define{metodo di Eulero esplicito}[di Eulero esplicito][metodo] o \Define{metodo di Eulero in avanti}[di Eulero in avanti][metodo] il metodo a un passo definito da
	\[
	\begin{cases}
		y_0 = y_0,\\
		y_{j + 1} = y_j + \Step f(t_j,y_j)\text{ per }i \in \mathbb{N}.
	\end{cases}
	\]
\end{Definition}
\begin{listing}
	\insertcode{octave}{"Metodi_numerici_per_equazioni_differenziali_ordinarie/EuleroEsplicito.m"}
	\caption{Metodo di Eulero esplicito implementato in \LanguageName{octave}.}
\end{listing}
\begin{Theorem}
	Il metodo di Eulero esplicito \`e un metodo di Runge-Kutta a $1$ stadio con \tableau
	\[
	\begin{array}{c|c}
	0	&	0\\
	\hline
		&	1
	\end{array}.
	\]
\end{Theorem}
\Proof Segue per verifica diretta. \EndProof
\begin{Theorem}
	Con le notazioni della definizione precedente, assumendo $y \in \CClass{2}$, il metodo di Eulero esplicito \`e consistente di ordine $1$.
\end{Theorem}
\Proof Abbiamo evidentemente $\LocalTruncationError_j = 0$. Inoltre, per ogni $i \in N - 1$ abbiamo $y(t_{j + 1}) - y(t_j) - \Step f(t_j,y_j) = y(t_{j + 1}) - y(t_j) - \Step y'(t_j) = \Step y'(t_j) + \frac{\Step^2}{2} y''(\xi) - \Step y'(t_j) = \BigO{\Step^2}$, dove $\xi \in (t_j,t_{j + 1})$. \EndProof
\begin{Theorem}
	Il metodo di Eulero esplicito
	\begin{itemize}
		\item ha funzione di stabilit\`a $\StabilityFunction(z) = 1 + z$;
		\item ha regione di stabilit\`a \`e $\StabilityRegion = \lbrace z | \AbsoluteValue{1 + z} < 1 \rbrace$.
	\end{itemize}
\end{Theorem}
\Proof Applichiamo il metodo di Eulero esplicito con passo $\Step$ al problema test di Dahlquist di parametro $\DahlquistParameter$: 
\[
\begin{cases}
	y_0 = y_0,\\
	y_{j + 1} = y_j + \Step \DahlquistParameter y_j\text{ per }j > 0.
\end{cases}
\]
\par Abbiamo dunque, per $j \in \mathbb{N}$, $y_j = (1 + \Step \DahlquistParameter)^j$: la successione \`e infinitesima se e solo se $\AbsoluteValue{1 + \Step \DahlquistParameter} < 1$, equivalente a $\AbsoluteValue{1 + z} < 1$. \EndProof

\subsubsection{Metodo di Eulero implicito.}
\label{MetodiNumericiPerEquazioniDifferenzialiOrdinarie_MetodoDiEuleroImplicito}
\begin{Definition}
	Dati un problema di Cauchy su un intervallo $I$
	\[
	\begin{cases}
		y' = f(t,y),\\
		y(t_0) = y_0,
	\end{cases}
	\]
	e una discretizzazione uniforme di $I$ di passo $\Step \in \RealPositive$ e $N$ nodi, chiamiamo \Define{metodo di Eulero implicito}[di Eulero implicito][metodo] o \Define{metodo di Eulero all'indietro}[di Eulero all'indietro][metodo] il metodo a un passo definito da
	\[
	\begin{cases}
		y_0 = y_0,\\
		y_{j + 1} = y_j + \Step \phi(t_j,y_j)\text{ per }i \in \mathbb{N},
	\end{cases}
	\]
	con
	\begin{itemize}
		\item $y_{j + 1} = \xi$ tale che $y_j + \Step f(t_{j + 1},\xi) = \xi$;
		\item $\phi(t_j,y_j) = \eta$ tale che $y_j + \Step \eta = y_{j + 1}$.
	\end{itemize}
\end{Definition}
\begin{listing}
	\insertcode{octave}{"Metodi_numerici_per_equazioni_differenziali_ordinarie/EuleroImplicito.m"}
	\caption{Metodo di Eulero implicito implementato in \LanguageName{octave}.}
\end{listing}
\begin{Theorem}
	Con le notazioni del teorema precedente, abbiamo
	\[
	\begin{cases}
		y_0 = y_0,\\
		y_{j + 1} = y_j + \Step f(t_{j + 1},y_{j + 1})\text{ per }i > 0.
	\end{cases}
	\]
\end{Theorem}
\Proof Il metodo descritto nell'enunciato si ottiene dalla definizione semplicemente sostituendo $y_{j + 1}$ a $\xi$. \EndProof
\begin{Theorem}
	Il metodo di Eulero implicito \`e un metodo di Runge-Kutta a $1$ stadio con \tableau
	\[
	\begin{array}{c|c}
	1	&	1\\
	\hline
		&	1
	\end{array}.
	\]
\end{Theorem}
\Proof Segue per verifica diretta. \EndProof
\begin{Theorem}
	Con le notazioni della definizione precedente, assumendo $y \in \CClass{2}$, il metodo di Eulero implicito \`e consistente di ordine $1$.
\end{Theorem}
\Proof Abbiamo evidentemente $\LocalTruncationError_j = 0$. Inoltre, per ogni $i \in N - 1$ abbiamo $y(t_{j + 1}) - y(t_j) - \Step f(t_{j + 1},y_{j + 1}) = y(t_{j + 1}) - y(t_j) - \Step y'(t_{j + 1}) = \Step y'(t_{j + 1}) + \frac{\Step^2}{2} y''(\xi) - \Step y'(t_{j + 1}) = \BigO{\Step^2}$, dove $\xi \in (t_j,t_{j + 1})$. \EndProof
\begin{Theorem}
	Il metodo di Eulero implicito
	\begin{itemize}
		\item ha funzione di stabilit\`a $\StabilityFunction(z) = (1 - z)^{-1}$;
		\item ha regione di stabilit\`a \`e $\StabilityRegion = \lbrace z | \AbsoluteValue{1 + z} > 1 \rbrace$ (e dunque \`e $A$-stabile).
	\end{itemize}
\end{Theorem}
\Proof Applichiamo il metodo di Eulero esplicito con passo $\Step$ al problema test di Dahlquist di parametro $\DahlquistParameter$: 
\[
\begin{cases}
	y_0 = y_0,\\
	y_{j + 1} = y_j + \Step \DahlquistParameter y_{j + 1}\text{ per }j > 0.
\end{cases}
\]
\par Abbiamo dunque, per $j \in \mathbb{N}$, $y_{j + 1} = (1 - \Step \DahlquistParameter)^{-1}y_j$ e dunque $y_j = (1 - \Step \DahlquistParameter)^{-j}$: la successione \`e infinitesima se e solo se $\AbsoluteValue{1 - \Step \DahlquistParameter} > 1$, equivalente a $\AbsoluteValue{1 - z} > 1$. \EndProof

\subsubsection{Metodo dei trapezi.}
\label{MetodiNumericiPerEquazioniDifferenzialiOrdinarie_MetodoDeiTrapezi}
\begin{Definition}
	Dati un problema di Cauchy su un intervallo $I$
	\[
	\begin{cases}
		y' = f(t,y),\\
		y(t_0) = y_0,
	\end{cases}
	\]
	e una discretizzazione uniforme di $I$ di passo $\Step \in \RealPositive$ e $N$ nodi, chiamiamo \Define{metodo dei trapezi}[dei trapezi][metodo] o \Define{metodo di Crank-Nicolson}[di Crank-Nicolson][metodo] il metodo a un passo definito da
	\[
	\begin{cases}
		y_0 = y_0,\\
		y_{j + 1} = y_j + \frac{\Step (f(t_j,y_j) + f(t_{j + 1},y_{j + 1}))}{2}\text{ per }i \in \mathbb{N}.
	\end{cases}
	\]
\end{Definition}
\begin{listing}
	\insertcode{octave}{"Metodi_numerici_per_equazioni_differenziali_ordinarie/Trapezi.m"}
	\caption{Metodo dei trapezi implementato in \LanguageName{octave}.}
\end{listing}
\begin{Theorem}
	Il metodo dei trapezi \`e un metodo di Runge-Kutta a $2$ stadi con \tableau
	\[
	\begin{array}{c|cc}
	0	&	0	&	0\\
	1	&	\frac{1}{2}	&	\frac{1}{2}\\
	\hline
		&	\frac{1}{2}	&	\frac{1}{2}
	\end{array}.
	\]
\end{Theorem}
\Proof Segue per verifica diretta. \EndProof
\begin{Theorem}
	Con le notazioni della definizione precedente, se $y \in \CClass{3}$, il metodo dei trapezi \`e consistente di ordine $2$.
\end{Theorem}
\Proof Utilizziamo le notazioni della definizione precedente. Abbiamo evidentemente $\LocalTruncationError_j = 0$. Inoltre, per ogni $i \in N - 1$, abbiamo $y(t_{j + 1}) - y(t_j) - \frac{\Step (f(t_j,y_j) + f(t_{j + 1}, y_{j + 1})}{2} = y(t_{j + 1}) - y(t_j) - \frac{\Step (y'(t_j) + y'(t_{j + 1}))}{2}$.
\par Sviluppiamo la serie di Taylor intorno a $\hat{t} = \frac{t_j + t_{j + 1}}{2}$. Abbiamo
\begin{itemize}
	\item $y(t_{j + 1}) = y(\hat{t}) + \frac{\Step}{2} y'(\hat{t}) + \frac{\Step^2}{8}y''(\hat{t}) + \frac{h^3}{48}y'''(\xi_{j + 1})$ per qualche $\xi_{j + 1} \in (\hat{t},t_{j + 1})$;
	\item $y(t_j) = y(\hat{t}) - \frac{\Step}{2} y'(\hat{t}) + \frac{\Step^2}{8}y''(\hat{t}) - \frac{h^3}{48}y'''(\xi_j)$ per qualche $\xi_j \in (t_j,\hat{t})$;
	\item $y'(t_{j + 1}) = y'(\hat{t}) + \frac{\Step}{2} y''(\hat{t}) + \frac{\Step^2}{8}y'''(\eta_{j + 1})$ per qualche $\eta_{j + 1} \in (\hat{t},t_{j + 1})$;
	\item $y'(t_j) = y'(\hat{t}) - \frac{\Step}{2} y''(\hat{t}) + \frac{\Step^2}{8}y'''(\eta_j)$ per qualche $\eta_j \in (t_j,\hat{t})$.
\end{itemize}
\par Dunque $y(t_{j + 1}) - y(t_j) - \frac{\Step (y'(t_j) + y'(t_{j + 1}))}{2} = \Step y'(\hat{t}) + \frac{h^3}{24}(y'''(\xi_{j + 1}) - y'''(\xi_j)) - \Step y'(\hat{t}) - \frac{h^3}{8}(y'''(\eta_{j + 1}) + y'''(\eta_j)) = \BigO{h^3}$. \EndProof
\begin{Theorem}
	Il metodo dei trapezi
	\begin{itemize}
		\item ha funzione di stabilit\`a $\StabilityFunction(z) = \frac{2 + z}{2 - z}$;
		\item ha regione di stabilit\`a \`e $\StabilityRegion = \lbrace z | \RealPart{z} < 0 \rbrace$ (e dunque \`e $A$-stabile).
	\end{itemize}
\end{Theorem}
\Proof Applichiamo il metodo dei trapezi con passo $\Step$ al problema test di Dahlquist di parametro $\DahlquistParameter$: 
\[
\begin{cases}
	y_0 = y_0,\\
	y_{j + 1} = y_j + \frac{\Step \DahlquistParameter (y_j +  y_{j + 1})}{2}\text{ per }j > 0.
\end{cases}
\]
\par Abbiamo dunque, per $j \in \mathbb{N}$, $(1 - \frac{\Step\DahlquistParameter}{2} y_{j + 1} = (1 + \frac{\Step \DahlquistParameter}{2})y_j$ e dunque $y_j = \left ( \frac{2 + \Step \DahlquistParameter}{2 - \Step\DahlquistParameter} \right )^j$: la successione \`e infinitesima se e solo se $\AbsoluteValue{\frac{2 + \Step \DahlquistParameter}{2 - \Step\DahlquistParameter}} < 1$, equivalente a $\RealPart{z} < 0$. \EndProof

\subsubsection{Metodo di Heun.}
\label{MetodiNumericiPerEquazioniDifferenzialiOrdinarie_MetodoDiHeun}
\begin{Definition}
	Dati un problema di Cauchy su un intervallo $I$
	\[
	\begin{cases}
		y' = f(t,y),\\
		y(t_0) = y_0,
	\end{cases}
	\]
	e una discretizzazione uniforme di $I$ di passo $\Step \in \RealPositive$ e $N$ nodi, chiamiamo \Define{metodo di Heun}[di Heun][metodo] il metodo a un passo definito da
	\[
	\begin{cases}
		y_0 = y_0,\\
		y_{j + 1} = y_j + \frac{\Step (f(t_j,y_j) + f(t_{j + 1},y_j + \Step f(t_j,y_j)))}{2}\text{ per }i \in \mathbb{N}.
	\end{cases}
	\]
\end{Definition}
\begin{Theorem}
	Il metodo di Heun \`e un metodo di Runge-Kutta a $2$ stadi con \tableau
	\[
	\begin{array}{c|cc}
	0	&	0	&	0\\
	1	&	1	&	0\\
	\hline
		&	\frac{1}{2}	&	\frac{1}{2}
	\end{array}.
	\]
\end{Theorem}
\Proof Segue per verifica diretta. \EndProof

