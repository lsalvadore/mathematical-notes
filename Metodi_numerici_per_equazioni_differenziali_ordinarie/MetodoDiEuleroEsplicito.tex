\subsubsection{Metodo di Eulero esplicito.}
\label{MetodiNumericiPerEquazioniDifferenzialiOrdinarie_MetodoDiEuleroEsplicito}
\begin{Definition}
	Dati un problema di Cauchy su un intervallo $I$
	\[
	\begin{cases}
		y' = f(t,y),\\
		y(t_0) = y_0,
	\end{cases}
	\]
	e una discretizzazione uniforme di $I$ di passo $\Step \in \RealPositive$ e $N$ nodi, chiamiamo \Define{metodo di Eulero esplicito}[di Eulero esplicito][metodo] o \Define{metodo di Eulero in avanti}[di Eulero in avanti][metodo] il metodo a un passo definito da
	\[
	\begin{cases}
		y_0 = y_0,\\
		y_{j + 1} = y_j + \Step f(t_j,y_j)\text{ per }i \in \mathbb{N}.
	\end{cases}
	\]
\end{Definition}
\begin{listing}
	\insertcode{octave}{"Metodi_numerici_per_equazioni_differenziali_ordinarie/EuleroEsplicito.m"}
	\caption{Metodo di Eulero esplicito implementato in \LanguageName{octave}.}
\end{listing}
\begin{Theorem}
	Il metodo di Eulero esplicito \`e un metodo di Runge-Kutta a $1$ stadio con \tableau
	\[
	\begin{array}{c|c}
	0	&	0\\
	\hline
		&	1
	\end{array}.
	\]
\end{Theorem}
\Proof Segue per verifica diretta. \EndProof
\begin{Theorem}
	Con le notazioni della definizione precedente, assumendo $y \in \CClass{2}$, il metodo di Eulero esplicito \`e consistente di ordine $1$.
\end{Theorem}
\Proof Abbiamo evidentemente $\LocalTruncationError_j = 0$. Inoltre, per ogni $i \in N - 1$ abbiamo $y(t_{j + 1}) - y(t_j) - \Step f(t_j,y_j) = y(t_{j + 1}) - y(t_j) - \Step y'(t_j) = \Step y'(t_j) + \frac{\Step^2}{2} y''(\xi) - \Step y'(t_j) = \BigO{\Step^2}$, dove $\xi \in (t_j,t_{j + 1})$. \EndProof
\begin{Theorem}
	Il metodo di Eulero esplicito
	\begin{itemize}
		\item ha funzione di stabilit\`a $\StabilityFunction(z) = 1 + z$;
		\item ha regione di stabilit\`a \`e $\StabilityRegion = \lbrace z | \AbsoluteValue{1 + z} < 1 \rbrace$.
	\end{itemize}
\end{Theorem}
\Proof Applichiamo il metodo di Eulero esplicito con passo $\Step$ al problema test di Dahlquist di parametro $\DahlquistParameter$: 
\[
\begin{cases}
	y_0 = y_0,\\
	y_{j + 1} = y_j + \Step \DahlquistParameter y_j\text{ per }j > 0.
\end{cases}
\]
\par Abbiamo dunque, per $j \in \mathbb{N}$, $y_j = (1 + \Step \DahlquistParameter)^j$: la successione \`e infinitesima se e solo se $\AbsoluteValue{1 + \Step \DahlquistParameter} < 1$, equivalente a $\AbsoluteValue{1 + z} < 1$. \EndProof
