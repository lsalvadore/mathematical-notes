\section{Definizioni e teoremi di base.}\label{DefinizioniETeoremiDiBase}
\begin{Definition}
	Sia $\Class$ una classe. Si definisce \Define{relazione $n$-aria}[$n$-aria][relazione] ($n \in \mathbb{N}$) su $\Class$ una qualsiasi parte $\Relation$ di $\Class^n$. Il numero $n$ is chiama \Define{ariet\`a}[di una relazione][ariet\`a] della relazione $\Relation$; la classe $\Class$ si chiama \Define{sostegno}[di una relazione][sostegno] della reazione $\Relation$. Se $(x_0, ..., x_n) \in \Relation$ si dice che $x_0, ..., x_n$ \NIDefine{verificano} $\Relation$, affermazione che denotiamo anche $\Relation[x_0,...,x_n]$; nel caso $\Relation$ sia una relazione binaria denotiamo questa affermazione anche $x_0 \Relation x_1$.
\end{Definition}
\begin{Definition}
	Sia $\Relation$ una relazione binaria su $\Class$. Definiamo \Define{relazione opposta}[opposta][relazione] di $\Relation$ la relazione $\Dual{\Relation}$ definita da $(x,y) \in \Dual{\Relation}$ se e solo se $(y,x) \in \Relation$. Per denotare le relazioni opposte si usa spesso lo stesso simbolo della relazione originale, ribaltato orizzontalmente.
\end{Definition}
\begin{Definition}
	Sia $\Relation$ una relazione binaria su una classe $\Class$. $\Relation$ si dice
	\begin{itemize}
		\item \Define{riflessiva}[riflessiva][relazione] quando $\ForAll{x \in \Class}{x \Relation x}$;
		\item \Define{simmetrica}[simmetrica][relazione] quando $\Implies{x \Relation y}{y \Relation x}$;
		\item \Define{antisimmetrica}[antisimmetrica][relazione] quando $\Implies{\And{x \Relation y}{y \Relation x}}{x = y}$;
		\item \Define{transitiva}[transitiva][relazione] quando $\Implies{\And{x \Relation y}{y \Relation z}}{x \Relation y}$.
	\end{itemize}
	Se \`e definita una seconda relazione $\Dual{\Relation}$ sulla stessa classe $\Class$
\end{Definition}
