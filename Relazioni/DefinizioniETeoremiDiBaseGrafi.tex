\subsection{Definizioni e teoremi di base.}
\label{DefinizioniETeoremiDiBaseGrafi}
\begin{Definition}
	Un \Define{grafo} \`e una coppia
  $\Graph = (\Nodes,\Edges)$ dove $\Nodes$ \`e
  una classe i cui elementi di chiamano
  \Define{nodi}[di un grafo][nodo],
  e $\Edges$ \`e una classe i cui elementi si dicono
  \Define{archi}[di un grafo][arco] tali che
	\begin{itemize}
		\item o $\Edges \subseteq \Nodes^2$ \`e una relazione, e allora il
      grafo si dice \Define{orientato}[orientato][grafo] o
     \Define{diretto}[diretto][grafo];
		\item o $\Edges$ \`e una classe i cui elementi sono tutti parti di
      $\Nodes$ di $2$ elementi, allora il grafo si
      dice \Define{non orientato}[non orientato][grafo] o
      \Define{non diretto}[non diretto][grafo]
	\end{itemize}
  Se $\Nodes$ \`e un insieme finito, allora, $\Graph$ si dice
  \Define{finito}[finito][grafo].
\end{Definition}
\begin{Definition}
  Dato un grafo non orientato $\Graph = (\Nodes,\Edges)$, chiamiamo
  \Define{grafo orientato associato}[orientato associato][grafo]
  a $\Graph$ il grafo $\Graph' = (\Nodes,\Edges')$ tale che
  $\lbrace \Node, \VarNode \rbrace \in \Edges$ se e solo se
  $\And{(\Node,\VarNode) \in \Edges'}{(\VarNode,\Node) \in \Edges'}$.
\end{Definition}
\begin{Definition}
	Sia $\Graph = (\Nodes,\Edges)$ un grafo. Se
  $\Node, \VarNode \in \Nodes$ sono nodi e $(\Node,\VarNode) \in \Edges$
  allora si dice che $\VarNode$ \`e \Define{collegato}[collegati][nodi]
  o \Define{connesso}[connessi][nodi] a $\Node$. Ora,
	\begin{itemize}
		\item se $\Graph$ \`e non orientato,
      \begin{itemize}
        \item chiamiamo \Define{vicinato} la
          parte di $\Nodes$ costituita dai nodi connessi a $\Node$, denotata
          $\GraphNeighborhood{\Node}$;
        \item \Define{vicino}[vicino][nodo] di $\Node$ un elemento di
          $\GraphNeighborhood{\Node}$;
        \item \Define{grado}[di un nodo in un grafo non orientato][grado] la
          cardinalit\`a di $\GraphNeighborhood{\Node}$, denotata
          $\GraphNodeDegree{\Node}$;
      \end{itemize}
		\item se $\Graph$ \`e orientato, chiamiamo
		  \begin{itemize}
			  \item \Define{vicinato uscente}[uscente][vicinato] la parte di
          $\Nodes$ costituita dai nodi connessi a $\Nodes$, denotato
        \item \Define{vicino uscente}[vicino uscente][nodo] di $\Node$ o
          \Define{figlio}[figlio][nodo] un elemento di
          $\GraphNeighborhoodOut{\Node}$;
        \item \Define{grado uscente}[di un nodo in un grafo orientato][grado]
          la cardinalit\`a di $\GraphNeighborhoodOut{\Node}$, denotata
          $\GraphNodeDegreeOut{\Node}$;
	  		\item \Define{vicinato entrante}[entrante][vicinato]
          $\GraphNeighborhoodIn{\Node}$ la parte di $\Nodes$ costituita
          da tutti i nodi a cui $\Node$ \`e connesso;
        \item \Define{vicino entrante}[vicino entrante][nodo] di $\Node$ o
          \Define{genitore}[genitore][nodo] un elemento di
          $\GraphNeighborhoodIn{\Node}$;
        \item \Define{grado entrante}[di un nodo in un grafo orientato][grado]
          la cardinalit\`a di $\GraphNeighborhoodIn{\Node}$, denotata
          $\GraphNodeDegreeOut{\Node}$.
		  \end{itemize}
	\end{itemize}
  La metafora genitore-figlio viene spesso estesa ad altri gradi di parentela
  con evidente significato.
\end{Definition}
\begin{Definition}
  Un nodo avente grado uscente nullo si chiama
  \Define{nodo pendente}[pendente][nodo]
  (\Define{\English{dangling node}}[\English{dangling}][\English{node}]).
\end{Definition}
\begin{Lemma}
 \TheoremName{Lemma della stretta di mano}[della stretta di mano][lemma]
  \TheoremName{Handshaking lemma}[handshaking][lemma]
  Sia $\Graph = (\Nodes,\Edges)$ un grafo finito:
  \begin{itemize}
    \item se $\Graph$ \`e orientato, allora
      $\sum_{\Node \in \Nodes} \GraphNodeDegreeOut{\Node}
      = \sum_{\Node \in \Nodes} \GraphNodeDegreeIn{\Node}
      = \Cardinality{\Edges}$;
    \item se $\Graph$ non \`e orientato, allora
      \begin{itemize}
        \item $\sum_{\Node \in \Nodes} \GraphNodeDegree{\Node}
          = 2\Cardinality{\Edges}$;
        \item la parte $\Nodes' \subseteq \Nodes$ di tutti i nodi
          di grado dispari ha cardinalit\`a pari.
      \end{itemize}
  \end{itemize}
\end{Lemma}
\Proof Supponiamo $\Graph$ orientato. Siano
$\Node, \VarNode \in \Nodes$ tali che
$(\Node,\VarNode) \in \Edges$:
$(\Node, \VarNode)$ \`e un arco entrante rispetto a
$\Node$ e uscente rispetto a $\VarNode$. Pertanto
$\sum_{\Node \in \Nodes} \GraphNodeDegreeOut{\Node}
= \sum_{\Node \in \Nodes} \GraphNodeDegreeIn{\Node}
= \Cardinality{\Edges}$.
\par Supponiamo ora $\Graph$ non orientato.
Considerando il grafo orientato associato a $\Graph$, abbiamo
$\sum_{\Node \in \Nodes} \GraphNodeDegree{\Node}
= 2 \Cardinality{\Edges}$.
\par Per l'ultima parte del teorema, basta osservare
che
$\sum_{\Node \in \Nodes} \GraphNodeDegree{\Node}
= \sum_{\Node \in \Nodes'} \GraphNodeDegree{\Node}
  + \sum_{\Node \in \Nodes \SetMin \Nodes '} \GraphNodeDegree{\Node}$,
\`e una quantit\`a pari, ma anche il termine
$\sum_{\Node \in \Nodes \SetMin \Nodes '} \GraphNodeDegree{\Node}$
\`e pari. \EndProof
\begin{Definition}
	Sia $\Graph = (\Nodes,\Edges)$ un grafo. Sia $(\Node_i)_{i \in n}$,
  con $n \in \NaturalExtended$ una successione di nodi tali che
  $\ForAll{i \in \NotZero{n}}
    {\lbrace \Node_{i - 1},\Node_i \rbrace \in \Edges}$.
  La successione di archi
  $(\lbrace \Node_{i - 1}, \Node_i \rbrace)_{i \in n}$
  si chiama
  \Define{cammino}[in un grafo][cammino]. Si dice che un cammino
  \Define{visita}[visitato][nodo] un nodo se il nodo appartiene ad uno
  degli archi del cammino. Inoltre, un cammino si dice
	\begin{itemize}
		\item \Define{semplice}[semplice][cammino] se tutti i suoi nodi sono
      distinti;
		\item un \Define{ciclo} se $n$ \`e finito e
      $\Node_0 = \Node_{n - 1}$.
	\end{itemize}
\end{Definition}
\begin{Definition}
	Un grafo $\Graph = (\Nodes,\Edges)$ si dice
  \Define{connesso}[connesso][grafo] quando per ogni
  $(\Node,\VarNode) \in \Nodes^2$ esiste un cammino di $\Graph$ avente
  $\Node$ come nodo iniziale e $\VarNode$ come nodo finale.
\end{Definition}
\begin{Definition}
	Dato un grafo $\Graph = (\Nodes,\Edges)$, i sottografi connessi
  massimali di $\Graph$ si chiamano
  \Define{componenti connesse}[connessa in un grafo][componente].
\end{Definition}
\begin{Definition}
	Un grafo si dice
	\begin{itemize}
		\item \Define{ciclico}[ciclico][grafo] se contiene un ciclo;
		\item \Define{aciclico}[aciclico][grafo] se non contiene nessun
      ciclo.
	\end{itemize}
\end{Definition}
\begin{Definition}
	Un cammino o un ciclo si dice
  \begin{itemize}
    \item \Define{cammino hamiltoniano}[hamiltoniano][cammino] o
      \Define{ciclo hamiltoniano}[hamiltoniano][ciclo]
      rispettivamente, se visita tutti i nodi di $\Graph$ una e sua sola
      volta;
    \item \Define{cammino euleriano}[euleriano][cammino] o
      \Define{ciclo euleriano}[euleriano][ciclo]
      rispettivamente, se comprende tutti gli archi di $\Graph$ una e
      sua sola volta.
  \end{itemize}
\end{Definition}
