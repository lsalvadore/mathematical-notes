\subsection{Alberi.}
\label{Relazioni_Alberi}
\begin{Definition}
	Chiamiamo \Define{albero} un grafo $\Graph = (\Nodes, \Edges)$,
	\begin{itemize}
		\item connesso;
		\item aciclico.
	\end{itemize}
  Se $\Graph$ \`e
  \begin{itemize}
    \item non orientato, allora l'albero si dice
      \Define{non orientato}[non orientato][albero];
    \item orientato, allora l'albero si dice
      \Define{orientato}[orientato][albero].
  \end{itemize}
  Qualora non si specifichi se l'albero orientato, esso si intende non
orientato.
\end{Definition}
\begin{Definition}
	Se $\Graph$ \`e un grafo le cui componenti connesse sono tutti alberi,
  allora $\Graph$ si dice una \Define{foresta}.
\end{Definition}
\begin{Definition}
	Consideriamo un albero $\Tree$. Definiamo
  \Define{dimensione}[di un albero][dimensione]
  di $\Tree$ il numero di nodi di cui esso \`e composto.
\end{Definition}
\begin{Definition}
  Un \Define{albero radicato}[radicato][albero] \`e una coppia
  $\RootedTree = (\Graph,\Root)$, dove
  \begin{itemize}
    \item $\Graph = (\Nodes, \Edges)$ \`e un albero;
    \item $\Root \in \Nodes$ \`e un nodo, detto \Define{radice}.
  \end{itemize}
  Inoltre,
  \begin{itemize}
    \item se l'albero \`e orientato in modo tale che la radice sia antenata
      di tutti gli altri nodi, l'albero si chiama \Define{arborescenza};
    \item se l'albero \`e orientato in modo tale che la radice sia discendente
      di tutti gli altri nodi, l'albero si chiama \Define{antiarborescenza}.
  \end{itemize}
\end{Definition}
\begin{Definition}
  Sia $\Arborescence$ un'arborescenza. Definiamo
  \begin{itemize}
    \item \Define{foglia} un nodo di $\Arborescence$ senza figli;
		\item \Define{altezza}[di un nodo][altezza] di un nodo il massimo numero di
      nodi da percorrere per passare da una foglia esclusa al nodo in questione
      incluso;
		\item \Define{profondit\`a}[di un nodo][profondit\`a] o
      \Define{livello}[di un nodo][livello] di un nodo il numero di
      nodi da percorrere per passare dalla radice esclusa al nodo in questione
      incluso;
    \item \Define{profondit\`a}[di un arborescenza][profondit\`a] di
      $\Arborescence$ il massimo delle profondit\`a dei suoi nodi;
    \item per ogni $n \in \mathbb{N}$,
      \Define{livello $n$-esimo}[di un'arborescenza][livello] di $\Arborescence$
      l'insieme di tutti i suoi nodi di livello $n$.
	\end{itemize}
\end{Definition}
\begin{Definition}
  Sia $n \in \NotZero{\mathbb{N}}$. Chiamiamo \Define{albero $n$-ario}
  un'arborescenza in cui ogni nodo ha al pi\`u $n$ figli.
\end{Definition}
\begin{Definition}
  Sia $\BinaryTree$ un albero binari. Gli eventuali figli di un nodo si
  distinguono arbitrariamente in
  \Define{figlio sinistro}[sinistro][figlio]
  e
  \Define{figlio destro}[destro][figlio]; anche quando un nodo ha un solo
  figlio, si usa indicarlo come sinistro o destro.
  \par Siano inoltre $\Node$ e $\VarNode$ due nodi di $\BinaryTree$ di stesso
  livello. Diciamo che $\Node$ \`e
  \Define{pi\`u a sinistra}[pi\`u a sinistra][nodo]
  di $\Node$
  o che $\VarNode$ \`e
  \Define{pi\`u a destra}[pi\`u a destra][nodo]
  di $\Node$ quando si verifica una delle due condizioni seguenti:
  \begin{itemize}
    \item $\Node$ e $\VarNode$ sono figlio sinistro e destro rispettivamente di
      uno stesso nodo;
    \item $\Node$ ha un antenato pi\`u a sinistra rispetto all'antenato di
      $\VarNode$ di stesso livello.
  \end{itemize}
\end{Definition}
\begin{Definition}
  Sia $\Arborescence$ un albero $n$-ario ($n \in \NotZero{\mathbb{N}}$).
  $\Arborescence$ si dice
  \begin{itemize}
    \item \Define{quasi completo}[$n$-ario quasi completo][albero]
      quando
      \begin{itemize}
        \item tutti i nodi con $0$ figli hanno lo stesso livello
          $L \in \mathbb{N}$;
        \item tutti i nodi di livello minore di $L$ hanno esattamente $n$ figli;
      \end{itemize}
    \item \Define{completo}[$n$-ario completo][albero]
      quando
      \begin{itemize}
        \item \`e quasi completo;
        \item denotata $L$ la sua profondit\`a, tutti i nodi di livello $L$
          hanno $0$ figli;
      \end{itemize}
    \item \Define{bilanciato}[bilanciato][albero]
      quando per ogni nodo di $\Arborescence$, i livelli dei sottoalberi di quel
      nodo differiscono al pi\`u di $1$.
  \end{itemize}
\end{Definition}
\begin{Definition}
  Sia $\BinaryTree$ un albero binario quasi completo. $\BinaryTree$ si dice
  \begin{itemize}
    \item \Define{completo da sinistra}[$n$-ario completo da sinistra][albero]
      quando denotata $L$ la sua profondit\`a, se $\Node$ \`e un nodo di
      livello $L$, allora tutti i nodi pi\`u a sinistra del genitore
      di $\Node$ hanno esattamente due figli;
    \item \Define{completo da destra}[$n$-ario completo da destra][albero]
      quando denotata $L$ la sua profondit\`a, se $\Node$ \`e un nodo di
      livello $L$, allora tutti i nodi pi\`u a destra del genitore
      di $\Node$ hanno esattamente due figli.
  \end{itemize}
\end{Definition}
