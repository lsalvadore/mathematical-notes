\subsection{Rappresentazione di Grafi.}
\label{Relazioni_RappresentazioneDiGrafi}
\begin{Definition}
  Sia $n \in \mathbb{N}$ e $\Graph$ un grafo finito orientato
  $\Graph = (n,\Edges)$ di $n$ nodi. Chiamiamo
  \Define{matrice di adiacenza}[di adiacenza][matrice]
  la famiglia
  $\AdjacencyMatrix
    = (\AdjacencyMatrix_i^j)_{(i,j) \in n \times n}$, dove
  $\AdjacencyMatrix_i^j =
    \begin{cases}
      0\text{ se }(i,j) \in \Edges,\\
      1\text{ se }(i,j) \in \Edges.
    \end{cases}$
  La matrice di adiacenza di un grafo finito non orientato si
  definisce analogamente per mezzo del grafo orientato associato.
\end{Definition}
\begin{Theorem}
  Sia $n \in \mathbb{N}$ e $\Graph$ un grafo finito orientato
  $\Graph = (n,\Edges)$ di $n$ nodi. La matrice di adiacenza
  contiene esattamente $2 \Cardinality{\Edges}$.
\end{Theorem}
\Proof Segue direttamente dal lemma della stretta di mano. \EndProof
\begin{Definition}
  Sia $n \in \mathbb{N}$ e $\Graph$ un grafo finito orientato
  $\Graph = (n,\Edges)$ di $n$ nodi.
  Chiamiamo \Define{lista di adiacenza}[di adiacienza][lista] un
  \Aarray\ di liste tale che l'$i$-esima ($i \in n$) lista dell'\Aarray\
  sia una lista di tutti i vicini (vicini uscenti nel caso orientato)
  del nodo $i$.
\end{Definition}
