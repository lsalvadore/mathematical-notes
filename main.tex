\documentclass{book}
\usepackage{it}
\title{Appunti di matematica}
\author{Lorenzo Salvadore}
\date{}
\makeindex
\begin{document}
	\maketitle
	\tableofcontents
	\part{Fondamenti}
	\chapter{Fondamenti della matematica}
	\section{Teoria degli insiemi.}\label{Insiemi}
\subsection{Teoria di Morse e Kelley (MK).}\label{MK}
\par Descriviamo la teoria degli insiemi di Morse e Kelley come una teoria del primo ordine con eguaglianza e un simbolo di relazione binaria $\in$ che verifica gli assiomi elencati di seguito. Salvo specificazioni contrarie, questa teoria sar\`a la nostra teoria degli insiemi di riferimento.
\begin{Definition}
	Ogni termine della teoria si chiama \Define{classe}. Quando vale la relazione $\Class_1 \in \Class_2$ diciamo che la classe $\Class_1$ \Define{appartiene}[appartenenza] alla classe $\Class_2$, che $\Class_1$ \`e \Define{elemento} di $\Class_2$ e che $\Class_2$ \Define{contiene}[contenimento] $\Class_1$. Denotiamo la relazione $\Not{\Class_1 \in \Class_2}$ col simbolo $\notin$. Se $\Set$ \`e una classe tale che sia verificata la relazione $(\exists \Class)(\Set \in \Class)$, allora $\Set$ \`e un \Define{insieme}, in caso contrario \`e una \Define{classe propria}[propria][classe].
\end{Definition}
\begin{Definition}
	Quando $(\Class_1 \in \Class_2) \Rightarrow (\Class_1 \in \Class_3)$ diciamo che $\Class_2$ \`e una \Define{sottoclasse} di $\Class_3$, denotato $\Class_2 \subseteq \Class_3$ e che $\Class_3$ \`e una \Define{sopraclasse} di $\Class_2$, denotato $\Class_3 \supseteq \Class_2$; se $\Class_2 \neq \Class_3$, allora le notazioni diventano rispettivamente $\Class_2 \subset \Class_3$ e $\Class_3 \supset \Class_2$. Inoltre
	\begin{itemize}
		\item se $\Class_2$ \`e un insieme, allora diciamo anche che $\Class_2$ \`e un \Define{sottoinsieme} di $\Class_3$;
		\item se $\Class_3$ \`e un insieme, allora diciamo anche che $\Class_3$ \`e un \Define{soprainsieme} di $\Class_2$.
	\end{itemize}
\end{Definition}
\begin{Axiom}
	\TheoremName{Assioma di estensionalit\`a}[di estensionalit\`a][assioma] Abbiamo $\Class_1 = \Class_2$ se e solo se $\Class_1 \subseteq \Class_2$ e $\Class_2 \subseteq \Class_1$.
\end{Axiom}
\par Per l'assioma di estensionalit\`a, dato un numero finito di classi, se esiste una classe che le contiene tutte senza contenere nient'altro, allora essa \`e unica: la denoteremo elencandone gli elementi tra parentesi graffe.
\begin{Axiom}
	\TheoremName{Schema di comprensione delle classi}[di comprensione delle classi][schema] Se $\phi(x)$ \`e una formula del primo ordine, allora esiste una classe $\Class$ tale che $\Set \in \Class \Leftrightarrow \And{(\exists \Class)(\Set \in \Class)}{\phi(\Set)}$.
\end{Axiom}
\par Data una formula $\phi(x)$ della forma descritta nell'assioma precedente, denotiamo la classe definita da $\phi(x)$ con $\lbrace x | \phi(x) \rbrace$.
\begin{Definition}
	Se la classe $\Class$ verifica la relazione $(\forall x)(x \notin \Class)$ allora diciamo che $\Class$ \`e \Define{vuota}[vuota][classe].
\end{Definition}
\begin{Theorem}
	La classe $\emptyset = \lbrace x | x \neq x \rbrace$ \`e vuota.
\end{Theorem}
\Proof Segue direttamente dallo schema di comprensione. \EndProof
\begin{Theorem}
	Sia $\Class_1$ una classe vuota e $\Class_2$ una classe qualsiasi: abbiamo $\Class_1 \subseteq \Class_2$.
\end{Theorem}
\Proof Sia $x \in \Class_1$: poich\'e, per definizione di classe vuota, $x \notin \Class_1$, ne deduciamo $x \in \Class_2$. \EndProof
\begin{Corollary}
	$\emptyset$ \`e l'unica classe vuota.
\end{Corollary}
\Proof Se $\Class$ \`e un'altra classe vuota, allora $\emptyset \subseteq \Class$ e $\Class \subseteq \emptyset$, da cui la tesi tramite l'assioma di estensionalit\`a. \EndProof
\par D'ora in poi denoteremo sempre con $\emptyset$ la classe vuota.
\begin{Theorem}
	Esiste la classe $\VonNeumannUniverse$, detta \Define{Universo di von Neumann}[di von Neumann][universo], i cui elementi sono tutti e soli gli insiemi.
\end{Theorem}
\Proof $\VonNeumannUniverse = \lbrace x | x \notin \emptyset \rbrace$. \EndProof
\begin{Definition}
	Se $\Class_1$ e $\Class_2$ sono due classi tali che non esista $x$ per cui $x \in \Class_1$ e $x \in \Class_2$, allora si dice che $\Class_1$ e $\Class_2$ sono \Define{disgiunte}[disgiunte][classi].
\end{Definition}
\begin{Axiom}
	\TheoremName{Assioma di regolarit\`a (o di fondazione)}[di regolarit\`a (o di fondazione)][assioma] Per ogni insieme $\Set_0$ esiste un elemento $\Set_1 \in \Set_0$ tale che $\Set_0$ e $\Set_1$ sono disgiunti.
\end{Axiom}
\begin{Axiom}
	\TheoremName{Assioma dell'accoppiamento}[della coppia][assioma] Dati due insiemi $\Set_1$ e $\Set_2$ esiste l'insieme $\lbrace \Set_1, \Set_2 \rbrace$.
\end{Axiom}
\begin{Theorem}
	Se $\Set$ \`e un insieme, esiste l'insieme $\lbrace \Set \rbrace$.
\end{Theorem}
\Proof $\lbrace \Set \rbrace$ si ottiene dall'assioma di accoppiamento prendendo $\Set_1 = \Set_2 = \Set$. \EndProof
\begin{Definition}
	Dati due insiemi $\Set_1$, $\Set_2$, la \Define{coppia} $(\Set_1,\Set_2)$ \`e l'insieme $\lbrace \lbrace \Set_1 \rbrace, \lbrace \Set_1, \Set_2 \rbrace \rbrace$. Per qualsiasi numero naturale\footnote{Non abbiamo ancora definito formalmente il concetto di numero naturale: ci affidiamo qui al concetto intuitivo metamatematico di ``numero usato per contare''.} $n$ maggiore\footnote{Non avendo definito l'insieme dei numeri naturali, non abbiamo definito neanche un loro ordinamento, n\'e l'operazione di sottrazione di $1$ utilizzata poco pi\`u avanti: persistiamo nell'appellarci all'intuizione metamatematica.} di $2$, la \Define{$n$-upla} $(\Set_1, ..., \Set_{n - 1}, \Set_n)$ \`e definita come la coppia $((\Set_1, ..., \Set_{n - 1}), \Set_n)$. Una \Define{terna} \`e una $3$-upla, una \Define{quadrupla} una $4$-upla, una \Define{quintupla} una $5$-upla e cos\`i via. Gli elementi che appaiono in una $n$-upla si chiamano \Define{componenti}[componente].
\end{Definition}
\begin{Definition}
	Chiamiamo \Define{classe funzionale}[funzionale][classe] una classe $\Function$ costituita da sole coppie tale che $\Implies{\And{(x,y) \in \Function}{(x,z) \in \Function}}{y = z}$.
\end{Definition}
\begin{Definition}
	Chiamiamo \Define{applicazione} o \Define{funzione} o ancora \Define{operatore}\footnote{Preferiremo il termine ``funzione'' in ambito numerico e il termine ``operatore'' quando gli argomenti sono altre funzioni.} il dato dei seguenti elementi:
	\begin{itemize}
		\item una classe funzionale $\Function$, detta \Define{grafo} dell'applicazione;
		\item una classe costituita esattamente da tutte le prime componenti delle coppie in $\Function$, detti \Define{argomenti}[argomento];
		\item una classe contenente tutte le seconde componenti delle coppie in funzione, dette \Define{immagini}[immagine].
	\end{itemize}
	Inoltre, dato un argomento $x$ di $\Function$, l'unica immagine $y$ tale che $(x,y) \in \Function$ si chiama \NIDefine{immagine} di $x$ tramite $\Function$ e si denota $\Function(x)$ o anche $\Function x$. Diciamo ancora che $\Function$ \Define{associa}[associazione] $y$ ad $x$. Per quei casi in cui non \`e tanto l'applicazione ad essere oggetto di interesse quanto l'associazione tra agormenti e immagini, definiamo il concetto di \Define{famiglia}, la quale \`e un'applicazione denotata $(m_i)_{i \in A}$, dove $A$ \`e l'insieme degli argomenti e $m_i$ \`e un termine, variabile a seconda del valore di $i$, che \`e l'immagine associata ad $i$ dalla famiglia stessa.
\end{Definition}
\begin{Axiom}
	\TheoremName{Assioma del limite della dimensione}[del limite della dimensione][assioma] $\Class$ \`e una classe propria se e solo se esiste una classe funzionale $\Function$ tale che per ogni $\Set_2 \in \VonNeumannUniverse$, esista uno e un solo $\Set_1 \in \Class$ con $(\Set_1,\Set_2) \in \Function$.
\end{Axiom}
\begin{Axiom}
	\TheoremName{Assioma dell'insieme delle parti}[dell'insieme delle parti][assioma] Dato un insieme $\Set$, esiste l'insieme $\lbrace x | x \subseteq \Set \rbrace$, d'ora in poi denotato $\PowerSet{\Set}$.
\end{Axiom}
\begin{Definition}
	Una classe $\Class$ si dice \Define{induttiva}[induttiva][classe] quando
	\begin{itemize}
		\item $\emptyset \in \Class$;
		\item $x \in \Class \Rightarrow x \cup \lbrace x \rbrace \in \Class$.
	\end{itemize}
\end{Definition}
\begin{Axiom}
	\TheoremName{Assioma dell'infinito}[dell'infinito][assioma] Esiste un insieme induttivo.
\end{Axiom}
\begin{Theorem}
	La classe vuota $\emptyset$ \`e un insieme, che chiameremo infatti d'ora in poi \Define{insieme vuoto}[vuoto][insieme].
\end{Theorem}
\Proof Dato che per l'assioma dell'infinito esiste almeno un insieme l'universo di von Neummann $\VonNeumannUniverse$ \`e non vuoto. Non pu\`o dunque esistere una classe funzionale $\Function$ tale che per ogni $\Set_2 \in \VonNeumannUniverse$, esista uno e un solo $\Set_1 \in \emptyset$ con $(\Set_1,\Set_2) \in \Function$. Dunque, per l'assioma di limite della dimensione, $\emptyset$ \`e un insieme. \EndProof
\par L'assioma dell'infinito \`e l'ultimo degli assiomi necessari a descrivere la teoria di Morse e Kelley. Gli assiomi che seguiranno devono essere considerati solo teoremi, che possono dunque essere aggiunti al sistema di assiomi senza danno. Essi rivestono comunque lo \textit{status} di assioma perch\'`e, come vedremo, possono a volte sostituire alcuni degli assiomi che abbiamo elencato o anche essere necessari in altre teorie degli insiemi.
\begin{Axiom}
	\TheoremName{Assioma dell'unione}[dell'unione][assioma] Dato un insieme $\Set$, esiste l'insieme $\lbrace x | (\exists y)(\And{y \in \Set}{x \in y}) \rbrace$, d'ora in poi denotato $\bigcup \Set$. Se $\Set$ ammette solo un numero finito di elementi $\Set_1, ..., \Set_n$ , allora denotiamo $\bigcup \Set$ anche con $\Set_1 \cup ... \cup \Set_n$.
\end{Axiom}
\begin{Theorem}
	L'assioma dell'unione \`e vero in MK.
\end{Theorem}
\begin{Axiom}
	\TheoremName{Schema di comprensione degli insiemi}[di comprensione degli insiemi][schema] Se $\phi(x)$ \`e una formula del primo ordine, allora per ogni insieme $\Set_0$ esiste un insieme $\Set$ tale che $x \in \Set \Leftrightarrow \And{x \in \Set_0}{\phi(x)}$.
\end{Axiom}
\begin{Axiom}
	\TheoremName{Assioma di rimpiazzamento} Se $\Function$ \`e una classe funzionale e un insieme e gli argomenti di $\Function$ costituiscono un insieme, allora le immagini di $\Function$ costituiscono un insieme, chiamato \Define{immagine}[immagine][insieme] di $\Function$ e denotato $\Image{\Function}$.
\end{Axiom}
\begin{Definition}
	Data una classe $\Class$, definiamo \Define{applicazione di scelta}[di scelta][applicazione] per $\Class$ un'applicazione $\Function$ avente le seguenti propriet\`a:
	\begin{itemize}
		\item gli argomenti di $\Function$ costituiscono la classe $\Class$;
		\item se $(x,y) \in \Function$, allora $y \in x$.
	\end{itemize}
\end{Definition}
\begin{Theorem}
	Con le notazioni della definizione precedente, se $\emptyset \in \Class$, allora non esistono applicazioni di scelta per $\Class$.
\end{Theorem}
\Proof Per definizione d'insieme vuoto, non esiste $y$ tale che $y \in \emptyset$. \EndProof
\begin{Axiom}
	\TheoremName{Assioma della scelta globale}[della scelta globale][assioma] Data una classe $\Class$ tale che $\emptyset \notin \Class$, esiste un'applicazione di scelta per $\Class$.
\end{Axiom}
\begin{Theorem}
	L'assioma di limite della dimensione \`e equivalente alla congiunzione dello schema di comprensione degli insiemei, degli assiomi di rimpiazzamento e della scelta globale.
\end{Theorem}

\subsection{Teoria di von Neumann, Bernays e G\"odel (NBG).}\label{NBG}
\par La teoria di von Neumann, Bernays e G\"odel \`e un'altra teoria degli insiemi, identica a quella di Morse e Kelley, ecceto che per lo schema di comprensione delle classi, che viene riformulato nel modo seguente.
\begin{Axiom}
	\TheoremName{Schema di comprensione delle classi (NBG)}[di comprensione delle classi (NBG)][schema] Se $\phi(x)$ \`e una formula del primo ordine che \`e priva di quantificatori sulle classi, allora esiste una classe $\Class$ tale che $\Set \in \Class \Leftrightarrow \phi(\Set)$.
\end{Axiom}
\par Naturalmente questa riformulazione produce una teoria degli insiemi pi\`u debole rispetto a quella di Morse e Kelley, ma questo indebolimento \`e ripagato dal fatto che la nuova teoria \`e finitamente assiomatizzabile. Diamo di seguito un esempio di un insieme finito di assiomi che pu\`o essere usato per sostituire lo schema di comprensione delle classi.
\begin{Axiom}
	\TheoremName{Assioma dell'appartenenza}[dell'appartenenza][assioma] Esiste una classe $\BelongingRelation$ tale che ne siano elementi tutte le coppie $(\Class_1,\Class_2$) tali che $\Class_1 \in \Class_2$ (non si esclude l'esistenza di altri elementi).
\end{Axiom}
\begin{Axiom}
	\TheoremName{Assioma dell'intersezione}[dell'intersezione][assioma] Date due classi $\Class_1$ e $\Class_2$ esiste la classe $\Class_1 \cap \Class_2$ tale che $x \in \Class_1 \cap \Class_2 \Leftrightarrow x \in \Class_1 \wedge x \in \Class_2$, detta \Define{intersezione} delle classi $\Class_1$ e $\Class_2$.
\end{Axiom}
\begin{Axiom}
	\TheoremName{Assioma del complementare}[del complementare][assioma] Data una classe $\Class$, esiste una classe $\Compl{\Class}$ tale che $x \in \Compl{\Class} \Leftrightarrow (\exists y)(x \in y) \wedge x \notin \Class$; vale a dire, $\Compl{\Class}$ \`e la classe i cui elementi sono tutti e soli gli insiemi (ma non le classi!) che non appartengono a $\Class$.
\end{Axiom}
\begin{Definition}
	Date due classi $\Class_1$, $\Class_2$, la classe $\Class_1 \cap \Compl{\Class_2}$ si chiama \Define{differenza} delle classi $\Class_1$ e $\Class_2$, denotata $\Class_1 \SetMin \Class_2$.
\end{Definition}
\begin{Axiom}
	\TheoremName{Assioma del dominio}[del dominio][assioma] Data una classe $\Class_1$, esiste la classe $\Class_2$ i cui elementi sono tutte e sole le prime componenti delle coppie appartenenti a $\Class_1$.
\end{Axiom}
\begin{Axiom}
	\TheoremName{Assioma del prodotto per $\VonNeumannUniverse$}[del prodotto per $\VonNeumannUniverse$][assioma] Per ogni classe $\Class_1$, esiste una classe $\Class_2$ i cui elementi sono tutte e sole le coppie $(x,y)$ dove $x \in \Class_1$ e $y \in \VonNeumannUniverse$. Denoteremo d'ora in poi la classe $\Class_2$ con la notazione $\Class_1 \times \VonNeumannUniverse$.
\end{Axiom}
\begin{Axiom}
	\TheoremName{Assioma della permutazione circolare}[della permutazione circolare][assioma] Per ogni classe $\Class_1$, esiste una classe $\Class_2$ tale per ogni terna $(x,y,z)$, $(x,y,z) \in \Class_2 \Leftrightarrow (y,z,x) \in \Class_2$.
\end{Axiom}
\begin{Axiom}
	\TheoremName{Assioma della trasposizione}[della trasposizione][assioma] Per ogni classe $\Class_1$, esiste una classe $\Class_2$ tale per ogni terna $(x,y,z)$, $(x,y,z) \in \Class_2 \Leftrightarrow (x,z,y) \in \Class_2$.
\end{Axiom}
\begin{Axiom}
	\TheoremName{Assioma di rimpiazzamento} Sia $\Function$ una classe funzionale. Se le prime componenti di tutte le coppie in $\Function$ costituiscono un insieme, allora le seconde componenti di tutte le coppie in $\Function$ costituiscono un secondo insieme.
\end{Axiom}

\subsection{Teoria di Zermelo e Fraenkel (ZF).}\label{ZFC}
\par La teoria di Zermelo e Fraenkel \`e la teoria degli insiemi di riferimento per ragioni storiche e didattiche, sebbene la nascita della teoria delle categorie abbia contribuito a rendere pi\`u popolare quella di von Neumann, Bernays e G\"odel. Si tratta di una teoria del primo ordine con uguaglianza che prevede solo il concetto di insieme, e non quello di classe: \`e sempre possibile parlare di classi, ma ci\`o deve essere fatto utilizzando complicate perifrasi il cui senso \`e ``tali termini non possono costituire un insieme'' (e dunque costituiscono una classe in NBG).
\par Come vedremo, molti assiomi sono comuni con le teorie precedenti, eventualmente restringendone la portata agli insiemi: per maggior chiarezza, riporteremo comunque esplicitamente tutti gli assiomi e le definizioni necessarie.
\begin{Definition}
	Ogni termine della teoria si chiama \Define{insieme}. Quando vale la relazione $\Set_1 \in \Set_2$ diciamo che l'insieme $\Set_1$ \Define{appartiene}[appartenenza] all'insieme $\Set_2$, che $\Set_1$ \`e \Define{elemento} di $\Set_2$ e che $\Set_2$ \Define{contiene}[contenimento] $\Set_1$. Denotiamo la relazione $\Not{\Set_1 \in \Set_2}$ col simbolo $\notin$.
\end{Definition}
\begin{Definition}
	Quando $(\Set_1 \in \Set_2) \Rightarrow (\Set_1 \in \Set_3)$ diciamo che $\Set_2$ \`e un \Define{sottoinsieme} di $\Set_3$, denotato $\Set_2 \subseteq \Set_3$ e che $\Set_3$ \`e un \Define{soprainsieme} di $\Set_2$, denotato $\Set_3 \supseteq \Set_2$; se $\Set_2 \neq \Set_3$, allora le notazioni diventano rispettivamente $\Set_2 \subset \Set_3$ e $\Set_3 \supset \Set_2$.
\end{Definition}
\begin{Axiom}
	\TheoremName{Assioma di estensionalit\`a (ZF)}[di estensionalit\`a (ZF)][assioma] Abbiamo $\Set_1 = \Set_2$ se e solo se $\Set_1 \subseteq \Set_2$ e $\Set_2 \subseteq \Set_1$.
\end{Axiom}
\par Per l'assioma di estensionalit\`a, dato un numero finito di insiemi, se esiste un insieme che li contiene tutti senza contenere nient'altro, allora esso \`e unico: lo denoteremo elencandone gli elementi tra parentesi graffe.
\begin{Axiom}
	\TheoremName{Schema di comprensione degli insiemi}[di comprensione degli insiemi][schema] Se $\phi(x)$ \`e una formula del primo ordine, allora per ogni insieme $\Set_0$ esiste un insieme $\Set$ tale che $x \in \Set \Leftrightarrow \And{x \in \Set_0}{\phi(x)}$.
\end{Axiom}
\par Dati una formula $\phi(x)$ e un insieme $\Set_0$ come nell'assioma precedente, denotiamo l'insieme $\Set$ con $\lbrace x \in \Set_0 | \phi(x) \rbrace$.
\begin{Definition}
	Se l'insieme $\Set$ verifica la relazione $(\forall x)(x \notin \Set)$ allora diciamo che $\Set$ \`e \Define{vuoto}[vuoto][insieme].
\end{Definition}
\begin{Axiom}
	Esiste un insieme vuoto $\emptyset$.
\end{Axiom}
\begin{Theorem}
	Sia $\Set_1$ un insieme vuoto e $\Set_2$ un insieme qualsiasi: abbiamo $\Set_1 \subseteq \Set_2$.
\end{Theorem}
\Proof Sia $x \in \Set_1$: poich\'e, per definizione di insieme vuoto, $x \notin \Set_1$, ne deduciamo $x \in \Set_2$. \EndProof
\begin{Corollary}
	$\emptyset$ \`e l'unico insieme vuoto.
\end{Corollary}
\Proof Se $\Set$ \`e un altro insieme vuoto, allora $\emptyset \subseteq \Set$ e $\Set \subseteq \emptyset$, da cui la tesi tramite l'assioma di estensionalit\`a. \EndProof
\begin{Definition}
	Se $\Set_1$ e $\Set_2$ sono due insiemi tali che non esista $x$ per cui  $x \in \Set_1$ e $x \in \Set_2$, allora si dice che $\Set_1$ e $\Set_2$ sono \Define{disgiunti}[disgiunti][insiemi].
\end{Definition}
\begin{Axiom}
	\TheoremName{Assioma di regolarit\`a (o di fondazione)}[di regolarit\`a (o di fondazione)][assioma] Per ogni insieme $\Set_0$ esiste un elemento $\Set_1 \in \Set_0$ tale che $\Set_0$ e $\Set_1$ sono disgiunti.
\end{Axiom}
\begin{Axiom}
	\TheoremName{Assioma dell'accoppiamento}[della coppia][assioma] Dati due insiemi $\Set_1$ e $\Set_2$ esiste l'insieme $\lbrace \Set_1, \Set_2 \rbrace$.
\end{Axiom}
\begin{Theorem}
	Se $\Set$ \`e un insieme, esiste l'insieme $\lbrace \Set \rbrace$.
\end{Theorem}
\Proof $\lbrace \Set \rbrace$ si ottiene dall'assioma di accoppiamento prendendo $\Set_1 = \Set_2 = \Set$. \EndProof
\begin{Definition}
	Dati due insiemi $\Set_1$, $\Set_2$, la \Define{coppia} $(\Set_1,\Set_2)$ \`e l'insieme $\lbrace \lbrace \Set_1 \rbrace, \lbrace \Set_1, \Set_2 \rbrace \rbrace$. Per qualsiasi numero naturale\footnote{Non abbiamo ancora definito formalmente il concetto di numero naturale: ci affidiamo qui al concetto intuitivo metamatematico di ``numero usato per contare''.} $n$ maggiore\footnote{Non avendo definito l'insieme dei numeri naturali, non abbiamo definito neanche un loro ordinamento, n\'e l'operazione di sottrazione di $1$ utilizzata poco pi\`u avanti: persistiamo nell'appellarci all'intuizione metamatematica.} di $2$, la \Define{$n$-upla} $(\Set_1, ..., \Set_{n - 1}, \Set_n)$ \`e definita come la coppia $((\Set_1, ..., \Set_{n - 1}), \Set_n)$. Una \Define{terna} \`e una $3$-upla, una \Define{quadrupla} una $4$-upla, una \Define{quintupla} una $5$-upla e cos\`i via. Gli elementi che appaiono in una $n$-upla si chiamano \Define{componenti}[componente].
\end{Definition}
\begin{Definition}
	Chiamiamo \Define{insieme funzionale}[funzionale][insieme] un insieme $\Function$ costituito da sole coppie tale che $\And{(x,y) \in \Function}{(x,z) \in \Function} \Rightarrow y = z$.
\end{Definition}
\begin{Axiom}
	\TheoremName{Assioma di rimpiazzamento} Se $\Function$ \`e un insieme funzionale e gli argomenti di $\Function$ costituiscono un insieme, allora anche le immagini di $\Function$ costituiscono un insieme, chiamato \Define{immagine}[immagine][insieme] di $\Function$ e denotato $\Image{\Function}$.
\end{Axiom}
\begin{Axiom}
	\TheoremName{Assioma dell'insieme delle parti}[dell'insieme delle parti][assioma] Dato un insieme $\Set$, esiste l'insieme $\lbrace x | x \subseteq \Set \rbrace$, d'ora in poi denotato $\PowerSet{\Set}$.
\end{Axiom}
\begin{Axiom}
	\TheoremName{Assioma dell'unione}[dell'unione][assioma] Dato un insieme $\Set$, esiste l'insieme $\lbrace x | (\exists y)(\And{y \in \Set}{x \in y}) \rbrace$, d'ora in poi denotato $\bigcup \Set$. Se $\Set$ ammette solo un numero finito di elementi $\Set_1, ..., \Set_n$ , allora denotiamo $\bigcup \Set$ anche con $\Set_1 \cup ... \cup \Set_n$.
\end{Axiom}
\begin{Definition}
	Un insieme $\Set$ si dice \Define{induttivo}[induttivo][insieme] quando
	\begin{itemize}
		\item $\emptyset \in \Set$;
		\item $x \in \Set \Rightarrow x \cup \lbrace x \rbrace \in \Set$.
	\end{itemize}
\end{Definition}
\begin{Axiom}
	\TheoremName{Assioma dell'infinito}[dell'infinito][assioma] Esiste un insieme induttivo.
\end{Axiom}
\par Presentiamo un ultimo assioma. Esso \`e oggetto di controversie: assumerlo vero porta a volte a risultati desiderabili, altre volte a risultati che appaiono controintuitivi. Per questo motivo, si evita il pi\`u possibile di ricorrere ad esso in una dimostrazione e, quando il suo uso \`e veramente inevitabile, ci\`o viene sottolineato.
\par Il seguente assioma viene spesso indicato con la sigla AC e, quando si suppone vero, alla sigla ZF che indica la teoria di Zermelo e Fraenkel si aggiunge una C ottenendo la sigla ZFC.
\begin{Axiom}
	\TheoremName{Assioma della scelta}[della scelta][assioma] Dato un insieme $\Set$ tale che $\emptyset \notin \Set$, esiste un insieme funzionale $\Function$ tale che
	\begin{itemize}
		\item gli argomenti di $\Function$ costituiscono l'insieme $\Set$;
		\item se $(x,y) \in \Function$, allora $x \in y$.
	\end{itemize}
\end{Axiom}


\input{Fondamenti_della_matematica/OperazioniBinarie.tex}
\section{Categorie.}\label{Categorie}
\subsection{Definizioni e nozioni di base.}\label{CategorieDefinizioniENozioniDiBase}
\begin{Definition}
	Una \Define{categoria} $\Category$ \`e una classe -- i cui elementi chiamiamo \Define{frecce}[freccia] o \Define{morfismi}[morfismo] -- munita di un'operazione binaria interna associativa $\cdot$ che chiamiamo \Define{composizione}[composizione di frecce]. Quando per due frecce $f,g \in \Category$ abbiamo $(f,g) \in \Dom{\cdot}$ diciamo che $f$ e $g$ sono \Define{componibili}[frecce componibili] o che $f$ \`e \NIDefine{componibile a sinistra} con $g$. Richiediamo inoltre che la composizione verifichi la seguente propriet\`a: per ogni freccia $f$, esistono frecce $u_d$ e $u_c$ tali che
		\begin{enumerate}
			\item per ogni freccia $g$ componibile a sinistra con $u_d$ [$u_c$] abbiamo $g \circ u_d = g$ [$g \circ u_c = g$],\label{identita_1}
			\item per ogni freccia $g$ componibile a destra con $u_d$ [$u_c$] abbiamo $u_d \circ g = g$ [$u_c \circ g = g$],\label{identita_2}
			\item $f$ \`e componibile a sinistra con $u_d$ e componibile a destra con $u_c$.
		\end{enumerate}
	Chiamiamo una freccia $u$ con le propriet\`a \ref{identita_1} e \ref{identita_2}, dove si pone $u_c = u_d = u$, \Define{identit\`a} o \Define{oggetto}: dunque richiediamo che per ogni freccia $f$ esistano un'identit\`a componibile a sinistra con $f$ e una componibile a destra.
\end{Definition}
\begin{Theorem}
	Sia $(\Category_i)_{i \in I}$ una famiglia di categorie tali che per ogni coppia di frecce di $\bigcap_{i \in I} \Category_i$ la composizione sia definita nella stessa maniera in tutte le categorie di $(\Category_i)_{i \in I}$. Allora le frecce di $\bigcap_{i \in I} \Category_i$ costituiscono una categoria.
\end{Theorem}
\Proof La verifica \`e immediata. \EndProof
\begin{Theorem}
	Il duale di una categoria \`e una categoria e le identit\`a della categoria originale e della categoria duale coincidono.
\end{Theorem}
\Proof Gli assiomi di categoria sono invarianti per dualit\`a. \EndProof
\begin{Theorem}\label{unicita_dominio}
	Sia $\Category$ una categoria e $f$ una freccia. Esiste un'unica identit\`a $u_d$ componibile a destra con $f$.
\end{Theorem}
\Proof L'esistenza \`e garantita dagli assiomi.
\par Siano $u_d$ e $v_d$ due identit\`a componibili a destra con $f$. Poich\'e $v_d$ \`e componibile a destra con $f = f u_d$, allora $v_d$ \`e anche componibile a destra con $u_d$. Abbiamo $u_d = u_d \circ v_d = v_d$ per definizione di identit\`a. \EndProof
\begin{Definition}
	Sia $f$ una freccia in una categoria $\Category$. Il \Define{dominio} di $f$, denotato $\Dom f$, \`e l'unica identit\`a componibile a destra con $f$. Il duale del dominio di $f$, denotato $\Cod f$, \`e chiamato \Define{codominio} di $f$ (\`e l'unica identit\`a componibile a sinistra con $f$). Chiamiamo \Define{parallele} due frecce che hanno lo stesso dominio e lo stesso codominio.
\end{Definition}
\begin{Theorem}
	Due frecce $f$ e $g$ sono componibili (in quest'ordine) se e solo se $\Dom f = \Cod g$.
\end{Theorem}
\Proof Siano $f$ e $g$ componibili. Poich\'e $f \circ \Dom f = f$, $\Dom f$ e $g$ sono componibili. Inoltre $\Dom f$ \`e un'identit\`a componibile a sinistra con $g$, dunque necessariamente $\Cod g = \Dom f$.
\par Viceversa, se $\Dom f = \Cod g$, allora, poich\'e $f$ \`e componibile a sinistra con $\Dom f$ e $\Cod g = \Dom f$ \`e componibile a sinistra con $g$, allora $f$ e $g$ sono componibili. \EndProof
\par D'ora in poi, invece di affermare che $f$ \`e una freccia tale che $\Dom f = a$ e $\Cod f = b$, scriveremo $f: a \rightarrow b$. Pi\`u generalmente, diamo la definizione seguente.
\begin{Definition}
	Un \Define{diagramma} \`e un disegno composto di frecce grafiche $\rightarrow$, eventualmente etichettate, che rappresentano parti di categorie secondo le seguenti regole:
	\begin{itemize}
		\item ogni freccia grafica corrisponde ad una ed una sola freccia di categoria;
		\item un'etichetta associata ad una freccia grafica \`e un nome per la freccia di categoria corrispondente;
		\item un'etichetta all'inizio [alla fine] di una freccia grafica \`e un nome per il dominio [codominio] della freccia di categoria corrispondente.
	\end{itemize}
	Un diagramma si dice \Define{commutativo}[commutativo][diagramma] se e solo qualsiasi coppia di sequenze di frecce grafiche consecutive che cominciano nello stesso punto e finiscono nello stesso punto, le composizioni delle corrispondenti frecce di categoria sono uguali.
\end{Definition}
\begin{Theorem}
	Sia $\Identity$ una identit\`a. Abbiamo $\Cod{\Identity} =
	\Dom{\Identity} = \Identity$.
\end{Theorem}
\Proof Abbiamo $\Cod{\Identity} = \Cod{\Identity} \Identity =
\Cod{\Identity} \Identity \Dom{\Identity} =
\Identity \Dom{\Identity} = \Dom{\Identity}$.
\par Inoltre, $\Cod{\Identity} = \Dom{\Identity}$ implica che
$\Identity$ \`e componibile con se stessa, dunque $\Identity$ deve
coincidere col proprio dominio e col proprio codominio. \EndProof
\begin{Corollary}
	Le identit\`a di una categoria sono tutti e soli i domini delle frecce della stessa categoria. 
\end{Corollary}
\Proof I domini sono identit\`a per definizione e ogni identit\`a \`e dominio di se stessa. \EndProof
\begin{Definition}
	Chiamiamo \Define{inversa sinistra}[inversa sinistra][freccia] di una freccia $f$ una freccia $r$ tale che $rf$ \`e un'identit\`a. Quando $f$ ammette un'inversa sinistra si dice che $f$ \`e \Define{invertibile a sinistra}[invertibile a sinistra][freccia].
\end{Definition}
\begin{Definition}
	Chiamiamo
	\begin{itemize}
		\item \Define{monomorfismo} una freccia semplificabile a sinistra;
		\item \Define{epimorfismo} una freccia semplificable a destra;
		\item \Define{retrazione} una freccia invertibile a destra;
		\item \Define{sezione} una freccia invertibile a sinistra;
		\item \Define{isomorfismo} una freccia invertibile;
		\item \Define{endomorfismo} una freccia $f$ tale che $\Dom f = \Cod f$;
		\item \Define{automorfismo} una freccia che \`e simultaneamente un endomorfismo e un isomorfismo.
	\end{itemize}
	Se $r$ e $s$ sono, rispettivamente, una retrazione e una sezione tale che $rs$ \`e un'identit\`a, diciamo che $r$ ed $s$ sono \Define{associate}[associate][frecce].
\end{Definition}
\begin{Definition}
	Un monomorfismo $m$ che \`e un'identit\`a e tale che per ogni altra identit\`a $i$ esiste una e una sola freccia $f$ di dominio $i$ e codominio $m$ si dice \Define{finale}[finale][freccia]; una freccia che soddisfa la condizione duale si dice \Define{iniziale}[iniziale][freccia]. Una freccia iniziale e finale si dice \Define{nulla}[nulla][freccia].
\end{Definition}
\begin{Theorem}\label{unicita_finale}
	Se una categoria ammette un oggetto finale, esso \`e unico a meno di isomorfismo.
\end{Theorem}
\Proof Siano $a$ e $b$ oggetti finali. Per la finalit\`a di $a$ esiste un'unica freccia $f: b \rightarrow a$ e per la finalit\`a di $g$ esiste un'unica freccia $g: a \rightarrow b$. Abbiamo necessariamente $f \circ g = \Id_a$ e $g \circ f = \Id_b$, dunque $f$ e $g$ sono isomorfismi, l'uno l'inverso dell'altro. \EndProof

\subsection{Funtori e trasformazioni naturali.}\label{FuntoriETrasformazioniNaturali}
\begin{Definition}
	Siano $\Category_1$ e $\Category_2$ due categorie. Un \Define{funtore} $T$ dalla categoria $\Category_1$ alla categoria $\Category_2$ \`e un'applicazione che associa ad ogni freccia $f$ di $\Category_1$ una e una sola freccia di $\Category_2$, denotata $T(f)$, in modo tale che, per ogni coppia di frecce componibili $f$ e $g$ di $\Category_1$, si abbia $T(fg) = T(f)T(g)$ e se $f$ \`e un'identit\`a, allora anche $T(f)$ \`e un'identit\`a.
\end{Definition}
\begin{Theorem}
	Dati due funtori $T$ e $S$, il primo dalla categoria $\Category_2$ nella categoria $\Category_3$, il secondo dalla categoria $\Category_1$ nella categoria $\Category_2$, l'applicazione che associa ad ogni freccia $f$ di $\Category_1$ la freccia $T(S(f))$ (che denoteremo anche pi\`u brevemente $TS(f)$) della categoria $\Category_3$ \`e un funtore dalla categoria $\Category_1$ alla categoria $\Category_3$.
\end{Theorem}
\Proof Siano $f$ e $g$ due frecce di $\Category_1$. Abbiamo $T(S(fg)) = T(S(f)S(g)) = TS(f)TS(g)$. \EndProof
\begin{Definition}
	Dati due funtori $T$ e $S$ come nel teorema precedente, il funtore $TS$ si chiama \Define{funtore composto}[composto][funtore] di $T$ ed $S$.
\end{Definition}
\begin{Definition}
	Data una categoria $\Category$, il funtore $\Id_\Category$ da $\Category$ in $\Category$ che ad ogni freccia associa la stessa freccia si chiama \Define{funtore identit\`a}[identit\`a][funtore] di $\Category$.
\end{Definition}
\begin{Theorem}
	I funtori costituiscono una categoria cos\`i definita:
	\begin{itemize}
		\item due funtori sono componibili se e solo la categoria d'arrivo del secondo coincide con la categoria di partenza del primo;
		\item se due funtori sono componibili, la loro composizione \`e definita essere il funtore composto.
	\end{itemize}
\end{Theorem}
\Proof La verifica degli assiomi di categoria \`e immediata. \EndProof
\par La corrispondenza che associa ad ogni categoria il suo funtore identit\`a \`e biunivoca, si pu\`o dunque dire che gli oggetti della categoria di tutti i funtori sono le categorie. Si usa dare un nome alle categorie a partire dai suoi oggetti: cos\`i la categoria di tutti i funtori \`e la categoria $\CatCateg$.
\par D'ora in poi, anzich\'e scrivere che $T$ \`e un funtore dalla categoria $\Category_1$ nella categoria $\Category_2$ potremo dunque scrivere che $T: \Category_1 \rightarrow \Category_2$ \`e un funtore.
\begin{Theorem}
	Esiste una e una sola categoria con un'unica freccia, a meno d'isomorfismo.
\end{Theorem}
\Proof Supponiamo che la categoria esista e denotiamo la sua unica freccia $u$. Necessariamente $\Dom u = \Cod u = u$ e dunque $u$ \`e componibile con se stessa; inoltre deve essere $u \circ u = u$. La verifica degli assiomi di categoria \`e immediata, cos\`i come la dimostrazione d'isomorfismo di due categorie qualsiasi di un'unica freccia ciascuna. \EndProof
\begin{Definition}
	L'unica categoria costituita da un'unica freccia viene denotata con $1$ e la sua unica freccia viene anch'essa denotata con $1$: la chiameremo \Define{categoria unit\`a}[unit\`a][categoria].
\end{Definition}
\begin{Theorem}
	Sia $T: \Category_1 \rightarrow \Category_2$ un funtore. L'applicazione $S: \Dual{\Category_1} \rightarrow\Dual{\Category_2}$ con grafo uguale a quello di $T$ \`e un funtore.
\end{Theorem}
\Proof Siano $f$ e $g$ frecce di $\Dual{\Category_1}$, componibili in quest'ordine. Allora $g$ e $f$ sono componibili in $\Category_1$ in quest'ordine e abbiamo $S(fg) = T(gf) = T(g)T(f) = S(g)s(f)$. Ma $S(g)S(f)$ in $\Category_2$ \`e il risultato della composizione $S(f)S(g)$ in $\Dual{\Category_2}$. \EndProof
\begin{Theorem}
	Sia $T: \Category_1 \rightarrow \Category_2$ un funtore. Abbiamo, per ogni freccia $f$ di $\Category_1$:
	\begin{itemize}
		\item $T(\Dom f) = \Dom T(f)$;
		\item $T(\Cod f) = \Cod T(f)$.
	\end{itemize}
\end{Theorem}
\Proof Per definizione di funtore $T(\Dom f)$ \`e un'identit\`a. Inoltre $T(f) \circ T(\Dom f) = T(f \circ \Dom f) = T(f)$, pertanto $\Dom T(f) = T(\Dom f)$. L'altra affermazione si ottiene per dualit\`a considerando $T: \Dual{\Category_1} \rightarrow \Dual{\Category_2}$. \EndProof
\begin{Definition}\label{def_trasfnat}
	Una \Define{trasformazione naturale} $\tau$ dal funtore $T: \Category_1 \rightarrow \Category_2$ nel funtore $S: \Category_1 \rightarrow \Category_2$ \`e un'applicazione che associa ad ogni identit\`a $c$ di $\Category_1$ una freccia $\tau(c)$ di $\Category_1$ in modo tale che il diagramma di figura \ref{fig_trasfnat} sia commutativo per qualsiasi scelta di $f$ in $\Category_1$.
\end{Definition}
\begin{figure}
	\center
	\begin{tikzcd}[column sep=6em, row sep=6em]
		T(\Dom f) \arrow{r}{\tau(T(\Dom f))} \arrow{d}[swap]{T(f)} & S(\Dom f) \arrow{d}{S(f)} \\
		T(\Cod f) \arrow{r}{\tau(T(\Cod f))} & S(\Cod f)
	\end{tikzcd}
	\caption{Definizione di trasformazione naturale.}
	\label{fig_trasfnat}
\end{figure}
\begin{Theorem}\label{natop1}
	Siano $\tau$ e $\sigma$ due trasformazioni naturali, la prima da $T: \Category_1 \rightarrow \Category_2$ in $R: \Category_1 \rightarrow \Category_2$, la seconda da $R$ a $S: \Category_1 \rightarrow \Category_2$. L'applicazione che associa a $c$ oggetto di $\Category_1$ la freccia $\tau(c)\sigma(c)$ \`e una trasformazione naturale da $T$ in $S$.
\end{Theorem}
\Proof Basta osservare che il diagramma di figura \ref{fig_natop1} \`e commutativo. \EndProof
\begin{figure}
	\center
	\begin{tikzcd}[column sep=6em, row sep=6em]
		T(\Dom f) \arrow{r}{\tau(T(\Dom f))} \arrow{d}[swap]{T(f)} & R(\Dom f) \arrow{r}{\sigma(R(\Dom f))} \arrow{d}{R(f)} & S(\Dom f) \arrow{d}{S(f)} \\
		T(\Cod f) \arrow{r}{\tau(T(\Cod f))} & R(\Cod f) \arrow{r}{\sigma(R(\Cod f))} & S(\Cod f)
	\end{tikzcd}
	\caption{Dimostrazione del teorema \ref{natop1}.}
	\label{fig_natop1}
\end{figure}
\begin{Theorem}\label{natop2}
	Siano $\tau$ e $\tau'$ trasformazioni naturali, la prima da $T: \Category_1 \rightarrow \Category_2$ in $S: \Category_1 \rightarrow \Category_2$, la seconda da $T': \Category_2 \rightarrow \Category_3$ in $S': \Category_2 \rightarrow \Category_3$. Per ogni oggetto $c$ di $\Category_1$ il diagramma di figura \ref{fig_natop2_1} \`e commutativo e l'applicazione che associa a $c$ la freccia $\tau'(S(c))T'(\tau(c))$ (la diagonale dello stesso diagramma) \`e una trasformarzione naturale da $T'T$ in $S'S$.
\end{Theorem}
\begin{figure}
	\center
	\begin{tikzcd}[column sep=6em, row sep=6em]
		T'(T(c)) \arrow{r}{\tau'(T(c))} \arrow{d}{T'(\tau(c))} & S'(T(c)) \arrow{d}{S'(\tau(c))}\\
		T'(S(c)) \arrow{r}{\tau'(S(c))} & S'(S(c))
	\end{tikzcd}
	\caption{Enunciato del teorema \ref{natop2}.}
	\label{fig_natop2_1}
\end{figure}
\Proof Per la prima parte del teorema \`e sufficiente ridisegnare il diagramma di figura \ref{fig_trasfnat} ponendo $f = \tau(c)$, per la seconda parte si deve osservare che il diagramma di figura \ref{fig_natop2_2} \`e commutativo.\EndProof
\begin{figure}
	\center
	\begin{tikzcd}[column sep=6em, row sep=6em]
		(T'T)(\Cod f) \arrow[red]{dd}{T'(\tau(\Cod f))} & (T'T)(\Dom f) \arrow{l}{(T'T)(f)} \arrow[red]{d}{T'(\tau(\Dom f))} \arrow{r}{\tau'(T(\Dom f))} & (S'T)(\Dom f) \arrow{d}{S'(\tau(\Dom f))}\\
		 & (T'S)(\Dom f) \arrow{dl}{(T'S)(f)} \arrow[red]{r}{\tau'(S(\Dom f))} & (S'S)(\Dom f) \arrow{d}{(S'S)(f)}\\
		(T'S)(\Cod f) \arrow[red]{rr}{\tau'(S(\Cod f))} & & (S'S)(\Cod f)
	\end{tikzcd}
	\caption{Dimostrazione del teorema \ref{natop2}.}
	\label{fig_natop2_2}
\end{figure}
\begin{Definition}
	Denotiamo $\tau \bullet \sigma$ la trasformazione naturale definita dal teorema \ref{natop1} e $\tau \circ \tau'$ la trasformazione naturale definita dal teorema \ref{natop2}.
\end{Definition}
\begin{Theorem}
	Le trasformazioni naturali costituiscono due categorie, una definita dall'operazione $\bullet$, l'altra dall'operazione $\circ$.
\end{Theorem}
\Proof La verifica degli assiomi di categoria \`e immediata (gli oggetti sono quelle trasformazioni naturali da un funtore allo stesso funtore che associano ad ogni oggetto della categoria di partenza la sua identit\`a, sono lo stesso oggetto). \EndProof

\subsection{Universalit\`a.}\label{Univarsalita}
\begin{Theorem}\label{comma_th}
	Siano i funtori $T: \Category_1 \rightarrow \Category_3$ e $S: \Category_2 \rightarrow \Category_3$. Definiamo la seguente struttura:
	\begin{itemize}
		\item chiamiamo ``frecce'' tutte le quadruple $(f_S,h,k,f_T)$, con $h$ freccia di $\Category_1$, $k$ freccia di $\Category_2$ e $f_T$ e $f_S$ frecce di $\Category_3$ per cui $f_T \circ T(h) = S(k) \circ f_S$ (diagramma di figura \ref{comma_1});
		\item diciamo che due ``frecce'' $(f_S,h,k,f_T)$ e $(f_S',h',k',f_T')$ sono ``componibili'' se e solo se $f_T = f_S'$, $h'$ e $h$ sono componibili in $\Category_1$ e $k'$ e $k$ sono componibili in $\Category_2$ (diagramma di figura \ref{comma_2});
		\item chiamiamo ``composizione'' l'operazione sulle ``frecce componibili'' cos\`i definita:  $(f_S,h,k,f_T) \circ (f_T,h',k',f_T') = (f_T',h' \circ h, k' \circ k,f_S)$ (perimetro esterno del diagramma di figura \ref{comma_2}).
	\end{itemize}
	Tale struttura \`e una categoria.
\end{Theorem}
\par La verifica degli assiomi \`e immediata. Verifichiamo esplicitamente solo che la composizione di due ``frecce'' \`e ancora una ``freccia'': $f_T' \circ T(h'h) = f_T' \circ T(h') \circ T(h) = S(k') \circ f_T \circ T(h) = S(k') \circ S(k) \circ f_S = S(k'k) \circ f_S$. \EndProof
\begin{figure}
	\center
	\begin{tikzcd}[column sep=6em, row sep=6em]
		{} \arrow{r}{T(h)} \arrow{d}{f_S} & {} \arrow{d}{f_T}\\
		{} \arrow{r}{S(k)} & {}
	\end{tikzcd}
	\caption{Enunciato del teorema \ref{comma_th}, diagramma 1.}
	\label{comma_1}
\end{figure}
\begin{figure}
	\center
	\begin{tikzcd}[column sep=6em, row sep=6em]
		{} \arrow{r}{T(h)} \arrow{d}{f_S} & {} \arrow{d}{f_T} \arrow{r}{T(h')} & {} \arrow{d}{f_T'}\\
		{} \arrow{r}{S(k)} & {} \arrow{r}{S(k')} & {}
	\end{tikzcd}
	\caption{Enunciato del teorema \ref{comma_th}, diagramma 2.}
	\label{comma_2}
\end{figure}
\begin{Definition}\label{comma_def}
	La categoria costruita nel teorema precedente si chiama \Define{categoria delle frecce}[delle frecce][categoria] o \Define{categoria dei morfismi}[dei morfismi][categoria]\footnote{In inglese si usa il termine \Define{comma category}[comma][categoria].} da $T$ in $S$, denotata $\CommaCategory{T}{S}$. Definiamo inoltre, per ogni $d,e \in \Category_3$, le categorie seguenti:
	\begin{itemize}
		\item $\CommaCategory{d}{S} = \CommaCategory{F_d}{S}$, dove $F_d$ \`e il funtore $F_d: 1 \rightarrow \Category_3$ definito da $F_d(1) = d$ (diagramma di figura \ref{comma_3});
		\item $\CommaCategory{T}{e} = \CommaCategory{T}{F_e}$, dove $F_e$ \`e il funtore $F_e: 1 \rightarrow \Category_3$ definito da $F_e(1) = e$ (diagramma di figura \ref{comma_4});
		\item $\CommaCategory{d}{e} = \CommaCategory{F_d}{F_e}$, dove $F_d$ e $F_e$ sono gli stessi funtori definiti nei due casi precedenti (diagramma di figura \ref{comma_5}).
	\end{itemize}
\end{Definition}
\begin{figure}
	\center
	\begin{tikzcd}[column sep=6em, row sep=6em]
		& {d} \arrow{dl}{f_S} \arrow{dr}{f_T} & \\
		{} \arrow{rr}{S(k)} & {} & {}
	\end{tikzcd}
	\caption{Definizione \ref{comma_def}, diagramma 1.}
	\label{comma_3}
\end{figure}
\begin{figure}
	\center
	\begin{tikzcd}[column sep=6em, row sep=6em]
		{} \arrow{rr}{T(h)} \arrow{dr}{f_S} & & {} \arrow{dl}{f_T}\\
		& {e} &
	\end{tikzcd}
	\caption{Definizione \ref{comma_def}, diagramma 2.}
	\label{comma_4}
\end{figure}
\begin{figure}
	\center
	\begin{tikzcd}[column sep=6em, row sep=6em]
		{d} \arrow{d}{f_S = f_T}\\
		{e}
	\end{tikzcd}
	\caption{Definizione \ref{comma_def}, diagramma 3.}
	\label{comma_5}
\end{figure}
\begin{Theorem}
	Data una categoria $\CommaCategory{T}{S}$, le sue identit\`a sono tutte e sole le frecce tali che (usiamo le stesse notazioni del teorema \ref{comma_th})
	\begin{itemize}
		\item $h$ e $k$ sono identit\`a;
		\item $f_S = f_T$.
	\end{itemize}
\end{Theorem}
\Proof \`E immediato verificare che se $h$ e $k$ sono identit\`a e $f_S = f_T$ allora $(f_T,h,k,f_S)$ \`e un'identit\`a.
\par Sia $(f_S,h,k,f_T)$ un'identit\`a. Poich\'e le identit\`a
sono componibili con loro stesse, deve essere $f_S = f_T$.
\par Sia poi $a$ una freccia componibile a sinistra e si consideri la
freccia $(g_S,a,\Dom k,f_S)$, dove $g_S$ \`e il funtore che ad ogni
freccia di $\Category_1$ associa $\Dom k$: componendola con
$(f_S,h,k,f_T)$ proviamo che $ah = h$. Analogamente proviamo che per ogni
freccia $b$ componibile a destra con $h$ abbiamo $hb = b$. Dunque $h$ \`e
un'identit\`a. Allo stesso modo si prova che $k$ \`e un'identit\`a.
\EndProof
\begin{Definition}\label{universale_def}
	Sia $c$ un oggetto della categoria $\Category_2$ e $S: \Category_1 \rightarrow \Category_2$ un funtore. Un oggetto iniziale nella categoria $\CommaCategory{c}{S}$ \`e detto \Define{universale}[universalit\`a] da $c$ in $S$. Inoltre, con riferimento al diagramma di figura \ref{universale_1} si dice ancora che $A$ e $\phi$ sono \NIDefine{universali} per $c$ ed $S$, che $A$ \`e un \Define{oggetto universale}[universale][oggetto] e che $\phi$ \`e una \Define{freccia universale}[universale][freccia]. Diciamo \NIDefine{universale} anche il duale di un'identit\`a universale\footnote{Alcuni autori distinguono i due concetti usando i termini \NIDefine{universale} e \Define{couniversale}[couniversale][freccia], ma non c'\`e accordo riguardo a quali dei due sarebbe quello universale e quale quello couniversale.}, rappresentato dal diagramma di figura \ref{universale_2}.
\end{Definition}
\begin{figure}
	\center
	\begin{tikzcd}[column sep=6em, row sep=6em]
		& c \arrow{dl}{\phi} \arrow{dr}{f} & \\
		S(A) \arrow{rr}{S(g)} & & S(B)
	\end{tikzcd}
	\caption{Definizione \ref{universale_def}, caso iniziale.}
	\label{universale_1}
\end{figure}
\begin{figure}
	\center
	\begin{tikzcd}[column sep=6em, row sep=6em]
		S(A) \arrow{dr}{\phi} & & S(B)\arrow{ll}{S(g)}\arrow{dl}{f} \\
		& c &
	\end{tikzcd}
	\caption{Definizione \ref{universale_def}, caso finale.}
	\label{universale_2}
\end{figure}
\par Nel seguito, salvo indicazioni contrarie, analizzeremo solo i casi iniziali di universalit\`a, mentre quelli finali seguiranno automaticamente per dualit\`a.
\begin{Theorem}
	Con le medesime notazioni della definizione \label{universale_def}, affermare che $(A,\phi)$ \`e universale in una categoria $\CommaCategory{c}{S}$ equivale a dire che per ogni scelta di $B \in \Category_1$ esiste una e una sola freccia $g \in \Category_1$ che faccia commutare il diagramma \ref{universale_1}.
\end{Theorem}
\Proof Segue direttamente dalle definizioni. \EndProof
\begin{Theorem}
	Un oggetto universale dall'identit\`a $c$ nella categoria $\Category_2$ nel funtore $S: \Category_1 \rightarrow \Category_2$, se esiste, \`e unico.
\end{Theorem}
\Proof Conseguenza immediata del teorema \ref{unicita_finale}. \EndProof
\begin{Definition}\label{def_proiezione}
	Data una categoria $\Category$ e un funtore $S$ avente per codominio $\Category$, chiamiamo \Define{proiezione} un epimorfismo universale da $c$ in $S$.
\end{Definition}
\begin{Theorem}
	Data una classe $\Class$, sia $\Category_1$ una categoria. Possiamo definie una categoria $\Category_2$
	\begin{itemize}
		\item scegliendo come frecce la classe di tutte le classi funzionali dove gli argomenti di $\Function$ costituiscono una classe $\Class$ e tutte le immagini di $\Function$ appartengono a $\Category_1$;
		\item definendo due frecce $f$ e $g$ componibili se e solo per ogni $x \in \Class$ sono componibili $f(x)$ e $g(x)$.
	\end{itemize}
\end{Theorem}
\Proof La verifica \`e immediata. \EndProof
\begin{Definition}
	Con le notazioni del teorema precedente, chiamiamo $\Category_2$ \Define{categoria potenza}[potenza][categoria] di $\Category_1$ in $\Class$, d'ora in poi denotata $\Category_1^\Class$.
\end{Definition}
\begin{Definition}
	Fissate una categoria $\Category$ e una classe $\Class$, definiamo \Define{funtore replicante}[replicante][funtore]\footnote{Questa terminologia \`e nostra.} il funtore da $\Category$ in $\Category^\Class$ che alla freccia $f$ associa la classe funzionale in cui tutte le immagini sono uguali a $f$.
\end{Definition}
\begin{Definition}
	Fissate una categoria $\Category$ e una classe $\Class$ e scelta una famiglia di oggetti $(\Object_i)_{i \in \Class} \in \Category^\Class$ chiamiamo
	\begin{itemize}
		\item \Define{oggetto prodotto}[prodotto][oggetto] della famiglia $(\Object_i)_{i \in \Class}$, denotato $\DirectProduct_{i \in \Class} \Object_i$, un oggetto universale da $\ReplicantFunctor$ in $(\Object_i)_{i \in \Class}$, dove $\ReplicantFunctor$ \`e il funtore replicante, e, fissato $i \in \Class$, \Define{proiezione}[in un prodotto][proiezione]\footnote{Queste proiezioni, pur essendo epimorfismi, non sono universali singolarmente, ma solo ``tutte insieme''. Non sono dunque universali nel senso della definizione \ref{def_proiezione}.} sul fattore $i$ l'$i$-esima componente della freccia universale;
		\item \Define{oggetto somma}[somma][oggetto] della famiglia $(\Object_i)_{i \in \Class}$, denotato $\DirectSum_{i \in \Class} \Object_i$, il duale dell'oggetto prodotto, vale a dire un oggetto universale da $(\Object_i)_{i \in \Class}$ in $\ReplicantFunctor$, dove $\ReplicantFunctor$ \`e il funtore replicante; fissato inoltre $i \in \Class$, chiamiamo \Define{immersione}[in una somma][immersione] nel fattore $i$ l'$i$-esima componente della freccia universale.
	\end{itemize}
\end{Definition}

\input{Fondamenti_della_matematica/CategoriaDegliInsiemi.tex}


	\chapter{Teoria delle relazioni}
	\section{Definizioni e teoremi di base.}\label{DefinizioniETeoremiDiBase}
\begin{Definition}
	Sia $\Class$ una classe. Si definisce \Define{relazione $n$-aria}[$n$-aria][relazione] ($n \in \mathbb{N}$) su $\Class$ una qualsiasi parte $\Relation$ di $\Class^n$. Il numero $n$ is chiama \Define{ariet\`a}[di una relazione][ariet\`a] della relazione $\Relation$; la classe $\Class$ si chiama \Define{sostegno}[di una relazione][sostegno] della reazione $\Relation$. Se $(x_0, ..., x_n) \in \Relation$ si dice che $x_0, ..., x_n$ \NIDefine{verificano} $\Relation$, affermazione che denotiamo anche $\Relation[x_0,...,x_n]$; nel caso $\Relation$ sia una relazione binaria denotiamo questa affermazione anche $x_0 \Relation x_1$.
\end{Definition}
\begin{Definition}
	Sia $\Relation$ una relazione binaria su $\Class$. Definiamo \Define{relazione opposta}[opposta][relazione] di $\Relation$ la relazione $\Dual{\Relation}$ definita da $(x,y) \in \Dual{\Relation}$ se e solo se $(y,x) \in \Relation$. Per denotare le relazioni opposte si usa spesso lo stesso simbolo della relazione originale, ribaltato orizzontalmente.
\end{Definition}
\begin{Definition}
	Sia $\Relation$ una relazione binaria su una classe $\Class$. $\Relation$ si dice
	\begin{itemize}
		\item \Define{riflessiva}[riflessiva][relazione] quando $\ForAll{x \in \Class}{x \Relation x}$;
		\item \Define{simmetrica}[simmetrica][relazione] quando $\Implies{x \Relation y}{y \Relation x}$;
		\item \Define{antisimmetrica}[antisimmetrica][relazione] quando $\Implies{\And{x \Relation y}{y \Relation x}}{x = y}$;
		\item \Define{transitiva}[transitiva][relazione] quando $\Implies{\And{x \Relation y}{y \Relation z}}{x \Relation y}$.
	\end{itemize}
	Se \`e definita una seconda relazione $\Dual{\Relation}$ sulla stessa classe $\Class$
\end{Definition}

\section{Grafi.}
\label{Relazioni_Grafi}
\subsection{Definizioni e teoremi di base.}
\label{DefinizioniETeoremiDiBaseGrafi}
\begin{Definition}
	Un \Define{grafo} \`e una coppia
  $\Graph = (\Nodes,\Edges)$ dove $\Nodes$ \`e
  una classe i cui elementi di chiamano
  \Define{nodi}[di un grafo][nodo],
  e $\Edges$ \`e una classe i cui elementi si dicono
  \Define{archi}[di un grafo][arco] tali che
	\begin{itemize}
		\item o $\Edges \subseteq \Nodes^2$ \`e una relazione, e allora il
      grafo si dice \Define{orientato}[orientato][grafo] o
     \Define{diretto}[diretto][grafo];
		\item o $\Edges$ \`e una classe i cui elementi sono tutti parti di
      $\Nodes$ di $2$ elementi, allora il grafo si
      dice \Define{non orientato}[non orientato][grafo] o
      \Define{non diretto}[non diretto][grafo]
	\end{itemize}
  Se $\Nodes$ \`e un insieme finito, allora, $\Graph$ si dice
  \Define{finito}[finito][grafo].
\end{Definition}
\begin{Definition}
  Dato un grafo non orientato $\Graph = (\Nodes,\Edges)$, chiamiamo
  \Define{grafo orientato associato}[orientato associato][grafo]
  a $\Graph$ il grafo $\Graph' = (\Nodes,\Edges')$ tale che
  $\lbrace \Node, \VarNode \rbrace \in \Edges$ se e solo se
  $\And{(\Node,\VarNode) \in \Edges'}{(\VarNode,\Node) \in \Edges'}$.
\end{Definition}
\begin{Definition}
	Sia $\Graph = (\Nodes,\Edges)$ un grafo. Se
  $\Node, \VarNode \in \Nodes$ sono nodi e $(\Node,\VarNode) \in \Edges$
  allora si dice che $\VarNode$ \`e \Define{collegato}[collegati][nodi]
  o \Define{connesso}[connessi][nodi] a $\Node$. Ora,
	\begin{itemize}
		\item se $\Graph$ \`e non orientato,
      \begin{itemize}
        \item chiamiamo \Define{vicinato} la
          parte di $\Nodes$ costituita dai nodi connessi a $\Node$, denotata
          $\GraphNeighborhood{\Node}$;
        \item \Define{vicino}[vicino][nodo] di $\Node$ un elemento di
          $\GraphNeighborhood{\Node}$;
        \item \Define{grado}[di un nodo in un grafo non orientato][grado] la
          cardinalit\`a di $\GraphNeighborhood{\Node}$, denotata
          $\GraphNodeDegree{\Node}$;
      \end{itemize}
		\item se $\Graph$ \`e orientato, chiamiamo
		  \begin{itemize}
			  \item \Define{vicinato uscente}[uscente][vicinato] la parte di
          $\Nodes$ costituita dai nodi connessi a $\Nodes$, denotato
        \item \Define{vicino uscente}[vicino uscente][nodo] di $\Node$ o
          \Define{figlio}[figlio][nodo] un elemento di
          $\GraphNeighborhoodOut{\Node}$;
        \item \Define{grado uscente}[di un nodo in un grafo orientato][grado]
          la cardinalit\`a di $\GraphNeighborhoodOut{\Node}$, denotata
          $\GraphNodeDegreeOut{\Node}$;
	  		\item \Define{vicinato entrante}[entrante][vicinato]
          $\GraphNeighborhoodIn{\Node}$ la parte di $\Nodes$ costituita
          da tutti i nodi a cui $\Node$ \`e connesso;
        \item \Define{vicino entrante}[vicino entrante][nodo] di $\Node$ o
          \Define{genitore}[genitore][nodo] un elemento di
          $\GraphNeighborhoodIn{\Node}$;
        \item \Define{grado entrante}[di un nodo in un grafo orientato][grado]
          la cardinalit\`a di $\GraphNeighborhoodIn{\Node}$, denotata
          $\GraphNodeDegreeOut{\Node}$.
		  \end{itemize}
	\end{itemize}
  La metafora genitore-figlio viene spesso estesa ad altri gradi di parentela
  con evidente significato.
\end{Definition}
\begin{Definition}
  Un nodo avente grado uscente nullo si chiama
  \Define{nodo pendente}[pendente][nodo]
  (\Define{\English{dangling node}}[\English{dangling}][\English{node}]).
\end{Definition}
\begin{Lemma}
 \TheoremName{Lemma della stretta di mano}[della stretta di mano][lemma]
  \TheoremName{Handshaking lemma}[handshaking][lemma]
  Sia $\Graph = (\Nodes,\Edges)$ un grafo finito:
  \begin{itemize}
    \item se $\Graph$ \`e orientato, allora
      $\sum_{\Node \in \Nodes} \GraphNodeDegreeOut{\Node}
      = \sum_{\Node \in \Nodes} \GraphNodeDegreeIn{\Node}
      = \Cardinality{\Edges}$;
    \item se $\Graph$ non \`e orientato, allora
      \begin{itemize}
        \item $\sum_{\Node \in \Nodes} \GraphNodeDegree{\Node}
          = 2\Cardinality{\Edges}$;
        \item la parte $\Nodes' \subseteq \Nodes$ di tutti i nodi
          di grado dispari ha cardinalit\`a pari.
      \end{itemize}
  \end{itemize}
\end{Lemma}
\Proof Supponiamo $\Graph$ orientato. Siano
$\Node, \VarNode \in \Nodes$ tali che
$(\Node,\VarNode) \in \Edges$:
$(\Node, \VarNode)$ \`e un arco entrante rispetto a
$\Node$ e uscente rispetto a $\VarNode$. Pertanto
$\sum_{\Node \in \Nodes} \GraphNodeDegreeOut{\Node}
= \sum_{\Node \in \Nodes} \GraphNodeDegreeIn{\Node}
= \Cardinality{\Edges}$.
\par Supponiamo ora $\Graph$ non orientato.
Considerando il grafo orientato associato a $\Graph$, abbiamo
$\sum_{\Node \in \Nodes} \GraphNodeDegree{\Node}
= 2 \Cardinality{\Edges}$.
\par Per l'ultima parte del teorema, basta osservare
che
$\sum_{\Node \in \Nodes} \GraphNodeDegree{\Node}
= \sum_{\Node \in \Nodes'} \GraphNodeDegree{\Node}
  + \sum_{\Node \in \Nodes \SetMin \Nodes '} \GraphNodeDegree{\Node}$,
\`e una quantit\`a pari, ma anche il termine
$\sum_{\Node \in \Nodes \SetMin \Nodes '} \GraphNodeDegree{\Node}$
\`e pari. \EndProof
\begin{Definition}
	Sia $\Graph = (\Nodes,\Edges)$ un grafo. Sia $(\Node_i)_{i \in n}$,
  con $n \in \NaturalExtended$ una successione di nodi tali che
  $\ForAll{i \in \NotZero{n}}
    {\lbrace \Node_{i - 1},\Node_i \rbrace \in \Edges}$.
  La successione di archi
  $(\lbrace \Node_{i - 1}, \Node_i \rbrace)_{i \in n}$
  si chiama
  \Define{cammino}[in un grafo][cammino]. Si dice che un cammino
  \Define{visita}[visitato][nodo] un nodo se il nodo appartiene ad uno
  degli archi del cammino. Inoltre, un cammino si dice
	\begin{itemize}
		\item \Define{semplice}[semplice][cammino] se tutti i suoi nodi sono
      distinti;
		\item un \Define{ciclo} se $n$ \`e finito e
      $\Node_0 = \Node_{n - 1}$.
	\end{itemize}
\end{Definition}
\begin{Definition}
	Un grafo $\Graph = (\Nodes,\Edges)$ si dice
  \Define{connesso}[connesso][grafo] quando per ogni
  $(\Node,\VarNode) \in \Nodes^2$ esiste un cammino di $\Graph$ avente
  $\Node$ come nodo iniziale e $\VarNode$ come nodo finale.
\end{Definition}
\begin{Definition}
	Dato un grafo $\Graph = (\Nodes,\Edges)$, i sottografi connessi
  massimali di $\Graph$ si chiamano
  \Define{componenti connesse}[connessa in un grafo][componente].
\end{Definition}
\begin{Definition}
	Un grafo si dice
	\begin{itemize}
		\item \Define{ciclico}[ciclico][grafo] se contiene un ciclo;
		\item \Define{aciclico}[aciclico][grafo] se non contiene nessun
      ciclo.
	\end{itemize}
\end{Definition}
\begin{Definition}
	Un cammino o un ciclo si dice
  \begin{itemize}
    \item \Define{cammino hamiltoniano}[hamiltoniano][cammino] o
      \Define{ciclo hamiltoniano}[hamiltoniano][ciclo]
      rispettivamente, se visita tutti i nodi di $\Graph$ una e sua sola
      volta;
    \item \Define{cammino euleriano}[euleriano][cammino] o
      \Define{ciclo euleriano}[euleriano][ciclo]
      rispettivamente, se comprende tutti gli archi di $\Graph$ una e
      sua sola volta.
  \end{itemize}
\end{Definition}

\subsection{Rappresentazione di Grafi.}
\label{Relazioni_RappresentazioneDiGrafi}
\begin{Definition}
  Sia $n \in \mathbb{N}$ e $\Graph$ un grafo finito orientato
  $\Graph = (n,\Edges)$ di $n$ nodi. Chiamiamo
  \Define{matrice di adiacenza}[di adiacenza][matrice]
  la famiglia
  $\AdjacencyMatrix
    = (\AdjacencyMatrix_i^j)_{(i,j) \in n \times n}$, dove
  $\AdjacencyMatrix_i^j =
    \begin{cases}
      0\text{ se }(i,j) \in \Edges,\\
      1\text{ se }(i,j) \in \Edges.
    \end{cases}$
  La matrice di adiacenza di un grafo finito non orientato si
  definisce analogamente per mezzo del grafo orientato associato.
\end{Definition}
\begin{Theorem}
  Sia $n \in \mathbb{N}$ e $\Graph$ un grafo finito orientato
  $\Graph = (n,\Edges)$ di $n$ nodi. La matrice di adiacenza
  contiene esattamente $2 \Cardinality{\Edges}$.
\end{Theorem}
\Proof Segue direttamente dal lemma della stretta di mano. \EndProof
\begin{Definition}
  Sia $n \in \mathbb{N}$ e $\Graph$ un grafo finito orientato
  $\Graph = (n,\Edges)$ di $n$ nodi.
  Chiamiamo \Define{lista di adiacenza}[di adiacienza][lista] un
  \Aarray\ di liste tale che l'$i$-esima ($i \in n$) lista dell'\Aarray\
  sia una lista di tutti i vicini (vicini uscenti nel caso orientato)
  del nodo $i$.
\end{Definition}

\subsection{Alberi.}
\label{Relazioni_Alberi}
\begin{Definition}
	Chiamiamo \Define{albero} un grafo $\Graph = (\Nodes, \Edges)$,
	\begin{itemize}
		\item connesso;
		\item aciclico.
	\end{itemize}
  Se $\Graph$ \`e
  \begin{itemize}
    \item non orientato, allora l'albero si dice
      \Define{non orientato}[non orientato][albero];
    \item orientato, allora l'albero si dice
      \Define{orientato}[orientato][albero].
  \end{itemize}
  Qualora non si specifichi se l'albero orientato, esso si intende non
orientato.
\end{Definition}
\begin{Definition}
	Se $\Graph$ \`e un grafo le cui componenti connesse sono tutti alberi,
  allora $\Graph$ si dice una \Define{foresta}.
\end{Definition}
\begin{Definition}
	Consideriamo un albero $\Tree$. Definiamo
  \Define{dimensione}[di un albero][dimensione]
  di $\Tree$ il numero di nodi di cui esso \`e composto.
\end{Definition}
\begin{Definition}
  Un \Define{albero radicato}[radicato][albero] \`e una coppia
  $\RootedTree = (\Graph,\Root)$, dove
  \begin{itemize}
    \item $\Graph = (\Nodes, \Edges)$ \`e un albero;
    \item $\Root \in \Nodes$ \`e un nodo, detto \Define{radice}.
  \end{itemize}
  Inoltre,
  \begin{itemize}
    \item se l'albero \`e orientato in modo tale che la radice sia antenata
      di tutti gli altri nodi, l'albero si chiama \Define{arborescenza};
    \item se l'albero \`e orientato in modo tale che la radice sia discendente
      di tutti gli altri nodi, l'albero si chiama \Define{antiarborescenza}.
  \end{itemize}
\end{Definition}
\begin{Definition}
  Sia $\Arborescence$ un'arborescenza. Definiamo
  \begin{itemize}
    \item \Define{foglia} un nodo di $\Arborescence$ senza figli;
		\item \Define{altezza}[di un nodo][altezza] di un nodo il massimo numero di
      nodi da percorrere per passare da una foglia esclusa al nodo in questione
      incluso;
		\item \Define{profondit\`a}[di un nodo][profondit\`a] o
      \Define{livello}[di un nodo][livello] di un nodo il numero di
      nodi da percorrere per passare dalla radice esclusa al nodo in questione
      incluso;
    \item \Define{profondit\`a}[di un arborescenza][profondit\`a] di
      $\Arborescence$ il massimo delle profondit\`a dei suoi nodi;
    \item per ogni $n \in \mathbb{N}$,
      \Define{livello $n$-esimo}[di un'arborescenza][livello] di $\Arborescence$
      l'insieme di tutti i suoi nodi di livello $n$.
	\end{itemize}
\end{Definition}
\begin{Definition}
  Sia $n \in \NotZero{\mathbb{N}}$. Chiamiamo \Define{albero $n$-ario}
  un'arborescenza in cui ogni nodo ha al pi\`u $n$ figli.
\end{Definition}
\begin{Definition}
  Sia $\BinaryTree$ un albero binari. Gli eventuali figli di un nodo si
  distinguono arbitrariamente in
  \Define{figlio sinistro}[sinistro][figlio]
  e
  \Define{figlio destro}[destro][figlio]; anche quando un nodo ha un solo
  figlio, si usa indicarlo come sinistro o destro.
  \par Siano inoltre $\Node$ e $\VarNode$ due nodi di $\BinaryTree$ di stesso
  livello. Diciamo che $\Node$ \`e
  \Define{pi\`u a sinistra}[pi\`u a sinistra][nodo]
  di $\Node$
  o che $\VarNode$ \`e
  \Define{pi\`u a destra}[pi\`u a destra][nodo]
  di $\Node$ quando si verifica una delle due condizioni seguenti:
  \begin{itemize}
    \item $\Node$ e $\VarNode$ sono figlio sinistro e destro rispettivamente di
      uno stesso nodo;
    \item $\Node$ ha un antenato pi\`u a sinistra rispetto all'antenato di
      $\VarNode$ di stesso livello.
  \end{itemize}
\end{Definition}
\begin{Definition}
  Sia $\Arborescence$ un albero $n$-ario ($n \in \NotZero{\mathbb{N}}$).
  $\Arborescence$ si dice
  \begin{itemize}
    \item \Define{quasi completo}[$n$-ario quasi completo][albero]
      quando
      \begin{itemize}
        \item tutti i nodi con $0$ figli hanno lo stesso livello
          $L \in \mathbb{N}$;
        \item tutti i nodi di livello minore di $L$ hanno esattamente $n$ figli;
      \end{itemize}
    \item \Define{completo}[$n$-ario completo][albero]
      quando
      \begin{itemize}
        \item \`e quasi completo;
        \item denotata $L$ la sua profondit\`a, tutti i nodi di livello $L$
          hanno $0$ figli;
      \end{itemize}
    \item \Define{bilanciato}[bilanciato][albero]
      quando per ogni nodo di $\Arborescence$, i livelli dei sottoalberi di quel
      nodo differiscono al pi\`u di $1$.
  \end{itemize}
\end{Definition}
\begin{Definition}
  Sia $\BinaryTree$ un albero binario quasi completo. $\BinaryTree$ si dice
  \begin{itemize}
    \item \Define{completo da sinistra}[$n$-ario completo da sinistra][albero]
      quando denotata $L$ la sua profondit\`a, se $\Node$ \`e un nodo di
      livello $L$, allora tutti i nodi pi\`u a sinistra del genitore
      di $\Node$ hanno esattamente due figli;
    \item \Define{completo da destra}[$n$-ario completo da destra][albero]
      quando denotata $L$ la sua profondit\`a, se $\Node$ \`e un nodo di
      livello $L$, allora tutti i nodi pi\`u a destra del genitore
      di $\Node$ hanno esattamente due figli.
  \end{itemize}
\end{Definition}

\subsection{Minimi alberi di ricoprimento.}
\label{Relazioni_MinimiAlberiDiRicoprimento}


\section{Relazioni d'ordine.}
\label{Relazioni_Ordini}
\subsection{Definizioni e teoremi di base.}\label{DefinizioniETeoremiDiBaseOrdini}
\begin{Definition}
	Una relazione binaria $\leq$ su una classe $\Class$ si chiama \Define{relazione d'ordine}[d'ordine][relazione] quando essa \`e
	\begin{itemize}
		\item riflessiva;
		\item antisimmetrica;
		\item transitiva.
	\end{itemize}
	Per ogni relazione d'ordine si usa introdurre un secondo simbolo ottenuto dal primo togliendovi il tratto pi\`u basso per significare oltre alla validit\`a della relazione anche la distinzione dei due elementi che la compongono: per esempio $x < y$ equivale a $\And{x \leq y}{x \neq y}$.
\end{Definition}
\begin{Definition}
	Una classe $\Poset$ su cui \`e definita una relazione d'ordine $\leq$ si chiama \Define{ordine parziale}[parziale][ordine].
\end{Definition}
\begin{Definition}
	Sia $\TotalOrder$ una classe su cui \`e definita una relazione d'ordine $\leq$ tale che $\ForAll{(x,y) \in \TotalOrder^2}{\Or{x \leq y}{y \leq x}}$. Diciamo che $\leq$ \`e una \Define{relazione d'ordine totale}[d'ordine totale][relazione] e che $\TotalOrder$ \`e un \Define{ordine totale}[totale][ordine].
\end{Definition}
\begin{Definition}
	Sia $f$ un'applicazione e siano definite due relazioni d'ordine, entrambe denotate con $\leq$, su $\Dom{f}$ e $\Cod{f}$. $f$ si dice
	\begin{itemize}
		\item \Define{crescente}[crescente][applicazione] quando $\Implies{x \leq y}{f(x) \leq f(y)}$;
		\item \Define{strettamente crescente}[strettamente crescente][applicazione] quando $\Implies{x < y}{f(x) < f(y)}$;
		\item \Define{decrescente}[decrescente][applicazione] quando $\Implies{x \leq y}{f(x) \geq f(y)}$;
		\item \Define{strettamente decrescente}[decrescente][applicazione] quando $\Implies{x < y}{f(x) > f(y)}$;
		\item \Define{monotona}[monotona][applicazione] se $f$ \`e crescente o se \`e decrescente;
		\item \Define{strettamente monotona}[strettamente monotona][applicazione] se $f$ \`e strettamente crescente o se \`e strettamente decrescente.
	\end{itemize}
\end{Definition}
\begin{Theorem}
	Un'applicazione $f$ decresente diventa crescente sostituendo una delle relazioni d'ordine con la relazione opposta.
\end{Theorem}
\Proof Segue direttamente dalle definizioni. \EndProof
\begin{Theorem}
	La composizione di applicazioni crescenti \`e crescente. 
\end{Theorem}
\Proof Siano $f$ e $g$ applicazioni crescenti e siano $x, y \in \Dom{f}$ tali che $x \leq y$. Abbiamo $f(x) \leq f(y)$ e quindi $(g \circ f)(x) = g(f(x)) \leq g(f(y)) = (g \circ f)(y)$. \EndProof
\begin{Theorem}
	Le applicazioni crescenti costituiscono una categoria.
\end{Theorem}
\Proof Basta verificare gli assiomi di categoria. Le identit\`a sono date dalle applicazioni identiche. \EndProof
\begin{Definition}
	Sia $\Poset$ un ordine parziale e sia $\Part \subseteq \Poset$. Definiamo
	\begin{itemize}
		\item \Define{maggiorante} di $\Part$ un elemento $\Majorant \in \Poset$ quando $\ForAll{x \in \Part}{x \leq \Majorant}$;
		\item \Define{minorante} di $\Part$ un elemento $\Minorant \in \Poset$ quando $\ForAll{x \in \Part}{x \geq \Minorant}$;
		\item \Define{elemento massimale}[massimale][elemento] di $\Part$ un elemento $\MaximalElement \in \Part$ tale che $\ForAll{x \in \Part}{\Implies{\MaximalElement \leq x}{\MaximalElement = x}}$;
		\item \Define{elemento minimmale}[minimale][elemento] di $\Part$ un elemento $\MinimalElement \in \Part$ tale che $\ForAll{x \in \Part}{\Implies{\MinimalElement \geq x}{\MinimalElement = x}}$;
		\item \Define{massimo} di $\Part$ un elemento $\Maximum \in \Part$ tale che $\ForAll{x \in \Part}{x \leq \Maximum}$;
		\item \Define{minimo} di $\Part$ un elemento $\Minimum \in \Part$ tale che $\ForAll{x \in \Part}{x \geq \Minimum}$.
	\end{itemize}
\end{Definition}
\begin{Theorem}
	Sia $\Poset$ un ordine parziale e sia $\Part \subseteq \Poset$. Se $\Part$ ammette massimo, allora esso \`e unico.
\end{Theorem}
\Proof Se $\Maximum_1, \Maximum_2 \in \Part$ sono entrambi massimi, allora $\Maximum_1 \leq \Maximum_2$ e $\Maximum_2 \leq \Maximum_1$, da cui $\Maximum_1 = \Maximum_2$. \EndProof
\begin{Corollary}
	Sia $\Poset$ un ordine parziale e sia $\Part \subseteq \Poset$. Se $\Part$ ammette minimo, allora esso \`e unico.
\end{Corollary}
\Proof Segue direttamente dal teorema precedente applicato all'ordine opposto di $\leq$. \EndProof
\begin{Theorem}
	Sia $\Poset$ un ordine parziale e sia $\Part \subseteq \Poset$. La classe dei maggioranti di $\Part$ ha minimo.
\end{Theorem}
\Proof
\begin{Corollary}
	Sia $\Poset$ un ordine parziale e sia $\Part \subseteq \Poset$. La classe dei minoranti ha massimo.
\end{Corollary}
\Proof Segue direttamente dal teorema precedente applicato all'ordine opposto di $\leq$. \EndProof
\begin{Definition}
	Sia $\Poset$ un ordine parziale e sia $\Part \subseteq \Poset$. Definiamo
	\begin{itemize}
		\item \Define{estremo superiore}[superiore][estremo] il minimo dei maggioranti di $\Part$;
		\item \Define{estremo inferiore}[inferiore][estremo] il massimo dei minoranti di $\Part$.
	\end{itemize}
\end{Definition}
\begin{Definition}
	Sia $\Poset$ un ordine parziale e sia $\Part \subseteq \Poset$. $\Part$ \`e una \Define{parte diretta}[diretta][parte] se per ogni $x, y \in \Part$ l'insieme $\lbrace x, y \rbrace$ ammette un maggiorante in $\Part$.
\end{Definition}
\begin{Definition}
	Sia $\Lattice$ un ordine parziale tale che per ogni $x, y \in \Lattice$ esistono gli estremi inferiori e superiori di $\lbrace x, y \rbrace$. $\Lattice$ si chiama \Define{reticolo}.
\end{Definition}
\begin{Definition}
	Siano
	\begin{itemize}
		\item $\Poset$ un ordine parziale;
		\item $N \in \NaturalExtended$;
		\item $(a_i)_{i \in N} \in \Poset^N$;
		\item $(b_i)_{i \in N - 1} \in \Poset^{N - 1}$.
	\end{itemize}
	Si dice che $(b_i)_{i \in N}$ \Define{separa}[separante][successione] $(a_i)_{i \in N}$ quando $\ForAll{i \in N - 1}{a_i \leq b_i \leq a_{i + 1}}$.
\end{Definition}

\input{Relazioni/Ordinali.tex}
\input{Relazioni/Cardinali.tex}

\section{Relazioni di equivalenza.}
\label{Relazioni_Equivalenze}
\subsection{Definizioni e teoremi di base.}\label{DefinizioniETeoremiDiBaseOrdini}
\begin{Definition}
	Una relazione $\EquivalenceRelation$ su una classe $\Class$ si dice \Define{relazione di equivalenza}[di equivalenza][relazione] se essa \`e
	\begin{itemize}
		\item riflessiva;
		\item simmetrica;
		\item transitiva.
	\end{itemize}
\end{Definition}



	\chapter{Fondamenti di algebra}
	\section{Magma e monoidi.}\label{Magma}
\begin{Definition}
	Un \Definition{magma} \`e il dato di
	\begin{itemize}
		\item una classe $\Magma$;
		\item un'operazione binaria interna $\cdot$.
	\end{itemize}
	Esso si dice inoltre
	\begin{itemize}
		\item \Define{moltiplicativo}[moltiplicativo][magma] quando l'operazione \`e denotata moltiplicativamente;
		\item \Define{additivo}[additivo][magma] quando l'operazione \`e denotata additivamente;
		\item \Define{assiociativo}[assiociativo][magma] quando l'operazione \`e associativa;
		\item \Define{commutativo}[commutativo][magma] quando l'operazione \`e commutativa;
		\item \Define{unitario}[unitario][magma] quando l'operazione ammette elemento neutro.
	\end{itemize}
	Infine chiamiamo \Define{monoide} un magma associativo e unitario.
	\par Un magma viene sempre denotato con lo stesso simbolo della classe che lo definisce, quando ci\`o non genera confusione.
\end{Definition}
\begin{Definition}
	Dati due magma $\Magma_1$, $\Magma_2$, un \Define{omomorfismo di magma}[di magma][omomorfismo] \`e un'applicazione $\Homomorphism: \Magma_1 \rightarrow \Magma_2$ tale che per ogni $x, y \in \Magma_1$ si abbia $\Homomorphism(xy) = \Homomorphism(x)\Homomorphism(y)$. La dicitura ``di magma'', utile per future distinzioni, viene spesso omessa quando questo non porta confusione, cos\`i come future analoghe diciture.
	\par Quando un'applicazione \`e un omomorfismo di magma si dice anche che essa \`e
	\begin{item}
		\item \Define{moltiplicativa}[moltiplicativa][applicazione] quando dominio e codominio sono due magma moltiplicativi;
		\item \Define{additiva}[additiva][applicazione] quando dominio e codominio sono due magma additivi.
	\end{item}
\end{Definition}
\begin{Theorem}
	Gli omomorfismi di magma costituiscono una categoria.
\end{Theorem}
\Proof Poich\'e gli omomorfismi sono applicazioni, le quali come sappiamo costituiscono la categoria $\CatClass$, \`e sufficiente provare che la composizione di due omomorfismi \`e un omomorfismo. Siano dunque $\Homomorphism_1$ e $\Homomorphism_2$ due omomorfismi e siano $x, y \in \Dom{\Homomorphism_1}$. Abbiamo $(\Homomorphism_2 \circ \Homomorphism_1)(xy) = \Homomorphism_2(\Homomorphism_1(xy)) = \Homomorphism_2(\Homomorphism_1(x) \Homomorphism_1(y)) = \Homomorphism_2(\Homomorphism_1(x)) \Homomorphism_2(\Homomorphism_1(y)) = (\Homomorphism_2 \circ \Homomorphism_1)(x) (\Homomorphism_2 \circ \Homomorphism_1)(y)$. \EndProof
\begin{Corollary}
	Costituiscono una categoria
	\begin{itemize}
		\item gli omomorfismi tra magma associativi;
		\item gli omomorfismi tra magma commutativi.
	\end{itemize}
\end{Corollary}
\Proof Conseguenza immediata del teorema precedente. \EndProof
\begin{Definition}
	Chiamiamo \Define{categoria dei magma}[dei magma][categoria] la categoria le cui frecce sono tutti gli omomorfismi di magma, d'ora in poi denotata $\CatMagma$. \`E inoltre definito il \Define{funtore dimenticante canonico dei magma}[dimenticante canonico dei magma] $\Forgetful: \CatMagma \rightarrow \CatClass$ che ad ogni magma associa la classe che lo definisce.
\end{Definition}
\begin{Definition}
	Chiamiamo
	\begin{itemize}
		\item \Define{categoria dei magma assiociativi}[dei magma associativi][categoria] la categoria le cui frecce sono tutti gli omomorfismi tra magma associativi, d'ora in poi denotata $\CatMagmaAss$;
		\item \Define{categoria dei magma commutativi}[dei magma commutativi][categoria] la categoria le cui frecce sono tutti gli omomorfismi tra magma commutativi, d'ora in poi denotata $\CatMagmaComm$.
	\end{itemize}
\end{Definition}
\begin{Definition}
	Dati due magma unitari $\Magma_1$, $\Magma_2$, un \Define{omomorfismo di magma unitari}[di magma unitari][omomorfismo] o \Define{omomorfismo unitario}[unitario][omomorfismo] \`e un omomorfismo $\UnitaryHomomorphism: \Magma_1 \rightarrow \Magma_2$ tale che $\UnitaryHomomorphism(1) = 1$.
\end{Definition}
\par Salvo quando diversamente specificato, un omomorfismo unitario sar\`a sempre supposto unitario.
\begin{Theorem}
	Gli omomorfismi di magma unitari costituiscono una categoria.
\end{Theorem}
\Proof Basta verificare che la composizione di due omomorfismi unitari $\UnitaryHomomorphism_1$ e $\UnitaryHomomorphism_2$ qualsiasi \`e un omomorfismo unitario. Sappiamo gi\`a che la composizione di omomorfismi \`e un omomorfismo. Abbiamo $(\UnitaryHomomorphism_2 \circ \UnitaryHomomorphism_1)(1) = \UnitaryHomomorphism_2(\UnitaryHomomorphism_1(1)) = \UnitaryHomomorphism_2(1) = 1$. \EndProof
\begin{Definition}
	Chiamiamo \Define{categoria dei magma unitari}[dei magma unitari][categoria] la categoria le cui frecce sono tutti gli omomorfismi di magma unitari, d'ora in poi denotata $\CatMagmaUn$.
\end{Definition}
\begin{Definition}
	Chiamiamo \Define{omomorfismo di monoidi} un omomorfismo unitario tra monoidi.
\end{Definition}
\begin{Theorem}
	Gli omomorfismi di monoidi costituiscono una categoria.
\end{Theorem}
\Proof Si tratta della categoria $\CatMagmaAss \cap \CatMagmaUn$. \EndProof
\begin{Definition}
	Chiamiamo \Define{categoria dei monoidi}[dei monoidi][categoria] la categoria le cui frecce sono tutti gli omomorfismi di monoidi, d'ora in poi denotata $\CatMonoid$.
\end{Definition}
\begin{Definition}
	Sia $(\Magma_i)_{i \in I}$ una famiglia di magma. Diciamo che i magma della famiglia sono \Define{compatibili}[compatibili][magma] quando, per ogni coppia $x, y \in \bigcap_{i \in I} \Magma_i$ si ha che $xy$ \`e lo stesso elemento sia secondo l'operazione definita in tutti i $\Magma_i$.
\end{Definition}
\begin{Theorem}
	L'inclusione canonica di un magma in un altro \`e un omomorfismo. Inoltre questo omomorfismo \`e unitario se i magma sono unitari.
\end{Theorem}
\Proof Segue direttamente dalle definizioni. \EndProof
\begin{Theorem}
	Sia $(\Magma_i)_{i \in I}$ una famiglia di magma compatibili. La classe $\bigcap_{i \in I} \Magma_i$, munita della struttura di magma indotta da ciascuno dei $\Magma_i$ \`e un sottomagma di tutti i $\Magma_i$.
\end{Theorem}
\Proof Immediata dalle definizioni. \EndProof
\begin{Definition}
	Dato un magma $\Magma$, una relazione d'equivalenza $\Equivalence$ si dice \Define{compatibile}[compatibile][relazione d'equivalenza] con la struttura di magma di $\Magma$ quando $\Equivalence$ \`e compatibile con l'operazione di $\Magma$.
\end{Definition}
\begin{Theorem}
	Siano dati un magma $\Magma$ e una relazione d'equivalenza $\Equivalence$ compatibile con la struttura di magma di $\Magma$. Esiste il magma quoziente $\Quotient{\Magma}{\Equivalence}$.
\end{Theorem}
\Proof Si tratta di dotare la classe $\Quotient{\Magma}{\Equivalence}$ di un'operazione binaria interna e dimostrare che la proiezione sul quoziente $\Projection: \Magma \rightarrow \Quotient{\Magma}{\Equivalence}$ \`e un omomorfismo.
\par Per la compatibilit\`a di $\Equivalence$ con la struttura di magma di $\Magma$, possiamo definire l'operazione binaria interna nel modo seguente: se $x, y \in \Magma$, allora $\EquivalenceClass{x}\EquivalenceClass{y} = \EquivalenceClass{xy}$. A questo punto segue direttamente per costruzione che $\Projection$ \`e un omomorfismo. \EndProof
\begin{Corollary}
	Se $\Magma$ \`e un magma associativo, allora ogni suo quoziente \`e anch'esso associativo.
\end{Corollary}
\Proof La dimostrazione del teorema precedente pu\`o essere ripetuta identicamente in $\CatMagmaAss$ anzich\'e in $\CatMagma$. \EndProof
\begin{Theorem}
	Siano dati un magma $\UnitaryMagma$ unitario e una relazione d'equivalenza $\Equivalence$ compatibile con la struttura di magma di $\Magma$. Il magma quoziente $\Quotient{\UnitaryMagma}{\Equivalence}$ \`e unitario.
\end{Theorem}
\Proof Basta provare che l'omomorfismo di proiezione sul quoziente $\Projection: \UnitaryMagma \rightarrow \Quotient{\UnitaryMagma}{\Equivalence}$ \`e unitario.
\par Proviamo che $\Projection{1}$ \`e l'elemento neutro di $\Quotient{\UnitaryMagma}{\Equivalence}$. Sia $\xi \in \Quotient{\UnitaryMagma}{\Equivalence}$: esiste $x \in \UnitaryMagma$ tale che $x \in \xi$. Abbiamo $\xi\Projection{1} = \Projection{x}\Projection{1} = \Projection{x \cdot 1} = \Projection{x} = \xi$ e, analogamente, $\Projection{1}\xi = \xi$. \EndProof
\begin{Corollary}
	Siano dati un monoide $\Monoid$ e una relazione d'equivalenza $\Equivalence$ compatibile con la struttura di monoide di $\Monoid$. Esiste il monoide quoziente $\Quotient{\Monoid}{\Equivalence}$.
\end{Corollary}
\Proof I teoremi precedenti provano che il magma quoziente $\Quotient{\Monoid}{\Equivalence}$ \`e un monoide. \EndProof
\begin{Theorem}
	Sia data una famiglia di magma $(\Magma_i)_{i \in I}$. Esiste il magma prodotto $\DirectProduct_{i \in I} \Magma_i$.
\end{Theorem}
\Proof Si tratta di dotare la classe $\DirectProduct_{i \in I} \Magma_i$ di un'operazione binaria interna e dimostrare che, per ogni $i \in I$, la proiezione sul fattore $i$ \`e un omomorfismo.
\par Dati $(x_i)_{i \in I}, (y_i)_{i \in I} \in \DirectProduct_{i \in I} \Magma_i$, definiamo $(x_i)_{i \in I}(y_i)_{i \in I} = (x_iy_i)_{i \in I}$. A questo punto segue direttamente per costruzione che le proiezioni sui fattori sono omomorfismi. \EndProof
\begin{Corollary}
	Se $(\Magma_i)_{i \in I}$ \`e una famiglia di magma tutti associativi, allora il loro prodotto \`e associativo.
\end{Corollary}
\Proof La dimostrazione del teorema precedente pu\`o essere ripetuta identicamente in $\CatMagmaAss$ anzich\'e in $\CatMagma$. \EndProof
\begin{Theorem}
	Sia data una famiglia di magma unitari $(\UnitaryMagma_i)_{i \in I}$. Il magma prodotto $\DirectProduct_{i \in I} \UnitaryMagma_i$ \`e unitario.
\end{Theorem}
\Proof Basta provare che, per ogni $i \in I$, la proieione sul fattore $i$ \`e unitaria.
\par Ci\`o segue immediatamente osservando che l'elemento neutro di $\DirectProduct_{i \in I} \Magma_i$ \`e $(1)_{i \in I}$. \EndProof
\begin{Corollary}
	Sia data una famiglia di monoidi $(\Monoid_i)_{i \in I}$. Esiste il monoide prodotto $\DirectProduct_{i \in I} \Monoid_i$.
\end{Corollary}
\Proof I teoremi precedenti provano che il magma prodotto $\DirectProduct_{i \in I} \Monoid_i$ \`e un monoide. \EndProof
\begin{Definition}
	Sia data una famiglia di magma unitari $(\UnitaryMagma_i)_{i \in I}$. Dato un qualsiasi elemento $(x_i)_{i \in I} \in \DirectProduct_{i \in I} \UnitaryMagma_i$ diciamo che esso ha \Define{supporto finito}[finito][supporto] se la sottoclasse $J = \lbrace j \in I | x_j \neq 1 \rbrace$ \`e una classe finita.
\end{Definition}
\begin{Theorem}
	Sia data una famiglia di magma unitari $(\UnitaryMagma_i)_{i \in I}$. Esiste il magma unitario somma $\DirectSum_{i \in I} \UnitaryMagma_i$.
\end{Theorem}
\Proof Definiamo $\DirectSum_{i \in I} \Magma_i$ come il sottomagma il cui sostegno \`e la classe di tutti gli elementi a supporto finito di $\DirectProduct_{i \in I} \Magma_i$. Si verifica facilmente che per, ogni $i \in I$, l'applicazione $\Injection_i: \Magma_i \rightarrow \DirectSum_{i \in I} \Magma_i$ che a $x \in \Magma_i$ associa $(y_j)_{j \in I}$, dove $y_j = \begin{cases} x\text{, se }i = j,\\ 1,\text{ altrimenti}.\end{cases}$ \`e un'immersione. \EndProof
\begin{Corollary}
	Se $(\Monoid_i)_{i \in I}$ \`e una famiglia di monoidi, allora la loro somma \`e un monoide.
\end{Corollary}
\Proof La dimostrazione del teorema precedente pu\`o essere ripetuta identicamente in $\CatMonoid$ anzich\'e in $\CatMagmaUn$. \EndProof
\begin{Theorem}
	Sia $\Homomorphism: \Magma_1 \rightarrow \Magma_2$ un omomorfismo tra i magma $\Magma_1$ e $\Magma_2$. Sono vere le seguenti affermazioni:
	\begin{itemize}
		\item $\Image{\Homomorphism}$ \`e un magma;
		\item se $\Magma_1$ \`e associativo, allora $\Image{\Homomorphism}$ \`e associativo;
		\item se $\Magma_1$ \`e commutativo, allora $\Image{\Homomorphism}$ \`e commutativo;
		\item se $\Magma_1$ \`e unitario, allora $\Image{\Homomorphism}$ \`e unitario;
		\item se $\Magma_1$ \`e un monoide, allora $\Image{\Homomorphism}$ \`e un monoide.
	\end{itemize}
\end{Theorem}
\Proof La prima affermazione segue direttamente dalle definizioni.
\par Supponiamo $\Magma_1$ associativo. Per ogni $x,y,z \in \Magma_1$, abbiamo $(\Homomorphism(x)\Homomorphism(y))\Homomorphism(z) = \Homomorphism(xy)\Homomorphism(z) = \Homomorphism((xy)z) = \Homomorphism(x(yz)) = \Homomorphism(x)\Homomorphism(yz) = \Homomorphism(x)(\Homomorphism(y)\Homomorphism(z))$.
\par Supponiamo $\Magma_1$ commutativo. Per ogni $x, y \in \Magma_2$, abbiamo $\Homomorphism(x)\Homomorphism(y) = \Homomorphism(xy) = \Homomorphism(yx) = \Homomorphism(y)\Homomorphism(x)$.
\par Supponiamo $\Magma_1$ unitario. Per ogni $x \in \Magma_1$, abbiamo $\Homomorphism(x)\Homomorphism(1) = \Homomorphism(x)$ e $\Homomorphism(1)\Homomorphism(x) = \Homomorphism(x)$, dunque $\Homomorphism(1)$ \`e elemento neutro in $\Magma_2$.
\par L'ultima affermazione segue dal fatto che un monoide \`e un magma associativo e unitario.\EndProof
\begin{Corollary}
	Sia $\Homomorphism: \Monoid_1 \rightarrow \Monoid_2$ un omomorfismo tra i monoidi $\Monoid_1$ e $\Monoid_2$. $\Image{\Homomorphism}$ \`e un monoide.
\end{Corollary}
\Proof Segue dal teorema precedente e dal fatto che un omomorfismo di monoidi \`e un omomorfismo di magma. \EndProof
\begin{Definition}
	Sia $\Homomorphism: \Monoid_1 \rightarrow \Monoid_2$ un omomorfismo tra magma unitari $\Magma_1$ e $\Magma_2$. Definiamo \Define{nucleo} di $\Homomorphism$, denotato $\Kernel{\Homomorphism}$ la classe $\Homomorphism^{-1}(1)$.
\end{Definition}
\begin{Theorem}
	Sia $\Homomorphism: \Magma_1 \rightarrow \Magma_2$ un omomorfismo tra i magma unitari $\Magma_1$ e $\Magma_2$. $\Kernel{\Homomorphism}$ \`e un sottomagma unitario.
\end{Theorem}
\Proof Abbiamo
\begin{itemize}
	\item se $x, y \in \Magma_1$ sono tali che $\Homomorphism(x) = \Homomorphism(y) = 1$, allora $\Homomorphism(xy) = 1$;
	\item $\Homomorphism(1) = 1$ per definizione di omomorfismo unitario. \EndProof
\end{itemize}
\begin{Corollary}
	Sia $\Homomorphism: \Monoid_1 \rightarrow \Monoid_2$ un omomorfismo tra i monoidi $\Monoid_1$ e $\Monoid_2$. $\Kernel{\Homomorphism}$ \`e un sottomonoide.
\end{Corollary}
\Proof I monoidi sono magma unitari e gli omomorfismi di monoidi sono omomorfismi di magma unitari. \EndProof
\begin{Theorem}
	Siano $\Monoid$ un monoide e $x, y \in \Monoid$ entrambi invertibili. Allora $xy$ \`e invertibile e il suo inverso \`e $y^{-1}x^{-1}$.
\end{Theorem}
\Proof $(xy)(y^{-1}x^{-1}) = x(yy^{-1})x^{-1} = xx^{-1} = 1$. \EndProof
\begin{Theorem}
	Siano $\Magma$ un magma unitario e $x \in \Magma$. Per qualsiasi omomorfismo di magma (non necessariamente unitari) $\Homomorphism$ tale che $\Dom{\Homomorphism} = \Magma$, se $x$ \`e invertibile, allora
	\begin{itemize}
		\item $f(x)$ \`e invertibile;
		\item $f(x)^{-1} = f(x^{-1})$.
	\end{itemize}
\end{Theorem}
\Proof L'esistenza di un elemento neutro in $\Cod{\Homomorphism}$ segue dal fatto che l'immagine di un magma unitario tramite un omomorfismo di magma \`e un magma unitario. Inoltre $f(x)f(x^{-1}) = f(xx^{-1}) = f(1) = 1$ e $f(x^{-1})f(x) = f(x^{-1}x) = f(1) = 1$. \EndProof
\begin{Theorem}
	Sia $\Magma$ un magma e sia $(x_i)_{i \in I}$ una famiglia di
	elementi di $\Magma$. Il pi\`u piccolo sottomagma $\Magma_0$ di
	$\Magma$ \`e il magma di tutti i prodotti di elementi di
	$(x_i)_{i \in I}$.
\end{Theorem}
\Proof I prodotti in questione costituiscono effettivamente un sottomagma
di $\Magma$ ed ogni sottomagma di $\Magma$ che contiene gli elementi di
$(x_i)_{i \in I}$ deve necessariamente contenere tutti i prodotti.
\EndProof
\begin{Definition}
	Sia $\Magma$ un magma e sia $(x_i)_{i \in I}$ una famiglia di
	elementi di $\Magma$. Il pi\`u piccolo sottomagma $\Magma_0$ di
	$\Magma$ tale che $\ForAll{i \in I}{x_i \in \Magma_0}$ si chiama
	\Define{magma generato}[generato][magma] da
	$(x_i)_{i \in \Magma_0}$.
\end{Definition}
\begin{Theorem}
	Sia $\Homomorphism: \Magma_0 \rightarrow \Magma_1$ un omomorfismo
	biettivo tra i magma $\Magma_0$ e $\Magma_1$. $\Homomorphism^{-1}$
	\`e un omomorfismo.
\end{Theorem}
\Proof Siano $a, b \in \Magma_1$. Abbiamo
$\Homomorphism(\Homomorphism^{-1}(a)\Homomorphism^{-1}(b)) =
\Homomorphism(\Homomorphism^{-1}(a))\Homomorphism(\Homomorphism^{-1}(b)) =
ab =
\Homomorphism(\Homomorphism^{-1}(ab))$.
Per l'iniettivit\`a di $\Homomorphism$, abbiamo
$\Homomorphism^{1}(a)\Homomorphism^{-1}(b) = \Homomorphism^{-1}(ab)$.
\EndProof
\begin{Definition}
	Chiamiamo \Define{notazione multi-indice}[multi-indice][notazione] la notazione che definisce le seguenti operazioni e relazioni sulle $n$-uple $\alpha, \beta \in \mathbb{N}^n$, dove $n \in \NotZero{\mathbb{N}}$:
	\begin{itemize}
		\item $\alpha + \beta = (\alpha_i + \beta_i)_{i \in n}$;
		\item $\alpha - \beta = (\alpha_i - \beta_i)_{i \in n}$;
		\item $\Coimplies{\alpha \leq \beta}{\ForAll{i \in n}{\alpha_i \leq \beta_i}}$;
		\item $\NaturalDegree{\alpha} = \sum_{i \in n} \alpha_i$;
		\item $\alpha! = \prod_{i \in n}\alpha_i!$;
		\item $\binom{\alpha}{\beta} = \frac{\alpha!}{(\alpha - \beta)!\beta!}$;
		\item posto $x = (x_i)_{i \in n}$, dove $x_i$ sono elementi di un magma, $x^\alpha = \prod_{i \in n} x_i^{\alpha_i}$.
	\end{itemize}
\end{Definition}

\section{Azioni.}\label{Azioni}
\begin{Definition}
	Siano $\Class$ e $\Operators$ due classi. Un'\Define{azione} di $\Operators$ su $\Class$ \`e un'applicazione $\Action: \Operators \rightarrow \Class^\Class$: diciamo in tal caso che $\Operators$ \NIDefine{agisce} su $\Class$ tramite $\Action$.
\end{Definition}
\begin{Theorem}
	Un'azione $\Action: \Operators \rightarrow \Class^\Class$ equivale ad un'operazione binaria $\cdot: \Operators \times \Class \rightarrow \Class$. Pi\`u precisamente,
	\begin{itemize}
		\item data $\Action: \Operators \rightarrow \Class^\Class$ esiste una e una sola operazione $\cdot: \Operators \times \Class \rightarrow \Class$, detta \Define{operazione indotta da $\Action$}[indotta da un'azione][operazione] tale che per ogni $\Operator \in \Operators$ e $x \in \Class$ si abbia $\Operator \cdot x = (\Action(\Operator))(x)$;
		\item data $\cdot: \Operators \times \Class \rightarrow \Class$ esiste una e una sola azione $\Action: \Operators \rightarrow \Class^\Class$, detta \Define{azione indotta da $\cdot$}[indotta da un'operazione][azione], tale che per ogni $\Operator \in \Operators$ e $x \in \Class$ si abbia $\Operator \cdot x = (\Action(\Operator))(x)$.
	\end{itemize}
\end{Theorem}
\Proof Segue immediatamente dalle definizioni. \EndProof
\begin{Definition}
	Sia $\Action: \Operators \rightarrow \Class^\Class$ un'azione tale che $\Operators$ sia un magma. Allora $\Action$ si dice un'\Define{azione sinistra}[sinistra][azione] se per ogni $\Operator_1, \Operator_2 \in \Action$ e $x \in \Class$ vale l'uguaglianza $\Action(\Operator_1\Operator_2)(x) = (\Action(\Operator_1) \circ \Action(\Operator_2))(x)$.
	\par Nel caso di un'azione sinistra si usa spesso identificare l'azione $\Action$ con l'operazione equivalente descritta dal teorema precedente, mentre nel caso di un'azione destra si usa spesso identificare $\Action$ con l'operazione opposta equivalente descritta dal teorema precedente, con evidenti vantaggi notazionali.
\end{Definition}
\begin{Definition}
	Sia dato un magma $\Operators$. $\Operators$ agisce su se stesso sia a sinistra che a destra tramite
	\begin{itemize}
		\item l'azione di \Define{traslazione a sinistra}[a sinistra][traslazione] indotta dall'operazione $\cdot$ del $\Operators$;
		\item l'azione di \Define{traslazione a destra}[a destra][traslazione] indotta dal duale dell'operazione $\cdot$ del $\Operators$.
	\end{itemize}
\end{Definition}
\begin{Definition}
	Sia data un'azione $\Action: \Operators \rightarrow \Class^\Class$ e un'elemento $x \in \Class$. Definiamo
	\begin{itemize}
		\item lo \Define{stabilizzatore} di $x$ rispetto ad $\Action$ come la classe $\Stabilizer{x}[\Action] = \lbrace \Operator \in \Operators | \Operator \cdot x = x \rbrace$;
		\item l'\Define{orbita} di $x$ rispetto ad $\Action$ come la classe $\Orbit{x}[\Action] = \lbrace y \in \Class | \Exists{\Operator \in \Operators}{\Operator \cdot x = y} \rbrace$.
	\end{itemize}
\end{Definition}
\begin{Definition}
	Sia data un'azione
	$\Action: \Operators \rightarrow \Class^\Class$.
	L'azione si dice
	\begin{itemize}
		\item
		\Define{transitiva}[transitiva][azione]
		quando
		$\ForAll{x \in \Class}
		{\ForAll{y \in \Class}
		{\Exists{\Operator \in \Operators}
		{y = \Operator \cdot x}}}$;
		\item
		\Define{semplicemente transitiva}[semplicemente transitiva][azione]
		quando
		$\ForAll{x \in \Class}
		{\ForAll{y \in \Class}
		{\ExistsOne{\Operator \in \Operators}
		{y = \Operator \cdot x}}}$.
	\end{itemize}
\end{Definition}
\begin{Definition}
	Siano $\Operators$ un magma e $\Class$ una classe qualsiasi. Un'\Define{azione di magma}[di magma][azione] di $\Operators$ su $\Class$ \`e un'azione del sostegno di $\Operators$ su $\Class$ che \`e anche un omomorfismo di $\Operators$ con $\Class^\Class$. Quando un magma $\Operators$ \NIDefine{agisce} su una classe $\Class$ intendiamo sempre che si tratta di un'azione di magma, salvo avvisi contrari.
	Diciamo anche che l'azione di $\Operators$ \`e
	\Define{distributiva}[distributiva][azione].
\end{Definition}
\begin{Definition}
	Siano $\Operators$ un magma unitario e $\Class$ una classe qualsiasi. Un'\Define{azione di magma unitario}[di magma unitario][azione] di $\Operators$ su $\Class$ \`e un'azione del sostegno di $\Operators$ su $\Class$ che \`e anche un omomorfismo di $\Operators$ con $\Class^\Class$. Quando un magma unitario $\Operators$ \NIDefine{agisce} su una classe $\Class$ intendiamo sempre che si tratta di un'azione di magma unitario, salvo avvisi contrari.
\end{Definition}
\begin{Definition}
	Siano $\Operators$ un monoide e $\Class$ una classe qualsiasi. Un'\Define{azione di monoide}[di monoide][azione] di $\Operators$ su $\Class$ \`e un'azione del sostegno di $\Operators$ su $\Class$ che \`e anche un omomorfismo di $\Operators$ con $\Class^\Class$. Quando un monoide $\Operators$ \NIDefine{agisce} su una classe $\Class$ intendiamo sempre che si tratta di un'azione di monoide, salvo avvisi contrari.
\end{Definition}
\begin{Definition}
	Sia data un'azione $\Action: \Operators \rightarrow \Class^\Class$ e una classe $\Operators_0 \subseteq \Operators$. Definiamo la \Define{classe fissa}[fissa][classe] di $\Class$ sotto l'azione di $\Operators_0$, denotata $\Fixed{\Class}{\Operators_0}$, la classe di tutti gli elementi di $\Class$ che rimangono fissi sotto l'azione di $\Operators_0$. Se $\Operators_0$ \`e un singoletto si usa anche scriverne l'unico elemento $\Operator$ nella notazione di classe fissa: $\Fixed{\Class}{\Operator}$.
\end{Definition}
\begin{Definition}
	Sia data un'azione di magma unitari
	$\Action: \Operators \rightarrow \Class^\Class$.
	L'azione si dice
	\begin{itemize}
		\item
		\Define{fedele}[fedele][azione]
		quando
		$\Implies{\And{\Operator \in \Operators}
		{\Operator \neq 1}}
		{\Fixed{\Class}{\Operator} \neq \Class}$;
		\item
		\Define{libera}[libera][azione]
		quando
		$\Implies{\And{\Operator \in \Operators}
		{\Operator \neq 1}}
		{\Fixed{\Class}{\Operator} = \emptyset}$.
	\end{itemize}
\end{Definition}


	\part{Algebra}
	\chapter{Teoria dei gruppi}
	\section{Definizioni e teoremi di base.}\label{DefinizioniETeoremiDiBase}
\begin{Definition}
	Si chiama \Define{gruppo} un monoide $\Group$ in cui tutti gli elementi sono invertibili. Un \Define{omomorfismo di gruppi}[di gruppi][omomorfismo] \`e un omomorfismo di monoidi dove dominio e codominio sono entrambi gruppi. L'\Define{ordine}[di un gruppo][ordine] di $\Group$, denotato $\GroupOrder{\Group}$, \`e la cardinalit\`a del suo sostegno. Se $\Subgroup$ \`e un sottogruppo di $\Group$, denotiamo tale relazione con $\Subgroup \IsSubgroup \Group$.
\end{Definition}
\begin{Theorem}
	Gli omomorfismi di gruppi costituiscono una categoria.
\end{Theorem}
\Proof Segue direttamente dall'esistenza della categoria $\CatMonoid$. \EndProof
\begin{Definition}
	Chiamiamo \Define{gruppo abeliano}[abeliano][gruppo] o \Define{commutativo}[commutativo][gruppo] un gruppo la cui operazione \`e commutativa.
\end{Definition}
\begin{Definition}
	Chiamiamo
	\begin{itemize}
		\item \Define{categoria dei gruppi}[dei gruppi][categoria] la categoria le cui frecce sono tutti gli omomorfismi di gruppi, d'ora in poi denotata $\CatGroup$;
		\item \Define{categoria dei gruppi abeliani}[dei gruppi abeliani][categoria] la categoria $\CatGroup \cap \CatMagmaComm$, d'ora in poi denotata $\CatAbelian$.
	\end{itemize}
\end{Definition}
\begin{Theorem}
	Esistono i seguenti gruppi:
	\begin{itemize}
		\item dati un gruppo $\Group$ e una relazione d'equivalenza $\Equivalence$ compatibile con la struttura di gruppo di $\Group$, il gruppo quoziente $\Quotient{\Group}{\Equivalence}$;
		\item data una famiglia di gruppi $(\Group_i)_{i \in I}$, i gruppi $\DirectProduct_{i \in I} \Group_i$ e $\DirectSum_{i \in I} \Group_i$.
	\end{itemize}
\end{Theorem}
\Proof Occorre verificare che i monoidi $\Quotient{\Group}{\Equivalence}$, $\DirectProduct_{i \in I} \Group_i$ e $\DirectSum_{i \in I}$ sono gruppi, vale a dire che ogni loro elemento \`e invertibile:
	\begin{itemize}
		\item se $\EquivalenceClass{x} \in \Quotient{\Group}{\Equivalence}$, osserviamo che $\EquivalenceClass{x}\EquivalenceClass{x^-1} = \EquivalenceClass{xx^-1} = \EquivalenceClass{1} = 1$;
		\item se $(x_i)_{i \in I} \in \DirectProduct_{i \in I} \Group_i$, allora $(x_i)_{i \in I} (x_i^{-1})_{i \in I} = (x_ix_i^{-1})_{i \in I} = (1)_{i \in I} = 1$;
		\item se $(x_i)_{i \in I} \in \DirectProduct_{i \in I}$ ha supporto finito, allora ha supporto finito anche $(x_i^{-1})_{i \in I}$. \EndProof
	\end{itemize}
\begin{Theorem}
	Sia $\Group$ un gruppo. Dato $\Subgroup \subseteq \Group$, $\Subgroup$ \`e un sottogruppo di $\Group$ se e solo se $\Subgroup \neq \emptyset$ e, per ogni $a,b \in \Subgroup$, abbiamo $ab^{-1} \in \Subgroup$.
\end{Theorem}
\Proof Occorre verificare i seguenti fatti:
\begin{itemize}
	\item $1 \in \Subgroup$: sia $a \in \Subgroup$, allora $aa^{-1} = 1 \in \Subgroup$;
	\item se $a \in \Subgroup$, allora $a^{-1} \in \Subgroup$: $a^{-1} = 1 \cdot a^{-1} \in \Subgroup$;
	\item se $a, b \in \Subgroup$, allora $ab \in \Subgroup$: per il punto precedente, $b^{-1} \in \Subgroup$, dunque $ab = a(b^{-1})^{-1} \in \Subgroup$. \EndProof
\end{itemize}
\par D'ora in poi, per indicare che $\Subgroup$ \`e un sottogruppo di $\Group$ useremo la notazione $\Subgroup \IsSubgroup \Group$.
\begin{Theorem}
	Siano $\Magma$ un magma, $\Group$ un gruppo e $\Homomorphism: \Group \rightarrow \Magma$ un omomorfismo di magma. $\Image{\Homomorphism}$ \`e un gruppo.
\end{Theorem}
\Proof Segue dal fatto che l'immagine di un monoide \`e un monoide e che l'immagine di un elemento invertibile \`e invertibile. \EndProof
\begin{Corollary}
	Siano $\Group_1, \Group_2$ gruppi e $\Homomorphism: \Group_1 \rightarrow \Group_2$ un omomorfismo. Abbiamo $\Image{\Homomorphism} \IsSubgroup \Group_2$.
\end{Corollary}
\Proof Un omomorfismo di gruppi \`e un omomorfismo di magma. \EndProof
\begin{Theorem}
	Siano $\Group_1, \Group_2$ gruppi e $\Homomorphism: \Group_1 \rightarrow \Group_2$ un omomorfismo. Abbiamo $\Kernel{\Homomorphism} \IsSubgroup \Group_1$.
\end{Theorem}
\Proof Un omomorfismo di gruppi \`e un omomorfismo di monoidi, quindi $\Kernel{\Homomorphism}$ \`e un sottomonoide di $\Group_1$. Inoltre, se $x \in \Kernel{\Homomorphism}$, allora $\Homomorphism(x^{-1}) = \Homomorphism(x)^{-1} = 1^{-1} = 1$. \EndProof
\begin{Theorem}
	Siano $\Group_1, \Group_2$ gruppi e
	$\Homomorphism: \Group_1 \rightarrow \Group_2$ un omomorfismo.
	$\Homomorphism$ \`e iniettivo se e solo se
	$\Kernel{\Homomorphism} = 1$.
\end{Theorem}
\Proof
Segue direttamente dalle definizioni di iniettit\`a e di omomorfismo che
se $\Homomorphism$ \`e iniettivo, allora
$\Kernel{\Homomorphism} = 1$.
\par
Assumiamo $\Kernel{\Homomorphism} = 1$ e siano $x, y \in \Group_1$ tali che
$\Homomorphism{x} = \Homomorphism{y}$.
Ne consegue $\Homomorphism{x}\Homomorphism{y}^{-1} = 1$.
Ma abbiamo anche $\Homomorphism{x}\Homomorphism{y}^{-1} =
\Homomorphism{xy^{-1}}$.
Quindi $xy^{-1} = 1$, da cui la tesi.
\EndProof
\begin{Theorem}
	Dato un oggetto $\Object$ in una categoria $\Category$ qualsiasi, la classe di tutti i suoi automorfismi costituisce un gruppo.
\end{Theorem}
\Proof Segue direttamente dalle definizioni. \EndProof
\begin{Definition}
	Dato un oggetto $\Object$ in una categoria $\Category$ qualsiasi, il gruppo di tutti i suoi automorfismi viene chiamato \Define{gruppo degli automorfismi di $\Object$}[degli automorfismi][gruppo] e denotato $\Automorphisms{\Object}$.
\end{Definition}

\section{Azioni di gruppi.}\label{Azioni}
\begin{Definition}
	Siano $\Operators$ un gruppo e $\Class$ una classe qualsiasi. Un'\Define{azione di gruppo}[di gruppo][azione] di $\Operators$ su $\Class$ \`e un'azione del sostegno di $\Operators$ su $\Class$ che \`e anche un omomorfismo di $\Operators$ con $\Class^\Class$. Quando un gruppo $\Operators$ \NIDefine{agisce} su una classe $\Class$ intendiamo sempre che si tratta di un'azione di gruppo, salvo avvisi contrari.
\end{Definition}
\begin{Theorem}
	Un'azione di gruppo \`e un'azione sinistra.
\end{Theorem}
\Proof Segue direttamente dalle definizioni. \EndProof
\begin{Theorem}
	Sia $\Operators$ un gruppo che agisce su una classe $\Class$ tramite $\Action$. Abbiamo $\Image{\Action} \subseteq \Automorphisms{\Class}$.
\end{Theorem}
\Proof Sia $\Homomorphism \in \Image{\Action}$. Sia $\Operator \in \Operators$ tale che $\Action{\Operator} = \Homomorphism$. Abbiamo $\Homomorphism\Action(\Operator^{-1}) = \Action{\Operator}\Action(\Operator{-1}) = \Action(\Operator\Operator^{-1}) = \Action(1) = 1$, dunque $\Homomorphism$ \`e invertibile. \EndProof
\begin{Theorem}
	Sia $\Operators$ un gruppo che agisce su una classe $\Class$ tramite $\Action$. Per ogni $x \in \Class$, abbiamo $\Stabilizer{x} \IsSubgroup \Operators$.
\end{Theorem}
\Proof Siano $\Operator_1, \Operator_2 \in \Stabilizer{x}$. Abbiamo $(\Operator_1\Operator_2^{-1})(x) = x$. \EndProof
\begin{Theorem}\label{action_equivalence}
	Sia $\Operators$ un gruppo. Un'azione di gruppo $\Action: \Operators \rightarrow \Class^\Class$ determina una relazione d'equivalenza $\Equivalence$ su $\Class$ nel modo seguente: per $x, y \in \Class$ si ha $\Coimplies{\Equivalence[x][y]}{\Exists{\Operator \in \Operators}{\Operator \cdot x = y}}$. Le classi d'equivalenza di $\Equivalence$ coincidono con le orbite.
\end{Theorem}
\Proof Occorre provare che $\Equivalence$ \`e una relazione d'equivalenza. In effetti, abbiamo
\begin{itemize}
	\item per ogni $x \in \Class$, $1 \cdot x = x$ perch\'e $\Action$ \`e un'azione di gruppo;
	\item per ogni $x, y \in \Class$ e $\Operator \in \Operators$ tali che $\Operator \cdot x = y$, $\Operator^{-1} \cdot y = \Operator^{-1} (\Operator x) = (\Operator^{-1} \Operator) x = 1 \cdot x = x$;
	\item per ogni $x, y, z \in \Class$ e $\Operator_1, \Operator_2 \in \Operators$ tali che $\Operator_1 \cdot x = y$ e $\Operator_2 \cdot y = z$, $(\Operator_2\Operator_1) \cdot x = \Operator_2 (\Operator_1 x) = z$.
\end{itemize}
\par La concidenza di orbite e classi d'equivalenza segue direttamente dalle definizioni. \EndProof
\begin{Definition}
	Sia $\Group$ un gruppo. L'azione di $\Group$ su se stesso che a $g \in \Group$ associa l'automorfismo $x \mapsto gxg^{-1}$, detto \Define{automorfismo interno}[interno][automorfismo] associato a $g$, si chiama \Define{azione di coniugio}[di coniugio][azione] tramite $g$, e se $y \in \Orbit{x}$ si dice che $x$ e $y$ sono \Define{coniugati}[coniugati][elementi]. L'insieme di tutti gli automorfismi interni si denota $\InnerAutomorphisms$.
\end{Definition}
\begin{Theorem}\label{action_conjugation}
	L'azione di coniugio di un gruppo $\Group$ su se stesso \`e un azione di gruppo.
\end{Theorem}
\Proof Per ogni $g, h, x \in \Group$, abbiamo $h(gxg^{-1})h^{-1} = (hg)x(hg)^{-1}$. \EndProof
\begin{Corollary}
	Dato un gruppo $\Group$, gli automorfismi interni di $\Group$ costituiscono un gruppo.
\end{Corollary}
\Proof Indicata con $\Action$ l'azione di coniugio di $\Group$, abbiamo $\InnerAutomorphisms{\Group} = \Image(\Action) \IsSubgroup \Automorphisms{\Group}$. \EndProof
\begin{Corollary}
	Il coniugio \`e una relazione d'equivalenza.
\end{Corollary}
\Proof Segue direttamente dai teoremi \ref{action_conjugation} e \ref{action_equivalence}. \EndProof
\begin{Lemma}
	\TheoremName{Lemma di Burnside}[di Burnside][lemma] Sia $\Operators$ un gruppo che agisce su una classe $\Class$. Indicata con $n$ la cardinalit\`a dell'insieme delle orbite determinate dall'azione di $\Operators$ su $\Class$ abbiamo $n\GroupOrder{\Operators} = \sum_{\Operator \in \Operators} \Cardinality{\Fixed{\Class}{\Operator}}$.
\end{Lemma}
\Proof Osserviamo che $\sum_{\Operator \in \Operators} \Cardinality{\Class}^{\Operator} = \Cardinality{\lbrace (\Operator,x) \in \Operators \times \Class | \Operator x = x \rbrace} = \sum_{x \in \Class} \Order{\Stabilizer{x}}$.
\Proof Sia $R \subseteq \Class$ una classe contenente esattamente un rappresentante di ogni orbita. Abbiamo $n\GroupOrder{\Operators} = \GroupOrder{\Operators}\Cardinality{R} = \sum_{r \in R} \GroupOrder{\Operators} = \sum_{r \in R} \GroupOrder{\Stabilizer{r}}\Orbit{r} = \sum_{x \in \Class} \GroupOrder{\Stabilizer{x}} = \Cardinality{\lbrace (\Operator,x) \ in \Operators \times \Class | \Operator x = x \rbrace} = \sum_{\Operator \in \Operators} \Cardinality{\Fixed{\Class}{\Operator}}$. \EndProof

\section{Sottogruppi normali e classi laterali.}\label{SottogruppiNormaliEQuozienti}
\begin{Definition}
	Sia $\Group$ un gruppo e $\NormalSubgroup \IsSubgroup \Group$. Se, per ogni automorfismo interno $\InnerAutomorphism$, abbiamo $\InnerAutomorphism(\NormalSubgroup) \subseteq \NormalSubgroup$, allora si dice che $\NormalSubgroup$ \`e un sottogruppo \Define{normale}[normale][sottogruppo] di $\Group$ e il fatto viene denotato con $\NormalSubgroup \IsNormalSubgroup \Group$.
\end{Definition}
\begin{Theorem}
	Siano $\Group$ un gruppo, $\InnerAutomorphism$ un automorfismo interno di $\Group$ e $\NormalSubgroup \IsNormalSubgroup \Group$. Abbiamo $\InnerAutomorphism(\NormalSubgroup) = \NormalSubgroup$.
\end{Theorem}
\Proof Basta provare che $\NormalSubgroup \subseteq \InnerAutomorphism(\NormalSubgroup)$. Supponiamo $x \in \NormalSubgroup$ e che $\InnerAutomorphism: x \mapsto gxg^{-1}$, con $g \in \Group$. Sia $y$ tale che $\InnerAutomorphism(y) = x$, che esiste in virt\`u del fatto che $\InnerAutomorphism$ \`e un automorfismo. Abbiamo $gyg^{-1} = x$, da cui $y = g^{-1}xg = (gg^{-1})g^{-1}xg(gg^{-1})$ e quindi $y = \InnerAutomorphism(g^{-1}g^{-1}xgg)$. Ma $x \in \NormalSubgroup$ per ipotesi quindi appartengono anche a $\NormalSubgroup$ gli elementi $g^{-1}xg$ e $g^{-1}g^{-1}xgg$ e quindi $y \in \InnerAutomorphism(\NormalSubgroup)$. \EndProof
\begin{Theorem}
	Siano $\Group_1, \Group_2$ gruppi e $\Homomorphism: \Group_1 \rightarrow \Group_2$ un omomofismo. Allora $\Kernel{\Homomorphism} \IsNormalSubgroup \Group_1$.
\end{Theorem}
\Proof Siano $x \in \Kernel{\Homomorphism}$ e $g \in \Group_1$. Abbiamo $\Homomorphism(gxg^{-1}) = \Homomorphism(g)\Homomorphism(x)\Homomorphism(g)^{-1} = 1$. \EndProof
\begin{Theorem}
	Sia $\Group$ un gruppo. Una relazione d'equivalenza $\Equivalence$ sul sostegno di $\Group$ \`e compatibile con la struttura di $\Group$ se e solo se
	\begin{itemize}
		\item $\EquivalenceClass{1} \IsNormalSubgroup \Group$;
		\item \`e equivalente alla relazione $xy^{-1} \in \EquivalenceClass{1}$ nelle variabili $x$ e $y$.
	\end{itemize}
\end{Theorem}
\Proof Supponiamo $\Equivalence$ sia compatibile con la struttura di $\Group$. Allora esiste il gruppo quoziente $\Quotient{\Group}{\Equivalence}$ e la proiezione sul quoziente \`e un omomorfismo, il cui nucleo \`e proprio $\EquivalenceClass{1}$ che \`e dunque un sottogruppo normale di $\Group$.
\par Siano ora $x, y \in \Group$. $x \Equivalence y$ equivale a $xy^{-1} \Equivalence 1$ e dunque a $xy^{-1} \in \EquivalenceClass{1}$.
\par Supponiamo ora invece che $\EquivalenceClass{1} \IsNormalSubgroup \Group$ e che la relazione di equivalenza $\Equivalence$ sia equivalente alla relazione $xy^{-1} \in \EquivalenceClass{1}$ nelle variabili $x$ e $y$. Siano $x, \xi, y, \eta \in \Group$ tali che $x \Equivalence \xi$ e $y \Equivalence \eta$. Abbiamo $x \xi^{-1} \in \EquivalenceClass{1}$ e $y \eta^{-1} \in \EquivalenceClass{1}$, da cui $xy(\xi\eta)^{-1} = xy\eta^{-1}\xi^{-1} = x (\xi^{-1}\xi) y\eta^{-1}\xi^{-1} = (x\xi^{-1}) (\xi y\eta^{-1}\xi^{-1}) \in \EquivalenceClass{1}$ e quindi $xy \Equivalence \xi\eta$. \EndProof
\par Il teorema precedente stabilisce che, dato un gruppo $\Group$, \`e equivalente fornire una relazione di equivalenza compatibile con la struttura di $\Group$ o il sottogruppo normale di $\Group$ che costituisce la classe di equivalenza di $1$: questo giustifica la definizione seguente.
\begin{Definition}
	Siano $\Group$ un gruppo e $\NormalSubgroup \IsNormalSubgroup \Group$. Si definisce \Define{gruppo quoziente di $\Group$ rispetto a $\NormalSubgroup$}[rispetto a un sottogruppo normale][sottogruppo], denotato $\Quotient{\Group}{\NormalSubgroup}$, il gruppo quoziente definito dalla relazione d'equivalenza $xy^{-1} \in \NormalSubgroup$ nelle variabili $x$ e $y$.
	Nel caso invece $\NormalSubgroup \IsSubgroup \Group$, ma non sia verificata la relazione $\NormalSubgroup \IsNormalSubgroup$, utilizziamo la stessa notazione per denotare lo stesso quoziente, che ha per\`o stavolta una struttura solo di classe quoziente, non di gruppo.
\end{Definition}
\begin{Definition}
	Siano $\Group$ un gruppo, $\Subgroup \IsSubgroup \Group$ e $x \in \Group$. L'immagine di $\Subgroup$ per l'azione di traslazione a sinistra di $\Group$ su se stesso, denotata $\Coset{x}{\Subgroup}$ si chiama \Define{classe laterale sinistra}[laterale][classe].
\end{Definition}
\begin{Theorem}
	Siano $\Group$ un gruppo, $\NormalSubgroup \IsNormalSubgroup \Group$ e $x \in \Group$. Allora $\Coset{x}{\NormalSubgroup} = \RightCoset{\NormalSubgroup}{x} = \EquivalenceClass{x} \in \Quotient{\Group}{\NormalSubgroup}$. D'altra parte, per ogni sottogruppo $\Subgroup \IsSubgroup \Group$, abbiamo $\Coimplies{\Subgroup \IsNormalSubgroup \Group}{\ForAll{x \in \Group}{\Coset{x}{\Subgroup} = \RightCoset{\Subgroup}{x}}}$.
\end{Theorem}
\Proof Per la prima parte del teorema, osserviamo che $y \in \Coset{x}{\NormalSubgroup}$ equivale all'affermare che vale l'equazione $y = xh$ per opportuno $h \in \NormalSubgroup$, e quindi a $xy^{-1} = h^{-1}$ e a $y \in \EquivalenceClass{x}$; un ragionamento analogo prova $\RightCoset{\NormalSubgroup}{x} = \EquivalenceClass{x}$.
\par Per la seconda parte, osserviamo che affermare che per ogni $x$ vale $\Coset{x}{\Subgroup} = \RightCoset{\Subgroup}{x}$ equivale ad affermare che per ogni $g \in \Subgroup$ esiste un $h \in \Subgroup$ tale che $xg = hx$; ci\`o \`e a sua volta equivalente a $xgx^{-1} = h$. Dunque, indicando per ogni $x \in \Group$ con $\InnerAutomorphism[x]$ l'automorfismo interno associato ad $x$, abbiamo $\ForAll{x \in \Group}{\Coimplies{x\Subgroup = \Subgroup x}{\InnerAutomorphism[x](\Subgroup) \subseteq \Subgroup}}$. \EndProof
\begin{Definition}
	Siano $\Group$ un gruppo e $\Subgroup \IsSubgroup \Group$. L'\Define{indice}[di un sottogruppo][indice] di $\Subgroup$ in $\Group$, denotato $\GroupIndex{\Group}{\Subgroup}$, \`e il numero di classi laterali di $\Subgroup \in \Group$.
\end{Definition}
\begin{Theorem}
	Sia $\Group$ un gruppo e $\Subgroup \IsSubgroup \Group$. Tutte le classi laterali di $\Subgroup$ sono equicardinali.
\end{Theorem}
\Proof Sia $x \Group$ e siano $g, h \in \Subgroup$. Per l'invertibilit\`a di $x$, abbiamo $xg = xh$ se e solo $g = h$. Ne consegue che $\Cardinality{\Subgroup} = \Cardinality{x\Subgroup}$. Analogamente si prova che $\Cardinality{\Subgroup} = \Cardinality{\Subgroup x}$. \EndProof
\begin{Corollary}
	\TheoremName{Teorema di Lagrange\footnote{Molti testi preferiscono indicare col nome di ``teorema di Lagrange'' l'enunciato che afferma che l'ordine di un sottogruppo divide l'ordine del gruppo intero: la nostra formulazione ha il vantaggio di comprendere il caso di gruppi di ordine infinito.}}[di Lagrange][teorema] Siano $\Group$ e $\Subgroup \IsSubgroup \Group$. Abbiamo $\GroupOrder{\Group} = \GroupIndex{\Group}{\Subgroup}\GroupOrder{\Subgroup}$.
\end{Corollary}
\Proof Segue direttamente dal teorema precedente e dal principio dei pastori. \EndProof
\begin{Theorem}
	Sia $\Operators$ un gruppo che agisce su una classe $\Class$. Per ogni $x \in \Class$, l'applicazione $f: \Quotient{\Operators}{\Stabilizer{x}} \rightarrow \Orbit{x}$ che alla classe laterale $\Coset{\Operator}{\Stabilizer{x}}$, con $\Operator \in \Operators$, associa $\Operator x$ \`e biunivoca.
\end{Theorem}
\Proof Siano $\Operator_1, \Operator_2 \in \Operators$ tali che $\Coset{\Operator_1}{\Stabilizer{x}} = \Coset{\Operator_2}{\Stabilizer{x}}$. Per opportuno $\Operator \in \Stabilizer{x}$ abbiamo $\Operator_1 = \Operator_2\Operator$, da cui $\Operator_1 x = \Operator_2\Operator x = \Operator_2 x$.
\par Siano ora $\Operator_1, \Operator_2 \in \Operators$ tali che $\Operator_1 x = \Operator_2 x$. Allora $\Operator_2^{-1}\Operator_1 \in \Stabilizer{x}$, cio\`e $\Coset{\Operator_1}{\Stabilizer{x}} = \Coset{\Operator_2}{\Stabilizer{x}}$.
\par Infine, sia $y \in \Orbit{x}$. Sia $\Operator \in \Operators$ tale che $\Operator x = y$. Chiaramente $y$ \`e l'immagine di $\Coset{\Operator}{\Stabilizer{x}}$.\EndProof
\begin{Definition}
	Sia $\Group$ un gruppo e sia
	$\Class \subseteq \Group$. Definiamo
	\Define{normalizzatore} di $\Class$,
	denotato $\Normalizer{\Class}$, la classe
	$\lbrace x \in \Group | x\Class = \Class x \rbrace$.
\end{Definition}
\begin{Theorem}
	Siano $\Group$ un gruppo e $\Class \subseteq \Group$.
	$\Normalizer{\Class}$ \`e un sottogruppo di $\Group$.
\end{Theorem}
\Proof
Siano $x, y \in \Normalizer{\Class}$ e sia
$a \in \Class$.
Per opportuni $b, c \in \Class$, abbiamo
$xy^{-1}a =
xy^{-1}ayy^{-1} =
xy^{-1}yby^{-1} =
xby^{-1} =
cxy^{-1}$.
Dunque $xy^{-1}\Class \subseteq \Class xy^{-1}\Class$.
Analogamente si dimostra il contenimento inverso e quindi
$\Normalizer{\Class} \IsSubgroup \Group$.
\EndProof
\begin{Theorem}
	Siano $\Group$ un gruppo e
	$\Subgroup \IsSubgroup \Group$.
	Abbiamo
	\begin{itemize}
		\item
		$\Subgroup \IsNormalSubgroup
		\Normalizer{\Subgroup}$;
		\item
		se $\Subgroup' \IsSubgroup \Group$
		tale che $\Subgroup \IsNormalSubgroup \Subgroup'$,
		allora $\Subgroup' \IsSubgroup \NormalSubgroup$.
	\end{itemize}
\end{Theorem}
\Proof
Siano $x \in \Subgroup$ e
$g \in \Normalizer{\Subgroup}$.
Per opportuno $y \in \Subgroup$, abbiamo
$gxg^{-1} =
ygg^{-1} = y \in \Subgroup$.
Dunque $\Subgroup \IsNormalSubgroup \Normalizer{\Subgroup}$.
\par
Sia $\Subgroup' \IsSubgroup \Group$ tale che
$\Subgroup \IsNormalSubgroup \Group$.
Sia $g \in \Subgroup'$. Per ogni $x \in \Subgroup$, esiste
$y \in \Subgroup$ tale che $gxg^{-1} = y$, che equivale a
$gx = yx$. Dunque $g \in \Normalizer{\Subgroup}$ e questo
conclude la dimostrazione.
\EndProof

\section{Teoremi d'isomorfismo.}\label{TeoremiDIsomorfismo}
\begin{Theorem}
\TheoremName{Primo teorema d'isomorfismo di gruppi}[primo d'isomorfismo di gruppi][teorema]
	Sia $\Homomorphism: \Group_1 \rightarrow \Group_2$ un omomorfismo
	tra i gruppi $\Group_1$ e $\Group_2$. Allora
	$\Isomorphic{\Quotient{\Group_1}{\Kernel{\Homomorphism}}}
	{\Image{\Homomorphism}}$.
\end{Theorem}
\Proof
Sia $\Homomorphism_0$ l'omomorfismo ottenuto da $\Homomorphism$
per passaggio al quoziente: esso \`e chiaramente suriettivo.
Inoltre $\Homomorphism_0$ \`e anche iniettivo perch\'e
$\Kernel(\Homomorphism_0) =
\lbrace \Kernel(\Homomorphism) \rbrace = 1$.
\EndProof
\begin{Theorem}
\TheoremName{Secondo teorema d'isomorfismo di gruppi}[secondo d'isomorfismo di gruppi][teorema]
	Siano
	\begin{itemize}
		\item
		$\Group$ un gruppo;
		\item
		$\Subgroup \IsSubgroup \Group$;
		\item
		$\NormalSubgroup \IsNormalSubgroup \Group$.
	\end{itemize}
	Abbiamo
	\begin{itemize}
		\item
		$\Subgroup\NormalSubgroup \IsSubgroup \Group$;
		\item
		$\NormalSubgroup \IsNormalSubgroup \Subgroup\NormalSubgroup$;
		\item
		$\Subgroup \cap \NormalSubgroup \IsNormalSubgroup \Subgroup$;
		\item
		l'applicazione
		$\Homomorphism: \Subgroup \rightarrow
		\Quotient{\Subgroup\NormalSubgroup}{\NormalSubgroup}$
		che ad $x \in \Subgroup$ associa $\Coset{x}{\NormalSubgroup} \in
		\Quotient{\Subgroup\NormalSubgroup}{\NormalSubgroup}$
		\`e un omomorfismo;
		\item
		l'omomorfismo $\Homomorphism_0:
		\Quotient{\Subgroup}{\Subgroup \cap \NormalSubgroup} \rightarrow
		\Quotient{\Subgroup\NormalSubgroup}{\NormalSubgroup}$,
		ottenuto da $\Homomorphism$ per passaggio al quoziente,
		\`e un isomorfismo,
		e dunque in particolare
		$\Isomorphic
		{\Quotient{\Subgroup}{\Subgroup \cap \NormalSubgroup}}
		{\Quotient{\Subgroup\NormalSubgroup}{\NormalSubgroup}}$.
		\end{itemize}
\end{Theorem}
\Proof
Siano $x_1, x_2 \in \Subgroup$ e $y_1, y_2 \in \NormalSubgroup$.
Abbiamo
$x_1y_1(x_2y_2)^{-1} =
x_1y_1y_2^{-1}x_2^{-1} =
x_1x_2^{-1}x_2y_1y_2^{-1}x_2^{-1}$.
Poich\'e $x_1x_2^{-1} \in \Subgroup$ e
$x_2y_1y_2^{-1}x_2^{-1} \in \NormalSubgroup$ per la normalit\`a
di $\NormalSubgroup \in \Group$, ne deduciamo che $\Subgroup\NormalSubgroup
\IsSubgroup \Group$.
\par
$\NormalSubgroup$ \`e normale in $\Group$ e dunque a maggior ragione lo \`e
in $\Subgroup\NormalSubgroup$.
\par
Siano $x \in \Subgroup \cap \NormalSubgroup$ e $y \in \Subgroup$.
Abbiamo $yxy^{-1} \in \Subgroup$ perch\'e $x,y \in \Subgroup$ e
$yxy^{-1} \in \NormalSubgroup$ per normalit\`a di $\NormalSubgroup$ in
$\Group$. Dunque $\Subgroup \cap \NormalSubgroup \IsNormalSubgroup \Subgroup$.
\par
$\Homomorphism$ \`e un omomorfismo perch\'e composizione dell'inclusione
canonica di $\Subgroup$ in $\Subgroup\NormalSubgroup$ con la proiezione canonica
di $\Subgroup\NormalSubgroup$ in
$\Quotient{\Subgroup\NormalSubgroup}{\NormalSubgroup}$.
\par
Il nucleo di $\Homomorphism$ \`e $\Subgroup \cap \NormalSubgroup$ e
$\Homomorphism$ \`e un omomomorfismo suriettivo.
Dunque $\Homomorphism_0$ \`e un isomorfismo.
\EndProof
\begin{Theorem}
\TheoremName{Terzo teorema d'isomorfismo di gruppi}[terzo d'isomorfismo di gruppi][teorema]
	Siano
	\begin{itemize}
		\item
		$\Group$ un gruppo;
		\item
		$\NormalSubgroup_1,
		\NormalSubgroup_2 \IsNormalSubgroup \Group$, con
		$\NormalSubgroup_1 \subseteq \NormalSubgroup_2$.
	\end{itemize}
	Allora
	\begin{itemize}
		\item
		esiste un'applicazione
		$\Homomorphism: \Quotient{\Group}{\NormalSubgroup_1}
		\rightarrow \Quotient{\Group}{\NormalSubgroup_2}$
		che associa a
		$\Coset{x}{\NormalSubgroup_1} \in
		\Quotient{\Group}{\NormalSubgroup_1}$
		la classe laterale
		$\Coset{x}{\NormalSubgroup_2} \in
		\Quotient{\Group}{\NormalSubgroup_2}$;
		\item
		$\Homomorphism$ \`e un omomorfismo;
		\item l'applicazione
		$\Homomorphism_0:
		\Quotient{\Quotient{\Group}{\NormalSubgroup_1}}
		{\Quotient{\NormalSubgroup_2}{\NormalSubgroup_1}}
		\rightarrow \Quotient{\Group}{\NormalSubgroup_2}$,
		ottenuta da $\Homomorphism$ per passaggio al quoziente,
		\`e un isomorfismo, e dunque in particolare
		$\Isomorphic{\Quotient
		{\Quotient{\Group}{\NormalSubgroup_1}}
		{\Quotient{\NormalSubgroup_2}{\NormalSubgroup_1}}}
		{\Quotient{\Group}{\NormalSubgroup_2}}$.
	\end{itemize}
\end{Theorem}
\Proof
L'esistenza di $\Homomorphism$ \`e garantita dal fatto che,
per ogni $x, y \in \Group$,
$\Implies{xy^{-1} \in \NormalSubgroup_1}{xy^{-1} \in \NormalSubgroup_2}$.
\par
Il fatto che $\Homomorphism$ sia un omomorfismo segue dal fatto che \`e un
omomorfismo la proiezione di $\Group$ al quoziente
$\Quotient{\Group}{\NormalSubgroup_2}$, nonch\'e
dalla struttura di gruppo definita su $\Quotient{\Group}{\NormalSubgroup_1}$.
\par
$\Homomorphism$ \`e un omomorfismo suriettivo per costruzione.
Inoltre il nucleo di $\Homomorphism$ \`e costituito da tutte le classi laterali
$\Coset{x}{\NormalSubgroup_1}$ con $x \in \NormalSubgroup_2$, le quali sono
esattamente gli elementi di $\Quotient{\NormalSubgroup_2}{\NormalSubgroup_1}$.
\EndProof
\begin{Theorem}
\TheoremName{Teorema di corrispondenza di sottogruppi}[di corrispondenza di sottogruppi ][teorema]
	Sia $\Group$ un gruppo e sia
	$\NormalSubgroup \IsNormalSubgroup \Group$.
	Indichiamo con $\Class_1$ la classe di tutti i sottogruppi
	di $\Group$ contenenti $\NormalSubgroup$ e con $\Class_2$ la classe
	di tutti i sottogruppi di $\Quotient{\Group}{\NormalSubgroup}$:
	\begin{itemize}
		\item
		se $X \in \Class_1$,
		allora $\Quotient{X}{\NormalSubgroup} \in \Class_2$;
		\item
		l'applicazione $\Map: \Class_1 \rightarrow \Class_2$ che a
		$X \in \Class_1$ associa
		$\Quotient{X}{\NormalSubgroup} \in \Class_2$
		\`e un isomorfismo d'ordine.
	\end{itemize}
\end{Theorem}
\Proof
Sia $X \in \Class_1$ e sia $a \in X$ tale che
$\Coset{a}{\NormalSubgroup} \in \Quotient{\Group}{\NormalSubgroup}$.
Gli elementi di $\Coset{a}{\NormalSubgroup}$ sono tutti e soli i prodotti di
$a$ con elementi $\NormalSubgroup$. Dunque
$\Coset{a}{\NormalSubgroup} \subseteq X$ e quindi
$\Coset{a}{\NormalSubgroup} \in \Quotient{X}{\NormalSubgroup}$.
\par
Se $X, Y \in \Class_1$ sono tali che $X \IsSubgroup Y$, allora
$\Quotient{X}{\NormalSubgroup} \IsSubgroup \Quotient{Y}{\NormalSubgroup}$.
Dunque $\Map$ \`e un'applicazione crescente.
\par
Siano ora $X, Y \in \Class_1$ qualsiasi e supponiamo
$\Quotient{X}{\NormalSubgroup} = \Quotient{Y}{\NormalSubgroup}$.
Abbiamo
$X =
\bigcup_{A \in \Quotient{X}{\NormalSubgroup}} A =
\bigcup_{A \in \Quotient{Y}{\NormalSubgroup}} A =
Y$. Quindi $\Map$ \`e un'applicazione iniettiva.
\par
Sia ora $Z \in \Class_2$ e poniamo $X = \bigcup_{A \in Z}$.
Per costruzione di $\Class_2$, $\NormalSubgroup \subseteq X$.
Siano poi $x, y \in X$. Allora
$\Coset{x}{\NormalSubgroup}, \Coset{y}{\NormalSubgroup} \in Z$ e,
poich\'e $Z$ \`e un gruppo,
$\Coset{xy^{-1}}{\NormalSubgroup} \in Z$, da cui $xy^{-1} \in X$.
Quindi $X \in \Class_1$.
Se verifica immediatamente che $\Map(X) = Z$. Quindi $\Map$ \`e
un'applicazione suriettiva.
\par
Dalla biettivit\`a e dalla cresenza di $\Map$ deduciamo che
$\Map$ \`e un isomorfismo d'ordine.
\EndProof

\section{Commutativit\`a.}\label{Commutativita}
\begin{Definition}
	Siano $\Group$ un gruppo e
	$\Class \subseteq \Group$. Chiamiamo
	\Define{centralizzatore}
	di $\Class$, denotato
	$\Centralizer{\Class}$ \`e l'insieme
	$\lbrace x \in \Group |
	\ForAll{a \in \Class}{xa = ax} \rbrace$.
\end{Definition}
\begin{Theorem}
	Siano $\Group$ un gruppo e $\Class \subseteq \Group$.
	Abbiamo $\Centralizer{\Class} \IsNormalSubgroup
	\Normalizer{Class}$.
\end{Theorem}
\Proof
Siano $x, y \in Centralizer{\Class}$ e
sia $a \in \Class$.
Abbiamo
$xy^{-1}a =
xy^{-1}ayy^{-1} =
xy^{-1}yay^{-1} =
xay^{-1} =
axy^{-1}$.
Dunque $xy^{-1} \in \Centralizer{\Class}$ e
$\Centralizer{\Class} \IsSubgroup \Group$.
Inoltre segue direttamente dalle definizioni
che $\Centralizer{\Class} \subseteq \Normalizer{\Class}$ e
dunque $\Centralizer{\Class} \IsSubgroup \Normalizer{\Class}$.
\par
Sia $g \in \Normalizer{\Class}$. Abbiamo,
per opportuno $b \in \Class$,
$gxg^{-1}a =
gxbg^{-1} =
gbxg^{-1} =
gbg^{-1}gxg^{-1} =
gg^{-1}agxg^{-1} =
agxg^{-1}$.
Quindi $\Centralizer{\Class} \IsNormalSubgroup
\Normalizer{\Class}$.
\EndProof
\begin{Theorem}
	Sia $\Group$ un gruppo e consideriamo
	la sua azione interna.
	Sia $x \in \Group$.
	Abbiamo $\Centralizer{x} =
	\Stabilizer{x}$.
\end{Theorem}
\Proof
Segue direttamente dalle definizioni.
\EndProof
\begin{Definition}
	Sia $\Group$ un gruppo.
	Definiamo
	\Define{centro}[di un gruppo][centro]
	il centralizzatore $\Centralizer{\Group}$
	dell'intero gruppo.
\end{Definition}
\begin{Theorem}
	Sia $\Group$ un gruppo.
	Abbiamo $\GroupCenter{\Group} \IsNormalSubgroup \Group$.
\end{Theorem}
\Proof
Siano $x, y \in \GroupCenter{\Group}$ e sia
$a \in \Group$.
\`E immediato verificare che
$(xy^{-1})a = a(xy^{-1})$.
\par
Altrettanto immediato \`e verificare che
$axa^{-1} = x \in \GroupCenter{\Group}$.
\EndProof
\begin{Theorem}
\TheoremName{Equazione delle classi}[delle classi][equazione]
	Sia $\Group$ un gruppo finito e
	sia $R \subseteq \Group$ tale da
	contenere uno e un solo rappresentate di
	ogni classe di coniugio di $\Group$.
	Abbiamo
	$\GroupOrder{\Group} =
	\GroupOrder{\GroupCenter{\Group}} +
	\sum_{x \in R \SetMin \GroupCenter{\Group}}
	\frac{\GroupOrder{\Group}}
	{\GroupOrder{\Centralizer{x}}}$.
\end{Theorem}
\Proof
Consideriamo l'azione interna di $\Group$.
Osserviamo che un elemento costituisce da solo
la propria classe di coniugio se e solo se appartiene
al centro.
Abbiamo
$\Cardinality{\Group} =
\sum_{x \in R} \Cardinality{\Orbit{x}} =
\sum_{x \in R} \frac{\GroupOrder{\Group}}
{\GroupOrder{\Stabilizer{x}}} =
\sum_{x \in R} \frac{\GroupOrder{\Group}}
{\GroupOrder{\Centralizer{x}}} =
\GroupOrder{\GroupCenter{\Group}} +
\sum_{x \in R \SetMin \GroupCenter{\Group}} \frac{\GroupOrder{\Group}}
{\GroupOrder{\Centralizer{x}}}$.
\EndProof
\begin{Definition}
	Sia $\Group$ un gruppo e siano
	$x, y \in \Group$. Definiamo
	\Define{commutatore} di $x$ e $y$,
	denotato $\Commutator{x}{y}$,
	l'elemento $x^{-1}y^{-1}xy$.
\end{Definition}
\begin{Theorem}
	Sia $\Group$ un gruppo e siano
	$x, y \in \Group$. $x$ e $y$ commutano
	se e solo se $\Commutator{x}{y} = 1$.
\end{Theorem}
\Proof
Segue direttamente dalle definizioni.
\EndProof
\begin{Definition}
	Sia $\Group$ un gruppo. Definiamo
	\Define{sottogruppo derivato}{derivato}[sottogruppo]
	di $G$, denotato $\DerivedSubgroup{\Group}$,
	il gruppo generato da tutti i commutatori di
	$\Group$.
\end{Definition}
\begin{Theorem}
	Sia $\Group$ un gruppo.
	Abbiamo $\DerivedSubgroup{\Group} \IsNormalSubgroup
	\Group$.
\end{Theorem}
\Proof
Consideriamo un generico elemento di $\Group$,
vale a dire un prodotto finito di commutatori
di $\Group$: $\prod_{i = 1}^n \Commutator{x_i}{y_i}$,
con $n \in \mathbb{N}$, $(x_i,y_i)_{i = 1}^n \in
\DirectProduct_{i = 1}^n (\Group \times \Group)$.
Sia inoltre $g \in \Group$.
\par
Abbiamo $g\left ( \prod_{i = 1}^n \Commutator{x_i}{y_i} \right )g^{-1} =
\prod_{i = 1}^n g\Commutator{x_i}{y_i}g^{-1} =
\prod_{i = 1}^n \Commutator{gx_ig^{-1}}{gy_ig^{-1}} \in
\DerivedSubgroup{\Group}$.
\EndProof
\begin{Definition}
	Sia $\Group$ un gruppo. Definiamo
	\Define{abelianizzato}[abelianizzato][gruppo]
	di $\Group$, denotato $\Abelianization{\Group}$,
	il quoziente $\Quotient{\Group}{\DerivedSubgroup{\Group}}$.
\end{Definition}
\begin{Theorem}
	Sia $\Group$ un gruppo.
	$\Abelianization{\Group}$ \`e un gruppo abeliano.
\end{Theorem}
\Proof
Immediata osservando che $\DerivedSubgroup{\Abelianization{\Group}} =
\lbrace 1 \rbrace$.
\EndProof

\section{Successioni esatte.}\label{SuccessioniEsatte}
\begin{Definition}
	Sia $(\Homomorphism_i)_{i \in n}$, con $n \in \omega + 1$, una
	successione di omomorfismi tali che, per ogni $i \in n$ tale che
	$i + 1 \in n$, si abbia
	$\Image{\Homomorphism_i} = \Kernel{\Homomorphism_{i + 1}}$. Si
	dice che la successione $(\Homomorphism_i)_{i \in n}$ \`e
	\Define{esatta}[esatta][successione], denotate
	$\Dom{\Homomorphism_0} \xrightarrow{\Homomorphism_0}
	\Dom{\Homomorphism_1} \xrightarrow{\Homomorphism_1}
	... \xrightarrow{\Homomorphism_{n - 2}}
	\Dom{\Homomorphism_{n - 1}} \xrightarrow{\Homomorphism_{n - 1}}
	\Dom{\Homomorphism_n} \xrightarrow{\Homomorphism_n}$.
\end{Definition}
\begin{Definition}
	Sia una successione esatta della forma
	$0 \xrightarrow{\Homomorphism_0}
	\Group_1 \xrightarrow{\Homomorphism_0}
	\Group_2 \xrightarrow{\Homomorphism_1}
	\Group_3 \xrightarrow{\Homomorphism_2}
	0$:
	si dice che tale successione \`e
	\Define{corta}[esatta corta][succesione].
\end{Definition}
\begin{Theorem}
	Sia la successione esatta corta
	$0 \rightarrow
	\Group_1 \xrightarrow{\Homomorphism_1}
	\Group_2 \xrightarrow{\Homomorphism_2}
	\Group_3 \rightarrow
	0$.
	Sono equivalenti i seguenti fatti:
	\begin{itemize}
		\item $\Homomorphism_1$ \`e una retrazione;
		\item $\Homomorphism_2$ \`e una sezione;
		\item $\Group_2$ \`e isomorfo a
		$\Group_1 \DirectSum \Group_3$ tramite
		un isomorfismo $\Isomorphism$ tale che
		$\Isomorphism \circ \Homomorphism_1 = \Identity_{\Group_1}$
		e $\Section \circ \Isomorphism = \Identity_{\Group_2}$.
	\end{itemize}
\end{Theorem}
\Proof Supponiamo $\Homomorphism$ possieda una retrazione $\Retraction$.

\section{Successioni di Jordan - Holder.}\label{SuccessioniDiJordanHolder}

\section{Prodotti semidiretti e diretti.}\label{ProdottiSemidirettiEDiretti}
\begin{Theorem}
	Siano $\Group_1$ e $\Group_2$ gruppi.
	Sia $\Homomorphism: \Group_2 \rightarrow \Automorphisms{\Group_1}$
	un omomorfismo.
	Il magma definito dalla classe $\Group_1 \times \Group_2$
	e dall'operazione $(x_1,y_1)(x_2,y_2) =
	(x_1\Homomorphism(y_1)(x_2),y_1y_2)$ \`e un gruppo.
\end{Theorem}
\Proof
Siano $x_1, x_2, x_3 \in \Group_1$ e
$y_1, y_2, y_3 \in \Group_2$.
Abbiamo
\begin{itemize}
	\item
	$((x_1,y_1)(x_2,y_2))(x_3,y_3) =
	(x_1\Homomorphism(y_1)(x_2),y_1y_2)(x_3,y_3) =
	(x_1\Homomorphism(y_1)(x_2)\Homomorphism(y_1y_2)(x_3),y_1y_2y_3) =
	(x_1\Homomorphism(y_1)(x_2)\Homomorphism(y_1)(\Homomorphism(y_2)(x_3)),y_1y_2y_3) =
	(x_1\Homomorphism(y_1)(x_2\Homomorphism(y_2)(x_3)),y_1y_2y_3) =
	(x_1,y_1)(x_2\Homomorphism(y_2)(x_3),y_2y_3) =
	(x_1,y_1)((x_2,y_2)(x_3,y_3))$;
	\item
	$(x_1,y_1)(1,1) =
	(x_1\Homomorphism(y_1)(1),y_1) =
	(x_1,y_1)$;
	\item
	$(1,1)(x_1,y_1) =
	(\Homomorphism(1)(x_1),y_1) =
	(x_1,y_1)$;
	\item
	$(x_1,y_1)((\Homomorphism{y_1})^{-1}(x_1^{-1}),y_1^{-1}) =
	(x_1\Homomorphism(y_1)((\Homomorphism(y_1))^{-1}(x_1^{-1}),y_1y_1^{-1}) =
	(1,1)$.
\end{itemize}
\EndProof
\begin{Definition}
	Con le notazioni del teorema precedente, chiamiamo il magma che abbiamo
	dimostrato essere un gruppo
	\Define{prodotto semidiretto}[semidiretto][prodotto]
	di $\Group_1$ e $\Group_2$ secondo $\Homomorphism$
	e lo denotiamo $\Group_1 \SemidirectProduct[\Homomorphism] \Group_2$, omettendo
	$\Homomorphism$ quando esso \`e chiaro dal contesto.
	Chiamiamo il prodotto $\Group_1 \times \Group_2$
	\Define{prodotto diretto}[diretto][prodotto]
	di $\Group_1$ e $\Group_2$ per distinguerlo pi\`u chiaramente
	dal prodotto semidiretto.
\end{Definition}
\begin{Theorem}
	Siano $\Group_1$ e $\Group_2$ gruppi e sia
	$\Homomorphism: \Group_2 \rightarrow \Automorphisms{\Group_1}$
	un omomorfismo.
	$\Group_1 \SemidirectProduct_{\Homomorphism} \Group_2 =
	\Group_1 \times \Group_2$ se e solo se
	$\Homomorphism$ \`e l'omomorfismo
	che a ogni $y \in \Group_2$ associa $\Identity[\Group_1] \in \Automorphisms{\Group_1}$.
\end{Theorem}
\Proof
Siano $x_1, x_2 \in \Group_1$ e $y_1, y_2 \in \Group_2$.
Abbiamo
$(x_1\Homomorphism(y_1)(x_2),y_1y_2)$
se e solo se
$x_1\Homomorphism(y_1)x_2 = x_1x_2$,
che equivale a
$\Homomorphism(y_1)x_2 = x_2$.
Quindi
$\Isomorphic{\Group_1 \SemidirectProduct{\Homomorphism} \Group_2}
{\Group_1 \times \Group_2}$ equivale a
$\ForAll{y \in \Group_2}
{\Homomorphism{y} = \Identity[\Group_1]}$.
\EndProof
\begin{Theorem}
	Sia $\Group$ un gruppo e siano
	$\NormalSubgroup$ e $\Subgroup$ gruppi tali che
	\begin{itemize}
		\item
		$\NormalSubgroup \IsNormalSubgroup \Group$;
		\item
		$\Subgroup \IsSubgroup \Group$;
		\item
		$\Group = \NormalSubgroup\Subgroup$;
		\item
		$\NormalSubgroup \cap \Subgroup = \lbrace 1 \rbrace$.
	\end{itemize}
	Sia inoltre $\InnerAutomorphism: \Subgroup \rightarrow \Automorphisms{\NormalSubgroup}$ l'azione per
	coniugio di $\Subgroup$ su $\NormalSubgroup$.
	L'applicazione
	$\Isomorphism: \NormalSubgroup \SemidirectProduct_{\InnerAutomorphism} \Subgroup \rightarrow
	\Group$
	che a $(x,y) \in \NormalSubgroup \SemidirectProduct_{\InnerAutomorphism} \Subgroup$ associa
	$(x,y) \in \Group$ \`e un isomorfismo.
\end{Theorem}
\Proof
Siano $x_1,x_2 \in \NormalSubgroup$ e $y_1,y_2 \in \Subgroup$.
Abbiamo $\Isomorphism((x_1,y_1)(x_2,y_2)) =
\Isomorphism((x_1y_1x_2y_1^{-1},y_1y_2)) =
x_1y_1x_2y_1^{-1}y_1y_2 =
x_1y_1x_2y_2 =
\Isomorphism((x_1,y_1))\Isomorphism((x_2,y_2))$.
Quindi $\Isomorphism$ \`e un omomorfismo.
\par
Poich\'e $\Group = \NormalSubgroup\Subgroup$,
$\Isomorphism$ \`e suriettivo.
\par
Sia $(x,y) \in \Kernel{\Isomorphism}$. Allora
$xy = 1$, da cui $y = x^{-1}$, ma poich\'e
$\NormalSubgroup \cap \Subgroup = \lbrace 1 \rbrace$,
non pu\`o che essere $x = y = 1$. Quindi $\Isomorphism$
\`e anche iniettivo e pertanto un isomorfismo. \EndProof
\begin{Corollary}
	Con le ipotesi del teorema precedente,
	$\Subgroup \IsNormalSubgroup \Group$
	equivale a $\Isomorphic{\Group}
	{\NormalSubgroup \times \Subgroup}$.
\end{Corollary}
\Proof
$\Group$ \`e isomorfo a $\NormalSubgroup \times \Subgroup$
se e solo se, per ogni $x \in \NormalSubgroup$ e ogni
$y \in \Subgroup$ abbiamo $yxy^{-1} = x$, che equivale a
$yxy^{-1}x = 1$.
\par
Se $\Subgroup \IsNormalSubgroup \Group$, allora
$yxy^{-1}x^{-1} \in \NormalSubgroup \cap \Subgroup$, quindi
certamente $yxy^{-1}x^{-1} = 1$.
\par
Assumiamo ora invece 
$\ForAll{x \in \NormalSubgroup}
{\ForAll{y \in \Subgroup}
{yxy^{-1}x^{-1} = 1}}$.
Sia $g \in \Group$. Per opportuni $a \in \NormalSubgroup$ e
$b \in Subgroup$ abbiamo $g = ab$.
Sia $y \in \Subgroup$.
Abbiamo $gyg^{-1} =
abyb^{-1}a^{-1} =
byb^{-1} \in \Subgroup$.
Dunque $\Subgroup \IsNormalSubgroup \Group$.
\EndProof


	\chapter{Teoria degli anelli}
	\section{Definizioni e nozioni di base.}\label{DefinizioniENozioniDiBase}
\begin{Definition}
	Un \Define{anello} \`e un gruppo commutativo che agisce distributivamente su se stesso sia a destra che a sinistra.
\end{Definition}
\par Denoteremo usualmente additivamente l'operazione di gruppo e moltiplicativamente le due azioni.
\begin{Definition}
	Sia $\Ring$ un anello. $\Ring$ si dice
	\begin{itemize}
		\item \Define{commutativo}[commutativo][anello] se le due azioni che definiscono $\Ring$ coincidono;
		\item \Define{unitario}[unitario][anello] o \Define{con identit\`a}[con identit\`a][anello] quando il magma indotto su $\Ring$ dalle due azioni \`e unitario.
	\end{itemize}
\end{Definition}
\begin{Definition}
	Sia $\Ring$ un anello unitario. Ogni elemento invertibile per il prodotto si dice \Define{unit\`a} o anche semplicemente \Define{invertibile}[invertibilie][elemento].
\end{Definition}
\begin{Definition}
	Si chiama \Define{corpo} un anello con identit\`a non banale in cui tutti gli elementi non nulli sono invertibili.
\end{Definition}
\begin{Definition}
	Un \Define{campo} \`e un corpo commutativo.
\end{Definition}
\begin{Theorem}
	Siano $\Ring$ un anello e $a \in \Ring$. Abbiamo
	\begin{itemize}
		\item $a \cdot 0 = 0$;
		\item $0 \cdot a = 0$.
	\end{itemize}
\end{Theorem}
\Proof Abbiamo $a \cdot 0 = a \cdot (0 + 0) = a \cdot 0 + a \cdot 0$ , da cui segue la prima affermazione sottraendo l'opposto di $a \cdot 0$. La dimostrazione della seconda affermazione \`e analoga. \EndProof
\begin{Theorem}
	Siano $\Ring$ un anello e $a, b \in \Ring$. Abbiamo
	\begin{itemize}
		\item $a \cdot (-b) = (-a) \cdot b = - ab$;
		\item $(-a) \cdot (-b) = ab$.
	\end{itemize}
\end{Theorem}
\Proof Abbiamo
\begin{itemize}
	\item $ab + a \cdot (-b) = a \cdot (b - b) = a \cdot 0 = 0$;
	\item $ab + (-a) \cdot b = (a - a) \cdot b = 0 \cdot b = 0$;
	\item $(-a) \cdot (-b) = (-(-a)) \cdot b = ab$. \EndProof
\end{itemize}
\begin{Definition}
	Dato un anello $\Ring$, siano $a, c \in \Ring$. $a$ si chiama \Define{divisore di $c$ sinistro}[divisore] se esiste un elemento $b \in \NotZero{\Ring}$ tale che $ab = c$. Se $a$ \`e divisore sinistro di $c$ si dice che $c$ \`e \Define{multiplo destro}[multiplo] di $a$ e che $a$ \NIDefine{divide $c$ a sinistra}, denotato $a|c$.
\end{Definition}
\par In un anello commutativo ci si riferisce a divisori e multipli senza ulteriori specificazioni, essendo le tre definizioni appena date equivalenti.
\begin{Definition}
	Dato un anello $\Ring$, siano $a, b \in \Ring$. Un divisore sinistro $d$ comune ad $a$ e $b$ si dice \Define{massimo comun divisore} sinistro quando, dato un qualsiasi divisore sinistro $e$ si ha che $e$ divide $d$. Inoltre, se $\Ring$ \`e un anello con identit\`a, due elementi si discono \Define{coprimi}[coprimo][elemento] a sinistra quando il loro massimo comun divisore sinistro \`e $1$.
\end{Definition}
\begin{Definition}
	Siano $\Ring$ un anello e $a \in \Ring$. $a$ si dice
	\begin{itemize}
		\item \Define{primo}[primo][elemento] se non \`e un'unit\`a e per ogni $b, c \in \Ring$ tali che $a$ divide $bc$ abbiamo $a|b \vee a |c$;
		\item \Define{irriducibile}[irriducibile][elemento] se non \`e $0$, non \`e un'unit\`a e non \`e prodotto di due elementi non invertibili;
		\item \Define{riducibile}[riducibile][elemento] se non \`e irriducibile e in tal caso una scrittura di $a$ come prodotto di due o pi\`u elementi non tutti invertibili si chiama \Define{riduzione}.
	\end{itemize}
\end{Definition}
\begin{Theorem}
	Sia $\Ring$ un anello. L'insieme dei divisori di $0$ e degli elementi invertibili sono disgiunti.
\end{Theorem}
\Proof Sia $a \in \Ring$ un divisore di $0$ invertibile. Allora esiste $b \in \NotZero{\Ring}$ tale che $ab = 0$, da cui $b = a^{-1} \cdot 0 = 0$, ma questo \`e assurdo. \EndProof

\section{Ideali.}\label{Ideali}
\begin{Definition}
	Sia $\Ring$ un anello. Un sottogruppo $\Ideal$ del gruppo additivo soggiacente a $\Ring$ si chiama \Define{ideale sinistro}[ideale] quando $(\forall x \in \Ring)(x\Ideal \subseteq \Ring)$.
\end{Definition}
\begin{Theorem}
	In un anello commutativo tutti gli ideali sono bilateri.
\end{Theorem}
\Proof La dimostrazione segue immediatamente dalla definizione. \EndProof
\par In virt\`u il teorema precedente, in un anello commutativo ci si riferisce sempre agli ideali senza specificare sinistro, destro o bilatero.
\begin{Definition}
	Un anello $\Ring$ si dice
	\begin{itemize}
		\item \Define{semplice sinistro}[semplice][anello] se i suoi unici ideali sinistri sono $\lbrace 0 \rbrace$ e $\Ring$;
		\item \Define{noetheriano sinistro}[noetheriano][anello] se ogni catena ascendente\footnote{Rispetto alla relazione di inclusione.} di ideali sinistri \`e stazionaria;
		\item \Define{artiniano sinistro}[artiniano][anello] se ogni catena discendente di ideali sinistri \`e stazionaria;
	\end{itemize}
\end{Definition}
\begin{Definition}
	Siano $(x_i)_{i \in I}$ elementi di un anello $\Ring$. Il pi\`u piccolo ideale sinistro [destro, bilatero] contenente gli elementi $(x_i)_{i \in I}$, denotato $\SpanIdeal{(x_i)_{i \in I}}$, si chiama \Define{ideale sinistro generato}[generato][ideale] da $(x_i)_{i \in I}$. Un ideale si dice
	\begin{itemize}
		\item \Define{principale}[principale][ideale] se \`e generato da un solo elemento;
		\item \Define{finitamente generato}[finitamente generato][ideale] se \`e generato da una famiglia finita di elementi.
	\end{itemize}
\end{Definition}
\begin{Theorem}
	Siano $\Ring$ un anello e $a \in \Ring$. L'ideale sinistro $\SpanIdeal{a}$ \`e l'insieme di tutti i multipli sinistri di $a$.
\end{Theorem}
\Proof \`E chiaro che tutti i multipli di $a$ apparengono a $\SpanIdeal{a}$: verifichiamo che $\SpanIdeal{a}$ sia un ideale. Siano $b, c \in \Ring$: abbiamo $ba - ca = (b - c)a$, dunque l'insieme dei multipli di $a$ costituisce un sottogruppo del gruppo additivo soggiacente ad $A$. Il resto della dimostrazione \`e evidente. \EndProof
\begin{Theorem}
	Un anello $\Noether$ \`e noetheriano sinistro se e solo se ogni suo ideale sinistro \`e finitamente generato.
\end{Theorem}
\Proof Se $\Noether$ ammette un ideale $\Ideal$ non finitamente generato, allora, dato un ideale $\Ideal_0 \subseteq \Ideal$ finitamente generato \`e sempre possibile un nuovo ideale finitamente generato $\Ideal_1$ tale che $\Ideal_0 \subseteq \Ideal_1 \subseteq \Ideal$ considerando come famiglia di generatori una qualunque famiglia finita di generatori di $\Ideal_0$ completata con un elemento $a \in \Ideal \SetMin \Ideal_0$.
\par Viceversa, supponiamo ogni ideale sia finitamente generato e consideriamo una catena crescente di ideali $(\Ideal_i)_{i \in I}$: l'unione $\bigcup_{i \in I} \Ideal_i$ \`e un ideale, necessariamente finitamente generato. Pertanto la catena diventa stazionaria dal primo indice $i$ a cui corrisponde un ideale che contiene tutti i generatori di $\bigcup_{i \in I} \Ideal_i$. \EndProof
\begin{Definition}
	Sia $\Ring$ un anello. Un ideale sinistro si dice \Define{massimale sinistro}[massimale][ideale] quando \`e un elemento massimale dell'insieme di tutti gli ideali propri sinistri di $\Ring$ ordinati per la relazioni d'inclusione.
\end{Definition}
\begin{Definition}
	Sia $\Ring$ un anello. Un ideale sinistro, destro o bilatero $\PrimeIdeal$ si dice \Define{primo}[primo][ideale] se, dati $a, b \in \Ring$, la condizione $(\forall c \in \Ring)(acb \in \PrimeIdeal)$ implica $a \in \PrimeIdeal \vee b \in \PrimeIdeal$
\end{Definition}
\begin{Definition}
	Sia $\Ring$ un anello. Un ideale $\PrimeIdeal$ si dice \Define{completamente primo}[completamente primo][ideale] se, dati $a, b \in \Ring$, la condizione $ab \in \PrimeIdeal$ implica $a \in \PrimeIdeal \vee b \in \PrimeIdeal$
\end{Definition}
\begin{Theorem}
	Ogni ideale sinistro completamente primo \`e primo.
\end{Theorem}
\Proof Siano $a$ e $b$ elementi di un anello $\Ring$ tali che $(\forall c \in \Ring)(acb \in \PrimeIdeal)$, dove $\PrimeIdeal$ \`e un ideale completamente primo. Abbiamo dunque $aab \in \PrimeIdeal$, da cui $a \in \PrimeIdeal \vee b \in \PrimeIdeal$. \EndProof
\begin{Theorem}
	Se $\Ring$ \`e un anello e $p \in \Ring$ \`e primo, allora l'ideale sinistro $\SpanIdeal{p}$ \`e completamente primo.
\end{Theorem}
\Proof Siano $a, b \in \Ring$ tali che $ab \in \SpanIdeal{p}$. Allora $ab$ \`e multiplo sinistro di $p$. \EndProof
\begin{Theorem}
	Siano $\Ring$ un anello e $\MaximalIdeal$ un suo ideale. $\MaximalIdeal$ \`e massimale se e solo se il quoziente $\Quotient{\Ring}{\MaximalIdeal}$ \`e un anello semplice.
\end{Theorem}
\Proof Il teorema \`e conseguenza immediata della corrispondenza biunivoca tra gli ideali di $\Quotient{\Ring}{\MaximalIdeal}$ e gli ideali di $\Ring$ contenenti $\MaximalIdeal$.
\begin{Corollary}
	Siano $\Ring$ un anello e $\MaximalIdeal$ un suo ideale massimale. $\Quotient{\Ring}{\MaximalIdeal}$ \`e
	\begin{itemize}
		\item un corpo se $\Ring$ \`e un anello unitario e $\MaximalIdeal$ \`e bilatero;
		\item un campo se $\Ring$ \`e un anello commutativo unitario.
	\end{itemize}
\end{Corollary}
\Proof Il quoziente $\Quotient{\Ring}{\MaximalIdeal}$ \`e un anello semplice. Se $a \in \Ring$ \`e invertibile, allora necessarimente $\Class{a} = \Ring$ e dunque $1 \in \Class{a}$, da cui $a$ \`e invertibile. Il resto del teorema si dimostra immediatamente. \EndProof
\begin{Theorem}
	Siano $\Ring$ un anello e $\PrimeIdeal$ un ideale. $\PrimeIdeal$ \`e un ideale completamente primo se e solo se nel quoziente $\Quotient{\Ring}{\PrimeIdeal}$ l'unico divisore di $0$ sinistro, destro o bilatero \`e $0$.
\end{Theorem}
\Proof Assumiamo inizialmente che $\PrimeIdeal$ sia completamente primo e dimostriamo solo il caso dei divisori di $0$ sinistri: per i destri la dimostrazione \`e analoga e dal caso sinistro e destro segue quello bilatero. Siano $a, b \in \Ring$ tali che $\Class{a} \Class{b} = 0$ e $\Class{b} \neq 0$. Allora $\Class{ab} = 0$, da cui $a \in \PrimeIdeal \vee b \in \PrimeIdeal$, cio\`e $\Class{a} = 0 \vee \Class{b} = 0$. Necessariamente, $\Class{a} = 0$.
\par Vediamo l'implicazione reciproca. Siano $a, b \in \Ring$ tali che $ab \in \PrimeIdeal$. Allora $\Class{ab} = \Class{a} \Class{b} = 0$, da cui $\Class{a} = 0 \vee \Class{b} = 0$, cio\`e $a \in \PrimeIdeal \vee b \in \PrimeIdeal$. \EndProof
\begin{Theorem}
	Ogni ideale sinistro, destro o bilatero massimale bilatero di un anello unitario \`e completamente primo.
\end{Theorem}
\Proof Sia $\Ring$ un anello unitario e $\MaximalIdeal$ un suo ideale massimale bilatero. Allora $\Quotient{\Ring}{\MaximalIdeal}$ \`e un corpo, dunque non esistono al suo interno divisori di $0$ e quindi $\MaximalIdeal$ \`e completamente primo. \EndProof
\begin{Lemma}
	\TheoremName{Lemma di Krull}[di Krull][lemma] Si assuma vero il lemma di Zorn. Dati un anello $\Ring$, un suo ideale $\Ideal$ e una classe moltiplicativamente chiusa $S \subseteq \Ring$ disgiunta da $\Ideal$, allora
	\begin{itemize}
		\item esiste un ideale massimale $\MaximalIdeal$ tale che $\Ideal \subseteq \MaximalIdeal$;
		\item esiste un ideale completamente primo $\PrimeIdeal$ tale che $\Ideal \subseteq \PrimeIdeal$ e $S \cap \PrimeIdeal = \emptyset$.
	\end{itemize}
\end{Lemma}
\Proof Il primo punto segue per immediata applicazione del lemma di Zorn dopo aver osservato che l'unione di una catena ascendente di ideali \`e un ideale.
\par Per il secondo punto, utilizzando ancora il lemma di Zorn, proviamo l'esistenza di un ideale massimale $\PrimeIdeal$ tra quelli che contengono $\Ideal$ ma sono disgiunti da $S$. Occore provare che $\PrimeIdeal$ \`e primo.
\par Siano $a, b \in \Ring$ tali che $ab \in \PrimeIdeal$ e supponiamo che n\'e $a$ n\'e $b$ appartengono a $\PrimeIdeal$. Allora gli ideali $\SpanIdeal{\PrimeIdeal,a}$ e $\SpanIdeal{\PrimeIdeal,b}$ incontrano $S$. Siano $x, y \in \Ring$ e $c, d \in \PrimeIdeal$ tali che $xa + c \in \SpanIdeal{\PrimeIdeal,a} \cap S$ e $yb + d \in \SpanIdeal{\PrimeIdeal,b} \cap S$. Abbiamo $(xa + c)(yb + d) = xyab + xad + cyb + cd \in \SpanIdeal{\PrimeIdeal,ab} \cap S$, contro l'ipotesi di massimalit\`a di $\PrimeIdeal$. Dunque almeno uno tra $a$ e $b$ deve appartenere a $\PrimeIdeal$.  \EndProof
\begin{Theorem}
	Siano $\Ideal_1$ e $\Ideal_2$ ideali sinistri in un anello $\Ring$. $\Ideal = \lbrace x \in \Ring | \Exists{a \in \Ideal_1}{\Exists{b \in \Ideal_2}{x = a + b}} \rbrace$ \`e un ideale sinistro.
\end{Theorem}
\Proof Siano $x, y \in \Ideal$. Esistono $a_x, a_y \in \Ideal_1$ e $b_x, b_y \in \Ideal_2$ tali che $x = a_x + b_x$ e $y = a_y + b_y$. Abbiamo $x - y = a_x + b_x - a_y - b_y = (a_x - a_y) + (b_x - b_y)$. Quindi $x - y \in \Ideal$.
\par Inoltre, se $\alpha \in \Ring$, allora $\alpha x = \alpha(a_x + b_x) = \alpha a_x + \alpha b_x$, da cui $\alpha x \in \Ideal$. \EndProof
\begin{Definition}
	Siano $\Ideal_1$ e $\Ideal_2$ ideali sinistri in un anello $\Ring$. L'ideale $\Ideal = \lbrace x \in \Ring | \Exists{a \in \Ideal_1}{\Exists{b \in \Ideal_2}{x = a + b}} \rbrace$ si chiama \Define{ideale somma}[somma][ideale] di $\Ideal_1$ e $\Ideal_2$, denotato $\Ideal_1 + \Ideal_2$.
\end{Definition}
\begin{Theorem}
	Siano $\Ideal_1$ e $\Ideal_2$ ideali sinistri in un anello $\Ring$. La classe $\Ideal$ di tutti gli elementi di $\Ring$ della forma $\sum_{n \in \mathbb{N}} a_nb_n$ con $(a_n)_{n \in \mathbb{N}} \in \DirectSum_{n \in \mathbb{N}} \Ideal_1$ e $(b_n)_{n \in \mathbb{N}} \in \DirectSum_{n \in \mathbb{N}} \Ideal_2$ \`e un ideale sinistro.
\end{Theorem}
\Proof La verifica \`e immediata.
\begin{Definition}
	Siano $\Ideal_1$ e $\Ideal_2$ ideali sinistri in un anello $\Ring$. L'ideale $\Ideal$ del teorema precedente si chiama \Define{ideale prodotto}[prodotto][ideale] di $\Ideal_1$ e $\Ideal_2$, denotato $\Ideal_1\Ideal_2$.
\end{Definition}
\begin{Definition}
	Sia $\Ideal$ un ideale sinistro in un anello $\Ring$. L'ideale intersezione di tutti gli ideali completamente primi che contengono $\Ideal$\footnote{La famiglia degli ideali primi \`e non vuota per il lemma di Krull.} si chiama \Define{radicale}[di un ideale][radicale] di $\Ideal$, denotato $\Radical{\Ideal}$. Inoltre un ideale si dice \Define{radicale}[radicale][ideale] quando coincide col suo radicale.
\end{Definition}
\begin{Theorem}
	Sia $\Ring$ un anello e $\Ideal$ un suo ideale. Abbiamo $\Radical{\Ideal} = \lbrace x \in \Ring | \Exists{n \in \mathbb{N}}{x^n \in \Ideal} \rbrace$.
\end{Theorem}
\Proof Sia ora un ideale completamente primo $\PrimeIdeal$ tale che $\Ideal \subseteq \PrimeIdeal$ e sia $x \in J$. Per opportuno $n \in \mathbb{N}$, $x^n \in \Ideal$, dunque $x^n \in \PrimeIdeal$ e quindi $x \in \PrimeIdeal$, da cui $J \subseteq \PrimeIdeal$. Quindi $J \subseteq \Radical{\Ideal}$.
\par Sia ora $r \in \Radical{\Ideal}$ tale che $r \notin J$. Allora, per definizione di $J$, l'insieme moltiplicativamente chiuso $S = \lbrace x | \Exists{n \in \NotZero{\mathbb{N}}}{x = r^n}$ \`e disgiunto da $\Ideal$. Esiste dunque un ideale completamente primo $\PrimeIdeal$ contenente $\Ideal$ e disgiunto da $S$. Ma $r \in \Radical{\Ideal} \subseteq \PrimeIdeal$, dunque $r \in \PrimeIdeal$, che \`e una contraddizione. Quindi $J = \Radical{\Ideal}$. \EndProof

\section{Anelli commutativi.}\label{AnelliCommutativi}
\subsection{Domini d'integrit\`a.}\label{DominiDIntegrita}
\begin{Definition}
	Un \Define{dominio d'integrit\`a}[d'integrit\`a][dominio] \`e un anello commutativo con identit\`a in cui $0$ \`e l'unico divisore di $0$.
\end{Definition}
\begin{Theorem}
	Sia $\IntegralDomain$ un dominio d'integrit\`a. Valgono le leggi di cancellazioni: se $a, b, c \in \IntegralDomain$, allora
	\begin{itemize}
		\item $ab = ac$ implica $b = c$;
		\item $ba = ca$ implica $b = c$.
	\end{itemize}
\end{Theorem}
\Proof Dimostriamo solo la prima legge: la seconda segue subito per commutativit\`a. Da $ab = ac$ segue $ab - ac = 0$, dunque $a(b - c) = 0$ e infine $b - c = 0$. \EndProof
\begin{Theorem}
	In un dominio d'integrit\`a $\IntegralDomain$ ogni elemento primo \`e irriducibile.
\end{Theorem}
\Proof Sia $p \in \IntegralDomain$ un elemento primo. Supponiamo $p = ab$, per opportuni $a, b \in \IntegralDomain$. $p$ \`e primo, quindi divide $a$ o divide $b$: per fissare le idee, diciamo che $p$ divide $a$. Allora, per opportuno $k \in \IntegralDomain$, $a = kp$, da cui $p = kpb$; ne deduciamo $p(1 - kb) = 0$: poich\'e $p \neq 0$, $b$ \`e invertibile e questo conclude la dimostrazione. \EndProof
\begin{Theorem}\label{DomIntIrridMax}
	Sia $\IntegralDomain$ un dominio d'integrit\`a. Un elemento $p \in \IntegralDomain$ \`e irriducibile se e solo se $\SpanIdeal{p}$ \`e massimale nell'insieme parzialmente ordinato per inclusione di tutti gli ideali principali di $\IntegralDomain$.
\end{Theorem}
\Proof Supponiamo $p$ irriducibile. Sia $a \in \PID$ tale che $\SpanIdeal{p} \subseteq \SpanIdeal{a} \subseteq \SpanIdeal{1}$. Abbiamo $p = ka$ per $k \in \PID$ opportuno, ma poich\'e $p$ \`e irriducibili, almeno uno tra $k$ ed $a$ deve essere un'unit\`a. Se $a$ \`e un'unit\`a allora $\SpanIdeal{a} = \SpanIdeal{1}$, se invece $k$ \`e un'unit\`a allora $a = k^{-1}p$, da cui $\SpanIdeal{p} = \SpanIdeal{a}$. \EndProof

\subsection{Domini a fattorizzazione unica.}\label{DominiAFattorizzazioneUnica}
\begin{Definition}
	Un \Define{dominio a fattorizzazione unica}[a fattorizzazione unica][dominio] $\UFD$ \`e un dominio di integrit\`a tale che ogni suo elemento $a \in \NotZero{\UFD}$ non invertibile si scrive in modo unico come prodotto di fattori irriducibili, a meno di unit\`a.
\end{Definition}
\begin{Theorem}
	In un dominio a fattorizzazione unica $\UFD$ ogni elemento irriducibile \`e primo.
\end{Theorem}
\Proof Sia $p \in \UFD$ irriducibile. Siano $a,b \in \UFD$ tali che $p|ab$. Il prodotto delle fattorizzazioni in irriducibili di $a$ e $b$ d\`a la fattorizzazione in irriducibili di $ab$: in quest'ultima appare necessariamente $p$, il quale appare dunque necessariamente nella fattorizzazione di $a$ o in quella di $b$, da cui $p|a \vee p|b$. \EndProof
\begin{Theorem}
	In un dominio a fattorizzazione unica $\UFD$ ogni famiglia finita di elementi ammette un massimo comun divisore.
\end{Theorem}
\Proof Si verifica immediatamente che il massimo comun divisore cercato \`e dato dal prodotto di tutti i fattori primi che compaiono in tutti gli elementi della famiglia finita (tenendo conto delle ripetezioni). \EndProof

\subsection{Domini ad ideali principali.}\label{DominiAdIdealiPrinciapli}
\begin{Definition}
	Un \Define{dominio ad ideali principali}[ad ideali principali][dominio] $\PID$ \`e un dominio d'integrit\`a tale che ogni suo ideale $\Ideal$ sia principale.
\end{Definition}
\begin{Theorem}
	Un dominio ad ideali principali \`e un anello noetheriano.
\end{Theorem}
\Proof Ogni ideale \`e finitamente generato. \EndProof
\begin{Theorem}
	In un dominio ad ideali principali $\PID$ ogni elemento irriducibile \`e primo.
\end{Theorem}
\Proof Se $\Irreducible \in \PID$ \`e irriducibile, allora l'ideale $\SpanIdeal{\Irreducible}$ \`e massimale in virt\`u del teorema \ref{DomIntIrridMax} e quindi primo. Quindi $\Irreducible$ \`e primo. \EndProof
\begin{Theorem}
	Ogni dominio ad ideali principali $\PID$ \`e un dominio a fattorizzazione unica.
\end{Theorem}
\Proof Proviamo prima che ogni $a \in \PID$ non invertibile \`e fattorizzabile: proveremo in un secondo momento che la fattorizzazione \`e unica a meno di invertibili. Se $a$ \`e irriducibile la dimostrazione d'esistenza della fattorizzazione \`e immediata. Se invece $a$ \`e riducibile, allora si scrive come prodotto di due elementi non invertibili $b, c \in \PID$ e abbiamo $\SpanIdeal{a} \subseteq \SpanIdeal{b}$ e $\SpanIdeal{a} \subseteq \SpanIdeal{c}$. Ripetendo il procedimento su $b$ e $c$ e sugli elementi che se ne deducono finch\'e possile costruiamo catene ascendenti di ideali: poich\'e $\PID$ \`e noetheriano, queste catene non possono essere continuate indefinitamente e si giunge dunque ad individuare una famiglia di ideali principali generati ciascuno da un elemento irriducibile; il prodotto di tutti questi elementi irriducibili \`e una fattorizzazione di $a$.
\par Dimostriamo ora l'unicit\`a della fattorizzazione. Consideriamo due fattorizzazioni in irriducibili di $a$: siano $(p_i)_{i = 1}^n$ ($n \in \mathbb{N}$) i fattori della prima fattorizzazione e $(q_j)_{j = 1}^m$ ($m \in \mathbb{N}$) i fattori della seconda fattorizzazione. Poich\'e in un dominio di integrit\`a valgono le leggi di cancellazione, possiamo supporre che tutti i fattori della prima fattorizzazione siano diversi da tutti i fattori della seconda fattorizzazione.
\par Consideriamo il fattore $p_1$. $p_1$ \`e primo ed esiste dunque un indice $j \in \lbrace 1, ..., m \rbrace$ tale che $p_1 | q_j$, ma per l'irriducibilit\`a di $q_j$ deve essere $p_1 = q_j$: i due fattori possono essere cancellati dall'equazione data dalle due fattorizzazione e si pu\`o procedere alla cancellazione di tutti i fattori $p_i$ ($i \in \lbrace 1, ..., n \rbrace$) uno alla volta ripetendo il procedimento: chiaramente le fattorizzazioni si esauriscono simultaneamente e coincidono. \EndProof
\begin{Theorem}
	Sia $\PID$ un dominio ad ideali principali e siano $a, b \in \PID$: $a$ e $b$ ammettono un massimo comun divisore.
\end{Theorem}
\Proof Segue direttamente dalle definizioni. \EndProof
\begin{Theorem}
	\TheoremName{Identit\`a di B\'ezout}[di B\'ezout][identit\`a] Sia $\PID$ un dominio ad ideali e siano $a, b \in \PID$. Scelto $d \in \MCD{a}{b}$, esistono $\xi, \eta \in \PID$ tali che $\xi a + \eta b = d$.
\end{Theorem}
\Proof $d$ \`e per definizione un generatore di $\SpanIdeal{a,b}$ ed ogni elemento di $\SpanIdeal{a,b}$ \`e della forma $\xi a + \eta b$ per $\xi, \eta \in \PID$ opportuni. \EndProof
\begin{Theorem}
	Sia $a \in \PID$. $\EquivalenceClass{b} \in \Quotient{\PID}{\SpanIdeal{a}}$ \`e invertibile se e solo se $b$ \`e coprimo con $a$.
\end{Theorem}
\Proof Se $b$ \`e coprimo con $a$, allora esistono $\xi, \eta \in \PID$ tali che $\xi a + \eta b = 1$. Se $\Projection: \PID \rightarrow \Quotient{\PID}{\SpanIdeal{a}}$ \`e la proiezione sul quoziente, allora dall'identit\`a di B\'ezout appena citata deduciamo $\Projection{\xi}\Projection{a} + \Projection{\eta}\Projection{b} = \Projection{1}$, e, essendo $\Projection{a} = 0$, $\EquivalenceClass{\eta}\EquivalenceClass{b} = 1$.
\par D'altra parte, se $\EquivalenceClass{b}$ \`e invertibile, allora esiste $\eta \in \PID$ tale che $\EquivalenceClass{\eta}\EquivalenceClass{b} = 1$, da cui $1 - \eta b \in \SpanIdeal{a}$ e quindi esiste $\xi \in \PID$ tale che $1 = \xi a + \eta b$, cio\`e che prova che $1 \in \MCD{a}{b}$. \EndProof
\begin{Theorem}
	Siano $\PID$ un dominio ad ideali principali e $a, b \in \PID$. L'identit\`a di $\PID$ \`e compatibile con le relazioni d'equivalenza in $x$ e $y$ seguenti:
	\begin{itemize}
		\item $x - y \in \SpanIdeal{ab}$ e $x - y \in \SpanIdeal{a}$;
		\item $x - y \in \SpanIdeal{ab}$ e $x - y \in \SpanIdeal{b}$.
	\end{itemize}
\end{Theorem}
\Proof Siano $x, y \in \PID$ tali che $x - y \in \SpanIdeal{ab}$. Allora $ab$ divide $x - y$, da cui $a$ divide $x - y$ e dunque $x - y \in \SpanIdeal{a}$; analogamente $x - y \in \SpanIdeal{b}$: l'identit\`a di $\PID$ \`e dunque compatibile con le relazioni d'equivalenza $x - y \in \SpanIdeal{ab}$ da una parte e $x - y \in \SpanIdeal{a}$ e $x - y \in \SpanIdeal{b}$ dall'altra. \EndProof
\begin{Theorem}
	\TheoremName{Teorema cinese del resto}[cinese del resto][teorema] Siano $\PID$ un dominio ad ideali principali e $a, b \in \PID$. Se $a$ e $b$ sono coprimi, allora $\Quotient{\PID}{\SpanIdeal{ab}} \Isomorphic \Quotient{\PID}{\SpanIdeal{a}} \times \Quotient{\PID}{\SpanIdeal{b}}$ e un isomorfismo \`e dato dall'applicazione $\phi: \Quotient{\PID}{\SpanIdeal{ab}} \rightarrow \Quotient{\PID}{\SpanIdeal{a}} \times \Quotient{\PID}{\SpanIdeal{b}}$ che a $\EquivalenceClass{x}[\SpanIdeal{ab}]$ associa $(\EquivalenceClass{x}[\SpanIdeal{a}],\EquivalenceClass{x}[\SpanIdeal{b}])$.
\end{Theorem}
\par Si verifica facilmente che $\phi$ \`e un omomorfismo. Inoltre $\phi$ \`e
\begin{itemize}
	\item iniettivo perch\'e, se $a$ e $b$ sono coprimi, le fattorizzazioni uniche di $a$ e di $b$ provano che $\Implies{x \in \SpanIdeal{a} \cup \SpanIdeal{b}}{x \in \SpanIdeal{ab}}$;
	\item suriettivo: sia $(\EquivalenceClass{x}[\SpanIdeal{a}],\EquivalenceClass{y}[\SpanIdeal{b}]) \in \Quotient{\PID}{\SpanIdeal{a}} \times \Quotient{\PID}{\SpanIdeal{b}}$. Esistono $\xi, \eta \in \PID$ tali che $\xi a + \eta b = 1$ (identit\`a di B\'ezout). Consideriamo $z = y \xi a + x \eta b$. Abbiamo $z = y ( 1 - x \eta b) + x \eta b = y - yx \eta b + x \eta b$, da cui $z - y \in \SpanIdeal{b}$ e analogamente si prova che $z - x \in \SpanIdeal{a}$. \EndProof 
\end{itemize}
\begin{Theorem}\label{ThCaratteristica}
	Sia $\Ring$ un anello. L'insieme delle soluzioni intere dell'equazione $x \cdot 1 = 0$, dove $1$ \`e l'identit\`a di $\Ring$, costituisce un ideale principale\footnote{$\mathbb{Z}$ \`e un dominio a ideali principali.}.
\end{Theorem}
\Proof Basta considerare l'omomorfismo $\phi: \mathbb{Z} \rightarrow \Ring$ definito tra i gruppi additivi tale che $\phi(1) = 1$: l'insieme delle soluzioni cercato \`e il nucleo di $\phi$. \EndProof
\begin{Definition}
	Il generatore non negativo dell'ideale descritto nel teorema precedente si chiama \Define{caratteristica} dell'anello $\Ring$, denotata $\Characteristic{\Ring}$.
\end{Definition}

\subsection{Domini Euclidei.}\label{DominiEuclidei}
\begin{Definition}
	Un \Define{dominio euclideo}[euclideo][dominio] $\EuclideanDomain$ \`e un anello commutativo con identit\`a in cui \`e definibile una funzione $\EuclideanDegree: \NotZero{\EuclideanDomain} \rightarrow \mathbb{N}$, detta \Define{grado}[in un dominio euclideo][grado], tale che
	\begin{itemize}
		\item ogni elemento $a \in \NotZero{\EuclideanDomain}$ si scrive, per ogni $b \in \NotZero{\EuclideanDomain}$, in modo unico nella forma $a = qb + r$, dove $q, r \in \EuclideanDomain$ e $\EuclideanDegree(r) < \EuclideanDegree(b) \vee r = 0$;
		\item dati due elementi qualsiasi $a, b \in \NotZero{\EuclideanDomain}$, \`e verificata la disuguaglianza $\EuclideanDegree(a) \leq \EuclideanDegree(ab)$.
	\end{itemize}
	Scrivere $a$ nella forma $qb + r$ si chiama effettuare la \Define{divisione euclidea} (o \Define{divisione con resto}) di $a$ per $b$ o anche semplicemente \Define{dividere} $a$ per $b$; $q$ si chiama \Define{quoziente} e $r$ \Define{resto}.
\end{Definition}
\par Sottolineamo il fatto che la funzione $\EuclideanDegree$ della definizione precedente \`e solo definibile, non definita: in particolare, uno stesso dominio euclideo pu\`o ammettere pi\`u funzioni grado\footnote{In effetti ne ammette infinite: \`e sufficiente moltiplicare una funzione grado per un intero positivo qualsiasi per ottenere una nuova funzione grado.}.
\begin{Theorem}
	Sia $\EuclideanDomain$ un dominio euclideo. Data una funziona grado $\EuclideanDegree$, gli elementi di grado minimo sono tutti e soli gli elementi invertibili.
\end{Theorem}
\Proof Sia $a \in \EuclideanDomain$. Abbiamo $a = a \cdot 1$, per cui $\EuclideanDegree(1) \leq \EuclideanDegree(a)$ e quindi il grado di $1$ \`e minimo in tutto l'anello.
\Proof Supponiamo ora $a \in \EuclideanDomain$ invertibile. Abbiamo $1 = aa^{-1}$, da cui $\EuclideanDegree(a) \leq \EuclideanDegree(1)$. Quindi $\EuclideanDegree(a) = \EuclideanDegree(1)$. \EndProof
\begin{Theorem}
	Ogni dominio euclideo $\EuclideanDomain$ \`e un dominio ad ideali principali.
\end{Theorem}
\Proof Sia $\Ideal$ un ideale e sia $a \in \Ideal$ un elemento di grado minimo in $\Ideal$. Sia $b \in \Ideal$ e consideriamo la divisione euclidea di $b$ per $a$: abbiamo $b = qa + r$ per $q, r \in \EuclideanDomain$ opportuni con $r$ nullo o di grado strettamente minore di $\EuclideanDegree(a)$. Abbiamo $r = b - qa \in \Ideal$, dunque, per la minimalit\`a del grado di $a$, $r$ \`e nullo. Ne consegue $\Ideal = \SpanIdeal{a}$.\EndProof
\begin{Theorem}
	Siano dati due elementi $a, b \in \NotZero{DomEucl}$. Si esegua il seguente algoritmo:
	\begin{enumerate}
		\item poniamo $n = 0$, $a_0 = a$, $b_0 = b$;
		\item effettuiamo la divisione con resto di $a_n$ per $b_n$ ottenendo $a_n = q_nb_n + r_n$, dove $q_n, r_n \in \EuclideanDomain$ ed $r_n = 0$ o $\EuclideanDegree(r_n) < \EuclideanDegree(b_n)$;\label{AlgEucl_1}
		\item se $r_n = 0$, allora l'algoritmo termina;
		\item se $r_n \neq 0$, allora poniamo $n = n + 1$, $a_n = b_{n - 1}$, $b_n = r_n$ e torniamo al punto \ref{AlgEucl_1}.
	\end{enumerate}
	L'algoritmo termina e, chiamato $N$ l'ultimo valore assunto da $n$, $b_N$ \`e il massimo comun divisore di $a$ e $b$.
\end{Theorem}
\Proof L'algoritmo termina necessariamente poich\'e ad ogni passo il grado di $b$ diminuisce, ma esso non pu\`o diminuire indefinitamente.
\par Per provare che $b_N$ \`e il massimo comun divisore di $a$ e $b$ procediamo per induzione su $N$. Se $N = 0$, allora $b$ divide $a$ ed \`e chiaro che $b$ \`e il massimo comun divisore di $a$ e $b$.
\par Supponiamo il teorema dimostrato per $N$ fissato e proviamolo per $N + 1$. Abbiamo, per ipotesi induttiva, che $b_N$ \`e il massimo comun divisore di $a_1$ e $b_1$. Abbiamo inoltre $a = q_0b + b_1$ e che $b_N = xa_1 + yb_1 = xb + yb_1$, per $x, y \in \EuclideanDomain$ opportuni, da cui $ya = yq_0b + b_N - xb$ e quindi $b_N = ya + (yq_0 -x)b$ e questo conclude la dimostrazione. \EndProof
\begin{Definition}
	L'algoritmo descritto nell'enunciato del teorema precedente si chiama \Define{algoritmo euclideo}[euclideo][algoritmo].
\end{Definition}
\par Osserviamo che l'algoritmo euclideo consente fornisce un metodo per costruire le identit\`a di B\'ezout in un dominio euclideo.

\subsection{Localizzazione di un anello commutativo.}\label{Localizzazione}
\begin{Theorem}
	Sia $\Ring$ un anello commutativo con identit\`a e $\LocalizationizationInverted \subseteq \Ring$ chiuso per il prodotto e contenente $1$. Definiamo su $\Ring \times \LocalizationizationInverted$ la relazione $\LocalizationizationEquivalence$ come segue: $((a,s) \LocalizationizationEquivalence (b, t)) \Leftrightarrow (\exists u \in \LocalizationizationInverted)(atu = bsu)$. La relazione $\LocalizationizationEquivalence$ \`e una relazione di equivalenza.
\end{Theorem}
\Proof La riflessiviti\`a si dimostra scegliendo $u = 1$, la simmetria segue direttamente dalla simmetria della definizione di $\LocalizationizationEquivalence$.
\par Vediamo la transitivit\`a. Supponiamo $(a,s) \LocalizationizationEquivalence (b,t)$ e $(b,t) \LocalizationizationEquivalence (c,r)$, con $a,b,c \in \Ring$ e $s,t,r \in \LocalizationizationInverted$. Per opportuni $u, v \in \LocalizationizationInverted$ abbiamo $atu = bsu$ e $brv = ctv$, da cui $aturv = bsurv = ctvsu$, con $tvu \in \LocalizationizationInverted$, da cui $(a,s) \LocalizationizationEquivalence (c,r)$. \EndProof
\begin{Definition}
	Sia $\Ring$ un anello commutativo con identit\`a e $\LocalizationizationInverted \subseteq \Ring$ chiuso per il prodotto e contenente $1$. Definiamo \Define{localizzazione} di $\Ring$ rispetto a $\LocalizationizationInverted$, l'anello $\Localization{\Ring}{\LocalizationizationInverted} = \Quotient{\Ring \times \LocalizationizationInverted}{\LocalizationizationEquivalence}$, dove
	\begin{itemize}
		\item $\LocalizationizationEquivalence$ \`e la relazione di equivalenza descritta dal teorema precedente;
		\item scegliamo di denotare la classe di equivalenza dell'elemento $(a,s) \in \Ring \times \LocalizationizationInverted$ con $\frac{a}{s}$;
		\item le operazioni di somma e prodotto sono definite come segue\footnote{La verifica degli assiomi di anello \`e diretta.}:
		\begin{itemize}
			\item $\frac{a}{s} + \frac{b}{t} = \frac{at + bs}{st}$;
			\item $\frac{a}{s} \cdot \frac{b}{t} = \frac{ab}{st}$.
		\end{itemize}
	\end{itemize}
\end{Definition}
\begin{Theorem}
	Sia $f: \Ring \rightarrow \Localization{\Ring}{\LocalizationizationInverted}$ l'applicazione che a $a \in \Ring$ associa $(a,1)$. Sono vere le seguenti affermazioni:
	\begin{itemize}
		\item $f$ \`e un omomorfismo;
		\item se $\LocalizationizationInverted$ non contiene divisori di $0$ allora $f$ \`e iniettivo.
	\end{itemize}
\end{Theorem}
\Proof Siano $a, b \in \Ring$. Abbiamo $f(a) + f(b) = \frac{a}{1} + \frac{b}{1} = \frac{a + b}{1} = f(a + b)$ e $f(a) \cdot f(b) = \frac{a}{1} \cdot \frac{b}{1} = \frac{ab}{1} = f(ab)$.
\par Vediamo la seconda affermazione. Occorre provare che $\Kernel{f} = \lbrace 0 \rbrace$. Sia $a \in \Ring$ tale che $f(a) = \frac{0}{1}$. Allora, per $u \in \LocalizationizationInverted$ opportuno, abbiamo $au = 0u = 0$. Poich\'e $u$ non \`e divisore di $0$, deve essere $a = 0$. \EndProof
\par Il teorema precedente fornisce, nei casi in cui $\LocalizationizationInverted$ non contiene divisori di $0$ un'\Define{identificazione canonica} di $\Ring$ in $\Localization{\Ring}{\LocalizationizationInverted}$.
\begin{Definition}
	Sia $\IntegralDomain$ un dominio d'integrit\`a: la sua localizzazione $\Localization{\IntegralDomain}{(\NotZero{\IntegralDomain})}$ si chiama \Define{campo dei quozienti} o \Define{campo delle frazioni} di $\IntegralDomain$.
\end{Definition}
\begin{Theorem}
	Sia $\IntegralDomain$ un dominio d'integrit\`a: il suo campo dei quozienti \`e un campo.
\end{Theorem}
\Proof \`E sufficiente verificare direttamente gli assiomi di campo. \EndProof

\section{Teoremi generali sugli anelli commutativi.}\label{TeoremiCommutativi}
\begin{Theorem}
\TheoremName{Teorema binomiale di Newton}[binomiale di Newton][teorema]
	Sia $\Ring$ un anello commutativo e siano $a,b \in \Ring$.
	Per ogni $n \in \mathbb{N}$ \`e verificata l'equazione
	$(a + b)^n = \sum_{i = 0}^n \binom{n}{i} a^ib^{n - i}$.
\end{Theorem}
\Proof
Per la distributivit\`a del prodotto rispetto alla somma,
$(a + b)^n$ \`e uguale a una somma di prodotti di $n$ fattori,
ciascuno dei quali pu\`o assumere il valore $a$ o $b$.
Poich\'e esistono $\binom{n}{i}$ prodotti
in cui esattamente $i$ fattori sono uguali ad $a$ e
i restanti sono uguali a $b$, si ha la tesi.
\EndProof
\begin{Theorem}
	Dato un anello $\Ring$ di caratteristica
	$\Characteristic{\Ring}$ finita,
	per ogni $a, b \in \Ring$,
	abbiamo $(a + b)^\Prime = a^\Prime + b^\Prime$.
\end{Theorem}
\Proof
Applichiamo la formula del binomio di Newton:
$(a + b)^{\Characteristic{\Ring}} =
\sum_{k = 0}^{\Characteristic{\Ring}}
\binom{\Characteristic{\Ring}}{k} a^kb^{n - k}$.
Per $k = 0$ e $k = \Characteristic{\Ring}$ otteniamo i termini
$b^{\Characteristic{\Ring}}$ e $a^{\Characteristic{\Ring}}$ rispettivamente.
Tutti gli altri termini hanno coefficiente divisible per
$\Characteristic{\Ring}$, e dunque nullo.
\EndProof
\begin{Theorem}
	Sia $\Ring$ un anello di caratteristica $\Prime \in \mathbb{Z}$ finita e
	sia $\Frobenius: \Ring \rightarrow \Ring$ l'applicazione che ad
	$x \in \Ring$ associa $x^{\Prime}$.
	$\Frobenius$ \`e un endomorfismo.
\end{Theorem}
\Proof
Per ogni $x,y \in \Ring$, abbiamo
$\Frobenius(x + y) =
(x + y)^{\Prime} =
x^\Prime + y^\Prime =
\Frobenius(x) + \Frobenius(y)$ e
$\Frobenius(xy) =
(xy)^\Prime =
x^\Prime y^\Prime =
\Frobenius(x)\Frobenius(y)$.
\EndProof
\begin{Definition}
	L'endomorfismo $\Frobenius$ del teorema precedente si chiama
	\Define{endomorfismo di Frobenius}[di Frobenius][endomorfismo].
\end{Definition}


\section{Serie di potenze e anelli di polinomi.}\label{SerieFormaliEAnelliDiPolinomi}
\subsection{Definizioni e teoremi di base.}\label{SeriePotenzeEAnelliDiPolinomiDefinizioniETeoremiDiBase}
\begin{Theorem}\label{ThSeriePotenze}
	Siano dati un anello $\Ring$ e un simbolo nuovo $x$. Sia $\PowerSeriesRing$ una struttura algebrica definita come segue:
	\begin{itemize}
		\item gli elementi di $\PowerSeriesRing$ sono tutte le scritture del tipo $\sum_{i \in \mathbb{N}} a_ix^i$, dove $(a_i)_{i \in \mathbb{N}} \in \DirectProduct_{i \in \mathbb{N}} \Ring$;
		\item \`e definita un'operazione $+$ tale che $\sum_{i \in \mathbb{N}} a_ix^i + \sum_{i \in \mathbb{N}} b_ix^i = \sum_{i \in \mathbb{N}} (a_i + b_i) x^i$;
		\item \`e definita un'operazione $\cdot$ tale che $\left ( \sum_{i \in \mathbb{N}} a_ix^i \right ) \left ( \sum_{i \in \mathbb{N}} b_ix^i \right ) = \sum_{i \in \mathbb{N}} c_i x^i$, dove per ogni $i \in \mathbb{N}$ poniamo $c_i.= \sum_{(j,k) \in \mathbb{N}^2 | j + k = i} a_jb_k$.
	\end{itemize}
	$\PowerSeriesRing$ \`e un anello.
\end{Theorem}
\Proof \`E sufficiente verificare che $\PowerSeriesRing$ verifica tutti gli assiomi di anello. \EndProof
\begin{Definition}
	Siano dati un anello $\Ring$ e un simbolo nuovo $x$. Definiamo
	\begin{itemize}
		\item \Define{anello delle serie di potenze}[delle serie di potenze][anello] in $x$ a coefficienti in $\Ring$, denotato $\PowerSeriesRing$, l'anello definito dal teorema \ref{ThSeriePotenze};
		\item \Define{serie di potenze} ogni elemento di $\PowerSeriesRing$;
		\item fissata una serie di potenza $p = \sum_{i \in \mathbb{N}} a_ix^i$, chiamiamo \Define{coefficienti}[coefficiente] gli elementi della successione $(a_i)_{i \in \mathbb{N}}$, \Define{termini}[termine] gli elementi della successione $(a_ix^i)_{i \in \mathbb{N}}$ e \Define{termine noto}[noto][termine] il coefficiente $a_0$.
	\end{itemize}
	Il nuovo simbolo $x$ viene spesso chiamato \Define{incognita}.
\end{Definition}
\par Data una serie di potenze $p = \sum_{i \in \mathbb{N}} a_i x^i$, \`e usuale semplificarne la notazione\footnote{Soprattutto nel caso di polinomi, \Cfr pi\`u avanti.} ponendo$x^0 = 1$, $(\forall i \in \mathbb{N})(0 \cdot x^i = 0)$ e omettendo dalla serie i termini nulli. Ci atterremo sempre a questa convenzione.
\begin{Definition}
	Sia un anello di serie di potenze $\PowerSeriesRing$. Definiamo la \Define{derivata formale} di una serie di potenze $p = \sum_{i \in \mathbb{N}} a_i x^i$ la serie di potenze $p = \sum_{i \in \mathbb{N}} (i + 1)a_{i + 1} x^i$.
\end{Definition}
\begin{Theorem}
	\TheoremName{Regola di Leibniz}[di Leibniz][regola] Sia un anello di serie di potenze $\PowerSeriesRing$ e siano $p,a,b \in \PowerSeriesRing$ tali che $p = ab$. Se $p',a',b'$ sono le derivate formali di $p, a, b$ rispettivamente, allora vale $p' = a'b + ab'$.
\end{Theorem}
\Proof Si verifica immediatamente per calcolo diretto dei coefficienti. \EndProof
\begin{Definition}
	Sia un anello di serie di potenze $\PowerSeriesRing$. Definiamo l'applicazione $\Deg: \PowerSeriesRing \rightarrow \Ext{\mathbb{N}}$ che ad ogni $p = \sum_{i \in \mathbb{N}} a_ix^i$ associa l'estremo superiore dell'insieme dei numeri naturali tali che $a_i \neq 0$: chiamiamo $\Deg p$ \Define{grado}[di una serie di potenze][grado] di $p$.
\end{Definition}
\begin{Theorem}
	Siano $\IntegralDomain$ un dominio di integrit\`a e $p, q \in \PowerSeries{\IntegralDomain}{x}$. Se $\Deg p$ e $\Deg q$ sono entrambi finiti, abbiamo $\Deg pq = \Deg p + \Deg q$.
\end{Theorem}
\Proof Siano $p = \sum_{i \in \mathbb{N}} a_ix^i$ e $q = \sum_{i \in \mathbb{N}} b_ix^i$. Il prodotto \`e per definizione $pq = \sum_{i \in \mathbb{N}} c_ix^i$ on $c_i = \sum_{(j,k) \in \mathbb{N}^2 | j + k = i} a_jb_k$. Ora, se $i = \Deg p + \Deg q$ abbiamo $c_i = a_{\Deg p}b_{\Deg q} \neq 0$ e, per ogni $l \geq i$, $c_l = 0$.\EndProof
\begin{Theorem}\label{ThAnelloPolinomi}
	Il sottoinsieme $\PolynomialsRing$ di $\PowerSeriesRing$ di tutti gli elementi $\sum_{i \in \mathbb{N}} a_ix^i$ tali che $(a_i)_{i \in \mathbb{N}}$ \`e una successione a supporto finito \`e un anello.
\end{Theorem}
\Proof \`E immediato che $\PolynomialsRing$ \`e chiuso per la somma e che l'opposto di ogni suo elemento appartiene ancora ad $\PolynomialsRing$. La chiusura di $\PolynomialsRing$ rispetto al prodotto segue dal teorema precedente. \EndProof
\begin{Definition}
	Siano dati un anello $\Ring$ e un simbolo nuovo $x$. Definiamo \Define{anello dei polinomi}[di polinomi][anello] in $x$ a coefficienti in $\Ring$, denotato $\PolynomialsRing$, l'anello definito dal teorema \ref{ThAnelloPolinomi}.
\end{Definition}
\par Spesso, sia nel caso delle serie di potenze che di quello dei polinomi, si desidera reiterare l'aggiunta di simboli nuovi un numero finito di volte: in tal caso si semplifica la notazione e il linguaggio scrivendo tutte insieme le nuove incognite (per esempio $\RingAdjunction{\PolynomialsRing}{x_1,x_2,x_3}$) e si parla di serie di potenze o polinomi \Define{multivariati}[multivariato][serie di potenze o polinomio] a coefficienti in $\Ring$, in opposizione alle serie di potenze e ai polinomi \Define{univariati}[univariato][serie di potenze o polinomio] definiti in precedenza.
\begin{Definition}
	Sia dato un anello di polinomi $\MultivariatePolynomialsRing$. Chiamiamo \Define{monomio} $\Monomial{x}$ ogni elemento dell'insieme di tutti i polinomi tali che tutti i suoi coefficienti siano nulli tranne uno il quale \`e uguale ad $1$, insieme che denoteremo con $\MonomialialsSet{\MultivariatePolynomialsRing}$.
\end{Definition}
\begin{Definition}
	Sia dato un anello di polinomi $\MultivariatePolynomialsRing$. Sono definite
	\begin{itemize}
		\item l'applicazione $\log: \MonomialialsSet{\MultivariatePolynomialsRing} \rightarrow \mathbb{N}^n$, che chiamiamo \Define{logaritmo monomiale}, che a $\prod_{i = 1, ..., n} x_i^{e_i}$ associa la $n$-upla $(e_1, ..., e_n)$;
		\item l'applicazione $\exp: \mathbb{N}^n \rightarrow \MonomialialsSet{\MultivariatePolynomialsRing}$, che chiamiamo \Define{esponenziale monomiale}, inversa del logaritmo monomiale;
		\item il \Define{grado}[di un monomio][grado] di ogni monomio $\prod_{i = 1, ..., n} x_i^{e_i}$ come $\Transposed{ \left ( \log \left ( \prod_{i = 1, ..., n} x_i^{e_i} \right ) \right )} (1, ..., 1)$.
	\end{itemize}
\end{Definition}
\begin{Theorem}
	Sia dato un anello di polinomi $\MultivariatePolynomialsRing$. Per ogni $\Monomial{x_1}, \Monomial{x_2} \in \MonomialialsSet{\MultivariatePolynomialsRing}$ abbiamo $\log{\Monomial{x_1} + \Monomial{x_2}} = \log{\Monomial{x_1}\Monomial{x_2}}$.
\end{Theorem}
\Proof Segue direttamente dalla definizione del logaritmo monomiale. \EndProof
\begin{Theorem}
	Sia dato un anello di polinomi $\PolynomialsRing$ e sia $v \in \Ring$. L'applicazione $\phi_a: \PolynomialsRing \rightarrow \Ring$ che a $p = \sum_{i \in \mathbb{N}} a_ix^i \in \PolynomialsRing$ associa $\sum_{i \in \mathbb{N}} a_iv^i$ \`e un omomorfismo.
\end{Theorem}
\Proof La verifica \`e immediata. \EndProof
\begin{Definition}
	Chiamiamo l'omomorfismo definito nel teorema precedente \Define{omomorfismo di valutazione}[di valutazione][omomorfismo] o \Define{omomorfismo di sostituzione}[di sostituzione][omomorfismo]. Con le notazioni del teorema precedente, denotiamo l'elemento $\phi_a(p)$ di $\Ring$ col simbolo $p(a)$. Inoltre, se $p(a) = 0$, diciamo che $a$ \`e \Define{radice}[radice di un polinomio] o \Define{zero}[zero di un polinomio] di $p$.
\end{Definition}

\subsection{Ordinamenti monomiali e ideali polinomiali.}\label{OrdinamentiMonomialiEIdealiPolinomiali}
\begin{Definition}
	Sia dato un anello di polinomi $\MultivariatePolynomialsRing$. Chiamiamo \Define{ordinamento monomiale} di $\MultivariatePolynomialsRing$ qualsiasi ordinamento di $\MonomialialsSet{\MultivariatePolynomialsRing}$ ottenuto trasportando un buon ordinamento invariante per traslazione dell'insieme $\mathbb{N}^n$.
\end{Definition}
\par Salvo indicazioni contrarie per anelli di polinomi univariati considereremo sempre l'ordinamento monomiale indotto dall'ordinamento classico di $\mathbb{N}$.
\begin{Theorem}\label{ordmonom_compatibilitaprodotto}
	Siano dati un anello di polinomi $\MultivariatePolynomialsRing$ e un suo ordinamento monomiale. Se $\Monomial{x_1}, \Monomial{x_2}, \Monomial{x_3} \in \MonomialialsSet{\MultivariatePolynomialsRing}$ e $\Monomial{x_1} \leq \Monomial{x_2}$, allora $\Monomial{x_3} \Monomial{x_1} \leq \Monomial{x_3} \Monomial{x_2}$.
\end{Theorem}
\Proof Abbiamo $\Monomial{x_1} \leq \Monomial{x_2}$ se e solo se $\log{\Monomial{x_1}} \leq \log{\Monomial{x_2}}$, che per l'invarianza per traslazione di $\mathbb{N}^n$ dell'ordinamento che induce l'ordinamento monomiale equivale a $\log{\Monomial{x_3}} + \log{\Monomial{x_1}} \leq \log{\Monomial{x_3}} + \log{\Monomial{x_2}}$ e dunque a $\log{\Monomial{x_3}\Monomial{x_1}} \leq \log{\Monomial{x_3}\Monomial{x_2}}$, vale a dire a $\Monomial{x_3}\Monomial{x_1} \leq \Monomial{x_3}\Monomial{x_2}$. \EndProof
\begin{Theorem}\label{ordmonom_min}
	Dato un qualsiasi ordinamento monomiale per $\MultivariatePolynomialsRing$, abbiamo $1 = \min \MonomialialsSet{\MultivariatePolynomialsRing}$.
\end{Theorem}
\Proof Supponiamo $\Monomial{x} \in \MonomialialsSet{\MultivariatePolynomialsRing}$ tale che $\Monomial{x} \leq 1$. Allora \`e immediato dimostrare per induzione che per ogni $i \in \mathbb{N}$, $\Monomial{x}^{i + 1} < \Monomial{x}^i$, da cui la famiglia $(\Monomial{x}^i)_{i \in \mathbb{N}}$  non ha minimo, in contraddizione col fatto che un ordinamento monomiale \`e un buon ordinamento. \EndProof
\begin{Definition}\label{polinomi_definizionitesta}
	Siano dati un anello di polinomi $\MultivariatePolynomialsRing$ e un suo ordinamento monomiale. Sia $p = \sum_{\Monomial{x} \in \MonomialialsSet{\MultivariatePolynomialsRing}} a_\Monomial{x} \Monomial{x} \in \MultivariatePolynomialsRing$. Definiamo
	\begin{itemize}
		\item \Define{supporto}[supporto di un polinomio] di $p$, denotato $\Supp{p}$, l'insieme di tutti i monomi $\Monomial{x}$ per cui $a_\Monomial{x} \neq 0$;
		\item \Define{monomio di testa}[di testa][monomio] di $p$, denotato $\LM{p}$, il massimo del supporto di $p$;
		\item \Define{coefficiente di testa}[di testa][coefficiente] di $p$, denotato $\LC{p}$, il coefficiente $a_{\LM{p}}$;
		\item \Define{termine di testa}[di testa][termine] di $p$, denotato $\LT{p}$, il termine $\LC{p}\LM{p}$;
		\item \Define{grado}[di un polinomio][grado] di $p$, denotato $\Deg{p}$ il grado di $\LM{p}$.
	\end{itemize}
\end{Definition}
\begin{Theorem}\label{teorema_divisione_polinomi}
	Siano $p \in \MultivariatePolynomialsRing$ e $(q_i)_{i \in I} \in \MultivariatePolynomialsRing^I$, dove $I$ \`e un insieme totalmente ordinato e finito, e supponiamo dato un ordinamento monomiale per $\MultivariatePolynomialsRing$. Consideriamo l'algoritmo seguente, dove inizialmente $r = 0$:
	\begin{enumerate}
		\item se $p \neq 0$, poniamo $t = \LT{p}$; altrimenti l'algoritmo termina;\label{divisione_estrazione}
		\item poniamo $j = \min \lbrace i: i \in I \rbrace \wedge \LT{q_i}|t \rbrace$ se l'insieme di cui si calcola il minimo \`e non vuoto, altrimenti si pone $r = r + t$ e si torna al passo \ref{divisione_estrazione};\label{divisione_ricerca}
		\item poniamo $p = p - \frac{t}{\LT{q_j}} q_j$ e torniamo al passo \ref{divisione_estrazione}.\label{divisione_somma}
	\end{enumerate}
	L'algoritmo termina per qualsiasi scelta di $p$ e di $(q_i)_{i \in I}$. Inoltre, nessun monomio in $\Supp{r}$ divide un qualche monomio della famiglia $(\LM{q_i})_{i \in I}$.
\end{Theorem}
\Proof Al passo \ref{divisione_somma} $\LC{p}$ viene azzerrato e pertanto il monomio di testa cambia. Per il teorema \ref{ordmonom_compatibilitaprodotto}, tutti i monomi del supporto del nuovo polinomio $p$, incluso il monomio di testa, sono minori del monomio di testa del vecchio polinomio. Poich\'e un ordinamento monomiale \`e un buon ordinamento, l'algoritmo deve terminare.
\par Il resto del teorema segue immediatamente dalla descrizione dell'algoritmo. \EndProof
\begin{Definition}
	In riferimento al teorema precedente, l'algoritmo descritto si chiama \Define{divisione}[tra polinomi][divisione] di $p$ rispetto a $(q_i)_{i \in I}$. Il polinomio $r$ ottenuto al termine dell'algoritmo si chiama \Define{resto}[resto di una divisione] della divisione.
\end{Definition}
\begin{Theorem}
	Sia $\Field$ un campo. L'anello $\FieldPolynomialsRing$ \`e un dominio euclideo, dove si pu\`o scegliere come funzione grado la funzione che associa ad ogni polinomio il suo grado nel senso della definizione \ref{polinomi_definizionitesta}.
\end{Theorem}
\Proof Siano $\Monomial{x_1}, \Monomial{x_2} \in \MonomialialsSet{\FieldPolynomialsRing}$. Per il teorema \ref{ordmonom_min}, abbiamo $1 \leq \Monomial{x_1}$, da cui, per il teorema \ref{ordmonom_compatibilitaprodotto}, $\Monomial{x_2} \leq \Monomial{x_1}\Monomial{x_2}$ e quindi $\Deg{\Monomial{x_2}} \leq \Deg{\Monomial{x_1}\Monomial{x_2}}$.
\par La possibilit\`a di effetturare una divisione euclidea segue direttamente dal teorema \ref{teorema_divisione_polinomi}. \EndProof

\subsection{Criteri di riducibilit\`a e d'irriducibilit\`a.}\label{CriteriDIrriducibilita}
\begin{Theorem}
	\TheoremName{Ruffini}[di Ruffini][teorema] Sia $\Ring$ un anello commutativo e sia $p \in \PolynomialsRing$. $a \in \Ring$ \`e radice di $p$ se e solo se il polinomio $x - a$ divide $p$.
\end{Theorem}
\Proof Supponiamo $a$ radice di $p$ e effettuiamo la divisione euclidea di $p$ per $x - a$ ottenendo un quoziente $q$ e un resto $r$ che \`e necessariamente un elemento di $\Ring$. Abbiamo $p = q(x - a) + r$, da cui $p(a) = r$, quindi $r = 0$ e $x - a$ divide $p$.
\par Se $x - a$ divide $p$, per opportuno $q \in \PolynomialsRing$ abbiamo $p = q(x - a)$, da cui $p(a) = 0$. \EndProof
\begin{Theorem}
	Se $\UFD$ \`e un dominio a fattorizzazione unica lo \`e anche ogni suo quoziente.
\end{Theorem}
\Proof Immediata considerando l'omomorfismo di proiezione di $\UFD$ sul quoziente \EndProof
\begin{Definition}
	Sia dato un anello di polinomi $\PolynomialsRing$ e un polinomio $p \in \PolynomialsRing$. Definiamo
	\begin{itemize}
		\item \Define{contenuto} di $p$, denotato $\Content{p}$ il massimo comun divisore dei coefficienti di $p$;
		\item \Define{primitivo}[primitivo][polinomio] $p$ quando $\Content{p} = 1$.
	\end{itemize}
\end{Definition}
\begin{Theorem}
	Sia $\Ring$ un anello commutativo e sia $p \in \PolynomialsRing$ primitivo. Siano inoltre $\MaximalIdeal$ un ideale massimale di $\Ring$ tale che $\LC{p} \notin \MaximalIdeal$ e $q \in \RingAdjunction{\Quotient{\Ring}{\MaximalIdeal}}{x}$ ottenuto da $p$ proiettandone i coefficienti in $\Quotient{\Ring}{\MaximalIdeal}$. Se $q$ \`e irriducibile allora \`e irriducibile anche $p$.
\end{Theorem}
\Proof L'irriducibilit\`a di $q$ implica $q \neq 0$. Dimostriamo che se $p$ \`e riducibile, allora lo \`e anche $q$.
\par Supponiamo $p = ab$, con $a,b \in \PolynomialsRing$ non invertibili e necessariamente di grado almeno $1$ per l'ipotesi di primitivit\`a. Proiettando i coefficienti di $a$ e di $b$ otteniamo $c,d \in \RingAdjunction{\Quotient{\Ring}{\PrimeIdeal}}{x}$ tali che $q = cd$. Poich\'e $q \neq 0$, non sono $0$ neanche $c$ e $d$.
\par Resta da vedere che $c$ e $d$ non sono invertibili. Poich\'e $\MaximalIdeal$ \`e massimale, $\Quotient{\Ring}{\MaximalIdeal}$ \`e un campo e $\RingAdjunction{\Quotient{\Ring}{\MaximalIdeal}}{x}$ \`e un dominio euclideo. Quindi gli unici polinomi invertibili in $\RingAdjunction{\Quotient{\Ring}{\MaximalIdeal}}{x}$ sono i polinomi di grado $0$.
\par Ora, necessariamente $\LC{p} = \LC{a} \LC{b}$ e poich\'e $\LC{p} \notin \MaximalIdeal$, anche $\LC{a}, \LC{b} \notin \MaximalIdeal$ e dunque anch\'e $c$ e $d$ hanno grado almeno $1$: abbiamo dunque ridotto il polinomio $q$. \EndProof
\begin{Theorem}
	\TheoremName{Lemma di Gauss sulla primitivit\`a}[di Gauss sulla primitivit\`a][lemma] Sia $\UFD$ un dominio a fattorizzazione unica. Il prodotto di due polinomi primitivi di $\UFDPolynomialsRing$ \`e primitivo.
\end{Theorem}
\Proof Siano $p, q \in \UFDPolynomialsRing$ primitivi. Supponiamo $\Content{pq} \neq 1$. Per la definizione di dominio a fattorizzazione unica, esiste un primo $P$ che divide $\Content{pq}$. D'altra parte, $P$ non pu\`o dividere tutti i coefficienti di $p$ n\'e tutti i coefficienti di $q$. Indicate con $(a_i)_{i \in \mathbb{N}}$ e $(b_i)_{i \in \mathbb{N}}$ le famiglie dei coefficienti di $p$ e $q$ rispettivamente, poniamo $I = \min \lbrace i \in \mathbb{N} | P \NotDivide a_i \rbrace$ e $J = \min i \in \mathbb{N} | P \NotDivide b_i \rbrace$. Consideriamo il coefficiente di $x^{I + J}$ in $pq$: esso \`e $\sum_{i + j = I + J} a_ib_j$. Per la minimalit\`a di $I$ e $J$, l'unico termine della somma precedente non divisibile per $P$ \`e $a_Ib_J$, ma allora, essendo $P$ primo e poich\'e le condizioni $P \NotDivide a_I$ e $P \NotDivide b_J$ escludono $a_Ib_J = 0$, $P$ non pu\`o dividere il coefficiente di $x^{I + J}$, in contraddizione con l'ipotesi $\Content{pq} \neq 1$. \EndProof
\begin{Theorem}
	\TheoremName{Lemma di Gauss sull'irriducibilit\`a}[di Gauss sull'irriducibilit\`a][lemma] Siano $\UFD$ un dominio a fattorizzazione unica e $\Field$ il suo campo dei quozienti. Un polinomio $p \in \UFDPolynomialsRing$ non costante \`e irriducibile se e solo se lo \`e in $\FieldPolynomialsRing$.
\end{Theorem}
\Proof Poich\'e $\UFD$ si identifica canonicamente con un sottoanello di $\Field$, \`e chiaro che che se $p$ \`e riducibile in $\UFDPolynomialsRing$ lo \`e anche in $\FieldPolynomialsRing$, a meno di essere invertibile in $\FieldPolynomialsRing$, cio\`e costante, ci\`o che \`e escluso dalle ipotesi. Proviamo che se $p$ \`e riducibile in $\FieldPolynomialsRing$ lo \`e anche in $\UFDPolynomialsRing$.
\par Se $p$ non \`e primitivo, allora si pu\`o scrivere nella forma $c p_0$, dove $p_0$ \`e primitivo e $c$ \`e un opportuno rappresentante di $\Content{p}$, che non \`e un'unit\`a: la fattorizzazione in irriducibili di $c$ contiene dunque un elemento irriducibile il quale \`e necessariamente irriducibile anche in $\UFDPolynomialsRing$. Quindi, supposto $p$ non primitivo, esso \`e necessariamente riducibile in $\UFDPolynomialsRing$ e dobbiamo dunque considerare solo il caso in cui $p$ \`e primitivo.
\par Supponiamo $p$ primitivo e riducibile: $p = ab$, dove $a,b \in \FieldPolynomialsRing$. Per opportuni elementi $d_a, d_b \in \Field$\footnote{Per esempio l'inverso del prodotto di tutti i denominatori di $a$ e $b$ rispettivamente.}, abbiamo $a = d_a a_0$ e $b = d_b b_0$, con $a_0, b_0 \in \UFDPolynomialsRing$, e scegliendo poi due rappresentanti $c_a, c_b \in \UFD$ di $\Content{a}$ e $\Content{b}$ rispettivamente abbiamo ancora $a = d_a c_a a_1$, $b = d_b c_b a_2$, con $a_1, b_1 \in \UFDPolynomialsRing$ primitivi. Abbiamo quindi $p = d_a c_a d_b c_b a_1 b_1$, dove $d_a c_a d_b c_b$ deve necessariamente appartenere a $\UFD$ (perch\'e appartengono a $\UFD$ sia i coefficienti di $p$ che di $a_1b_1$). Poich\'e $a_1$ e $b_1$ hanno lo stesso grado di $a$ e $b$ essi non sono invertibili come non lo erano $a$ e $b$ e abbiamo dunque esibito una riduzione di $p$ in $\UFDPolynomialsRing$. \EndProof
\begin{Theorem}
	Un anello $\UFD$ \`e un dominio a fattorizzazione unica se e solo se lo \`e il corripondente anello di polinomi $\UFDPolynomialsRing$.
\end{Theorem}
\Proof Poich\'e $\UFD$ si identifica canonicamente ad un sottoanello di $\UFDPolynomialsRing$, \`e chiaro che se $\UFD$ non \`e un dominio a fattorizzazione unica non lo \`e neanche $\UFDPolynomialsRing$.
\par Supponiamo $\UFD$ dominio a fattorizzazione unica e sia $p \in \UFDPolynomialsRing$. Denotato con $\Field$ il campo dei quozienti di $\UFD$, abbiamo gi\`a dimostrato che $\FieldPolynomialsRing$ \`e un dominio euclideo e dunque un dominio a fattorizzazione unica. Ora, dalla fattorizzazione unica di $p$ in $\FieldPolynomialsRing$ possiamo ricavare una fattorizzazione in irriducibili in $\UFDPolynomialsRing$ moltiplicando per un opportuno elemento $c \in \UFD$ come gi\`a indicato nella dimostrazione del lemma di Gauss sull'irriducibilit\`a: tutti i polinomi irriducibili di questa fattorizzazione sono irriducibili anche in $\UFDPolynomialsRing$ per il lemma di Gauss, all'eccezione forse di $c$, che \`e un'unit\`a in $\FieldPolynomialsRing$, ma ammette una fattorizzazione in irriducibili in $\UFDPolynomialsRing$ in quanto $c$ appartiene a $\UFD$ che \`e un dominio a fattorizzazione unica. Abbiamo dunque costruito una fattorizzazione in irriducibili di $p$, la quale non pu\`o che essere unica, poich\'e se esistessero due fattorizzazioni in $\UFDPolynomialsRing$ ne esisterebbero due anche nel dominio a fattorizzazione unica $\FieldPolynomialsRing$. \EndProof
\begin{Theorem}
	Dato un dominio a fattorizzazione unica $\UFD$ la fattorizzazione di ogni monomio di $\UFDPolynomialsRing$ \`e costituita da soli monomi.
\end{Theorem}
\Proof Immediata per induzione sul grado dei monomi. \EndProof
\begin{Theorem}
	\TheoremName{Criterio di Eisenstein (generalizzato)}[di Eisenstein (generalizzato)][criterio] Sia $\IntegralDomain$ un dominio d'integrit\`a. Sia $\Polynomial = \sum_{i \in \mathbb{N}} a_ix^i \in \RingAdjunction{\IntegralDomain}{x}$ primitivo. Se esiste un ideale primo $\PrimeIdeal \subseteq \IntegralDomain$ tale che
	\begin{itemize}
		\item $\LC{\Polynomial} \notin \PrimeIdeal$;
		\item per ogni $i \in \mathbb{N}$ se $a_i \neq 0$ e $a_i \neq \LC{\Polynomial}$, allora $a_i \in \PrimeIdeal$;
		\item $a_0 \notin \PrimeIdeal^2$;
	\end{itemize}
	allora $\Polynomial$ non \`e il prodotto di due polinomi entrambi non costanti in $\RingAdjunction{\IntegralDomain}{x}$.
\end{Theorem}
\Proof Supponiamo $\Polynomial = qr$ per opportuni $q = \sum_{i \in \mathbb{N}} b_ix^i$ e $r = \sum_{i \in \mathbb{N}} c_ix^i$ non costanti. Consideriamo l'omomorfismo di proiezione $\Projection: \IntegralDomain \rightarrow \Quotient{\IntegralDomain}{\PrimeIdeal}$. Definiamo
\begin{itemize}
	\item $\bar{\Polynomial} = \sum_{i \in \mathbb{N}} \Projection{a_i}x^i = \Projection{a_N} x^N$, dove $N = \Deg{\Polynomial}$;
	\item $\bar{q} = \sum_{i \in \mathbb{N}} \Projection{b_i}x^i$;
	\item $\bar{r} = \sum_{i \in \mathbb{N}} \Projection{c_i}x^i$.
\end{itemize}
\par Ora, come conseguenza del fatto che $\Projection$ \`e un omomorfismo, deve essere $\bar{\Polynomial} = \bar{q}\bar{r}$ e poich\`e $\RingAdjunction{\Quotient{\IntegralDomain}{\PrimeIdeal}}{x}$ \`e un dominio d'integrit\`a $\bar{q}$ e $\bar{r}$ devono essere costituiti da un solo termine ciascuno corrispondente al termine di testa di $q$ ed $r$ rispettivamente, diciamo $\bar{q} = \Projection{b_M}x^M$ e $\bar{q} = \Projection{c_L}x^L$ per $M, L \in \mathbb{N}$.
\par Per ipotesi deve essere $M \neq 0$ e $L \neq 0$. Ma allora entrambi i termini noti di $q$ ed $r$ appartengono a $\PrimeIdeal$ e dunque $a_0 \in \PrimeIdeal^2$, contro le ipotesi.
\par Dunque $\Polynomial$ non \`e prodotto di due polinomi entrambi non costanti. \EndProof
\begin{Corollary}
	\TheoremName{Criterio di Eisenstein}[di Eisenstein][criterio] Sia $\UFD$ un dominio a fattorizzazione unica. Sia $\Polynomial = \sum_{i \in \mathbb{N}} a_ix^i \in \UFDPolynomialsRing$ primitivo. Se esiste un elemento primo $\Prime \in \UFD$ tale che
	\begin{itemize}
		\item $\Prime\NotDivide\LC{\Polynomial}$;
		\item $\Prime$ divide tutti i coefficienti non nulli di $\Polynomial - \LT{\Polynomial}$;
		\item $\Prime^2\NotDivide a_0$;
	\end{itemize}
	allora $\Polynomial$ \`e irriducibile.
\end{Corollary}
\Proof Segue dal teorema precedente applicato a $\Polynomial$ visto come elemento di $\FieldPolynomialsRing$ e dal lemma di Gauss. \EndProof



	\chapter{Classificazione di gruppi}
	\section{Gruppi ciclici.}\label{GruppiCiclici}
\begin{Definition}
	Un gruppo generato da un solo elemento si dice
	\Define{ciclico}[ciclico][gruppo].
\end{Definition}
\begin{Theorem}
	Sia $\Group$ un gruppo finito tale che
	$\GroupOrder{\Group}$ sia primo.
	$\Group$ \`e ciclico.
\end{Theorem}
\Proof
Sia $\Subgroup \IsSubgroup \Group$.
Per il teorema di Lagrange, $\GroupOrder{\Subgroup}$
non pu\`o essere che $1$ o $\GroupOrder{\Group}$.
\EndProof
\begin{Theorem}
	Sia $\Group$ un gruppo ciclico e
	sia $\Subgroup \IsSubgroup \Group$.
	$\Subgroup$ \`e ciclico.
\end{Theorem}
\Proof
Sia $\Generator \in \Group$ tale che
$\Group = \SpanGroup{\Generator}$.
Sia $n = \min \lbrace k \in \NotZero{\mathbb{N}} |
\Generator^k \in \Subgroup \rbrace$.
/par
Sia ora $x \in \Subgroup$.
Per opportuno $m \in \mathbb{N}$ abbiamo
$x = \Generator^m$.
Effettuiamo la divisione euclidea di $m$ per $n$:
abbiamo
$m = qn + r$ per $q, r \in \mathbb{Z}$ con
$0 \leq r < n$.
Poich\'e $\Generator^r = \Generator^{m - qn} =
\Generator^m \left ( \Generator^n \right )^{- q}$,
per la minimalit\`a di $n$ deve essere $r = 0$.
\par
Ne deduciamo che $\Generator^n$ genera $\Subgroup$.
\EndProof
\begin{Theorem}
	Il gruppo additivo $\mathbb{Z}$ \`e ciclico
	e generato da $1$.
\end{Theorem}
\Proof
Sia $n \in \mathbb{Z}$.
Abbiamo $n \cdot 1 = n$.
\EndProof
\begin{Theorem}
	Sia $\Group$ un gruppo ciclico.
	$\Group$ \`e isomorfo a $\mathbb{Z}$ o
	a un suo quoziente.
\end{Theorem}
\Proof
Basta considerare l'omomorfismo
$\Homomorphism: \mathbb{Z} \rightarrow \Group$
che a $n \in \mathbb{Z}$ associa $\Generator^n$,
dove $\Generator$ \`e un generatore di $\Group$.
$\Homomorphism$ \`e suriettivo.
La tesi segue dunque dal primo teorema d'isomorfismo.
\EndProof
\begin{Corollary}
	Tutti i gruppi ciclici sono abeliani.
\end{Corollary}
\Proof
$\mathbb{Z}$ \`e un gruppo abeliano e tutti i
quozienti di gruppi abeliani sono abeliani.
\EndProof
\begin{Definition}
	Chiamiamo \Define{funzione $\EulerFunction$ di Eulero}[$\EulerFunction$ di Eulero][funzione] o anche \Define{funzione toziente}[toziente][funzione] la funzione $\EulerFunction: \NotZero{\mathbb{N}} \rightarrow \mathbb{N}$ che a $n \in \mathbb{Z}$ associa il numero di interi minori di $n$ coprimi con $n$.
\end{Definition}
\begin{Theorem}
	Sia $n \in \mathbb{N}$. Il numero di elementi invertibili in $\Quotient{\mathbb{Z}}{\SpanIdeal{n}}$ \`e $\EulerFunction(n)$.
\end{Theorem}
\Proof Segue direttamente dalla definizione di $\EulerFunction$ e dal fatto che $\EquivalenceClass{x} \in \Quotient{\mathbb{Z}}{\SpanIdeal{n}}$ \`e invertibile se e solo se $x$ \`e coprimo con $n$. \EndProof
\begin{Corollary}
	Sia $\Prime \in \mathbb{Z}$ primo. $\Quotient{\mathbb{Z}}{\SpanIdeal{\Prime}}$ \`e un campo.
\end{Corollary}
\Proof Segue direttamente dal teorema precedente e dalla definizione di campo. \EndProof
\par Del corollario precedente sar\`a fornita un'altra dimostrazione in materia di teoria dei campi.
\begin{Theorem}
	Sia $n \in \mathbb{N}$. $\EquivalenceClass{x} \in \Quotient{\mathbb{Z}}{\SpanIdeal{n}}$ genera il gruppo ciclico $\Quotient{\mathbb{Z}}{\SpanIdeal{n}}$ se e solo se $\MCD{x}{n} = 1$.
\end{Theorem}
\Proof $\SpanGroup{\EquivalenceClass{x}}$ \`e per definizione l'insieme di tutti gli elementi $a\EquivalenceClass{x} \in \Quotient{\mathbb{Z}}{\SpanIdeal{n}}$ con $a \in \mathbb{Z}$. Poich\'e $x$ \`e coprimo con $n$, \`e invertible $\EquivalenceClass{x}$. Sia $\EquivalenceClass{y} \in \Quotient{\mathbb{Z}}{\SpanIdeal{n}}$ e scegliamo $a \in \EquivalenceClass{yx^{-1}}$: abbiamo allora $a\EquivalenceClass{x} = \EquivalenceClass{ax} = \EquivalenceClass{a}\EquivalenceClass{x} = \EquivalenceClass{yx^{-1}}\EquivalenceClass{x} = \EquivalenceClass{yx^{-1}x} = \EquivalenceClass{y}$. \EndProof
\begin{Corollary}
	Per ogni $n \in \mathbb{N}$, il numero di generatori del gruppo ciclico $\Quotient{\mathbb{Z}}{\SpanIdeal{n}}$ \`e $\EulerFunction(n)$.
\end{Corollary}
\Proof Segue direttamente dal teorema precedente. \EndProof
\begin{Theorem}
	La funzione $\EulerFunction$ di Eulero \`e moltiplicativa.
\end{Theorem}
\Proof Siano $m, n \in \NotZero{\mathbb{N}}$. Per le fattorizzazioni di $m$ ed $n$, ogni $x \in \mathbb{N}$ \`e coprimo con $mn$ se e solo se \`e coprimo anche con $m$ ed $n$ simultaneamente.
\par Ora, per ogni $x \in \mathbb{N}$ tale che $x < mn$ l'applicazione che ad $x$ associa $\EquivalenceClass{x}[\SpanIdeal{mn}] \in \Quotient{\mathbb{Z}}{\SpanIdeal{mn}}$ \`e biunivoca e, per l'isomorfismo dato dal teorema cinese del resto, lo \`e anche l'applicazione che ad $x$ associa $(\EquivalenceClass{x}[\SpanIdeal{m}],\EquivalenceClass{x}[\SpanIdeal{n}]) \in \Quotient{\mathbb{Z}}{\SpanIdeal{m}} \times \Quotient{\mathbb{Z}}{\SpanIdeal{n}}$ ed infine quella che ad $x$ associa la coppia $(y,z) \in \mathbb{N}^2$ dove $y$ e $z$ sono gli unici rappresentanti di $\EquivalenceClass{x}[\SpanIdeal{m}]$ con $y < m$ e di $\EquivalenceClass{x}[\SpanIdeal{n}]$ con $z < n$ rispettivamente. Essendo $x$ coprimo con $m$, deve esserlo anche $y$ e analogamente $z$ deve essere coprimo con $n$: questo conclude la dimotrazione. \EndProof
\begin{Theorem}
	Sia $\Prime \in \mathbb{Z}$ primo e positivo. Per ogni $n \in \NotZero{\mathbb{N}}$, abbiamo $\EulerFunction(\Prime^n) = \Prime^{n - 1}(\Prime - 1)$.
\end{Theorem}
\Proof Tra tutti gli interi positivi minori di $\Prime^n$, sono coprimi con $\Prime^n$ tutti coloro che non sono multipli di $\Prime$, e i multipli di $\Prime$ sono esattamente $\Prime^{n - 1}$. Di conseguenza $\EulerFunction(\Prime^n) = \Prime^n - \Prime^{n - 1} = \Prime^{n - 1}(\Prime - 1)$. \EndProof
\begin{Theorem}
	Sia $n \in \NotZero{\mathbb{N}}$, abbiamo $\EulerFunction(n) = n \prod_{\Prime \in I} \left ( 1 - \frac{1}{p} \right )$, dove $I$ \`e l'insieme di tutti i fattori primi di $n$.
\end{Theorem}
\Proof Abbiamo $\EulerFunction(n) = \prod_{\Prime \in I} \EulerFunction(\Prime^{e_\Prime})$, dove, per ogni $\Prime \in I$ $e_\Prime$ \`e il numero di volte in cui compare $\Prime$ nella fattorizzazione di $n$. Dunque $\EulerFunction = \prod_{\Prime \in I} \Prime^{e_\Prime - 1}(\Prime - 1) = \prod_{\Prime \in I} \Prime^{e_\Prime} \Prime^{- 1}(\Prime - 1) = n \prod_{\Prime \in I} \left ( 1 - \frac{1}{\Prime} \right )$. \EndProof
\begin{Theorem}
	Sia $\Group$ un gruppo ciclico finito,
	generato da $\Generator \in \Group$.
	Per ogni $d | \GroupOrder{\Group}$,
	$\SpanGroup{\Generator^{\frac{\GroupOrder{\Group}}{d}}}$
	\`e l'unico sottogruppo di $\Group$ di ordine $d$.
\end{Theorem}
\Proof
Si verifica facilmente che
$\SpanGroup{\Generator^{\frac{\GroupOrder{\Group}}{d}}} \IsSubgroup
\Group$ e che
$\GroupOrder{\SpanGroup{\Generator^{\frac{\GroupOrder{\Group}}{d}}}} = d$.
Sia $\Subgroup \IsSubgroup \Group$ di ordine $d$.
Sia $k \in \mathbb{N}$ il minimo naturale tale che
$\Subgroup = \SpanGroup{\Generator^k}$.
Poich\'e l'ordine di $\Subgroup$ \`e $d$ e
l'ordine di $\Group$ \`e $n$,
abbiamo
$\Generator^{kd} = \Generator^{n} = 1$.
Dunque, per opportuno $q \in \mathbb{N}$, abbiamo
$kd = qn$ o equivalentemente $k = \frac{qn}{d}$.
Poich\'e $\frac{n}{d} | k$, abbiamo
$\Subgroup \IsSubgroup \SpanGroup{\Generator^{\frac{n}{d}}}$,
ma dato che
$\GroupOrder{\Subgroup} \IsSubgroup \GroupOrder{\SpanGroup{\Generator^{\frac{n}{d}}}}$,
deve essere
$\Subgroup = \SpanGroup{\Generator^{\frac{n}{d}}}$.
\EndProof
\begin{Corollary}
	Sia $\Group$ un gruppo ciclico finito.
	$\Group$ ammette tanti sottogruppi
	quanti sono i divisori di $\GroupOrder{\Group}$.
\end{Corollary}
\Proof
Segue direttamente dal teorema precedente.
\EndProof
\begin{Corollary}
	Per ogni $n \in \NotZero{\mathbb{N}}$, abbiamo
	$n = \sum_{d \in \mathbb{N},\\d|n} \EulerFunction(d)$.
\end{Corollary}
\Proof
Consideriamo il gruppo ciclico di ordine $n$
$\Quotient{\mathbb{Z}}{\SpanIdeal{n}}$.
Ogni elemento \`e generatore di un sottogruppo
di
$\Quotient{\mathbb{Z}}{\SpanIdeal{n}}$.
Per il teorema precedente, per ogni intero $d \in \mathbb{N}$
che divide $n$ il gruppo $\Quotient{\mathbb{Z}}{\SpanIdeal{n}}$
ammette esattamente un sottogruppo $\Subgroup_d$.
Pertanto, indicato con $G_d$ l'insieme dei generatori di
$\Subgroup_d$, abbiamo
$\Quotient{\mathbb{Z}}{\SpanIdeal{n}} =
\bigcup_{d \in \mathbb{N},\\d|n} G_d$
da cui, passando alle cardinalit\`a e essendo i $G_d$ tutti
disgiunti,
$n = \sum_{d \in \mathbb{N},\\d|n} \EulerFunction(d)$.
\EndProof

\section{Gruppi simmetrici e alternanti.}\label{GruppiSimmetrici}
\begin{Definition}
	Sia $\Class$ una classe.
	Si chiama \Define{gruppo simmetrico}[simmetrico][gruppo]
	di $Class$ il gruppo $\Automorphisms{\Class}$.
	Quando $\Class$ \`e l'insieme di tutt i numeri naturali
	da $1$ a $n \in \mathbb{N}$, allora denotiamo
	$\SymmetricGroup{n}$ il gruppo simmetrico di $\Class$.
	\par
	Gli elementi di $\Automorphisms{\Class}$ si chiamano
	\Define{permutazioni}[permutazione].
	Sia $\Permutation \in \Automorphisms{Class}$ tale che la classe
	$\Class \SetMin \Fixed{\Class}{\Permutation}$ sia finita:
	diciamo che $\Permutation$ \`e \Define{finita}[finita][permutazione]
	si usa indicare $\Permutation$ con la notazione
	$\Permutation[x_1 & ... & x_k][y_1 & ... & y_k]$
	dove $(x_i)_{i = 1}^k$ \`e una famiglia di elementi
	di $\Class$ che comprende
	$\Class \SetMin \Fixed{\Class}{\Permutation}$ e
	$(y_k)_{i = 1}^k = (\Permutation(x_k))_{i = 1}^k$.
\end{Definition}
\begin{Theorem}
	Siano $\Class_1$ e $\Class_2$ due classi.
	Se $\Class_1$ e $\Class_2$ sono equicardinali,
	allora $\Automorphisms{\Class_1}$ e $\Automorphisms{\Class_2}$
	sono gruppi isomorfi.
\end{Theorem}
\Proof
Sia $Map: \Class_1 \rightarrow \Class_2$
un'applicazione biettiva.
Sia $\Isomorphism:
\Automorphisms{\Class_1} \rightarrow
\Automorphisms{\Class_2}$
l'applicazione che a $\Automorphism \in \Automorphisms{\Class_1}$
associa $\Map\Automorphism\Map^{-1} \in \Automorphisms{\Class_2}$.
Segue direttamente dalle definizioni che $\Isomorphism$
\`e un isomorfismo.
\EndProof
\begin{Corollary}
	Sia $\Class$ un insieme finito: $\Isomorphic{\Automorphisms{\Class}}
	\Isomorphic{\SymmetricGroup{\Cardinality{\Class}}}$.
\end{Corollary}
\Proof
Segue immediatamente dal teorema precedente.
\EndProof
\begin{Theorem}
	Per ogni $n \in \mathbb{N}$, abbiamo
	$\GroupOrder{\SymmetricGroup{n}} = \Factorial{n}$.
\end{Theorem}
\Proof
\`E immediato notare che
$\GroupOrder{\SymmetricGroup{0}} = 1$ e
$\GroupOrder{\SymmetricGroup{1}} = 1$.
\par
Supponiamo il teorema provato per $\SymmetricGroup{n}$
e proviamolo per $\SymmetricGroup{n + 1}$.
Consideriamo l'applicazione
$\Map: \SymmetricGroup{n} \times \lbrace 1, ..., n + 1 \rbrace \rightarrow
\SymmetricGroup{n + 1}$
che a $(\Permutation,m) \ in \SymmetricGroup{n} \times \lbrace 1, ..., n + 1 \rbrace$
associa $\Map(\Permutation,m)$ definita da
$\Map(\Permutation,m) =
\begin{cases}
(\Cycle{m & n + 1}\Permutation)(x)\text{ se }x \leq n;\\
m\text{ se }x = n + 1.
\end{cases}$
\`E immediato verificare che l'applicazione \`e biettiva.
Dunque $\GroupOrder{\SymmetricGroup{n + 1}} =
\GroupOrder{\SymmetricGroup{n}} \Cardinality{\lbrace 1, ..., n + 1 \rbrace} =
\Factorial{n}(n + 1) = \Factorial{n + 1}$.
\EndProof
\begin{Theorem}
\TheoremName{Teorema di Cayley}[di Cayley][teorema]
	Sia $\Group$ un gruppo. Esiste
	$\Subgroup \IsSubgroup \Automorphisms{\Group}$,
	dove $\Automorphisms{Group}$ \`e il gruppo delle
	permutazioni del sostegno di $\Group$ (e non il gruppo
	degli automorfismi del gruppo $\Group$),
	tale che $\Isomorphic{\Group}{\Subgroup}$.
\end{Theorem}
\Proof
\`E sufficiente prendere come $\Subgroup$
l'immagine dell'azione di traslazione a sinistra di
$\Group$ su se stesso: l'azione \`e fedele, da cui l'isomorfismo.
\EndProof
\begin{Definition}
	Siano $\Class$ una classe e $\Permutation \in \Automorphisms{\Class}$.
	Diciamo che $\Permutation$ \`e un \Define{ciclo} quando l'azione di
	$\SpanGroup{\Permutation}$ su $\Class$ prevede un'unica orbita di
	cardinalit\`a maggiore o uguale di $2$.
	Se tale orbita \`e finita, cio\`e se
	$\Permutation =
	\Permutation[x_1 & ... & x_{k - 1} & x_k][x_2 & ... & x_k & x_1]$,
	allora $\Permutation$ si chiama \Define{ciclo finito}[finito][ciclo] o
	\Define{$k$-ciclo}{$k$-ciclo}[ciclo] e si
	denota in maniera pi\`u compatta
	$\Cycle{x_1 & ... & x_k}$.
	Un $2$-ciclo si chiama pi\`u comunemente
	\Define{trasposizione}.
	La \Define{lunghezza}[di un ciclo][lunghezza] di un ciclo \`e la cardinalit\`a
	della sua unica orbita.
\end{Definition}
\begin{Theorem}
	Sia $\Permutation$ un ciclo finito.
	$\Permutation$ \`e prodotto di trasposizioni.
\end{Theorem}
\Proof
Sia $\Permutation = \Cycle{x_1 & ... & x_k}$.
Abbiamo
$\Permutation = \prod_{i = 1}^{k - 1} \Cycle{x_i & x_{i + 1}}$.
\EndProof
\begin{Definition}
	Siano $\Permutation_1$ e $\Permutation_2$
	due cicli di $\Automorphisms{\Class}$.
	Se $\Class \SetMin \Fixed{\Class}{\Permutation_1} \cap
	\Class \SetMin \Fixed{\Class}{\Permutation_2}$ si dice
	che $\Permutation_1$ e $\Permutation_2$ sono
	\Define{cicli disgiunti}[disgiunti cicli].
\end{Definition}
\begin{Theorem}
	Ogni famiglia di cicli disgiunti commuta.
\end{Theorem}
\Proof
Segue immediatamente dalle definizioni.
\EndProof
\begin{Theorem}
	Siano $\Class$ una classe e $\Permutation \in \Automorphisms{\Class}$.
	$\Permutation$ pu\\`o essere scritto come prodotto\footnote{Eventualmente infinito, nel qual caso il prodotto `e l'unica permutazione che prolunga tutti i fattori.} di cicli disgiunti
	in modo unico a meno dell'ordine dei fattori e di moltiplicazione per $1$-cicli;
	pi\`u precisamente, ogni gruppo simmetrico \`e generato dai suoi cicli.
\end{Theorem}
\Proof
Consideriamo l'azione di $\SpanGroup{\Permutation}$ su $\Class$.
Costruiamo la famiglia di permutazioni
$(\Permutation_i)_{i \in \Quotient{\Class}{\SpanGroup{\Permutation}}}$
tale che $\Permutation_i(x) =
\begin{cases}
\Permutation(x)\text{, se }x \in i,\\
x\text{, altrimenti}.
\end{cases}$
\`E immediato verificare che
$\prod_{i \in \Quotient{\Class}{\SpanGroup{\Permutation}}} \Permutation_i =
\Permutation$ e che tutti i $\Permutation_i$ sono cicli disgiunti.
D'altra parte, sia
$(\Permutation_i)_{i \in I}$ una famiglia di permutazioni disgiunte
tali che $\prod_{i \in I} \Permutation_i = \Permutation$.
Sia $x \in \Class$. Per ipotesi, $x$ \`e un punto fisso.
di $\Permutation$ o esiste un'unica permutazione $\Permutation_i$,
con $i \in I$,
che trasforma $x$ in un altro punto $\Permutation_i(x) = \Permutation(x)$.
Inoltre, poich\'e $\Permutation_i$ \`e un ciclo, abbiamo
$\ForAll{n \in \mathbb{N}}{\Permutation_i^n(x) = \Permutation^n(x)}$.
Ma questo equivale a dire che $\Permutation_i$ \`e una delle
permutazioni della scomposizione precedentemente costruita.
Poich\'e quanto appena affermato vale per tutti gli $x \in \Class$,
tutte le permutazioni della scomposizione $\prod_{i \in I} \Permutation_i$
sono presenti nella scomposizione precedentemente costruita; e poich\'e
nessun prodotto di una sottofamiglia dei fattori della scomposizione precedente
ha $\Permutation$ per prodotto, le due scomposizioni coincidono necessariamente.
\EndProof
\begin{Corollary}
	Siano $\Class$ una classe finita e $\Permutation \in \Automorphisms{\Class}$.
	$\Permutation$ pu\`o essere scritto come prodotto di trasposizioni;
	pi\`u precisamente, ogni gruppo simmetrico finito \`e generato dalle sue
	trasposizioni.
\end{Corollary}
\Proof
Segue direttamente dal teorema precedente e dal fatto
che ogni ciclo pu\`o essere scritto come prodotto di
trasposizioni.
\EndProof
\begin{Theorem}
	Siano $\Class$ una classe e
	$\Permutation, \Permutation_1 \in \Automorphisms{Class}$.
	Se $\Permutation$ \`e un ciclo, allora il suo coniugato
	$\Permutation_1\Permutation\Permutation_1^{-1}$ \`e un ciclo
	di medesima lunghezza.
\end{Theorem}
\Proof
Sia $x \in \Class$.
Si verifica direttamente che
$x \in \Fixed{\Class}{\SpanGroup{\Permutation_1\Permutation\Permutation_1^{-1}}}$
se e solo se
$\Permutation_1^{-1}(x) \in \Fixed{\Class}{\SpanGroup{\Permutation}}$.
\par
Proviamo che
$\Class \SetMin
\Fixed{\Class}{\SpanGroup{\Permutation_1\Permutation\Permutation_1^{-1}}}$ costituisce
un'unica orbita per l'azione di
$\SpanGroup{\Permutation_1\Permutation\Permutation_1^{-1}}$.
Siano $x, y \in
\Class \SetMin
\Fixed{\Class}{\SpanGroup{\Permutation_1\Permutation\Permutation_1^{-1}}}$.
Poich\'e $\Permutation$ \`e un ciclo e per quanto appena provato riguardo ai punti fissi
di $\Permutation_1\Permutation\Permutation_1^{-1}$, esiste $n \in \mathbb{Z}$ tale che
$\Permutation_1^{-1}(y) = \Permutation^n\Permutation_1^{-1}(x)$, da cui
$y = \Permutation_1\Permutation^n\Permutation_1^{-1}(x)$ e quindi
$\Class \SetMin
\Fixed{\Class}{\SpanGroup{\Permutation_1\Permutation\Permutation_1^{-1}}}$ costituisce
l'unica orbita per l'azione di
$\SpanGroup{\Permutation_1\Permutation\Permutation_1^{-1}}$.
\par
L'uguaglianza della lunghezza dei due cicli segue dall'equicardinalit\`a
delle classi dei punti fissi.
\EndProof
\begin{Corollary}
	Sia $\Class$ una classe e sia
	$\Permutation \in \Automorphisms{\Class}$.
	Tutte le permutazioni coniugate a $\Permutation$
	si scompogono in cicli di ugual numero e uguali lunghezze.
\end{Corollary}
\Proof
Segue immediatamente dal teorema precedente,
considerando che la permutazione coniugata a $\Permutation$ tramite
una qualsiasi permutazione $\Permutation_1$ \`e il prodotto dei
coniugati dei fattori della scomposizione in cicli disgiunti di $\Permutation$
per mezzo della stessa permutazione $\Permutation_1$.
\EndProof
\begin{Definition}
	Sia $\Class$ un ordinamento totale
	e sia $\Permutation \in \Automorphisms{\Class}$.
	Definiamo \Define{inversione} una qualsiasi coppia
	$(x,y) \in \Class^2$ tale che
	$x < y$ e $\Permutation(x) > \Permutation(y)$.
\end{Definition}
\begin{Definition}
	Sia $\Class$ un ordinamento totale finito.
	Definiamo \Define{segno}[di una permutazione][segno] o
	\Define{segnatura}[di una permutazione][segnatura] o
	\Define{parit\`a}[di una permutationze][parit\`a]
	l'applicazione $\PermutationSign: \Automorphisms{\Class} \rightarrow
	\Invertible{\Quotient{\mathbb{Z}}{\SpanIdeal{3}}}$ che a
	$\Permutation \in \Automorphisms{\Class}$
	associa $\PermutationSign(\Permutation) =
	\begin{cases}
		1\text{, se il numero di inversioni di }\Permutation\text{ \`e pari};\\
		-1\text{, se il numero di inversioni di }\Permutation\text{ \`e dispari}.
	\end{cases}$
	Inoltre diciamo che una permutazione $\Permutation$ \`e
	\Define{pari}[pari][permutazione]
	quando $\PermutationSign{\Permutation} = 1$ e
	\Define{dispari}[dispari][permutazione]
	quando $\PermutationSign{\Permutation} = -1$.
\end{Definition}
\begin{Theorem}
	Sia $\Class$ un ordinamento totale finito
	e siano $\Permutation, \Transposition \in \Automorphisms{\Class}$.
	Se $\Transposition$ \`e una trasposizione, allora
	$\PermutationSign{\Transposition\Permutation} =
	- \PermutationSign{\Permutation}$.
\end{Theorem}
\Proof
Sia $\Transposition = \Cycle{i & j}$.
Tutte le inversioni di $\Permutation$ che non coinvolgono n\'e $i$ n\'e $j$
sussistono in $\Transposition\Permutation$.
Inoltre sussitono tutte le inversioni che coinvolgono $i$ e un elemento maggiore di $j$
e tutte le inversioni che coinvolgono $j$ e un elemento minore di $i$.
Restono solo da valutare tutte le inversioni che coinvolgono $i$ o $j$ con un elemento
$x \in [i,j]$.
\par
Supponiamo che le inversioni di $i$ con elementi di $[i,j]$ nella permutazione
$\Permutation$ siano $n_i$: nella permutazione $\Transposition\Permutation$
gli elementi di $[i,j]$ che costituiscono inversione con $i$ sono tutti e soli quelli
che non la costituiscono in $\Permutation$ e sono diversi da $i$; sono dunque in numero
$j - i - n_i$.
Analogamente, se $n_j$ \`e il numero di inversioni di $j$ con elementi di $[i,j]$,
allora le inversioni di $\Transposition\Permutation$ che coinvolgono $j$ e un elemento
di $[i,j]$ sono $j - i - n_j$.
La differenza tra il numero di inversioni di $\Permutation$ e quello di $\Transposition\Permutation$
\`e dunque $n_i + n_j - (j - i - n_i) - (j - i - n_j) \pm 1 =
2n_i + 2n_j - 2j + 2i \pm 1$ (il pi\`u o il meno dipende da se la coppia $(i,j)$
sia o meno un'inversione in $Permutation$), che \`e un numero dispari, da cui la tesi.
\EndProof
\begin{Corollary}
	Sia $\Class$ un ordinamento totale finito
	e sia $\Permutation \in \Automorphisms{\Class}$.
	$\Permutation$ \`e
	\begin{itemize}
		\item
		pari se e solo se si scrive come prodotto di un numero pari di trasposizioni;
		\item
		dispari se e solo se si scrive come prodotto di un numero dispari di trasposizioni.
	\end{itemize}
	In particolare, la parit\`a del numero di fattori in ogni scrittura di $\Permutation$
	come prodotto di trasposizione \`e la stessa.
\end{Corollary}
\Proof
Segue dal fatto che $\Permutation$ si scrive come prodotto di trasposizioni
e che ogni trasposizione \`e dispari.
\EndProof
\begin{Corollary}
	Sia $\Class$ un ordinamento totale finito.
	L'applicazione $\PermutationSign: \Automorphisms{\Class} \rightarrow
	\Invertible{\Quotient{\mathbb{Z}}{\SpanIdeal{3}}}$ \`e un omomorfismo.
\end{Corollary}
\Proof
Segue direttamente dal corollario precedente considerando
le permutazioni come prodotti di trasposizioni.
\EndProof
\begin{Definition}
	Chiamiamo $n$-esimo
	\Define{gruppo alternante}[alternante][gruppo]
	il nucleo dell'applicazione
	$\PermutationSign: \SymmetricGroup{n} \rightarrow
	\Invertible{\Quotient{\mathbb{Z}}{\SpanIdeal{3}}}$.
\end{Definition}
\begin{Theorem}
	Per ogni $n \in \mathbb{N}$,
	il gruppo alternante $\AlternatingGroup{n}$ \`e un
	sottogruppo normale di $\SymmetricGroup{n}$ di indice $2$.
\end{Theorem}
\Proof
La normalit\`a di $\AlternatingGroup{n}$ segue dal
fatto che il nucleo di un omomorfismo \`e sempre un sottogruppo
normale del suo dominio.
\par
Siano $\Permutation_1, \Permutation_2 \in \SymmetricGroup{n} \SetMin \AlternatingGroup{n}$.
$\Permutation_1$ e $\Permutation_2$ sono permutazioni dispari,
pertanto $\Permutation_1\Permutation_2^{-1} \in \AlternatingGroup{n}$, da cui
$\SymmetricGroup{n} \SetMin \AlternatingGroup{n}$ \`e l'unica classe laterale di
$\AlternatingGroup{n}$ oltre a $\AlternatingGroup{n}$ stesso.
\EndProof

\section{Gruppi di Sylow.}\label{GruppiDiSylow}
\begin{Definition}
	Dato $\Prime \in \mathbb{Z}$ primo,
	si chiama \Define{$\Prime$-gruppo} un qualsiasi gruppo
	in cui tutti gli elementi hanno per ordine una potenza di $\Prime$;
	inoltre, dato un gruppo finito $\Group$,
	chiamiamo \Define{$\Prime$-Sylow}
	un $\Prime$-sottogruppo di $\Group$ di ordine
	coprimo col suo indice.
\end{Definition}
\begin{Theorem}
	Sia $\Prime \in \mathbb{Z}$ primo.
	Sia $\Group$ un $\Prime$-gruppo.
	$\GroupCenter{\Group}$ non \`e banale.
\end{Theorem}
\Proof
Per l'equazione delle classi,
$\GroupOrder{\GroupCenter{\Group}} =
\GroupCenter{\Group} +
sum_{x \in R \SetMin \GroupCenter{\Group}}
\frac{\GroupOrder{\Group}}{\Stabilizer{x}}$,
dove l'azione \`e l'azione di coniugio e
$R$ \`e un sistema di rappresentanti delle orbite.
\par
Il membro sinistro e la sommatoria sono entrambi
congruenti a $0$ modulo $\Prime$, dunque lo stesso
deve valere anche per $\GroupCenter{\Group}$, che non
pu\`o dunque essere costituito da un solo elemento.
\EndProof
\begin{Corollary}
	Sia $\Prime \in \mathbb{Z}$ primo.
	Se $\Group$ \`e un $\Prime$-gruppo di ordine
	$\Prime^2$, allora $\Group$ \`e abeliano.
\end{Corollary}
\Proof
Per il teorema precedente, $\GroupCenter{\Group}$
non \`e banale.
\par
Supponiamo $\GroupOrder{\GroupCenter{\Group}} = p$.
Allora esiste $x \in \Group \SetMin \GroupCenter{\Group}$ e
$\GroupCenter{\Group}$ \`e un sottogruppo proprio
di $\Centralizer{x}$, il quale ha dunque cardinalit\`a
$\Prime^2$. Ma allora $\Centralizer{x} = \Group$ e
$x \in \GroupCenter{\Group}$, contro le ipotesi.
Dunque deve essere $\GroupCenter{\Group} = \Group$
e $\Group$ \`e abeliano.
\EndProof
\begin{Theorem}
\TheoremName{Teorema di Cauchy per i gruppi abeliani}[di Cauchy per i gruppi
abeliani][teorema]
	Sia $\Group$ un gruppo abeliano e sia
	$\Prime \in \mathbb{Z}$ primo tale che
	$\Prime | \GroupOrder{\Group}$.
	Esiste $x \in \Group$ tale che
	$\GroupOrder{x} = \Prime$.
\end{Theorem}
\Proof
Procediamo per induzione su $\GroupOrder{\Group}$.
\par
Per caso base, scegliamo $\GroupOrder{\Group} = \Prime$.
$\Group$ \`e allora isomorfo al gruppo ciclico
$\Quotient{\mathbb{Z}}{\SpanIdeal{\Prime}}$,
ove ogni elemento non nullo ha per ordine $\Prime$.
\par
Supponiamo il teorema provato per
$\Prime \leq \GroupOrder{\Group} < n \in \mathbb{Z}$
e proviamolo per $\GroupOrder{\Group} = n$.
Sia $x \in \Group$ tale che $x \neq 1$:
$\Prime$ divide $\GroupOrder{\SpanGroup{x}}$ o
$\GroupIndex{\Group}{\SpanGroup{x}}$.
\par
Se $\Prime | \GroupOrder{\SpanGroup{x}}$, allora
$x^{\frac{\GroupOrder{x}}{\Prime}}$ ha ordine $\Prime$.
\par
Se $\Prime | \GroupIndex{\Group}{\SpanGroup{x}}$,
allora, per ipotesi induttiva, esiste $y \in \Group$ tale che
$\Coset{y}{\SpanGroup{x}} \in \Quotient{\Group}{\SpanGroup{x}}$ abbia
ordine $\Prime$. Ma allora $\Prime | \GroupOrder{y}$ e quindi,
$y^{\frac{\GroupOrder{x}}{\Prime}}$ ha ordine $\Prime$.
\EndProof
\begin{Theorem}
	Sia $\Prime \in \mathbb{Z}$ primo e sia
	$\Group$ un $\Prime$-gruppo finito.
	Sia $n \in \mathbb{N}$ tale che $\Prime^n | \GroupOrder{\Group}$.
	Esiste un sottogruppo normale
	$\NormalSubgroup \IsNormalSubgroup \Group$ tale che
	$\GroupOrder{\NormalSubgroup} = \Prime^n$.
\end{Theorem}
\Proof
Se $n = 1$, la tesi segue immediatamente per applicazione del teorema di Cauchy
per i gruppi abeliani.
\par
Assumiamo la tesi provata per $n$ e proviamola per $n + 1$.
Assumiamo anche $\Prime^{n + 1} | \GroupOrder{\Group}$.
Poich\'e $\GroupCenter{\Group}$ \`e abeliano,
per il teorema di Cauchy esiste un sottogruppo $\NormalSubgroup$ di
$\GroupCenter{\Group}$ di ordine $\Prime$; inoltre abbiamo
$\NormalSubgroup \IsNormalSubgroup \Group$.
Quindi $\Quotient{\Group}{\NormalSubgroup}$ ha ordine
$\frac{\GroupOrder{\Group}}{\Prime}$.
\par
Per ipotesi induttiva, esiste
$\Subgroup \IsSubgroup \Quotient{\Group}{\NormalSubgroup}$
di ordine
$\GroupOrder{\Subgroup} = \Prime^n$.
Per il teorema di corrispondenza di gruppi,
$\bigcup_{X \in \Subgroup} X$ \`e un sottogruppo di $\Group$:
il suo ordine \`e
$\GroupOrder{\Subgroup} \GroupOrder{\NormalSubgroup} =
\Prime^n \Prime = \Prime^{n + 1}$.
\EndProof
\begin{Theorem}
\TheoremName{Primo teorema di Sylow}[primo di Sylow][teorema]
	Sia $\Group$ un gruppo finito.
	Per ogni $\Prime \in \mathbb{Z}$ primo tale che
	$\Prime | \GroupOrder{\Group}$, $\Group$ ammette
	un $\Prime$-Sylow $\Sylow$.
\end{Theorem}
\Proof
Siano $\Prime \in \mathbb{Z}$ primo e $n \in \mathbb{N}$
tale che $\Prime^n | \Cardinality{\Group}$, ma
$\Prime^{n + 1} \NotDivide \Cardinality{\Group}$. Poniamo
$m = \frac{\GroupOrder{\Group}}{\Prime^n}$.
\par
Sia $\Class$ l'insieme di tutti i sottoinsiemi di
$\Group$ di cardinalit\`a $\Prime^n$.
Consideriamo l'azione di moltiplicazione a sinistra
di $\Group$ su $\Class$: per ogni $\Operator \in \Group$
e $X \in \Class$ abbiamo $\Operator \cdot X =
\lbrace y \in \Group | \Exists{x \in X}
{\Operator x = y} \rbrace$.
\par
Ora, per ogni $X \in \Class$ e $x \in \Class$,
abbiamo, per ogni $\Operator \in \Stabilizer{X}$,
$\Operator x \in X$, da cui $\GroupOrder{\Stabilizer{X}} \leq
\Cardinality{X} = \Prime^n$.
\par
La cardinalit\`a di $\Class$ \`e data da
$\binom{\Prime^n m}{\Prime^n}$. Per il teorema binomiale,
$\binom{\Prime^n m}{\Prime^n}$ \`e il coefficiente
che accompagna il monomio $x^{Prime^n}y{\Prime^n m -\Prime^n}$ nel polinomio
$(x + y)^{\Prime^n m} \in \RingAdjunction{\mathbb{Z}{\SpanIdeal{\Prime}}}{x,y}$.
Abbiamo
$(x + y)^{\Prime^n m} =
(x^{\Prime^n} + y^{\Prime^n})^m$,
quindi, modulo $\Prime$, abbiamo
$\binom{\Prime^n m}{\Prime^n} \Congruent
\binom{m}{1} = m$.
Ne deduciamo che esiste $X \in \Class$ tale che
$\Cardinality{\Orbit{X}}$ non sia multipla di $\Prime$;
dunque $\GroupOrder{\Stabilizer{X}}$ contiene tanti fattori
$\Prime$ quanti $\GroupOrder{\Group}$.
Poich\'e abbiamo gi\`a provato che $\GroupOrder{\Stabilizer{X}} \leq
\Prime^n$, $\Stabilizer{X}$ \`e un $\Prime$-Sylow di $\Group$.
\EndProof
\begin{Corollary}
	Siano $\Group$ un gruppo finito,
	$n, \Prime \in \mathbb{Z}$ con
	$\Prime$ primo.
	Se $\Prime^n | \GroupOrder{\Group}$, allora
	$\Group$ ammette un sottogruppo $\Subgroup$
	di ordine $\GroupOrder{\Subgroup} = \Prime^n$.
\end{Corollary}
\Proof
Segue immediatamente dall'esistenza di un $\Prime$-Sylow e
dal fatto che ogni $\Prime$-gruppo finito ammette sottogruppi di ordine
$\Prime^n$ per tutti gli $n \in \mathbb{N}$ tali che $\Prime^n$ divida
l'ordine del $\Prime$-gruppo.
\EndProof
\begin{Corollary}
\TheoremName{Teorema di Cauchy per i gruppi}[di Cauchy per i gruppi][teorema]
	Siano $\Group$ un gruppo finito e
	$\Prime \in \mathbb{Z}$ un primo tale che
	$\Prime | \Cardinality{\Group}$.
	Esiste un elemento $x \in \Group$ di ordine $\Prime$.
\end{Corollary}
\Proof
L'enunciato del teorema coincide col caso $n = 1$ del corollario precedente.
\EndProof
\begin{Corollary}
	Sia $\Prime \in \mathbb{Z}$ primo.
	Sia $\Group$ un $\Prime$-gruppo.
	Esiste $n \in \mathbb{N}$ tale che
	$\GroupOrder{\Group} = \Prime^n$.
\end{Corollary}
\Proof
Se $\Prime' \in \mathbb{Z}$ \`e un primo tale che
$\Prime' | \GroupOrder{\Group}$,
allora per il teorema di Cauchy esiste in $\Group$ un elemento di
ordine $\Prime'$.
Quindi, per definizione di $\Prime$-gruppo,
deve essere $\Prime' = \Prime$.
\EndProof
\begin{Theorem}
\TheoremName{Secondo teorema di Sylow}[secondo di Sylow][teorema]
	Siano $\Group$ un gruppo finito e 
	$\Prime \in \mathbb{Z}$ un primo tale che
	$\Prime | \Cardinality{\Group}$.
	Tutti i $\Prime$-Sylow di $\Group$ sono coniugati tra loro.
\end{Theorem}
\Proof
Sia $\Sylow_1$ un $\Prime$-Sylow e
sia $\Set$ l'insieme di tutti i sottogruppi di $\Group$
coniugati a $\Sylow$.
Consideriamo l'azione di coniugio di $\Group$ su $\Set$.
\par
Poich\'e $\Stabilizer{\Sylow_1} = \Normalizer{\Sylow_1}$,
abbiamo $\Sylow_1 \IsSubgroup \Stabilizer{\Sylow_1}$ e dunque
$\Cardinality{\Set} =
\Cardinality{\Orbit{\Sylow_1}} =
\GroupIndex{\Group}{\Stabilizer{\Sylow_1}}$
\`e coprimo con $\Prime$.
\par
Sia ora $\Sylow_2$ un altro $\Prime$-Sylow e
consideriamo l'azione di coniugio di $\Sylow_2$ su $\Set$.
Abbiamo, per ogni $X \in \Set$,
$\Cardinality{\Orbit{x}} =
\frac{\GroupOrder{\Sylow_2}}{\Stabilizer{\Orbit{X}}} |
\GroupOrder{\Sylow_2}$.
Dunque tutte le orbite per l'azione di $\Sylow_2$ hanno per carinalit\`a
una potenza di $\Prime$. Poich\'e $\Set$ non \`e multiplo di $\Prime$ deve
esistere $X \in \Set$ tale che $\Orbit{X} = \lbrace X \rbrace$.
Pertanto, $\Sylow_2 \IsSubgroup \Normalizer{X}$
\par
Ora, poich\'e $\Sylow_2 \IsSubgroup \Normalizer{X}$, $\Sylow_2 X$ \`e un
$\Prime$-gruppo. Poich\'e $X$ \`e un $\Prime$-Sylow, non solo
$\Sylow_2 X = \Sylow_2$, ma anche $\Sylow_2 X = X$. Dunque $\Sylow_2 = X$
e $\Sylow_2$ risulta coniugato a $\Sylow_1$.
\EndProof
\begin{Theorem}
\TheoremName{Terzo teorema di Sylow}[terzo di Sylow][teorema]
	Sia $\Group$ un gruppo finito e
	$\Prime \in \mathbb{Z}$ un primo tale che
	$\Prime | \Cardinality{\Group}$.
	Sia $n$ il numero di $\Prime$-Sylow di $\Group$.
	Indicato con $\Sylow$ un qualsiasi $\Prime$-Sylow di $\Group$,
	abbiamo
	\begin{itemize}
		\item $n \Congruent 1 (\Modulo \Prime)$;
		\item $n = \Index{\Group}{\Normalizer{\Sylow}}$
	\end{itemize}
\end{Theorem}
\Proof
Riprendiamo le mosse dalla dimostrazione del secondo teorema di
Sylow. Abbiamo provato incidentalmente che $\Sylow_2$ \`e l'unico punto
fisso di dell'azione di $\Sylow_2$ sull'insieme $\Set$ e dunque che tutte
le orbite distinte da $\lbrace \Sylow_2 \rbrace$ hanno cardinalit\`a che
\`e una potenza di $\Prime$. Quindi, in particolare,
$n \Congruent 1 (\Modulo \Prime)$.
\par
Abbiamo gi\`a provato nella dimostrazione del teorema precedente che $n =
\Index{\Group}{\Normalizer{\Sylow}}$.
\EndProof

\section{Gruppi finiti di ordine piccolo.}\label{GruppiPiccoli}
\begin{Theorem}
	Sia $\Group$ un gruppo. Allora
	\begin{itemize}
		\item
		se $\GroupOrder{\Group} = 2$, allora $\Isomorphic{\Group}{\mathbb{Z}{2}}$;
		\item
		se $\GroupOrder{\Group} = 3$, allora $\Isomorphic{\Group}{\mathbb{Z}{3}}$;
		\item
		se $\GroupOrder{\Group} = 5$, allora $\Isomorphic{\Group}{\mathbb{Z}{5}}$;
		\item
		se $\GroupOrder{\Group} = 7$, allora $\Isomorphic{\Group}{\mathbb{Z}{7}}$.
	\end{itemize}
\end{Theorem}
\Proof
Segue direttamente dal fatto che $2$, $3$, $5$ e $7$ sono primi in $\mathbb{Z}$.
\EndProof


	\chapter{Teoria dei moduli}
	\section{Definizioni e nozioni di base.}\label{DefinizioniENozioniDiBase}
\begin{Definition}
	Dato un anello $\Ring$, chiamiamo \Define{modulo sinistro}[modulo] su $\Ring$
  o, pi\`u brevemente, $\Ring$-modulo sinistro il dato di
	\begin{itemize}
		\item un gruppo additivo $\Module$;
		\item un'azione sinistra di $\Ring$ su $\Module$.
	\end{itemize}
	Gli elementi di $\Ring$ si chiamano
  \Define{scalari}[scalare] e dunque l'anello $\Ring$ si chiama
  \Define{anello degli scalari}[degli scalari][anello].
  Se inoltre $\Ring$ \`e un campo, allora diciamo che $\Module$ \`e uno spazio
  vettoriale o un $\Ring$-spazio vettoriale.
  Gli elementi di uno spazio vettoriale si chiamano \Define{vettori}[vettore].
  Inoltre,
  \begin{itemize}
    \item se $\Ring = \mathbb{R}$, allora il modulo si dice
          \Define{reale}[reale][modulo];
    \item se $\Ring = \mathbb{C}$, allora il modulo si dice
          \Define{complesso}[complesso][modulo].
  \end{itemize}
\end{Definition}
\par Identificheremo sempre un modulo col gruppo soggiacente. Nel seguito,
salvo eccezioni esplicite, tutti i moduli saranno moduli sinistri.
\begin{Definition}
	Dati due moduli $\Module_1$ e $\Module_2$ su uno stesso anello $\Ring$ chiamiamo \Define{applicazione lineare}[lineare][applicazione] un omomorfismo dei gruppi additivi $\Module_1$ e $\Module_2$ compatibile con le rispettive azioni.
\end{Definition}
\begin{Theorem}
	Fissato un anello $\Ring$ le applicazioni lineari tra $\Ring$-moduli sinistri costituiscono una categoria.
\end{Theorem}
\Proof Basta provare che la composizione di applicazioni lineari tra $\Ring$-moduli sinistri \`e un'applicazione lineare. Siano dunque $\Linear_1$ e $\Linear_2$ due applicazioni lineari componibili. La loro composizione $\Linear_2 \circ \Linear_1$ \`e un omomorfismo di gruppi perch\'e gli omomorfismi di gruppi costituiscono una categoria ed \`e compatibile con le azioni di $\Dom{\Linear_1}$ e $\Cod{\Linear_2}$ perch\'e per ogni $\Scalar \in \Ring$ abbiamo $(\Linear_2 \circ \Linear_1)(\Scalar\ModuleElement) = \Linear_2(\Scalar\Linear_1(\ModuleElement)) = \Scalar((\Linear_2 \circ \Linear_1)(\ModuleElement))$. \EndProof
\begin{Definition}
	Dato un anello $\Ring$ chiamiamo \Define{categoria dei $\Ring$-moduli sinistri}[dei $\Ring$-moduli sinistri][categoria] la categoria $\CatLeftModule{\Ring}$ di tutte le applicazioni lineare tra $\Ring$-moduli sinistri. Se $\Ring$ \`e un campo chiamiamo ancora la stessa categoria \Define{categoria dei $\Ring$-spazi vettoriali}[dei $\Ring$-spazi vettoriali][categoria] e la denotiamo $\CatVectorialSpace{\Ring}$.
\end{Definition}
\begin{Definition}
	Siano $(\Module_i)_{i \in n}$ ($n \in \mathbb{N}$) una famiglia di $\Ring$-moduli e
	$\Module$ un ulteriore $\Ring$-modulo. Un'applicazione
	$\Multilinear: \bigtimes_{i \in n} \Module_i \rightarrow \Module$ si
	dice \Define{multilineare}[multilineare][applicazione] quando \`e
	lineare ogni sua $i$-esima ($i \in I$) applicazione parziale.
	Inoltre,
	\begin{itemize}
		\item se $\Module = \Ring$ allora $\Multilinear$ si chiama
		\Define{forma multilineare}[multilineare][forma];
		\item se $n = 2, 3, ...$, allora l'applicazione si dice
		rispettivamente \NIDefine{bilineare},
		\NIDefine{trilineare} ecc. o anche $n$-lineare;
		\item le forme lineari si chiamano anche
		\Define{funzionali}[funzionale] o, nel caso in cui $\Ring$ sia un campo e dunque $\Module_0$ uno spazio vettoriale, \Define{covettori}[covettore].
	\end{itemize}
	Denotiamo
	\begin{itemize}
		\item $\Multilinears{(\Module_i)_{i \in n}}{\Module}$ la classe di tutte le applicazioni multilineari di dominio $\bigtimes_{i \in n} \Module_i$ e codominio $\Module$;
		\item $\Homomorphisms{\Module_0}{\Module} = \Multilinears{\Module_0}{\Module}$;
		\item $\MultilinearForms{(\Module_i)_{i \in n}} = \Multilinears{(\Module_i)_{i \in n}}{\Ring}$;
		\item $\Dual{\Module} = \MultilinearForms{\Module}$; $\Dual{\Module}$ si chiama \Define{modulo duale}[duale][modulo] di $\Module$ e $\Dual{\Dual{\Module}}$ \Define{modulo biduale}[biduale][modulo].
	\end{itemize}
\end{Definition}
\begin{Definition}
	Siano $(\Module_i)_{i \in n}$ ($n \in \mathbb{N}$) e $\Module$ $\Ring$-moduli. Siano poi $(\Multilinear_0,\Multilinear_1,\Scalar) \in \Module^{\prod_{i \in n} \Module_i} \times \Module^{\prod_{i \in n} \Module_i} \times \Ring$. Definiamo
	\begin{itemize}
		\item la \Define{somma}[di applicazioni multilineari][somma] $\Multilinear_0 + \Multilinear_1$ come l'applicazione di $\Module^{\prod_{i \in n} \Module_i}$ che a $\ModuleElement \in \prod_{i \in n} \Module_i$ associa $\Multilinear_0(\ModuleElement) + \Multilinear_1(\ModuleElement) \in \Module$;
		\item il \Define{prodotto per scalare}[per scalare di applicazioni multilineari][prodotto] $\Scalar \Multilinear_0$ come l'applicazione di $\Module^{\prod_{i \in n} \Module_i}$ che a $\ModuleElement \in \prod_{i \in n} \Module_i$ associa $\Scalar \Multilinear_0(\ModuleElement) \in \Module$.
	\end{itemize}
\end{Definition}
\begin{Theorem}
	Siano $(\Module_i)_{i \in n}$ e $\Module$ $\Ring$-Moduli. $\Multilinears{(\Module_i)_{i \in n}}{\Module}$ con le operazioni di somma e prodotto per scalare della definizione precedenti \`e un $\Ring$-modulo.
\end{Theorem}
\Proof Segue direttamente dalle definizioni. \EndProof
\begin{Theorem}
	Sia $\Module$ un $\Ring$-modulo e sia $\VarModule \subseteq \Module$ non vuoto. $\VarModule$ \`e un sottomodulo di $\Module$ se e solo se
	\begin{itemize}
		\item $\ForAll{(\ModuleElement,\VarModuleElement) \in \Module^2}{\ModuleElement - \VarModuleElement \in \Module}$;
		\item $\ForAll{(\Scalar,\ModuleElement) \in \Ring \times \Module}{\Scalar\ModuleElement \in \Module}$.
	\end{itemize}
\end{Theorem}
\Proof La validit\`a delle due propriet\`a nel caso $\VarModule$ sia un modulo segue direttamente dalla definizione di modulo.
\par Viceversa, se $\VarModule$ verifica le due propriet\`a, allora per la prima \`e un gruppo abeliano e per la seconda la restrizione dell'azione di $\Ring$ su $\Module$ a $\VarModule$ \`e un'azione su $\VarModule$. \EndProof
\begin{Theorem}
	Sia $\Linear: \Module \rightarrow \Module'$ un'applicazione
	lineare tra i $\Ring$-moduli $\Module$ e $\Module'$. Sono
	$\Ring$-moduli $\Kernel{\Linear}$ e $\Image{\Linear}$.
\end{Theorem}
\Proof Poich\'e un'applicazione lineare \`e un omomorfismo di gruppi
abeliani, $\Kernel{\Linear}$ e $\Image{\Linear}$ sono gruppi abeliani.
\par Inoltre, dato $\Scalar \in \Ring$, abbiamo
\begin{itemize}
	\item per $\ModuleElement \in \Kernel{\Linear}$,
	$\Linear(\Scalar\ModuleElement) =
	\Scalar\Linear(\ModuleElement) = 0$, quindi
	$\Scalar\ModuleElement \in \Kernel{\Linear}$;
	\item per $\ModuleElement' \in \Image{\Linear}$,
	dato $\ModuleElement \in \Module$ tale che
	$\Linear(\ModuleElement) = \ModuleElement'$,
	$\Scalar\ModuleElement' =
	\Scalar\Linear(\ModuleElement) =
	\Linear(\Scalar\ModuleElement) \in \Image{\Linear}$. \EndProof
\end{itemize}
\begin{Definition}
	Siano $\Module$ e $\VarModule$ $\Ring$-moduli. Chiamiamo \Define{equazione lineare}[lineare][equazione] un'equazione della forma $\Linear(x) = \VarModuleElement$ dove
	\begin{itemize}
		\item $\Linear: \Module \rightarrow \VarModule$ \`e un'applicazione lineare;
		\item $x \in \Module$ \`e l'incognita;
		\item $\VarModuleElement \in \VarModule$.
	\end{itemize}
	L'equazione si dice \Define{omogenea}[lineare omogenea][equazione] quando $\VarModuleElement = 0$. Inoltre, data l'equazione lineare $\Linear(x) = \VarModuleElement$, chiamiamo \Define{equazione omogenea associata}[omogenea associata][equazione] l'equazione $\Linear(x) = 0$.
\end{Definition}
\begin{Theorem}
	Siano
	\begin{itemize}
		\item $\Module$ e $\VarModule$ $\Ring$-moduli;
		\item $\Linear: \Module \rightarrow \VarModule$ $\Ring$-moduli;
		\item $\VarModuleElement \in \VarModule$;
		\item $\ParticularSolution \in \Module$ una soluzione di $\Linear(x) = \VarModuleElement$, che chiameremo \Define{soluzione particolare}[particolare di un'equazione lineare][soluzione].
	\end{itemize}
	$\EqSolution \in \Module$ \`e soluzione di $\Linear(x) = \VarModuleElement$ se e solo se \`e della forma $\EqSolution = \ParticularSolution + \HomogeneousSolution$ per qualche soluzione $\HomogeneousSolution$ dell'equazioine omogenea associata.
\end{Theorem}
\Proof Se $\EqSolution = \ParticularSolution + \HomogeneousSolution$ per qualche soluzioen $\HomogeneousSolution$ dell'equazione omogenea associata, alloa $\Linear(\EqSolution) = \Linear(\ParticularSolution) + \Linear(\HomogeneousSolution) = \Linear(\ParticularSolution) + \Linear(\HomogeneousSolution) = \VarModuleElement$.
\par Se invece $\EqSolution$ \`e soluzione di $\Linear(x) = \VarModuleElement$, allora $\Linear(\EqSolution - \ParticularSolution) = \Linear(\EqSolution) - \Linear(\ParticularSolution) = \VarModuleElement - \VarModuleElement = 0$. \EndProof

\section{Algebre.}
\label{Moduli_Algebre}
\begin{Definition}
	Chiamiamo \Define{algebra} un $\Ring$-modulo su cui \`e definita un'operazione binaria interna $\cdot$ bilineare.
\end{Definition}

\section{Combinazioni lineare e famiglie libere}\label{CombinazioniLineariEFamiglieLibere}
\begin{Definition}
	Sia $\Module$ un $\Ring$-modulo. Date una famiglia finita
	$(\ModuleElement_i)_{i \in I}$ di elementi di $\Module$ e una
	famiglia di scalari $(\Scalar_i)_{i \in I}$ l'espressione
	$\sum_{i \in I} \Scalar_i \ModuleElement_i$ si chiama
	\Define{combinazione lineare}[lineare][combinazione]. In questo
	caso, gli scalari $\Scalar_i$ si chiamano
	\Define{coefficienti}[di una combinazione lineare][coefficiente].
\end{Definition}
\begin{Definition}
	Siano $\Module$ un $\Ring$-modulo, $(\ModuleElement_i)_{i \in I}$ una famiglia di elementi di $\Module$ e $(\Scalar_i)_{i \in I}$ una famiglia di scalari. La combinazione lineare $\sum_{i \in I} \Scalar_i \ModuleElement_i$ si chiama
	\begin{itemize}
		\item \Define{combinazione conica}[conica][combinazione] quando $\ForAll{i \in I}{\Scalar_i \geq 0}$ (si presume che $\Ring$ sia un anello su cui \`e definito un ordinamento);
		\item \Define{combinazione convessa}[convessa][combinazione] quando \`e conica e $\sum_{i \in I} = 1$.
	\end{itemize}
\end{Definition}
\begin{Definition}
	Sia $\Module$ un $\Ring$-modulo e sia $\Class \subseteq \Module$. $\Class$ si dice \Define{convesso}[convessa][classe] quando contiene ogni combinazione convessa di qualsiasi insieme finito non vuoto dei suoi punti.
\end{Definition}
\begin{Definition}
	Sia $\Class$ una parte convessa di un $\Ring$-modulo $\Module$. Si chiama \Define{punto estremo}[estremo][punto] un punto $\ExtremePoint \in \Class$ che non risulta essere combinazione convessa di due punti distinti di $\Class$.
\end{Definition}
\begin{Theorem}
	Sia $\Module$ un $\Ring$-modulo e sia $\Class \subseteq \Module$. $\Class$ \`e convesso se e solo se contiene ogni combinazione convessa di qualsiasi insieme di due suoi punti.
\end{Theorem}
\Proof Basta provare che se qualsiasi combinazione convessa di qualsiasi insieme di due punti di $\Class$ appartiene ancora a $\Class$, allora qualsiasi combinazione convessa di qualsiasi insieme finito non vuoto di punti di $\Class$ appartiene a $\Class$.
\par Sia $(\ModuleElement_i)_{i \in n}$ ($n \in \mathbb{N}$) una famiglia di elementi di $\Class$. Se $n \leq 2$ non c'\`e nulla da dimostrare.
\par Supponiamo il teorema dimostrato per $n = N$ e proviamolo per $n = N + 1$. Allora, data una famiglia di scalari $(\Scalar_i)_{i \in N + 1}$, abbiamo $\sum_{i \in N + 1} \Scalar_i \ModuleElement_i = \sum_{i \in N} \Scalar_i \ModuleElement_i + \Scalar_N \ModuleElement_N$.
\par Se $\Scalar_N = 0$, allora la combinazione convessa appartiene a $\Class$ per ipotesi induttiva, altrimenti essa \`e uguale a $(1 - \Scalar_N) \sum_{i \in N} \frac{\Scalar_i}{1 - \Scalar_N} \ModuleElement_i + \Scalar_N \ModuleElement_N$, che \`e una combinazione lineare di due punti di $\Class$ (il primo per ipotesi induttiva, il secondo per ipotesi). \EndProof
\begin{Theorem}
	Sia $\Module$ un $\Ring$-modulo e sia
	$\Class \subseteq \Module$. La classe di tutte le combinazioni
	lineari di elementi di $\Class$ \`e un sottomodulo di $\Module$.
\end{Theorem}
\Proof \`E sufficiente verificare direttamente che tale classe verifica
gli assiomi di modulo. \EndProof
\begin{Definition}
	Sia $\Module$ un $\Ring$-modulo e sia
	$\Class \subseteq \Module$. La classe
	\begin{itemize}
		\item di tutte le combinazioni lineari di elementi di $\Class$ si chiama
	\Define{modulo generato}[generato][modulo] da $\Class$, denotato
	$\SpanModule{\Class}$; se inoltre $\Class$ \`e un insieme finito, allora si dice che $\SpanModule{\Class}$ \`e \Define{finitamente generato}[finitamente generato][modulo];
		\item di tutte le combinazioni convesse di elementi di $\Class$ si chiama \Define{inviluppo convesso}[convesso][inviluppo] di $\Class$, denotato $\ConvexHull{\Class}$;
		\item di tutte le combinazioni coniche di elementi di $\Class$ si chiama \Define{inviluppo conico}[conico][inviluppo] da $\Class$, denotato $\ConicalHull{\Class}$;
	\end{itemize}
\end{Definition}
\begin{Definition}
	Sia $\Module$ un $\Ring$-modulo. $\Class \subseteq \Module$ si
	dice \Define{linearmente indipendente}[lineare][indipendenza] o
	\Define{libera}[libera][famiglia]
	quando, per ogni famiglia finita $(\ModuleElement_i)_{i \in I}$
	di elementi di $\Class$, dati scalari $(\Scalar_i)_{i \in I}$,
	abbiamo
	$\Implies{\sum_{i \in I} \Scalar_i \ModuleElement_i = 0}{\ForAll{
	i \in I}{\Scalar_i = 0}}$. Se $\Class$ non \`e linearmente
	indipendente, allora si dice che \`e
	\Define{linearmente dipendente}[lineare][dipendenza].
\end{Definition}
\begin{Definition}
	Sia $\Module$ un $\Ring$-modulo. Una famiglia
	$(\ModuleBase_i)_{i \in I}$ di elementi di $\Module$ si dice
	\Define{base}[di un modulo][base] quando
	\begin{itemize}
		\item \`e linearmente indipendente;
		\item genera $\Module$.
	\end{itemize}
\end{Definition}
\begin{Theorem}
	Sia $\Module$ un $\Ring$-modulo. Se 
	$(\ModuleBase_i)_{i \in I} \in \Module^I$ \`e una base di
	$\Module$ allora \`e anche una classe massimale linearmente
	indipendente.
\end{Theorem}
\Proof Sia $(\ModuleBase_i)_{i \in I}$ una base e sia $\ModuleElement \in
\Module$ qualsiasi. Esiste $(\Scalar_i)_{i \in I} \in \Module^I$ tale che
$\ModuleElement = \sum_{i \in I} \Scalar_i \ModuleBase_i$, dunque
$\sum_{i \in I} \Scalar_i \ModuleBase_i - \ModuleElement = 0$. \EndProof
\begin{Corollary}
	Sia $\Module$ un $\Ring$-modulo, con $\Ring$ corpo. Allora
	$(\ModuleBase_i)_{i \in I} \in \Module^I$ \`e una base di
	$\Module$ se e solo se \`e classe massimale di elementi
	linearmente indipendenti.
\end{Corollary}
\Proof Supponiamo $(\ModuleBase_i)_{i \in I}$ sia una classe massimale
linearmente indipendente. Allora, dato $\ModuleElement \in \Module$,
esistono scalari $(\Scalar_i)_{i \in I} \in \Module^I$ e $\Scalar \in
\Module$ non tutti nulli tali che $\sum_{i \in I} \Scalar_i \ModuleBase_i
+ \Scalar \ModuleElement = 0$. Necessariamente, $\Scalar \neq 0$, poich\'e
in caso contrario per l'indipendenza lineare di
$(\ModuleBase_i)_{i \in I}$, tutti gli $(\Scalar_i)_{i \in I}$ sarebbero
anch'essi nulli. Abbiamo dunque
$\ModuleElement = - \sum_{i \in I} \Scalar_i\Scalar^{-1} \ModuleBase_i$.
\EndProof
\begin{Theorem}
	Sia $\Module$ un $\Ring$-modulo, con $\Ring$ corpo. Se \`e vero
	l'assioma della scelta, allora $\Module$ ammette una base.
\end{Theorem}
\Proof Consideriamo la classe di tutte le parti di $\Module$ linearmente
indipendenti, ordinata per inclusione. Poich\'e l'unione di una catena di
parti linearmente indipendenti \`e ancora linearmente indipendente, ogni
catena ha un maggiorante e dunque, per il lemma di Zorn, esiste una parte
massimale $\ModuleBase$.
\par Dunque $\ModuleBase$ \`e una base. \EndProof
\begin{Corollary}
	Sia $\LinearSpace$ un $\Field$-spazio vettoriale. Se \`e vero
	l'assioma della scelta, $\LinearSpace$ ammette una base.
\end{Corollary}
\Proof $\Field$ \`e un corpo. \EndProof
\begin{Lemma}
	\TheoremName{Algoritmo di estrazione di base}
	Sia $(\ModuleElement_i)_{i \in n}$, con $n \in
	\NotZero{\mathbb{N}}$ una famiglia finita di generatori di un
	$\Ring$-modulo $\Module$. Esiste una base $\ModuleBase$ di
	$Module$ costituita esclusivamente da elementi di
	$(\ModuleElement_i)_{i \in n}$.
\end{Lemma}
\Proof Per ogni $i \in n$, poniamo
\[
  \ModuleBase_i =
  \begin{cases}
	  \lbrace \ModuleElement_0 \rbrace&\text{ se }i = 0,\\
  	\ModuleBase_{i - 1}&\text{ se }\And{i > 0}{\ModuleElement_i \in
	  	\SpanModule{\ModuleBase_{i - 1}}},\\
  	\ModuleBase_{i - 1} \cup \lbrace \ModuleElement_i \rbrace
	  	&\text{ se }\And{i > 0}{\ModuleElement_i \notin
  		\SpanModule{\ModuleBase_{i - 1}}}.
  \end{cases}
\]
Infine, poniamo $\ModuleBase = \ModuleBase_{n - 1}$. Per costruzione,
$\ModuleBase$ \`e costituito solo da elementi di
$(\ModuleElement_i)_{i \in n}$, \`e lineramente indipendente e ogni
generatore escluso \`e combinazione lineare dei generatori inclusi, per
cui $\ModuleBase$ genera $\Module$. Quindi $\ModuleBase$ \`e una base.
\EndProof
\begin{listing}
	\insertcode{octave}{"Moduli/EstraiBase.m"}
	\caption{Implementazione dell'algoritmo di estrazione di base in \LanguageName{octave}
    per uno spazio vettoriale reale o complesso.}
\end{listing}
\begin{Corollary}
	Sia $\Module$ un $\Ring$-modulo finitamente generato. Allora
	$\Module$ ammette una base finita.
\end{Corollary}
\Proof Segue immediatamente dal teorema precedente.
\begin{Lemma}
	\TheoremName{Algoritmo di completamento a base}
	Sia $\Module$ un $\Ring$-modulo finitamente generato e siano
	\begin{itemize}
		\item $\Set_1 \subseteq \Module$ finito e linearmente
		indipendente;
		\item $\Set_2 \subseteq \Module$ finito e tale che
		$\SpanModule{\Set_2} = \Module$.
	\end{itemize}
	Allora esiste una base $\ModuleBase$ di $\Module$ tale che
	$\Set_1 \subseteq \ModuleBase \subseteq \Set_1 \cup \Set_2$.
\end{Lemma}
\Proof Siano $(\ModuleElement_i)_{i \in n}$ ($n \in \mathbb{N}$) gli
elementi di $\Set_2$ e poniamo, per ogni $i \in \mathbb{N}$,
\[
  \ModuleBase_i =
  \begin{cases}
	  \Set_1&\text{ se }i = 0,\\
  	\ModuleBase_{i - 1}&\text{ se }\And{i > 0}{\ModuleElement_i \in
	  	\SpanModule{\ModuleBase_{i - 1}}},\\
  	\ModuleBase_{i - 1} \cup \lbrace \ModuleElement_i \rbrace
	  	&\text{ se }\And{i > 0}{\ModuleElement_i \notin
		  \SpanModule{\ModuleBase_{i - 1}}}.
  \end{cases}
\]
Infine, poniamo $\ModuleBase = \ModuleBase_{n - 1}$. Per costruzione,
$\ModuleBase$ \`e costituito solo da elementi di
$\Set_1 \cup \Set_2$, contiene tutti gli elementi di $\Set_1$, \`e
lineramente indipendente e ogni generatore escluso \`e combinazione
lineare dei generatori inclusi e degli elementi di $\Set_1$, per
cui $\ModuleBase$ genera $\Module$. Quindi $\ModuleBase$ \`e una base.
\EndProof
\begin{Lemma}
	\TheoremName{Lemma di Steinitz} Sia $\Module$ un $\Ring$-modulo,
	con $\Ring$ corpo, e siano $(\ModuleBase_i)_{i \in n}$ e
	$(\ModuleElement_j)_{j \in m}$, con $n, m \in \mathbb{N}$, tali
	che
	\begin{itemize}
		\item $(\ModuleBase_i)_{i \in n}$ sia linearmente
		indipendente;
		\item $\ModuleElement_j)_{j \in m}$ generi $\Module$.
	\end{itemize}
	Allora
	\begin{itemize}
		\item $n \leq m$;
		\item esiste un insieme $\Set$ tale che
		\begin{itemize}
			\item
		$\bigcup_{i \in n} \lbrace \ModuleBase_i \rbrace \subseteq
		\Set \subseteq
		\bigcup_{i \in n} \lbrace \ModuleBase_i \rbrace \cup
		\bigcup_{j \in m} \lbrace \ModuleElement_j \rbrace$
			\item $\Cardinality{\Set} = m$;
			\item $\SpanModule{\Set} = \Module$.
		\end{itemize}
	\end{itemize}
\end{Lemma}
\Proof Consideriamo $\Class = \bigcup_{j \in m} \lbrace \ModuleElement_j
\rbrace \cup \lbrace \ModuleBase_0 \rbrace$. Poich\'e $(\ModuleElement_j)_{j \in m}$
genera $\Module$, esistono scalari $(\Scalar_j)_{j \in m} \in \Ring^m$ non
tutti nulli tali che
$\sum_{j \in m} \Scalar_j \ModuleElement_j + \ModuleBase_0 = 0$. Sia
$k \in m$ tale che $\Scalar_k \neq 0$, abbiamo allora
$\ModuleElement_k =
- sum_{\And{j \in m}{j \neq k}} \Scalar_j\Scalar_k^{-1} \ModuleElement_j -
\Scalar_k^{-1} \ModuleBase_0$.
Dunque anche
$\bigcup_{\And{j \in m}{j \neq k}} \lbrace \ModuleElement_j \rbrace \cup
\lbrace \ModuleBase_0 \rbrace$ genera $\Module$: abbiamo sostituito un
elemento di $(\ModuleElement_j)_{j \in m}$ con uno di
$(\ModuleBase_i)_{i \in n}$. Buttiamo via l'elemento originale
$\ModuleElement_k$ e rinominiamo $\ModuleElement_k$ l'elemento
originale $\ModuleBase_0$. Rinomiano inoltre $\ModuleBase_{i - 1}$ ogni
$\ModuleBase_i$ con $i \in \NotZero{n}$.
\par Il processo \`e a questo punto ripetibile per ogni
elemento di $(\ModuleBase_i)_{i \in \NotZero{n}}$. Inoltre, l'indipendenza
lineare dei $(\ModuleBase_i)_{i \ n}$ originale garantisce
la possibilit\`a di scegliere $k$ in ogni passaggio in modo tale che
$\ModuleElement_k$ sia membro della famiglia
$(\ModuleElement_j)_{j \in m}$ originale. Avvalendoci di questa
possibilit\`a, riusciamo, per ogni $i \in n$, a sostituire un elemento
della famiglia $(\ModuleElement_j)_{j \in m}$ originale con
$\ModuleBase_i$ in modo tale che la nuova famiglia
$(\ModuleElement_j)_{j \in m}$
\begin{itemize}
	\item contenga tutti gli elementi della famiglia
	$(\ModuleBase_i)_{i \in n}$ originale;
	\item sia composta esattamente da $m$ elementi (e dunque
	$n \leq m$);
	\item abbia i restanti $m - n$ elementi apparteneti alla famiglia
	$(\ModuleElement_j)_{j \in m}$ originale;
	\item generi $\Module$. \EndProof
\end{itemize}
\begin{Corollary}
	Sia $\Module$ un $\Ring$-modulo, con $\Ring$ corpo,
	finitamente generato. Tutte le basi sono equicardinali.
\end{Corollary}
\Proof Siano $(\ModuleBase_i)_{i \in n}$ e $(\ModuleElement_j)_{j \in m}$
due basi di $\Module$. Per il lemma di Steinitz, abbiamo sia $n \leq m$
che $m \leq n$, da cui $n = m$. \EndProof
\begin{Theorem}
	Sia $\Module$ un $\Ring$-modulo. Se $\Module$ ammette una base,
	allora esso \`e libero e generato, come modulo libero, dalla base
	stessa.
\end{Theorem}
\Proof Sia $(\ModuleBase_i)_{i \in I}$ una base di $\Module$. Per ogni
applicazione lineare $\Linear: \Module \rightarrow \Module'$, dove
$\Module'$ \`e un modulo qualsiasi, e per ogni
$\ModuleElement \in \Module$, fissati scalari $(\Scalar_i)_{i \in I}$ tali
che $\ModuleElement = \sum_{i \in I} \Scalar_i\ModuleBase_i$, abbiamo
$\Linear(\ModuleElement) =
\Linear(\sum_{i \in I} \Scalar_i\ModuleBase_i) =
\sum_{i \in I} \Scalar_i \Linear(\ModuleBase_i)$. \EndProof
\begin{Theorem}
	Sia $\Module$ un $\Ring$-modulo che ammette una base
	$(\ModuleBase_i)_{i \in I}$. Per ogni $\ModuleElement \in
	\Module$, esistono unici scalari $(\Scalar_i)_{i \in I}$ tali che
	$\ModuleElement = \sum_{i \in I} \Scalar_i \ModuleBase_i$.
\end{Theorem}
\Proof Supponiamo esistano due famiglie di scalari $(\Scalar_i)_{i \in I}$
e $(\Scalar'_i)_{i \in I}$ tali che $\ModuleElement =
\sum_{i \in I} \Scalar_i \ModuleBase_i =
\sum_{i \in I} \Scalar'_i \ModuleBase_i$. Allora abbiamo
$\sum_{i \in I} (\Scalar_i - \Scalar'_i) \ModuleBase_i = 0$ e, per
l'indipendenza lineare di $(\ModuleBase_i)_{i \in I}$, necessiamente
$(\ForAll{i \in I}{\Scalar_i = \Scalar'_i}$. \EndProof
\begin{Definition}
	Sia $\Module$ un $\Ring$-modulo che ammette una base
	$(\ModuleBase_i)_{i \in I}$. Dato $\ModuleElement \in \Module$,
	gli unici scalari $(\Scalar_i)_{i \in I}$ tali che
	$\ModuleElement = \sum_{i \in I} \Scalar_i \ModuleBase_i$ si
	chiamano \Define{coordinate} di $\ModuleElement$ rispetto a
	$(\ModuleBase_i)_{i \in I}$.
\end{Definition}

\section{Coni convessi e lineari.}\label{ConiConvessiELineari}
\begin{Definition}
	Sia $\Module$ un $\Ring$-modulo e sia $\Class \subseteq \Module$. $\Class$ si chiama
	\begin{itemize}
		\item \Define{cono lineare}[lineare][cono] se contiene ogni combinazione lineare di ogni insieme costituito da un solo punto di $\Class$; 
		\item \Define{cono convesso}[convesso][cono] se contiene ogni combinazione convessa di ogni insieme costituito da due soli punti di $\Class$. 
	\end{itemize}
\end{Definition}
\begin{Definition}
	Un cono lineare che \`e inviluppo conico di un numero finito di vettori si dice \Define{finitamente generato}[finitamente generato][cono].
\end{Definition}
\begin{Theorem}
	Un cono convesso \`e una parte convessa.
\end{Theorem}
\Proof Segue direttamente dalla definizione di cono convesso. \EndProof
\begin{Theorem}
	Sono vere le seguenti affermazioni:
	\begin{itemize}
		\item un cono convesso \`e un cono lineare;
		\item un inviluppo conico \`e un cono convesso;
		\item un cono finitamente generato \`e un cono convesso.
	\end{itemize}
\end{Theorem}
\Proof Sia $\ConvexCone$ un cono convesso e siano $\ModuleElement \in \ConvexCone$ e $\Scalar \in \Ring$, dove $\Ring$ \`e l'anello degli scalari del modulo in cui giace il cono convesso. Abbiamo $\Scalar\ModuleElement = \frac{\Scalar}{2}\ModuleElement + \frac{\Scalar}{2}\ModuleElement$.
\par La seconda e la terza affermazione seguono direttamente dalle definizioni. \EndProof

\section{Applicazioni multilineari.}\label{ApplicazioniMultilineari}
\begin{Definition}
	Siano $(\Module_i)_{i \in n}$ ($n \in \mathbb{N}$) una famiglia di $\Ring$-moduli e
	$\Module$ un ulteriore $\Ring$-modulo. L'applicazione
	$\Multilinear: \bigtimes_{i \in n} \Module_i \rightarrow \Module$ si si dice \Define{antisimmetrica}[multilineare antisimmetrica][forma] quando, per ogni coppia $(j,k) \in n^2$ tale che $j \neq k$ e ogni $(x_i)_{i \in \mathbb{N}}, (y_i)_{i \in \mathbb{N}} \in \bigtimes_{i \in n} \Module_i$  tali che $x_i = \begin{cases} y_j\text{, se } i = k,\\ y_k\text{, se } i = j,\\ y_i\text{ altrimenti,}\end{cases}$ abbiamo $\Multilinear((x_i)_{i \in n}) = - \Multilinear((y_i)_{i \in n})$.
\end{Definition}
\begin{Definition}
	Siano $(\Module_i)_{i \in n}$ ($n \in \mathbb{N}$) $\Ring$-moduli.
	Sia $\AlternatingForm \in \MultilinearForms{(\Module_i)_{i \in n}}$ ($n \in \mathbb{N}$) una forma multilineare sui $\Ring$-moduli $(\Module_i)_{i \in n}$. $\AlternatingForm$ si dice \Define{alternante}[multilineare alternante][forma] quando per ogni $(x_i)_{i \in n} \in \bigtimes_{i \in n} \Module_i$ tale che $\Exists{(j,k) \in n^2}{\And{j \neq k}{x_j = x_k}}$ abbiamo $\MultilinearForm((x_i)_{i \in n}) = 0$. Denotiamo $\AlternatingForms{(\Module)_{i \in n}}$ la classe di tutte le forme multilineari alternanti sui $\Ring$-moduli $(\Module_i)_{i \in n}$.
\end{Definition}
\begin{Theorem}
	Siano $(\Module_i)_{i \in n}$ ($n \in \mathbb{N}$) $\Ring$-moduli. Per ogni $k \in \NotZero{\mathbb{N}}$ le forme multilineari alternanti su $(\Module_i)_{i \in k}$ costituiscono un sottomodulo di $\MultilinearForms{(\Module_i)_{i \in k}}$.
\end{Theorem}
\Proof Siano $\AlternatingForm, \VarAlternatingForm \in \AlternatingForms{(\Module_i)_{i \in n}}$. Sia inoltre $\Scalar \in \Field$. Si verifica immediatamente che $\AlternatingForm - \VarAlternatingForm, \Scalar\AlternatingForm \in \AlternatingForms{(\Module_i)_{i \in n}}$. \EndProof
\begin{Theorem}
	Siano $(\Module_i)_{i \in n}$ ($n \in \mathbb{N}$) $\Ring$-moduli.
	Sia $\AlternatingForm \in \MultilinearForms{(\Module_i)_{i \in n}}$ ($n \in \mathbb{N}$) una forma multilineare alternante sui $\Ring$-moduli $(\Module_i)_{i \in n}$. $\AlternatingForm$ \`e antisimmetrica.
\end{Theorem}
\Proof Siano $(x_i)_{i \in \mathbb{N}}, (y_i)_{i \in \mathbb{N}} \in \bigtimes_{i \in n} \Module_i$  tali che $x_i = \begin{cases} y_j\text{, se } i = k,\\ y_k\text{, se } i = j,\\ y_i\text{ altrimenti,}\end{cases}$ per opportuni $j,k \in n^2$ distinti.
\par Poniamo, per ogni $i \in n$, $z_i = \begin{cases} x_j + x_k\text{, se } i \in \lbrace j, k,\\ x_i\text{ altrimenti.}\end{cases}$ Abbiamo $\AlternatingForm((z_i)_{i \in n}) = 0$, ma anche $\AlternatingForm((z_i)_{i \in n}) = \AlternatingForm((x_i)_{i \in n}) + \AlternatingForm((y_i)_{i \in n})$, da cui $\AlternatingForm((x_i)_{i \in n}) = - \AlternatingForm((y_i)_{i \in n})$. \EndProof
\begin{Theorem}
	Siano $(\Module_i)_{i \in n}$ ($n \in \mathbb{N}$) $\Ring$-moduli.
	Sia $\AlternatingForm \in \MultilinearForms{(\Module_i)_{i \in n}}$ ($n \in \mathbb{N}$) una forma multilineare antisimmetrica sui $\Ring$-moduli $(\Module_i)_{i \in n}$. Se $\Characteristic{\Ring} \neq 2$, allora $\AlternatingForm$ \`e alternante.
\end{Theorem}
\Proof Sia $(x_i)_{i \in n} \in \bigtimes_{i \in n} \Module_i$ tale che $\Exists{(j,k) \in n^2}{\And{j \neq k}{x_j = x_k}}$. Abbiamo $\AlternatingForm((x_i)_{i \in n}) = - \AlternatingForm((x_i)_{i \in n})$ e quindi $2 \AlternatingForm((x_i)_{i \in n}) = 0$ e dunque $\AlternatingForm((x_i)_{i \in n}) = 0$. \EndProof
\begin{Theorem}
	Siano $(\Module_i)_{i \in n}$ ($n \in \mathbb{N}$) $\Ring$-moduli.
	Sia $\AlternatingForm \in \AlternatingForms{(\Module_i)_{i \in n}}$. Sia $(\ModuleElement_i)_{i \in n} \in \bigtimes_{i \in n} \Module_i$. Ora
	\begin{itemize}
		\item se $(\ModuleElement_i)_{i \in n}$ \`e una famiglia di vettori dipendenti, allora $\AlternatingForm((\ModuleElement_i)_{i \in n}) = 0$;
		\item se $\Module$ \`e un $\Ring$-modulo tale che $\ForAll{i \in n}{\Module = \Module_i}$ e $(\ModuleElement_i)_{i \in n}$ \`e una base, allora $\AlternatingForm((\ModuleElement_i)_{i \in n}) = 0$ se e solo se $\AlternatingForm = 0$.
	\end{itemize}
\end{Theorem}
\Proof Supponiamo $(\ModuleElement_i)_{i \in n}$ sia una famiglia di vettori dipendenti. Sia $j \in n$ tale che per opportuni scalari $(\Scalar_i)_{\substack{i \in n\\i \neq j}} \in \Field^{n - \lbrace j \rbrace}$, $\ModuleElement_j = \sum_{\substack{i \in n,\\ i \neq j}} \Scalar_i \ModuleElement_i$. Abbiamo $\AlternatingForm(\ModuleElement_0, ..., \ModuleElement_j, ..., \ModuleElement_{n - 1}) = \sum_{\substack{i \in n\\i \neq j}} \Scalar_i \AlternatingForm(\ModuleElement_0, ..., \ModuleElement_i, ..., \ModuleElement_{n - 1}) = 0$.
\par Supponiamo ora invece che i $(\Module_i)_{i \in n}$ siano tutti uguali a un unico $\Ring$-modulo $\Ring$ di cui $(\ModuleElement_i)_{i \in n}$ sia una base. Allora, per ogni famiglia di elementi $(\VarModuleElement_i)_{i \in n}$, $\AlternatingForm((\VarModuleElement_i)_{i \in n})$ \`e combinazione lineare di termini della forma $\AlternatingForm((\ModuleElement_{\Permutation(i)})_{i \in n})$ con $\Permutation \in \SymmetricGroup{n}$, i quali sono per ipotesi tutti nulli. Dunque $\AlternatingForm = 0$. \EndProof

\section{Prodotti tensoriali.}\label{ProdottiTensoriali}
\begin{Definition}
	Siano $(\Module_i)_{i \in I}$ $\Ring$-moduli, con $\Ring$ anello commutativo. Chiamiamo \Define{prodotto tensoriale}[tensoriale][prodotto] dei $(\Module_i)_{i \in I}$ una coppia costituita da un $\Ring$-modulo denotato $\bigotimes_{i \in I} \Module_i$ e un'applicazione $\otimes \in \Multilinears{(\Module_i)_{i \in I}}{\bigotimes_{i \in I} \Module_i}$ che sia universale da $\bigtimes_{i \in I} \Module_i$ nel funtore $\Functor: \CatModule \rightarrow \CatClass$ che ad ogni $\Linear \in \CatModule$ associa $\Functor(\Linear): \Dom{\Linear} \rightarrow f$, dove $f$ \`e l'applicazione $f: \Multilinears{(\Module_i)_{i \in I}}{\Dom{\Linear}} \rightarrow \Multilinears{(\Module_i)_{i \in I}}{\Cod{\Linear}}$ che a $\Multilinear \in \Multilinears{(\Module_i)_{i \in I}}{\Dom{\Linear}}$ associa $\Linear \circ \Multilinear \in \Multilinears{(\Module_i)_{i \in I}}{\Cod{\Linear}}$.
\end{Definition}
\begin{Theorem}%Questo teorema \`e la definizione pi\`u comune di prodotto tensoriale e dunque \`e quasi certamente giusto. Se non fosse coerente con la definizione data sopra, sospettare maggiormente della definizione. 
	Siano $(\Module_i)_{i \in  I}$ $\Ring$-moduli, con $\Ring$ anello commutativo. Siano $\Module$ un ulteriore $\Ring$ modulo e $\Multilinear_\otimes \in \Multilinears{(\Module_i)_{i \in I}}{\Module}$. $(\Module,\Multilinear_\otimes)$ \`e prodotto tensoriale dei $(\Module_i)_{i \in I}$ se e solo se per ogni $\Ring$-modulo $\VarModule$ e $\Multilinear \in \Multilinears{(\Module_i)_{i \in I}}{\VarModule}$ esiste una e una sola applicazione lineare $\Linear$ tale che $\Linear \circ \Multilinear_\otimes = \Multilinear$.
\end{Theorem}
\begin{Theorem}
	Siano $(\Module_i)_{i \in n}$ ($n \in \mathbb{N}$) $\Ring$-moduli, con $\Ring$ anello commutativo. Il prodotto tensoriale $\bigotimes_{i \in n} \Module_i$ esiste ed \`e unico a meno di isomorfismo.
\end{Theorem}
\Proof 
\par L'unicit\`a segue direttamente dall'universalit\`a.
\begin{Theorem}\label{th_tensor_product_properties}
	Il prodotto tensoriale, visto come operazione tra moduli, \`e
	\begin{itemize}
		\item commutativo;
		\item associativo;
		\item distributivo rispetto alla somma diretta.
	\end{itemize}
\end{Theorem}
\Proof Siano $(\Module_i)_{i \in I}$ $\Ring$-moduli, con $\Ring$ anello commutativo.
\par La commutativit\`a segue dalla commutativit\`a del diagramma \ref{tensor_product_commutative}, con $\Permutation \in \Automorphisms{I}$.
\begin{figure}
	\center
	\begin{tikzcd}[column sep=6em, row sep=6em]
		{\prod_{i \in I} \Module_i} \arrow{r}{(\ModuleElement_i)_{i \in I} \mapsto (\ModuleElement_{\Permutation(i)})_{i \in I}} \arrow{d}{\otimes} & {\prod_{i \in I} \Module_{\Permutation(i)}} \arrow{r}{(\ModuleElement_{\Permutation(i)})_{i \in I} \mapsto (\ModuleElement_i)_{i \in I}}\arrow{d}{\otimes} & {\prod_{i \in I} \Module_i} \arrow{d}{\otimes}\\
		{\bigotimes_{i \in I} \Module_i} \arrow{r}{} & {\bigotimes_{i \in I} \Module_{\Permutation(i)}} \arrow{r}{} & {\bigotimes_{i \in I} \Module_i}
	\end{tikzcd}
	\caption{Dimostrazione del teorema \ref{th_tensor_product_properties}: commutativit\`a.}
	\label{tensor_product_commutative}
\end{figure}
%\par L'associativit\`a segue dalla commutativit\`a del diagramma \ref{tensor_product_associative}.
%\begin{figure}
%	\center
%	\begin{tikzcd}[column sep=6em, row sep=6em]
%		{(\Module_0 \times \Module_1) \times \Module_2} \arrow{r}{((\ModuleElement_0,\ModuleElement_1),\ModuleElement_2) \mapsto (\ModuleElement_0,(\ModuleElement_1,\ModuleElement_2))} \arrow{d}{((\ModuleElement_0,\ModuleElement_1),\ModuleElement_2) \mapsto (\ModuleElement_0 \otimes \ModuleElement_1, \ModuleElement_2)} &
%		{\Module_0 \times (\Module_1 \times \Module_2)} \arrow{r}{(\ModuleElement_0,(\ModuleElement_1,\ModuleElement_2)) \mapsto ((\ModuleElement_0,\ModuleElement_1),\ModuleElement_2))} \arrow{d}{(\ModuleElement_0,(\ModuleElement_1,\ModuleElement_2)) \mapsto (\ModuleElement_0, \ModuleElement_1 \otimes \ModuleElement_2)} &
%		{(\Module_0 \times \Module_1) \times \Module_2} \arrow{d}{((\ModuleElement_0,\ModuleElement_1),\ModuleElement_2) \mapsto (\ModuleElement_0 \otimes \ModuleElement_1, \ModuleElement_2)}\\
		
%		{(\Module_0 \otimes \Module_1) \times \Module_2} \arrow{r}{((\ModuleElement_0,\ModuleElement_1),\ModuleElement_2) \mapsto (\ModuleElement_0,(\ModuleElement_1,\ModuleElement_2))} \arrow{d}{((\ModuleElement_0,\ModuleElement_1),\ModuleElement_2) \mapsto (\ModuleElement_0 \otimes \ModuleElement_1, \ModuleElement_2)} &
%		{\Module_0 \times (\Module_1 \times \Module_2)} \arrow{r}{(\ModuleElement_0,(\ModuleElement_1,\ModuleElement_2)) \mapsto ((\ModuleElement_0,\ModuleElement_1),\ModuleElement_2))} \arrow{d}{(\ModuleElement_0,(\ModuleElement_1,\ModuleElement_2)) \mapsto (\ModuleElement_0, \ModuleElement_1 \otimes \ModuleElement_2)} &
%		{(\Module_0 \times \Module_1) \times \Module_2} \arrow{d}{((\ModuleElement_0,\ModuleElement_1),\ModuleElement_2) \mapsto (\ModuleElement_0 \otimes \ModuleElement_1, \ModuleElement_2)}\\
%	\end{tikzcd}
%	\caption{Dimostrazione del teorema \ref{th_tensor_product_properties}: associativit\`a.}
%	\label{tensor_product_associative}
%\end{figure}

\section{Proiezioni.}\label{Proiezioni}
\begin{Definition}
	Si chiama \Define{proiezione} o \Define{proiettore} un endomorfismo
  lineare idempotente.
\end{Definition}
\begin{Theorem}
	Sia $\Projection$ una proiezione sul $\Ring$-modulo $\Module$. $\Module$ \`e isomorfo a $\Kernel{\Projection} \oplus \Image{\Projection}$.
\end{Theorem}
\Proof Sia $\ModuleElement \in \Module$. Abbiamo $\ModuleElement = (\ModuleElement - \Projection(\ModuleElement)) + \Projection(\ModuleElement))$. Inoltre $\Projection(\ModuleElement - \Projection(\ModuleElement)) = \Projection(\ModuleElement) - \Projection^2(\ModuleElement) = \Projection(\ModuleElement) - \Projection(\ModuleElement) = 0$. Quindi $\ModuleElement$ si scrive come somma di un elemento di $\Kernel{\Projection}$ e di uno di $\Image{\Projection}$.
\par Sia ora $\ModuleElement \in \Kernel{\Projection} \cap \Image{\ModuleElement}$. Abbiamo $\Projection(\ModuleElement) = 0$ e, per opportuno $\ModuleElement_0 \in \Module$, $\Projection(\ModuleElement_0) = \ModuleElement$. Quindi $\ModuleElement = \Projection(\ModuleElement_0) = \Projection^2(\ModuleElement_0) = \Projection(\ModuleElement) = 0$. \EndProof
\begin{Definition}
	Sia $\Projection$ una proiezione. Diciamo che $\Projection$ \`e una \Define{proiezione}[su un sottomodulo lungo un altro sottomodulo][proiezione] sul sottospazio $\Image{\Projection}$ lungo $\Kernel{\Projection}$.
\end{Definition}

\section{Annullatori.}
\label{TeoriaDeiModuli_Annullatori}
\begin{Definition}
	Sia $\Module$ un $\Ring$-modulo. Definiamo \Define{annullatore} di $\Part \subseteq \Module$, denotato $\Annihilator{\Part}$, la classe di tutti i funzionali di $\Module$ che si annullanno in tutti i punti di $\Part$.
\end{Definition}
\begin{Theorem}
	Con le notazioni della definizione precedente, $\Annihilator{\Part}$ \`e un sottomodulo di $\Dual{\Module}$.
\end{Theorem}
\Proof Siano $\Functional, \VarFunctional \in \Dual{\Module}$ e $\Scalar \in \Ring$. Abbiamo, per ogni $x \in \Part$,
\begin{itemize}
	\item $(\Functional - \VarFunctional)(x) = \Functional(x) - \VarFunctional(x) = 0$;
	\item $(\Scalar\Functional)(x) = \Scalar \Functional(x) = 0$. \EndProof
\end{itemize}


	\chapter{Algebra di matrici}
	\section{Definizioni e teoremi di base.}
\label{Matrici_DefinizioniETeoremiDiBase}
\begin{Definition}
	Sia $\Ring$ un anello. Fissati interi $m, n \in \mathbb{N}$ una
	famiglia di elementi di $\Ring$ doppiamente indicizzata
	$\Matrix = (\MatrixElement_{i,j})_{(i,j) \in m \times n}$ si chiama \Define{matrice}. Inoltre
	\begin{itemize}
		\item gli elementi $(\MatrixElement_{i,j})_{(i,j) \in m \times n}$ si chiamano
		\Define{coefficienti}[di una matrice][coefficiente]: li denoteremo anche $(\Matrix_{i,j})_{(i,j)\in m \times n}$;
		\item la \Define{dimensione}[di una matrice][dimensione]
		di $\Matrix$ \`e $m \times n$;
		\item fissato $i \in m$, la famiglia
		$(\MatrixElement_{i,j})_{j \in n}$ si chiama $i$-esima
		\Define{riga}[di una matrice][riga] di $\Matrix$, denotata $\Matrix_i$;
		\item fissato $j \in m$, la famiglia
		$(\MatrixElement_{i,j})_{i \in m}$ si chiama $j$-esima
		\Define{colonna}[di una matrice][colonna] di $\Matrix$, denotata $\Matrix^j$.
	\end{itemize}
	Denotiamo $\Ring^{m \times n}$ la classe di tutte le matrici di dimensione $m \times n$. Denoteremo la matrice $\Matrix$ anche disponendone gli elementi in una tabella, le cui righe e colonne corrispondono alle righe e colonne della matrice come sopra definito, racchiusa da parentesi tonde o quadre:
$$\left (
\begin{array}{ccc}
	\MatrixElement_{0,0}	&	\cdots	&	\MatrixElement_{0,n} \\
	\vdots	&	\ddots	&	\vdots \\
	\MatrixElement_{m,0}	&	\cdots	&	\MatrixElement_{m,n}
\end{array}
\right )$$
o
$$\left [
\begin{array}{ccc}
	\MatrixElement_{0,0}	&	\cdots	&	\MatrixElement_{0,n} \\
	\vdots	&	\ddots	&	\vdots \\
	\MatrixElement_{m,0}	&	\cdots	&	\MatrixElement_{m,n}
\end{array}
\right ].$$
	Identificheremo spesso $\Ring^{m \times n}$ con $(\Ring^m)^n$.
\end{Definition}
\begin{Definition}
	Sia una matrice $\Matrix = (\MatrixElement_{i,j})_{(i,j) \in m \times n}$ ($(m,n) \in \mathbb{N}^2$) a coefficienti in un anello $\Ring$. Definiamo $\Matrix$
	\begin{itemize}
		\item \Define{matrice riga}[riga][matrice] se $m = 0$, e in tal caso denotiamo pi\`u brevemente $\Matrix$ come la famiglia singolarmente indicizzata $(\MatrixElement_j)_{j \in n}$, con $\ForAll{j \in n}{\MatrixElement_j = \MatrixElement_{0,j}}$ e identifichiamo $\Ring^{0 \times n}$ con $\Ring^n$;
		\item \Define{matrice colonna}[colonna][matrice] se $n = 0$, e in tal caso denotiamo pi\`u brevemente $\Matrix$ come la famiglia singolarmente indicizzata $(\MatrixElement_i)_{i \in m}$, con $\ForAll{i \in m}{\MatrixElement_i = \MatrixElement_{i,0}}$ e identifichiamo $\Ring^{m \times 0}$ con $\Ring^m$.
	\end{itemize}
\end{Definition}
\begin{Definition}
	Sia una matrice $\Matrix = (\MatrixElement_{i,j})_{(i,j) \in m \times n}$ ($(m,n) \in \mathbb{N}^2$) a coefficienti in un anello $\Ring$. Definiamo $\VarMatrix = (\MatrixElement_{i,j})_{(i,j) \in A \times B}$, con $A \subseteq m$ e $B \subseteq n$, \Define{sottomatrice} di $\Matrix$.
\end{Definition}
\begin{Definition}
	Siano un anello $\Ring$, interi $m, n, p \in \mathbb{N}$ e siano
	\begin{itemize}
		\item $A, B \in \Ring^{m \times n}$ con $A = (A_{i,j})_{(i,j) \in m \times n}$ e $B = (B_{i,j})_{(i,j) \in m \times n}$;
		\item sia $C \in \Ring^{n \times p}$ con $C = (C_{j,k})_{(j,k) \in n \times p}$;
		\item $\Scalar \in \Ring$.
	\end{itemize}
	Definiamo
	\begin{itemize}
		\item la \Define{somma}[di matrice][somma] come
		$A + B = (A_{i,j} + B_{i,j})_{(i,j) \in m \times n}$;
		\item il \Define{prodotto}[di matrice][prodotto] come
		$AC = (\sum_{j = 1}^n A_{i,j}C_{j,k})_{i,k}$;
		\item il
		\Define{prodotto per scalare}[di matrice per scalare][prodotto] come
		$\Scalar A = (\Scalar A_{i,j}))_{(i,j) \in m \times n}$.
	\end{itemize}
\end{Definition}
\begin{Theorem}
	Siano $(m,n) \in \mathbb{N}^2$ e $\Ring$ un anello. $\Ring^{m \times n}$ munito delle operazioni di somma e prodotto per scalare \`e un $\Ring$-modulo.
\end{Theorem}
\Proof Segue direttamente dalle definizioni. \EndProof
\begin{Corollary}
	Siano $\Ring$ un anello e $n \in \mathbb{N}$. $\Ring^n$ \`e un $\Ring$-modulo.
\end{Corollary}
\Proof $\Ring^n = \Ring^{n \times 0}$. \EndProof
\begin{Definition}
	Siano $\Ring$ un anello e $(n,m) \in \mathbb{N}^2$. Introduciamo $n$ incognite $(x_i)_{i \in n}$ e poniamo $X = (x_i)_{i \in n}$. Siano inoltre $A \in \Ring^{m \times n}$ e $B \in \Ring^n$. L'equazione $AX = B$ si chiama \Define{sistema lineare di equazione}[lineare di equazioni][sistema]. Inoltre, chiamiamo
	\begin{itemize}
		\item $A$ \Define{matrice del sistema}[di un sistema lineare][matrice] e i suoi coefficienti \Define{coefficienti del sistema}[di un sistema lineare][coefficienti];
		\item $B$ \Define{vettore dei termini noti}[dei termini noti][vettore] e le sue componenti \Define{termini noti}[noto][termine];
		\item $C \in \Ring^n$ tale che $AC = B$ \Define{soluzione del sistema}[di un sistema lineare][soluzione].
	\end{itemize}
	Se su $\Ring$ \`e definito un ordinamento $\leq$, allora introduciamo anche le notazioni $AX \leq B$ e $AX \geq B$, che chiamiamo \Define{sistema lineare di disequazioni}[lineare di disequazioni][sistema]. Definiamo per i sistemi di disequazioni gli stessi termini appena introdotti per i sistemi di equazioni, modificiano la definizione di soluzione come segue: $C \in \Ring^n$ \`e soluzione di $AX \leq B$ quando $\ForAll{i \in m}{(AC)_i \leq B_i}$ (e analogamente per il caso $AX \geq B$).
\end{Definition}
\begin{Theorem}
	Siano $(m,n) \in \mathbb{N}^2$ e $\Ring$ un anello. Sia $M \in \Ring^{m \times n}$. L'applicazione $\Linear: \Ring^n \rightarrow \Ring^m$ \`e un'applicazione lineare.
\end{Theorem}
\Proof Segue direttamente dalle definizioni. \EndProof

\section{Matrici quadrate.}
\label{Matrici_MatriciQuadrate}
\begin{Definition}
	Fissato un anello $\Ring$ e un numero naturale $n \in \mathbb{N}$, una matrice $\Matrix \in \Ring^{n \times n}$ si dice \Define{quadrata}[quadrata][matrice]. Definiamo inoltre
	\begin{itemize}
		\item $n$ \Define{ordine}[di una matrice quadrata][ordine];
		\item $(\Matrix_i^i)_{i \in n}$ \Define{diagonale principale}[principale][diagonale];
		\item $(\Matrix_{n - 1 - i}^i)_{i \in n}$ \Define{antidiagonale} o \Define{diagonale secondaria}[secondaria][diagonale].
	\end{itemize}
\end{Definition}

\section{Matrici triangolari.}
\label{Matrici_MatriciTriangolari}
\begin{Definition}
	Siano $\Ring$ un anello, $n, m \in \mathbb{N}$ e $\Matrix \in \Ring^{n \times m}$.
  Diciamo che $\Matrix$ \`e
	\begin{itemize}
		\item \Define{triangolare superiore}[triangolare superiore][matrice] quando
      $\ForAll{(i,j) \in n \times m}{\Implies{i > j}{\Matrix_i^j = 0}}$:
      nel caso quadrato abbiamo
\[
\lmatrix
\begin{array}{ccc}
\Matrix_0^0 & \cdots & \Matrix_{n - 1}^{n - 1}\\
& \ddots & \vdots\\
& & \Matrix_{n - 1}^{n - 1}
\end{array}
\rmatrix;
\]
		\item \Define{triangolare strettamente superiore}[triangolare strettamente superiore][matrice] quando
      $\ForAll{(i,j) \in n \times m}{\Implies{i \geq j}{\Matrix_i^j = 0}}$:
      nel caso quadrato abbiamo
\[
\lmatrix
\begin{array}{cccc}
0 & \Matrix_0^1 & \cdots & \Matrix_0^{n - 1}\\
& 0 & \ddots & \vdots\\
& & 0 & \Matrix_{n - 2}^{n - 1}\\
& & & 0
\end{array}
\rmatrix;
\]
		\item \Define{triangolare inferiore}[triangolare inferiore][matrice] quando
      $\ForAll{(i,j) \in n \times m}{\Implies{i < j}{\Matrix_i^j = 0}}$:
      nel caso quadrato abbiamo
\[
\lmatrix
\begin{array}{ccc}
\Matrix_0^0 & &\\
\vdots & \ddots &\\
\Matrix_{n - 1}^0 & \cdots & \Matrix_{n - 1}^{n - 1}\\
\end{array}
\rmatrix;
\]
		\item \Define{triangolare strettamente inferiore}[triangolare strettamente inferiore][matrice] quando
    $\ForAll{(i,j) \in n \times m}{\Implies{i \leq j}{\Matrix_i^j = 0}}$:
      nel caso quadrato abbiamo
\[
\lmatrix
\begin{array}{cccc}
0 & & &\\
\Matrix_1^0 & \ddots & &\\
\vdots & \ddots & \ddots &\\
\Matrix_{n - 1}^0 & \cdots & \Matrix_{n - 1}^{n - 2} & 0\\
\end{array}
\rmatrix;
\]
	\end{itemize}
\end{Definition}
\begin{Theorem}
  Siano
  \begin{itemize}
    \item $m, n, p \in \NotZero{\mathbb{N}}$;
    \item $\Ring$ un anello;
    \item $\TriangularMatrix \in \Ring^{m \times n}$;
    \item $\VarTriangularMatrix \in \Ring^{n \times p}$;
    \item $\Matrix = \TriangularMatrix\VarTriangularMatrix$.
  \end{itemize}
  Allora
  \begin{itemize}
    \item se $\TriangularMatrix$ e $\VarTriangularMatrix$ sono triangolari
      inferiori, $\Matrix$ \`e triangolare inferiore;
    \item se $\TriangularMatrix$ e $\VarTriangularMatrix$ sono triangolari
      strettamente inferiori, $\Matrix$ \`e triangolare strettamente inferiore.
  \end{itemize}
\end{Theorem}
\Proof Sia $(k,j) \in m \times p$. Abbiamo
$\Matrix_k^j
= \sum_{l \in n} \TriangularMatrix_k^l \VarTriangularMatrix_l^j$.
\par Ora,
\begin{itemize}
  \item se $\TriangularMatrix$ e $\VarTriangularMatrix$ sono triangolari
    inferiori,
    $\ForAll{(k,l) \in m \times n}{
      \Implies{k < l}{\TriangularMatrix_k^l = 0}}$;
  \item se $\TriangularMatrix$ e $\VarTriangularMatrix$ sono triangolari
    strettamente inferiori,
    $\ForAll{(k,l) \in m \times n}{
      \Implies{k \leq l}{\TriangularMatrix_k^l = 0}}$.
\end{itemize}
\par Ne segue immediatamente la tesi. \EndProof
\begin{Definition}
	Siano $\Ring$ un anello, $n, m \in \mathbb{N}$ e $\Matrix \in \Ring^{n \times m}$.
  Sia $(\tau_k)_{k \in n}$ una successione tale che, per ogni $k \in n$,
  \begin{itemize}
    \item $\Implies{j < \tau_k}{\Matrix_k^j = 0}$;
    \item $\Matrix_k^{\tau_k} \neq 0$.
  \end{itemize}
  Diciamo che $\Matrix$ \`e una
  \Define{matrice a gradini}[a gradini][matrice]
  quando la successione $(\tau_k)_{k \in n}$ \`e strettamente crescente.
\end{Definition}
\begin{Theorem}
  Con le notazioni della definizione precedente, $\Matrix$ \`e una matrice
  triangolare superiore.
\end{Theorem}
\Proof Procediamo per induzione su $n$. Il teorema \`e immediato se $n = 1$.
\par Supponiamo il teorema dimostrato per $n = N \in \mathbb{N}$ e proviamolo
per $n = N + 1$. La sottomatrice $\Matrix'$ ottenuta da $\Matrix$
eliminando la prima riga e la prima colonna \`e ancora una matrice a gradini
e dunque vi si pu\`o applicare l'ipotesi induttiva: $\Matrix'$ \`e
triangolare superiore.
\par Poich\'e $\Matrix$ \`e a gradini, tutti gli elementi
$(\Matrix_k^0)_{k = 1}^{n - 1}$ devono essere nulli. Ne consegue che
$\Matrix$ \`e triangolare superiore. \EndProof

\section{Matrici $N$-diagonali.}
\label{Matrici_MatriciNDiagonali}
\begin{Definition}
	Siano $\Ring$ un anello, $n \in \mathbb{N}$ e $\Matrix \in \Ring^{n \times n}$. Fissato $N \in \mathbb{N}$, chiamiamo $\Matrix$ \Define{$2N + 1$-diagonale}[$N$-diagonale][matrice] quando $\ForAll{(i,j) \in n \times n}{\Implies{\AbsoluteValue{i - j} > N}{\Matrix_i^j = 0}}$. In particolare, chiamiamo $\Matrix$
\begin{itemize}
	\item \Define{diagonale}[diagonale][matrice] se \`e $1$-diagonale:
\[
\lmatrix
\begin{array}{ccc}
\Matrix_0^0 & &\\
& \ddots &\\
& & \Matrix_{n - 1}^{n - 1}
\end{array}
\rmatrix;
\]
	\item \Define{tridiagonale}[tridiagonale][matrice] se \`e $3$-diagonale:
\[
\lmatrix
\begin{array}{ccccc}
\Matrix_0^0 & \Matrix_0^1 & & &\\
\Matrix_1^0 & \ddots & \ddots & &\\
& \ddots  & \ddots & \ddots &\\
& & \ddots & \ddots & \Matrix_{n - 2}^{n - 1}\\
& & & \Matrix_{n - 1}^{n - 2} & \Matrix_{n - 1}^{n - 1}
\end{array}
\rmatrix;
\]
\end{itemize}
e cos\`i via.
\end{Definition}
\begin{Theorem}
	Siano $\Ring$ un anello, $n \in \mathbb{N}$, $\Matrix, \DiagonalMatrix \in \Ring^{n \times n}$, con $\DiagonalMatrix$ diagonale. Abbiamo, posto $A = \DiagonalMatrix\Matrix$ e $B = \Matrix\DiagonalMatrix$,
	\begin{itemize}
		\item $A_i^j = \DiagonalMatrix_i^i\Matrix_i^j$;
		\item $B_i^j = \Matrix_i^j\DiagonalMatrix_j^j$.
	\end{itemize}
\end{Theorem}
\Proof Segue per calcolo diretto. \EndProof

\section{Matrici di toeplitz.}
\label{Matrici_MatriciDiToeplitz}
\begin{Definition}
	Siano $\Ring$ un anello, $n \in \mathbb{N}$ e $\Matrix \in \Ring^{n \times n}$. Si dice che $\Matrix$ \`e una \Define{matrice di Toeplitz}[di Toeplitz][matrice] o \Define{matrice a diagonali costanti}[a diagonali costanti][matrice] quando
	\[
		\ForAll{(i,j,k,l) \in n \times n \times n \times n}{\Implies{i - k = j - l}{\Matrix_i^j = \Matrix_k^l}}.
	\]
\end{Definition}
\begin{Definition}
	Siano $\Ring$ un anello, $n \in \mathbb{N}$ e $\Matrix \in \Ring^{n \times n}$. Si dice che $\Matrix$ \`e una \Define{matrice di Hankel}[di Hankel][matrice] quando
	\[
		\ForAll{(i,j,k,l) \in n \times n \times n \times n}{\Implies{i - k = l - j}{\Matrix_i^j = \Matrix_k^l}}.
	\]
\end{Definition}
\begin{Theorem}
	Se $\Matrix$ \`e una matrice di Toeplitz, allora anche $\Transposed{\Matrix}$ \`e una matrice di Toeplitz.
\end{Theorem}
\Proof Segue direttamente dalle definizioni. \EndProof
\begin{Theorem}
	Le matrici di Hankel sono simmetriche.
\end{Theorem}
\Proof Segue direttamente dalle definizioni. \EndProof
\begin{Theorem}
	Siano $\Ring$ un anello, $n \in \mathbb{N}$ e $\Matrix \in \Ring^{n \times n}$. Sia inoltre $\PermutationMatrix \in \Ring^{n \times n}$ la matrice di permutazione definita da $\PermutationMatrix_i^j = \begin{cases}1\text{ se }i = n - j,\\0\text{ altrimenti};\end{cases}$, vale a dire
	\[
		\lmatrix
		\begin{array}{ccc}
		& & 1\\
		& \iddots &\\
		1 & &
		\end{array}
		\rmatrix.
	\]
	Abbiamo:
	\begin{itemize}
		\item se $\Matrix$ \`e una matrice di Toeplitz, allora $\PermutationMatrix \Matrix$ e $\Matrix \PermutationMatrix$ sono matrici di Hankel;
		\item se $\Matrix$ \`e una matrice di Hankel, allora $\PermutationMatrix \Matrix$ e $\Matrix \PermutationMatrix$ sono matrici di Toeplitz.
	\end{itemize}
\end{Theorem}
\Proof Fissato $(i,j,k,l) \in n \times n \times n \times n$, abbiamo
\begin{itemize}
	\item $(\PermutationMatrix \Matrix)_{i,j} = \Matrix_{n - 1 - i,j}$;
	\item $(\PermutationMatrix \Matrix)_{k,l} = \Matrix_{n - 1 - k,l}$.
\end{itemize}
\par Inoltre
\begin{itemize}
	\item se $i - k = j - l$, allora $n - 1 - i - (n - 1 - k) = k - i = j - l$, da cui se $\Matrix$ \`e matrice di Toepliz allora $\PermutationMatrix \Matrix$ \`e matrice di Hankel;
	\item se $i - k = l - j$, allora $n - 1 - i - (n - 1 - k) = k - i = l - j$, da cui se $\Matrix$ \`e matrice di Hankel allora $\PermutationMatrix \Matrix$ \`e matrice di Toeplitz.
\end{itemize}
\par Il resto del teorema segue per trasposizione da quanto gi\`a dimostrato. \EndProof

\section{Matrici trasposte hermitiane e trasposte.}
\label{Matrici_MatriciTrasposteHermitianeETrasposte}
\begin{Definition}
	Siano $\Ring$ un anello e $(n,m) \in \mathbb{N}^2$. Sia $\Matrix = (\MatrixElement_{i,j})_{(i,j) \in m \times n} \in \Ring^{m \times n}$. Chiamiamo
	\begin{itemize}
		\item \Define{matrice trasposta}[trasposta][matrice] di $\Matrix$, denotata $\Transposed{\Matrix}$, la matrice $\Transposed{\Matrix} = (\MatrixElement_{j,i})_{(j,i) \in n \times m} \in \Ring^{n \times m}$;
		\item se $\Ring = \mathbb{C}$, \Define{matrice trasposta coniugata}[trasposta coniugata][matrice] o \Define{trasposta hermitiana}[trasposta hermitiana][matrice] o ancora \Define{matrice aggiunta}[aggiunta][matrice] di $\Matrix$, denotata $\HermitianTransposed{\Matrix}$, la matrice $\HermitianTransposed{\Matrix} = (\Conjugate{\MatrixElement_{j,i}})_{(j,i) \in n \times m} \in \Ring^{n \times m}$.
	\end{itemize}
\end{Definition}
\begin{Definition}
	Siano $\Ring$ un anello, $n \in \mathbb{N}$ e $\Matrix = \in \Ring^{n \times n}$. Diciamo che $\Matrix$ \`e
	\begin{itemize}
		\item \Define{simmetrica}[simmetrica][matrice] quando $\Matrix = \Transposed{\Matrix}$;
		\item se $\Ring = \mathbb{C}$, \Define{hermitiana}[hermitiana][matrice] o \Define{autoaggiunta}[autoaggiunta][matrice] quando $\Matrix = \HermitianTransposed{\Matrix}$;
		\item se $\Ring = \mathbb{C}$, \Define{antihermitiana}[antihermitiana][matrice] quando $\Matrix = - \HermitianTransposed{\Matrix}$.
		\item
	\end{itemize}
\end{Definition}
\begin{Definition}
	Siano $\Ring$ un anello, $n \in \mathbb{N}$ e $\Matrix = \in \Ring^{n \times n}$. Diciamo che $\Matrix$ \`e
	\begin{itemize}
		\item \Define{ortogonale}[ortogonale][matrice] quando $\Matrix$ \`e invertibile e $\Matrix^{-1} = \Transposed{\Matrix}$;
		\item \Define{unitaria}[unitaria][matrice] quando $\Matrix$ \`e invertibile e $\Matrix^{-1} = \HermitianTransposed{\Matrix}$;
	\end{itemize}
\end{Definition}
\begin{Definition}
	Siano $n \in \mathbb{N}$ e $\Matrix = \in \mathbb{C}^{n \times n}$. Diciamo che $\Matrix$ \`e \Define{normale}[normale][matrice] quando $\Matrix\HermitianTransposed{\Matrix} = \HermitianTransposed{\Matrix}\Matrix$.
\end{Definition}
\begin{Theorem}
	Tutte le matrici unitarie sono normali.
\end{Theorem}
\Proof Segue direttamente dalle definizioni. \EndProof

\section{Matrici di permutazione.}
\label{Matrici_MatriciDiPermutazione}
\begin{Definition}
	Dato una anello con identit\`a $\RingWithIdentity$, chiamiamo \Define{matrice di permutazione}[di permutazione][matrice] una matrice quadrata $\PermutationMatrix = (\PermutationMatrixElement_{i,j})_{(i,j) \in n \times n} \in \RingWithIdentity^{n \times n}$ ($n \in \NotZero{\mathbb{N}}$) tale che
\begin{itemize}
	\item $\ForAll{(i,j) \in n \times n}{\PermutationMatrixElement_{i,j} \in \lbrace 0, 1 \rbrace}$;
	\item $\ForAll{i \in n}{\ForAll{(j,j') \in n \times n}{\Implies{\And{\PermutationMatrixElement_{i,j} = 1}{\PermutationMatrixElement_{i,j'} = 1}}{j = j'}}}$;
	\item $\ForAll{j \in n}{\ForAll{(i,i') \in n \times n}{\Implies{\And{\PermutationMatrixElement_{i,j} = 1}{\PermutationMatrixElement_{i',j} = 1}}{i = i'}}}$.
\end{itemize}
\end{Definition}
\begin{Theorem}
	Se $\PermutationMatrix$ \`e una matrice di permutazione, lo \`e anche $\Transposed{\PermutationMatrix}$.
\end{Theorem}
\Proof Segue direttamente dalle definizioni. \EndProof
\begin{Theorem}
	Sia $\PermutationMatrix$ una matrice quadrata di ordine $n$. $\PermutationMatrix$ \`e di permutazione se e solo se esistono uniche permutazioni $\Permutation, \VarPermutation \in \SymmetricGroup{n}$ tali che
	\begin{itemize}
		\item $\ForAll{i \in n}{\PermutationMatrix_{\Permutation(i)} = \Identity_i}$;
		\item $\ForAll{j \in n}{\PermutationMatrix^{\VarPermutation(j)} = \Identity^j}$.
	\end{itemize}
\end{Theorem}
\Proof $\Identity$ \`e una matrice di permutazione e modificandone l'ordine delle righe mantiene la propriet\`a di avere un solo $1$ per ogni riga e per ogni colonna. Dunque, se $\ForAll{i \in n}{\PermutationMatrix_i = \Identity_{\Permutation{i}}}$, allora $\PermutationMatrix$ \`e di permutazione.
\par Supponiamo $\PermutationMatrix$ di permutazione e definiamo la permutazione $\Permutation \in \SymmetricGroup{n}$ che a $i \in n$ associa l'indice dell'unica riga di $\PermutationMatrix$ che contiene un $1$ in posizione $i$. Abbiamo allora $\PermutationMatrix_{\Permutation(i)} = \Identity_i$.
\par L'unicit\`a segue dal fatto che tutte le righe di $\PermutationMatrix$
da una parte e di $\Identity$ dall'altra sono distinte.
\par Il resto del teorema segue direttamente per trasposizione. \EndProof
\begin{Definition}
  Con le notazioni del teorema precedente, chiamiamo
  \begin{itemize}
    \item $\Permutation^{-1}$
      \Define{permutazione associata alle righe}[associata alle righe di una matrice di permutazione][permutazione];
    \item $\VarPermutation^{-1}$
      \Define{permutazione associata alle colonne}[associata alle colonne di una matrice di permutazione][permutazione].
  \end{itemize}
\end{Definition}
\begin{Theorem}
	Siano
	\begin{itemize}
		\item $\RingWithIdentity$ un anello con identit\`a;
		\item $m, n \in \mathbb{N}$;
		\item $\Matrix \in \RingWithIdentity^{m \times n}$;
		\item $\PermutationMatrix \in \RingWithIdentity^{m \times m}$ matrice di
      permutazione;
		\item $\VarPermutationMatrix \in \RingWithIdentity^{n \times n}$ matrice di
      permutazione;
    \item $\Permutation \in \SymmetricGroup{m}$ permutazione associata alle
      righe di $\PermutationMatrix$;
    \item $\VarPermutation \in \SymmetricGroup{n}$ permutazione associata alle
      colonne di $\VarPermutationMatrix$.
	\end{itemize}
	Abbiamo
	\begin{itemize}
		\item $\ForAll{i \in m}{(\PermutationMatrix\Matrix)_i = \Matrix_{\Permutation(i)}}$;
		\item $\ForAll{j \in n}{(\Matrix\VarPermutationMatrix)^{j} = \Matrix^{\VarPermutation(j)}}$.
	\end{itemize}
\end{Theorem}
\Proof Abbiamo
\begin{itemize}
	\item per ogni $j \in n$, $(\PermutationMatrix\Matrix)_{\Permutation^{-1}(i)}^j = \sum_{k \in m} \PermutationMatrix_{\Permutation^{-1}(i)}^k \Matrix_k^j = \sum_{k \in m} \Identity_i^k \Matrix_k^j = \Matrix_i^j$ e quindi $\ForAll{i \in m}{(\PermutationMatrix\Matrix)_{\Permutation^{-1}(i)} = \Matrix_i}$;
	\item per ogni $i \in m$, $(\Matrix\VarPermutationMatrix)_i^{\VarPermutation^{-1}(j)} = \sum_{k \in n} \Matrix_i^k\VarPermutationMatrix_k^{\VarPermutation^{-1}(j)} = \sum_{k \in n} \Matrix_i^k \Identity_k^j = \Matrix_i^j$ e quindi $\ForAll{j \in n}{(\Matrix\VarPermutationMatrix)^{\VarPermutation^{-1}(j)} = \Matrix^j}$. \EndProof
\end{itemize}
\begin{Theorem}
	Siano $\RingWithIdentity$ un anello con identit\`a, $n \in \mathbb{N}$ e $\PermutationMatrix \in \RingWithIdentity^{n \times n}$ una matrice di permutazione. $\PermutationMatrix$ \`e una matrice ortogonale (unitaria nel caso $\RingWithIdentity = \mathbb{C}$).
\end{Theorem}
\Proof Segue per calcolo diretto. \EndProof

\section{Prodotto e somma di Kronecker}
\label{Matrici_ProdottoESommaDiKronecker}
\begin{Definition}
  Fissato un anello $\Ring$, siano
  \begin{itemize}
    \item siano $m, n, p, q \in \NotZero{\mathbb{N}}$;
    \item $A \in \Ring^{m \times n}$;
    \item $B \in \Ring^{p \times q}$.
  \end{itemize}
  Chiamiamo
  \Define{prodotto di Kronecker}[di Kronecker][prodotto]
  la matrice $A \KroneckerProduct B \in \Ring^{mp \times nq}$ definita da
  $(A \KroneckerProduct B)_k^j = A_{q_1}^{q_2} B_{r_1}^{r_2}$
  al variare di $(k,j) \in mp \times nq$, dove
  \begin{itemize}
    \item $q_1$ \`e il quoziente della divisione euclidea di $k$ per $p$;
    \item $r_1$ \`e il resto della divisione euclidea di $k$ per $p$;
    \item $q_2$ \`e il quoziente della divisione euclidea di $j$ per $q$;
    \item $r_2$ \`e il resto della divisione euclidea di $j$ per $q$;
  \end{itemize}
  vale a dire
  \[
    A \KroneckerProduct B =
    \lmatrix
    \begin{array}{ccc}
      A_0^0 B & \cdots & A_0^{n - 1} B,\\
      \vdots & \ddots \vdots,\\
      A_{m - 1}^0 B & \cdots & A_{m - 1}^{n - 1} B,\\
    \end{array}
    \rmatrix.
  \]
  Se $\Ring$ \`e un anello unitario, chiamiamo inoltre
  \Define{somma di Kronecker}[di Kronecker][somma]
  la matrice
  $A \KroneckerSum B
  = \Identity \KroneckerProduct A + B \KroneckerProduct \Identity$.
\end{Definition}
\begin{Theorem}
  Fissato un anello $\Ring$, siano
  \begin{itemize}
    \item siano $m, n, p, q, s, t \in \NotZero{\mathbb{N}}$;
    \item $A \in \Ring^{m \times n}$;
    \item $B \in \Ring^{p \times q}$;
    \item $C \in \Ring^{n \times s}$;
    \item $D \in \Ring^{q \times t}$.
  \end{itemize}
  Abbiamo
  \[
    (A \KroneckerProduct B)(C \KroneckerProduct D)
    = AC \KroneckerProduct BD.
  \]
\end{Theorem}
\Proof Abbiamo, per $(k,j) \in mq \times pt$,
posto
\begin{itemize}
    \item $q_1$ \`e il quoziente della divisione euclidea di $k$ per $p$;
    \item $r_1$ \`e il resto della divisione euclidea di $k$ per $p$;
    \item $q_2$ \`e il quoziente della divisione euclidea di $j$ per $q$;
    \item $r_2$ \`e il resto della divisione euclidea di $j$ per $q$;
    \item $q_3$ \`e il quoziente della divisione euclidea di $l$ per $q$;
    \item $r_3$ \`e il resto della divisione euclidea di $l$ per $q$;
\end{itemize}
\begin{align*}
  ((A \KroneckerProduct B)(C \KroneckerProduct D))_k^j
  &= (A \KroneckerProduct B)_k^l (C \KroneckerProduct D)_l^j,\\
  &= A_{q_1}^{q_3} B_{r_1}^{r_3} C_{q_3}^{q_4} D_{r_3}^{r_4},\\
  &= A_{q_1}^{q_3} C_{q_3}^{q_4} B_{r_1}^{r_3} D_{r_3}^{r_4},\\
  &= AC \KroneckerProduct BD.\text{ \EndProof}
\end{align*}
\begin{Corollary}
  Fissato un anello $\Ring$, siano
  \begin{itemize}
    \item siano $n, p \in \NotZero{\mathbb{N}}$;
    \item $A \in \Ring^{n \times n}$ invertibile;
    \item $B \in \Ring^{p \times p}$ invertibile.
  \end{itemize}
  Abbiamo $(A \KroneckerProduct B)^{-1} = A^{-1} \KroneckerProduct B^{-1}$.
\end{Corollary}
\Proof Abbiamo
\begin{align*}
  (A \KroneckerProduct B)(A^{-1} \KroneckerProduct B^{-1})
  &= AA^{-1} \KroneckerProduct BB^{-1},\\
  &= \Identity_n \KroneckerProduct \Identity_p,\\
  &= \Identity_{np},\\
  &= A^{-1}A \KroneckerProduct B^{-1}B,\\
  &= (A^{-1} \KroneckerProduct B^{-1})(A \KroneckerProduct B).\text{ \EndProof}
\end{align*}


	\chapter{Teoria dei campi}
	\section{Definizioni e teoremi di base.}\label{DefinizioniETeoremiDiBase}
\begin{Definition}
	Si chiama \Define{corpo} un anello con identit\`a non banale in cui tutti gli elementi non nulli sono invertibili.
\end{Definition}
\par Nel seguito denoteremo con $+$ l'operazione rispetto alla quale il sostegno del corpo costituisce un gruppo commutativo, con $\cdot$ l'altra operazione e impiegheremo le notazioni coerenti per gli elementi neutri e gli inversi.
\begin{Definition}
	Un \Define{campo} \`e un corpo commutativo.
\end{Definition}
\begin{Theorem}
	Se $\Prime \in \mathbb{Z}$ \`e primo, allora $\Quotient{\mathbb{Z}}{(\Prime)}$ \`e un campo.
\end{Theorem}
\Proof Segue dal fatto che $\mathbb{Z}$ \`e un dominio euclideo. \EndProof
\begin{Theorem}
	Sia $\Field$ un campo. Allora
	\begin{itemize}
		\item se $\Field = \mathbb{Q}$, allora $\Characteristic{\Field} = 0$;
		\item se $\Field = \Quotient{\mathbb{Z}}{(\Prime)}$, dove $\Prime$ \`e un numero primo, allora $\Characteristic{\Field} = \Prime$;
		\item $\Characteristic{\Field}$ \`e un numero nullo o primo;
		\item se $\Characteristic{\Field} = \Prime$, dove $\Prime$ \`e un numero primo, allora $\Field$ contiene un sottocampo isomorfo a $\Quotient{\mathbb{Z}}{(\Prime)}$.
		\item se $\Characteristic{\Field} = 0$, allora $\Field$ contiene un sottocampo\footnote{Un sottocampo di $\Field$ \`e un sottoinsieme di $\Field$ che \`e un campo con le operazioni date dalla restrizione delle operazioni definite su $\Field \times \Field$, \Cfr definizione \ref{DefEstensione}.} $\Field'$ isomorfo a $\mathbb{Q}$;
	\end{itemize}
\end{Theorem}
\Proof Le prime due affermazioni sono ovvie.
\par Siano $n, m \in \mathbb{Z}$ e supponiamo $\phi(nm) = 0$, dove $\phi: \mathbb{Z} \rightarrow \Field$ che a $n \in \mathbb{Z}$ associa $n \cdot 1$. Supponiamo $\phi(m) \neq 0$. Allora $\Characteristic{\Field}$ divide necessariamente $n$, quindi $\Kernel{\phi}$ \`e un ideale primo e $\Characteristic{\Field}$ \`e necessariamente primo o nullo: questo prova la terza affermazione.
\par La quarta affermazione segue direttamente dal fatto che l'omomorfismo dedotto da $\phi$ per passaggio al quoziente \`e iniettivo.
\par Per l'ultima affermazione, sempre l'omomorfismo dedotto da $\phi$ per passaggio al quotidiano dimostra che $\phi(\mathbb{Z})$ \`e un anello isomorfo a $\mathbb{Z}$, pertanto il campo dei quozienti di $\phi(\mathbb{Z})$ \`e isomorfo al campo dei quozienti di $\mathbb{Z}$, cio\`e $\mathbb{Q}$. \EndProof
\begin{Theorem}
	Sia $\phi: \Field_1 \rightarrow \Field_2$ un omomorfismo. Se $\phi \neq 0$, allora $\phi$ \`e iniettivo.
\end{Theorem}
\Proof Il nucleo di un omomorfismo di anelli con unit\`a, e dunque di campi, \`e un ideale e gli unici ideali di un campo sono $\SpanIdeal{0}$ e $\SpanIdeal{1}$. \EndProof

\section{Estensioni.}\label{Estensioni}
\subsection{Concetti generali.}\label{EstensioniConcettiGenerali}
\begin{Definition}\label{DefEstensione}
	Siano $\Field_1$ e $\Field_2$ campi tali che
	\begin{itemize}
		\item $\Field_1 \subseteq \Field_2$;
		\item $\Field_2$ induca sul sottoinsieme $\Field_1$ una struttura di campo coincidente con quella di $\Field_1$;
	\end{itemize}
	allora
	\begin{itemize}
		\item $\Field_1$ \`e un \Define{sottocampo} di $\Field_2$;
		\item $\Field_2$ \`e un'\Define{estensione} di $\Field_1$;
		\item la notazione $\FieldExtension{\Field_2}{\Field_1}$ indica l'\NIDefine{estensione del campo} $\Field_1$ ad opera di $\Field_2$.
	\end{itemize}
\end{Definition}
\begin{Definition}\label{DefSottoestensione}
	Siano le estensioni di campi $\FieldExtension{\Field_2}{\Field_1}$ e $\FieldExtension{\Field_3}{\Field_1}$. Se $\Field_3$ \`e un'estensione di $\Field_2$, allora $\FieldExtension{\Field_2}{\Field_1}$ \`e una \Define{sottoestensione} di $\FieldExtension{\Field_3}{\Field_1}$.
\end{Definition}
\begin{Definition}
	Sia una successione di campi $(\Field_n)_{n \in \mathbb{N}}$ tale che, per $n > 0$, il campo $\Field_n$ sia un'estensione di $\Field_{n - 1}$: tale successione si chiama \Define{torre d'estensioni}.
\end{Definition}
\begin{Definition}
	Sia l'estensione di campi $\FieldExtension{\Field_2}{\Field_1}$ e sia $S \subseteq \Field_2$. La notazione $\FieldAdjunction{\Field_1}{S}$ indica il pi\`u piccolo sottocampo di $\Field_2$ contenente $\Field_1 \cup S$. Nel caso in cui $S$ sia un insieme finito $\lbrace x_1, ..., x_n \rbrace$ si scrive semplicemente $\FieldAdjunction{\Field_1}{x_1, ..., x_n}$ e, se $S = \lbrace x \rbrace$ \`e un singoletto, si dice che l'estensione $\FieldExtension{\FieldAdjunction{\Field_1}{x}}{\Field_1}$ \`e \Define{semplice}{estensione}.
\end{Definition}
\begin{Theorem}\label{ThEstensioneUnione}
	Con le notazioni della definizione precedente, indicata con $\mathcal{S}$ la famiglia di tutte le parti finite di $S$, abbiamo $\FieldAdjunction{\Field_1}{S} = \bigcup_{T \in \mathcal{S}} \FieldAdjunction{\Field_1}{T}$.
\end{Theorem}
\Proof Chiaramente $\bigcup_{T \in \mathcal{S}} \FieldAdjunction{\Field_1}{S} \subseteq \FieldAdjunction{\Field_1}{S}$. D'altra parte, $S \subseteq \bigcup_{T \in \mathcal{S}} \FieldAdjunction{\Field_1}{S}$ e si verifica facilmente che $\bigcup_{T \in \mathcal{S}} \FieldAdjunction{\Field_1}{S}$ \`e un campo, dunque $\FieldAdjunction{\Field_1}{S} \subseteq \bigcup_{T \in \mathcal{S}} \FieldAdjunction{\Field_1}{S}$ e questo completa la dimostrazione. \EndProof
\begin{Definition}
	Sia l'estensione di campi $\FieldExtension{\Field_2}{\Field_1}$ e sia $\alpha \in \Field_2$. $\alpha$ si dice
	\begin{itemize}
		\item \Define{algebrico}[algebrico][elemento] (su $\Field_1$) se esiste un polinomio in $p \in \Field_1[x]$ tale che $p(\alpha) = 0$;
		\item \Define{trascendente}[algebrico][elemento] (su $\Field_1$) altrimenti.
	\end{itemize}
	Inoltre, se ogni $\alpha \in \Field_2$ \`e algebrico su $\Field_1$ si dice che $\FieldExtension{\Field_2}{\Field_1}$ \`e un'\Define{estensione algebrica}[algebrica][estensione].
\end{Definition}
\begin{Theorem}
	Sia l'estensione di campi $\FieldExtension{\Field_2}{\Field_1}$. $\Field_2$ \`e uno spazio vettoriale su $\Field_1$.
\end{Theorem}
\Proof La verifica degli assiomi di spazio vettoriale \`e diretta. \EndProof
\begin{Definition}
	Sia l'estensione di campi $\FieldExtension{\Field_2}{\Field_1}$. La dimensione del $\Field_1$-spazio vettoriale $\Field_2$ si indica con $\FieldDegree{\Field_2}{\Field_1}$ e si chiama \Define{grado dell'estensione}; se il grado \`e finito, l'estensione si dice \Define{finita}[finita][estensione]. Inoltre, se $\alpha \in \Field_2$ \`e algebrico su $\Field_1$ e $\Field_2 = \FieldAdjunction{\Field_1}{\alpha}$, allora si dice che $\alpha$ \`e un elemento \NIDefine{di grado $\FieldExtension{\Field_2}{\Field_1}$} su $\Field_1$.
\end{Definition}
\begin{Theorem}
	Se l'estensione $\FieldExtension{\Field_2}{\Field_1}$ \`e finita allora \`e anche algebrica.
\end{Theorem}
\Proof Sia $\alpha \in \Field_2$ e sia $n = \FieldDegree{\Field_2}{\Field_1}$. La famiglia di vettori $(\alpha^k)_{k = 0}^n$ \`e dipendente ed esiste dunque un polinomio di grado minore o uguale a $n$ che si annulla in $\alpha$. \EndProof
\begin{Theorem}
	Sia l'estensione di campi $\FieldExtension{\Field_2}{\Field_1}$ e sia $\alpha \in \Field_2$. Abbiamo
	\begin{itemize}
		\item se $\alpha$ \`e algebrico l'insieme dei polinomi di $\Field_1[x]$ che si annullano in $\alpha$ costituisce un ideale non nullo (principale\footnote{$\Field_1[x]$ \`e un dominio a ideali principali, dunque l'ideale \`e necessariamente principale.});
		\item se $\alpha$ \`e trascendente $\Isomorphic{\Field_1(x)}{\Field_1[x]}$.
	\end{itemize}
\end{Theorem}
\Proof Si considera l'omomorfismo $\phi: \Field_1[x] \rightarrow \Field_1(x)$ che al polinomio $p$ associa $p(\alpha)$. Ora,
\begin{itemize}
	\item se $\alpha$ \`e algebrico il nucleo di $\phi$ contiene un elemento non nullo e $\Kernel{\phi}$ \`e un ideale non nullo;
	\item se invece $\alpha$ \`e trascendente $\Kernel{\phi}$ \`e l'ideale non nullo: $\phi$ \`e dunque iniettivo; essendo anche surgettivo, \`e un isomorfismo. \EndProof
\end{itemize}
\begin{Definition}
	Sia l'estensione di campi $\FieldExtension{\Field_2}{\Field_1}$. Dato $\alpha \in \Field_2$ algebrico, il generatore monico dell'ideale del teorema precedente \`e detto
  \Define{polinomio minimo}[minimo di un elemento algebrico][polinomio]
  di $\alpha$ su $\Field_1$ e lo denoteremo $\MinimalPolynomial{\alpha}[\Field_1]$, omettendo $\Field_1$ quando il campo \`e ovvio dal contesto. Diciamo inoltre che tutte le radici di $\MinimalPolynomial{\alpha}$ sono \Define{radici coniugate} ad $\alpha$.
\end{Definition}
\begin{Theorem}
	Data l'estensione semplice $\FieldExtension{\FieldAdjunction{\Field_1}{\alpha}}{\Field_1}$, abbiamo $\FieldDegree{\FieldAdjunction{\Field_1}{\alpha}}{\Field_1} = \begin{cases} \Deg \MinimalPolynomial{\alpha}\text{, se } \alpha \text{\`e algebrico},\\ \infty\text{ altrimenti}. \end{cases}$ Una base \`e data dalla famiglia $(\alpha^k)_{k = 0}^{\FieldDegree{\FieldAdjunction{\Field_1}{\alpha}}{\Field_1}}$.
\end{Theorem}
\Proof Sia $n = \Deg \MinimalPolynomial{\alpha}$: il polinomio $\MinimalPolynomial{\alpha}$ prova che gli $n$ elementi della famiglia $(\alpha^k)_{k = 0}^n$ sono dipendenti, dunque $\FieldDegree{\Field_1(\alpha)}{\Field_1} \leq n$. D'altra parte, nessun polinomio di grado minore di $n$ si annulla in $\alpha$, dunque la famiglia $(\alpha^k)_{k = 0}^{n - 1}$ \`e composta da $n$ vettori indipendenti e ne segue la tesi. \EndProof
\begin{Theorem}\label{thEstensioniTorri}
	Siano le estensioni di campi $\FieldExtension{\Field_3}{\Field_2}$ e $\FieldExtension{\Field_2}{\Field_1}$, allora
	\begin{itemize}
		\item $\Field_3$ \`e un'estensione di $\Field_1$;
		\item se $\FieldExtension{\Field_3}{\Field_2}$ e $\FieldExtension{\Field_2}{\Field_1}$ sono finite, allora \`e finita anche $\FieldExtension{\Field_3}{\Field_1}$ e abbiamo $\FieldDegree{\Field_3}{\Field_1} = \FieldDegree{\Field_3}{\Field_2} \cdot \FieldDegree{\Field_2}{\Field_1}$;
		\item se $\FieldExtension{\Field_3}{\Field_2}$ e $\FieldExtension{\Field_2}{\Field_1}$ sono algebriche, allora \`e algebrica anche $\FieldExtension{\Field_3}{\Field_1}$.
	\end{itemize}
\end{Theorem}
\Proof \`E chiaro che $\Field_1 \subseteq \Field_3$ e che $\Field_3$ induce su $\Field_1$ una struttura di campo coincidente con quella di $\Field_1$.
\par Siano $n = \FieldDegree{\Field_3}{\Field_2}$ e $m = \FieldDegree{\Field_2}{\Field_1}$. Sia $(b_k)_{k = 1}^n$ una base del $\Field_2$-spazio vettoriale $\Field_3$ e sia $(c_l)_{l = 1}^m$ una base del $\Field_1$-spazio vettoriale $\Field_2$: si verifica facilmente che la famiglia $(b_k c_l)_{k = 1, ..., n\\l = 1, ..., m}$ \`e una base del $\Field_1$-spazio vettoriale $\Field_3$.
\par Infine, se $\alpha \in \Field_3$ \`e algebrico su $\Field_2$, allora annulla un polinomio $p$ a coefficienti in $\Field_2$: sia $P$ l'insieme (finito) dei coefficienti di $p$. Poich\'e $\FieldExtension{\Field_2(P \cup \lbrace \alpha \rbrace)}{\Field_2(P)}$ e $\FieldExtension{\Field_2(P)}{\Field_1}$ sono estensioni finite, \`e finita anche l'estensione $\FieldExtension{\Field_2(P \cup \lbrace \alpha \rbrace)}{\Field_1}$ e dunque algebrica; ne segue che \`e algebrica anche $\FieldExtension{\Field_3}{\Field_1}$. \EndProof
\begin{Definition}
	Siano le estensioni di campi $\FieldExtension{\Field_3}{\Field_1}$ e $\FieldExtension{\Field_3}{\Field_2}$. Si definisce \Define{composto}[composto][campo] di $\Field_1$ e $\Field_2$, denotato $\FieldComposition{\Field_1}{\Field_2}$, il pi\`u piccolo sottocampo di $\Field_3$ contenente $\Field_1$ e $\Field_2$.
\end{Definition}
\begin{Theorem}
	Con le notazioni della definizione precedente, $\FieldComposition{\Field_1}{\Field_2} = \FieldAdjunction{\Field_1}{\Field_2} = \FieldAdjunction{\Field_2}{\Field_1}$.
\end{Theorem}
\Proof Immediata dalle definizioni. \EndProof
\begin{Theorem}\label{th_estensionealgebricamentegenerata}
	Sia l'estensione di campi $\FieldExtension{\Field_2}{\Field_1}$ e sia $S \subseteq \Field_2$. $\FieldExtension{\FieldAdjunction{\Field_1}{S}}{\Field_1}$ \`e algebrica se e solo se ogni elemento di $S$ \`e algebrico su $\Field_1$.
\end{Theorem}
\Proof Sia $\gamma \in \FieldAdjunction{\Field_1}{S}$. Allora per il teorema \ref{ThEstensioneUnione} esiste $T \subseteq S$ tale che $T$ \`e finito e $\gamma \in \FieldAdjunction{\Field_1}{T}$.Per la finitezza di $T$, $\FieldExtension{\FieldAdjunction{\Field_1}{T}}{\Field_1}$ \`e algebrica, dunque $\gamma$ \`e algebrico su $\Field_1$. Il viceversa \`e immediato. \EndProof
\begin{Exercice}
	Sia l'estensione di campi $\FieldExtension{\Field_2}{\Field_1}$, con $\Characteristic \Field_1 \neq 2$. Esiste $\Delta \in \Field_2$ tale Se $\alpha$ \`e un elemento di grado $2$ su $\Field_1$ e $\Delta$ il discriminante di $\MinimalPolynomial{\alpha}$, allora $\FieldAdjunction{\Field_1}{\alpha} = \FieldAdjunction{\Field_1}{\sqrt{\Delta}}$.
\end{Exercice}
\Solution $\alpha$ \`e soluzione dell'equazione di secondo grado a coefficienti in $\Field_1$ data da $\MinimalPolynomial{\alpha} = 0$. Poich\'e $\Characteristic \Field_1 \neq 2$, possiamo risolvere l'equazione con la formula classica e, ricavandonoe $\sqrt{\Delta}$, si prova che $\sqrt{\Delta} \in \FieldAdjunction{\Field_1}{\alpha}$. Un argomento dimensionale prova che le due estensioni di $\Field_1$ coincidono. \EndSolution
\begin{Exercice}
	Sia l'estensione di campi $\FieldExtension{\Field_2}{\Field_1}$. L'applicazione $\psi: \frac{\Field_1}{{\Invertible{\Field_1}}^2} \rightarrow \mathcal{F}$, dove $\mathcal{F}$ \`e l'insieme di tutte le sottoestensioni quadratiche $\FieldExtension{\Field_3}{\Field_1}$ di $\FieldExtension{\Field_2}{\Field_1}$, tale che $\psi(x{\Invertible{\Field_1}}^2) = \FieldAdjunction{\Field_1}{\sqrt{x}}$ \`e biettiva.
\end{Exercice}
\Solution Proviamo la suriettivit\`a di $\psi$. Sia $\FieldExtension{\Field_3}{\Field_1}$ un'estensione quadratica: essa \`e finita e dunque necessariamente algebrica; inoltre, essendo quadratica, \`e necessariamente semplice. Sia dunque $\alpha$ tale che $\Field_3 = \FieldAdjunction{\Field_1}{\alpha}$. Per il risultato dell'esercicio precedente, esiste $\Delta \in \Field_1$ tale che $\psi(\sqrt{\Delta}{\Invertible{\Field_1}}^2 = \FieldAdjunction{\Field_1}{\alpha}$.
\par Vediamo l'iniettivit\`a. Siano $\alpha, \beta \in \Field_1$ tali che $\FieldAdjunction{\Field_1}{\sqrt{\alpha}} = \FieldAdjunction{\Field_1}{\sqrt{\beta}}$. Allora, per opportuni $a, b \in \Field_1$, $\sqrt{\beta} = a + b \sqrt{\alpha}$, da cui $\beta = a^2 + 2ab\sqrt{\alpha} + b^2\alpha$; poich\'e $\beta \in \Field_1$, deve essere $ab = 0$.
\par Ora, se $a = 0$, allora $\beta = b^2\alpha$, da cui $\beta\alpha^{-1} = b^2 \in {\Invertible{\Field_1}}^2$; se invece $b = 0$, allora $\beta = a^2$, quindi $\FieldAdjunction{\Field_1}{\sqrt{\beta}} = \Field_1$ e quindi, di nuovo $\beta\alpha^{-1} \in {\Invertible{\Field_1}}^2$ e questo conclude la dimostrazione. \EndSolution
\begin{Exercice}
	Fornire un esempio di estensione algebrica non finita.
\end{Exercice}
\Solution Consideriamo l'insieme $S$ di tutte le radici $n$-esime di $2$ al variare di $n \in \mathbb{N}$: il campo $\FieldAdjunction{\mathbb{Q}}{S}$ \`e un'estensione algebrica non finita di $\mathbb{Q}$.
\par L'estensione \`e algebrica perch\'e $\sqrt[n]{2}$ \`e radice del polinomio $x^n - 2$; il grado dell'estensione \`e infinito perch\'e $x^n - 2$ \`e un polinomio irriducibile su $\mathbb{Q}$ per il criterio di Eisenstein e quindi $\FieldExtension{\FieldAdjunction{\mathbb{Q}}{S}}{\mathbb{Q}}$ contiene la sottoestensione $\FieldExtension{\FieldAdjunction{\mathbb{Q}}{\sqrt[n]{2}}}{\mathbb{Q}}$ di grado $n$ arbitrariamente grande. \EndSolution

\subsection{Trasferimenti di propriet\`a.}\label{TrasferimentiDiProprieta}
\begin{Definition}\label{defEstensioniTrasferimentoTorre}
	Data una torre d'estensioni $\Field_1 \subseteq \Field_2 \subseteq \Field_3$, se la validit\`a della propriet\`a $\Proposition$ per le estensioni $\FieldExtension{\Field_2}{\Field_1}$ e $\FieldExtension{\Field_3}{\Field_2}$ implica la validit\`a di $\Proposition$ per l'estensione $\FieldExtension{\Field_3}{\Field_1}$, allora si dice che $\Proposition$ \Define{si trasferisce lungo le torri}[lungo le torri][trasferimento di propriet\`a].
\end{Definition}
\begin{Definition}\label{defEstensioniTrasferimentoTraslazione}
	Date le torri d'estensioni $\Field_1 \subseteq \Field_2 \subseteq \FieldComposition{\Field_2}{\Field_3}$ e $\Field_1 \subseteq \Field_3 \subseteq \FieldComposition{\Field_2}{\Field_3}$, se la validit\`a della propriet\`a $\Proposition$ per l'estensione $\FieldExtension{\Field_2}{\Field_1}$ implica la validit\`a di $\Proposition$ per l'estensione $\FieldExtension{\FieldComposition{\Field_2}{\Field_3}}{\Field_3}$, allora si dice che $\Proposition$ \Define{si trasferisce per traslazione}[per traslazione][trasferimento di propriet\`a].
\end{Definition}
\begin{Definition}\label{defEstensioniTrasferimentoComposizione}
	Date le torri d'estensioni $\Field_1 \subseteq \Field_2 \subseteq \FieldComposition{\Field_2}{\Field_3}$ e $\Field_1 \subseteq \Field_3 \subseteq \FieldComposition{\Field_2}{\Field_3}$, se la validit\`a della propriet\`a $\Proposition$ per le estensioni $\FieldExtension{\Field_2}{\Field_1}$ e $\FieldExtension{\Field_3}{\Field_1}$ implica la validit\`a di $\Proposition$ per l'estensione $\FieldExtension{\FieldComposition{\Field_2}{\Field_3}}{\Field_1}$, allora si dice che $\Proposition$ \Define{si trasferisce per composizione}[per composizione][trasferimento di propriet\`a].
\end{Definition}
\begin{Theorem}
	Se la propriet\`a $\Proposition$  si trasferisce lungo le torri e per traslazione allora si trasferisce anche per composizione.
\end{Theorem}
\Proof Riprendiamo le notazioni della definizione \ref{defEstensioniTrasferimentoComposizione} e supponiamo $\Proposition$ verificata per $\FieldExtension{\Field_2}{\Field_1}$. Trasferiamo la propriet\`a $\Proposition$ per traslazione da $\FieldExtension{\Field_2}{\Field_1}$ a $\FieldExtension{\FieldComposition{\Field_2}{\Field_3}}{\Field_3}$. La tesi segue trasferendo $\Proposition$ lungo la torre $\Field_1 \subseteq \Field_3 \subseteq \FieldComposition{\Field_2}{\Field_3}$. \EndProof
\begin{Theorem}
	La propriet\`a di finitezza delle estensioni si trasferisce lungo le torri, per traslazione e per composizione.
\end{Theorem}
\Proof \`E sufficiente provare il trasferimento lungo le torri e per traslazione.
\par Il trasferimento lungo le torri \`e gi\`a stato dimostrato dal teorema \ref{thEstensioniTorri}. Per il traferimento per traslazione osserviamo che una famiglia finita di generatori del $\Field_1$-spazio vettoriale $\Field_2$ genera anche il $\Field_3$-spazio vettoriale $\FieldComposition{\Field_2}{\Field_3}$ e dunque $\FieldDegree{\FieldComposition{\Field_2}{\Field_3}}{\Field_3} \leq \FieldDegree{\Field_2}{\Field_1}$. \EndProof
\begin{Theorem}
	La propriet\`a di algebricit\`a delle estensioni si trasferisce lungo le torri, per traslazione e per composizione.
\end{Theorem}
\Proof \`E sufficiente provare il trasferimento lungo le torri e per traslazione.
\par Il trasferimento lungo le torri \`e gi\`a stato dimostrato dal teorema \ref{thEstensioniTorri}. Per il trasferimento per traslazione, riprendendo le notazioni della definizione \ref{defEstensioniTrasferimentoTraslazione}, supponiamo che $\FieldExtension{\Field_2}{\Field_1}$ sia algebrica. Abbiamo $\FieldComposition{\Field_2}{\Field_3} = \FieldAdjunction{\Field_3}{\Field_2}$; ora, tutti gli elementi di $\Field_2$ sono algebrici su $\Field_1$ e dunque a maggior ragione su $\Field_3$, quindi $\FieldExtension{\FieldComposition{\Field_2}{\Field_3}}{\Field_3}$ \`e un'estensione algebrica. \EndProof

\subsection{Chiusura algebrica, campi algebricamente chiusi e estensioni di omomorfismi.}\label{ChiusureAlgebricheECampiAlgebricamenteChiusi}
\begin{Theorem}\label{thChiusuraAlgebricaRelativa}
	Sia l'estensione di campi $\FieldExtension{\Field_2}{\Field_1}$. L'insieme $\Field_3$ di tutti gli elementi di $\Field_2$ algebrici su $\Field_1$, munito delle operazioni di somma e prodotto indotte da quelle di $\Field_2$ \`e un'estensione algebrica di $\Field_1$. Inoltre, se $\Field_3 \neq \Field_2$, allora $\FieldExtension{\Field_3}{\Field_2}$ \`e trascendente.
\end{Theorem}
\Proof Per la prima parte del teorema \`e sufficiente dimostrare che $\Field_3$ \`e un campo. Ma questo \`e immediato perch\'e
\begin{itemize}
	\item se $\alpha, \beta \in \Field_3$, allora $\alpha - \beta \in \FieldAdjunction{\Field_1}{\alpha, \beta} \subseteq \Field_3$;
	\item se $\alpha, \beta \in \Field_3 - \lbrace 0 \rbrace$, allora $\alpha\beta^{-1} \in \FieldAdjunction{\Field_1}{\alpha, \beta} \subseteq \Field_3$.
\end{itemize}
\par La trascendenza di $\FieldExtension{\Field_3}{\Field_2}$ segue dal fatto che l'algebricit\`a si trasferisce lungo le torri. \EndProof
\begin{Definition}
	Con le notazioni e le ipotesi del teorema \ref{thChiusuraAlgebricaRelativa}, $\Field_3$ si dice la \Define{chiusura algebrica}[algebrica (relativa)][chiusura] di $\Field_1$ in $\Field_2$.
\end{Definition}
\begin{Definition}
	Sia $\Field$ un campo. Si dice che $\Field$ \`e \Define{algebricamente chiuso}[algebricamente chiuso][campo] quando ogni polinomio $p \in \RingAdjunction{\Field}{x}$ ammette una radice in $\Field$.
\end{Definition}
\begin{Theorem}
	Sia $\Field$ un campo. Sono equivalenti i seguenti fatti:
	\begin{itemize}
		\item $\Field$ \`e algebricamente chiuso;
		\item ogni polinomio $p \in \RingAdjunction{\Field}{x} \SetMin \lbrace 0 \rbrace$ si spezza in fattori lineari;
		\item il polinomio $p \in \RingAdjunction{\Field}{x}$ \`e irriducibile se e solo se $\Deg{p} = 1$.
	\end{itemize}
\end{Theorem}
\Proof Assumiamo che $\Field$ sia algebricamente chiuso e sia $p \in \RingAdjunction{\Field}{x}$. Se $p$ \`e di grado $1$ esso \`e gi\`a un fattore lineare. Supponiamo che ogni polinomio di grado $n < \Deg{p}$ si spezzi in fattori lineare: $p$ ammette una radice $\alpha \in \Field$, dunque, per il teorema di Ruffini, $p$ si spezza nel fattore lineare $x - \alpha$ e in un polinomio $q$ di grado minore di $\Deg{p}$, il quale si spezza a sua volta in fattori lineari.
\par \`E immediato provare che la seconda affermazione implica la prima e che le ultime due sono equivalenti. \EndProof
\begin{Definition}
	Sia l'estensione di campi $\FieldExtension{\Field_2}{\Field_1}$. Si dice che $\Field_2$ \`e una \Define{chiusura algebrica}[algebrica][chiusura] di $\Field_1$ quando
	\begin{itemize}
		\item $\FieldExtension{\Field_2}{\Field_1}$ \`e un'estensione algebrica;
		\item $\Field_2$ \`e algebricamente chiuso.
	\end{itemize}
	Determinata una chiusura algebrica di un campo $\Field$, essa si denota con $\AlgebraicClosure{\Field}$.
\end{Definition}
\begin{Theorem}\label{th_esistenzaradici}
	Siano $\Field_1$ un campo $(p_i)_{i \in \lbrace 1, ..., n \rbrace}$ una famiglia di polinomi di $\RingAdjunction{\Field_1}{x}$. Allora esiste un'estensione algebrica $\Field_2$ di $\Field_1$ tale che per ogni $i \in \lbrace 1, ..., n \rbrace$ il polinomio $p_i$ ammette una radice in $\Field_2$
\end{Theorem}
\Proof Dimostriamo il teorema applicando la costruzione di estensioni algebriche di Leopold Kronecker (1823 -- 1891).
\par Supponiamo $n = 1$ e sia $f$ un fattore irriducibile di $p_1$. Allora $\SpanIdeal{f}$ \`e massimale nel dominio euclideo $\RingAdjunction{\Field_1}{x}$ e $\Field_2 = \Quotient{\RingAdjunction{\Field_1}{x}}{\SpanIdeal{f}}$ \`e un campo. Poich\'e $\Field_1$ \`e isomorfo a un sottocampo di $\Field_2$ per mezzo della proiezione canonica $\Projection$, ne consegue che $\Field_2$ pu\`o essere visto come un'estensione di $\Field_1$ e che $p$ si identifica col polinimio di $\RingAdjunction{\Field_2}{x}$ ottenuto da $p$ cambiando ogni coefficiente nella sua classe di equivalenza.
\par Inoltre il polinomio $p$, visto come un polinomio di $\RingAdjunction{\Field_2}{x}$, valutato in $\Projection{x}$ restituisce $\Projection{p} = 0$: il fattore $f$, e dunque il polinomio $p$, ammette una radice in $\Field_2$.
\par Per vedere l'algebricit\`a dell'estensione, \`e sufficiente notare che \`e finita: la famiglia $\left ( \Projection{x^i} \right )_{i = 0}^{\Deg f - 1}$ ne costituisce una base.
\par Se $n > 1$ il teorema segue per induzione. \EndProof
\begin{Theorem}
	Sia $\Field_0$ un campo. Assunto vero l'assioma della scelta, esiste una chiusura algebrica $\Field$ di $\Field_0$.
\end{Theorem}
\Proof Questa dimostrazione \`e di Emil Artin (1898 -- 1962).
\par Costruiamo una successione di estensioni algebriche $(\Field_i)_{i \in \mathbb{N}}$ nel modo seguente.
\par $\Field_0$ \`e dato per ipotesi. Sia $i \in \mathbb{N}$ e supponiamo costruito $\Field_i$: costruiamo $\Field_{i + 1}$. Consideriamo l'insieme $S$ di tutti i polinomi di $\Field_i$ di grado almeno $1$ e per ogni polinomio $p \in S$ introduciamo una nuova incognita $x_p$. Consideriamo inoltre l'anello di polinomi $\RingAdjunction{\Field_i}{(x_p)_{p \in S}}$ e l'ideale $\Ideal = \SpanIdeal{(p(x_p))_{p \in S}}$: dimostriamo che $\Ideal$ \`e un ideale proprio di $\RingAdjunction{\Field_i}{(x_p)_{p \in S}}$.
\par Supponiamo che $\Ideal$ non sia un ideale proprio. Allora esiste un insieme finito $T \subseteq S$ tale che $1 = \sum_{p \in T} a_p p(x_p)$ per un'opportuna famiglia di coefficienti $(a_p)_{p \in T} \in \Field_i^T$. Per il teorema \ref{th_esistenzaradici}, esite un campo $\tilde{\Field_i}$ che estende $\Field_i$ tale che ogni $p \in T$ vi ammetta una radice $\rho_p$. Applichiamo un omomorfismo di valutazione $\phi: \RingAdjunction{\tilde{\Field_i}}{(x_p)_{p \in T}} \rightarrow \tilde{\Field_i}$ tale che per ogni $p \in T$ abbiamo $x_p = \rho_p$: abbiamo allora $1 = \phi(1) = \sum_{p \in T} a_p p(\rho_p) = 0$, ma questo \`e assurdo perch\'e $\Field_1$ \`e un campo e quindi $0 \neq 1$.
\par $\Ideal$ \`e dunque un ideale proprio di $\RingAdjunction{\Field_i}{(x_p)_{p \in S}}$. Pertanto $\Ideal$ \`e contenuto in un ideale massimale $\MaximalIdeal$. Poniamo allora $\Field_{i + 1} = \Quotient{\RingAdjunction{\Field_i}{(x_p)_{p \in S}}}{\MaximalIdeal}$. Ora,
\begin{itemize}
	\item $\Field_{i + 1}$ \`e un campo perch\'e $\MaximalIdeal$ \`e massimale;
	\item $\Field_{i + 1}$ estende $\Field_i$ perch\'e $\Field_i$ si immerge canonicamente in $\RingAdjunction{\Field_1}{(x_p)_{p \in S}}$ che si immerge canonicamente in $\Field_{i + 1}$;
	\item $\FieldExtension{\Field_{i + 1}}{\Field_i}$ \`e un'estensione algebrica perch\'e $\Field_{i + 1} = \RingAdjunction{\Field_i}{(\Projection{x_p})_{p \in S}}$ \`e algebricamente generata (\Cfr\ teorema \ref{th_estensionealgebricamentegenerata}).
\end{itemize}
\par Poniamo infine $\Field = \bigcup_{i \in \mathbb{N}} \Field_i$, che \`e chiaramente un campo. Ora,
\begin{itemize}
	\item $\Field$ \`e un'estensione algebrica di $\Field_0$: infatti se $\alpha \in \Field$, allora $\alpha \in \Field_i$ per opportuno $i \in \mathbb{N}$, ma $\FieldExtension{\Field_i}{\Field_0}$ \`e un'estensione algebrica perch\'e l'algebricit\`a si trasferisce lungo le torri, quindi $\alpha$ \`e algebrico su $\Field_0$;
	\item $\Field$ \`e algebricamente chiuso: infatti qualsiasi polinomio $p \in \RingAdjunction{\Field}{x}$ include un numero finito di coefficienti i quali sono tutti contenuti in $\Field_i$ per $i \in \mathbb{N}$ opportuno, cosicch\'e $p \in \RingAdjunction{\Field_i}{x}$, e $p$ ammette una radice in $\Field_{i + 1}$ per costruzione ($\Projection{x_p}$). \EndProof
\end{itemize}
\begin{Theorem}
	Sia $\Field_0$ un campo infinito e $\Field$ una sua chiusura algebrica. Abbiamo $\Cardinality{\Field_0} = \Cardinality{\Field}$.
\end{Theorem}
\Proof Riprendiamo le costruzioni e le notazioni del teorema precedente e dimostriamo che per ogni $i \in \mathbb{N}$, $\Cardinality{\Field_{i + 1}} = \Cardinality{\Field_i}$.
\par In primo luogo osserviamo che $\Cardinality{\Field_i} = \Cardinality{\RingAdjunction{\Field_i}{x}}$. Infatti $\RingAdjunction{\Field_i}{x}$ \`e unione numerabile degli insiemi di tutti i polinomi di grado al pi\`u $n$, con $n$ che varia in $\mathbb{N}$: la sua cardinalit\`a \`e dunque minore o uguale a $\aleph_0 \cdot \sum_{n \in \mathbb{N}} \Cardinality{\Field_i}^n = \aleph_0 \cdot \Cardinality{\Field_i} = \Cardinality{\Field_i}$. Dato che $\Field_i \subseteq \RingAdjunction{\Field_i}{x}$, se ne conclude che $\Cardinality{\Field_i} = \Cardinality{\RingAdjunction{\Field_i}{x}}$.
\par Inoltre, ogni elemento di $\Field_{i + 1}$ \`e radice di un qualche polinomio in $\RingAdjunction{\Field_i}{x}$, dunque $\Cardinality{\Field_{i + 1}} \leq \Cardinality{\RingAdjunction{\Field_i}{x}} = \Cardinality{\Field_i}$ e da $\Field_i \subseteq \Field_{i + 1}$ si deduce che $\Cardinality{\Field_i} = \Cardinality{\Field_{i + 1}}$.
\par La dimostrazione \`e conclusa osservando che $\Field$ \`e unione numerabile di una famiglia di insiemi equicardinali a $\Field_0$. \EndProof
\begin{Theorem}
	Sia $\Field_1$ un campo e $\Field_2$ un campo algebricamente chiuso che estende $\Field_1$. La chiusura algebrica $\Field$ di $\Field_1$ in $\Field_2$ \`e chiusura algebrica di $\Field_1$.
\end{Theorem}
\Proof $\FieldExtension{\Field}{\Field_1}$ \`e algebrica per costruzione. Proviamo che $\Field$ \`e algebricamente chiuso.
\par Sia $p \in \RingAdjunction{\Field}{x} \subseteq \RingAdjunction{\Field_2}{x}$: il polinomio $p$ ammette una radice $a \in \Field_2$. Pertanto $\FieldExtension{\FieldAdjunction{\Field_2}{a}}{\Field}$ \`e un'estensione algebrica e dato che l'algebricit\`a si trasferisce lungo le torri, $\FieldExtension{\FieldAdjunction{\Field_2}{a}}{\Field_1}$ \`e algebrica. \`E quindi algebrico su $\Field_1$ l'elemento $a$, che appartiene dunque a $\Field$. \EndProof
\begin{Theorem}
	\TheoremName{Teorema fondamentale dell'algebra} Il campo $\mathbb{C}$ \`e algebricamente chiuso.
\end{Theorem}
\Proof Sia $p \in \RingAdjunction{\mathbb{C}}{x}$. Poniamo $f: \mathbb{C} \rightarrow \mathbb{R}^+$ che a $z \in \mathbb{C}$ associa $\Abs{p(z)}$: l'applicazione $f$ \`e continua.
\par Abbiamo $p = x^{\Deg{p}}\sum_{i = 0}^{\Deg p}a_ix^{i - \Deg{p}}$, da cui $f(z) \rightarrow \infty$ per $\Abs{z} \rightarrow \infty$.
\par Scegliamo $R > 0$ in modo tale che $\Abs{z} > R$ implichi $f(z) > \inf f + 1$; osserviamo che l'insieme $I$ dei numeri complessi di modulo minore o uguale di $R$ \`e compatto (\`e isomorfo al cerchio di centro $0$ e raggio $R$ di $\mathbb{R}^2$, chiuso e limitato): $f$ assume dunque un minimo in un punto $\mu \in I$, che \`e necessariamente un minimo assoluto.
\par Supponiamo ora che $f(\mu) \neq 0$. Introduciamo un nuovo polinomio $q \in \RingAdjunction{\mathbb{C}}{x}$ definito come $\frac{p(\mu - z)}{f(\mu)}$, insieme ad una nuova funzione $g: \mathbb{C} \rightarrow \mathbb{R}^+$ che a $z \in \mathbb{C}$ associa $\Abs{q(z)}$: abbiamo $g(z) = \Abs{\frac{p(\mu - z)}{f(\mu)}} = \frac{f(\mu - z)}{f(\mu)}$ e quindi $g$ ha un minimo assoluto in $0$ di valore $1$. Supponiamo $q = \sum_{i \in \mathbb{N}} a_ix^i$, dove $(a_i)_{i \in \mathbb{N}} \in \mathbb{C}^\mathbb{N}$, e poniamo $r = \min_{\NotZero{\mathbb{N}}} \lbrace i \in \mathbb{N} | a_i \neq 0 \rbrace$.
\par Sia $\lambda > 0$. Abbiamo, posto $\zeta \in \mathbb{C}$ tale che $\zeta^r = a_r^{-1}$,
\begin{align}
	g(- \lambda \zeta) &\leq \Abs{a_0 - a_r \lambda^r a_r^{-1}} + \Abs{\sum_{i \in \mathbb{N}} a_{r + 1 + i}\lambda^{r + 1 + i}\zeta^{r + 1 + i}},\\
	&= 1 - \lambda^r (1 - \lambda\Abs{\sum_{i \in \mathbb{N}} a_{r + 1 +i}\lambda^i\zeta^{r + 1 + i}}).
\end{align}
\par Ora, per $\lambda$ sufficientemente piccolo, l'espressione tra parentesi \`e poco meno di $1$ e quindi poco meno di $1$ \`e il secondo termine, cosicch\'e $g(- \lambda \zeta) < 1$, che \`e assurdo. Quindi $f(\mu) = 0$. \EndProof
\par Nel seguito, dato un omomorfismo di campi $\phi: \Field_1 \rightarrow \Field_2$ e un polinomio $p = \sum_{i \in \mathbb{N}} a_ix^i \in \RingAdjunction{\Field_1}{x}$, la notazione $\ImagePolynomial{\phi}{p}$ indicher\`a il polinomio $\sum_{i \in \mathbb{N}} \phi(a_i)x^i$.
\begin{Definition}
	Siano dati tre campi $\Field_1$, $\Field_2$ e $\Field_3$ e un'omomorfismo $\Immersion: \Field_1 \rightarrow \Field_3$ tali che
	\begin{itemize}
		\item $\Field_2$ estenda $\Field_1$;
		\item $\Immersion$ sia un'immersione.
	\end{itemize}
	Chiamiamo \Define{omomorfismo relativo}\footnote{Questa terminologia non pare molto frequente. Noi l'abbiamo presa da van der Waerden.} a $\Immersion$ su $\Field_1$ un omomorfismo $\RelativeHomomorphism: \Field_2 \rightarrow \Field_3$ tale che $\Restricted{\RelativeHomomorphism}{\Field_1} = \Immersion$ . La specifica dell'omomorfismo $i$ e del campo $\Field_1$ viene spesso omessa, soprattutto quando $\Field_1 \subseteq \Field_3$ ed $\Immersion$ \`e l'immersione canonica.
	\par Qualora $\Field_2 = \Field_3$, allora chiamiamo $\RelativeHomomorphism$ \Define{automorfismo di $\Field_2$ relativo}[relativo][automorfismo] a $\Immersion$  su $\Field_1$ e denotiamo $\RelativeAutomorphisms{\Field_2}{\Field_1}$ la classe di tutti questi automorfismi.
\end{Definition}
\begin{Theorem}\label{th_esistenzaomomorfismirelativi_0}
	Siano $\Field_1$ e $\Field_2$ campi tali che $\FieldExtension{\Field_2}{\Field_1}$ sia un'estensione algebrica semplice generata da $\alpha \in \Field_2$. Sia $\Field$ un ulteriore campo e sia $\Immersion: \Field_1 \hookrightarrow \Field$ un'immersione. L'applicazione $\phi: \Field_1 \cup \lbrace \alpha \rbrace \rightarrow \Field$ tale che $\Restricted{\psi}{\Field_1} = \Immersion$ pu\`o essere estesa ad un omomorfismo $\RelativeHomomorphism: \Field_2 \rightarrow \Field$ se e solo se $\phi(\alpha)$ \`e radice di $\ImagePolynomial{\Immersion}{\MinimalPolynomial{\alpha}}$.
\end{Theorem}
\Proof Per definizione di omomorfismo e per le ipotesi su $\alpha$, se $\MinimalPolynomial{\alpha} = \sum_{i \in \mathbb{N}} a_i x^i$, abbiamo $\ImagePolynomial{\Immersion}{\MinimalPolynomial{\alpha}}{\phi(\alpha)} = \sum_{i \in \mathbb{N}} \phi(a_i) \phi(\alpha)^i = \phi(\sum_{i \in \mathbb{N}} a_i \alpha^i) = \phi(0) = 0$. Dunque ogni omomorfismo relativo $\RelativeHomomorphism$ manda $\alpha$ in una radice di $\ImagePolynomial{\Immersion}{\MinimalPolynomial{\alpha}}$.
\par Viceversa, supponiamo che $\phi(\alpha)$ sia radice di $\ImagePolynomial{\Immersion}{\MinimalPolynomial{\alpha}}$. Vogliamo definire l'applicazione $\RelativeHomomorphism$ come l'applicazione che a $\sum_{i \in \mathbb{N}} c_i \alpha^i$ associa $\sum_{i \in \mathbb{N}} \Immersion(c_i) \phi(\alpha)^i$, ma per farlo, dobbiamo verificare che $\sum_{i \in \mathbb{N}} c_i \alpha^i = \sum_{i \in \mathbb{N}} d_i \alpha^i$ implica $\sum_{i \in \mathbb{N}} \Immersion(c_i) \phi(\alpha)^i = \sum_{i \in \mathbb{N}} \Immersion(d_i) \phi(\alpha)^i$.
\par In effetti, assumiamo $\sum_{i \in \mathbb{N}} c_i \alpha^i = \sum_{i \in \mathbb{N}} d_i \alpha^i$. Definiamo il polinomio $p = \sum_{i \in \mathbb{N}} (c_i - d_i)x^i$: $\alpha$ \`e radice di $p$ e dunque $\MinimalPolynomial{\alpha}|p$. Se ne deduce che $\ImagePolynomial{\Immersion}{\MinimalPolynomial{\alpha}}|\ImagePolynomial{\Immersion}{p}$ e, per le ipotesi su $\phi(\alpha)$, $\ImagePolynomial{\Immersion}{p}(\alpha) = 0$, da cui $\sum_{i \in \mathbb{N}} \Immersion(c_i) \phi(\alpha)^i = \sum_{i \in \mathbb{N}} \Immersion(d_i) \phi(\alpha)^i$.
\par La verifica che l'applicazione $\RelativeHomomorphism$ sia effettivamente un omomorfismo \`e immediata. \EndProof
\begin{Corollary}\label{th_esistenzaomomorfismirelativi}
	Siano $\Field_1$ e $\Field_2$ campi tali che $\FieldExtension{\Field_2}{\Field_1}$ sia un'estensione algebrica semplice generata da $\alpha \in \Field_2$. Sia $\Field$ un ulteriore campo e sia $\phi: \Field_1 \hookrightarrow \Field$ un'immersione. Il numero di omomorfismi $\psi: \Field_2 \rightarrow \Field$ che estendono $\phi$ \`e uguale al numero di elementi di $\Field$ che sono radici del polinomio $\ImagePolynomial{\phi}{\MinimalPolynomial{\alpha}}$.
\end{Corollary}
\Proof Poich\'e un omomorfismo da $\Field_2$ in $\Field$ \`e anche un'applicazione lineare tra i $\Field_1$-spazio vettoriali $\Field_2$ e $\Field$, ad ogni elemento di $m \in \Field_2$ corrisponde al pi\`u un omomorfismo $\psi$ che estende $\phi$ tale che $\psi(\alpha) = m$. \EndProof
\begin{Corollary}
	Siano $\Field_1$ e $\Field_2$ campi tali che $\FieldExtension{\Field_2}{\Field_1}$ sia un'estensione algebrica semplice generata da $\alpha \in \Field_2$. Sia $\Immersion: \Field_1 \hookrightarrow \AlgebraicClosure{\Field_1}$ un'immersione. Il numero di omomorfismi relativi $\RelativeHomomorphism: \Field_2 \rightarrow \AlgebraicClosure{\Field_1}$ che estendono \`e uguale al numero di radice distinte di $\MinimalPolynomial{\alpha}$ in $\AlgebraicClosure{\Field_1}$.
\end{Corollary}
\Proof \`E conseguenza immediata del teorema precedente. \EndProof
\begin{Theorem}
	Siano $\Field_1$ e $\Field_2$ campi tali che $\FieldExtension{\Field_2}{\Field_1}$ sia un'estensione algebrica. Sia $\Field$ un campo algebricamente chiuso e sia $\Immersion: \Field_1 \hookrightarrow \Field$ un'immersione. Se \`e vero l'assioma della scelta, esiste un omomorfismo relativo $\RelativeHomomorphism: \Field_2 \rightarrow \Field$.
\end{Theorem}
\Proof Consideriamo l'insieme $I = \lbrace (\tilde{\Field},\tilde{\RelativeHomomorphism}) | \text{$\tilde{\Field}$ \`e un'estensione di $\Field_1$, $\tilde{\Field} \subseteq \Field_2$ e $\tilde{\RelativeHomomorphism}: \tilde{\Field} \rightarrow \Field$ \`e un omomorfismo relativo}$. Ordiniamo parzialmente $I$ definendo $(\tilde{\Field_1},\tilde{\RelativeHomomorphism_1}) \leq (\tilde{\Field_2},\tilde{\RelativeHomomorphism_2})$ se e solo se $\tilde{\Field_1} \subseteq \tilde{\Field_2}$ e $\tilde{\RelativeHomomorphism_2}$ prolunga $\tilde{\RelativeHomomorphism_1}$ (cio\`e \`e un omomorfismo relativo a $\tilde{\RelativeHomomorphism_1}$).
\par $I$ \`e non vuoto perch\'e $(\Field_1,\Immersion) \in I$.
\par Ogni catena ascendente di $(\tilde{\Field_i},\tilde{\RelativeHomomorphism_i})_{i \in S} \in I^S$, dove $S$ \`e un insieme qualsiasi, \`e maggiorata da $\left ( \bigcup_{i \in S} \tilde{\Field_i}, \tilde{\RelativeHomomorphism} \right )$, dove $\tilde{\RelativeHomomorphism}$ \`e l'unica applicazione da $\bigcup_{i \in S} \tilde{\Field_i}$ in $\Field$ che prolunga tutte le $\tilde{\RelativeHomomorphism_i}$. Quindi per il lemma di Zorn $I$ ammette un elemento massimale $(\Field_M,\RelativeHomomorphism_M)$.
\par Concludiamo la dimostrazione provando che $\Field_M = \Field_2$. Se fosse diversamente, esisterebbe $\alpha \in \SetMin{\Field_2}{\Field_M}$ e $(\FieldAdjunction{\Field_M}{\alpha},\tilde{\RelativeHomomorphism_M})$, dove $\tilde{\RelativeHomomorphism_M}$ \`e un omomorfismo che prolunga $\RelativeHomomorphism_M$ (la sua esistenza \`e garantita dal corollario precedente) sarebbe un elemento di $I$ strettamente maggiore dell'elemento massimale $(\Field_M,\RelativeHomomorphism_M)$, e questo sarebbe assurdo. \EndProof
\begin{Theorem}\label{thUnicitaChiusuraAlgebrica}
	Sia $\Field$ un campo. $\Field$ ammette una e una sola chiusura algebrica a meno di isomorfismo.
\end{Theorem}
\Proof Siano $\AlgebraicClosure{\Field}_1$ e $\AlgebraicClosure{\Field}_2$ due chiusure algebriche di $\Field$. Per il teorema precedente, esiste un omomorfismo relativo $\RelativeHomomorphism: \AlgebraicClosure{\Field}_1 \rightarrow \AlgebraicClosure{\Field}_2$.
\par $\RelativeHomomorphism$ \`e un omomorfismo iniettivo perch\'e non nullo.
\par Proviamo che $\RelativeHomomorphism$ \`e suriettivo.
\par $\RelativeHomomorphism(\AlgebraicClosure{\Field}_1)$ \`e un campo. Sia $p \in \RingAdjunction{\RelativeHomomorphism(\AlgebraicClosure{\Field}_1)}{x}$. Esite $q \in \RingAdjunction{\AlgebraicClosure{\Field}_1}{x}$ tale che $\ImagePolynomial{\RelativeHomomorphism}{q} = p$. Poich\'e $\AlgebraicClosure{\Field}_1$ \`e algebricamente chiuso, esiste $\alpha \in \AlgebraicClosure{\Field}_1$ tale che $q(\alpha) = 0$. Si verifica subito che $p(\RelativeHomomorphism(\alpha)) = 0$ e quindi $\RelativeHomomorphism(\AlgebraicClosure{\Field}_1)$ \`e algebricamente chiuso.
\par Ora, sia $\beta \in \AlgebraicClosure{\Field}_2$. Poich\'e $\beta$ \`e algebrico su $\Field$, esso \`e algebrico anche su $\RelativeHomomorphism{\AlgebraicClosure{\Field}_1}$ e dunque appartiene a $\phi{\AlgebraicClosure{\Field}_1}$, essendo esso algebricamente chiuso. Ne consegue che $\Field_2 \subseteq \RelativeHomomorphism{\AlgebraicClosure{\Field}_1}$ e dato che vale anche il contenimento inverso, $\Field_2 = \RelativeHomomorphism{\AlgebraicClosure{\Field}_1}$. \EndProof

\subsection{Campi di spezzamento ed estensioni normali.}\label{CampiDiSpezzamento}
\begin{Definition}
	Siano
	\begin{itemize}
		\item $\Field$ un campo;
		\item $I$ un insieme qualsiasi;
		\item $(p_i)_{i \in I} \in \RingAdjunction{\Field}{x}^I$.
	\end{itemize}
	Si definisce \Define{campo di spezzamento}[di spezzamento][campo] di $(p_i)_{i \in I}$ su $\Field$ un campo $\SplittingField$ tale che
	\begin{itemize}
		\item $\SplittingField$ estende $\Field$;
		\item ogni $p_i$ ($i \in I$) si spezza in fattori lineari in $\SplittingField$;
		\item l'insieme di tutte le radici di tutti i $p_i$ al variare di $i \in I$ genera $\FieldAdjunction{\SplittingField}{\Field}$.
	\end{itemize}
\end{Definition}
\begin{Theorem}
	Siano
	\begin{itemize}
		\item $\Field_1$ un campo;
		\item $I$ un insieme qualsiasi;
		\item $(p_i)_{i \in I} \in \RingAdjunction{\Field_1}{x}^I$;
		\item $\Field_2$ un campo in cui tutti i $p_i$ ($i \in I$) si spezzano in fattori lineari;
	\end{itemize}
	Esiste un unico campo di spezzamento $\FieldSpezzamento$ di $(p_i)_{i \in I}$ tale che $\FieldSpezzamento \subseteq \Field_2$.
\end{Theorem}
\Proof Sia $A$ l'insieme di tutte le radici di tutti i polinomi $(p_i)_{i \in I}$ nella chiusura algebrica $\Field_2$. Necessariamente $\FieldSpezzamento = \FieldAdjunction{\Field}{A}$. \EndProof
\begin{Theorem}
	Siano
	\begin{itemize}
		\item $\Field$ un campo;
		\item $I$ un insieme qualsiasi;
		\item $(p_i)_{i \in I} \in \RingAdjunction{\Field_1}{x}^I$;
		\item $\FieldSpezzamento$ un campo di spezzamento di $(p_i)_{i \in I}$ su $\Field$;
		\item $\FieldSpezzamento_S$ il campo di spezzamento di $(p_i)_{i \in S}$ su $\Field$ contenuto in $\FieldSpezzamento$.
	\end{itemize}
	Abbiamo,
	\begin{itemize}
		\item $\FieldSpezzamento = \bigcup_{S \subseteq I | \Cardinality{S} \in \mathbb{N}} \FieldSpezzamento_S = \prod_{S \subseteq I | \Cardinality{S} \in \mathbb{N}} \FieldSpezzamento_S$;
		\item $\FieldSpezzamento = \bigcup_{S \subseteq I | \Cardinality{S} = 1} \FieldSpezzamento_S = \prod_{S \subseteq I | \Cardinality{S} = 1} \FieldSpezzamento_S$;
	\end{itemize}
\end{Theorem}
\Proof Immediata dalle definizioni. \EndProof
\begin{Theorem}\label{th_pre_unicitacampospezzamento}
	Siano
	\begin{itemize}
		\item $\Field$ un campo;
		\item $I$ un insieme qualsiasi;
		\item $(p_i)_{i \in I} \in \RingAdjunction{\Field_1}{x}^I$;
		\item $\FieldSpezzamento_1$ e $\FieldSpezzamento_2$ campi di spezzamento di $(p_i)_{i \in I}$ su $\Field$;
		\item $\RelativeHomomorphism: \FieldSpezzamento_1 \rightarrow \AlgebraicClosure{\FieldSpezzamento_2}$ un omommorfismo relativo.
	\end{itemize}
	Abbiamo $\Image \RelativeHomomorphism = \FieldSpezzamento_2$.
\end{Theorem}
\Proof \`E sufficiente provare il teorema nel caso in cui $I$ \`e un singoletto: infatti il teorema precedente permette di concludere considerando, con le notazioni del teorema precedente, che $\RelativeHomomorphism \left ( \bigcup_{S \subseteq I | \Cardinality{S} = 1} \FieldSpezzamento_S \right ) = \bigcup_{S \subseteq I | \Cardinality{S} = 1} \RelativeHomomorphism(\FieldSpezzamento_S)$.
\par Denotiamo con $p$ l'unico polinomio della famiglia $(p_i)_{i \in I}$. In $\FieldSpezzamento_1$ abbiamo $p = c \prod_{i = 1}^{\Deg{p}} (x - \alpha_i)$, dove $c \in \Field$ e $(\alpha)_{i = 1}^{\Deg{p}}$ sono tutte le radici di $p$ (con molteplicit\`a); tramite $\RelativeHomomorphism$ abbiamo poi $\ImagePolynomial{\RelativeHomomorphism}{p} = p = c \prod_{i = 1}^{\Deg{p}} (x - \RelativeHomomorphism(\alpha_i)$: vediamo dunque che $p$ si spezza in fattori lineari in $\Image \RelativeHomomorphism$, da cui $\FieldSpezzamento_2 = \FieldAdjunction{\Field}{(\RelativeHomomorphism(\alpha_i))_{i = 1}^{\Deg{p}}} \subseteq \Image \RelativeHomomorphism$. L'inclusione inversa si prova osservando che \`e sufficiente specifiare le immagini $(\RelativeHomomorphism(\alpha_i))_{i = 1}^{\Deg{p}}$ per definire completamente l'omomorfismo $\RelativeHomomorphism$. \EndProof
\begin{Corollary}
	Siano
	\begin{itemize}
		\item $\Field$ un campo;
		\item $I$ un insieme qualsiasi;
		\item $(p_i)_{i \in I} \in \RingAdjunction{\Field_1}{x}^I$;
		\item $\FieldSpezzamento_1$ e $\FieldSpezzamento_2$ campi di spezzamento di $(p_i)_{i \in I}$ su $\Field$.
	\end{itemize}
	$\FieldSpezzamento_1$ e $\FieldSpezzamento_2$ sono isomorfi.
\end{Corollary}
\Proof Basta prendere un omomorfismo relativo $\RelativeHomomorphism: \FieldSpezzamento_1 \rightarrow \AlgebraicClosure{\FieldSpezzamento_2}$: per il teorema precedente, cambiandone il codominio in $\FieldSpezzamento_2$ otteniamo un isomorfismo. \EndProof
\begin{Definition}
	Sia $\FieldExtension{\Field_2}{\Field_1}$ un'estensione di campi algebrica. Se per ogni omomorfismo relativo $\RelativeHomomorphism: \Field_2 \rightarrow \AlgebraicClosure{\Field_2}$ abbiamo $\Image \RelativeHomomorphism = \Field_2$, allora chiamiamo \Define{normale}[normale][estensione] l'estensione $\FieldExtension{\Field_2}{\Field_1}$.
\end{Definition}
\begin{Theorem}
	Sia $\FieldExtension{\Field_2}{\Field_1}$ un'estensione di campi algebrica e sia $\RelativeHomomorphism: \Field_2 \rightarrow \AlgebraicClosure{\Field_2}$ un omomorifsmo relativo. Se $\Image \RelativeHomomorphism \subseteq \Field_2$, allora l'estensione $\FieldExtension{\Field_2}{\Field_1}$ \`e normale.
\end{Theorem}
\Proof Sia $\alpha \in \Field_2$. $\RelativeHomomorphism$ \`e un omomorfismo relativo dunque trasforma radici del polinomio minimo $\MinimalPolynomial{\alpha} \in \RingAdjunction{\Field_1}{x}$ in $\Field_2$ in altre radici del medesimo polinomio, stavolta in $\AlgebraicClosure{\Field_2}$, ma poich\'e $\Image \RelativeHomomorphism \subseteq \Field_2$, anche quest'ultime devono appartenere a $\Field_2$. Dato inoltre che l'insieme di radici di $\MinimalPolynomial{\alpha}$ contenute in $\Field_2$ \`e finito e che $\RelativeHomomorphism$ \`e iniettiva, $\RelativeHomomorphism$ deve essere biettiva e dunque esiste $\beta \in \Field_2$ tale che $\RelativeHomomorphism(\beta) = \alpha$. \EndProof
\begin{Theorem}
	Sia $\FieldExtension{\Field_2}{\Field_1}$ un'estensione algebrica. Sono fatti equivalenti:
	\begin{itemize}
		\item $\FieldExtension{\Field_2}{\Field_1}$ \`e normale;
		\item se $\Polynomial \in \RingAdjunction{\Field_1}{x}$ \`e irriducibile e ammette una radice in $\Field_2$, allora $\Polynomial$ si spezza in fattori lineari in $\RingAdjunction{\Field_2}{x}$;
		\item $\Field_2$ \`e il campo di spezzamento di una famiglia di polinomi $(\Polynomial_i)_{i \in I} \in \RingAdjunction{\Field_1}{x}^I$, dove $I$ \`e un insieme opportuno, su $\Field_1$.
	\end{itemize}
\end{Theorem}
\Proof Supponiamo che $\FieldExtension{\Field_2}{\Field_1}$ sia normale e sia $\Polynomial \in \RingAdjunction{\Field_1}{x}$ irriducibile. Sia $\alpha \in \Field_2$ una radice di $\Polynomial$: abbiamo $\MinimalPolynomial{\alpha} = \LC{\Polynomial}^{-1} p$. Fissata una radice $\beta \in \AlgebraicClosure{\Field_2}$ di $\MinimalPolynomial{\alpha}$, consideriamo l'omomorfismo relativo $\RelativeHomomorphism_1: \FieldAdjunction{\Field_1}{\alpha} \rightarrow \AlgebraicClosure{\Field_2}$ tale che $\RelativeHomomorphism_1(\alpha) = \beta$ e dunque un ulteriore omomorfismo $\RelativeHomomorphism_2: \Field_2 \rightarrow \AlgebraicClosure{\Field_2}$, stavolta relativo a $\RelativeHomomorphism_1$: per la normalit\`a di $\FieldExtension{\Field_2}{\Field_1}$, abbiamo $\Image \RelativeHomomorphism_2 = \Field_2$, da cui $\beta \in \Field_2$. Dunque $p$ si spezza in fattori lineari in $\Field_2$.
\par Supponiamo ora che ogni polinomio $\Polynomial \in \RingAdjunction{\Field_1}{x}$ irriducibile che ammetta una radice in $\Field_2$ si spezzi in fattori lineari in $\RingAdjunction{\Field_2}{x}$. Consideriamo la famiglia di polinomi $(\MinimalPolynomial{\alpha})_{\alpha \in \Field_2}$, dove per ogni $\alpha \in \Field_2$, $\MinimalPolynomial{\alpha}$ \`e il polinomio minimo di $\alpha$ su $\Field_1$: per le nostre ipotesi, il campo di spezzamento $\FieldSpezzamento$ di $(\MinimalPolynomial{\alpha})_{\alpha \in \Field_2}$ \`e sottoinsieme di $\Field_2$, mentre l'inclusione inversa \`e immediata per costruzione della famiglia $(\MinimalPolynomial{\alpha})_{\alpha \in \Field_2}$.
\par Supponiamo infine che $\Field_2$ sia il campo di spezzamento di una famiglia di polinomi $(\Polynomial_i)_{i \in I} \in \RingAdjunction{\Field_1}{x}^I$, dove $I$ \`e un insieme opportuno. Per il teorema \ref{th_pre_unicitacampospezzamento}, dove prendiamo entrambi i campi di spezzamento uguali a $\Field_2$, $\FieldExtension{\Field_2}{\Field_1}$. \EndProof
\begin{Corollary}
	Sia $\Field$ un campo. L'estensione $\FieldExtension{\AlgebraicClosure{\Field}}{\Field}$ \`e normale.
\end{Corollary}
\Proof Ogni polinomio $\Polynomial \in \RingAdjunction{\Field}{x}$ si spezza in fattori lineari in $\AlgebraicClosure{\Field}$. \EndProof
\begin{Theorem}
	Tutte le estensioni algebriche di grado $2$ sono normali.
\end{Theorem}
\Proof Sia $\FieldExtension{\Field_2}{\Field_1}$ un'estensione di grado $2$. Sia $\alpha \in \Field_2 \SetMin \Field_1$: necessariamente il polinomio minimo di $\alpha$ \`e di grado $2$ e dunque ammette in $\AlgebraicClosure{\Field_1}$ al pi\`u due radici distinte. Ora, se $c \in \Field_1$ \`e il termine noto di $\MinimalPolynomial{\alpha}$, allora si verifica immediatamente considerando il campo di spezzamento di $\MinimalPolynomial{\alpha}$ su $\Field_1$ che $\frac{c}{\alpha}$ \`e l'altra radice di $\MinimalPolynomial{\alpha}$ oltre ad $\alpha$. Se ne deduce che $\Field_2$ \`e il campo di spezzamento del polinomio $\MinimalPolynomial{\alpha}$ su $\Field_1$. \EndProof
\begin{Theorem}
	La normalit\`a non si trasferisce lungo le torri.
\end{Theorem}
\Proof Consideriamo la torre d'estensioni $\mathbb{Q} \subseteq \FieldAdjunction{\mathbb{Q}}{\sqrt{2}} \subseteq \FieldAdjunction{\mathbb{Q}}{\sqrt[4]{2}}$. $\FieldExtension{\FieldAdjunction{\mathbb{Q}}{\sqrt{2}}}{\mathbb{Q}}$ \`e normale perch\'e $\FieldAdjunction{\mathbb{Q}}{\sqrt{2}}$ \`e il campo di spezzamento di $x^2 - 2$ su $\mathbb{Q}$ e $\FieldExtension{\FieldAdjunction{\mathbb{Q}}{\sqrt[4]{2}}}{\FieldAdjunction{\mathbb{Q}}{\sqrt{2}}}$ \`e normale perch\'e $\FieldAdjunction{\mathbb{Q}}{\sqrt[4]{2}}$ \`e campo di spezzamento di $x^2 - \sqrt{2}$ su $\FieldAdjunction{\mathbb{Q}}{\sqrt{2}}$.
\par Ma $\FieldExtension{\FieldAdjunction{\mathbb{Q}}{\sqrt[4]{2}}}{\mathbb{Q}}$ non \`e normale perch\'e in $\mathbb{C}$, estensione algebricamente chiusa di $\FieldAdjunction{\mathbb{Q}}{\sqrt[4]{2}}$, il polinomio minimo di $\sqrt[4]{2}$ su $\mathbb{Q}$ \`e $x^4 - 2$ (irriducibile per il criterio di Eisenstein) ha la radice $i\sqrt[4]{2}$ che non appartiene a $\FieldAdjunction{\mathbb{Q}}{\sqrt[4]{2}}$. \EndProof
\begin{Theorem}
	Consideriamo la torre d'estensioni $\Field_1 \subseteq \Field_2 \subseteq \Field_3$. Se $\FieldExtension{\Field_3}{\Field_1}$ \`e normale, allora
	\begin{itemize}
		\item $\FieldExtension{\Field_3}{\Field_2}$ \`e normale;
		\item non necessariamente $\FieldExtension{\Field_2}{\Field_1}$ \`e normale.
	\end{itemize}
\end{Theorem}
\Proof Per la prima parte, osserviamo che la normalit\`a dell'estensione $\FieldExtension{\Field_3}{\Field_1}$ implica che $\Field_3$ sia il campo di spezzamento di una famiglia di polinomi a coefficienti in $\Field_1$, i quali appartengono a maggior ragione a $\Field_2$, e dunque $\FieldExtension{\Field_3}{\Field_2}$ \`e normale.
\par Per la seconda parte considerimano la torre d'estensioni $\mathbb{Q} \subseteq \FieldAdjunction{\mathbb{Q}}{\sqrt[3]{2}} \subseteq \FieldAdjunction{\mathbb{Q}}{\sqrt[3]{2},\zeta_3}$, dove $\zeta_3 = \frac{- 1 + i\sqrt{3}}{2}$ \`e una radice cubica non reale dell'unit\`a\footnote{Stiamo anticipando la notazione delle radici primitive dell'unit\`a, \Cfr\ \ref{radici_primitive_def}.}.
\par Essa verifica le ipotesi dell'enunciato: infatti $\pm \zeta_3 \sqrt[3]{2}$ e $- \sqrt[3]{2}$ sono tutte e sole le radici del polinomio $x^3 - 2$ e dunque $\FieldAdjunction{\mathbb{Q}}{\sqrt[3]{2}}$ contiene il campo di spezzamento di $x^3 - 2$ su $\mathbb{Q}$ ed \`e addirittura uguale ad esso perch\'e $\zeta_3 = \frac{- \sqrt[3]{2}}{- \zeta_3 \sqrt[3]{2}}$.
\par D'altra parte, $x^3 - 2$ \`e un polinomio irriducibile per il criterio di Eisenstein e questo prova che $\FieldExtension{\FieldAdjunction{\mathbb{Q}}{\sqrt[3]{2}}}{\mathbb{Q}}$ non \`e normale. \EndProof
\begin{Theorem}
	La normalit\`a si trasferisce per traslazione.
\end{Theorem}
\Proof Siano date le torri d'estensioni $\Field_1 \subseteq \Field_2 \subseteq \FieldComposition{\Field_2}{\Field_3}$ e $\Field_1 \subseteq \Field_3 \subseteq \FieldComposition{\Field_2}{\Field_3}$. Consideriamo un omomorifsmo relativo $\RelativeHomomorphism: \FieldComposition{\Field_2}{\Field_3} \rightarrow \AlgebraicClosure{\FieldComposition{\Field_2}{\Field_3}}$ su $\Field_3$. Osservato che $\RelativeHomomorphism$ ristretto a $\Field_2$ \`e un omomorfismo relativo su $\Field_1$, abbiamo $\RelativeHomomorphism(\FieldComposition{\Field_2}{\Field_3}) = \FieldComposition{\RelativeHomomorphism(\Field_2)}{\RelativeHomomorphism{\Field_3}} = \FieldComposition{\Field_2}{\Field_3}$, da cui $\FieldExtension{\FieldComposition{\Field_2}{\Field_3}}{\Field_3}$ \`e normale. \EndProof
\begin{Theorem}
	La normalit\`a si trasferisce per composizione.
\end{Theorem}
\Proof Siano date le torri d'estensioni $\Field_1 \subseteq \Field_2 \subseteq \FieldComposition{\Field_2}{\Field_3}$ e $\Field_1 \subseteq \Field_3 \subseteq \FieldComposition{\Field_2}{\Field_3}$. Sia $\RelativeHomomorphism: \FieldComposition{\Field_2}{\Field_3} \rightarrow \AlgebraicClosure{\Field_2}$ un omomorfismo relativo su $\Field_1$. Abbiamo $\RelativeHomomorphism(\FieldComposition{\Field_2}{\Field_3}) = \FieldComposition{\RelativeHomomorphism(\Field_2)}{\RelativeHomomorphism(\Field_3)} = \FieldComposition{\Field_2}{\Field_3}$. \EndProof
\begin{Theorem}
	Siano dati i campi $\Field_1, \Field_2, \Field_3$ tali che $\Field_2$ e $\Field_3$ estendano $\Field_1$ e che entrambe le estensioni siano normali. $\FieldExtension{\Field_2 \cap \Field_3}{\Field_1}$ \`e normale.
\end{Theorem}
\Proof Siano date le torri d'estensioni $\Field_1 \subseteq \Field_2 \subseteq \FieldComposition{\Field_2}{\Field_3}$ e $\Field_1 \subseteq \Field_3 \subseteq \FieldComposition{\Field_2}{\Field_3}$. Sia $\RelativeHomomorphism: \Field_2 \cap \Field_3 \rightarrow \AlgebraicClosure{\Field_2}$ un omomorfismo relativo su $\Field_1$. Abbiamo, ricordano che $\RelativeHomomorphism$ \`e un'applicazione iniettiva, $\RelativeHomomorphism(\Field_2 \cap \Field_3) = \RelativeHomomorphism(\Field_2) \cap \RelativeHomomorphism(\Field_3) = \Field_2 \cap \Field_3$. \EndProof
\begin{Definition}
	Sia l'estensione di campo $\FieldExtension{\Field_2}{\Field_1}$. Si chiama \Define{chiusura normale}[normale][chiusura] di $\Field_1$ rispetto a $\Field_2$ la pi\`u piccola estensione\footnote{L'unicit\`a, a meno di isomorfismo, segue direttamente dal teorema di unicit\`a dei campi di spezzamento.} $\Field_3$ di $\Field_2$ tale che $\FieldExtension{\Field_3}{\Field_1}$ \`e normale: la denoteremo $\NormalClosure{\Field_1}[\Field_2]$, omettendo il campo $\Field_2$ quando esso \`e chiaro dal contesto.
\end{Definition}
\begin{Theorem}
	Sia l'estensione di campo $\FieldExtension{\Field_2}{\Field_1}$. La chiusura normale di $\Field_2$ rispetto a $\Field_1$ \`e il campo $\Field_3 = \BigComposition{\phi \in I}{\phi(\Field_2)}$, dove $I$ \`e l'insieme di tutte le immersioni $\phi: \Field_2 \rightarrow \AlgebraicClosure{\Field_2}$ (non necessariamente omomorfismi relativi).
\end{Theorem}
\Proof Segue immediatamente dalla costruzione di $\Field_3$ che $\Field_2 \subseteq \Field_3$.
\par Proviamo che $\Field_3$ \`e normale. In effetti, se $\RelativeHomomorphism: \Field_3 \rightarrow \AlgebraicClosure{\Field_1}$ \`e un omomorfismo relativo a $\Field_1$ abbiamo $\RelativeHomomorphism(\Field_3) = \BigComposition{\phi \in I}{(\RelativeHomomorphism \circ \phi)(\Field_2)} \subseteq \Field_3$, poich\'e la composizione di immersioni \`e un'immersione.
\par Proviamo ora che $\Field_3 \subseteq \NormalClosure{\Field_2}$. Fissata $\phi \in I$, sia $\RelativeHomomorphism: \NormalClosure{\Field_2} \rightarrow \AlgebraicClosure{\Field_1}$ un omomorfismo relativo a $\phi$: abbiamo $\phi(\Field_2) \subseteq \RelativeHomomorphism(\NormalClosure{\Field_2}) = \NormalClosure{\Field_2}$. Per l'arbitrariet\`a di $\phi \in I$, abbiamo $\Field_3 \subseteq \NormalClosure{\Field_2}$. \EndProof
\begin{Theorem}	
	Se l'estensione di campo $\FieldExtension{\Field_2}{\Field_1}$ \`e finita, allora \`e finita anche $\FieldExtension{\NormalClosure{\Field_2}}{\Field_1}$.
\end{Theorem}
\Proof Se $\FieldExtension{\Field_2}{\Field_1}$ \`e finita allora \`e algebrica e generata da una famiglia di elementi algebrici $(\alpha_i)_{i = 1}^n$, $n \in \mathbb{N}$. Quindi $\NormalClosure{\Field_2}$ deve essere il campo di spezzamento dei polinomi minimi $(\MinimalPolynomial{\alpha_i})_{i =1}^n$ ed \`e finitamente generata e quindi finita. \EndProof
\begin{Definition}
	Sia l'estensione di campo $\FieldExtension{\Field_2}{\Field_1}$. Si chiama \Define{nucleo normale} di $\Field_2$ rispetto a $\Field_1$ la pi\`u grande sottoestensione\footnote{L'unicit\`a, a meno di isomorfismo, segue direttamente dal teorema di unicit\`a dei campi di spezzamento.} $\Field_3$ di $\Field_2$ tale che $\FieldExtension{\Field_3}{\Field_1}$ \`e normale: la denoteremo $\NormalClosure{\Field_2}$.
\end{Definition}
\begin{Theorem}
	Sia l'estensione di campo $\FieldExtension{\Field_2}{\Field_1}$. Il nucleo normale di $\Field_2$ rispetto a $\Field_1$ \`e il campo $\Field_3 = \bigcap_{\phi \in I}{\phi(\Field_2)}$, dove $I$ \`e l'insieme di tutte le immersioni $\phi: \Field_2 \rightarrow \AlgebraicClosure{\Field_2}$ (non necessariamente omomorfismi relativi).
\end{Theorem}
\Proof Segue immediatamente dalla costruzione di $\Field_3$ che $\Field_2 \subseteq \Field_3$.
\par Proviamo che $\Field_3$ \`e normale. In effetti, se $\RelativeHomomorphism: \Field_3 \rightarrow \AlgebraicClosure{\Field_2}$ \`e un omomorfismo relativo a $\Field_2$ abbiamo $\RelativeHomomorphism(\Field_3) = \bigcap_{\phi \in I}{(\RelativeHomomorphism \circ \phi)(\Field_2)} \subseteq \Field_3$, poich\'e la composizione di immersioni \`e un'immersione.
\begin{Exercice}
	Sia l'estensione normale $\FieldExtension{\Field_2}{\Field_1}$ e sia $p \in \RingAdjunction{\Field_1}{x}$ tale che
	\begin{itemize}
		\item $p$ sia monico e irriducibile su $\Field_1$;
		\item $p = \prod_{i = 1}^n p_i$, con $n \in \mathbb{N}$, sia la fattorizzazione in irriducili di $p$ in $\RingAdjunction{\Field_2}{x}$.
	\end{itemize}
	Per ogni coppia $(i,j) \in [1,n]^2 \cap \mathbb{N}^2$, esiste un omomorfismo relativo $\RelativeHomomorphism: \Field_2 \rightarrow \AlgebraicClosure{\Field_2}$ tale che $\ImagePolynomial{\RelativeHomomorphism}{p_i} = p_j$.
\end{Exercice}
\Solution Siano, in $\AlgebraicClosure{\Field_2}$, $\alpha_i$ radice di $p_i$ e $\alpha_j$ radice di $p_j$. $\alpha_i$ e $\alpha_j$ sono radici coniugate del polinomio $p \in \RingAdjunction{\Field_1}{x}$, dunque esiste un omomorfismo relativo $\RelativeHomomorphism: \Field_2 \rightarrow \AlgebraicClosure{\Field_1}$ tale che $\RelativeHomomorphism(\alpha_i) = \alpha_j$. D'altra parte, poich\'e l'estensione $\FieldExtension{\Field_2}{\Field_1}$ \`e normale, $\RelativeHomomorphism$ \`e un isomorfismo sull'immagine $\Field_2$, cio\`e un automorfismo: $\ImagePolynomial{\RelativeHomomorphism}{p_i}$ \`e dunque un polinomio irriducibile che ha per radice $\alpha_j$; poich\'e inoltre $p_i$ \`e necessariamente monico, deve esserlo anche $\ImagePolynomial{\RelativeHomomorphism}{p_i}$, che coincide dunque con $p_j$. \EndProof

\subsection{Estensioni separabili e puramente inseparabili.}\label{EstensioniSeparabiliEPuramenteInseparabili}
\begin{Definition}
	Dato un campo $\Field$, un polinomio $p \in \NotZero{\RingAdjunction{\Field}{x}}$ si dice
	\begin{itemize}
		\item \Define{separabile}[separabile][polinomio] quando esso ammette tutte radici distinte in $\AlgebraicClosure{\Field}$;
		\item \Define{puramente inseparabile}[puramente inseparabile][polinomio] quando esso ammette un'unica radice in $\AlgebraicClosure{\Field}$;
	\end{itemize}
\end{Definition}
\begin{Theorem}
	\TheoremName{Criterio della derivata}[della derivata][criterio] Dato un campo $\Field$, un polinomio $p \in \RingAdjunction{\Field}{x}$ di grado strettamente positivo ha radici multiple nella chiusura algebrica $\AlgebraicClosure{\Field}$ se e solo se $\SpanIdeal{p,p'} \neq 1$, dove $p'$ \`e la derivata formale di $p$.
\end{Theorem}
\Proof Supponiamo $\alpha$ sia una radice di $p$: allora, per il teorema di Ruffini, $p = (x - \alpha)q$, con $q \in \RingAdjunction{\AlgebraicClosure{\Field}}{x}$ opportuno. Abbiamo dunque $p' = q + (x - \alpha)q'$ e $\alpha$ \`e radice doppia di $p$ se e solo se $x - \alpha$ divide $q$ e quindi $p'$ in $\RingAdjunction{\AlgebraicClosure{\Field}}{x}$. Dunque $p$ ammette una radice doppia se e solo se $\SpanIdeal{p,p'} \neq 1$ in $\RingAdjunction{\AlgebraicClosure{\Field}}{x}$.
\par D'altra parte, $\SpanIdeal{p,p'} = 1$ in $\RingAdjunction{\Field}{x}$ se e solo se esistono polinomi $a, b \in \RingAdjunction{\Field}{x}$ tali che $ap + bp' = 1$. Dunque, $\SpanIdeal{p,p'} \neq 1$ in $\RingAdjunction{\Field}{x}$ se e solo se $\SpanIdeal{p,p'} \neq 1$ in $\RingAdjunction{\AlgebraicClosure{\Field}}{x}$ se e solo se $p$ ammette una radice doppia. \EndProof
\begin{Corollary}
	Nelle ipotesi del teorema precedente, se $p$ \`e irriducibile allora ammette radici multiple se e solo se $p' = 0$.
\end{Corollary}
\Proof $p$ ammette radici multiple se e solo se $\SpanIdeal{p,p'} \neq 1$, ma gli $\SpanIdeal{p}$ \`e massimale, dunque $\SpanIdeal{p,p'} = \SpanIdeal{p}$ e $p' \in \SpanIdeal{p}$. Poich\'e la derivata formale di un polinomio ha grado strettamente minore del polinomio originale o \`e nulla, necessariamente $p' = 0$. \EndProof
\begin{Theorem}\label{th_separabilita}
	Dato un campo $\Field$, sia $p \in \RingAdjunction{\Field}{x}$ un polinomio irriducibile. Abbiamo
	\begin{itemize}
		\item se $\Characteristic{\Field} = 0$, allora $p$ \`e separabile;
		\item se $\Characteristic{\Field} = P$, con $P \in \mathbb{Z}$ primo, allora, posto $r = \max \lbrace k \in \mathbb{N} | (\exists q \in \RingAdjunction{\Field}{x})(p(x) = q(x^{P^k}))$ e $q \in \RingAdjunction{\Field}{x}$ tale che $p(x) = q(x^{P^r})$,
		\begin{itemize}
			\item $q$ \`e irriducibile e separabile;
			\item la moltiplicit\`a di ogni radice di $p$ su $\AlgebraicClosure{\Field}$ \`e $P^r$;
			\item le radici di $p$ su $\AlgebraicClosure{\Field}$ sono le radici $P^r$-esime delle radici di $q$.
		\end{itemize}
	\end{itemize}
\end{Theorem}
\Proof Il caso di caratteristica nulla si dimostra immediatamente: $p$ irriducibile \`e separabile se e solo se la sua derivata formale $p'$ \`e nonnulla, cio\`e se e solo se $p$ non \`e costante, ma nessun polinomio irriducibile \`e costante.
\par $q$ \`e irriducibile perch\'e lo \`e $p$ (una riduzione di $q$ indurrebbe un'analoga riduzione di $p$).
\par $q$ irriducibile \`e separabile se e solo se la sua derivata formale $q'$ \`e non nulla. Assumiamo $q' = 0$ e supponiamo $q = \sum_{k \in \mathbb{N}} a_kx^k$. Abbiamo $q' = \sum_{k \in \mathbb{N}} (k + 1)a_{k + 1}x^k = 0$, da cui, per ogni $k \in \mathbb{N}$, $(k + 1)a_{k + 1} = 0$. Quindi, per ogni $k \in \mathbb{N}$, $a_{k + 1} = 0$ o $P$ divide $k + 1$; pi\`u precisamente, abbiamo $a_k \neq 0$ solo se $P$ divide $k$. Di conseguenza, $q$ \`e un polinomio in $x^P$, contraddicendo la definizione di $q$.
\par Siano $(\alpha_i)_{i = 1}^n$, con $n \in \mathbb{N}$, tutte le radici di $q$. Abbiamo, in $\RingAdjunction{\AlgebraicClosure{\Field}}{x}$, $q = c \prod_{i = 1}^n (x - \alpha_i)$, dove $c = \LC{p}$; da cui $p = c \prod_{i = 1}^n \left (x^{P^r} - \alpha_i \right ) = c \prod_{i = 1}^n \left (x^{P^r} - \beta_i^{P^r} \right ) = c \prod_{i = 1}^n (x - \beta_i)^{P^r}$, dove per ogni $i = 1, ..., n$, $\beta_i$ \`e radice $P^r$-esima di $\alpha_i$. \EndProof
\begin{Corollary}
	Dato un campo $\Field$ di caratteristica $P > 0$, il polinomio irriducibile $p \in \RingAdjunction{\Field}{x}$ \`e separabile se e solo se, posto $r = \max \lbrace k \in \mathbb{N} | (\exists q \in \RingAdjunction{\Field}{x})(p(x) = q(x^{P^k}))$, abbiamo $r = 0$.
\end{Corollary}
\Proof Segue direttamente dal teorema precedente. \EndProof
\begin{Corollary}
	Dato un campo $\Field$, sia $p \in \RingAdjunction{\Field}{x}$ un polinomio irriducibile. Tutte le radici di $p$ hanno la stessa molteplicit\`a.
\end{Corollary}
\Proof Segue direttamente dal teorema precedente. \EndProof
\begin{Theorem}
	Sia $\Field$ un campo di caratteristica $\Characteristic{\Field} = \Prime > 0$ e $\Polynomial \in \RingAdjunction{\Field}{x}$ un polinomio irriducibile e monico. Abbiamo che $\Polynomial$ \`e puramente inseparabile se e solo se esistono $\beta \in \Field$ e $r \in \NotZero{\mathbb{N}}$ tali che $\Polynomial = x^{\Prime^r} - \beta$.
\end{Theorem}
\Proof Se $\Polynomial$ ammette un'unica radice $\alpha$ e $m \geq 2$ \`e la molteplicit\`a di questa radice, allora $\Polynomial = (x - \alpha)^m$. D'altra parte, abbiamo gi\`a dimostrato che $m$ \`e una potenza di $\Prime$, dunque esiste $r \geq 1$ tale che $\Polynomial = (x - \alpha)^{\Prime^r} = x^{\Prime^r} - \alpha^{\Prime^r}$. \EndProof
\begin{Definition}
	Sia $\FieldExtension{\Field_2}{\Field_1}$ un'estensione di campo. $\alpha \in \Field_2$ si dice
	\begin{itemize}
		\item \Define{separabile}[separabile][elemento] su $\Field_1$ quando $\MinimalPolynomial{\alpha}$ \`e separabile;
		\item \Define{puramente inseparabile}[puramente inseparabile][elemento] su $\Field_1$ quando $\MinimalPolynomial{\alpha}$ \`e puramente separabile.
	\end{itemize}
	Inoltre, se $\FieldExtension{\Field_2}{\Field_1}$ \`e un'estensione algebrica tale che
	\begin{itemize}
		\item tutti gli elementi sono separabili, allora diciamo che l'estensione \`e \Define{separabile}[separabile][estensione];
		\item tutti gli elementi sono puramente inseparabili, allora diciamo che l'estensione \`e \Define{puramente inseparabile}[inseparabile][estensione].
	\end{itemize}
\end{Definition}
\begin{Definition}
	Sia $\Field$ un campo. Se ogni estensione algebrica di $\Field$ \`e separabile, diciamo che $\Field$ \`e \Define{perfetto}[perfetto][campo].
\end{Definition}
\begin{Theorem}
	I campi di caratteristica $0$ e i campi di Galois sono perfetti.
\end{Theorem}
\Proof Abbiamo dimostrato che tutti i campi di caratteristica $0$ sono separabili e, poich\'e ogni estensione di un campo ne mantiene la caratteristica, ogni campo di caratteristica $0$ \`e perfetto.
\par Consideriamo ora invece un campo di Galois $\GaloisField{P^n}$ di caratteristica $P$, con $n \in \mathbb{N}$ e sia $p \in \RingAdjunction{\GaloisField{P^n}}{x}$. Per il teorema \ref{th_separabilita} $p$ \`e separabile se e solo se $r = \max \lbrace k \in \mathbb{N} | (\exists q \in \RingAdjunction{\Field}{x})(p(x) = q(x^{P^k}))$ \`e nullo. Proviamo quindi che necessariamente $r = 0$. In effetti, se $r > 1$, sfruttando il piccolo teorema di Fermat, applicabile in virt\`u della finitezza di $\GaloisField{P^n}$, abbiamo, per opportuni coefficienti $(a_k)_{k \in \mathbb{N}}$, $p = \sum_{k \in \mathbb{N}} a_kX^k = \sum_{k \in \mathbb{N}} a_{P^{kr}} x^{P^{kr}} = \sum_{k \in \mathbb{N}} a_{P^{kr}}^{P^{kr}} x^{P^{kr}} = \left ( \sum_{k \in \mathbb{N}} a_{P^{kr}}x \right )^{P^{kr}}$, in contraddizione con l'irriducibilit\`a di $p$. \EndProof
\begin{Theorem}
	Sia $\FieldExtension{\Field_2}{\Field_1}$ un'estensione puramente inseparabile: $\FieldExtension{\Field_2}{\Field_1}$ \`e anche normale.
\end{Theorem}
\Proof Sia $\Polynomial \in \FieldAdjunction{\Field_1}{x}$ un polinomio irriducibile che ammette una radice $\alpha \in \Field_2$: allora $\Polynomial$ \`e, a meno di moltiplicazione per una costante, il polinomio minimo di $\alpha$ su $\Field_1$, quindi \`e puramente separabile e si spezza in fattori lineari tutti uguali a $(x - \alpha)$. \EndProof
\begin{Theorem}
	Sia $\FieldExtension{\Field_2}{\Field_1}$ un'estensione simultaneamente separabile e puramente inseparabile. Allora $\Field_1 = \Field_2$.
\end{Theorem}
\Proof Immediata. \EndProof
\begin{Definition}
	Sia $\FieldExtension{\Field_2}{\Field_1}$ un'estensione di campo algebrica. Definiamo
	\begin{itemize}
		\item \Define{grado di separabilit\`a}[di separabilit\`a][grado] o \Define{grado separabile}[separabile][grado] di $\FieldExtension{\Field_2}{\Field_1}$ il numero (eventualmente anche infinito) di omomorfismi $\RelativeHomomorphism: \Field_2 \rightarrow \AlgebraicClosure{\Field_1}$ relativo ad una stessa immersione: lo denotiamo $\SeparableDegree{\Field_2}{\Field_1}$;
		\item \Define{grado di inseparabilit\`a}[di inseparabilit\`a][grado] o \Define{grado inseparabile}[inseparabile][grado] di $\FieldExtension{\Field_2}{\Field_1}$ il rapporto $\frac{\FieldDegree{\Field_2}{\Field_1}}{\SeparableDegree{\Field_2}{\Field_1}}$ se $\FieldExtension{\Field_2}{\Field_1}$ \`e finita (altrimenti il grado inseparabile non viene definito): lo denotatiamo $\UnseparableDegree{\Field_2}{\Field_1}$.
		\end{itemize}
\end{Definition}
\begin{Theorem}
	Sia $\Field$ un campo e $\alpha \in \AlgebraicClosure{\Field}$. Sia $m$ la comune moltiplicit\`a di tutte le radici di $\MinimalPolynomial{\alpha}$: abbiamo $m = \UnseparableDegree{\FieldAdjunction{\Field}{\alpha}}{\Field}$.
\end{Theorem}
\Proof Abbiamo gi\`a provato che $\SeparableDegree{\FieldAdjunction{\Field}{\alpha}}{\Field}$ \`e uguale al numero di radici distinte di $\MinimalPolynomial{\alpha}$ in $\AlgebraicClosure{\Field}$: il teorema di Ruffini garantisce che questo numero sia propio $\frac{\FieldDegree{\FieldAdjunction{\Field}{\alpha}}{\Field}}{m}$. \EndProof
\begin{Theorem}
	Sia la torre d'estensioni di campo $\Field_1 \subseteq \Field_2 \subseteq \Field_3 \subseteq \AlgebraicClosure{\Field_1}$, con $\FieldExtension{\Field_2}{\Field_1}$ e $\FieldExtension{\Field_3}{\Field_2}$ finite. Allora $\SeparableDegree{\Field_3}{\Field_1} = \SeparableDegree{\Field_3}{\Field_2} \cdot \SeparableDegree{\Field_2}{\Field_1}$ e $\UnseparableDegree{\Field_3}{\Field_1} = \UnseparableDegree{\Field_3}{\Field_2} \cdot \UnseparableDegree{\Field_2}{\Field_1}$.
\end{Theorem}
\Proof Siano $\RelativeHomomorphism_1: \Field_2 \rightarrow \AlgebraicClosure{\Field_1}$ e $\bar{\RelativeHomomorphism}_2: \Field_3 \rightarrow \AlgebraicClosure{\Field_2} = \AlgebraicClosure{\Field_1}$ omomorfismi relativi alle rispettive identit\`a. Sia $\RelativeHomomorphism_1^1: \Field_3 \rightarrow \AlgebraicClosure{\Field_2} = \AlgebraicClosure{\Field_1}$ un omomorfismo relativo a $\RelativeHomomorphism_1$.
\par $\RelativeHomomorphism_1^1 \circ \bar{\RelativeHomomorphism}_2: \Field_3 \rightarrow \AlgebraicClosure{\Field_1}$ \`e un omomorfismo relativo per $\Field_1$: infatti si verifica subito che $\Restricted{(\RelativeHomomorphism_1^1 \circ \bar{\RelativeHomomorphism}_2)}{\Field_1}$ \`e l'identit\`a di $\Field_1$.
\par Inoltre ogni coppia $(\RelativeHomomorphism_1,\bar{\RelativeHomomorphism}_2)$ porta a un omomorfiso relativo $\RelativeHomomorphism_1^1 \circ \bar{\RelativeHomomorphism}_2$ diverso. In effetti, sia $(\RelativeHomomorphism_1',\bar{\RelativeHomomorphism}_2')$ un 'altra coppia analoga, da cui analogamente costruiamo $\bar{\RelativeHomomorphism}_1^1 \circ \bar{\RelativeHomomorphism}_2'$. Supponiamo $\bar{\RelativeHomomorphism}_1^1 \circ \bar{\RelativeHomomorphism}_2' = \RelativeHomomorphism_1^1 \circ \bar{\RelativeHomomorphism}_2$, allora per ogni $a \in \Field_2$, $\bar{\RelativeHomomorphism}_1^1(a) = (\RelativeHomomorphism_1^1 \circ \bar{\RelativeHomomorphism}_2)(\bar{\RelativeHomomorphism}_2'^{-1}(a)) = (\RelativeHomomorphism_1^1 \circ \bar{\RelativeHomomorphism}_2)(a) = \RelativeHomomorphism_1^1(a)$: ne consegue che le restrizioni di $\bar{\RelativeHomomorphism}_1^1$ e $\RelativeHomomorphism_1^1$ a $\Field_2$, cio\`e $\RelativeHomomorphism_1'$ e $\RelativeHomomorphism_1$ sono uguali. Per l'iniettivit\`a di $\RelativeHomomorphism_1'$ e $\RelativeHomomorphism_1$, deve essere anche $\bar{\RelativeHomomorphism}_2' = \bar{\RelativeHomomorphism}_2$.
\par D'altra parte, sia $\RelativeHomomorphism_3: \Field_3 \rightarrow \AlgebraicClosure{\Field_1}$ un omomorfismo relativo. La sua restrizione $\RelativeHomomorphism_1$ a $\Field_2$ \`e un omomorfismo relativo e $\RelativeHomomorphism_3$ \`e un omomorfismo relativo a $\RelativeHomomorphism_1$: se ripetiamo la costruzione precedente ponendo $\RelativeHomomorphism_1^1 = \RelativeHomomorphism_3$ e $\bar{\RelativeHomomorphism}_2$ uguale all'inclusione canonica, allora vediamo che $\RelativeHomomorphism_3$ si pu\`o mettere nella stessa forma degli omomorfismi relativi precedentemente descritti, che sono in numero esattamente $\SeparableDegree{\Field_3}{\Field_2} \cdot \SeparableDegree{\Field_2}{\Field_1}$.
\par La moltiplicativit\`a dei gradi inseparabili segue direttametne da quella dei gradi separabili. \EndProof
\begin{Theorem}
	Sia l'estensione di campo $\FieldExtension{\Field_2}{\Field_1}$ finita, con $\Characteristic{\Field_1} > 0$. Il grado inseparabile di $\FieldExtension{\Field_2}{\Field_1}$ \`e una potenza di $\Characteristic{\Field_1}$.
\end{Theorem}
\Proof L'estensione $\FieldExtension{\Field_2}{\Field_1}$ \`e finita e dunque finitamente generata da una famiglia di elementi $(\alpha_k)_{k = 1}^n$ di $\Field_2$. Definiamo la torre di estensioni $(\Field^{(k)})_{k = 0}^n$ come $\Field_{(k)} = \begin{cases} \Field_1\text{ se }k = 0,\\ \FieldAdjunction{\Field^{(k - 1)}}{\alpha_k}\text{ altrimenti}.\end{cases}$
\par Per ogni $k = 0, ..., n - 1$, abbiamo che $\UnseparableDegree{\Field^{(k + 1)}}{\Field^{(k)}}$ \`e potenza di $\Characteristic{\Field_1}$, da cui $\UnseparableDegree{\Field_2}{\Field_1}$ \`e potenza di $\Characteristic{\Field_1}$. \EndProof
\begin{Theorem}
	\TheoremName{Teorema di caratterizzazione delle estensioni separabili finite}[di caratterizzazione delle estensioni separabili finite][teorema] Sia l'estensione di campo $\FieldExtension{\Field_2}{\Field_1}$ finita. Sono equivalenti le seguenti affermazioni:
	\begin{itemize}
		\item $\FieldExtension{\Field_2}{\Field_1}$ \`e separabile;
		\item esite una famiglia finita di elementi $(\alpha_k)_{k = 1}^n$ di $\Field_2$, con $n \in \mathbb{N}$, separabili su $\Field_1$ tali che $\Field_2 = \FieldAdjunction{\Field_1}{(\alpha_k)_{k = 1}^n}$;
		\item $\FieldDegree{\Field_2}{\Field_1} = \SeparableDegree{\Field_2}{\Field_1}$.
	\end{itemize}
\end{Theorem}
\Proof Se \`e vera la prima affermazione, segue direttamente dalle definizioni di finitezza e separabilit\`a di un'estensione di campo la validit\`a della seconda affermazione.
\par Se \`e vera la seconda affermazione, allora la terza affermazione segue considerando la torre d'estensioni $(\Field^{(k)})_{k = 0}^n$, con $\Field^{(k)} = \begin{cases} \Field_1\text{ se }k = 0,\\ \FieldAdjunction{\Field^{(k - 1)}}{\alpha_k}\text{ altrimenti}.\end{cases}$.
\par Assumiamo vera la terza affermazione e proviamo la prima. Possiamo assumere $\Characteristic{\Field_1} > 0$. Sia $\alpha \in \Field_2$ e consideriamo la torre d'estensioni $\Field_1 \subseteq \FieldAdjunction{\Field_1}{\alpha} \subseteq \Field_2$. Abbiamo $\FieldDegree{\Field_2}{\Field_1} = \FieldDegree{\FieldAdjunction{\Field_1}{\alpha}}{\Field_1}\FieldDegree{\Field_2}{\FieldAdjunction{\Field_1}{\alpha}}$ e $\SeparableDegree{\Field_2}{\Field_1} = \SeparableDegree{\FieldAdjunction{\Field_1}{\alpha}}{\Field_1}\SeparableDegree{\Field_2}{\FieldAdjunction{\Field_1}{\alpha}}$, da cui $\FieldDegree{\FieldAdjunction{\Field_1}{\alpha}}{\Field_1}\FieldDegree{\Field_2}{\FieldAdjunction{\Field_1}{\alpha}} = \SeparableDegree{\FieldAdjunction{\Field_1}{\alpha}}{\Field_1}\SeparableDegree{\Field_2}{\FieldAdjunction{\Field_1}{\alpha}}$. Poich\'e il grado inseparabile di un'estensione finita \`e una potenza della caratteristica dei campi, dall'ultima uguaglianza deduciamo che $\FieldDegree{\FieldAdjunction{\Field_1}{\alpha}}{\Field_1} = \SeparableDegree{\FieldAdjunction{\Field_1}{\alpha}}{\Field}$ e dunque $\alpha$ \`e separabile. \EndProof
\begin{Corollary}
	Sia $\Field$ un campo e $S \subseteq \AlgebraicClosure{\Field}$ non vuoto. $\FieldExtension{\FieldAdjunction{\Field}{S}}{\Field}$ \`e separabile se e solo se \`e separabile ogni elemento di $S$. Inoltre, se $\FieldExtension{\FieldAdjunction{\Field}{S}}{\Field}$ \`e separabile allora $\FieldDegree{\FieldAdjunction{\Field}{S}}{\Field} = \SeparableDegree{\FieldAdjunction{\Field}{S}}{\Field}$, incluso il caso in cui l'estensione \`e infinita.
\end{Corollary}
\Proof Se $\FieldExtension{\FieldAdjunction{\Field}{S}}{\Field}$ \`e separabile allora, per definizione, \`e separabile ogni elemento di $S$.
\par Supponiamo separabile ogni elemento di $S$ e sia $\alpha \in \Field{S}$. Allora $\alpha$ appartiene a qualche sottoestensione di $\FieldAdjunction{\Field}{S}$ generata da un sottoinsieme finito di $S$ e dunque finita essa stessa: per il teorema precedente, tale estensione \`e anche separabile, da cui la separabilit\`a di $\alpha$ su $\Field$.
\par Rimane solo da dimostrare che se $\FieldExtension{\FieldAdjunction{\Field}{S}}{\Field}$ \`e separabile, allora grado e grado separabile coincidono. Il caso finito \`e gi\`a stato provato dal teorema precedente: vediamo dunque il caso infinito. Se $\FieldExtension{\FieldAdjunction{\Field}{S}}{\Field}$ \`e infinita, allora per ogni $k \in \mathbb{N}$ esiste una sottoestensione finita $\Field_k$ tale che $\FieldDegree{\Field^k}{\Field} \geq k$, che \`e separabile. Per tanto, per ogni $k \in \mathbb{N}$, $\SeparableDegree{\FieldAdjunction{\Field}{S}}{\Field} > \SeparableDegree{\Field_k}{\Field} \geq k$. \EndProof
\begin{Theorem}	
	Se l'estensione di campo $\FieldExtension{\Field_2}{\Field_1}$ \`e separabile, allora \`e separabile anche $\FieldExtension{\NormalClosure{\Field_2}}{\Field_1}$,dove
	$\NormalClosure{\Field_2}$ \`e la chiusura normale di $\Field_2$ in $\AlgebraicClosure{\Field_2}$.
\end{Theorem}
\Proof Se $\FieldExtension{\Field_2}{\Field_1}$ \`e separabile allora \`e generata da una famiglia di elementi separabili $(\alpha_i)_{i = 1}^n$, $n \in \mathbb{N}$. Quindi $\NormalClosure{\Field_2}$ deve essere il campo di spezzamento dei polinomi minimi $(\MinimalPolynomial{\alpha_i})_{i =1}^n$ ed \`e separabilmente generata e quindi separabile. \EndProof
\begin{Exercice}
	Sia $\Field$ un campo e $S \subseteq \AlgebraicClosure{\Field}$ non vuoto. Dimostrare che $\FieldDegree{\FieldAdjunction{\Field}{S}}{\Field} = \SeparableDegree{\FieldAdjunction{\Field}{S}}{\Field}$ non implica necessarimente la separabilit\`a di $\FieldExtension{\FieldAdjunction{\Field}}{S}$.
\end{Exercice}
\Solution Per il teorema di caratterizzazione delle estensioni separabili finite, possiamo cercare un controesempio solamente tra le estensioni infinite.
\par Fissato un primo $p \in \mathbb{Z}$ con $p > 0$, poniamo $\Field = \FieldAdjunction{\GaloisField{p}}{t}$, dove $t$ \`e un nuovo simbolo qualsiasi, e $S = \AlgebraicClosure{\Field}$.
\par Consideriamo la sottoestensione $\FieldAdjunction{\FieldAdjunction{\GaloisField{p}}{t}}{\AlgebraicClosure{\GaloisField{p}}} = \FieldAdjunction{\AlgebraicClosure{\GaloisField{p}}}{t}$ di $\FieldExtension{\FieldAdjunction{\AlgebraicClosure{\GaloisField{p}}}{t}}{\FieldAdjunction{\GaloisField{p}}{t}}$. Essa \`e
\begin{itemize}
	\item separabile: perch\'e i campi di Galois sono perfetti e $\FieldAdjunction{\AlgebraicClosure{\GaloisField{p}}}{t} = \FieldAdjunction{\GaloisField{p}}{\AlgebraicClosure{\GaloisField{p}} \cup \lbrace t \rbrace}$;
	\item infinita: perch\'e contiene $\AlgebraicClosure{\GaloisField{p}}$;
	\item di grado separabile infinito: per il corollario precedente.
\end{itemize}
\par Ne consegue che l'estensione $\FieldExtension{\FieldAdjunction{\AlgebraicClosure{\GaloisField{p}}}{t}}{\FieldAdjunction{\GaloisField{p}}{t}}$, evidentemente infinita (tutte le potenze di $t$ sono indipendenti) ha grado separabile infinito. Tuttavia non \`e separabile: infatti il polinomio $x^p - t$ ammette una sola radice di molteplicit\`a $p$ in $\AlgebraicClosure{\FieldAdjunction{\GaloisField{p}}{t}}$. \EndSolution
\begin{Theorem}
	\TheoremName{Teorema di caratterizzazione delle estensioni puramente inseparabili}[di caratterizzazione delle estensioni puramente inseparabili][teorema] Sia l'estensione di campo $\FieldExtension{\Field_2}{\Field_1}$ con $\Characteristic{\Field_1} = \Prime > 0$. Sono equivalenti le seguenti affermazioni:
	\begin{itemize}
		\item $\FieldExtension{\Field_2}{\Field_1}$ \`e puramente inseparabile;
		\item esiste $S \subseteq \Field_2$ tale che tutti gli elementi di $S$ sono puramente inseparabili e $\Field_2 = \FieldAdjunction{\Field_1}{S}$;
		\item $\SeparableDegree{\Field_2}{\Field_1} = 1$;
		\item per ogni $\alpha \in \Field_2$ esiste $r \in \mathbb{N}$ tale che $\alpha^{\Prime^r} \in \Field_1$.
	\end{itemize}
\end{Theorem}
\Proof La prima affermazione implica la seconda perch\'e $\Field_2 = \FieldAdjunction{\Field_1}{\Field_2}$. La seconda affermazione implica la terza perch\'e un omomorfismo relativo $\RelativeHomomorphism: \Field_2 \rightarrow \AlgebraicClosure{\Field_1}$ non pu\`o che trasformare ogni generatore in $S$ in se stesso e dunque coincide con l'identit\`a. La terza affermazione implica la quarta perch\'e per ogni $\alpha \in \Field_2$ abbiamo $\SeparableDegree{\FieldAdjunction{\Field_1}{\alpha}}{\Field_1} | \SeparableDegree{\Field_2}{\Field_1} = 1$, da cui $\SeparableDegree{\FieldAdjunction{\Field_1}{\alpha}}{\Field_1}$ e quindi $\MinimalPolynomial{\alpha}$ \`e puramente inseparabile e dunque della forma $x^{\Prime^r} - \alpha^{\Prime^r}$ per $r \in \mathbb{N}$ opportuno. La quarta affermazione implica la prima perch\'e per ogni $\alpha$, fissato $r \in \mathbb{N}$ tale che $\alpha^{\Prime^r} \in \Field_1$, abbiamo $\MinimalPolynomial{\alpha} | x^{\Prime^r} - \alpha^{\Prime^r}$. \EndProof
\begin{Theorem}
	Sia $\Field_1 \subseteq \Field_2 \subseteq \Field_3$ una torre d'estensioni. Allora
	\begin{itemize}
		\item $\FieldExtension{\Field_3}{\Field_1}$ \`e separabile se e solo se lo sono $\FieldExtension{\Field_2}{\Field_1}$ e $\FieldExtension{\Field_3}{\Field_2}$. In particolare, la separabilit\`a si trasferisce alle torri;
		\item $\FieldExtension{\Field_3}{\Field_1}$ \`e puramente inseparabile se e solo se lo sono $\FieldExtension{\Field_2}{\Field_1}$ e $\FieldExtension{\Field_3}{\Field_2}$. In particolare, la pura inseparabilit\`a si trasferisce alle torri.
	\end{itemize}
\end{Theorem}
\Proof Nel caso l'estensione $\FieldExtension{\Field_3}{\Field_1}$ sia finita, il teorema segue dalla moltiplicativit\`a dei gradi separabili e dall'uguaglianza dei gradi e dei gradi separabili per estensioni separabili.
\par Vediamo il caso in cui $\FieldExtension{\Field_3}{\Field_1}$ sia infinita.
\par Supponiamo $\FieldExtension{\Field_3}{\Field_1}$ separabile. Sia $\alpha \in \Field_3$ e $\MinimalPolynomial{\alpha}[\Field_1]$ il polinomio minimo di $\alpha$ su $\Field_1$, necessariamente separabile, e sia $\MinimalPolynomial{\alpha}[\Field_2]$ il suo polinomio minimo su $\Field_2$: per definizione di polinomio minimo, $\MinimalPolynomial{\alpha}[\Field_2]$ divide $\MinimalPolynomial{\alpha}[\Field_1]$ e dunque \`e separabile. Dunque $\FieldExtension{\Field_3}{\Field_2}$ \`e separabile. La separabilit\`a di $\FieldExtension{\Field_2}{\Field_1}$ segue direttamente da $\Field_2 \subseteq \Field_3$.
\par Supponiamo ora $\FieldExtension{\Field_3}{\Field_2}$ e $\FieldExtension{\Field_2}{\Field_1}$ separabili. Sia $\alpha \in \Field_3$ e sia $\MinimalPolynomial{\alpha}[\Field_2]$ il polinomio minimo di $\alpha$ su $\Field_2$. Consideriamo l'insieme finito $S$ di tutti i coefficienti di $\MinimalPolynomial{\alpha}[\Field_2]$ e la torre d'estensioni $\Field_1 \subseteq \FieldAdjunction{\Field_1}{S} \subseteq \FieldAdjunction{\Field_1}{S \cup \lbrace \alpha \rbrace}$: si tratta di estensioni algebriche finite. Ora, $\FieldExtension{\FieldAdjunction{\Field_1}{S}}{\Field_1}$ \`e separabile perch\'e $\FieldAdjunction{\Field_1}{S} \subseteq \Field_2$ e $\FieldExtension{\FieldAdjunction{\Field_1}{S \cup \lbrace \alpha \rbrace}}{\FieldAdjunction{\Field_1}{S}}$ \`e anch'essa separabile perch\'e $\MinimalPolynomial{\alpha}[\Field_2] = \MinimalPolynomial{\alpha}[\FieldAdjunction{\Field_1}{S}]$.
\par Per quanto riguarda la pura inseparabilit\`a, tutto segue direttamente dal teorema precedente e da quanto appena dimostrato per la separabilit\`a. \EndProof
\begin{Theorem}
	La separabilit\`a  e la pura inseparabilit\`a si trasferiscono per traslazione. 
\end{Theorem}
\Proof Consideriamo le torri d'estensioni $\Field_1 \subseteq \Field_2 \subseteq \FieldComposition{\Field_2}{\Field_3}$ e $\Field_1 \subseteq \Field_3 \subseteq \FieldComposition{\Field_2}{\Field_3}$ e supponiamo $\FieldExtension{\Field_2}{\Field_1}$ separabile. La separabilit\`a di $\FieldExtension{\FieldComposition{\Field_2}{\Field_3}}{\Field_3}$ segue direttamente dal fatto che $\FieldComposition{\Field_2}{\Field_3} = \FieldExtension{\Field_3}{\Field_2}$ e dal fatto che se gli elementi di $\Field_2$ sono tutti separabili su $\Field_1$ allora lo sono anche su $\Field_3 \supseteq \Field_1$.
\par La dimostrazione della trasferibilit\`a della pura inseparabilit\`a \`e identica \textit{mutatis mutandis}. \EndProof
\begin{Theorem}
	La separabilit\`a e la pura inseparabilit\`a si trasferiscono per composizione.
\end{Theorem}
\Proof Segue direttamente dalla trasferibilit\`a lungo le torri e per traslazione. \EndProof
\begin{Theorem}
	Sia l'estensione di campo $\FieldExtension{\Field_2}{\Field_1}$ separabile. Allora \`e separabile anche $\FieldExtension{\NormalClosure{\Field_2}}{\Field_1}$.
\end{Theorem}
\Proof Sia $\alpha \in \NormalClosure{\Field_2}$. Poich\'e $\NormalClosure{\Field_2} = \prod_{\phi \in I} \phi(\Field_2)$, dove $I$ \`e l'insieme di tutte le immersioni di $\Field_2$ in $\AlgebraicClosure{\Field_2}$. Poich\'e le immersioni sono isomorfismi sull'immagine, ogni $\phi{\Field_2}$ \`e separabile. Abbiamo necessariamente $\alpha \in \prod_{\phi \in J} \phi{\Field_2}$ per un opportuno sottoinsieme finito $J \subseteq I$, da cui, poich\'e la separabilit\`a si trasferisce per composizione, $\alpha$ \`e separabile. \EndProof
\begin{Theorem}
	Data un'estensione di campo $\FieldExtension{\Field_2}{\Field_1}$, $\SeparableClosure{\Field_1}[\Field_2] = \lbrace \alpha \in \Field_2 | \alpha\text{ \`e separabile su }\Field_1 \rbrace$ \`e un campo, e dunque la massima sottoestensione separabile di $\FieldExtension{\Field_2}{\Field_1}$.
\end{Theorem}
\Proof Siano $\alpha, \beta \in \Field_2$ separabili: bisogna provare che $\alpha + \beta$, $\alpha\beta$, $- \beta$ e $\alpha^{-1}$ (nell'ipotesi che $\alpha \neq 0$) sono ancora separabili. Ma tutti questi elementi appartengono a $\FieldAdjunction{\Field_1}{\alpha,\beta}$, che \`e separabile in quanto separabilmente generato. \EndProof
\begin{Definition}
	Con le notazioni del teorema precedente, $\SeparableClosure{\Field_1}[\Field_2]$ si definisce \Define{chiusura separabile}[separabile][chiusura] di $\Field_1$ in $\Field_2$, omettendo $\Field_2$ quando l'estensione \`e evidente.
\end{Definition}
\begin{Theorem}
	Sia $\FieldExtension{\Field_2}{\Field_1}$ un'estensione algebrica. Sia $\Field$ un campo tale che
	\begin{itemize}
		\item $\Field_1 \subseteq \Field \subseteq \Field_2$ sia una torre d'estensioni;
		\item $\FieldExtension{\Field}{\Field_1}$ sia separabile;
		\item $\FieldExtension{\Field_2}{\Field}$ sia puramente inseparabile.
	\end{itemize}
	Allora $\Field = \SeparableClosure{\Field_1}$. Inoltre
	\begin{itemize}
		\item $\SeparableDegree{\Field_2}{\Field_1} = \FieldDegree{\SeparableClosure{\Field_1}}{\Field_1}$;
		\item se $\FieldExtension{\Field_2}{\Field_1}$ \`e finita, allora $\UnseparableDegree{\Field_2}{\Field_1} = \UnseparableDegree{\Field_2}{\SeparableClosure{\Field_1}}$;
		\item se $\FieldExtension{\Field_2}{\Field_1}$ \`e normale, allora $\FieldExtension{\SeparableClosure{\Field_1}{\Field}}$ \`e normale.
	\end{itemize}
\end{Theorem}
\Proof Poich\'`e $\FieldExtension{\Field}{\Field_1}$ \`e separabile, deve essere $\Field \subseteq \SeparableClosure{\Field_1}$ e $\FieldExtension{\SeparableClosure{\Field_1}}{\Field}$ \`e separabile. Ma poich\'e $\FieldExtension{\Field_2}{\Field}$ \`e puramente inseparabile, anche $\FieldExtension{\SeparableClosure{\Field_1}}{\Field}$ deve essere puramente inseparabile. Dunque $\Field = \SeparableClosure{\Field_1}$.
\par Abbiamo $\SeparableDegree{\Field_2}{\Field_1} = \SeparableDegree{\Field_2}{\SeparableClosure{\Field_1}}\SeparableDegree{\SeparableClosure{\Field_1}}{\Field_1} = \FieldDegree{\SeparableClosure{\Field_1}}{\Field_1}$ e, se $\FieldExtension{\Field_2}{\Field_1}$ \`e finita, $\UnseparableDegree{\Field_2}{\Field_1} = \UnseparableDegree{\Field_2}{\SeparableClosure{\Field_1}}\UnseparableDegree{\SeparableClosure{\Field_1}}{\Field_1} = \UnseparableDegree{\Field_2}{\SeparableClosure{\Field_1}}$.
\par Infine se $\alpha \in \SeparableClosure{\Field_1}$, allora $\MinimalPolynomial{\alpha}$ si spezza in fattori lineari in $\RingAdjunction{\Field_2}[x]$ i quali sono tutti distinti in virt\`u della separabilit\`a di $\alpha$. Inoltre tutte le radici coniugate ad $\alpha$ sono separabili e dunque appartengono a $\SeparableClosure{\Field_1}$. Quindi $\FieldExtension{\SeparableClosure{\Field_1}}{\Field_1}$ \`e separabile. \EndProof
\begin{Definition}
	Sia $\Field$ un campo di caratterstica $\Prime > 0$. Per ogni $n \in \mathbb{N}$ definiamo $\Field^{\Prime^{-n}}$ il campo di spezzamento della classe di tutti polinomi i polinomi della forma $x^{\Prime^{n}} - c \in \RingAdjunction{\Field}{x}$ con $c \in \Field$. Il campo $\PerfectClosure{\Field} = \bigcup_{n \in \mathbb{N}} \Field^{\Prime^{-n}}$ si dice \Define{chiusura perfetta}[perfetta]{chiusura} o \Define{chiusura puramente inseparabile}[puramente inseparabile][chiusura] di $\Field$.
\end{Definition}
\begin{Theorem}
	Sia $\Field_1$ un campo di caratteristica $\Prime > 0$ e sia $\FieldExtension{\Field_2}{\Field_1}$ una sua estensione algebrica. Allora
	\begin{itemize}
		\item la chiusura perfetta $\PerfectClosure{\Field_1}$ \`e perfetta;
		\item $\Field_1$ \`e perfetto se e solo se $\Field_1 = \PerfectClosure{\Field_1}$;
		\item $\FieldDegree{\PerfectClosure{\Field_1}}{\Field_1} = \begin{cases} 1\text{ se }\Field_1 = \PerfectClosure{\Field_1},\\\infty\text{ altrimenti;} \end{cases}$
		\item se $\Field_1$ \`e perfetto lo \`e anche $\Field_2$;
		\item se $\Field_2$ \`e perfetto \`e $\FieldExtension{\Field_2}{\Field_1}$ \`e finito allora \`e perfetto anche $\Field_1$.
	\end{itemize}
\end{Theorem}
\Proof Sia $\alpha \in \AlgebraicClosure{\PerfectClosure{\Field_1}}$. Sia $r \geq 0$ tale che $\MinimalPolynomial{\alpha} = b + \sum_{k \in \NotZero{\mathbb{N}}} a_k x^{k{\Prime^r}} \in \RingAdjunction{\Field_1}{x}$. Per definizione di chiusura perfetta, esiste $b \in \Field_1$ tale che $b = a_0^{\Prime^r}$. Abbiamo allora $\MinimalPolynomial{\alpha} = \left ( \sum_{k \in \mathbb{N}} a_k x^k \right )^{\Prime^r}$. Per l'irriducibilit\`a di $\MinimalPolynomial{\alpha}$, necessariamente $r = 0$ e quindi $\alpha$ \`e separabile. Dunque $\PerfectClosure{\Field_1}$ \`e perfetto.
\par \`E chiaro che se $\Field_1 = \PerfectClosure{\Field_1}$ allora $\Field_1$ \`e perfetto. D'altra parte assumiamo $\Field_1$ perfetto. Fissato $n \in \mathbb{N}$, consideriamo il polinomio $\Polynomial(x) = x^{\Prime^n} - c \in \RingAdjunction{\Field_1}{x}$ con $c \in \Field_1$. Sia $\alpha \in \AlgebraicClosure{\Field_1}$ una radice di $\Polynomial$: abbiamo $\Polynomial = x^{\Prime^n} - \alpha^{\Prime^n} = (x - \alpha)^{\Prime^n}$. Poich\'e $\Field_1$ \`e perfetto, il polinomio minimo di $\alpha$ su $\Field_1$ \`e separabile e poich\'e deve dividere $\Polynomial$ non pu\`o che essere $x - \alpha$: quindi $\alpha \in \Field_1$. Ne consegue che $\Field_1 = \PerfectClosure{\Field_1}$.
\par Se $\Field_1 = \PerfectClosure{\Field_1}$ segue direttamente dalle definizioni che $\FieldDegree{\PerfectClosure{\Field_1}}{\Field_1} = 1$. Supponiamo $\Field_1 \neq \PerfectClosure{\Field_1}$. Allora esistono $n \in \mathbb{N}$ e $c \in \Field_1$ tali che il polinomio $\Polynomial(x) = x^{\Prime^n} - c$ sia irriducibile: sia $\alpha \in \AlgebraicClosure{\Field_1}$ l'unica radice di $\Polynomial$. Allora, per ogni $k \in \mathbb{N}$, \`e irriducibile il polinomio $x^{\Prime^{n + k}} - c \in \RingAdjunction{\Field_1}{x}$, poich\'e se ammettesse una radice $\beta$, necessariamente unica, in $\Field_1$, avremmo allora $\alpha = \beta^{\Prime^k}$ in quanto $\Polynomial(\beta^{\Prime^k}) = \beta^{{\Prime^k}^{\Prime^n}} - c = \beta^{\Prime^{n + k}} - c = 0$ e, quindi, $\alpha \in \Field_1$ contro l'ipotesi di irriducibilit\`a di $\Polynomial$. Pertanto, per ogni $k \in \mathbb{N}$, abbiamo $\FieldDegree{\PerfectClosure{\Field_1}}{\Field_1} \geq \Prime^{n + k}$.
\par Supponiamo $\Field_1$ perfetto. Sia $\alpha \in \AlgebraicClosure{\Field_2}$. Per perfezione di $\Field_1$, il polinomio minimo $\MinimalPolynomial{\alpha}{\Field_1}$ ha tutte le radici distinte e a maggior ragione ha tutte le radici distinte $\MinimalPolynomial{\alpha}{\Field_2}$. Dunque $\alpha$ \`e separabile su $\Field_2$ e $\Field_2$ \`e perfetto.
\par Abbiamo $\PerfectClosure{\Field_1} \subseteq \FieldComposition{\PerfectClosure{\Field_1}}{\Field_2} \subseteq \PerfectClosure{\Field_2} = \Field_2$, dunque $\FieldDegree{\PerfectClosure{\Field_1}}{\Field_1} \leq \FieldDegree{\Field_2}{\Field_1}$ e necessariamente $\FieldDegree{\PerfectClosure{\Field_1}}{\Field_1} = 1$, da cui $\Field_1 = \PerfectClosure{\Field_1}$ \`e perfetto. \EndProof
\begin{Theorem}
	Sia $\Field$ un campo. Sia $\Field'$ un campo tale che
	\begin{itemize}
		\item $\Field \subseteq \Field' \subseteq \AlgebraicClosure{\Field}$ sia una torre d'estensioni;
		\item $\FieldExtension{\Field'}{\Field}$ sia puramente inseparabile;
		\item $\FieldExtension{\AlgebraicClosure{\Field}}{\Field'}$ sia separabile.
	\end{itemize}
	Allora $\Field = \PerfectClosure{\Field_1}$.
\end{Theorem}
\Proof Segue direttamente dalla definizione di chiusura perfetta che $\Field' \subseteq \PerfectClosure{\Field}$.
\par D'altra parte, se $\alpha \in \PerfectClosure{\Field} \SetMin \Field'$, allora $\alpha \in \AlgebraicClosure{\Field} \SetMin \Field'$ e $\alpha$ \`e simultaneamente separabile su $\Field'$ e puramente inseparabile su $\Field$ e dunque a maggior ragione su $\Field'$: quindi $\alpha \in \Field'$, ma questo \`e assurdo. Quindi $\Field' = \PerfectClosure{\Field}$. \EndProof
\begin{Theorem}
	Sia $\FieldExtension{\Field_2}{\Field_1}$ un'estensione normale. Se $\Field$ \`e un campo tale che
	\begin{itemize}
		\item $\Field_1 \subseteq \Field \subseteq \Field_2$ sia una torre d'estensioni;
		\item $\FieldExtension{\Field}{\Field_1}$ sia puramente inseparabile;
		\item $\FieldExtension{\Field_2}{\Field}$ sia separabile.
	\end{itemize}
	Allora $\Field = \Fixed{\Field_2}{\RelativeAutomorphisms{\Field_2}{\Field_1}}$.
\end{Theorem}
\Proof Per l'ipotesi di normalit\`a, ad ogni omomorfismo relativo $\RelativeHomomorphism: \Field_2 \rightarrow \AlgebraicClosure{\Field_1}$ corrisponde un automorfismo $\bar{\RelativeHomomorphism}: \Field_2 \rightarrow \Field_2$ per semplice sostituzione del codominio; viceversa ogni automorfismo di $\Field_2$ che lascia fisso $\Field_1$ diventa un omomorfismo relativo sostiuendone il codominio con $\AlgebraicClosure{\Field_1}$.
\par Abbiamo chiaramente $\Field_1 \subseteq \Field \subseteq \Field_2$. Inoltre
\begin{itemize}
	\item $\FieldExtension{\Field}{\Field_1}$ \`e puramente inseparabile perch\'e ogni omomorfismo relativo $\RelativeHomomorphism: \Field \rightarrow \AlgebraicClosure{\Field_1}$ pu\`o essere esteso ad un omomorfismo relativo da $\Field_2$ in $\AlgebraicClosure{\Field_1}$, il quale corrisponde ad un automorfismo in $\RelativeAutomorphisms{\Field_2}{\Field_1}$ che lascia fisso ogni punto di $\Field$;
	\item $\FieldExtension{\Field_2}{\Field}$ \`e separabile: consideriamo l'azione di $A$ su $\Field_2$ e sia $\alpha \in \Field_2$. L'orbita di $\alpha$ \`e naturalmente finita. Sia $\Polynomial = \prod_{\bar{\RelativeHomomorphism} \in A} (x - \bar{\RelativeHomomorphism}(\alpha))$: poich\'e per ogni $\tau \in A$ l'applicazione che ad ogni $\bar{\RelativeHomomorphism} \in A$ associa $\tau\bar{\RelativeHomomorphism}$ \`e iniettiva, allora applicando $\tau$ ai coefficienti di $\Polynomial$ -- o equivalentemente ai termini $\bar{\RelativeHomomorphism}(\alpha)$ -- si ottiene lo stesso polinomio $\Polynomial$, quindi $\Polynomial \in \RingAdjunction{\Field}{x}$. Inoltre, per costruzione $\Polynomial$ \`e un polinomio separabile avente $\alpha$ per radice, dunque $\alpha$ \`e separabile su $\Field$;
	\item $\Field$ \`e l'unico campo che verifica le tre propriet\`a del teorema: supponiamo infatti esista un altro campo $\Field'$ che verifica le stesse propriet\`a. Poich\'e $\Field'$ \`e puramente inseparabile, necessariamente $\Field' \subseteq \Field$ per definizione di $\Field$, da cui $\FieldExtension{\Field}{\Field'}$ \`e puramente inseparabile. Poich\'e inoltre $\FieldExtension{\Field_2}{\Field'}$ \`e separabile, allora \`e separabile anche $\FieldExtension{\Field}{\Field'}$. Quindi $\Field' = \Field$. \EndProof
\end{itemize}
\begin{Exercice}
	Trovare un'estensione $\FieldExtension{\Field_2}{\Field_1}$ per cui non esista nessun campo $\Field$ tale che
	\begin{itemize}
		\item $\Field_1 \subseteq \Field \subseteq \Field_2$ sia una torre d'estensioni;
		\item $\FieldExtension{\Field}{\Field_1}$ sia puramente inseparabile;
		\item $\FieldExtension{\Field_2}{\Field}$ sia separabile.
	\end{itemize}
\end{Exercice}
\Solution Poniamo $\Field_1 = \FieldAdjunction{\GaloisField{2}}{x,y}$ e $\Field_2 = \FieldAdjunction{\Field_1}{\alpha}$ dove $\alpha$ \`e una radice del polinomio $\Polynomial = t^4 + xt^2 + y \in \RingAdjunction{\RingAdjunction{\GaloisField{2}}{x,y}}{t}$. L'ideale $\SpanIdeal{x,y}$ \`e primo in $\RingAdjunction{\GaloisField{2}}{x,y}$ e ne consegue per il criterio di Eisenstein che $\Polynomial$ \`e primo. Dunque $\FieldDegree{\Field_2}{\Field_1} = 4$.
\par Inoltre $\Polynomial = \Polynomial_0(t^2)$, dove $\Polynomial_0(t) = t^2 + xt + y$, da cui $\UnseparableDegree{\Field_2}{\Field_1} = 2$, da cui $\FieldExtension{\SeparableClosure{\Field_1}}{\Field_1}$ \`e normale.
\par D'altra parte, abbiamo $\Polynomial = (t - \alpha)^2(t - \beta^2)$ per opportuno $\beta \in \AlgebraicClosure{\Field_1}$. Supponiamo $\FieldExtension{\Field_2}{\Field_1}$ normale. Allora $\alpha^2\beta^2 = y$ e $\alpha^2 + \beta^2 = (\alpha + \beta)^2 = x$, da cui $\alpha\beta = \sqrt{y}$ e $\alpha + \beta = \sqrt{x}$. Pertanto, poich\'e per normalit\`a $\beta \in \Field_2$, anche $\sqrt{x}, \sqrt{y} \in \Field_2$ e dunque $\UnseparableDegree{\FieldAdjunction{\Field_1}{\sqrt{x}}}{\Field_1} = \UnseparableDegree{\FieldAdjunction{\Field_1}{\sqrt{x},\sqrt{y}}}{\FieldAdjunction{\Field_1}{\sqrt{x}}} = 2$, da cui $\UnseparableDegree{\FieldAdjunction{\Field_1}{\sqrt{x},\sqrt{y}}}{\Field_1} = 4$, ma questo \`e assurdo, dunque $\FieldExtension{\Field_2}{\Field_1}$ non \`e normale. \EndProof
\\\Solution Supponiamo che il campo $\Field$ dell'enunciato esista. Abbiamo allora le seguenti torri d'estensioni:
\begin{itemize}
	\item $\Field \subseteq \FieldComposition{\Field}{\SeparableClosure{\Field_1}} \subseteq \Field_2$,
	\item $\Field_1 \subseteq \FieldComposition{\Field}{\SeparableClosure{\Field_1}} \subseteq \Field_2$;
\end{itemize}
ne deduciamo che $\FieldExtension{\Field_2}{\FieldComposition{\Field}{\SeparableClosure{\Field_1}}}$ \`e simultaneamente puramente inseparabile e separabile, e dunque che $\Field_2 = \FieldComposition{\Field}{\SeparableClosure{\Field_1}}$. Ora, poich\'e $\FieldExtension{\SeparableClosure{\Field_1}}{\Field_1}$ \`e normale, allora \`e normale anche $\FieldExtension{\Field_2}{\Field_1}$ perch\'e la normalit\`a si trasferisce per composizione, ma questo \`e assurdo. Dunque il campo $\Field$ non esiste. \EndSolution
\begin{Definition}
	Sia l'estensione di campi $\FieldExtension{\Field_2}{\Field_1}$. Si dice che $\Field_2$ \`e una \Define{chiusura algebrica separabile}[algebrica separabile][chiusura] di $\Field_1$ quando $\FieldExtension{\Field_2}{\Field_1}$ \`e un'estensione algebrica separabile tale che ogni polinomio separabile a coefficienti in $\Field_2$ ammetta una radice in $\Field_2$ (o, equivalentemente per il teorema di Ruffini, tutte le radici appertengono a $\Field_2$). Determinata una chiusura algebrica separabile di un campo $\Field$, essa si denota con $\SeparableAlgebraicClosure{\Field}$.
\end{Definition}
\begin{Definition}
	Sia l'estensione di campi $\FieldExtension{\Field_2}{\Field_1}$. Si dice che $\Field_2$ \`e una \Define{chiusura algebrica puramente inseparabile}[algebrica puramente inseparabile][chiusura] di $\Field_1$ quando $\FieldExtension{\Field_2}{\Field_1}$ \`e un'estensione algebrica puramente inseparabile tale che ogni polinomio puramente inseparabile a coefficienti in $\Field_2$ ha la sua unica radice in $\Field_2$. Determinata una chiusura algebrica separabile di un campo $\Field$, essa si denota con $\UnseparableAlgebraicClosure{\Field}$.
\end{Definition}
\begin{Theorem}
	Sia $\Field$ un campo. La chiusura algebrica separabile $\SeparableAlgebraicClosure{\Field}$ di $\Field$ coincide con la chiusura separabile di $\Field$ in $\AlgebraicClosure{\Field}$.
\end{Theorem}
\Proof Per semplicit\`a notazionale indichiamo con $\Field_1$ la chiusura algebrica separabile di $\Field$ e con $\Field_2$ la chiusura separabile di $\Field$ in $\AlgebraicClosure{\Field}$.
\par $\Field_1$ \`e separabile, quindi \`e un sottocampo di $\Field_2$.
\par D'altra parte, sia $\alpha \in \AlgebraicClosure{\Field}$ separabile su $\Field$: ci\`o significa che il polinomio minimo di $\alpha$ su $\Field$ \`e separabile. Ma $\MinimalPolynomial{\alpha}(x) \in \RingAdjunction{\Field_1}{x}$, dunque tutte le radici di $\MinimalPolynomial{\alpha}$ appartengono a $\Field_1$, inclusa $\alpha$. \EndProof
\begin{Theorem}
	Sia $\Field$ un campo. La chiusura algebrica puramente inseparabile $\UnseparableAlgebraicClosure{\Field}$ di $\Field$ coincide con la chiusura puramente inseparabile di $\Field$.
\end{Theorem}
\Proof Per semplicit\`a notazionale indichiamo con $\Field_1$ la chiusura algebrica puramente inseparabile di $\Field$ e con $\Field_2$ la chiusura perfetta di $\Field$.
\par $\Field_1$ \`e puramente inseparabile, quindi \`e un sottocampo di $\Field_2$.
\par D'altra parte, sia $\alpha \in \AlgebraicClosure{\Field}$ puramente inseparabile su $\Field$: ci\`o significa che il polinomio minimo di $\alpha$ su $\Field$ \`e puramente inseparabile. Ma $\MinimalPolynomial{\alpha}(x) \in \RingAdjunction{\Field_1}{x}$, dunque l'unica radice $\alpha$ di $\MinimalPolynomial{\alpha}$ appartiene a $\Field_1$. \EndProof
\begin{Theorem}
	Sia $\FieldExtension{\Field_2}{\Field_1}$ un'estensione di campi con $\Characteristic{\Field_1} = \Prime$. Sia $\alpha \in \Field_2$ algebrico. $\alpha$ \`e separabile su $\Field_1$ se e solo se $\FieldAdjunction{\Field_1}{\alpha} = \FieldAdjunction{\Field_1}{\alpha^\Prime}$.
\end{Theorem}
\Proof \`E immediato vedere che in ogni caso $\FieldAdjunction{\Field_1}{\alpha^\Prime} \subseteq \FieldAdjunction{\Field_1}{\alpha}$.
\par Supponiamo $\alpha$ separabile su $\Field_1$. Consideriamo il polinomio $\Polynomial(x) = x^\Prime - \alpha^\Prime = (x - \alpha)^\Prime \in \RingAdjunction{\FieldAdjunction{\Field_1}{\alpha_\Prime}}{x}$: esso ha $\alpha$ per unica radice; ne deduciamo che $\MinimalPolynomial{\alpha}[\FieldAdjunction{\Field_1}{\alpha^\Prime}]$ divide $\Polynomial$. Ma poich\'e $\alpha$ \`e separabile su $\Field_1$, lo \`e a maggior ragione su $\FieldAdjunction{\Field_1}{\alpha^\Prime}$ e dunque $\MinimalPolynomial{\alpha}[\FieldAdjunction{\Field_1}{\alpha^\Prime}] = x - \alpha$, da cui $\alpha \in \FieldAdjunction{\Field_1}{\alpha_\Prime}$.
\par Supponiamo ora viceversa che $\FieldAdjunction{\Field_1}{\alpha} = \FieldAdjunction{\Field_1}{\alpha^\Prime}$ e proviamo che $\alpha$ \`e separabile su $\Field_1$. Supponiamo $\alpha$ non separabile. Allora il polinomio minimo di $\alpha$ \`e della forma $\MinimalPolynomial{\alpha}(x) = \Polynomial(x^{\Prime^r})$, dove $\Polynomial \in \RingAdjunction{\Field_1}{x}$ \`e irriducibile e $r > 0$. Abbiamo allora $\FieldAdjunction{\Field_1}{\alpha} = \PolynomialDegree{\MinimalPolynomial{\alpha}} = \PolynomialDegree{\Polynomial} \Prime^r$, ma anche, poich\'e $\Polynomial(x^{\Prime^{r - 1}})$ \`e polinomio minimo di $\alpha^\Prime$, $\FieldAdjunction{\Field_1}{\alpha} = \FieldAdjunction{\Field_1}{\alpha^{\Prime^r}} = \PolynomialDegree{\Polynomial(x^{\Prime^{r - 1}})} = \PolynomialDegree{\Polynomial} \Prime^{r - 1}$. Pertanto $\alpha$ deve essere separabile. \EndProof
\begin{Theorem}
	Sia $\FieldExtension{\Field_2}{\Field_1}$ un'estensione di campi. Siano $\alpha, \beta \in \Field_2$ tali che
	\begin{itemize}
		\item $\alpha$ sia separabile;
		\item $\beta$ sia puramente inseparabile.
	\end{itemize}
	Abbiamo $\FieldAdjunction{\Field_1}{\alpha,\beta} = \FieldAdjunction{\Field_1}{\alpha + \beta} = \FieldAdjunction{\Field_1}{\alpha\beta}$.
\end{Theorem}
\Proof L'esistenza di un elemento puramente inseparabile implica $\Characteristic{\Field_1} = \Prime > 0$. L'esercizio precedente prova che, per ogni $r \in \mathbb{N}$, $\FieldAdjunction{\Field_1}{\alpha^{\Prime^r}} \in \FieldAdjunction{\Field_1}{\alpha}$. Inoltre, poich\'e $\beta$ \`e puramente inseparabile, esiste $k \in \mathbb{N}$ tale che $\beta^{\Prime^k} \in \Field_1$.
\par Abbiamo dunque $(\alpha\beta)^{\Prime^k} \in \FieldAdjunction{\Field_1}{\alpha\beta}$, da cui $\alpha^{\Prime^k} = \frac{(\alpha\beta)^{\Prime^k}}{\beta^{\Prime^k}} \in \FieldAdjunction{\Field_1}{\alpha\beta}$ e pertanto $\FieldAdjunction{\Field_1}{\alpha} = \FieldAdjunction{\Field_1}{\alpha^{\Prime^k}} \subseteq \FieldAdjunction{\Field_1}{\alpha\beta}$. Infine quindi $\FieldAdjunction{\Field_1}{\alpha\beta} = \FieldAdjunction{\Field_1}{\alpha, \beta}$.
\par La dimostrazione della relazione $\FieldAdjunction{\Field_1}{\alpha + \beta} = \FieldAdjunction{\Field_1}{\alpha, \beta}$ \`e del tutto analoga. \EndProof
\\\Proof Offriamo una seconda soluzione. Estendiamo ogni omomorfismo relativo $\RelativeHomomorphism: \FieldAdjunction{\Field_1}{\alpha} \rightarrow \AlgebraicClosure{\Field_1}$ ad un omomorfismo relativo $\tilde{\RelativeHomomorphism}: \FieldAdjunction{\Field_1}{\alpha,\beta} \rightarrow \AlgebraicClosure{\FieldAdjunction{\Field_1}{\beta}} = \AlgebraicClosure{\Field_1}$. Tutte le estensioni cos\`i ottenute sono distinte su $\FieldAdjunction{\Field_1}{\alpha + \beta}$, infatti, estendendo due omomorfismi relativi $\RelativeHomomorphism_1$ e $\RelativeHomomorphism_2$ abbiamo $\tilde{\RelativeHomomorphism_1}(\alpha + \beta) = \RelativeHomomorphism_1{\alpha} + \beta$ e $\tilde{\RelativeHomomorphism_2}(\alpha + \beta) = \RelativeHomomorphism_2{\alpha} + \beta$ e dunque $\tilde{\RelativeHomomorphism_1}(\alpha + \beta) = \tilde{\RelativeHomomorphism_2}(\alpha + \beta)$ implica $\RelativeHomomorphism_1(\alpha) = \RelativeHomomorphism_2(\alpha)$ e quindi $\RelativeHomomorphism_1 = \RelativeHomomorphism_2$.
\par Pertanto $\SeparableDegree{\FieldAdjunction{\Field_1}{\alpha + \beta}}{\Field_1} \geq \SeparableDegree{\FieldAdjunction{\Field_1}{\alpha + \beta}}{\FieldAdjunction{\Field_1}{\beta}} \geq \SeparableDegree{\FieldAdjunction{\Field_1}{\alpha}}{\Field_1} = \SeparableDegree{\FieldAdjunction{\Field_1}{\alpha, \beta}}{\Field_1} \geq \SeparableDegree{\FieldAdjunction{\Field_1}{\alpha + \beta}}{\Field_1}$ e quindi $\SeparableDegree{\FieldAdjunction{\Field_1}{\alpha + \beta}}{\Field_1} = \SeparableDegree{\FieldAdjunction{\Field_1}{\alpha}}{\Field_1}$, da cui $\alpha \in \FieldAdjunction{\Field_1}{\alpha + \beta}$ e quindi infine $\FieldAdjunction{\Field_1}{\alpha, \beta} = \FieldAdjunction{\Field_1}{\alpha + \beta}$. La dimostrazione dell'uguaglianza $\FieldAdjunction{\Field_1}{\alpha, \beta} = \FieldAdjunction{\Field_1}{\alpha\beta}$ \`e analoga. \EndProof
\\\Proof Una terza soluzione consiste nell'osservare che la separabilit\`a di $\FieldExtension{\FieldAdjunction{\Field_1}{\alpha}}{\Field_1}$ e la pura inseparabilit\`a di $\FieldExtension{\FieldAdjunction{\Field_1}{\beta}}{\Field_1}$ si trasferiscono entrambe per traslazione sia a $\FieldExtension{\FieldAdjunction{\Field_1}{\alpha,\beta}}{\FieldAdjunction{\Field_1}{\alpha\beta}}$ che a $\FieldExtension{\FieldAdjunction{\Field_1}{\alpha,\beta}}{\FieldAdjunction{\Field_1}{\alpha + \beta}}$, da cui la tesi. \EndSolution
\begin{Theorem}
	Sia $\Field$ un campo tale che $\Characteristic{\Field} = \Prime > 0$. L'endomorfismo di Frobenius $\Frobenius: \Field \rightarrow \Field$ \`e un automorfismo se e solo se $\Field$ \`e perfetto.
\end{Theorem}
\Proof \`E sufficiente provare che $\Frobenius$ \`e surgettivo se e solo se $\Field$ \`e perfetto.
\par Supponiamo $\Frobenius$ surgettivo. Sia $\alpha \in \AlgebraicClosure{\Field}$. Sia $r \geq 0$ tale che $\MinimalPolynomial{\alpha} = b + \sum_{k \in \NotZero{\mathbb{N}}} a_k x^{k{\Prime^r}} \in \RingAdjunction{\Field}{x}$. Per la surgettivit\`a di $\Frobenius$, esiste $b \in \Field$ tale che $b = a_0^{\Prime^r}$. Abbiamo allora $\MinimalPolynomial{\alpha} = \left ( \sum_{k \in \mathbb{N}} a_k x^k \right )^{\Prime^r}$. Per l'irriducibilit\`a di $\MinimalPolynomial{\alpha}$, necessariamente $r = 0$ e quindi $\alpha$ \`e separabile.
\par Suppponiamo ora invece che $\Field$ sia perfetto. Sia $\alpha \in \Field$ e sia $\Polynomial = x^\Prime - \alpha \in \RingAdjunction{\Field}{x}$. Sia $\beta \in \AlgebraicClosure{\Field}$ radice di $\Polynomial$. Abbiamo allora $\Polynomial = x^\Prime - \beta^\Prime = (x - \beta)^\Prime$: ne deduciamo, per la perfezione di $\Field$, che $\MinimalPolynomial{\beta} = x - \beta$ e dunque che $\beta \in \Field$. \EndProof

\subsection{Campi di Galois.}\label{CampiDiGalois}
\begin{Theorem}
	Se $\Field$ \`e un campo finito,
	allora $\Characteristic{\Field} > 0$.
\end{Theorem}
\Proof Immediata per definizione di caratteristica. \EndProof
\begin{Theorem}
	Per ogni $a \in \AlgebraicClosure{\GaloisField{\Prime^n}}$,
	con $\Prime \in \mathbb{Z}$ primo e $n \in \mathbb{N}$,
	e per ogni $k \in \mathbb{Z}$, $a$ \`e l'unica radice
	$\Prime^k$-esima di $a^{\Prime^k}$.
\end{Theorem}
\Proof $\GaloisField{\Prime^n} \subseteq
\AlgebraicClosure{\GaloisField{\Prime^n}}$, dunque
$\Characteristic{\AlgebraicClosure{\GaloisField{\Prime^n}}} =
\Characteristic{\GaloisField{\Prime^n}} = \Prime$.
Per definizione una radice $\Prime^k$-esima di $a$ \`e una soluzione
dell'equazione $x^{\Prime^k} - a^{\Prime^k} = 0$, la quale equivale a
$(x - a)^{\Prime^k} = 0$. \EndProof
\begin{Theorem}
	Se $\Field$ \`e un campo,
	e $\Group$ un sottogruppo finito del
	gruppo moltiplicativo $\Invertible{\Field}$,
	allora $\Group$ \`e ciclico.
\end{Theorem}
\Proof
Sia $D$ l'insieme dei divisori di $\GroupOrder{\Group}$ e,
per ogni $d \in D$, sia $\psi(d)$ il numero di elementi di ordine $d$ in
$\Group$.
Fissiamo $d \in D$ e supponiamo $\psi(d) \neq 0$: scegliamo
$a \in \Group$ tale che $\Order{a} = d$.
Il sottogruppo $\SpanGroup{a}$ ha ordine $d$,
dunque tutti i suoi elementi sono radici del polinomio
$\Polynomial(x) = x^d - 1 \in \FieldAdjunction{\Field}{x}$;
non solo: $\Polynomial$ ammette al pi\`u $d$ radici distinte,
dunque $\SpanGroup{a}$ \`e l'insieme di tutte e sole le radici di
$\Polynomial$.
\par
Pertanto, per ogni $d \in D$,
o $\psi(d) = 0$,
o $\psi(d)$ \`e uguale al numero di generatori di
$\Isomorphic{\SpanGroup{a}}{\Quotient{\mathbb{Z}}{\SpanIdeal{d}}}$,
che \`e $\EulerFunction(d)$, dove $\EulerFunction$ \`e la funzione di
Eulero.
Ora, abbiamo sia $\sum_{d \in D} \psi(d) = \GroupOrder{\Group}$ che
$\sum_{d \in D} \EulerFunction(d) = \GroupOrder{\Group}$, dunque,
per ogni $d \in D$, deve essere $\psi(d) = \EulerFunction(d)$:
ne consegue che
$\Group$ \`e ciclico.
\EndProof
\begin{Corollary}
	Sia $\Field$ un campo finito.
	Il gruppo moltiplicativo $\Invertible{\Field}$ \`e
	un grupp ciclico.
\end{Corollary}
\Proof
Immediata dal teorema precedente.
\EndProof
\begin{Corollary}
	Fissati un campo $\Field$, $\Prime \in \mathbb{Z}$ primo e
	$n \in \NotZero{\mathbb{N}}$,
	l'insieme $\Roots_n$ di tutte le radici del polinomio
	$\Polynomial(x) = x^n - 1$ costituisce un sottogruppo
	moltiplicativo ciclico di $\Invertible{\Field}$.
\end{Corollary}
\Proof
$\Roots_n$ \`e un sottogruppo moltiplicativo di $\Invertible{\Field}$
perch\'e se $a, b \in \Field$ sono tali che $a^n = b^n = 1$, allora
$(ab^{-1})^n = a^nb^{-n} = 1$. La ciclicit\`a segue dalla finitezza di
$\Roots_n$ e dal teorema precedente.
\EndProof
\begin{Definition}\label{radici_primitive_def}
	Con le notazioni del corollario precedente,
	chiamiamo
	\Define{radici primitive $n$-esime dell'unit\`a}[primitiva dell'unit\`a][radice]
	i generatori di $\Roots_n$.
	Chiamiamo inoltre
	\Define{$n$-esimo polinomio ciclotomico}[ciclotomico][polinomio]
	il polinomio
	$\CyclotomicPolynomial{n}(x) =
	\prod_{\PrimitiveRoot \in \PrimitiveRoots[n]}
	(x - \PrimitiveRoot)$.
\end{Definition}
\begin{Theorem}
	Se $\Field$ \`e un campo finito,
	allora la sua cardinalit\`a \`e potenza di un primo
	$\Prime \in \mathbb{Z}$.
\end{Theorem}
\Proof
Supponiamo che $\Field$ sia un campo di caratteristica
$\Prime \in \mathbb{Z}$ prima.
Allora $\Field$ \`e un'estensione di
$\Quotient{\mathbb{Z}}{\SpanIdeal{\Prime}}$ e dunque uno spazio vettoriale
finito su $\Quotient{\mathbb{Z}}{\SpanIdeal{\Prime}}$: la cardinalit\`a di
$\Field$ \`e dunque necessarimamente una potenza di $P$.
\EndProof
\begin{Theorem}
	Se $\Prime \in \mathbb{Z}$ \`e primo e $n \in \NotZero{\mathbb{N}}$, allora esiste uno e un solo campo di cardinalit\`a $\Prime^n$.
\end{Theorem}
\Proof Consideriamo il polinomio $\Polynomial = x^{\Prime^n} - x \in \RingAdjunction{\Quotient{\mathbb{Z}}{\SpanIdeal{p}}}{x}$. La derivata formale di $\Polynomial$ \`e $\Polynomial' = \Prime^nx^{\Prime^n - 1} - 1 = -1$ dunque, per il criterio della derivata, $\Polynomial$ ha tutte le $\Prime^n$ radici distinte in $\Closure{\RingAdjunction{\Quotient{\mathbb{Z}}{\SpanIdeal{p}}}{x}}$: sia $\mathcal{R}$ l'insieme di tutte queste radici.
\par Siano $\alpha, \beta \in \mathcal{R}$: abbiamo $\Polynomial(\alpha + \beta) = (\alpha + \beta)^{\Prime^n} - \alpha - \beta = \alpha^{\Prime^n} + \beta^{\Prime^n} - \alpha - \beta = 0$, $\Polynomial(\alpha\beta) = (\alpha\beta)^{\Prime^n} - \alpha\beta = \alpha^{\Prime^n}\beta^{\Prime^n} - \alpha\beta = \alpha\beta - \alpha\beta = 0$ e $\Polynomial(\alpha^{-1}) = (\alpha^{-1})^{\Prime^n} - \alpha^{-1} = (\alpha^{\Prime^n})^{-1} - \alpha^{-1} = \alpha^{-1} - \alpha^{-1} = 0$. Ne deduciamo che $\mathcal{R}$ \`e un campo, di esattamente $\Prime^n$ elementi.
\par Assumiamo ora che $\Field$ sia una campo di cardinalit\`a $\Prime^n$. Poich\'e il gruppo moltiplicativo $\Invertible{\Field}$ \`e ciclico, ogni $\alpha \in \Invertible{\Field}$ \`e radice dell'equazione $x^{\Prime^n - 1} = 1$ e quindi ogni $\alpha \in \Field$ \`e radice del polinomio $\Polynomial = x^{\Prime^n} - x$: ma allora $\Field$ \`e necessariamente la pi\`u piccola estensione di $\Quotient{\mathbb{Z}}{\SpanIdeal{\Prime}}$ che contiene tutte le radici di $\Polynomial$, vale a dire il campo di spezzamento di $\Polynomial$. L'unicit\`a del campo finito segue dunque dal teorema di unicit\`a del campo di spezzamento. \EndProof
\begin{Definition}
	Chiamiamo \Define{campo di Galois}[di Galois][campo] un campo finito: denotiamo $\GaloisField{n}$ dove $n \in \mathbb{N}$ \`e la cardinalit\`a del campo, l'unico campo di Galois di cardinalit\`a $n$.
\end{Definition}
\begin{Theorem}
	Sia $\Prime \in \mathbb{Z}$ primo.
	L'insieme $\bigcup_{n \in \NotZero{\mathbb{N}}}
	\GaloisField{\Prime^n}$
	\begin{itemize}
		\item \`e dotato di una ed una sola struttura di campo
		compatibile con le strutture di campo dei singoli
		$(\GaloisField{\Prime^n})_{n \in \NotZero{\mathbb{N}}}$;
		\item \`e chiusura algebrica di $\GaloisField{\Prime^n}$
		per ogni $n \in \mathbb{N}$;
		\item per ogni $n \in \mathbb{N}$, abbiamo
		$\FieldDegree{\AlgebraicClosure{\GaloisField{\Prime^n}}}
		{\GaloisField{\Prime^n}} = \infty$.
	\end{itemize}
\end{Theorem}
\Proof Per il primo punto, \`e sufficiente osservare che $\bigcup_{n \in \NotZero{\mathbb{N}}} \GaloisField{\Prime^n} = \FieldAdjunction{\GaloisField{\Prime}}{S}$, dove $S$ \`e l'insieme di tutte le radici primitive $\Prime^n$-esime dell'unit\`a al variare di $n \in \NotZero{\mathbb{N}}$: si tratta dunque di un'estensione di $\GaloisField{\Prime^n}$ per ogni $n \in \NotZero{\mathbb{N}}$.
\par $\bigcup_{n \in \NotZero{\mathbb{N}}} \GaloisField{\Prime^n}$ \`e un'estensione algebrica di qualunque $\GaloisField{\Prime^n}$ con $n \in \mathbb{N}$ perch\'e algebricamente generata. Proviamo che \`e anche algebricamente chiusa: sia $\Polynomial = \sum_{k \in \mathbb{N}} a_kx^k \in \RingAdjunction{\bigcup_{n \in \NotZero{\mathbb{N}}} \GaloisField{\Prime^n}}{x}$. Poich\'e la famiglia $(a_k)_{k \in \mathbb{N}}$ ha supporto finito, esiste $K \in \mathbb{N}$ tale che $\Polynomial \in \RingAdjunction{\GaloisField{\Prime^K}}{x}$ (per esempio si pu\`o prendere $K$ uguale al prodotto di una famiglia di elementi $n_k$ tali che per ogni $k \in \mathbb{N}$ si abbia $a_k \in \GaloisField{\Prime^{n_k}}$). Esiste allora, nella chiusura algebrica di $\RingAdjunction{\GaloisField{\Prime^K}}{x}$ una radice $\alpha$ di $\Polynomial$: ma allora $\FieldAdjunction{\GaloisField{\Prime^K}}{\alpha}$ \`e un campo finito e dunque $\alpha \in \bigcup_{n \in \NotZero{\mathbb{N}}} \GaloisField{\Prime^n}$.
\par Per quanto riguarda il grado di $\FieldExtension{\AlgebraicClosure{\GaloisField{\Prime^n}}}{\GaloisField{\Prime^n}}$, qualunque sia $n \in \NotZero{\mathbb{N}}$, per il teorema di moltiplicativit\`a dei gradi basta provare che $\AlgebraicClosure{\GaloisField{\Prime^n}}$ \`e un campo infinito, ma ci\`o segue subito dal fatto che se fosse finito, allora esisterebbe un campo di Galois di cardinalit\`a maggiore e stessa caratteristica di $\AlgebraicClosure{\GaloisField{\Prime^n}}$, in contraddizione col fatto che il sostegno di $\AlgebraicClosure{\GaloisField{\Prime^n}}$ \`e l'insieme di tutti i campi di Galois di caratteristica $\Prime$. \EndProof

\begin{Theorem}
	Sia $\FieldExtension{\Field_2}{\Field_1}$ un'estensione di campo finita. $\FieldExtension{\Field_2}{\Field_1}$ \`e semplice se e solo se ammette solo un un numero finito di sottoestensioni.
\end{Theorem}
\Proof Supponiamo che $\Field_1$ sia un campo di Galois.
Allora anche $\Field_2$ \`e un campo di Galois e l'estensione
$\FieldExtension{\Field_2}{\Field_1}$ risulta generata da un qualsiasi generatore del gruppo degli elementi invertibili di $\Field_2$.
D'altra parte, un'eventuale sottoestensione di $\FieldExtension{\Field_2}{\Field_1}$ deve avere grado che divide $\FieldDegree{\Field_2}{\Field_1}$ e, poich\'e sono in numero finito sia i divisori di $\FieldDegree{\Field_2}{\Field_1}$ sia i polinomi di $\FieldAdjunction{\Field_1}{x}$ di grado fissato, sono in numero finito anche le sottoestensioni stesse.
\par Supponiamo dunque che $\Field_1$ sia un campo infinito.
\par Sia $\alpha \in \AlgebraicClosure{\Field_1}$ tale che $\Field_2 = \FieldAdjunction{\Field_1}{\alpha}$. Sia $f$ l'applicazione che ad ogni campo $\Field$ tale che $\Field_1 \subseteq \Field \subseteq \Field_2$ associa il polinomio $\MinimalPolynomial{\alpha}[\Field]$, polinomio minimo di $\alpha$ in $\Field$. Per ogni campo $\Field$, $\MinimalPolynomial{\alpha}[\Field]$ deve dividere $\MinimalPolynomial{\alpha}[\Field_1]$, quindi il codominio di $f$ \`e finito.
\par Fissiamo il campo $\Field$ tale che $\Field_1 \subseteq \Field \subseteq \Field_2$. Sia $S$ l'insieme di tutti i coefficienti non nulli del polinomio $f(\Field)$. Abbiamo chiaramente $\FieldAdjunction{\Field_1}{S} \subseteq \Field$. Inoltre $f(\FieldAdjunction{\Field_1}{S}) = f(\Field)$, da cui $\FieldDegree{\Field_2}{\FieldAdjunction{\Field_1{S}}} = \FieldDegree{\Field_2}{\Field}$, da cui $\Field = \FieldAdjunction{\Field_1}{S}$. Pertanto se, $\Field^{(1)}, \Field^{(2)} \in \Dom{f}$ sono tali che $f(\Field^{(1)}) = f(\Field^{(2)}$, allora $\Field^{(1)} = \Field^{(2)}$, vale a dire che $f$ \`e un'applicazione iniettiva: per la finitizza del codominio di $f$, si deduce che \`e finito anche il dominio.
\par Supponiamo invece che $\FieldExtension{\Field_2}{\Field_1}$ ammetta solo un numero finito di sottoestensioni e supponiamo $S \subseteq \Field_2$ tale che $\Field_2 = \FieldAdjunction{\Field_1}{S}$. Procediamo per induzione su $\Cardinality{S}$.
\par Se $\Cardinality{S} = 1$ la tesi \`e immediata. Supponiamo la tesi provata per $\Cardinality{S} = n > 1$. Allora scegliamo $\alpha, \beta \in S$ con $\alpha \neq \beta$. Consideriamo i campi $\FieldAdjunction{\Field_1}{\alpha + t \beta}$ al variare di $t \in \Field_1$; essi
\begin{itemize}
	\item verificano tutti la condizione $\Field_1 \subseteq \FieldAdjunction{\Field_1}{\alpha + t \beta} \subseteq \Field_2$ e dunque sono in numero finito;
	\item sono tutte definite da generatori diversi perch\'e se, per $t_1, t_2 \in \Field_1$, abbiamo $\alpha + t_1 \beta = \alpha + t_2 \beta$, allora abbiamo necessariamente $t_1 = t_2$.
\end{itemize}
\par Dunque, per opportuni $t_1, t_2 \in \Field_1$ con $t_1 \neq t_2$ abbiamo $\FieldAdjunction{\Field_1}{\alpha + t_1 \beta} = \FieldAdjunction{\Field_1}{\alpha + t_2 \beta}$. Ne deduciamo $\beta = (t_1 - t_2)^{-1} (\alpha + t_1 \beta - \alpha - t_2 \beta) \in \FieldAdjunction{\Field_1}{\alpha + t_1 \beta}$ e $\alpha = \alpha + t_1 \beta - t_1 \beta \in \FieldAdjunction{\Field_1}{\alpha + t_1 \beta}$. Dunque $\Field_2 = \FieldAdjunction{\Field_1}{\lbrace \alpha + t_1 \beta} \cup S \SetMin \lbrace{\alpha, \beta \rbrace}$ e la tesi segue per ipotesi induttiva. \EndProof
\subsection{Teorema dell'elemento primitivo.}\label{TeoremaDellElementoPrimitivo}
\begin{Theorem}
	\TheoremName{Teorema dell'elemento primitivo}[dell'elemento primitivo][teorema] Sia $\FieldExtension{\Field_2}{\Field_1}$ un'estensione di campo finita e separabile: l'estensione \`e anche semplice.
\end{Theorem}
\Proof Supponiamo prima che $\Field_2$ sia un campo di Galois. Allora $\Invertible{\Field_2}$ \`e ciclico e quindi, scelto $\zeta \in \Invertible{\Field_2}$, abbiamo $\Field_2 = \FieldAdjunction{\Field_1}{\zeta}$.
\par Supponiamo dunque che $\Field_2$ sia un campo infinito.
\par Sia $S$ un insieme di generatori dell'estensione $\FieldAdjunction{\Field_2}{\Field_1}$: l'insieme $S$ \`e chiaramente finito. Procediamo per induzione su $\Cardinality{S}$.
\par Se $\Cardinality{S} = 1$ la tesi \`e immediata. Supponiamo la tesi provata per $\Cardinality{S} = n > 1$. Allora possiamo scegliere $\beta \in S$ e porre $T = S \SetMin \lbrace \beta \rbrace$. Abbiamo dunque la torre d'estensioni $\Field_1 \subseteq \FieldAdjunction{\Field_1}{T} \subseteq \FieldAdjunction{\Field_1}{S} = \Field_2$, dove, per ipotesi induttiva, $\FieldAdjunction{\Field_1}{T} = \FieldAdjunction{\Field_1}{\alpha}$, per opportuno $\alpha \in \FieldAdjunction{\Field_1}{T}$. Dunque $\Field_2 = \FieldAdjunction{\Field_1}{\alpha,\beta}$.
\par Poich\'e l'estensione \`e separabile, abbiamo $\SeparableDegree{\Field_2}{\Field_1} = \FieldDegree{\Field_2}{\Field_1}$, da cui esiste solo un numero finito di omomorfismi relativi $(\RelativeHomomorphism_k)_{k = 1}^K$, con $K \in \mathbb{N}$, da $\Field_2$ in $\AlgebraicClosure{\Field_1}$.
\par Consideriamo il polinomio $\Polynomial(x) = \prod_{k < j} (\RelativeHomomorphism_k(\alpha) + \RelativeHomomorphism_k(\beta)x - \RelativeHomomorphism_j(\alpha) - \RelativeHomomorphism_j(\beta)x)$. Se fosse $\Polynomial = 0$, allora esisterebbero $k, j \in \mathbb{N}$ con $k \neq j$ tali che $\RelativeHomomorphism_k(\alpha) + \RelativeHomomorphism_k(\beta)x = \RelativeHomomorphism_j(\alpha) + \RelativeHomomorphism_j(\beta)x$, da cui dedurremmo $\RelativeHomomorphism_k(\alpha) = \RelativeHomomorphism_j(\alpha)$ e $\RelativeHomomorphism_k(\beta) = \RelativeHomomorphism_j(\beta)$ e, poich\'e $\alpha$ e $\beta$ generano l'estensione $\FieldExtension{\Field_2}{\Field_1}$, $\RelativeHomomorphism_k = \RelativeHomomorphism_j$, ma questo \`e assurdo. Dunque $\Polynomial \neq 0$ e, poich\'e $\Polynomial$ ammette solo un numero finito di radici mentre $\Field_2$ \`e un campo infinito, esiste $t \in \Field_2$ tale che $\Polynomial(t) \neq 0$.
\par Poniamo $\gamma = \alpha + \beta t$ e proviamo che $\Field_2 = \FieldAdjunction{\Field_1}{\gamma}$. Per costruzione di $t$, abbiamo che $\RelativeHomomorphism_k(\gamma) \neq \RelativeHomomorphism_j(\gamma)$ per ogni $k \neq j$, dunque $\FieldDegree{\FieldAdjunction{\Field_1}{\gamma}}{\Field_1} \geq \FieldDegree{\Field_2}{\Field_1}$, e poich\'e $\FieldAdjunction{\Field_1}{\gamma} \subseteq \Field_2$, vale $\FieldDegree{\FieldAdjunction{\Field_1}{\gamma}}{\Field_1} = \FieldDegree{\Field_2}{\Field_1}$ e dunque $\Field_2 = \FieldAdjunction{\Field_1}{\gamma}$. \EndProof
\begin{Exercice}
	Sia $\Prime \in \mathbb{Z}$. Provare che l'estensione $\FieldExtension{\FieldAdjunction{\GaloisField{\Prime}}{X,Y}}{\FieldAdjunction{\GaloisField{\Prime}}{X^\Prime,Y^\Prime}}$ ammette infinite sottoestensioni.
\end{Exercice}
\Solution Consideriamo il polinomio $\Polynomial = t^\Prime - X^\Prime \in \RingAdjunction{\FieldAdjunction{\GaloisField{\Prime}}{X^\Prime,Y^\Prime}}{t}$: esso si annulla in $X$, che \`e la sua unica radice in $\AlgebraicClosure{\FieldAdjunction{\GaloisField{\Prime}}{X^\Prime,Y^\Prime}}$, quindi abbiamo $\SeparableDegree{\FieldAdjunction{\GaloisField{\Prime}}{X^\Prime,Y^\Prime,X}}{\FieldAdjunction{\GaloisField{\Prime}}{X^\Prime,Y^\Prime}} = 1$, da cui $\FieldDegree{\FieldAdjunction{\GaloisField{\Prime}}{X^\Prime,Y^\Prime,X}}{\FieldAdjunction{\GaloisField{\Prime}}{X^\Prime,Y^\Prime}}$ \`e una potenza di $\Prime$, necessariamente $\Prime$ stesso dato che $\EuclideanDegree \Polynomial = \Prime$.
\par Analogamente si prova che $\SeparableDegree{\FieldAdjunction{\GaloisField{\Prime}}{X^\Prime,Y^\Prime,X,Y}}{\FieldAdjunction{\GaloisField{\Prime}}{X^\Prime,Y^\Prime,X}} = 1$ e $\FieldDegree{\FieldAdjunction{\GaloisField{\Prime}}{X^\Prime,Y^\Prime,X,Y}}{\FieldAdjunction{\GaloisField{\Prime}}{X^\Prime,Y^\Prime,X}} = p$. Dunque $\SeparableDegree{\FieldAdjunction{\GaloisField{\Prime}}{X^\Prime,Y^\Prime,X,Y}}{\FieldAdjunction{\GaloisField{\Prime}}{X^\Prime,Y^\Prime}} = 1$ e $\FieldDegree{\FieldAdjunction{\GaloisField{\Prime}}{X^\Prime,Y^\Prime,X,Y}}{\FieldAdjunction{\GaloisField{\Prime}}{X^\Prime,Y^\Prime}} = \Prime^2$.
\par Consideriamo l'endomorfismo di Frobenius $\Frobenius: \FieldAdjunction{\GaloisField{\Prime}}{X,Y} \rightarrow \FieldAdjunction{\GaloisField{\Prime}}{X^\Prime,Y^\Prime}$. Abbiamo $\FieldAdjunction{\GaloisField{\Prime}}{X^\Prime,Y^\Prime} = \Image{\Frobenius}$. Dunque, per ogni $\alpha \in \FieldAdjunction{\GaloisField{\Prime}}{X^\Prime,Y^\Prime} \SetMin \FieldAdjunction{\GaloisField{\Prime}}{X,Y}$, il polinomio $t^\Prime - \alpha^\prime$ appartiene a $\RingAdjunction{\FieldAdjunction{\GaloisField{\Prime}}{X^\Prime,Y^\Prime}}{t}$ e quindi $\FieldDegree{\FieldAdjunction{\GaloisField{\Prime}{X^\Prime,Y^\Prime, \alpha}}}{\FieldAdjunction{\GaloisField{\Prime}{X^\Prime,Y^\Prime}}} = \Prime \neq \Prime^2$. Quindi $\FieldExtension{\FieldAdjunction{\GaloisField{\Prime}}{X,Y}}{\FieldAdjunction{\GaloisField{\Prime}}{X^\Prime,Y^\Prime}}$ non \`e un'estensione semplice e ammette infinite sottoestensioni. \EndProof
\begin{Theorem}
	Siano $\Field$ un campo, $S \subseteq \AlgebraicClosure{\Field}$ un insieme finito di elementi separabili su $\Field$ e $\beta \in \AlgebraicClosure{\Field}$. L'estensione $\FieldAdjunction{\Field}{S \cup \lbrace \beta \rbrace}$ \`e semplice.
\end{Theorem}
\Proof Se $\Field$ \`e finito allora $\FieldAdjunction{\Field}{S \cup \lbrace \beta \rbrace}$ \`e finito e semplice per la ciclicit\`a del suo gruppo moltiplicativo. Supponiamo dunque $\Field$ infinito.
\par Per il teorema dell'elemento primitivo, esiste $\alpha \in S$ tale che $\FieldAdjunction{\Field}{S} = \FieldAdjunction{\Field}{\alpha}$. Dunque $\FieldAdjunction{\Field}{S \cup \lbrace \beta \rbrace} = \FieldAdjunction{\Field}{\alpha,\beta}$.
\par Abbiamo $\SeparableDegree{\FieldAdjunction{\Field}{\alpha,\beta}}{\Field} < \FieldDegree{\FieldAdjunction{\Field}{\alpha,\beta}}{\Field}$, da cui esiste solo un numero finito di omomorfismi relativi $(\RelativeHomomorphism_k)_{k = 1}^K$, con $K \in \mathbb{N}$, da $\FieldAdjunction{\Field}{\alpha,\beta}$ in $\AlgebraicClosure{\Field}$.
\par Consideriamo il polinomio $\Polynomial(x) = \prod_{k < j} (\RelativeHomomorphism_k(\alpha) + \RelativeHomomorphism_k(\beta)x - \RelativeHomomorphism_j(\alpha) - \RelativeHomomorphism_j(\beta)x)$. Se fosse $\Polynomial = 0$, allora esisterebbero $k, j \in \mathbb{N}$ con $k \neq j$ tali che $\RelativeHomomorphism_k(\alpha) + \RelativeHomomorphism_k(\beta)x = \RelativeHomomorphism_j(\alpha) + \RelativeHomomorphism_j(\beta)x$, da cui dedurremmo $\RelativeHomomorphism_k(\alpha) = \RelativeHomomorphism_j(\alpha)$ e $\RelativeHomomorphism_k(\beta) = \RelativeHomomorphism_j(\beta)$ e, poich\'e $\alpha$ e $\beta$ generano l'estensione $\FieldExtension{\FieldAdjunction{\alpha,\beta}}{\Field}$, $\RelativeHomomorphism_k = \RelativeHomomorphism_j$, ma questo \`e assurdo. Dunque $\Polynomial \neq 0$ e, poich\'e $\Polynomial$ ammette solo un numero finito di radici mentre $\FieldAdjunction{\Field}{\alpha,\beta}$ \`e un campo infinito, esiste $t \in \FieldAdjunction{\Field}{\alpha,\beta}$ tale che $\Polynomial(t) \neq 0$.
\par Poniamo $\gamma = \alpha + \beta t$ e proviamo che $\FieldAdjunction{\Field}{\alpha,\beta} = \FieldAdjunction{\Field}{\gamma}$. Per costruzione di $t$, abbiamo che $\RelativeHomomorphism_k(\gamma) \neq \RelativeHomomorphism_j(\gamma)$ per ogni $k \neq j$, dunque $\SeparableDegree{\FieldAdjunction{\Field}{\gamma}}{\Field_1} \geq \SeparableDegree{\FieldAdjunction{\Field}{\alpha,\beta}}{\Field}$, e poich\'e $\FieldAdjunction{\Field}{\gamma} \subseteq \FieldAdjunction{\Field}{\alpha,\beta}$, vale $\SeparableDegree{\FieldAdjunction{\Field}{\gamma}}{\Field} = \SeparableDegree{\FieldAdjunction{\Field}{\alpha,\beta}}{\Field}$ e dunque $\FieldAdjunction{\Field}{\alpha,\beta} = \FieldAdjunction{\Field}{\gamma}$. \EndProof
\begin{Lemma}
	Sia $\Field$ un campo. Abbiamo $\AlgebraicClosure{\Field} = \FieldComposition{\SeparableAlgebraicClosure{\Field}\PerfectClosure{\Field}}$.
\end{Lemma}
\Proof Abbiamo le seguenti torri d'estensioni:
\begin{itemize}
	\item $\Field \subseteq \SeparableAlgebraicClosure{\Field} \subseteq \FieldComposition{\SeparableAlgebraicClosure{\Field}\PerfectClosure{\Field}} \subseteq \AlgebraicClosure{\Field}$;
	\item $\Field \subseteq \PerfectClosure{\Field} \subseteq \FieldComposition{\SeparableAlgebraicClosure{\Field}\PerfectClosure{\Field}} \subseteq \AlgebraicClosure{\Field}$.
\end{itemize}
\par Per trasferimento per traslazione abbiamo che
\begin{itemize}
	\item $\FieldExtension{\FieldComposition{\SeparableAlgebraicClosure{\Field}}{\PerfectClosure{\Field}}}{\SeparableAlgebraicClosure{\Field}}$ \`e puramente inseparabile;
	\item $\FieldExtension{\FieldComposition{\SeparableAlgebraicClosure{\Field}}{\PerfectClosure{\Field}}}{\PerfectClosure{\Field}}$ \`e separabile;
\end{itemize}
e dunque che $\FieldExtension{\AlgebraicClosure{\Field}}{\FieldComposition{\SeparableAlgebraicClosure{\Field}}{\PerfectClosure{\Field}}}$ \`e simultaneamente separabile e puramente inseparabile, dunque banale. \EndProof
\begin{Theorem}
	Sia $\FieldExtension{\Field_2}{\Field_1}$ una estensione di campi algebrica tale che ogni polinomio in $\RingAdjunction{\Field_1}{x}$ ammetta una radice in $\Field_2$. Allora $\Field_2$ \`e chiusura algebrica di $\Field_1$.
\end{Theorem}
\Proof Per il lemma precedente, basta provare che se $\alpha \in \AlgebraicClosure{\Field}$ \`e separabile o puramente inseparabile, allora $\alpha \in \Field_2$.
\par Se $\alpha$ \`e puramente inseparabile, allora $\alpha \in \Field_2$ perch\'e unica radice del suo polinomio minimo su $\Field_1$.
\par Supponiamo invece $\alpha$ separabile e sia $\Field$ il campo di spezzamento di $\MinimalPolynomial{\alpha}[\Field_1]$ su $\Field_1$. $\FieldExtension{\Field}{\Field_1}$ \`e un'estensione finita e separabile, dunque semplice per il teorema dell'elemento primitivo. Sia $\beta \in \Field$ tale che $\Field = \FieldAdjunction{\Field_1}{\beta}$. Il polinomio minimo $\MinimalPolynomial{\beta}[\Field_1]$ ammette una radice $\gamma \in \Field_2$. D'altra parte, per ragioni di grado delle estensioni, $\FieldAdjunction{\Field_1}{\gamma} = \Field$, dunque $\alpha \in \Field = \FieldAdjunction{\Field_1}{\gamma} \subseteq \Field_2$. \EndProof


\section{Teoria di Galois.}\label{Galois}
\subsection{Corrispondenza di Galois.}\label{CorrispondenzaDiGalois}
\begin{Definition}
	Sia $\FieldExtension{\Field_2}{\Field_1}$ un'estensione di campi normale e separabile: diciamo che $\FieldExtension{\Field_2}{\Field_1}$ \`e un'\Define{estensione di Galois}[di Galois][estensione]. Chiamiamo inoltre \Define{gruppo di Galois}[di Galois][gruppo] di $\FieldExtension{\Field_2}{\Field_1}$, denotato $\GaloisGroup{\Field_2}{\Field_1}$, il gruppo $\RelativeAutomorphisms{\Field_2}{\Field_1}$, i cui elementi chiamiamo \Define{automorfismi di Galois}[di Galois][automorfismo].
\end{Definition}
\begin{Definition}
	Dato un campo $\Field$, chiamiamo $\GaloisGroup{\AlgebraicClosure{\Field}}{\Field}$ \Define{gruppo assoluto di Galois}[assoluto di Galois][gruppo] di $\Field$, o anche pi\`u brevemente \Define{gruppo assoluto}[assoluto][gruppo] di Galois.
\end{Definition}
\begin{Theorem}
	Sia $\FieldExtension{\Field_2}{\Field_1}$ un'estensione di Galois. Abbiamo $\GroupOrder{\GaloisGroup{\Field_2}{\Field_1}} = \FieldDegree{\Field_2}{\Field_1}$.
\end{Theorem}
\Proof Sia $\RelativeHomomorphism: \Field_2 \rightarrow \AlgebraicClosure{\Field_1}$ un omomorfismo relativo. Per normalit\`a di $\FieldExtension{\Field_2}{\Field_1}$, $\Image{\RelativeHomomorphism} = \Field_2$. \`E dunque possibile associare ad ogni omomorfismo relativo di $\FieldExtension{\Field_2}{\Field_1}$ un automorfismo in $\GaloisGroup{\Field_2}{\Field_1}$ modificandone il codominio e mantenendo lo stesso grafo: la corrispondenza \`e chiaramente biunivoca.
\par Dunque $\GroupOrder{\GaloisGroup{\Field_2}{\Field_1}} = \SeparableDegree{\Field_2}{\Field_1}$ e, poich\'e $\FieldExtension{\Field_2}{\Field_1}$ \`e separabile, $\GroupOrder{\GaloisGroup{\Field_2}{\Field_1}} = \FieldDegree{\Field_2}{\Field_1}$. \EndProof
\begin{Theorem}
	Sia $\Field_1 \subseteq \Field_2 \subseteq \Field_3$ una torre d'estensioni di campo. Se $\FieldExtension{\Field_3}{\Field_1}$ \`e di Galois, allora
	\begin{itemize}
		\item $\FieldExtension{\Field_3}{\Field_2}$ \`e di Galois;
		\item $\FieldExtension{\Field_2}{\Field_1}$ \`e di Galois se e solo se $\FieldExtension{\Field_2}{\Field_1}$ \`e normale;
	\end{itemize}
	Inoltre l'essere estensione di Galois \`e una propriet\`a che si trasferisce per traslazione e per composizione, ma non lungo le torri.
\end{Theorem}
\Proof Tutt le affermazioni seguono direttamente dagli analoghi teoremi sulla normalit\`a e la separabilit\`a. \EndProof
\begin{Theorem}
	Sia $\Field_1 \subseteq \Field_2 \subseteq \Field_3$ una torre d'estensioni di campo con $\FieldExtension{\Field_3}{\Field_1}$ di Galois. Abbiamo
	\begin{itemize}
		\item $\GaloisGroup{\Field_3}{\Field_2} \IsSubgroup \GaloisGroup{\Field_3}{\Field_1}$;
		\item se $\GaloisAutomorphism \in \GaloisGroup{\Field_3}{\Field_1}$, allora $\GaloisAutomorphism\GaloisGroup{\Field_3}{\Field_2}\GaloisAutomorphism^{-1} = \GaloisGroup{\Field_3}{\GaloisAutomorphism{\Field_2}}$;
		\item se $\FieldExtension{\Field_2}{\Field_1}$ \`e di Galois, allora l'applicazione $f: \GaloisGroup{\Field_3}{\Field_1} \rightarrow \GaloisGroup{\Field_2}{\Field_1}$ che a $\GaloisAutomorphism \in \GaloisGroup{\Field_3}{\Field_1}$ associa $\Restricted{\GaloisAutomorphism}{\Field_2} \in \GaloisGroup{\Field_2}{\Field_1}$ \`e surgettiva.
	\end{itemize}
\end{Theorem}
\Proof Un automorfismo di $\Field_3$ che lascia fisso $\Field_2$ a maggior ragione lascia fisso $\Field_1$, dunque $\GaloisGroup{\Field_3}{\Field_2} \IsSubgroup \GaloisGroup{\Field_3}{\Field_1}$.
\par Per ogni automorfismo di Galois $\GaloisAutomorphism \in \GaloisGroup{\Field_3}{\Field_1}$ e ogni $\GaloisAutomorphism_0 \in \GaloisGroup{\Field_3}{\Field_2}$ abbiamo
\begin{itemize}
	\item se $x \in \GaloisAutomorphism(\Field_2)$, allora $(\GaloisAutomorphism \circ \GaloisAutomorphism_0 \circ \GaloisAutomorphism^{-1})(x) = x$;
	\item se $x \in \Field_2$ \`e tale che $(\GaloisAutomorphism \circ \GaloisAutomorphism_0 \circ \GaloisAutomorphism^{-1})(x) = x$, allora $\GaloisAutomorphism_0(\GaloisAutomorphism^{-1}(x)) = \GaloisAutomorphism^{-1}(x)$, da cui $x \in \GaloisAutomorphism{\Field_2}$.
\end{itemize}
Quindi $\GaloisAutomorphism\GaloisGroup{\Field_3}{\Field_2}\GaloisAutomorphism^{-1} = \GaloisGroup{\Field_3}{\GaloisAutomorphism{\Field_2}}$.
\par Ogni automorfismo in $\GaloisGroup{\Field_2}{\Field_1}$ ha lo stesso grafo di un'immersione da $\Field_2$ in $\AlgebraicClosure{\Field_1}$, la quale si pu\`o estendere ad omomorfismo relativo da $\Field_3$ in $\AlgebraicClosure{\Field_1}$, la cui immagine \`e $\Field_3$ per la normalit\`a di $\FieldExtension{\Field_3}{\Field_1}$. \EndProof
\begin{Lemma}
	\TheoremName{Lemma di Artin}[di Artin][lemma]\footnote{Questa denominazione non \`e molto comune.} Siano $\Field$ un campo e $\Group \IsSubgroup \Automorphisms{\Field}$. Abbiamo
	\begin{itemize}
		\item se $\Group$ \`e finito, allora $\FieldExtension{\Field}{\Fixed{\Field}{\Group}}$ \`e un'estensione di Galois finita con $\Group$ per gruppo di Galois;
		\item se $\Group$ non \`e finito e $\FieldExtension{\Field}{\Fixed{\Field}{\Group}}$ \`e algebrica, allora $\FieldExtension{\Field}{\Fixed{\Field}{\Group}}$ \`e un estensione di Galois infinita e $\Group \IsSubgroup \GaloisGroup{\Field}{\Fixed{\Field}{\Group}}$.
	\end{itemize}
\end{Lemma}
\Proof Proviamo che $\FieldExtension{\Field}{\Fixed{\Field}{\Group}}$ \`e
\begin{itemize}
	\item algebrica anche nel caso finito;
	\item di Galois in ogni caso.
\end{itemize}
\par Sia $\alpha \in \Field$. La famiglia $(\GaloisAutomorphism(\alpha))_{\GaloisAutomorphism \in \Group}$ \`e finita e ne indichiamo con $S$ il supporto. Definiamo il polinomio $\Polynomial_{\alpha}(x) = \prod_{s \in S}(x - s)$: esso appartiene a $\RingAdjunction{\Fixed{\Field}{\Group}}{x}$, \`e separabile e ha $\alpha$ per radice, dunque $\alpha$ \`e algebrico e separabile su $\Fixed{\Field}{\Group}$. Inoltre, $\Field$ \`e il campo di spezzamento della famiglia di polinomi $(\Polynomial_\alpha)_{\alpha \in L}$, dunque $\FieldExtension{\Field}{\Fixed{\Field}{\Group}}$ \`e normale. Ne consegue che $\FieldExtension{\Field}{\Fixed{\Field}{\Group}}$ \`e un'estensione di Galois.
\par Per costruzione di $\Fixed{\Field}{\Group}$, abbiamo $\Group \IsSubgroup \GaloisGroup{\Field}{\Fixed{\Field}{\Group}}$.
\par Se $\Group$ non \`e finito, allora a maggior ragione non \`e finito $\GaloisGroup{\Field}{\Fixed{\Field}{\Group}}$ e dunque non lo \`e l'estensione $\FieldExtension{\Field}{\Fixed{\Field}{\Group}}$.
\par Supponiamo $\Group$ finito. Per ogni $\alpha \in \Field$, il polinomio $\Polynomial_\alpha$ prova che $\FieldDegree{\FieldAdjunction{\Fixed{\Field}{\Group}}{\alpha}}{\Fixed{\Field}{\Group}} \leq \GroupOrder{\Group}$: sia $\alpha_0 \in \Field$ tale che $\FieldDegree{\FieldAdjunction{\Fixed{\Field}{\Group}}{\alpha_0}}{\Fixed{\Field}{\Group}} \leq \GroupOrder{\Group} = \max_{\alpha \in \Field} \FieldDegree{\FieldAdjunction{\Fixed{\Field}{\Group}}{\alpha}}{\Fixed{\Field}{\Group}} \leq \GroupOrder{\Group}$.
\par Supponiamo esista $\beta \in \Field$ tale che $\beta \notin \FieldAdjunction{\Fixed{\Field}{\Group}}{\alpha_0}$. Allora, per il teorema dell'elemento primitivo, $\FieldAdjunction{\Fixed{\Field}{\Group}}{\alpha_0,\beta}$, che ha necessarimante grado maggiore di $\FieldAdjunction{\Fixed{\Field}{\Group}}{\alpha_0}$ \`e semplice, ma ci\`o contraddice la definizione di $\alpha_0$. Dunque $\Field = \FieldAdjunction{\Fixed{\Field}{\Group}}{\alpha_0}$.
\par Abbiamo infine $\GroupOrder{\Group} \leq \GroupOrder{\GaloisGroup{\Field}{\Fixed{\Field}{\Group}}} = \FieldDegree{\Field}{\Fixed{\Field}{\Group}} = \FieldDegree{\FieldAdjunction{\Fixed{\Field}{\Group}}{\alpha_0}}{\Fixed{\Field}{\Group}} \leq \GroupOrder{\Group}$. Dunque $\Group$ e $\GaloisGroup{\Field}{\Fixed{\Field}{\Group}}$ hanno lo stesso ordine e pertanto coincidono. \EndProof
\begin{Corollary}
	Sia $\FieldExtension{\Field_2}{\Field_1}$ un'estensione normale. Abbiamo
	\begin{itemize}
		\item $\FieldExtension{\Field_2}{\Fixed{\Field_2}{\RelativeAutomorphisms{\Field_2}{\Field_1}}}$ \`e un'estensione di Galois con gruppo di Galois $\RelativeAutomorphisms{\Field_2}{\Field_1}$;
		\item se $\FieldExtension{\Field_2}{\Field_1}$ \`e di Galois, allora $\Field_1 = \Fixed{\Field_2}{\RelativeAutomorphisms{\Field_2}{\Field_1}}$.
	\end{itemize}
\end{Corollary}
\Proof L'estensione $\FieldExtension{\Field_2}{\Fixed{\Field_2}{\RelativeAutomorphisms{\Field_2}{\Field_1}}}$ \`e normale perch\'e \`e normale $\FieldExtension{\Field_2}{\Field_1}$ e \`e separabile per un teorema precedentemente dimostrato, dunque \`e di Galois. Per il lemma di Artin, $\RelativeAutomorphisms{\Field_2}{\Field_1} \IsSubgroup \GaloisGroup{\Field_2}{\Fixed{\Field_2}{\RelativeAutomorphisms{\Field_2}{\Field_1}}} = \RelativeAutomorphisms{\Field_2}{\Fixed{\Field_2}{\RelativeAutomorphisms{\Field_2}{\Field_1}}} \IsSubgroup \RelativeAutomorphisms{\Field_2}{\Field_1}$. Quindi $\GaloisGroup{\Field_2}{\Fixed{\Field_2}{\RelativeAutomorphisms{\Field_2}{\Field_1}}} = \RelativeAutomorphisms{\Field_1}{\Field_2}$.
\par Se inoltre $\FieldExtension{\Field_2}{\Field_1}$ \`e di Galois, allora \`e separabile e dunque anche $\FieldExtension{\Fixed{\Field_2}{\RelativeAutomorphisms{\Field_2}{\Field_1}}}{\Field_1}$ \`e separabile; ma poich\'e essa \`e anche puramente inseprabile, necessariamente \`e banale. \EndProof
\begin{Theorem}
	\TheoremName{Teorema fondamentale della teoria di Galois (caso finito)}[fondamentale della teoria di Galois (caso finito)][teorema] Sia $\FieldExtension{\Field_2}{\Field_1}$ un'estensione di Galois. Siano le classi
	\begin{itemize}
		\item $G = \lbrace \Group \in \PowerSet{\GaloisGroup{\Field_2}{\Field_1}} | \Group \IsSubgroup \GaloisGroup{\Field_2}{\Field_1} \rbrace$;
		\item $F = \lbrace \Field \in \PowerSet{\Field_2} | \Field_1 \subseteq \Field \subseteq \Field_2\text{ e $\Field$ \`e un campo} \rbrace$.
	\end{itemize}
	Definiamo le applicazioni
	\begin{itemize}
		\item $\phi: G \rightarrow F$ tale che $\phi(\Group) = \Fixed{\Field_2}{\Group}$;
		\item $\psi: F \rightarrow G$ tale che $\psi(\Field) = \GaloisGroup{\Field_2}{\Field}$.
	\end{itemize}
	Abbiamo i seguenti risultati:
	\begin{itemize}
		\item $\phi \circ \psi = \Identity$;
		\item se $\FieldExtension{\Field_2}{\Field_1}$ \`e finita, allora $\psi \circ \phi = \Identity$;
		\item dato $\Field \in F$, $\FieldExtension{\Field}{\Field_1}$ \`e normale se e solo se $\psi(\Field) = \GaloisGroup{\Field_2}{\Field} \IsNormalSubgroup \GaloisGroup{\Field_2}{\Field_1}$;
		\item dato $\Field \in F$, se $\FieldExtension{\Field}{\Field_1}$ normale allora $\Isomorphic{\GaloisGroup{\Field}{\Field_1}}{\Quotient{\GaloisGroup{\Field_2}{\Field_1}}{\GaloisGroup{\Field_2}{\Field}}}$.
	\end{itemize}
\end{Theorem}
\Proof Sia $\Field \in F$. Abbiamo $(\phi \circ \psi)(\Field) = \phi(\GaloisGroup{\Field_2}{\Field}) = \Fixed{\Field_2}{\GaloisGroup{\Field_2}{\Field}}$, da cui, per il corollario precedente, $(\phi \circ \psi)(\Field) = \Field$.
\par Sia $\Group \in G$. Abbiamo $(\psi \circ \phi)(\Group) = \psi(\Fixed{\Field_2}{\Group}) = \GaloisGroup{\Field_2}{\Fixed{\Field_2}{\Group}}$, da cui, per il lemma di Artin, se $\Group$ \`e finito allora $(\psi \circ \phi)(\Group) = \Group$.
\par Ora $\FieldExtension{\Field}{\Field_1}$ \`e normale se e solo se, per ogni omomorfismo relativo $\RelativeHomomorphism$ di $\Field$ rispetto a $\Field_1$, abbiamo $\RelativeHomomorphism(\Field) = \Field$.  Supponiamo $\FieldExtension{\Field}{\Field_1}$ sia effettivamente normale e sia $\GaloisAutomorphism \in \GaloisGroup{\Field_2}{\Field_1}$: allora $\GaloisAutomorphism(\Field) = \Field$. Viceversa, se ogni automorfisma di Galois $\GaloisAutomorphism{\Field_2}{\Field_1}$ \`e tale che $\GaloisAutomorphism(\Field) = \Field$, allora dato che ogni omomorfismo relativo di $\Field$ rispetto a $\Field_1$ si estende ad un automorfismo di $\GaloisGroup{\Field_2}{\Field_1}$, allora $\FieldExtension{\Field}{\Field_1}$ \`e normale. Dunque $\FieldExtension{\Field}{\Field_1}$ \`e normale se e solo se $\ForAll{\GaloisAutomorphism \in \GaloisGroup{\Field_2}{\Field_1}}{\GaloisAutomorphism{\Field} = \Field}$.
\par D'altra parte $\GaloisGroup{\Field_2}{\Field} \IsNormalSubgroup \GaloisGroup{\Field_2}{\Field_1}$ se e solo se, per ogni automorfismo di Galois $\GaloisAutomorphism \in \GaloisGroup{\Field_2}{\Field_1}$, abbiamo $\GaloisAutomorphism\GaloisGroup{\Field_2}{\Field}\GaloisAutomorphism^{-1} = \GaloisGroup{\Field_2}{\GaloisAutomorphism(\Field)} = \GaloisGroup{\Field_2}{\Field}$. Dunque $\GaloisGroup{\Field_2}{\Field} \IsNormalSubgroup \GaloisGroup{\Field_2}{\Field_1}$ se e solo se $\ForAll{\GaloisAutomorphism \in \GaloisGroup{\Field_2}{\Field_1}}{\GaloisAutomorphism\GaloisGroup{\Field_2}{\Field}\GaloisAutomorphism^{-1} = \GaloisGroup{\Field_2}{\GaloisAutomorphism(\Field)} = \GaloisGroup{\Field_2}{\Field}}$.
\par Si conclude, per l'iniettivit\`a di $\psi$ che $\FieldExtension{\Field}{\Field_1}$ \`e normale se e solo se $\GaloisGroup{\Field_2}{\Field} \IsNormalSubgroup \GaloisGroup{\Field_2}{\Field_1}$.
\par Infine l'isomorfismo tra $\GaloisGroup{\Field}{\Field_1}$ e $\Quotient{\GaloisGroup{\Field_2}{\Field_1}}{\GaloisGroup{\Field_2}{\Field}}$ si deduce dall'omomorfismo surgettivo che ad ogni automorfismo di $\GaloisGroup{\Field_2}{\Field_1}$ associa la sua restrizione a $\Field$. \EndProof
\begin{Definition}
	Le corrispondenze definite dal teorema precedente dalle applicazioni $\phi$ e $\psi$ del teorema precedente prendono entrambe il nome di \Define{corrispondenza di Galois}[di Galois][corrispondenza].
\end{Definition}
\par
Possiamo utilizzare la corrispondenza di Galois per fornire una dimostrazione alternativa del teorema dell'elemento primitivo.
\begin{Corollary}
\TheoremName{Teorema dell'elemento primitivo}[dell'elemento primitivo][teorema]
	Sia $\FieldExtension{\Field_2}{\Field_1}$ un'estensione di campi finita e separabile.
	$\FieldExtension{\Field_2}{\Field_1}$ \`e semplice.
\end{Corollary}
\Proof
Consideriamo la chiusura normale $\NormalClosure{\Field_2}$ di $\Field_2$ in $\AlgebraicClosure{\Field_2}$:
$\FieldExtension{\NormalClosure{\Field_2}}{\Field_1}$ \`e separabile e dunque di Galois; inoltre \`e anche finita.
Il gruppo $\GaloisGroup{\NormalClosure{\Field_2}}{\Field_1}$ \`e dunque finito e ammette pertanto solo un numero finito di sottogruppi.
Ne deduciamo che $\FieldExtension{\NormalClosure{\Field_2}}{\Field_1}$ ammette solo un numero finito di estensioni.
Dunque anche $\FieldExtension{\Field_2}{\Field_1}$ ammette solo un numero finito di estensioni.
Ne consegue che $\FieldExtension{\Field_2}{\Field_1}$ \`e semplice.
\EndProof
\begin{Corollary}
	Sia $\FieldExtension{\Field_2}{\Field_1}$ un'estensione finita di Galois e
	siano $\Group_1, \Group_2 \IsSubgroup \GaloisGroup{\Field_2}{\Field_1}$.
	Abbiamo
	\begin{itemize}
		\item
		$\Coimplies
		{\Fixed{\Field_2}{\Group_1}
		\subseteq \Fixed{\Field_2}{\Group_2}}
		{\Group_2 \subseteq \Group_1}$;
		\item
		$\FieldComposition
		{\Fixed{\Field_2}{\Group_1}}
		{\Fixed{\Field_2}{\Group_2}} =
		\Fixed{\Field_2}{\Group_1 \cap \Group_2}$;
		\item
		$\Fixed{\Field_2}{\Group_1} \cap
		\Fixed{\Field_2}{\Group_2} =
		\Fixed{\Field_2}
		{\SpanGroup{\Group_1,\Group_2}}$.
	\end{itemize}
\end{Corollary}
\Proof
Il primo punto segue direttamente dal teorema fondamentale della teoria di Galois.
\par
Per il secondo punto osserviamo che
$\Group_1 \cap \Group_2 \subseteq \Group_1$,
da cui $\Fixed{\Field_2}{\Group_1} \subseteq \Fixed{\Field_2}{\Group_1 \cap \Group_2}$
e analogamente
$\Fixed{\Field_2}{\Group_2} \subseteq \Fixed{\Field_2}{\Group_1 \cap \Group_1}$.
Ne deduciamo
$\FieldComposition{\Fixed{\Field_2}{\Group_1}}{\Fixed{\Field_2}{\Group_2}}
\subseteq \Fixed{\Field_2}{\Group_1 \cap \Group_2}$.
D'altra parte abbiamo
$\GaloisGroup{\Field_2}
{\FieldComposition{\Fixed{\Field_2}{\Group_1}}{\Fixed{\Field_2}{\Group_2}}}
\subseteq
\GaloisGroup{\Field_2}{\Fixed{\Field_2}{\Group_1}} \cap
\GaloisGroup{\Field_2}{\Fixed{\Field_2}{\Group_2}}$,
da cui, tramite la corrispondenza di Galois,
$\FieldComposition{\Fixed{\Field_2}{\Group_1}}{\Fixed{\Field_2}{\Group_2}}
\supseteq
\Fixed{\Field_2}{\Group_1} \cap \Fixed{\Field_2}{\Group_2}
= \Fixed{\Field_2}{\Group_1 \cap \Group_2}$.
\par
Per il terzo punto osserviamo che
$\Group_1 \subseteq \SpanGroup{\Group_1,\Group_2}$,
da cui $\Fixed{\Field_2}{\SpanGroup{\Group_1,\Group_2}} \subseteq \Fixed{\Field_2}{\Group_1}$
e analogamente
$\Fixed{\Field_2}{\SpanGroup{\Group_1,\Group_2}} \subseteq \Fixed{\Field_2}{\Group_2}$.
Ne deduciamo
$\Fixed{\Field_2}{\SpanGroup{\Group_1,\Group_2}}
\subseteq \Fixed{\Field_2}{\Group_1} \cap \Fixed{\Field_2}{\Group_2}$.
D'altra parte ogni elemento di $\Field_2$ che viene lasciato fisso da ogni
automorfismo di $\Group_1$ e di $\Group_2$ viene lasciato fisso anche da
ogni automorfismo di $\SpanGroup{\Group_1,\Group_2}$.
Quindi abbiamo anche
$\Fixed{\Field_2}{\Group_1}
\cap
\Fixed{\Field_2}{\Group_2}
\subseteq
\Fixed{\Field_2}{\SpanGroup{\Group_1}{\Group_2}}$.
\EndProof
\begin{Theorem}
	Con le notazioni del teorema precedente, esistono estensioni di Galois infinite $\FieldExtension{\Field_2}{\Field_1}$ tali che $\phi$ non sia iniettiva; e dunque in particolare tali che $\psi \circ \phi \neq \Identity$.
\end{Theorem}
\Proof
Consideriamo $\Field = \GaloisField{\Prime}$ per qualche $\Prime \in \mathbb{Z}$ primo.
L'estensione $\FieldExtension{\AlgebraicClosure{\Field}}{\Field}$ \`e
\begin{itemize}
	\item normale perch\'e $\AlgebraicClosure{\Field}$ \`e la chiusura algebrica di $\Field$;
	\item separabile perch\'e i campi di Galois sono perfetti.
\end{itemize}
Dunque $\FieldExtension{\AlgebraicClosure{\Field}}{\Field}$ \`e un'estensione di Galois.
Consideriamo l'endomorfismo di Frobenius
$\Frobenius: \AlgebraicClosure{\Field} \rightarrow \AlgebraicClosure{\Field}$.
Essendo $\AlgebraicClosure{\Field}$ algebricamente chiuso di caratteristica $\Prime$, ogni suo elemento ammette una e una sola radice $\Prime$-esima, dunque $\Frobenius$ \`e un automorfismo.
Ora, \`e chiaro che per ogni
$\alpha \in \Field$ abbiamo
$\Frobenius{\alpha} = \alpha^\Prime = \alpha$.
Abbiamo dunque $\SpanGroup{\Frobenius} \IsSubgroup \GaloisGroup{\AlgebraicClosure{\Field}}{\Field}$.
Inoltre, $\alpha \in \AlgebraicClosure{\Field}$ \`e tale che $\Frobenius{\alpha} = \alpha$ se e solo se $\alpha^\Prime = \alpha$ cio\`e se e solo se $\alpha \in \Field$, da cui
$\Fixed{\AlgebraicClosure{\Field}}{\SpanGroup{\Frobenius}} = \Field = \Fixed{\AlgebraicClosure{\Field}}{\GaloisGroup{\AlgebraicClosure{\Field}}{\Field}}$.
\par Supponiamo $\phi$ inettiva.
Allora deve essere $\SpanGroup{\Frobenius} =
\GaloisGroup{\AlgebraicClosure{\Field}}{\Field}$.
\par Consideriamo l'omomorfismo che a $n \in \mathbb{Z}$ associa $\Frobenius^n \in \SpanGroup{\Frobenius}$: esso \`e chiaramente suriettivo.
Inoltre, poich\'e $\AlgebraicClosure{\Field}$ \`e un campo finito e l'equazione $x^{\Prime^n} = x$ ha infinite soluzioni se e solo se $n = 0$, allora l'omomorfismo \`e anche iniettivo.
Dunque $\SpanGroup{\Frobenius}$ \`e isomorfo a $\mathbb{Z}$.
Ora, tutti i sottogruppi di $\mathbb{Z}$ sono finiti,
dunque tutte le estensioni di campo del tipo
$\FieldExtension{\AlgebraicClosure{\Field}}{\Field}$
sono finite per la suriettivit\`a di $\phi$ ed il lemma di Artin.
\par
Tuttavia $\Field' = \GaloisField{\Prime^2}$ \`e un campo tale che $\FieldExtension{\Field'}{\Field}$ \`e finito e dunque $\FieldExtension{\AlgebraicClosure{\Field}}{\Field}$ \`e infinito.
Questo \`e assurdo: dunque
$\SpanGroup{\Frobenius} \neq \GaloisGroup{\AlgebraicClosure{\Field}}{\Field}$
e $\phi$ non \`e un'applicazione iniettiva.
\EndProof
\begin{Theorem}
	Siano date le seguenti torri d'estensione:
	\begin{itemize}
		\item $\Field_1 \subseteq
		\Field_2 \subseteq
		\AlgebraicClosure{\Field_1}$,
		con $\FieldExtension{\Field_2}{\Field_1}$
		finita e di Galois;
		\item $\Field_1 \subseteq
		\Field_3 \subseteq
		\AlgebraicClosure{\Field_1}$.
	\end{itemize}
	Abbiamo $\Isomorphic
	{\GaloisGroup{\FieldComposition{\Field_2}{\Field_3}}
	{\Field_2}}
	{\GaloisGroup{\Field_2}{\Field_2 \cap \Field_3}}$.
\end{Theorem}
\Proof
Sia $\GaloisAutomorphism \in
\GaloisGroup{\FieldComposition{\Field_2}{\Field_3}}{\Field_2}$.
Poich\'e $\FieldExtension{\Field_2}{\Field_1}$ \`e normale,
$\Restricted{\GaloisAutomorphism}{\Field_2}(\Field_2) =
\Field_2$.
Inoltre $\Field_2 \cap \Field_3 \subseteq \Field_2$,
Esiste dunque un'applicazione
$\Homomorphism:
\GaloisGroup{\FieldComposition{\Field_2}{\Field_3}}{\Field_2}
\rightarrow
\GaloisGroup{\Field_2}{\Field_2 \cap \Field_3}$:
\`e immediato verificare che si tratta di un omomorfismo.
\par
$\Homomorphism$ \`e iniettivo perch\'e se due automorfismi di
$\GaloisGroup{\FieldComposition{\Field_2}{\Field_3}}{\Field_2}$
coincidono su $\Field_2$, allora, coincidendo anche su $\Field_3$,
devono coincidere su $\FieldComposition{\Field_2}{\Field_3}$.
\par
Proviamo che $\Homomorphism$ \`e anche suriettivo.
Abbiamo $\Image{\Homomorphism} \subseteq \GaloisGroup{\Field_2}{\Field_2 \cap \Field_3}$,
dunque $\Field_2 \cap \Field_3 \subseteq \Fixed{\Field_2}{\Image{\Homomorphism)}}$.
Inoltre, sia $\alpha \in \Fixed{\Field_2}{\Image{\Homomorphism}}$.
Abbiamo $\alpha \in \Field_2$ perch\'e $\Fixed{\Field_2}{\Image{\Homomorphism}} \subseteq
\Field_2$.
Inoltre, per costruzione, per ogni $\GaloisAutomorphism \in
\GaloisGroup{\FieldComposition{\Field_2}{\Field_3}}{\Field_3}$ abbiamo
$\Homomorphism(\GaloisAutomorphism) (\alpha) = \alpha$, da cui
$\alpha \in \Field_3$.
Ne consegue $\Field_2 \cap \Field_3 = \Fixed{\Field_2}{\Image{\Homomorphism}}$ e quindi,
per la corrispondenza di Galois e a finitezza di $\FieldExtension{\Field_2}{\Field_1}$,
la suriettivit\`a di $\Homomorphism$.
\EndProof
\begin{Theorem}
	Siano $\FieldExtension{\Field_2}{\Field_1}$ e
	$\FieldExtension{\Field_3}{\Field_1}$ estensioni di Galois.
	Allora l'omomorfismo
	$\Homomorphism
	\GaloisGroup{\FieldComposition{\Field_2}{\Field_3}}{\Field_1}
	\rightarrow
	\GaloisGroup{\Field_2}{\Field_1} \times \GaloisGroup{\Field_3}{\Field_1}$,
	che a $\GaloisAutomorphism \in
	\GaloisGroup{\FieldComposition{\Field_2}{\Field_3}}{\Field_1}$
	associa $(\GaloisAutomorphism_1,\GaloisAutomorphism_2 \in
	\GaloisGroup{\Field_2}{\Field_1} \times \GaloisGroup{\Field_3}{\Field_1}$,
	dove $\GaloisAutomorphism_1$ e $\GaloisAutomorphism_2$ hanno lo stesso grafo di
	$\Restricted{\GaloisAutomorphism}{\Field_2}$ e $\Restricted{\GaloisAutomorphism}{\Field_3}$,
	\`e iniettiva.
	Inoltre, se $\Field_2 \cap \Field_3 = \Field_1$, allora \`e anche suriettiva.
	\par
	Infine, se $\FieldExtension{\Field_2}{\Field_1}$ e $\FieldExtension{\Field_3}{\Field_1}$
	sono finite, allora la biettivit\`a dell'applicazione implica
	$\Field_2 \cap \Field_3 = \Field_1$.
\end{Theorem}
\Proof
L'iniettivit\`a segue subito dal fatto che automorfismi di
$\GaloisGroup{\FieldComposition{\Field_2}{\Field_3}}{\Field_1}$
che coincidono su $\Field_2$ e $\Field_3$ coincidono anche su
$\FieldComposition{\Field_2}{\Field_3}$.
\par
La suriettivit\`a nel caso $\Field_2 \cap \Field_3 = \Field_1$
segue dal fatto che
$\Image{\Homomorphism} =
= \lbrace (\GaloisAutomorphism_1, \GaloisAutomorphism_2) \in
\GaloisGroup{\Field_2}{\Field_1} \times \GaloisGroup{\Field_3}{\Field_1} |
\Restricted{\GaloisAutomorphism_1}{\Field_2 \cap \Field_3} =
\Restricted{\GaloisAutomorphism_2}{\Field_2 \cap \Field_3}$.
\par
Supponiamo $\FieldExtension{\Field_2}{\Field_1}$ e
$\FieldExtension{\Field_3}{\Field_1}$ finite.
Per la corrispondenza di Galois, abbiamo che la biettivit\`a di
$\Homomorphism$ equivale all'equicardinalit\`a di
$\GaloisGroup{\FieldComposition{\Field_2}{\Field_3}}{\Field_1}$
e
$\GaloisGroup{\Field_2}{\Field_1} \times \GaloisGroup{\Field_3}{\Field_1}$
dunque a
$\FieldDegree{\FieldComposition{\Field_2}{\Field_3}}{\Field_1} =
\FieldDegree{\Field_2}{\Field_1}
\FieldDegree{\Field_3}{\Field_1}$,
a sua volta equivalente per il teorema precedente a
$\FieldDegree{\Field_3}{\Field_2 \cap \Field_3} =
\FieldDegree{\Field_3}{\Field_1}$ e quindi a
$\Field_2 \cap \Field_3 = \Field_1$.
\EndProof
\begin{Theorem}
	Siano $\Field$ un campo e sia
	$\Polynomial \in \RingAdjunction{\Field}{x}$ separabile.
	Sia $\Set \subseteq \AlgebraicClosure{\Field}$ l'insieme di tutte
	le radici di $\Polynomial$.
	Allora
	\begin{itemize}
		\item l'applicazione
		$\Homomorphism:
		\GaloisGroup{\FieldAdjunction{\Field}{\Set}}{\Field}
		\rightarrow
		\Automorphisms{\Set} \Isomorphic
		\SymmetricGroup{\Cardinality{\Set}}$
		 che a $\GaloisAutomorphism \in
		\GaloisGroup{\FieldAdjunction{\Field}{\Set}}{\Field}$
		 associa
		$\Restricted{\GaloisAutomorphism}{\Set} \in
		\Automorphisms{\Set}$
		\`e un omomorfismo iniettivo;
		\item $\Polynomial$ \`e irriducibile se e solo se
		$\GaloisGroup{\FieldAdjunction{\Field}{\Set}}{\Field}$
		agisce transitivamente su $\Set$.
	\end{itemize}
\end{Theorem}
\Proof
$\Homomorphism$ \`e un omomorfismo perch\'e gli automorfismi di Galois
trasformano radici di polinomi a coefficienti nel campo di base nelle
medesime radici. Inoltre \`e un omomorfismo iniettivo perch\'e se un
automorfismo di Galois lascia fisse tutte le radici di $\Polynomial$,
allora deve essere l'identit\`a/
\par
Affermare che $\Polynomial$ \`e irriducibile equivale ad affermare che \`e
polinomio minimo di ogni sua radice. La transitivit\`a dell'azione di
$\GaloisGroup{\FieldAdjunction{\Field}{\Set}}{\Field}$ segue dunque
dal teorema \ref{th_esistenzaomomorfismirelativi_0}.
\EndProof
\begin{Corollary}
	Sia $\FieldExtension{\Field_2}{\Field_1}$ un'estensione di Galois
	finita.
	Abbiamo $\GaloisGroup{\Field_2}{\Field_1} \IsSubgroup
	\SymmetricGroup{\FieldDegree{\Field_2}{\Field_1}}$.
\end{Corollary}
\Proof Per il teorema dell'elemento primitivo, esiste
$\alpha \in \Field_2$ tale che
$\Field_2 = \FieldAdjunction{\Field_1}{\alpha}$. Dunque $\Field_2$ \`e
il campo di spezzamento  del polinimio minimo $\MinimalPolynomial{\alpha}$
e la tesi segue dal teorema precedente.
\EndProof
\Proof
Segue direttamente dal teorema di Cayley.
\EndProof
\begin{Theorem}
	\TheoremName{Teorema fondamentale dell'algebra} Il campo $\mathbb{C}$ \`e algebricamente chiuso.
\end{Theorem}
\Proof Supponiamo $\Field$ campo tale che $\FieldExtension{\Field}{\mathbb{C}}$ sia algebrica. La chiusura normale di $\Field$ su $\mathbb{R}$ ha grado finito, dunque $\FieldExtension{\NormalClosure{\Field}}{\mathbb{C}}$ ha ancora grado finito. Poich\'e inoltre $\Characteristic{\mathbb{R}} = 0$, $\FieldExtension{\NormalClosure{\Field}}{\mathbb{C}}$ \`e separabile e dunque di Galois.
\par Poich\'e $\FieldDegree{\mathbb{C}}{\mathbb{R}} = 2$ divide $\FieldDegree{\NormalClosure{\Field}}{\mathbb{R}}$, esiste un $2$-Sylow $\Sylow[2]$ sottogruppo di $\GaloisGroup{\NormalClosure{\Field}}{\mathbb{R}}$, a cui corrisponde una sottoestensione $\FieldExtension{\NormalClosure{\Field}}{\VarField}$, di grado uguale all'ordine di $\Sylow[2]$. Dunque $\FieldExtension{\VarField}{\mathbb{R}}$ \`e un estensione di grado dispari. Poich\'e non esistono polinomi irriducibili non lineari di grado dispari su $\mathbb{R}$, necessarimente $\VarField = \mathbb{R}$, da cui $\GaloisGroup{\NormalClosure{\Field}}{\mathbb{C}}$ ha ordine potenza di $2$. Se quest'ultimo ordine \`e almeno $2$, allora $\GaloisGroup{\NormalClosure{\Field}}{\mathbb{C}}$ ammette un sottogruppo di ordine $2$, a cui corrisponde una estensione $\FieldExtension{\VarField'}{\mathbb{C}}$ di grado due. Ma tutti i polinomi di $\RingAdjunction{\mathbb{C}}{x}$ di grado $2$ sono riducibili. Dunque deve essere $\FieldDegree{\NormalClosure{\Field}}{\mathbb{C}} = 1$, da cui $\Field = \mathbb{C}$. \EndProof

\subsection{Estensioni abeliane, cicliche, ciclotomiche.}
\label{EstensioniAbelianeCiclicheCiclotomiche}
\begin{Definition}
	Sia $\FieldExtension{\Field_2}{\Field_1}$
	un'estensione di Galois. L'estensione si dice
	\begin{itemize}
		\item \Define{abeliana}[abeliana][estensione] quando
		$\GaloisGroup{\Field_2}{\Field_1}$ \`e un gruppo abeliano;
		\item \Define{ciclica}[ciclica][estensione] quando
		$\GaloisGroup{\Field_2}{\Field_1}$ \`e un gruppo ciclico.
	\end{itemize}
\end{Definition}
\begin{Theorem}
	Abelianit\`a e ciclicit\`a non sono propriet\`a che si
	trasferiscono lungo le torri.
\end{Theorem}
\Proof
Segue immediatamente dal fatto che la propriet\`a di essere di Galois non
si trasferisce lungo le torri.
\EndProof
\begin{Theorem}
	Sia $\Field_1 \subseteq \Field_2 \subseteq \Field_3$ una torre
	d'estensioni con $\FieldExtension{\Field_2}{\Field_1}$ normale. Se
	\begin{itemize}
		\item $\FieldExtension{\Field_3}{\Field_1}$ \`e abeliana,
		allora $\FieldExtension{\Field_3}{\Field_2}$ e
		$\FieldExtension{\Field_2}{\Field_1}$ sono abeliane;
		\item $\FieldExtension{\Field_3}{\Field_1}$ \`e ciclica,
		allora $\FieldExtension{\Field_3}{\Field_2}$ e
		$\FieldExtension{\Field_2}{\Field_1}$ sono cicliche;
	\end{itemize}
\end{Theorem}
\Proof
$\FieldExtension{\Field_3}{\Field_2}$ e
$\FieldExtension{\Field_2}{\Field_1}$ sono estensioni di Galois: la prima
perch\'e \`e di Galois $\FieldExtension{\Field_3}{\Field_2}$, la seconda
per lo stesso motivo unito alla normalit\`a dell'estensione stessa.
\par
L'abelianit\`a o ciclit\`a di
$\FieldExtension{\Field_3}{\Field_2}$ segue dal fatto che sottogruppi di
gruppi abeliani o ciclici sono abeliani o ciclici rispettivamente.
\par
L'abelianit\`a o ciclit\`a di
$\FieldExtension{\Field_2}{\Field_1}$ segue dal fatto che quozienti di
gruppi abeliani o ciclici sono abeliani o ciclici rispettivamente.
\EndProof
\begin{Theorem}
	Consideriamo le estenioni
	$\FieldExtension{\Field_2}{\Field_1}$ e
	$\FieldExtension{\Field_3}{\Field_1}$, con
	$\FieldExtension{\Field_2}{\Field_1}$ finita.
	Abbiamo
	\begin{itemize}
		\item
		se $\FieldExtension{\Field_2}{\Field_1}$ \`e abeliana,
		allora
		$\FieldExtension{\FieldComposition{\Field_2}{\Field_3}}
		{\Field_3}$ \`e abeliana;
		\item
		se $\FieldExtension{\Field_2}{\Field_1}$ \`e ciclica,
		allora
		$\FieldExtension{\FieldComposition{\Field_2}{\Field_3}}
		{\Field_3}$ \`e ciclica.
	\end{itemize}
\end{Theorem}
\Proof
Segue immediatamente dal fatto che
$\GaloisGroup{\FieldComposition{\Field_2}{\Field_3}}{\Field_3} \Isomorphic
\GaloisGroup{\Field_2}{\Field_2 \cap \Field_3} \IsSubgroup
\GaloisGroup{\Field_2}{\Field_1}$ e dal fatto che sottogruppi di gruppi
abeliani o ciclici sono abeliani o ciclici rispettivamente.
\EndProof
\begin{Theorem}
	Siano le estensioni
	$\FieldExtension{\Field_2}{\Field_1}$ e
	$\FieldExtension{\Field_3}{\Field_1}$.
	Abbiamo,
	\begin{itemize}
		\item
		se
		$\FieldExtension{\Field_2}{\Field_1}$ e
		$\FieldExtension{\Field_3}{\Field_1}$ sono abeliane,
		allora
		$\FieldExtension{\FieldComposition{\Field_2}{\Field_3}}
		{\Field_1}$ \`e abeliana;
		\item
		se
		$\FieldExtension{\Field_2}{\Field_1}$ e
		$\FieldExtension{\Field_3}{\Field_1}$ sono ciclice,
		e
		$\FieldDegree{\Field_2}{\Field_1}$ e
		$\FieldDegree{\Field_3}{\Field_1}$ sono coprimi,
		allora
		$\FieldExtension{\FieldComposition{\Field_2}{\Field_3}}
		{\Field_1}$ \`e ciclica.
	\end{itemize}
\end{Theorem}
\Proof
Abbiamo provato che
$\GaloisGroup{\FieldComposition{\Field_2}{\Field_3}}
{\Field_1}$ si immerge in
$\GaloisGroup{\Field_2}{\Field_1} \times
\GaloisGroup{\Field_3}{\Field_1}$, il quale \`e
\begin{itemize}
	\item un gruppo abeliano se i fattori sono abeliani;
	\item un gruppo ciclico se i fattori sono ciclici e di ordini
	coprimi.
\end{itemize}
\EndProof
\begin{Theorem}
	Sia $\Field$ un campo algebricamente chiuso e
	$\Automorphism \in \Automorphisms{\Field}$.
	Sia $\Field_0$ un'estensione finita di
	$\Fixed{\Field}{\SpanGroup{\Automorphism}}$.
	$\Field_0$ \`e ciclica.
\end{Theorem}
\Proof
Poich\'e $\SpanGroup{\Automorphism}$ \`e un gruppo ciclico,
\`e un gruppo ciclico anche $\SpanGroup{\Automorphism_0}$, dove
$\Automorphism_0$ \`e l'autormofismo di $\Field_0$ con lo stesso
grafo di $\Restricted{\Automorphism}{\Field_0}$.
\par
Abbamo allora $\Field = \Fixed{\Field_0}{\SpanGroup{\Automorphism_0}}$ e,
per il lemma di Artin, $\FieldExtension{\Field_0}{\Field}$ \`e
un'estensione di Galois con gruppo di Galois $\SpanGroup{\Automorphism_0}$
e dunque ciclica.
\EndProof
\begin{Theorem}
	Sia $a \in \mathbb{N}$. $\FieldAdjunction{\mathbb{Q}}{\sqrt{-a}}$
	non si immerge in un'estensione ciclica di $\mathbb{Q}$ di grado
	multiplo di $4$.
\end{Theorem}
\Proof
Sia la torre d'estensione
$\mathbb{Q} \subseteq
\FieldAdjunction{\mathbb{Q}}{\sqrt{-a}} \subseteq
\Field$,
con $\FieldExtension{\Field}{\mathbb{Q}}$ estensione ciclica
di grado $4d$, con $d \in \NotZero{\mathbb{N}}$.
Procediamo per induzione su $d$.
\par
Supponiamo $d = 1$.
$\Field \cap \mathbb{R} \neq \Field$ perch\'e
$\sqrt{-a} \notin \mathbb{R}$.
Inoltre $\Field \cap \mathbb{R} \neq \mathbb{Q}$ perch\'e
\par
Quindi
$\FieldDegree{\Field}{\Field \cap \mathbb{R}} = 2 =
\FieldDegree{\Field}{\FieldAdjunction{\mathbb{Q}}{\sqrt{-a}}}$:
$\Field$ ammette dunque sottocampi distinti di grado $2$ che estendono
$\mathbb{Q}$, in contraddizione con la ciclicit\`a di
$\GaloisField{\Field}{\mathbb{Q}}$ e il teorema di corrispondenza di
Galois.
\par
Sia $d > 1$, all'estensione
$\FieldExtension{\Field}{\FieldAdjunction{\mathbb{Q}}{\sqrt{-a}}}$
corrisponde un sottogruppo ciclico
$\Subgroup \IsSubgroup \GaloisGroup{\Field}{\mathbb{Q}}$ di ordine $2d$.
Sia poi $\Subgroup_0 \IsSubgroup \Subgroup$ di ordine $d$: a $\Subgroup_0$
corrisponde un'estensione di campo $\FieldExtension{\Field}{\Field_0}$,
con
$\FieldAdjunction{\mathbb{Q}}{\sqrt{-a}} \subseteq \Field_0$ e tale che
$\FieldDegree{\Field_0}{\FieldAdjunction{\mathbb{Q}}{\sqrt{-a}}} = 2$.
Abbiamo dunque la torre d'estensioni
$\mathbb{Q} \subseteq \FieldAdjunction{\mathbb{Q}}{\sqrt{-a}} \subseteq
\Field_0$ con
$\FieldDegree{\Field_0}{\mathbb{Q}} = 4$.
L'estensione $\FieldDegree{\Field_0}{\mathbb{Q}}$ \`e anche ciclica
perch\'e il suo gruppo di Galois \`e isomorfo al quoziente
$\Quotient{\GaloisGroup{\Field}{\mathbb{Q}}}
{\GaloisGroup{\Field}{\Field_0}}$.
\par
Ci si riconduce dunque al caso $d = 1$.
\EndProof



	\part{Topologia}
	\chapter{Topologia generale}
	\section{Definizioni e teoremi di base.}\label{DefinizioniETeoremiDiBase}
\begin{Definition}
	Sia $\TopologicalSpace$ una classe. Si dice che $\Topology$ \`e
	una \Define{topologia} di $\TopologicalSpace$ quando
	\begin{itemize}
		\item $\ForAll{\Open \in \Topology}{\Open \subseteq
		\TopologicalSpace}$;
		\item $\ForAll{\Part \subseteq \Topology}{\bigcup_{\Open
		\in \Part} \Open \in \Topology}$;
		\item $\ForAll{\Part \subseteq \Topology}{\Implies{\Cardinality{\Part}
		\leq \aleph_0}{{\bigcap_{\Open \in \Part} \Open \in
		\Topology}}}$;
	\end{itemize}
	La coppia $(\TopologicalSpace,\Topology)$ si chiama
	\Define{spazio topologico}[topologico][spazio] e ogni elemento di
	$\Topology$ si chiama \Define{aperto}.
\end{Definition}
\begin{Definition}
	Siano $(\TopologicalSpace_1,\Topology_1)$ e
	 $(\TopologicalSpace_2, \Topology_2)$ spazi topologici.
	 Un'applicazione
	 $\Continuous:  \TopologicalSpace_1 \rightarrow
	 \TopologicalSpace_2$ tale che
	$\Implies{\Open \in \Topology_2}{\Continuous^{-1}(\Open) \in
	 \Topology_1)}$ si dice \Define{continua}[continua][applicazione].
\end{Definition}
\begin{Theorem}
	Le applicazioni continue costituiscono una categoria.
\end{Theorem}
\Proof \`E sufficiente provare che la composizione di applicazioni
 continue \`e continua. Siano
\begin{itemize}
	\item $\Continuous_1: \TopologicalSpace_1 \rightarrow
	\TopologicalSpace_2$;
	\item $\Continuous_2: \TopologicalSpace_2 \rightarrow
	\TopologicalSpace_3$;
\end{itemize}
applicazioni continue tra gli spazi topologici
$(\TopologicalSpace_1,\Topology_1)$,
$(\TopologicalSpace_2,\Topology_2)$ e
$(\TopologicalSpace_3,\Topology_3)$.
Sia $\Open \in \Topology_3$. Abbiamo
$(\Continuous_2 \circ \Continuous_1)^{-1}(\Open) =
\Continuous_1^{-1}(\Continuous_2^{-1}(\Open)) \in \Topology_1$.
Dunque $\Continuous_2 \circ \Continuous_1$ \`e continua. \EndProof
\begin{Definition}
	Chiamiamo \Define{categoria degli spazi topologici}[degli spazi topologici][categoria]
	la categoria costituita da tutte le applicazioni continue, d'ora
	in poi denotata $\CatTopologicalSpace$. \`E inoltre definito il
	funtore dimenticante canonico $\Forgetful: \CatTopologicalSpace
	\rightarrow \CatClass$ che ad ogni spazio topologico
	$(\TopologicalSpace,\Topology)$ associa la classe
	$\TopologicalSpace$. 
\end{Definition}
\begin{Definition}
	Chiamiamo \Define{omeomorfismi} gli isomorfismi di
	$\CatTopologicalSpace$.
\end{Definition}
\begin{Definition}
	Sia $(\TopologicalSpace,\Topology)$ uno spazio topologico.
	Definiamo \Define{chiuso} una classe $\Closed \subseteq
	\TopologicalSpace$ tale che $\Compl{\Closed} \in \Topology$.
\end{Definition}
\begin{Theorem}
	Sia $(\TopologicalSpace,\Topology)$, uno spazio topologico.
	$\emptyset$ e $\TopologicalSpace$ sono aperti e chiusi.
\end{Theorem}
\Proof Basta provare che $\emptyset, \TopologicalSpace \in \Topology$.
Abbiamo $\bigcup_{\Part \in \emptyset} \Part = \emptyset$ e
$\bigcap_{\Part \in \emptyset} \Part = \TopologicalSpace$. \EndProof
\begin{Definition}
	Siano $(\TopologicalSpace_1,\Topology_1)$ e
	$(\TopologicalSpace_2,\Topology_2)$ spazi topologici. Sia inoltre
	$\Continuous: \TopologicalSpace_1 \rightarrow \TopologicalSpace_2$
	un'applicazione qualsiasi.
	Dato un punto $x \in \TopologicalSpace_1$ definiamo
	\Define{intorno} di $x$ una classe $\Neighborhood$ tale che
	\begin{itemize}
		\item $\Neighborhood \subseteq \TopologicalSpace_1$;
		\item $\Exists{\Open \in \Topology}{\And{x \in \Open}{\Open \subseteq
		\Neighborhood}}$.
	\end{itemize}
	Denoteremo $\Neighborhoods{x}[\Topology_1]$ la classe di tutti gli intorni di $x$ rispetto alla topologia $\tau_1$, omettendo la spcifica della topologia ogni qual volta essa sia chiara dal contesto.
	Inoltre, diciamo che $\Continuous$ \`e
	\Define{continua}[continua in un punto][applicazione] quando
	per ogni $\Neighborhood_{\Continuous(x)} \in \Neighborhoods{\Continuous(x)}$ intorno di
	$\Continuous(x)$ esiste $\Neighborhood_x \in \Neighborhoods{x}$ intorno di $x$ tale che
	$\Continuous(\Neighborhood_x) \subseteq \Neighborhood_{\Continuous{x}}$.
\end{Definition}
\begin{Definition}
	Siano
	\begin{itemize}
		\item $(\TopologicalSpace,\Topology)$ uno spazio
		 topologico;
		\item $\Part \subseteq \TopologicalSpace$;
		\item $x \in \TopologicalSpace$.
	\end{itemize}
	Il punto $x$ si dice
	\begin{itemize}
		\item \Define{interno}[interno][punto] a $\Part$ quando
		$\Part$ \`e un intorno di $x$;
		\item \Define{esterno}[esterno][punto] a $\Part$ quando
		$\Compl{\Part}$ \`e un intorno di $x$;
		\item \Define{aderente}[aderente][punto] o
		\Define{di frontiera}[di frontiera][punto] quando
		per ogni intorno $\Neighborhood \in \Neighborhoods{x}$ si verificano
		simultaneamente $\Neighborhood \cap \Part \neq \emptyset$
		e $\Neighborhood \cap \Compl{\Part} \neq \emptyset$.
	\end{itemize}
	Inoltre definiamo
	\begin{itemize}
		\item l'\Define{interno} di $\Part$, denotato
		 $\Interior{\Part}$, la classe di tutti i punti interni
		 di $\Part$;
		\item l'\Define{esterno} di $\Part$, denotato
		 $\Exterior{\Part}$, la classe di tutti i punti esterni
		 di $\Part$;
		\item la \Define{frontiera} o \Define{bordo} o ancora
		\Define{contorno} di $\Part$, denotato
		$\Boundary{\Part}$, la classe di tutti i punti di
		 frontiera di $\Part$.
	\end{itemize}
\end{Definition}
\begin{Theorem}
	Siano
	\begin{itemize}
		\item $(\TopologicalSpace,\Topology)$ uno spazio
		 topologico;
		\item $\Part \subseteq \TopologicalSpace$.
	\end{itemize}
	Le classi non vuote tra $\Interior{\Part}$, $\Exterior{\Part}$ e
	$\Boundary{\Part}$ costituiscono una partizione.
\end{Theorem}
\Proof Sia $x \in \TopologicalSpace$.
\begin{itemize}
	\item se $x \in \Interior{\Part}$, allora $x \in \Part$ per cui
	$\Compl{\Part}$ non pu\`o essere intorno di $x$. Inoltre
	$\Interior{\Part}$ \`e un intorno di $x$ disgiunto da
	$\Compl{\Part}$, dunque $x \notin \Boundary{\Part}$;
	\item se $x \in \Exterior{\Part}$, allora
	 $x \in \Compl{\Part}$ per cui
	$\Part$ non pu\`o essere intorno di $x$. Inoltre
	$\Exterior{\Part}$ \`e un intorno di $x$ disgiunto da
	$\Part$, dunque $x \notin \Boundary{\Part}$;
	\item se $x \in \Boundary{\Part}$, allora n\'e $\Part$ n\'e
	 $\Compl{\Part}$ possono essere intorni di $x$. \EndProof
\end{itemize}
\begin{Theorem}
	Siano
	\begin{itemize}
		\item $(\TopologicalSpace,\Topology)$ uno spazio
		 topologico;
		\item $\Part \subseteq \TopologicalSpace$.
	\end{itemize}
	$\Interior{\Part}$ \`e il pi\`u grande aperto sottoclasse di
	$\Part$.	
\end{Theorem}
\Proof Sia $\Open \in \Topology$ tale che $\Open \subseteq \Part$. Sia
$x \in \Open$. $\Part$ \`e intorno di $x$, dunque $x \in
\Interior{\Part}$. \EndProof
\begin{Definition}
	Siano
	\begin{itemize}
		\item $(\TopologicalSpace,\Topology)$ uno spazio
		 topologico;
		\item $\Part \subseteq \TopologicalSpace$.
	\end{itemize}
	Si definisce \Define{chiusura}	di $\Part$, denotata
	$\Closure{\Part}$, il pi\`u piccolo chiuso tale che
	$\Part \subseteq \Closure{\Part}$.
\end{Definition}
\begin{Theorem}
	Siano $(\TopologicalSpace,\Topology)$ uno spazio topologico e $\Part \subseteq \TopologicalSpace$.
	Abbiamo $\Closure{\Part} = \Part \cup \Boundary{\Part}$;
\end{Theorem}
\Proof Abbiamo $\Compl{\Part \cup \Boundary{\Part}} = \Exterior{\Part} =
\Interior{\Compl{\Part}}$, dunque $\Part \cup \Boundary{\Part}$ \`e un chiuso.
\par Sia $\Closed$ un chiuso tale che $\Part \subseteq \Closed$. Dunque
$\Compl{\Closed}$ \`e un aperto tale che $\Compl{\Closed}
\subseteq \Compl{\Part}$, dunque $\Compl{\Closed} \subseteq
\Interior{\Compl{\Part}} = \Compl{\Part \cup \Boundary{\Part}}$, da cui
$\Part \cup \Boundary{\Part} \subseteq \Closed$. \EndProof

\section{Sottospazi topologici.}\label{SottospaziTopologici}
\begin{Theorem}
	Sia $(\TopologicalSpace,\Topology)$ uno spazio topologico e sia
	$\TopologicalSpace_0 \subseteq \TopologicalSpace$. $\Topology$ induce
	una topologia su $\TopologicalSpace_0$ tramite la famiglia di
	classi $\Topology_0 = \lbrace \Open_0 \subseteq
	\TopologicalSpace_0 | \Exists{\Open \in \Topology}{\Open_0 = \Open
	\cap \TopologicalSpace_0} \rbrace$
\end{Theorem}
\Proof Sia $(\Open^{(i)}_0)_{i \in I}$ una famiglia di elementi di
$\Topology_0$ e, per ogni $i \in I$, sia $\Open^{(i)} \in \Topology$ tale
che $\Open^{(i)}_0 = \Open^{(i)} \cap \TopologicalSpace_0$.
Abbiamo
\begin{itemize}
	\item $\bigcup_{i \in I} \Open^{(i)}_0 =
	\bigcup_{i \in I} (\Open^{(i)} \cap \TopologicalSpace_0) =
	\left ( \bigcup_{i \in I} \Open^{(i)} \right ) \cap
	\TopologicalSpace_0$;
	\item $\bigcap_{i \in I} \Open^{(i)}_0 =
	\bigcap_{i \in I} (\Open^{(i)} \cap \TopologicalSpace_0) =
	\left ( \bigcap_{i \in I} \Open^{(i)} \right ) \cap
	\TopologicalSpace_0$.
\end{itemize}
\par Dunque l'unione arbitraria e l'intersezione finita  di elementi di
$\Topology_0$ restituisce un elemento di $\Topology_0$. \EndProof
\begin{Theorem}
	Sia $(\TopologicalSpace,\Topology)$ uno spazio topologico e sia
	$(\TopologicalSpace_0,\Topology_0)$ un suo sottospazio.
	I chiusi di $\TopologicalSpace_0$ sono tutti e soli i
	$\Closed_0 \subseteq \TopologicalSpace_0$ tali che esista un
	chiuso $\Closed$ in $\TopologicalSpace$ per cui $\Closed_0 =
	\Closed \cap \Topology_0$. Fissato $x \in \TopologicalSpace_0$,
	gli intorni di $x$ per $\Topology_0$ sono tutti e soli i
	$\Neighborhood_{x,\Topology_0} \subseteq \TopologicalSpace_0$  tali che esista
	$\Neighborhood_{x,\TopologicalSpace} \in \Neighborhoods{x}[\TopologicalSpace]$ tale che
	$\Neighborhood_{x,\TopologicalSpace_0} = \Neighborhood_{x,\TopologicalSpace} \cap
	\TopologicalSpace_0$.
\end{Theorem}
\Proof Sia $\Closed_0$ chiuso in $\TopologicalSpace_0$. Allora
$\TopologicalSpace_0 \SetMin \Closed_0$ \`e aperto in
$\TopologicalSpace_0$ e dunque traccia di un aperto $\Open$ di
$\TopologicalSpace$. Abbiamo $
(\TopologicalSpace \SetMin \Open) \cap \TopologicalSpace_0 =
\TopologicalSpace_0 \SetMin (\Open \cap \TopologicalSpace_0) =
\TopologicalSpace_0 \SetMin (\TopologicalSpace_0 \SetMin \Closed_0) =
\Closed_0$. Quindi $\Closed$ \`e traccia di del chiuso $\TopologicalSpace
\SetMin \Open$ di $\TopologicalSpace$.
\par Sia d'altra parte $\Closed$ chiuso in $\TopologicalSpace$. Abbiamo
$\TopologicalSpace_0 \SetMin (\Closed \cap \TopologicalSpace_0) =
\TopologicalSpace \SetMin (\Closed \cap \TopologicalSpace_0) =
(\TopologicalSpace \SetMin \Closed) \cap \TopologicalSpace_0$, dunque la
traccia di $\Closed$ in $\TopologicalSpace_0$ \`e chiusa in
$\TopologicalSpace_0$.
\par Sia $\Neighborhood_{x,\Topology_0}$ intorno di $x$ in
$\TopologicalSpace_0$. Allora esiste $\Open_0$ aperto di
$\TopologicalSpace_0$ tale che $x \in \Open_0$ e $\Open_0 \subseteq
\Neighborhood_{x,\Topology}{0}$. Sia $\Open$ aperto di $\TopologicalSpace$ tale che
$\Open_0 = \Open \cap \TopologicalSpace_0$.
$\Open \cup \Neighborhood_{x,\Topology_0}$ \`e un intorno di $x$ in
$\TopologicalSpace$ e $\Neighborhood_{x,\Topology_0}$ ne \`e chiaramente la
traccia su $\TopologicalSpace_0$.
\par Sia infine $\Neighborhood_{x,\Topology}$ un intorno di $x$ in
$\TopologicalSpace$. Allora esiste $\Open \in \Topology$ tale che $x \in
\Open$ e $\Open \subseteq \Neighborhood_{x,\Topology}$. Chiaramente la traccia di
$\Neighborhood_{x,\Topology}$ su $\TopologicalSpace_0$ contiene l'aperto $\Open \cap
\TopologicalSpace_0$ di $\TopologicalSpace_0$ e $x \in \Open \cap
\TopologicalSpace_0$. \EndProof

\section{Basi e ricoprimenti.}\label{BasiERicoprimenti}
\begin{Definition}
	Sia $(\TopologicalSpace,\Topology)$ uno spazio e sia, per
	un'opportuna classe $I$, $(\TopologicalBase_i)_{i \in I}$ una
	famiglia di sottoclassi di $\TopologicalSpace$.
	$(\TopologicalBase_i)_{i \in I}$ si dice
	\begin{itemize}
		\item \Define{ricoprimento} di $\TopologicalSpace$ quando
		$\bigcup_{i \in I} \TopologicalBase_i =
		\TopologicalSpace$;
		\item \Define{ricoprimento aperto}[aperto][ricoprimento]
		quando $(\TopologicalBase_i)_{i \in I}$ \`e un
		ricoprimento e, per ogni $i \in I$, $\TopologicalBase_i$
		\`e aperto;
		\item \Define{ricoprimento chiuso}[chiuso][ricoprimento]
		quando $(\TopologicalBase_i)_{i \in I}$ \`e un
		ricoprimento e, per ogni $i \in I$, $\TopologicalBase_i$
		\`e chiuso;
		\item \Define{base}[di uno spazio topologico][base] quando
		quando $(\TopologicalBase_i)_{i \in I}$ \`e un
		ricoprimento aperto e ogni aperto di $\Topology$ \`e
		unione di elementi di $(\TopologicalBase_i)_{i \in I}$.
	\end{itemize} 
\end{Definition}
\begin{Theorem}\label{th_BaseDiUnaTopologia}
	Sia $\TopologicalSpace$ una classe e sia
	$(\TopologicalBase_i)_{i \in I}$ una famiglia di parti di
	$\TopologicalSpace$ tali che
	\begin{itemize}
		\item $\bigcup_{i \in I} \TopologicalBase_i =
		\TopologicalSpace$;
		\item fissati $i, j \in I$, $\Implies{x \in
		\TopologicalBase_i \ cap \TopologicalBase_j}{\Exists{k \in
		I}{\And{x \in \TopologicalBase_k}{\TopologicalBase_k
		\subseteq \TopologicalBase_i \cap \TopologicalBase_j}}}$.
	\end{itemize}
	Allora $\TopologicalBase_i$ \`e base di una topologia per
	$\TopologicalSpace$.
\end{Theorem}
\Proof La topologia $\Topology$ in questione non pu\`o che essere quella i
cui elementi sono tutte e sole le unioni di elementi di
$(\TopologicalBase_i)_{i \in I}$: proviamo che si tratta effettivamente di
una topologia.
\par A tal fine basta provare che l'intersezione di due unioni di elementi
 di $(\TopologicalBase_i)_{i \in I}$ \`e ancora unione di elementi di
$(\TopologicalBase_i)_{i \in I}$. Abbiamo
$\left (\bigcup_{i \in I_1} \TopologicalBase_i \right ) \cap
\left (\bigcup_{i \in I_2} \TopologicalBase_i \right ) =
\bigcup_{(i,j) \in I_1 \times I_2}
(\TopologicalBase_i \cap \TopologicalBase_j) =
\bigcup_{(i,j) \in I_1 \times I_2}
\bigcup_{x \in (\TopologicalBase_i \cap \TopologicalBase_j)}
\TopologicalBase_{x,i,j}$, dove $\TopologicalBase_{x,i,j}$ \`e un elemento
di $(\TopologicalBase_i)_{i \in I}$ tale che
$x \in \TopologicalBase_{x,i,j}$ e $\TopologicalBase_{x,i,j} \subseteq
(\TopologicalBase_i \cap \TopologicalBase_j)$. \EndProof
\begin{Definition}
	Sia $(\TopologicalSpace,\Topology)$ uno spazio e sia
	$x \in \TopologicalSpace$. Una famiglia di intorni
	$(\Neighborhood_{x,i})_{i \in I} \in \Neighborhoods{x}^{I}$ si chiama
	\Define{base locale}[locale][base] o
	\Define{sistema fondamentale di intorni}[fondamentale di intorni][sistema]
	quando per ogni intorno $\Neighborhood_x \in \Neighborhoods{x}$ di $x$ esiste $i \in I$
	tale che $\Neighborhood_{x,i} \subseteq
	\Neighborhood_x$.
\end{Definition}

\section{Assiomi di separazione.}\label{AssiomiDiSeparazione}
\begin{Axiom}
	\TheoremName{Assiomi di separazione}
	Sia $(\TopologicalSpace,\Topology)$ uno spazio topologico.
	Si dice che $\TopologicalSpace$ \`e
	\begin{itemize}
		\item \Define{$T_0$}[$T_0$][spazio]
		quando per ogni coppia $(x_1,x_2) \in
		\TopologicalSpace$ esiste un aperto $\Open$ tale che
		$\And{x_1 \in \Open}{x_2 \notin \Open}$ o
		$\And{x_2 \in \Open}{x_1 \notin \Open}$; in questo
		caso $\TopologicalSpace$ si chiama
		\Define{spazio di Kolmogorov}[di Kolmogorov][spazio];
		\item \Define{$T_1$}[$T_1$][spazio]
		quando per ogni coppia $(x_1,x_2) \in
		\TopologicalSpace$ esiste due aperti $\Open_1$ e $\Open_2$
		tali che
		$\And{x_1 \in \Open_1}{x_2 \notin \Open_1}$ e
		$\And{x_2 \in \Open_2}{x_1 \notin \Open_2}$;
		\item \Define{$T_2$}[$T_2$][spazio]
		quando per ogni coppia $(x_1,x_2) \in
		\TopologicalSpace$ esiste due aperti disgiunti $\Open_1$ e
		$\Open_2$ tali che
		$\And{x_1 \in \Open_1}{x_2 \notin \Open_1}$ e
		$\And{x_2 \in \Open_2}{x_1 \notin \Open_2}$; in questo
		caso $\TopologicalSpace$ si chiama
		\Define{spazio di Hausdorff}[di Hausdorff][spazio];
		\item \Define{$T_3$}[$T_3$][spazio] quando per ogni
		$x \in \TopologicalSpace$ e chiuso $\Closed$ tale che
		$x \notin \Closed$ esistono due aperti disgiunti $\Open_1$
		e $\Open_2$ tali che $x \in \Open_1$ e $\Closed
		\subseteq \Open_2$;
		\item \Define{$T_4$}[$T_4$][spazio] quando per ogni
		dati due chiusi disgiunti $\Closed_1$ e $\Closed_2$ tale
		che $x \notin \Closed$ esistono due aperti disgiunti
		$\Open_1$ e $\Open_2$ tali che $\Closed_1 \subseteq
		\Open_1$ e $\Closed_2 \subseteq \Open_2$.
	\end{itemize}
\end{Axiom}
\begin{Definition}
	Sia $(\TopologicalSpace,\Topology)$ uno spazio topologico.
	$\TopologicalSpace$ si dice \Define{regolare}[regolare][spazio] se
	\`e $T_1$ e $T_3$.
\end{Definition}

\section{Assiomi di numerabilita.}\label{AssiomiDiNumerabilita}
\begin{Axiom}
	\TheoremName{Primo assioma di numerabilit\`a}
	Sia $(\TopologicalSpace,\Topology)$ uno spazio topologico.
	$\TopologicalSpace$ verifica il primo assioma di numerabilit\`a
	se ogni punto di $\TopologicalSpace$ ammette una base locale
	numerabile.
\end{Axiom}
\begin{Axiom}
	\TheoremName{Secondo assioma di numerabilit\`a}
	Sia $(\TopologicalSpace,\Topology)$ uno spazio topologico.
	$\TopologicalSpace$ verifica il secondo assioma di numerabilit\`a
	se ammette una base numerabile.
\end{Axiom}
\begin{Definition}
	Sia $(\TopologicalSpace,\Topology)$ uno spazio topologico.
	$\Class \subseteq \TopologicalSpace$ si dice
	\Define{denso}[densa][classe] quando interseca ogni aperto non
	vuoto.
\end{Definition}
\begin{Definition}
	Sia $(\TopologicalSpace,\Topology)$ uno spazio topologico.
	$\TopologicalSpace$ si dice
	\Define{separabile}[separabile][spazio] se esiste
	$\Class \subseteq \TopologicalBase$ denso e numerabile.
\end{Definition}
\begin{Theorem}
	Sia $(\TopologicalSpace,\Topology)$ uno spazio topologico e sia $x \in \TopologicalSpace$. Se $x$ ammette una base locale numerabile $(\LocalBase_n)_{n \in \mathbb{N}}$, allora $x$ ammette anche una base locale numerabile non crescente.
\end{Theorem}
\Proof Una base locale non crescente \`e data da $(\VarLocalBase_n)_{n \in \mathbb{N}}$, con $\VarLocalBase_n = \bigcap_{i \in n} \LocalBase_i$. \EndProof

\section{Successioni e limiti.}
\label{Topologia_SuccessioniELimiti}
\begin{Definition}
	Sia $(\TopologicalSpace,\Topology)$ uno spazio topologico.
	La successione $(a_n)_{n \in \mathbb{N}}$
	\Define{converge}[convergente][successione] a un punto
	$p \in \TopologicalSpace$
	quando per ogni intorno $\Neighborhood_p \in \Neighborhoods{p}$ esiste
	$N \in \mathbb{N}$ tale che $\Implies{n \geq N}{
	a_n \in \Neighborhood_p}$.
	Se $(a_n)_{n \in \mathbb{N}}$ converge a un punto, allora $(a_n)_{n \in \mathbb{N}}$ si dice \Define{convergente}[convergente][successione].
\end{Definition}
\begin{Definition}
	Sia $(\TopologicalSpace,\Topology)$ uno spazio topologico
	e sia $(a_n)_{n \in \mathbb{N}}$ una successione
	convergente a un unico punto $p \in \TopologicalSpace$. Diciamo
	che $p$ \`e il
	\Define{limite}[di una successione][limite] della successione,
	denotato $\lim_{n \rightarrow \infty} a_n = p$.
	Diciamo anche che $(a_n)_{n \in \mathbb{N}}$ tende a $p$, in simboli
	$a_n \rightarrow p$, per $n$ che tende a infinito,
	in simboli $n \rightarrow \infty$.
\end{Definition}
\begin{Theorem}
	Sia $(\TopologicalSpace,\Topology)$ uno spazio topologico e
  $(a_n)_{n \in \mathbb{N}} \in \TopologicalSpace^\mathbb{N}$ una succesione
  che converge a un punto $p$. Ogni sottosuccessione
  $(a_{f(n)})_{n \in \mathbb{N}}$,
  con $f: \mathbb{N} \rightarrow \mathbb{N}$ applicazione crescente,
  converge anch'essa a $p$.
\end{Theorem}
\Proof Sia $\Neighborhood_p \in \Neighborhoods{p}$. Esiste $N \in \mathbb{N}$ tale che $\Implies{n > N}{a_n \in \Neighborhood_p}$. A maggior ragione, $\Implies{n > N}{a_{f(n)} \in \Neighborhood_p}$ e dunque $(a_{f(n)})_{n \in \mathbb{N}}$ converge a $p$. \EndProof
\begin{Theorem}
	Sia $(\TopologicalSpace,\Topology)$ uno spazio topologico di
	Hausdorff e sia $(\SequenceElement_i)_{i \in \mathbb{N}}$ una
	successione convergente a due punti $p_1, p_2 \in
	\TopologicalSpace$. Abbiamo $p_1 = p_2$.
\end{Theorem}
\Proof Supponiamo $p_1 \neq p_2$. Allora esistono due intorni aperti
disgiunti $\Open_1, \Open_2 \in \Topology$ tali che $p_1 \in \Open_1$ e
$p_2 \in \Open_2$. Ma allora, per $n \in \mathbb{N}$ grande a sufficienza,
abbiamo $\Implies{m \geq n}{a_m \in \Open_1 \cap \Open_2}$, ma questo \`e
assurdo. Dunque deve essere $p_1 = p_2$. \EndProof
\begin{Definition}
	Sia $(\TopologicalSpace,\Topology)$ uno spazio topologico.
	Siano $\Part \subseteq \TopologicalSpace$ e $(a_n)_{n \in \mathbb{N}} \in \TopologicalSpace^\mathbb{N}$.
	$\AccumulationPoint \in \TopologicalSpace$ \`e
	\begin{itemize}
		\item \Define{punto di accumulazione}[di accumulazione per una parte][punto] di $\Part$ quando $\ForAll{\Neighborhood \in \Neighborhoods{p}}{\Neighborhood \cap \Part \neq \emptyset}$.
		\item \Define{punto di accumulazione}[di accumulazione per una successione][punto] di $(a_n)_{n \in \mathbb{N}}$ quando $\ForAll{(\Neighborhood,n) \in \Neighborhoods{p} \times\mathbb{N}}{\Exists{m \geq n}{a_m \in \Neighborhood}}$.
	\end{itemize}
\end{Definition}
\begin{Theorem}
	Sia $(\TopologicalSpace,\Topology)$ uno spazio topologico.
	Se $\AccumulationPoint$ \`e punto di accumulazione per $(a_n)_{n \in \mathbb{N}} \in \TopologicalSpace^\mathbb{N}$, allora \`e punto di accumulazione anche per $\bigcup_{n \in \mathbb{N}} \lbrace a_n \rbrace$.
\end{Theorem}
\Proof Segue direttamente dalle definizioni. \EndProof
\begin{Theorem}
	Sia $(\TopologicalSpace,\Topology)$ uno spazio topologico.
	Se $(a_n)_{n \in \mathbb{N}} \in \TopologicalSpace^\mathbb{N}$ converge a $\AccumulationPoint \in \TopologicalSpace$, allora \`e punto di accumulazione per $(a_n)_{n \in \mathbb{N}}$.
\end{Theorem}
\Proof Segue direttamente dalle definizioni. \EndProof
\begin{Theorem}
	Sia $(\TopologicalSpace,\Topology)$ uno spazio topologico e sia $\Part \subseteq \TopologicalSpace$. $\Closure{\Part}$ contiene tutti e soli i suoi punti di accumulazioni.
\end{Theorem}
\par Supponiamo $x \in \Closure{\Part}$. Allora $\Or{x \in \Part}{x \in \Boundary{\Part}}$: se $x \in \Part$, allora ogni intorno di $x$ contiene $x \in \Part$ ed \`e dunque non disgiunto da $\Part$; se invece $x \in \Boundary{\Part}$, allora per definizione di bordo ogni intorno di $x$ interseca $\Part$.
\par Supponiamo $\AccumulationPoint \in \TopologicalSpace$ di accumulazione per $\Part$. Allora $\ForAll{\Neighborhood \in \Neighborhoods{x}}{\Neighborhood \cap \Part \neq \emptyset}$. e dunque $x \in \Compl{\Exterior{\Part}} = \Part \cup \Boundary{\Part} = \Closure{\Part}$.  \EndProof
\begin{Theorem}
	Siano $(\TopologicalSpace,\Topology)$ uno spazio topologico che soddisfa il primo assioma di numerabilit\`a e $\Part \subseteq \TopologicalSpace$. Se \`e vero l'assioma della scelta, allora $\AccumulationPoint \in \TopologicalSpace$ \`e di accumulazione per $\Part$ se e solo se esiste $(a_n)_{n \in \mathbb{N}} \in \Part^\mathbb{N}$ convergente a $\AccumulationPoint$.
\end{Theorem}
\Proof Sia $\AccumulationPoint \in \TopologicalSpace$ di accumulazione per $\Part$ e sia $(\LocalBase_n)_{n \in \mathbb{N}}$ una base locale decrescente di $\AccumulationPoint$. Per ogni $n \in \mathbb{N}$, sia $a_n \in \LocalBase_n$: $(a_n)_{n \in \mathbb{N}}$ converge a $\AccumulationPoint$.
\par Segue immediatamente dalle definizioni che se $\AccumulationPoint$ \`e tale che $(a_n)_{n \in \mathbb{N}} \in \Part^\mathbb{N}$ converga a $\AccumulationPoint$, allora $\AccumulationPoint$ \`e di accumulazione per $(a_n)_{n \in \mathbb{N}}$ e a maggior ragione per $\Part$. \EndProof
\begin{Definition}
	Siano
	\begin{itemize}
		\item $(\TopologicalSpace,\Topology)$ e $(\VarTopologicalSpace,\VarTopology)$ spazi topologici;
		\item $\Part \subseteq \TopologicalSpace$;
		\item $f: \Part \rightarrow \VarTopologicalSpace$;
		\item $\AccumulationPoint$ punto di accumulazione per $\Part$;
		\item $\Limit \in \VarTopologicalSpace$.
	\end{itemize}
	Se $\ForAll{\Neighborhood_\Limit \in \Neighborhoods{\Limit} \cap \VarTopology}{\Exists{\Neighborhood_\AccumulationPoint \in \Neighborhoods{\AccumulationPoint} \cap \Topology}{f(\Neighborhood_\AccumulationPoint \cap \Part) \subseteq \Neighborhood_\Limit}}$, allora diciamo che $f(x)$ \Define{converge}[di una funzione in un punto][convergenza] a $\Limit$, in simboli $f(x) \rightarrow \Limit$, per $x$ che tende a $\AccumulationPoint$, in simboli $x \rightarrow \AccumulationPoint$. Se $L$ \`e l'unico punto a cui converge $f(x)$ per $x \rightarrow \AccumulationPoint$, allora chiamiamo $\Limit$ \Define{limite}[di una funzione in un punto][limite] di $f(x)$ per $x \rightarrow \AccumulationPoint$, denotato $\lim_{x \rightarrow \AccumulationPoint} f(x) = \Limit$.
\end{Definition}
\begin{Theorem}
	Siano
	\begin{itemize}
		\item $(\TopologicalSpace,\Topology)$ spazio topologico che verifica il primo assioma di numerabilit\`a;
		\item $(\VarTopologicalSpace,\VarTopology)$ spazio topologico;
		\item una successione $(a_n)_{n \in \mathbb{N}} \in \TopologicalSpace^\mathbb{N}$ convergente a $\AccumulationPoint \in \TopologicalSpace$;
		\item $f: \bigcup_{n \in \mathbb{N}} \lbrace a_n \rbrace \rightarrow \VarTopologicalSpace$.
	\end{itemize}
	Se \`e vero l'assioma della scelta, $(f(a_n))_{n \in \mathbb{N}}$ converge a $\Limit$ se e solo se $f(x)$ converge a $\Limit$ per $x \rightarrow \AccumulationPoint$.
\end{Theorem}
\Proof Per prima cosa osserviamo che $\AccumulationPoint$ \`e di accumulazione per il dominio di $f$.
\par Supponiamo $f(x) \rightarrow \Limit$ per $x \rightarrow \AccumulationPoint$. Sia $\Neighborhood_\Limit \in \Neighborhoods{\Limit} \cap \VarTopology$: per opportuno $\Neighborhood_\AccumulationPoint \in \Neighborhoods{\AccumulationPoint} \cap \Topology$, abbiamo $f \left (\Neighborhood_\AccumulationPoint \cap \bigcup_{n \in \mathbb{N}} \lbrace a_n \rbrace \right ) \subseteq \Neighborhood_\Limit$. Ma $a_n \rightarrow \AccumulationPoint$ per $n \rightarrow \infty$ e quindi esiste $N \in \mathbb{N}$ tale che $\Implies{n > N}{a_n \in \Neighborhood_\AccumulationPoint}$, da cui $\Implies{n > N}{f(a_n) \in \Neighborhood_\Limit}$. Ne deduciamo $f(a_n) \rightarrow \Limit$ per $n \rightarrow \infty$.
\par Supponiamo ora $\lim_{n \rightarrow \infty} f(a_n) = \Limit$. Supponiamo inoltre che $f(x)$ non converga a $\Limit$ per $x \rightarrow \AccumulationPoint$. Allora esiste $\Neighborhood_\Limit \in \Neighborhoods{\Limit} \cap \VarTopology$ tale che $\ForAll{\Neighborhood_\AccumulationPoint \in \Neighborhoods{\AccumulationPoint} \cap \Topology}{f \left ( \Neighborhood_\AccumulationPoint \cap \bigcup_{m \in \mathbb{N}} \lbrace a_m \rbrace \right ) \cap \Compl{\Neighborhood_\Limit} \neq \emptyset}$. Sia $(\LocalBase_n)_{n \in \mathbb{N}}$ un sistema fondamentale di intorni aperti di $\AccumulationPoint$. Per ogni $n \in \mathbb{N}$, abbiamo dunque
$f\left (\LocalBase_n \cap \bigcup_{m \in \mathbb{N}} \lbrace a_m \rbrace \right ) \cap \Compl{\Neighborhood_\Limit} \neq \emptyset$, da cui $\LocalBase_n \cap \bigcup_{m \in \mathbb{N}} \lbrace a_m \rbrace \cap f^{-1}(\Compl{\Neighborhood_\Limit}) \neq \emptyset$.
\par Quindi $\ForAll{n \in \mathbb{N}}{\Exists{m \in \mathbb{N}}{a_m \in \LocalBase_n \cap f^{-1}(\Compl{\Neighborhood_\Limit})}}$. D'altra parte, lo stesso ragionamento pu\`o essere ripetuto per ogni sottosuccesione di $(a_n)_{n \in \mathbb{N}}$ del tipo $(a_{n + k})_{n \in \mathbb{N}}$ per qualsiasi $k \in \mathbb{N}$ e sempre con lo stesso intorno $\Neighborhood_\Limit$ di $\Limit$ e quindi $\ForAll{n \in \mathbb{N}}{\ForAll{N \in \mathbb{N}}{\Exists{m \in \mathbb{N} \SetMin N}{a_m \in \LocalBase_n \cap f^{-1}(\Compl{\Neighborhood_\Limit})}}}$.
\par Possiamo quindi definire la sottosuccessione $(a_{g(n)})_{n \in \mathbb{N}} \in \TopologicalSpace^{\mathbb{N}}$, con $g: \mathbb{N} \rightarrow \mathbb{N}$ applicazione strettamente crescente, tale che $\ForAll{n \in \mathbb{N}}{a_{g(n)} \in \LocalBase_n \cap f^{-1}(\Compl{\Neighborhood_\Limit})}$. Allora
\begin{itemize}
	\item $(f(a_{g(n)}))_{n \in \mathbb{N}}$ converge a $\Limit$ in quanto sottosuccessione di $(f(a_n))_{n \in \mathbb{N}}$;
	\item $(f(a_{g(n)}))_{n \in \mathbb{N}}$ non converge a $\Limit$ perch\'e $\ForAll{n \in \mathbb{N}}{f(a_{g(n)}) \notin \Neighborhood_\Limit}$.
\end{itemize}
Ma questo \`e assurdo, dunque $f(x)$ deve converge a $\Limit$ per $x \rightarrow \AccumulationPoint$. \EndProof

\section{Spazi vettoriali topologici}
\label{Topologia_SpaziVettorialiTopologici}
\begin{Definition}
	Sia $\LinearSpace$ un $\Field$-spazio vettoriale e $\Topology$ una topologia su $\LinearSpace$ tale che
	\begin{itemize}
		\item la somma di vettori $+: \LinearSpace^2 \rightarrow \LinearSpace$ sia continua;
		\item il prodotto per scalare $\cdot: \Field \times \LinearSpace \rightarrow \LinearSpace$ sia continuo.
	\end{itemize}
	Chiamiamo $(\LinearSpace,\Topology)$ \Define{spazio vettoriale topologico}[vettoriale topologico][spazio].
\end{Definition}
\begin{Definition}
	Sia $(\LinearSpace,\Topology)$ uno spazio vettoriale topologico.
	Una parte $\Part \subseteq \LinearSpace$ si dice \Define{limitata}[limitata][parte] se per ogni intorno $\Neighborhood \in \Neighborhoods{0}$ esiste $\Scalar \in \Field$ tale che $\Part \subseteq \Scalar \Neighborhood$. Un'applicazione a valori in $\LinearSpace$ si dice \Define{limitata}[limitata][applicazione] se \`e limitata la sua immagine.
\end{Definition}

\section{Spazi metrici.}\label{SpaziMetrici}
\begin{Definition}
	Sia $\MetricSpace$ una classe e $\Distance:
\MetricSpace^2 \rightarrow \mathbb{R}^+$ un'applicazione
tale che:
	\begin{itemize}
		\item $\ForAll{(x,y) \in \MetricSpace^2}{
		\Coimplies{\Distance(x,y) = 0}{x = y}}$;
		\item $\ForAll{(x,y) \in \MetricSpace^2}{
		\Distance(x,y) = \Distance(y,x)}$ (simmetria);
		\item $\ForAll{(x,y,z) \in \MetricSpace^3}{
		\Distance(x,y) \leq \Distance(x,z) + \Distance(z,y)}$
		(disuguaglianza triangolare).
	\end{itemize}
	$\Distance$ si chiama \Define{distanza} o \Define{metrica} su
	$\MetricSpace$ e la coppia $(\MetricSpace,\Distance)$ si chiama
	\Define{spazio metrico}[metrico][spazio].
\end{Definition}
\begin{Definition}
	Sia $(\MetricSpace,\Distance)$ uno spazio metrico. Dati $c \in
	\MetricSpace$ e $r \in \mathbb{R}^+$ si definiscono
	\begin{itemize}
		\item \Define{palla aperta}[aperta][palla] centrata di $x$ di raggio $r$
	la classe $\OpenBall{c}{r} = \lbrace x \in \MetricSpace |
	\Distance(x,c) < r \rbrace$;
		\item \Define{palla chiusa}[chiusa][palla] centrata di $x$ di raggio $r$
	la classe $\ClosedBall{c}{r} = \lbrace x \in \MetricSpace |
	\Distance(x,c) \leq r \rbrace$.
	\end{itemize}
\end{Definition}
\begin{Theorem}
	Sia $(\MetricSpace,\Distance)$ uno spazio metrico.
	Le palle aperte di $\MetricSpace$ costituiscono una base per una
	topologia su $\MetricSpace$
\end{Theorem}
\Proof Sia $c \in \MetricSpace$. Abbiamo $\MetricSpace =
\bigcup_{r \in \mathbb{R}^+} \OpenBall{c}{r}$.
\par Siano $c_1, c_2 \in \MetricSpace$, $r_1, r_2 \in \mathbb{R}^+$ e sia
$c \in \OpenBall{c_1}{r_1} \cap \OpenBall{c_2}{r_2}$. Scegliamo
$r < \min{r_1 - \Distance(c,c_1),r_2 - \Distance(c,c_2)}$.
Se $x \in \OpenBall{c}{r}$, allora
\begin{itemize}
	\item $\Distance(x,c_1) \leq \Distance(x,c) + \Distance(c,c_1)
	\leq r + \Distance(c,c_1)
	< r_1 - \Distance(c,c_1) + \Distance(c,c_1)
	< r_1$;
	\item $\Distance(x,c_2) \leq \Distance(x,c) + \Distance(c,c_2)
	\leq r + \Distance(c,c_2)
	< r_2 - \Distance(c,c_2) + \Distance(c,c_2)
	< r_2$.
\end{itemize}
Quindi, $\OpenBall{c}{r} \subseteq
\OpenBall{c_1}{r_1} \cap \OpenBall{c_2}{r_2}$ e per il teorema
\ref{th_BaseDiUnaTopologia}, le palle aperte costituiscono una base per
una topologia su $\MetricSpace$. \EndProof
\begin{Definition}
	Sia $(\MetricSpace,\Distance)$ uno spazio metrico.
	La topologia di $\MetricSpace$ per cui le palle aperte
	costituiscono una base si chiama
	\Define{topologia indotta}[indotta da metrica][topolgia] da
	$\Distance$. Se la distanza $\Distance$ \`e la distanza euclidea,
	allora la topolgia indotta si chiama anche
	\Define{topologia euclidea}[euclidea][topologia].
\end{Definition}
\par Salvo avvisi contrari, quando considereremo una topologia definita su uno spazio metrico si tratter\`a sempre delle topologia indotta dalla metrica del medesimo spazio. 
\begin{Definition}
  Due norme su uno stesso spazio vettoriale si dicono
  \Define{equivalenti}[equivalenti][norme]
  quando le distanze da esse indotte inducono la stessa topologia.
\end{Definition}
\begin{Theorem}
	Sia $(\MetricSpace,\Distance)$ uno spazio metrico.
	Sia $x \in \MetricSpace$. La famiglia di palle aperte $(\OpenBall{x}{r})_{r > 0}$ costituisce una base locale per $x$.
\end{Theorem}
\Proof Segue immediatamente dalle definizioni. \EndProof
\begin{Theorem}
	Sia $(\MetricSpace,\Distance)$ uno spazio metrico.
	$\MetricSpace$ \`e uno spazio di Hausdorff.
\end{Theorem}
\Proof Per ogni coppia di punti di $(x,y) \in \MetricSpace^2$, \`e sufficiente considerare gli aperti $\OpenBall{x}{r}$ e $\OpenBall{y}{r}$, con $r < \Distance{x}{y}$. \EndProof
\begin{Definition}
	Sia $(\MetricSpace,\Distance)$ uno spazio metrico e sia $(a_i)_{i \in \mathbb{N}} \in \MetricSpace^\mathbb{N}$. $(a_i)_{i \in \mathbb{N}}$ si chiama \Define{successione di Cauchy}[di Cauchy][successione] o \Define{successione fondamentale}[fondamentale][successione] quando $\ForAll{\epsilon > 0}{\Exists{N \in \mathbb{N}}{\Implies{\And{n > N}{m > N}}{\Distance(a_n,a_m) < \epsilon}}}$.
\end{Definition}
\begin{Theorem}
	Sia $(\MetricSpace,\Distance)$ uno spazio metrico.
	Sia $(a_i)_{i \in \mathbb{N}} \in \MetricSpace^\mathbb{N}$ una successione \`e convergente: $(a_i)_{i \in \mathbb{N}}$ \`e di Cauchy.
\end{Theorem}
\Proof Fissiamo $\epsilon > 0$ e poniamo $L = \lim_{i \rightarrow \infty} a_i$. Sia $N \in \mathbb{N}$ tale che $\Implies{i > N}{a_i \in \OpenBall{L}{\frac{\epsilon}{2}}}$.
\par Siano $n, m > N$. Abbiamo $\Distance{a_n}{a_m} \leq \Distance{a_n}{L} + \Distance{a_m}{L} < \frac{\epsilon}{2} + \frac{\epsilon}{2} = \epsilon$. \EndProof
\begin{Definition}
	Sia $(\MetricSpace,\Distance)$ uno spazio metrico.
	Se ogni successione di Cauchy in $\MetricSpace$ \`e convergente, allora $\MetricSpace$ si dice \Define{completo}[metrico completo][spazio].
\end{Definition}
\begin{Theorem}
	Sia $(\MetricSpace,\Distance)$ uno spazio metrico.
	Sia $(a_i)_{i \in \mathbb{N}} \in \MetricSpace^\mathbb{N}$ \`e una successione convergente a $L \in \MetricSpace$ se e solo se $\ForAll{\epsilon > 0}{\Exists{N \in \mathbb{N}}{\Implies{i > N}{\Distance{a_i}{L} < \epsilon}}}$.
\end{Theorem}
\Proof Se $(a_i)_{i \in \mathbb{N}}$ converge a $L$, allora per definizione di convergenza $\ForAll{\epsilon > 0}{\Exists{N \in \mathbb{N}}{\Implies{i > N}{a_i \in \OpenBall{L}{\epsilon}}}}$ e $a_i \in \OpenBall{L}{\epsilon}$ equivale a $\Distance{a_i}{L} < \epsilon$.
\par Viceversa, assumiamo $\ForAll{\epsilon > 0}{\Exists{N \in \mathbb{N}}{\Implies{i > N}{\Distance{a_i}{L} < \epsilon}}}$. Sia $\Neighborhood \in \Neighborhoods{L}$: poich\'e $(\OpenBall{L}{r})_{r \in \mathbb{R}^+}$ costituisce una famiglia fondamentale di intorni di per $L$, esiste $\epsilon > 0$ tale che abbiamo $\OpenBall{L}{\epsilon} \subseteq \Neighborhood$: scegliendo $N \in \mathbb{N}$ tale che $\Implies{i > N}{\Distance{a_i}{L} < \epsilon}$, abbiamo, per ogni $i > n$, $a_i \in \OpenBall{a_i}{\epsilon} \subseteq \Neighborhood$. \EndProof
\begin{Definition}
	Sia $(\MetricSpace,\Distance)$ uno spazio metrico.
	Una parte $\Part \subseteq \MetricSpace$ si dice \Define{limitata}[limitata][parte] se \`e contenuta in una palla chiusa di $\MetricSpace$.
\end{Definition}
\begin{Theorem}
	Sia $(\MetricSpace,\Distance)$ uno spazio metrico.
	Sia $(a_i)_{i \in \mathbb{N}} \in \MetricSpace^\mathbb{N}$ una successione convergente. $(a_i)_{i \in \mathbb{N}}$ \`e limitata.
\end{Theorem}
\Proof Fissiamo $\epsilon > 0$. Poniamo $L = \lim_{i \rightarrow \infty} a_i$.
\par Siano $N \in \mathbb{N}$ tale che $\Implies{i > N}{\Distance{a_i}{L} < \epsilon}$ e $m = \max_{i \leq N} \Distance{a_i}{L}$. Poniamo $M = \max \lbrace \epsilon, m \rbrace$: la successione $(a_i)_{i \in \mathbb{N}}$ \`e interamente contenuta in $\ClosedBall{L}{M}$. \EndProof
\begin{Definition}
	Siano $(\MetricSpace,\Distance)$ e
  $(\VarMetricSpace,\Distance')$ due spazi metrici e
  $f: \MetricSpace \rightarrow \VarMetricSpace$.
  Se \`e verifcata la proposizione
  \[
    \ForAll{\epsilon > 0}
    {\Exists{\delta > 0}
    {\ForAll{(x_1,x_2) \in \MetricSpace^2}
    {\Implies{\Distance(x_1,x_2) < \delta}
    {\Distance'(f(x_1),f(x_2)) < \epsilon}}}},
  \]
  allora diciamo che $f$ \`e
  \Define{uniformemente continua}[uniformemente continua][applicazione].
\end{Definition}
\begin{Theorem}
	Con le notazioni del teorema precedente, se $f$ \`e uniformemente continua, allora $f$ \`e anche continua.
\end{Theorem}
\Proof Siano $x_0 \in \MetricSpace$ e $\epsilon > 0$. Per l'uniforme continuit\`a abbiamo $\Exists{\delta > 0}{\ForAll{x \in \Dom{f}}{\Implies{\Distance(x,x_0) < \delta}{\Distance(f(x),f(x_0)) < \epsilon}}}$ che equivale alla continuit\`a di $f$ in $x_0$. \EndProof
\begin{Definition}
	Siano $(\MetricSpace,\Distance)$ e $(\VarMetricSpace,\Distance)$ due spazi metrici e $f: \MetricSpace \rightarrow \VarMetricSpace$. Se esiste $L > 0$ tale che $\ForAll{(x,y) \in \MetricSpace}{\Distance(f(x),f(y)) \leq L \Distance(x,y)}$, allo si dice che $f$ verifica la \Define{condizione di Lipschitz}[di Lipschitz][condizione], che \`e \Define{lipschitziana}[lipschitziana][funzione] o ancora che \`e \Define{$L$-Lipschitz continua}[$L$-lipschitz continua][funzione].
\end{Definition}
\begin{Theorem}
	Con le notazioni del teorema precedente, se $f$ \`e lipschitziana, allora \`e anche uniformemente continua.
\end{Theorem}
\Proof Sia $\epsilon > 0$. Per ogni $(x,y) \in \MetricSpace^2$ tale che $\Distance(x,y) < L^{-1} \epsilon$ abbiamo $\Distance(f(x),f(y)) \leq L \Distance(x,y) < \epsilon$. \EndProof
\begin{Theorem}
  Sia $\NormedSpace$ un $\Field$-spazio vettoriale normato: la sua norma
  $\Norm{\cdot}$ \`e un'applicazione uniformente continua.
\end{Theorem}
\Proof Fissiamo $\epsilon > 0$ e poniamo $\delta = \epsilon$.
\par Siano $x, y \in \NormedSpace$ tali che
$\Norm{x - y} < \epsilon$.
Abbiamo, per la disugualianza triangolare,
$\Norm{x} - \Norm{y} \leq \Norm{x - y} < \epsilon$. \EndProof

\section{Connessione.}\label{Connessione}
\begin{Definition}
	Sia $(\TopologicalSpace,\Topology)$ uno spazio topologico.
	$\TopologicalSpace$ si dice \Define{connesso}[connesso][spazio]
	quando $\TopologicalSpace$ e $\emptyset$ sono gli unici due
	aperti ad essere anche chiusi. Uno spazio topologico non connesso
	si dice \Define{disconesso}[disconnesso][spazio].
\end{Definition}
\begin{Theorem}
	Sia $(\TopologicalSpace,\Topology)$ uno spazio topologico.
	Sono equivalenti le seguenti proposizioni:
	\begin{itemize}
		\item $\TopologicalSpace$ \`e disconnesso;
		\item esistono $\Open_1, \Open_2 \in \Topology$ disgiunti
		e non vuoti tali che $\Open_1 \cup \Open_2 =
		\TopologicalSpace$.
		\item esistono chiusi $\Closed_1, \Closed_2$ disgiunti e
		non vuoti tali che $\Closed_1 \cup \Closed_2 =
		\TopologicalSpace$.
	\end{itemize}
\end{Theorem}
\Proof Supponiamo $\TopologicalSpace$ disconesso. Allora esiste
$\Open \in \Topology$ chiuso. Dunque $\Compl{\Open} \in \Topology$.Abbiamo
 $\Open \cup \Compl{\Open} = \TopologicalSpace$.
Quindi $\TopologicalSpace$ \`e unione di due aperti disgiunti.
\par Supponiamo $\TopologicalSpace = \Open_1 \cup \Open_2$ con $\Open_1,
\Open_2  \in \Topology$ non vuoti e disgiunti. Allora $\Open_1 =
\Compl{\Open_2}$ e $\Open_2 = \Compl{\Open_1}$ sono chiusi disgiunti.
\par Supponiamo $\Closed_1, \Closed_2$ siano chiusi disgiunti non vuoti
tali che $\TopologicalSpace = \Closed_1 \cup \Closed_2$. Allora $\Closed_1
= \Compl{\Closed_2}$ \`e aperto non vuoto e non uguale a
$\TopologicalSpace$ dunque $\TopologicalSpace$ \`e disconnesso. \EndProof
\begin{Theorem}
	Sia $\Continuous: \TopologicalSpace \rightarrow \VarTopologicalSpace$ un'applicazione continua tra gli spazi topologici $\TopologicalSpace$ e $\VarTopologicalSpace$. Se $\TopologicalSpace$ \`e connesso, allora \`e connesso anche $\Image{\Continuous}$.
\end{Theorem}
\Proof Sia $\Part \subseteq \Image{\Continuous}$ aperto e chiuso non vuoto di $\Image{\Continuous}$: esistono un aperto $\Open$ e un chiuso $\Closed$ in $\VarTopologicalSpace$ tali che $\Part = \Open \cap \Image{\Continuous} = \Closed \cap \Image{\Continuous}$. Abbiamo $\Continuous^{-1}(\Part) = \Continuous^{-1}(\Open) = \Continuous^{-1}(\Closed)$ aperto e chiuso in $\TopologicalSpace$ e dunque, essendo esso anche non vuoto, $\Continuous^{-1}(\Part) = \TopologicalSpace$. Pertanto $\Part = \Image{\Continuous}$. \EndProof
\begin{Definition}
	Sia $(\TopologicalSpace,\Topology)$ uno spazio topologico e sia
  $\Interval \subseteq \mathbb{R}$ un intervallo.
  Chiamiamo
  \Define{curva}
  un'applicazione continua
  $\Curve: \Interval \rightarrow \TopologicalSpace$.
  I punti
  \begin{itemize}
    \item $\Curve(\inf \Interval)$, se $\inf \Interval \in \Interval$;
    \item $\Curve(\sup \Interval)$, se $\sup \Interval \in \Interval$;
  \end{itemize}
  si chiamamo
  \Define{estremi}[di una curva][estremo]
  di $\Curve$, l'immagine $\Image{\Curve}$ si chiama
  \Define{supporto}[di una curva][supporto].
  \par $\Curve$ si dice
  \begin{itemize}
    \item \Define{chiusa}[chiusa][curva]
    quando $\Interval$ \`e chiuso e
    $\Curve(\inf \Interval) = \Curve(\sup \Interval)$;
    \item \Define{semplice}[semplice][curva]
    quando
    $\ForAll{(x,y) \in \Interior{\Interval} \times \Interval}{
    \Implies{\Curve(x) = \Curve{y}}{x = y}}$.
  \end{itemize}
\end{Definition}
\begin{Definition}
	Sia $(\TopologicalSpace,\Topology)$ uno spazio topologico.
  Chiamiamo
  \Define{cammino}
  o
  \Define{arco}
  una curva $\Path$ tale che $\Dom{\Path} = [0,1]$.
\end{Definition}
\begin{Definition}
	Sia $(\TopologicalSpace,\Topology)$ uno spazio topologico.
	$\TopologicalSpace$ si dice
	\Define{connesso per cammini}[connesso per cammini][spazio]
  quando,
	dati due punti $a, b \in \TopologicalSpace$ esiste un arco
	$\Path: [0,1] \rightarrow \TopologicalSpace$ tale
	che $\Path(0) = a$ e $\Path(1) = b$.
\end{Definition}
\begin{Definition}
	Sia $(\TopologicalSpace,\Topology)$ uno spazio topologico.
	Si chiamano
	\begin{itemize}
		\item \Define{componente connessa}[connessa][componente]
		 tutti i sottospazi connessi massimali di
		$\TopologicalSpace$;
		\item \Define{componente connessa per cammini}[connessa per cammini][componente]
		tutti i sottospazi connessi per cammini massimali di
		$\TopologicalSpace$.
	\end{itemize}
\end{Definition}
\begin{Theorem}
	Sia $(\TopologicalSpace,\Topology)$ uno spazio topologico.
	Le componenti connesse di $\TopologicalSpace$ ne determinano una
	partizione.
\end{Theorem}
\Proof Siano $\ConnectedComponent_1$ e $\ConnectedComponent_2$ due
componenti connesse. 
\begin{Theorem}
	Sia $(\TopologicalSpace,\Topology)$ uno spazio topologico.
	Le componenti connesse per cammini di $\TopologicalSpace$ ne
	determinano una partizione.
\end{Theorem}
\begin{Theorem}
	$[0,1]$ \`e connesso per la topologia euclidea.
\end{Theorem}
\Proof Supponiamo $\Closed_0$ e $\Closed_1$ chiusi non vuoti disgiunti tali che $[0,1] = \Closed_0 \cup \Closed_1$. Poich\'e $\Closed_1$ \`e chiuso, $\inf \Closed_1 \in \Closed_1$.
\par Supponiamo $0 \in \Closed_0$. Se $\inf \Closed_1 = 0$, allora $\Closed_0 \cap \Closed_1 \neq \emptyset$.
\par Supponiamo invece $\inf \Closed_1 \neq 0$. Allora $\Closed_0 \cap [0,\inf{\Closed_1}]$ \`e chiuso e dunque $\inf{\Closed_1} \in \Closed_0$.
\par Quindi, in ogni caso $\Closed_0 \cap \Closed_1 \neq \emptyset$, contro l'ipotesi. \EndProof
\begin{Theorem}
	I connessi di $\mathbb{R}$ sono tutti e soli gli intervalli.
\end{Theorem}
\Proof
\begin{Theorem}
	Sia $\Continuous: \ConnectedSpace \rightarrow \mathbb{R}$ un'applicazione continua con $\ConnectedSpace$ spazio topologico connesso. Siano $x, y \in \ConnectedSpace$ tali che $\Continuous(x) \leq \Continuous(y)$. Allora $\ForAll{\eta \in [\Continuous(x),\Continuous(y)]}{\Exists{\xi \in \ConnectedSpace}{\Continuous(\xi) = \eta}}$.
\end{Theorem}
\begin{Corollary}
	\TheoremName{Teorema dei valori intermedi}[dei valori intermedi][teorema] Sia $\Continuous: [a,b] \rightarrow \mathbb{R}$ ($a,b \in \mathbb{R}$, $a \leq b$) un'applicazione continua. Allora $\ForAll{\eta \in [\Continuous(a),\Continuous(b)] \cup [\Continuous(b),\Continuous(a)]}{\Exists{\xi \in \ConnectedSpace}{\Continuous(\xi) = \eta}}$.
\end{Corollary}
\Proof Segue direttamente dal teorema precedente. \EndProof
\begin{Corollary}
	\TheoremName{Teorema di esistenza degli zeri o di Bolzano}[di esistenza degli zeri o di Bolzano][teorema] Sia $\Continuous: [a,b] \rightarrow \mathbb{R}$ ($a,b \in \mathbb{R}$, $a \leq b$) un'applicazione continua tale che $\Continuous(a) \Continuous(b) < 0$. Allora $\Exists{\xi \in [a,b]}{\Continuous(\xi) = 0}$.
\end{Corollary}
\Proof Segue direttamente dal corollario precedente osservando che $\Continuous(a) \Continuous(b) < 0$ equivale a $\Or{\Continuous(a) < 0 < \Continuous(b)}{\Continuous(b) < 0 < \Continuous(a)}$. \EndProof
\begin{Theorem}
	Sia $(\TopologicalSpace,\Topology)$ uno spazio connesso per cammini.
	Esso \`e anche connesso.
\end{Theorem}
\Proof 

\section{Compattezza.}\label{Compattezza}
\begin{Definition}
	Sia $(\CompactSpace,\Topology)$ uno spazio topologico.
	$\CompactSpace$ si dice \Define{compatto}[compatto][spazio]
	quando ogni ricoprimento aperto di $\TopologicalSpace$ ammette un
	sottoricoprimento finito.
\end{Definition}
\begin{Theorem}
	Sia $(\TopologicalSpace,\Topology)$ uno spazio topologico e sia $\CompactSpace \subseteq \TopologicalSpace$ un sottospazio topologico. $\CompactSpace$ \`e compatto se e solo se ogni ricoprimento aperto di $\CompactSpace$ in $\TopologicalSpace$ ammette un sottoricoprimento finito.
\end{Theorem}
\Proof Sia $\CompactSpace$ compatto e sia $(\Open_i)_{i \in I}$ un ricoprimento aperto di $\CompactSpace$ in $\TopologicalSpace$. Allora $(\Open_i \cap \CompactSpace)_{i \in I}$ \`e un ricoprimento aperto di $\CompactSpace$ in $\CompactSpace$ e ammette dunque un sottoricoprimento finito $(\Open_i \cap \CompactSpace)_{i \in J}$ ($J \subseteq I$). Quindi $(\Open_i)_{i \in J}$ \`e un sottoricoprimento di $(\Open_i)_{i \in I}$ di $\CompactSpace$.
\par Supponiamo adesso che ogni ricoprimento aperto di $\CompactSpace$ ammetta un sottoricoprimento finito. Sia $(\Open_i)_{i \in I}$ un ricoprimento aperto di $\CompactSpace$ in $\CompactSpace$. Esistono aperti $(\Open_i')_{i \in I}$ in $\TopologicalSpace$ tali che $\ForAll{i \in I}{\Open_i = \Open_i' \cap \CompactSpace}$. $(\Open_i')_{i \in I}$ \`e un ricoprimento aperto di $\CompactSpace$ ed ammette dunque un sottoricoprimento finito $(\Open_i')_{i \in J}$ ($J \subseteq I$). Allora $(\Open_i)_{i \in J}$ \`e un sottoricoprimento finito di $(\Open_i)_{i \in I}$ per $\CompactSpace$. Quindi $\CompactSpace$ \`e compatto. \EndProof
\begin{Theorem}
	Sia $\Continuous: \CompactSpace \rightarrow \VarTopologicalSpace$ un'applicazione continua tra gli spazi topologici $\CompactSpace$ e $\VarTopologicalSpace$. Se $\CompactSpace$ \`e compatto, allora \`e compatto anche $\Image{\Continuous}$.
\end{Theorem}
\Proof Sia $(\Open_i)_{i \in I}$ un ricoprimento aperto di $\Image{\Continuous}$ in $\VarTopologicalSpace$. Allora $(\Continuous^{-1}(\Open_i))_{i \in I}$ \`e un ricoprimento aperto di $\CompactSpace$: per la compattezza di $\CompactSpace$, ne esiste un sottoricoprimento finito $(\Continuous^{-1}(\Open_i))_{i \in J}$ ($J \subseteq I$) di $\CompactSpace$. $(\Open_i)_{i \in J}$ \`e allora un sottoricoprimento finito di $(\Open_i)_{i \in I}$ per $\Image{\Continuous}$. \EndProof
\begin{Theorem}
	Sia $(\TopologicalSpace,\Topology)$ uno spazio topologico compatto e sia $\Closed$ chiuso in $\TopologicalSpace$. $\Closed$ \`e compatto.
\end{Theorem}
\Proof Sia $(\Open_i)_{i \in I}$ un ricoprimento aperto di $\Closed$. $\Complement{\Closed}$ \`e aperto, quindi la famiglia di aperti $(\Open_i)_{i \in I}$ allargata con $\Complement{\Closed}$ \`e un ricoprimento aperto di $\TopologicalSpace$. Esiste dunque una sottofamiglia finita $(\Open_i)_{i \in J}$ ($J \subseteq I$) tale che $\Complement{\Closed} \cup \bigcup_{i \in J} \Open_i = \TopologicalSpace$. Ne deduciamo che $(\Open_i)_{i \in J}$ \`e un sottoricoprimento finito di $(\Open_i)_{i \in I}$ per $\Closed$. \EndProof
\begin{Theorem}
	Sia $(\MetricSpace,\Distance)$ uno spazio metrico e sia $\CompactSpace \subseteq \MetricSpace$ compatto: $\CompactSpace$ \`e chiuso e limitato.
\end{Theorem}
\Proof Fissato $O \in \MetricSpace$, consideriamo il ricoprimento aperto $(\OpenBall{O}{n})_{n \in \mathbb{N}}$ di $\CompactSpace$. Per la compattezza di $\CompactSpace$, esiste un sottoricoprimento finito di $(\OpenBall{O}{n})_{n \in N}$ ($N \subseteq \mathbb{N}$) per $\CompactSpace$ e dunque $\CompactSpace \subseteq \OpenBall{O}{\max N}$. Ne deduciamo che $\CompactSpace$ \`e limitato.
\par Proviamo ora che $\Complement{\CompactSpace}$ \`e aperto. Sia $P \in \Complement{\CompactSpace}$. La famiglia di aperti $(\OpenBall{x}{r})_{(x,r) \in \CompactSpace \times [0,\Distance(x,P)[}$ \`e un ricoprimento aperto di $\CompactSpace$, ne esiste dunque un sottoricoprimento finito $(\OpenBall{x}{r})_{(x,r) \in A \times B)}$ ($A \subseteq \CompactSpace$, $B \subseteq [0, \Distance(x,P)[$).
\par Consideriamo $\OpenBall{P}{R}$, con $0 < R < \min_{(x,r) \in A \times B} (\Distance(P,x) - r)$. Sia $Q \in \OpenBall{P}{R}$: per ogni $(x,r) \in A \times B$ abbiamo, per la disuguaglianza triangolare, $\Distance(Q,x) \geq \Distance(P,x) - \Distance(Q,P)$. Ora, abbiamo $\Distance(Q,P) < R < \min_{(x,r) \in A \times B} (\Distance(P,x) - r) \leq \Distance(P,x) - \max B$, da cui $\Distance(Q,x) > \max B > r$. Quindi $\ForAll{(x,r) \in A \times B}{Q \notin \OpenBall{x}{r}}$ e dunque $Q \notin \CompactSpace$. Allora $\Complement{\CompactSpace}$ \`e aperto. \EndProof
\begin{Theorem}
	$[0,1]$ \`e compatto per la topologia euclidea.
\end{Theorem}
\Proof Sia $(\Open_i)_{i \in I}$ un ricoprimento aperto di $[0,1]$ e sia $X = \lbrace x \in \RealNonNegative | \Exists{J \subseteq I}{\And{\Cardinality{J} < \aleph_0}{[0,t] \subseteq \bigcup_{i \in J} \Open_i}} \rbrace$.
\par Sicuramente $0 \in X$, quindi $X \neq \emptyset$. Consideriamo $\sup X$.
\par Supponiamo $\sup X \leq 1$. Allora, per oppurtuno $j \in I$ abbiamo $\sup X \in \Open_j$. Sia $\delta > 0$ tale che $\sup X \in ]\sup X - \delta, \sup X + \delta[ \subseteq \Open_j$. Sia $y \in ]\sup X - \delta, \sup X$. Esiste un sottoricoprimento finito $(\Open_i)_{i \in J}$ ($J \subseteq I$) di $(\Open_i)_{i \in I}$ per $[0,y]$. Ma $(\Open_i)_{i \in J \cup \lbrace j \rbrace}$ \`e allora un sottoricoprimento finito per $[0,\sup X + t]$ per ogni $t \in \mathbb{R}$ tale che $0 \leq t < \delta$, e dunque $\sup X + t \in X$, ma questo \`e assurdo.
\par Dunque $\sup X > 1$ e sia $s \in X$ tale che $1 \leq s \leq \sup X$. Allora esiste $J \subseteq I$ finito tale che $[0,1] \subseteq [0,s] \subseteq \bigcup_{i \in J} \Open_i$. \EndProof
\begin{Corollary}
	Un sottospazio topologico $\CompactSpace \subseteq \mathbb{R}$ della retta euclidea \`e compatto se e solo se \`e chiuso e limitato.
\end{Corollary}
\Proof Abbiamo gi\`a provato che in uno spazio metrico i compatti sono chiusi e limitati.
\par Supponiamo $\CompactSpace \subseteq \mathbb{R}$ chiuso e limitato. Poich\'e $\CompactSpace$ \`e limitato, per opportuno $R > 0$ abbiamo $\CompactSpace \subseteq [-R,R]$: $\CompactSpace$ \`e allora un chiuso in un compatto, pe pertanto compatto. \EndProof
\begin{Corollary}
	\TheoremName{Teorema di Weierstrass} Sia $\Continuous: \CompactSpace \rightarrow \mathbb{R}$ un'applicazione continua, con $\CompactSpace$ compatto. $\Continuous$ ammette massimo e minimo.
\end{Corollary}
\Proof L'immagine di $\Continuous$ \`e compatta e dunque chiusa e limitata. Pertanto ammette massimo e minimo. \EndProof


	\chapter{Topologia algebrica}
	\section{Teorema di Perron-Frobenius.}
\label{CalcoloScientifico_TeoremaDiPerronFrobenius}
\begin{Theorem}
  \TheoremName{Teorema di Perron-Frobenius}[di Perron-Frobenius][teorema]
  Siano
  \begin{itemize}
    \item $n \in \NotZero{\mathbb{N}}$;
    \item $\Matrix \in \mathbb{C}^{n \times n}$ non negativa.
  \end{itemize}
  Allora
  \begin{itemize}
    \item $\SpectralRadius{\Matrix} \in \Spectrum{\Matrix}$;
    \item esiste un autovettore sinistro $\VarEigenvector$ tale che
      $\ForAll{k \in n}{\VarEigenvector \geq 0}$;
    \item esiste un autovettore destro $\Eigenvector$ tale che
      $\ForAll{k \in n}{\Eigenvector \geq 0}$.
  \end{itemize}
  Inoltre,
  \begin{itemize}
    \item se $\Matrix$ \`e irriducibile, allora
      \begin{itemize}
        \item $\SpectralRadius{\Matrix}$ \`e un autovalore semplice;
        \item esiste un autovettore sinistro $\VarEigenvector$ tale che
          $\ForAll{k \in n}{\VarEigenvector > 0}$;
        \item esiste un autovettore destro $\Eigenvector$ tale che
          $\ForAll{k \in n}{\Eigenvector > 0}$;
      \end{itemize}
      \item se $\Matrix$ \`e positiva, allora
        $\SpectralRadius{\Matrix}$
        \`e l'unico autovalore di modulo massimo di $\Matrix$.
  \end{itemize}
\end{Theorem}

\section{\English{Page rank}.}
\label{CalcoloScientifico_PageRank}
Il modello denominato
\Define{\English{page rank}}
\`e un modello nato per classificare l'importanza delle pagine
\English{web}, da cui il nome, ma che pu\`o essere riproposto per ogni
rete. Noi lo presenteremo nel suo contesto originale delle reti.
\par Sia $\PagesNumber$ il numero totale di tutte le pagine di una rete data.
Identificheremo ogni pagina con un numero naturale da $0$ a
$\PagesNumber - 1$, dove $\PagesNumber$ \`e il numero delle pagine stesse.
\par Esso viene costruito a partire da un grafo orientato
$\Graph = (\PagesNumber,\Edges)$ dove
$(\Node,\VarNode) \in \PagesNumber \times \PagesNumber$ appartiene a $\Edges$
se e solo se la pagina $\Node$ contiene un collegamento ipertestuale
alla pagina $\VarNode$.
\par Denotiamo $\AdjacencyMatrix$ la matrice di adiacenza di $\Graph$.
\par Lo scopo del modello di \English{page rank} \`e determinare un
\Define{vettore di pesi}[dei pesi (\English{page rank})][vettore]
$\Weights$: la componente $\Weights_k$ indica l'importanza della pagina $k$.
\par Definiamo i vettori
\begin{itemize}
  \item $\AllOnes = \Transposed{(1 \cdots 1)}$;
  \item $\PageDegree = \AdjacencyMatrix \AllOnes$;
  \item $\MatrixPageDegree \in \mathbb{R}^{\PagesNumber \times \PagesNumber}$
    tale che
    \[
      \ForAll{(j,k) \in \PagesNumber \times \PagesNumber}{
        \MatrixPageDegree_j^k =
        \begin{cases}
          0&\text{ se } j \neq k,\\
          \PageDegree_k&\text{ se } j = k;
        \end{cases}}
    \]
  \item $\Matrix = \MatrixPageDegree^{-1}\AdjacencyMatrix$.
\end{itemize}
La componente $\PageDegree_k$ ($k \in \PagesNumber$) indica il grado uscente
della pagina $k$.
\par Vorremmo costruire un modello per cui ogni pagina trasmette equamente
la sua importanza a tutte le pagine da esse puntata e l'importanza di una pagina
\`e esattamente la somma delle importanze cos\`i trasmesse tramite i
collegamenti.
\par Come vedremo, \`e utile ipotizzare che non esistano pagine pendenti
nella rete in esame; questa ipotesi \`e facilmente realizzabile con una delle due strategie seguenti, nessuna delle quali cambia sostanzialmente
il senso del modello della rete:
\begin{itemize}
  \item aggiungiamo ad ogni pagina pendente un collegamento a tutte
    le altre pagine;
  \item intoduciamo una pagina aggiuntiva $\PagesNumber$ che punta a tutte 
  le altre pagine ed \`e puntata da tutte le altre pagine:
  \[
    \ForAll{k > 0}{\AdjacencyMatrix_\PagesNumber^k
    = \AdjacencyMatrix_k^\PagesNumber = 1}.
  \]
\end{itemize}
\par Assumeremo dunque d'ora in poi il seguente assioma.
\begin{Axiom}
  La rete non contiene nodi pendenti.
\end{Axiom}
\begin{Theorem}
  $\Weights$ \`e autovettore sinistro di $\Matrix$.
\end{Theorem}
\Proof Per ogni pagina $k \in \PagesNumber$, abbiamo
\[
  \Weights_k = \sum_{j \in \PagesNumber} \AdjacencyMatrix_j^k \frac{\Weights_j}{\PageDegree_j},
\]
equivalente a
\[
  \Weights = \MatrixPageDegree^{-1}\AdjacencyMatrix\Weights,
\]
vale a dire
\[
  \Transposed{\Weights} = \Transposed{\Weights}\Matrix.\text{ \EndProof}
\]
\begin{Theorem}
  Abbiamo
  \begin{itemize}
    \item $1$ \`e autovalore di $\Matrix$ relvativo a $\AllOnes$;
    \item $\SpectralRadius{\Matrix} = 1$.
  \end{itemize}
\end{Theorem}
\Proof Abbiamo
$\Matrix\AllOnes
= \MatrixPageDegree^{-1}\AdjacencyMatrix\AllOnes
= \MatrixPageDegree^{-1}\PageDegree
= \AllOnes$.
\par Inoltre,
$\Norm{\Matrix}[\infty]
= \max_{k \in \PagesNumber} \sum_{j \in \PagesNumber}
  \AbsoluteValue{\Matrix_k^j}
= \max_{k \in \PagesNumber} \sum_{j \in \PagesNumber}
  \Matrix_k^j
= \max_{k \in \PagesNumber} \sum_{j \in \PagesNumber}
  \frac{\AdjacencyMatrix_k^j}{\PageDegree_k}
= 1$,
e quindi, se $\Eigenvector \in \mathbb{C}^\PagesNumber$ \`e autovettore
di $\Matrix$ relativo a $\Eigenvalue$, allora
$\AbsoluteValue{\Eigenvalue}\Norm{\Eigenvector}[\infty]
= \Norm{\Eigenvalue\Eigenvector}[\infty]
= \Norm{\Matrix\Eigenvector}[\infty]
\leq \Norm{\Matrix}[\infty]\Norm{\Eigenvector}[\infty]$,
da cui
$\AbsoluteValue{\Eigenvalue}
\leq \Norm{\Matrix}[\infty] = 1$. \EndProof
\begin{Theorem}
  Se $\Matrix$ ha un unico autovalore di modulo massimo, qualsiasi successione
  $(\Weights^{(k)})_{k \in \mathbb{N}}$ tale che
  \begin{itemize}
    \item $\Weights^{(0)}$ non \`e ortogonale a $\AllOnes$;
    \item per ogni $k > 0$,
      $\Weights^{(k)} = \Transposed{\Matrix}\Eigenvector^{(k - 1)}$;
    \item se $\Weights^{(0)}$ non ha componenti negative, allora
      $\ForAll{k \in \mathbb{N}}{
      \Norm{\Weights^{(k)}}[1] = \Norm{\Weights^{(0)}}[1]}$;
  \end{itemize}
  converge a un autovettore sinistro di $\Matrix$, cio\`e a $\Weights$
  a meno di prodotto per scalare.
\end{Theorem}
\Proof Riscriviamo il metodo delle potenze applicandolo alla matrice
$\Transposed{\Matrix}$ e al vettore $\Weights^{(0)}$ con una qualsiasi
tolleranza $\Tollerance$:
\begin{enumerate}
  \item\label{PageRank_MetodoDellePotenze_1} si ponga $j = 0$,
    $\Weights_0 = \Weights^{(0)}$,
    $\Eigenvalue_0^{(0)} =
    = \Transposed{\Weights_0}\Matrix\Weights_{0}$;
  \item\label{PageRank_MetodoDellePotenze_2} si ponga
    $\tilde{\Weights}_{j + 1} = \Matrix \Weights_j$;
  \item\label{PageRank_MetodoDellePotenze_3} si ponga
    $\Weights_{j + 1}
    = \frac{\tilde{\Weights}_{j + 1}}{\Norm{\tilde{\Weights_{j + 1}}}}$;
  \item\label{PageRank_MetodoDellePotenze_4} si ponga $\Eigenvalue_0^{(j + 1)}
    = \Transposed{\Weights_{j + 1}}\Matrix\Weights_{j + 1}$;
  \item\label{PageRank_MetodoDellePotenze_5} se
    $\AbsoluteValue{\Eigenvalue_0^{(j + 1)} - \Eigenvalue_0^{(j)}}
    > \Tollerance$
    si incrementi $j$ di $1$ e si torni al punto
    \ref{PageRank_MetodoDellePotenze_2}, altrimenti l'algoritmo termina
    restituendo la coppia $(\Eigenvalue_0^{(j + 1)},\Weights_{j + 1})$.
\end{enumerate}
\par Per ogni $k \in \mathbb{N}$, abbiamo
$\Weights_k = \Weights^{(k)}$ a meno di moltiplicazione per scalare.
\par Supponiamo $\Weights^{(0)}$ non abbia nessuna componente negativa.
Si dimostra immediatamente per induzione, utilizzando anche la non negativit\`a
di $\Matrix$, che per ogni $k \in \mathbb{N}$,
$\Norm{\Weights^{(k)}}[1]
= \Transposed{\Weights^{(k)}}\AllOnes
= \Transposed{\Weights^{(0)}}\AllOnes
\Norm{\Weights^{(0)}}[1]$. \EndProof
\par Con riferimento a quest'ultimo teorema, grazie alla costanza della
norma dei vettori della successione
$(\Weights^{(k)})_{k \in \mathbb{N}}$
\`e possibile calcolare direttamente l'autovettore sinistro $\Weights$
senza il passo di normalizzazione \ref{PageRank_MetodoDellePotenze_3}
evitando qualsiasi rischio di traboccamento dei moduli.
\par Definiamo adesso
\begin{itemize}
  \item $\gamma \in (0,1)$;
  \item $\Vector \in \mathbb{R}^\PagesNumber$ tale che
    $\ForAll{k \in \PagesNumber}{\Vector_k > 0}$ e
    $\Norm{\Vector}[1] = 1$;
  \item $A = \gamma\Matrix + (1 - \gamma)\AllOnes\Transposed{\Vector}$.
\end{itemize}
\begin{Theorem}
  La matrice $A$ \`e positiva.
\end{Theorem}
\Proof Fissiamo $(k,j) \in \PagesNumber \times \PagesNumber$:
$A_k^j$ \`e combinazione convessa di $\Matrix_k^j$ e
$\sum_{k \in \PagesNumber} \Vector_k$ con coefficienti non nulli,
dunque \`e strettamente positivo. \EndProof
\begin{Theorem}
  $\AllOnes$ \`e autovettore di $A$ relativo all'autovalore $1$.
\end{Theorem}
\Proof Abbiamo
$A\AllOnes
= \gamma\Matrix\AllOnes + (1 - \gamma)(\AllOnes\Transposed{\Vector})\AllOnes
= \gamma\AllOnes
  + (1 - \gamma)\AllOnes(\Transposed{\Vector}\AllOnes)
= \gamma\AllOnes + (1 - \gamma)\AllOnes
= \AllOnes$. \EndProof
\begin{Theorem}
  \TheoremName{Teorema di Brauer}[di Brauer][teorema]
  Siano
  \begin{itemize}
    \item $n \in \NotZero{\mathbb{N}}$;
    \item $J \in \mathbb{C}^{n \times n}$;
    \item $\Eigenvector$ autovettore di $J$ relativo all'autovalore
      $\Eigenvalue \in \Spectrum{J}$;
    \item $\VarVector \in \mathbb{C}^n$;
    \item $K = J + \Eigenvector\HermitianTransposed{\VarVector}$.
  \end{itemize}
  Abbiamo
  \[
    \Spectrum{K}
    = \Spectrum{J}
      \SetMin \lbrace \Eigenvalue \rbrace
      \cup \lbrace \Eigenvalue + \HermitianTransposed{\VarVector}\Eigenvector \rbrace.
  \]
\end{Theorem}
\Proof Sia $\UnitaryMatrix \in \mathbb{C}^n$ unitaria tale che
$\SchurNormalForm = \HermitianTransposed{\UnitaryMatrix} J \UnitaryMatrix$
sia una forma normale di Schur di $J$ tale che
$\SchurNormalForm_0^0 = \Eigenvalue$.
\par Denotando $\CanonicalBase^0$ il primo vettore della base canonica, abbiamo
\begin{align}
  \HermitianTransposed{\UnitaryMatrix} K \UnitaryMatrix
  &= \HermitianTransposed{\UnitaryMatrix} J \UnitaryMatrix
    + \HermitianTransposed{\UnitaryMatrix}
      \Eigenvector\HermitianTransposed{\VarVector}
      \UnitaryMatrix,\\
  &= \SchurNormalForm
    + \CanonicalBase^0\HermitianTransposed{\VarVector}
      \UnitaryMatrix.
\end{align}
\par Sia ora $k \in n$. Supponiamo $k = 0$: abbiamo
\[
  (\CanonicalBase^0\HermitianTransposed{\VarVector}\UnitaryMatrix)_0
  = \HermitianTransposed{\VarVector}\UnitaryMatrix.
\]
Se invece $k \neq 0$, allora, per ogni $j \in n$
\[
  (\CanonicalBase^0\HermitianTransposed{\VarVector}\UnitaryMatrix)_k
  = 0.
\]
Dunque tutti gli elementi diagonali della matrice triangolare superiore
$\HermitianTransposed{\UnitaryMatrix} K \UnitaryMatrix$,
forma normale di Schur,
coincidono con quelli di $\SchurNormalForm$ tranne
$(\HermitianTransposed{\UnitaryMatrix} J \UnitaryMatrix)_0^0$
che viene sostituito da
\[
  \Eigenvalue + \HermitianTransposed{\VarVector}\UnitaryMatrix\CanonicalBase^0
  = \Eigenvalue + \HermitianTransposed{\VarVector}\Eigenvector.
  \text{ \EndProof}
\]
\begin{Theorem}
  $1$ \`e autovalore semplice e di modulo massimo di $A$.
\end{Theorem}
\Proof Ricordiamo che
$A = \gamma\Matrix + (1 - \gamma)\AllOnes\Transposed{\Vector}$.
$\AllOnes$ \`e autovettore relativo a $1$ di $\Matrix$ e $1$ \`e autovalore
semplice e di modulo massimo di $\Matrix$ per il teorema di Perron-Frobenius.
Ne deduciamo che
$\AllOnes$ \`e autovettore relativo a $\gamma$ di $\gamma\Matrix$ e
$\gamma$ \`e autovalore semplice e di modulo massimo di $\gamma\Matrix$
per il teorema di Perron-Frobenius.
\par Abbiamo
$\gamma + (1 - \gamma)\Transposed{\AllOnes}\Vector
= \gamma + (1 - \gamma)\Norm{\Vector}[1] = 1$.
Dunque, per il teorema di Bauer, $1$ \`e autovalore semplice e di modulo
massimo di $A$. \EndProof

\section{Vibrazioni.}
\label{CalcoloScientifico_Vibrazioni.}
\subsubsection{Vibrazioni di corde elastiche.}
\label{CalcoloScientifico_VibrazioniDiCordeElastiche}
\par Consideriamo una successione finita di
$N \in \NotZero{\mathbb{N}}$ punti materiali
$(\PointMass_k)_{k \in N}$ di massa $(\Mass_k)_{k \in N}$
e coordinate $(X_k)_{k \in N}$ (nel piano e nello spazio indifferentemente)
tali che
\begin{itemize}
  \item i punti $\PointMass_0$ e $\PointMass_{N - 1}$ siano vincolati
    nella loro posizione al tempo iniziale $t = 0$;
  \item per ogni $k \in N - 1$, $\PointMass_k$ e $\PointMass_{k + 1}$
    sono collegati da una corda elastica di costante elastica
    $\ElasticConstant_k$;
  \item per ogni $k \in \NotZero{(N - 1)}$, $\PointMass_k$ \`e soggetto
    ad una forza d'attrito dinamico con coefficiente $\Friction_k$.
\end{itemize}
\par Sia $k \in \NotZero{(N - 1)}$. Per la seconda legge di Newton, abbiamo
\[
  \Mass_k X_k = - \ElasticConstant X_k - \Friction_k \dot{X_k}.
\]
Inoltre, abbiamo le condizioni al bordo
\[
  \begin{cases}
    \ForAll{t \in \RealNonNegative}{X_0(t) = X_0(0)},\\
    \ForAll{t \in \RealNonNegative}{X_{N - 1}(t) = X_{N - 1}(0)},\\
    \dot{X_0} = \dot{X_{N - 1}} = 0.
  \end{cases}
\]
\par Poniamo ora
\begin{itemize}
  \item $M \in \mathbb{R}^{N \times N}$ uguale alla matrice diagonale i cui
    elementi sono la successione delle masse:
    \[
      M =
      \lmatrix
      \begin{array}{ccc}
        \Mass_0 & &\\
        & \ddots &\\
        & & \Mass_{N - 1}
      \end{array}
      \rmatrix.
    \]
  \item $R \in \mathbb{R}^{N \times N}$ uguale alla matrice diagonale i cui
    elementi sono la successione dei coefficienti di attrito:
    \[
      R =
      \lmatrix
      \begin{array}{ccc}
        \Friction_0 & &\\
        & \ddots &\\
        & & \Friction_{N - 1}
      \end{array}
      \rmatrix.
    \]
  \item $K \in \mathbb{R}^{N \times N}$ uguale alla matrice tridiagonale i
    cui elementi sono
    $K_k^j =
    \begin{cases}
      - \ElasticConstant_k&\text{ se }k = j + 1,\\
      \ElasticConstant_k + \ElasticConstant_{k + 1}&\text{ se }k = j,\\
      - \ElasticConstant_{k + 1}&\text{ se }k = j - 1,
    \end{cases}$
    vale a dire
    \[
      K =
      \lmatrix
      \begin{array}{cccccc}
        \ElasticConstant_0 + \ElasticConstant_1 & - \ElasticConstant_1 &&&&\\
        - \ElasticConstant_0 & \ddots & \ddots &&&\\
        & \ddots & \ddots & \ddots & &\\
        && \ddots & \ddots & - \ElasticConstant\\
        &&& - \ElasticConstant & \ElasticConstant + \ElasticConstant
      \end{array}
      \rmatrix.
    \]
\end{itemize}

\subsubsection{Vibrazioni di membrane elastiche.}
\label{CalcoloScientifico_VibrazioniDiMembraneElastiche}



	\part{Geometria}
	\chapter{Geometria lineare}
	\section{Definizioni e teoremi di base.}
\label{GeometriaLineare_DefinizioniETeoremiDiBase}
\begin{Definition}
	Sia $\Module$ un $\Ring$-modulo. La cardinalit\`a comune di tutte
	le basi di $\Module$, denotata $\Dimension{\Module}$, si chiama
	\Define{dimensione}[di un modulo][dimensione] di $\Module$.
\end{Definition}
\begin{Definition}
	Sia $\LinearSpace$ un $\Field$-spazio vettoriale di dimensione finita e sia $\LinearSpace_0 \subseteq \LinearSpace$ sottospazio vettoriale. Si definisce \Define{codimensione}[di un sottospazio vettoriale in uno spazio vettoriale finito][codimensione], denotata $\Codimension{\LinearSpace_0}$, la differenza $\Dimension{\LinearSpace} - \Dimension{\LinearSpace_0}$.
\end{Definition}
\begin{Definition}
	Sia $\LinearSpace$ un $\Field$-spazio vettoriale finito. Un sottospazio vettoriale $\Hyperplane$ di $\LinearSpace$ di codimensione $1$ si chiama \Define{iperpiano lineare}[lineare][iperpiano] di $\LinearSpace$.
\end{Definition}
\begin{Theorem}
	Sia $\LinearSpace$ un $\Field$-spazio vettoriale di dimensione finita e sia $\LinearSpace_0 \subseteq \LinearSpace$ sottospazio vettoriale tale che $\Dimension{\LinearSpace_0} = \Dimension{\LinearSpace}$. Allora $\LinearSpace_0 = \LinearSpace$.
\end{Theorem}
\Proof Supponiamo $(\LinearBase_i)_{i \in I}$ sia una base di $\LinearSpace_0$ e sia $\Vector \in \LinearSpace$. $\Vector$ non \`e linearmente indipendente da $(\LinearBase_i)_{i \in I}$ poich\'e se lo fosse $\LinearSpace$ avrebbe dimensione maggiore di $\Dimension{\LinearSpace_0}$, contro le ipotesi. Dunque $\Vector$ \`e combinazione lineare dei $(\LinearBase_i)_{i \in I}$, da cui $\Vector \in \LinearSpace_0$. \EndProof
\begin{Theorem}
	Sia $\Linear: \LinearSpace \rightarrow \LinearSpace'$ un'applicazione lineare tra i $\Field$-spazi vettoriali $\LinearSpace$ e $\LinearSpace'$ di dimensione finita. Abbiamo $\Dimension{\LinearSpace} = \Dimension{\Kernel{\Linear}} + \Dimension{\Image{\Linear}}$.
\end{Theorem}
\Proof Sia $(\LinearBase_i)_{i \in \Dimension{\Image{\Linear}}}$ una base di $\Image{\Linear}$. Allora $(\Linear^{-1}(\LinearBase_i))_{i \in \Dimension{\Image{\Linear}}}$ \`e una famiglia indiependente di vettori di $\LinearSpace$:  completiamola a base aggiungendo gli elementi $(\LinearBase_i)_{i \in \Dimension{\LinearSpace} \SetMin \Dimension{\Image{\Linear}}}$. Necessariamente, per ogni $i \in \Dimension{\LinearSpace} \SetMin \Dimension{\Image{\Linear}}$, abbiamo $\Linear(\LinearBase_i) = 0$. Dunque $(\LinearBase_i)_{i \in \Dimension{\LinearSpace} \SetMin \Dimension{\Image{\Linear}}}$ \`e una base di $\Kernel{\Linear}$. \EndProof
\begin{Corollary}
	Sia $\Linear: \LinearSpace \rightarrow \LinearSpace'$ un'applicazione lineare tra i $\Field$-spazi vettoriali $\LinearSpace$ e $\LinearSpace'$ di dimensione finita e uguale.
	Sono equivalenti
	\begin{itemize}
		\item $\Linear$ \`e iniettiva;
		\item $\Linear$ \`e suriettiva;
		\item $\Linear$ \`e un isomorfismo.
	\end{itemize}
\end{Corollary}
\Proof Provata l'equivalenza tra iniettivit\`a e suriettivit\`a, l'equivalenza con la biettivit\`a discende direttamente dalle definizioni.
\par Affermare che $\Linear$ \`e iniettiva equivale ad affermare che $\Dimension{\Kernel{\Linear}} = \lbrace 0 \rbrace$, da cui $\Dimension{\Image{\Linear}} = \Dimension{\LinearSpace}$ e dunque $\Image{\Linear} = \LinearSpace'$.
\par Affermare che $\Linear$ \`e suriettiva equivale ad affermare che $\Dimension{\Image{\Linear}} = \Dimension{\LinearSpace'}$, da cui $\Dimension{\Kernel{\Linear}} = 0$ e dunque $\Kernel{\Linear} = \lbrace 0 \rbrace$. \EndProof
\begin{Theorem}
	\TheoremName{Formula di Grassmann}[di Grassmann][formula] Siano
	\begin{itemize}
		\item $\LinearSpace$ un $\Field$-spazio vettoriale finito;
		\item $\VarLinearSpace_1$ e $\VarLinearSpace_2$ sottospazi vettoriali di $\LinearSpace$.
	\end{itemize}
	Abbiamo
	\[
		\LinearDimension{(\VarLinearSpace_1 + \VarLinearSpace_2)} + \LinearDimension{(\VarLinearSpace_1 \cap \VarLinearSpace_2)} = \LinearDimension{\VarLinearSpace_1} + \LinearDimension{\VarLinearSpace_2}.
	\]
\end{Theorem}
\begin{Theorem}
  Siano
  \begin{itemize}
    \item $\LinearSpace$ un $\Field$-spazio vettoriale finito;
    \item $\LinearSubspace, \VarLinearSubspace \subseteq \LinearSpace$
      sottospazi vettoriali di $\LinearSpace$;
    \item $a = \LinearDimension{\LinearSubspace}$;
    \item $b = \LinearDimension{\LinearSpace}
              - \LinearDimension{\VarLinearSubspace} < a$;
    \item $n = \LinearDimension{\LinearSpace}$;
    \item $(\LinearBase_j)_{j \in a - 1}$ una
      famiglia di vettori di $\LinearSpace$ tale che
      \begin{itemize}
        \item $(\LinearBase_j)_{j = 0}^{a - 1}$ una
          base di $\LinearSubspace$;
        \item $(\LinearBase_j)_{j = b}^{n - 1}$
          una base di $\VarLinearSubspace$;
        \item $(\LinearBase_j)_{j = b}^{a - 1}$ una
          base di $\LinearSubspace \cap \VarLinearSubspace$.
      \end{itemize}
  \end{itemize}
  $(\LinearBase_j)_{j \in n}$ \`e una base di
  $\LinearSpace$.
\end{Theorem}
\Proof Sia
$(\Scalar_j)_{j \in n} \in \Field^n$ tale che
$\sum_{j \in n} \Scalar_j\LinearBase_j = 0$.
\par Abbiamo
$\sum_{j = b}^{a - 1} \Scalar_j\LinearBase_j =
- \sum_{j = 0}^{b - 1} \Scalar_j\LinearBase_j
- \sum_{j = a}^{n - 1} \Scalar_j\LinearBase_j$.
\par A primo membro abbiamo una combinazione lineare di vettori di
$\LinearSubspace \cap \VarLinearSubspace$, dunque deve essere
$\sum_{j = 0}^{b - 1} \Scalar_j\LinearBase_j
+ \sum_{j = a}^{n - 1} \Scalar_j\LinearBase_j = 0$ e
$\ForAll{j \in a \SetMin b}{\Scalar_j = 0}$.
\par Poniamo
$\Vector = \sum_{j = 0}^{b - 1} \Scalar_j\LinearBase_j
= - \sum_{j = a}^{n - 1} \Scalar_j\LinearBase_j$,
ma il primo membro \`e una combinazione lineare di vettori
$\LinearSubspace$ e il secondo una combinazione lineare di vettori di
$\VarLinearSubspace$ e pertanto
$\Vector \in \LinearSubspace \cap \VarLinearSubspace$.
Poich\'e per\`o tale vettore deve essere linearmente indipendente da
$(\LinearBase_j)_{j = b}^{a - 1}$, deve essere $v = 0$ e infine
$\ForAll{j \in n}{\Scalar_j = 0}$. \EndProof
\begin{Corollary}
  Siano
  \begin{itemize}
    \item $\LinearSpace$ un $\Field$-spazio vettoriale finito;
    \item $\LinearSubspace, \VarLinearSubspace \subseteq \LinearSpace$
      sottospazi vettoriali di $\LinearSpace$ tali che
      \begin{itemize}
        \item $\LinearSubspace \cap \VarLinearSubspace = \lbrace 0 \rbrace$;
        \item $\LinearDimension{\LinearSubspace}
                + \LinearDimension{\VarLinearSubspace}
                = \LinearDimension{\LinearSpace}$.
      \end{itemize}
  \end{itemize}
  Abbiamo
  \[
    \LinearSubspace \DirectSum \VarLinearSubspace = \LinearSpace.
  \]
\end{Corollary}
\Proof Segue direttamente dal teorema precedente dopo aver costruito una base
$(\LinearBase_j)_{j \in \LinearDimension{\LinearSpace}}$ tale che
\begin{itemize}
  \item $(\LinearBase_j)_{j \in \LinearDimension{\LinearSubspace}}$ una
    base di $\LinearSubspace$;
  \item $(\LinearBase_j)_{j \in
    \LinearDimension{\LinearSpace} \SetMin \LinearDimension{\LinearSubspace}}$
    una base di $\VarLinearSubspace$. \EndProof
\end{itemize}
\begin{Theorem}
	Siano $\LinearSpace$ un $\Field$-spazio vettoriale di dimensione $n$ e sia $\LinearBase = (\LinearBase_i)_{i \in n}$ una base di $\LinearSpace$. Siano
	\begin{itemize}
		\item $f: \LinearBase \rightarrow \Field^n$ l'applicazione che a $\Vector \in \LinearBase$ associa le sue coordinate in $\Field^n$. $f$ \`e un isomorfismo;
		\item $g: \Field^n \rightarrow \LinearBase$ l'applicazione che a $(\Scalar_i)_{i \in n}$ associa il vettore $\sum_{i \in n} \Scalar_i \LinearBase_i$.
	\end{itemize}
	$f$ e $g$ sono isomorfismi lineari tra i $\Field$-spazi vettoriali $\LinearSpace$ e $\Field^n$, l'uno l'inverso dell'altro.
\end{Theorem}
\Proof Immediata per verifica diretta. \EndProof
\begin{Definition}
	Con le notazioni del teorema precedente, l'isomorfismo $f$ si chiama \Define{isomorfismo di coordinate}[di coordinate][isomorfismo] del $\Field$-spazio vettoriale $\LinearSpace$.
\end{Definition}
\begin{Theorem}
	Siano $\LinearSpace_1$ e $\LinearSpace_2$ $\Field$-spazi vettoriali di dimensione $n_1$ e $n_2$ rispettivamente. Siano $(\LinearBase_i)_{i \in n_1}$ e $(\VarLinearBase_j)_{j \in n_2}$ basi di di $\LinearSpace_1$ e $\LinearSpace_2$ rispettivamente. Denotiamo con $f_1$ e $f_2$ gli isomorfismi di coordinate di $\LinearSpace_1$ e $\LinearSpace_2$ rispettivamente. Siano
	\begin{itemize}
		\item $\Linear: \LinearSpace_1 \rightarrow \LinearSpace_2$ un'applicazione lineare;
		\item $A$ la matrice tale che $\ForAll{i \in n_1}{A^i = f_2(\Linear(\LinearBase_i))}$. Abbiamo $\ForAll{\Vector \in \LinearSpace_1}{A f_1(\Vector) = f_2(\Linear(\Vector))}$.
	\end{itemize}
\end{Theorem}
\Proof Sia $\Vector \in \LinearSpace_1$ e siano $(\Scalar_i)_{i \in n_1}$ le coordinate di $\Vector$ rispetto a $(\LinearBase_i)_{i \in n_1}$. Abbiamo $A f_1(\Vector) = A f_1(\sum_{i \in n_1} \Scalar_i \LinearBase_i) = \sum_{i \in n_1} \Scalar_i A^i = \sum_{i \in n_1} \Scalar_i f_2(\Linear(\LinearBase_i)) = f_2(\Linear(\Vector))$. \EndProof
\begin{Definition}
	Con le notazioni del teorema precedente, diciamo che la matrice $A$ \Define{rappresenta}[che rappresenta un'applicazione lineare][matrice] l'applicazione lineare $\Linear$ rispetto alle basi $(\LinearBase_i)_{i \in n_1}$ e $(\VarLinearBase_j)_{j \in n_2}$ o ancora che $A$ \`e la \Define{matrice associata}[associata a un'applicazione lineare][matrice] all'applicazione lineare $\Linear$ rispetto alle basi $(\LinearBase_i)_{i \in n_1}$ e $(\VarLinearBase_j)_{j \in n_2}$.
\end{Definition}
\begin{Definition}
	Sia $\LinearSpace$ uno spazio vettoriale di dimensione $n \in \mathbb{R}$ e siano $(\LinearBase_i)_{i \in n}$ e $(\VarLinearBase_j)_{j \in n}$ due sue basi distinte. La matrice associata all'identit\`a di $\LinearSpace$ rispetto alle basi $(\LinearBase_i)_{i \in n}$ e $(\VarLinearBase_j)_{j \in n}$ si chiama \Define{matrice di cambiamento di base}[di cambiamento di base][matrice] dalla base $(\LinearBase_i)_{i \in n}$ alla base $(\VarLinearBase_j)_{j \in n}$.
\end{Definition}
\begin{Theorem}
	Siano $\LinearSpace_0$ e $\LinearSpace_1$ due $\Field$-spazi vettoriali di dimensione $n_0$ e $n_1$ rispettivamente. Siano $(\LinearBase_i)_{i \in n_0}$ e $(\VarLinearBase_j)_{j \in n_2}$ basi di $\LinearSpace_0$ e $\LinearSpace_1$ rispettivamente. L'applicazione $f: \Homomorphisms{\LinearSpace_0}{\LinearSpace_1} \rightarrow \Field^{n_1 \times n_0}$ che a $\Linear \in \Homomorphisms{\LinearSpace_0}{\LinearSpace_1}$ associa la matrice che rappresenta $\Linear$ rispetto a $(\LinearBase_i)_{i \in n_0}$ e $(\LinearBase_j)_{j \in n_1}$ \`e un isomorfismo lineare. 
\end{Theorem}
\Proof
\begin{Definition}
	Siano $(\LinearSpace_i)_{i \in n}$ ($n \in \mathbb{N}$) una famiglia di $\Field$-spazi vettoriali e $\VarLinearSpace$ un ulteriore $\Field$-spazio vettoriale. Sia, per ogni $i \in n$,, $(\LinearBase_{i,j_i})_{j_i \in \Dimension{\LinearSpace_i}}$ una base di $\LinearSpace_i$ e sia $(\VarLinearBase_j)_{j \in \Dimension{\LinearSpace}}$ una base di $\LinearSpace$. Siano $\Multilinear \in \Multilinears{(\LinearSpace_i)_{i \in n}}{\VarLinearSpace}$ e $(\Scalar_{j_1,...,j_n}^j)_{(j_1,...,j_n,j) \in \bigtimes_{i \in n} \Dimension{\LinearSpace_i} \times \Dimension{\VarLinearSpace}}$ una famiglia di scalari tale che $\Multilinear((\LinearBase_{i,j_i})_{i \in n})) = \sum_{j \in \Dimension{\LinearSpace}} \Scalar_{j_1,...,j_n}^j \VarLinearBase_j$. Chiamiamo $(\Scalar_{j_1,...,j_n}^j)$ \Define{coordinate}[di un'applicazione multilineare][coordinate] di $\Multilinear$ rispetto alle basi fissate.
\end{Definition}
\begin{Theorem}
	Siano $(\LinearSpace_i)_{i \in n}$ ($n \in \mathbb{N}$) una famiglia di $\Field$-spazi vettoriali e $\VarLinearSpace$ un ulteriore $\Field$-spazio vettoriale. Sia, per ogni $i \in n$,, $(\LinearBase_{i,l})_{l \in \Dimension{\LinearSpace_i}}$ una base di $\LinearSpace_i$ e sia $(\VarLinearBase_j)_{j \in \Dimension{\LinearSpace}}$ una base di $\VarLinearSpace$. Una famiglia di scalari $(\Scalar_{j_1,...,j_n}^j)_{(j_1,...,j_n,j) \in \bigtimes_{i \in n} \Dimension{\LinearSpace_i} \times \Dimension{\LinearSpace}}$ costituisce le coordinate di una e una sola applicazione multilineare rispetto alle basi fissate e l'applicazione che associa a $(\Scalar_{j_1,...,j_n}^j)_{(j_1,...,j_n} \in \bigtimes_{i \in n} \Dimension{\LinearSpace_i} \times \Dimension{\LinearSpace}$ l'applicazione multilineare corrispondente \`e un isomorfismo di $\Field^{\Dimension{\LinearSpace} \prod_{i \in n}\Dimension{\LinearSpace_i}}$ su $\Multilinears{(\LinearSpace_i)_{i \in n}}{\VarLinearSpace}$.
\end{Theorem} 
\Proof Sia $(\Vector_i)_{i \in n}$ un generico elemento di $\bigtimes_{i \in n} \LinearSpace_i$. Per ogni $i \in n$, siano $(a_{i,j_i})_{j_i \in \Dimension{\LinearSpace_i}}$ le coordinate di $\Vector_i$ rispetto a $(\LinearBase_{i,j_i})_{j_i \in \Dimension{\LinearSpace_i}}$. Abbiamo $\Multilinear((\Vector_i)_{i \in n}) = \Multilinear \left ( \left (\sum_{j_i \in \Dimension{\LinearSpace_i}} a_{i,j_i} \LinearBase_{i,j_i} \right )_{i \in n} \right ) = \sum_{(j_1, ..., j_n) \in \bigtimes_{i \in n} \Dimension{\LinearSpace_i}} \prod_{i \in n} a_{i,j_i} \Multilinear((\LinearBase_i,j_i)_{i \in n}) = \sum_{(j_1,...,j_n) \in \bigtimes_{i \in n} \Dimension{\LinearSpace_i}} \prod_{i \in n} a_{i,j_i} \left ( \sum_{j \in \Dimension{\LinearSpace}} \Scalar_{j_1,...,j_n}^j \VarLinearBase_j \right )$.
\par D'altra parte, dati $(\Scalar_{j_1,...,j_n}^j)_{(j_1,...,j_n) \in \bigtimes_{i \in n} \Dimension{\LinearSpace_i} \times \Dimension{\LinearSpace}}$, si verifica facilmente essere lineare l'applicazione che a $(\Vector_i)_{i \in n} \in \bigtimes \LinearSpace_i$ associa $\sum_{(j_1,...,j_n) \in \bigtimes_{i \in n} \Dimension{\LinearSpace_i}} \prod_{i \in n} a_{i,j_i} \left ( \sum_{j \in \Dimension{\LinearSpace}} \Scalar_{j_1,...,j_n}^j \VarLinearBase_j \right )$, dove per per ogni $i \in n$ il vettore $(a_{i,j_i})_{j_i \in \Dimension{\LinearSpace_i}}$ \`e il vettore delle coordinate di $\Vector_i$ rispetto a $(\LinearBase_i)_{i \in \Dimension{\LinearSpace_i}}$.
\par Si verifica facilmente che la corrispondenza del teorema \`e lineare e che la corrispondenza che ad un'applicazione multilineare ne associa le coordinate ne \`e l'inversa. \EndProof
\begin{Corollary}
	Con le notazioni del teorema precedente, abbiamo $\Dimension{\Multilinears{(\LinearSpace_i)_{i \in n}}{\LinearSpace}} = \Dimension{\LinearSpace} \prod_{i \in n} \Dimension{\LinearSpace_i}$.
\end{Corollary}
\Proof Segue direttamente dall'isomorfismo del teorema precedente. \EndProof

\section{Funzionali.}\label{Funzionali}
\begin{Theorem}
	Sia $\LinearSpace$ un $\Field$-spazio vettoriale e sia $(\LinearBase_i)_{i \in n}$ ($n \in \mathbb{N}$)
	una base di $\LinearSpace$. Sia $(\LinearBase^i)_{i \in n}$ la famiglia
	di funzionali definita, per ogni $(i,j) \in n \times n$ da
	$\LinearBase^i(\LinearBase_j) =
	\begin{cases}
		0\text{, se }i \neq j,\\
		1\text{, se }i = j.
	\end{cases}$
	$(\LinearBase^i)_{i \in n}$ costituisce una base di
	$\Dual{\LinearSpace}$.
\end{Theorem}
\Proof Sia $(\Scalar_i)_{i \in n} \in \Field^n$ tale che
$\sum_{i \in n} \Scalar_i \LinearBase^i = 0$. Per ogni $j \in n$, abbiamo
$(\sum_{i \in n} \Scalar_i \LinearBase^i)(\LinearBase_j) =
\sum_{i \in n} \Scalar_i \LinearBase^i(\LinearBase_j) =
\Scalar_j$. Dunque, necessariamente, $\ForAll{i \in n}{\Scalar_i = 0}$.
\par Sia $\Functional \in \Dual{\LinearSpace}$ e sia $(\Scalar_i)_{i \in n} \in \Field^n$.
Abbiamo
$(\sum_{i \in n} \Functional(\LinearBase_i)\LinearBase^i)
(\sum_{i \in n} \Scalar_i\LinearBase_i)
= \sum_{i \in n} \Scalar_i \Functional(\LinearBase_i)
= \Functional(\Scalar_i \LinearBase_i)$. Dunque $(\LinearBase^i)_{i \in n}$
genera $\LinearSpace$. \EndProof
\begin{Definition}
	Con le notazioni del teorema precedente,
	$(\LinearBase^i)_{i \in n}$ si dice
	\Define{base duale}[duale][base] di $(\LinearBase_i)_{i \in n}$.
\end{Definition}
\begin{Theorem}
	Sia $\LinearSpace$ un $\Field$-spazio vettoriale e sia $(\LinearBase_i)_{i \in n}$
	una base di $\LinearSpace$. L'applicazione lineare $\Linear$ definita
	da $\ForAll{i \in n}{\Linear(\LinearBase_i) = \LinearBase^i}$ \`e un isomorfismo.
\end{Theorem}
\Proof $\Linear$ \`e un'applicazione lineare iniettiva tra spazi finiti. \EndProof
\begin{Lemma}
	Sia $\LinearSpace$ un $\Field$-spazio vettoriale. $\Linear'': \LinearSpace \rightarrow \Field^\Dual{\LinearSpace}$ l'applicazione definita da $\ForAll{\Vector \in \LinearSpace}{\ForAll{\Functional \in \Dual{\LinearSpace}}{{(\Linear''(\Vector))(\Functional) = \Functional(\Vector)}}}$. Abbiamo $\Image{\Linear''} \subseteq \Dual{\Dual{\LinearSpace}}$.
\end{Lemma}
\Proof Occorre dimostrare che, per ogni $\Vector \in \LinearSpace$, $\Linear''(\Vector)$ \`e lineare.
\par Siano $\Functional_0, \Functional_1 \in \Dual{\LinearSpace}$ e $\Scalar \in \Field$. Abbiamo
\begin{itemize}
	\item $\Linear''(\Vector)(\Functional_0 + \Functional_1) = (\Functional_0 + \Functional_1)(\Vector) = \Functional_0(\Vector) + \Functional_1(\Vector) = \Linear''(\Vector)(\Functional_0) + \Linear''(\Vector)(\Functional_1)$;
	\item $\Linear''(\Vector)(\Scalar \Functional_0) = (\Scalar \Functional_0)(\Vector) = \Scalar \Functional_0(\Vector) = \Scalar \Linear''(\Vector)(\Functional_0)$. \EndProof
\end{itemize} 
\begin{Theorem}
	Sia $\LinearSpace$ un $\Field$-spazio vettoriale e sia $(\LinearBase_i)_{i \in n}$ ($n \in \mathbb{N}$) una base di $\LinearSpace$. Denotiamo $(\LinearBase_{(i)})_{i \in n}$ la base duale di $(\LinearBase^i)_{i \in n}$. Siano
	\begin{itemize}
		\item $\Linear: \LinearSpace \rightarrow \Dual{\LinearSpace}$ l'applicazione lineare definita da $\ForAll{i \in n}{\Linear(\LinearBase_i) = \LinearBase^i}$;
		\item $\Linear': \Dual{\LinearSpace} \rightarrow \Dual{\Dual{\LinearSpace}}$ l'applicazione lineare definita da $\ForAll{i \in n}{\Linear'(\LinearBase^i) = \LinearBase_{(i)}}$.
		\item $\Linear'': \LinearSpace \rightarrow \Dual{\Dual{\LinearSpace}}$ l'applicazione definita da $\ForAll{\Vector \in \LinearSpace}{\ForAll{\Functional \in \Dual{\LinearSpace}}{{(\Linear''(\Vector))(\Functional) = \Functional(\Vector)}}}$.
	\end{itemize}
	Abbiamo
	\begin{itemize}
		\item $\Linear'' = \Linear' \circ \Linear$;
		\item $\Linear''$ \`e un isomorfismo;
		\item $\Linear''$ non dipende dalla base fissata $(\LinearBase_i)_{i \in n}$.
	\end{itemize}
\end{Theorem}
\Proof Sia $\Vector \in \LinearSpace$ e siano $(\Scalar_i)_{i \in n} \in \Field^n$ le coordinate di $\Vector$ rispetto a $(\LinearBase_i)_{i \in n}$. Inoltre, sia $\Functional \in \LinearSpace$ e siano $(\Scalar^i)_{i \in n} \in \Field^n$ le coordinate di $\Functional$ rispetto a $(\LinearBase^i)_{i \in n}$.
\par Abbiamo
$(\Linear' \circ \Linear)(\Vector)(\Functional) =
\Linear' \left ( \Linear \left ( \sum_{i \in n} \Scalar_i \LinearBase_i \right ) \right )(\Functional) =
\Linear' \left ( \sum_{i \in n} \Scalar_i \LinearBase^i \right )(\Functional) =
\left ( \sum_{i \in n} \Scalar_i \LinearBase_{(i)} \right ) \left ( \sum_{i \in n} \Scalar^i \LinearBase^i \right ) =
\sum_{i \in n} \Scalar_i \Scalar^i$,
ma anche
$\Linear''(\Vector)(\Functional) =
\Functional(\Vector) =
\left ( \sum_{i \in n} \Scalar_i \LinearBase^i \right ) \left ( \sum_{i \in n} \Scalar^i \LinearBase_i \right ) =
\sum_{i \in n} \Scalar_i \Scalar^i$.
Quindi effettivamente $\Linear'' = \Linear ' \circ \Linear$.
\par $\Linear''$ \`e allora un isomorfismo in quanto composizione di isomorfismi e non dipende dalla base fissata per costruzione. \EndProof
\begin{Definition}
	L'isomorfismo $\Linear''$ del teorema precedente si chiama
	\Define{isomorfismo canonico sul biduale}[canonico sul biduale][isomorfismo].
\end{Definition}
\par Nel seguito ogni spazio vettoriale verr\`a sempre identificato col suo biduale.
\begin{Theorem}
	Siano $(\LinearSpace_i)_{i \in n}$ ($n \in \mathbb{N}$) una famiglia di $\Field$-spazi vettoriali e $\LinearSpace$ un ulteriore $\Field$-spazio vettoriale. L'applicazione $\Linear: \Multilinears{(\LinearSpace_i)_{i \in n}}{\LinearSpace} \rightarrow \Multilinears{(\LinearSpace_{i \in n}),\Dual{\LinearSpace}}{\Field}$ che a $\Multilinear \in \Multilinears{(\LinearSpace_i)_{i \in n}}{\LinearSpace}$ associa la forma multilineare che a $((\Vector_i)_{i \in n},\Functional) \in \bigoplus_{i \in n}\LinearSpace_i \oplus \LinearSpace$ associa $\Functional(\Multilinear((\Vector_i)_{i \in n})) \in \Field$ \`e un isomorfismo, indipendente dalla scelta di qualsiasi base.
\end{Theorem}
\Proof Siano $\Multilinear_0, \Multilinear_1 \in \Multilinears{(\LinearSpace_i)_{i \in n}}{\LinearSpace}$. $\Linear(\Multilinear_0 + \Multilinear_1)$ \`e la forma multilineare che a $((\Vector_i)_{i \in n}, \Functional) \in \bigoplus_{i \in n} \LinearSpace_i \oplus \LinearSpace$ associa $\Functional((\Multilinear_0 + \Multilinear_1)((\Vector_i)_{i \in n})) = \Functional(\Multilinear_0((\Vector_i)_{i \in n})) + \Functional(\Multilinear_1((\Vector_i)_{i \in n})$, quindi effettivamente $\Linear(\Multilinear_0) + \Linear(\Multilinear_1) = \Linear(\Multilinear_0 + \Multilinear_1)$.
\par Sia inoltre $\Scalar \in \Field$. Allora $\Linear(\Scalar\Multilinear_0)$ \`e la forma multilineare che a $((\Vector_i)_{i \in n},\Functional)$ associa $\Functional((\Scalar \Multilinear_0)((\Vector_i)_{i \in n})) = \Scalar \Functional(\Multilinear_0((\Vector_i)_{i \in n}))$, quindi effettivamente $\Scalar \Linear(\Multilinear_0) = \Linear(\Scalar \Multilinear_0)$. $\Linear$ \`e pertanto un'applicazione multilineare.
\par Supponiamo ora $\Multilinear \in \Kernel{\Linear}$. Allora, $\ForAll{\Functional \in \Dual{\LinearSpace}}{\ForAll{(\Vector_i \in n) \in \bigoplus{i \in n} \LinearSpace_i}{\Functional(\Multilinear((\Vector_i)_{i \in n}) = 0}}$. Quindi $\ForAll{(\Vector_i \in n) \in \bigoplus{i \in n} \LinearSpace_i}{\Multilinear((\Vector_i)_{i \in n}) = 0}$, da quindi $\Multilinear = 0$.
\par $\Linear$ \`e allora un'applicazione lineare iniettiva tra spazi finiti equidimensionali, e pertanto un isomorfismo. Esso \`e indipendente dalla scelta di qualsiasi base per costruzione. \EndProof

\section{Tensori.}\label{Tensori}
\begin{Theorem}
	Siano $(\LinearSpace_i)_{i \in n}$ ($n \in \mathbb{N}$) $\Field$-spazi vettoriali e, per ogni $i \in n$, sia $(\LinearBase_{i,j_i})_{j_i \in \Dimension{\LinearSpace_i}}$ una base di $\LinearSpace_i$.
	Sia $\otimes: \bigtimes_{i \in n} \LinearSpace_i \rightarrow \MultilinearForms{(\Dual{\LinearSpace_i})_{i \in n}}$ l'applicazione che a $(\Vector_i)_{i \in n} \in \bigoplus_{i \in n} \LinearSpace_i$ associa $\bigotimes_{i \in n} \Vector_i \in \MultilinearForms{(\Dual{\LinearSpace_i})_{i \in n}}$ che a $(\Functional_i)_{i \in n} \in \bigoplus_{i \in n} \Dual{\LinearSpace_i}$ associa $\prod_{i \in n} \Functional_i(\Vector_i)$.
 Gli elementi $\left (\bigotimes_{i \in n} \LinearBase_{i,j_i} \right )_{(j_i)_{i \in n} \in \prod_{i \in n} \Dimension{\LinearSpace_i}}$, fissato un ordinamento totale, costituiscono una base di $\MultilinearForms{(\Dual{\LinearSpace_i})_{i \in n}}$.
\end{Theorem}
\Proof Siano $(\Scalar_{(j_i)_{i \in n}})_{(j_i)_{i \in n} \in \prod_{i \in n} \Dimension{\LinearSpace_i}}$ scalari tali che
$\sum_{(j_i)_{i \in n} \in \prod_{i \in n} \Dimension{\LinearSpace_i}} \Scalar_{(j_i)_{i \in n}} \bigotimes_{i \in n} \LinearBase_{i,j_i} = 0$.
\par Per ogni $i \in n$, denotiamo $(\LinearBase^{i,j_i})_{j_i \in \Dimension{\LinearSpace_i}}$ la base duale di $(\LinearBase_{i,j_i})_{j_i \in \Dimension{\LinearSpace_i}}$. Abbiamo, fissato $(J_i)_{i \in n} \in \prod_{i \in n} \Dimension{\LinearSpace_i}$:
$\left (\sum_{(j_i)_{i \in n} \in \prod_{i \in n} \Dimension{\LinearSpace_i}} \Scalar_{(j_i)_{i \in n}} \bigotimes_{i \in n} \LinearBase_{i,j_i} \right ) ( (\LinearBase^{i,J_i})_{i \in n} ) =
\sum_{(j_i)_{i \in n} \in \prod_{i \in n} \Dimension{\LinearSpace_i}} \Scalar_{(j_i)_{i \in n}} \bigotimes_{i \in n} \LinearBase_{i,j_i}( (\LinearBase^{i,J_i})_{i \in n} ) =
\sum_{(j_i)_{i \in n} \in \prod_{i \in n} \Dimension{\LinearSpace_i}} \Scalar_{(j_i)_{i \in n}} \bigotimes_{i \in n} \prod_{i \in n} \LinearBase^{i,J_i})(\LinearBase_{i,j_i}) =
\Scalar_{(J_i)_{i \in n}} = 0$.
\par Dunque $\left ( \bigotimes_{i \in n} \LinearBase_{i,j_i} \right )_{(j_i)_{i \in n} \in \prod_{i \in n} \Dimension{\LinearSpace_i}}$ \`e effettivamente una famiglia indipendente. Inoltre essa \`e costituita da $\prod_{i \in n} \Dimension{\LinearSpace_i}$ elementi, che \`e esattamente la dimensione di $\MultilinearForms{(\Dual{\LinearSpace_i})_{i \in n}}$, e dunque essa \`e una base. \EndProof
\begin{Corollary}
	Siano $(\LinearSpace_i)_{i \in n}$ ($n \in \mathbb{N}$) $\Field$-spazi vettoriali.
	Sia $\otimes: \bigtimes_{i \in n} \LinearSpace_i \rightarrow \MultilinearForms{(\Dual{\LinearSpace_i})_{i \in n}}$ l'applicazione che a $(\Vector_i)_{i \in n} \in \bigtimes_{i \in n} \LinearSpace_i$ associa $\bigotimes_{i \in n} \Vector_i \in \MultilinearForms{(\Dual{\LinearSpace_i})_{i \in n}}$ che a $(\Functional_i)_{i \in n} \in \bigtimes_{i \in n} \Dual{\LinearSpace_i}$ associa $\prod_{i \in n} \Functional_i(\Vector_i)$.
	$(\MultilinearForms{(\Dual{\LinearSpace_i})_{i \in n}}, \otimes)$ \`e il prodotto tensoriale $\bigotimes_{i \in n} \LinearSpace_i$.
\end{Corollary}
\Proof Siano $\VarLinearSpace$ un $\Field$-spazio vettoriale e $\Multilinear \in \Multilinears{(\LinearSpace_i)_{i \in n}}{\VarLinearSpace}$: occorre provare che esiste una e una sola applicazione lineare $\Linear: \MultilinearForms{(\Dual{\LinearSpace_i})_{i \in n}} \rightarrow \VarLinearSpace$ tale che $\Multilinear = \Linear \circ \otimes$.
\par Siano, per ogni $i \in n$, $(\LinearBase_{i,j_i})_{j_i \in \Dimension{\LinearSpace_i}}$ una base di $\LinearSpace_i$ e $(\VarLinearBase_j)_{j \in \Dimension{\VarLinearSpace}}$ una base di $\VarLinearSpace$. Definiamo $\Linear: \bigotimes_{i \in n} \LinearSpace_i \rightarrow \VarLinearSpace$ come l'unica applicazione lineare tale che $\ForAll{(j_i)_{i \in n} \in \prod_{i \in n} \Dimension{\LinearSpace_i}}{\Linear(\bigotimes_{i \in n} \LinearBase_{i,j_i}) = \Multilinear((\LinearBase_{i,j_i})_{i \in n})}$. Si verifica immediatamente che $\Multilinear = \Linear \circ \otimes$.
\par L'unicit\`a di $\Linear$ segue dalla necessit\`a della condizione $\ForAll{(j_i)_{i \in n} \in \prod_{i \in n} \Dimension{\LinearSpace_i}}{\Linear(\bigotimes_{i \in n} \LinearBase_{i,j_i}) = \Multilinear((\LinearBase_{i,j_i})_{i \in n})}$ perch\'e possa essere $\Multilinear = \Linear \circ \otimes$. \EndProof
\begin{Theorem}
	Siano $\LinearSpace$ e $\VarLinearSpace$ $\Field$-spazi vettoriali e sia $\Multilinear: \LinearSpace \times \VarLinearSpace \rightarrow \LinearSpace \otimes \VarLinearSpace$ l'applicazione che a $(\Vector,\VarVector) \in \LinearSpace \times \VarLinearSpace$ associa $\Vector \otimes \VarVector \in \LinearSpace \otimes \VarLinearSpace$. $\Multilinear$ \`e un'applicazione bilineare e $\ForAll{(\Vector,\VarVector) \in \LinearSpace \times \VarLinearSpace}{\Coimplies{\Vector \otimes \VarVector = 0}{\Or{\Vector = 0}{\VarVector = 0}}}$.
\end{Theorem}
\Proof Siano
\begin{itemize}
	\item $\Vector_0, \Vector_1 \in \LinearSpace$;
	\item $\VarVector \in \VarLinearSpace$;
	\item $\Scalar \in \Field$.
\end{itemize}
Abbiamo, per ogni $(\Functional,\VarFunctional) \in \Dual{\LinearSpace} \times \Dual{\VarLinearSpace}$,
\begin{itemize}
	\item $((\Vector_0 + \Vector_1) \otimes \VarVector)(\Functional,\VarFunctional) = \Functional(\Vector_0 + \Vector_1)\VarFunctional(\VarVector) = \Functional(\Vector_0)\VarFunctional(\VarVector) + \Functional(\Vector_1)\VarFunctional(\VarVector) = (\Vector_0 \otimes \VarVector)(\Functional,\VarFunctional) + (\Vector_1 \otimes \VarVector)(\Functional,\VarFunctional)$;
	\item $((\Scalar\Vector_0) \otimes \VarVector)(\Functional,\VarFunctional) = \Functional(\Scalar\Vector_0)\VarFunctional(\VarVector) = \Scalar\Functional(\Vector_0)\VarFunctional(\VarVector) = \Scalar(\Vector_0 \otimes \VarVector)(\Functional,\VarFunctional)$.
\end{itemize}
Dunque la prima applicazione parziale di $\Multilinear$ \`e lineare. Analogamente si dimostra lineare la seconda applicazione parziale di $\Multilinear$. Ne consegue che $\Multilinear$ \`e un'applicazione bilineare.
\par Infine $\Vector \otimes \VarVector = 0$ se e solo se $\ForAll{(\Functional,\VarFunctional) \in \Dual{\LinearSpace} \times \Dual{\VarLinearSpace}}{(\Vector \otimes \VarVector)(\Functional,\VarFunctional) = 0}$. Ma abbiamo $(\Vector \otimes \VarVector)(\Functional,\VarFunctional) = \Functional(\Vector)\VarFunctional(\VarVector) = 0$ se e solo se $\Functional(\Vector) = 0$ o $\VarFunctional(\VarVector) = 0$ e dunque se e solo se $\Vector = 0$ o $\VarVector = 0$. \EndProof
\begin{Corollary}
	Sia $\LinearSpace$ un $\Field$-spazio vettoriale finito. Sia $\Linear: \LinearSpace \rightarrow \LinearSpace \otimes \Field$ l'applicazione che a $\Vector \in \LinearSpace$ associa $\Vector \otimes 1$. $\Linear$ \`e un isomorfismo.
\end{Corollary}
\Proof Il teorema precedente dimostra che $\Linear$ \`e lineare e iniettiva. Inoltre $\LinearSpace$ e $\LinearSpace \otimes \Field$ sono equidimensionali. \EndProof
\begin{Definition}
	Sia $\LinearSpace$ un $\Field$-spazio vettoriale e sia $(h,k) \in \mathbb{N}$. Definiamo \Define{tensore di tipo $(h,k)$}[di tipo $(h,k)$][tensore] su $\LinearSpace$
	\begin{itemize}
		\item un qualsiasi scalare se $h = k = 0$;
		\item un elemento $\Tensor \in \left ( \bigotimes_{i \in h} \LinearSpace \right ) \otimes \left ( \bigotimes_{i \in k} \Dual{\LinearSpace} \right )$ altrimenti. 
	\end{itemize}
	Denoteremo nel seguito $\TensorsSpace{\LinearSpace}{h}{k} = \left ( \bigotimes_{i \in h} \LinearSpace \right ) \otimes \left ( \bigotimes_{i \in k} \Dual{\LinearSpace} \right )$ lo spazio di tutti i tensori di tipo $(h,k)$ su $\LinearSpace$.
	\par Inoltre, fissata una base $(\LinearBase_i)_{i \in \Dimension{\LinearSpace}}$ di $\LinearSpace$, denotiamo le cordinate di un tensore $\Tensor \in \TensorsSpace{\LinearSpace}{h}{k}$ rispetto alla base fissata con $\Tensor^{i_1,...,i_h}_{j_1,...,j_k}$ (nel campo degli scalari assumiamo la base costituita solo dall'unit\`a).
\end{Definition}
\begin{Example}
	Dato un $\Field$-spazio vettoriale $\LinearSpace$, abbiamo
	\begin{itemize}
		\item $\TensorsSpace{\LinearSpace}{0}{0} = \Field$;
		\item $\TensorsSpace{\LinearSpace}{1}{0} = \LinearSpace$;
		\item $\TensorsSpace{\LinearSpace}{0}{1} = \Dual{\LinearSpace}$;
		\item fissato $n \in \mathbb{N}$, $\TensorsSpace{\LinearSpace}{0}{n} = \MultilinearForms{(\LinearSpace)_{i \in n}}$.
	\end{itemize}
\end{Example}
\begin{Lemma}
	Siano $\LinearSpace$ e $\VarLinearSpace$ $\Field$-spazi vettoriali. Abbiamo $\Homomorphisms{\LinearSpace}{\VarLinearSpace} = \Dual{\LinearSpace} \otimes \VarLinearSpace$
\end{Lemma}
\Proof Abbiamo $\Dual{\LinearSpace} \otimes \VarLinearSpace = \MultilinearForms{\LinearSpace,\Dual{\VarLinearSpace}} = \Multilinears{\LinearSpace}{\VarLinearSpace} = \Homomorphisms{\LinearSpace}{\VarLinearSpace}$. \EndProof
\begin{Theorem}
	Sia $\LinearSpace$ un $\Field$-spazio vettoriale. Abbiamo $\TensorsSpace{\LinearSpace}{1}{1} = \Endomorphisms{\LinearSpace}$.
\end{Theorem}
\Proof Abbiamo $\TensorsSpace{\LinearSpace}{1}{1} = \LinearSpace \otimes \Dual{\LinearSpace} = \Dual{\LinearSpace} \otimes \LinearSpace = \Homomorphisms{\LinearSpace}{\LinearSpace} = \Endomorphisms{\LinearSpace}$. \EndProof
\begin{Definition}
	Dato un $\Field$-spazio vettoriale $\LinearSpace$, definiamo \Define{$\delta$ di Kronecker}[di Kronecker][$\delta$] il tensore di tipo $(1,1)$ denotato $\KroneckerDelta{i}{j}$ definito da $\KroneckerDelta{i}{j} = \begin{cases}0\text{ quando }i \neq j,\\1\text{ quando}i = j.\end{cases}$
\end{Definition}
\begin{Definition}
	Quando in un termine un indice appare due volte la \Define{notazione di Einstein}[di Einstein][notazione] prevede che si sottointenda davanti al termine un simbolo di sommatoria riferito proprio a quell'indice, il quale assume tutti i valori che hanno senso, salvo avvisi contrari.
\end{Definition}
\par In presenza di tensori, assumeremo sempre la notazione di Einstein, salvo avvisi contrari.

\section{Forme bilineari e sesquilineari.}
\label{GeometriaLineare_FormeBilineariESesquilineari}
\begin{Definition}
	Sia $\LinearSpace$ un $\Field$-spazio vettoriale. Si definisce \Define{forma quadratica}[quadratica][forma] un'applicazione $\QuadraticForm: \LinearSpace \rightarrow \Field$ tale che
	\begin{itemize}
		\item $\ForAll{(\Scalar,\Vector) \in \Field \times \LinearSpace}{\QuadraticForm(\Scalar \Vector) = \Scalar^2 \QuadraticForm(\Vector)}$;
		\item $\QuadraticForm(\Vector_0 + \Vector_1) - \QuadraticForm(\Vector_0) - \QuadraticForm(\Vector_1)$ \`e una forma bilineare.
	\end{itemize}
\end{Definition}
\begin{Theorem}
	Sia $\LinearSpace$ un $\Field$-spazio vettoriale e sia $\QuadraticForm: \LinearSpace \rightarrow \Field$ una forma quadratica. L'applicazione $\BilinearForm: \LinearSpace^2 \rightarrow \Field$ che a $(\Vector_0,\Vector_1) \in \LinearSpace^2$ associa $\frac{\QuadraticForm(\Vector_0 + \Vector_1) - \QuadraticForm(\Vector_0) - \QuadraticForm(\Vector_1)}{2}$ \`e una forma bilineare.
\end{Theorem}
\Proof Segue direttamente dalle definizioni. \EndProof
\begin{Definition}
	Con le notazioni del teorema precedente, $\BilinearForm$ si chiama \Define{forma bilineare associata}[bilineare associata][forma] alla forma quadratica $\QuadraticForm$.
\end{Definition}
\begin{Theorem}
	Sia $\LinearSpace$ un $\Field$-spazio vettoriale e sia $\BilinearForm: \LinearSpace^2 \rightarrow \Field$ una forma bilineare. L'applicazione $\QuadraticForm: \LinearSpace \rightarrow \Field$ che a $\Vector \in \LinearSpace$ associa $\BilinearForm(\Vector,\Vector)$: $\QuadraticForm$ \`e una forma quadratica.
\end{Theorem}
\Proof Siano $\Scalar \in \Field$ e $\Vector \in \LinearSpace$. Abbiamo $\QuadraticForm(\Scalar \Vector) = \BilinearForm(\Scalar \Vector, \Scalar \Vector) = \Scalar^2 \BilinearForm(\Vector,\Vector) = \Scalar^2\QuadraticForm(\Vector)$.
\par Sia $(\Vector_0,\Vector_1) \in \LinearSpace^2$. Abbiamo $\QuadraticForm(\Vector_0 + \Vector_1) - \QuadraticForm(\Vector_0) - \QuadraticForm(\Vector_1) = \BilinearForm(\Vector_0 + \Vector_1, \Vector_0 + \Vector_1) - \BilinearForm(\Vector_0, \Vector_0) - \BilinearForm(\Vector_1, \Vector_1) = \BilinearForm(\Vector_0,\Vector_0) + \BilinearForm(\Vector_1,\Vector_0) + \BilinearForm(\Vector_0,\Vector_1) + \BilinearForm(\Vector_1, \Vector_1) - \BilinearForm(\Vector_0,\Vector_0) - \BilinearForm(\Vector_1,\Vector_1) = \BilinearForm(\Vector_0,\Vector_1) + \BilinearForm(\Vector_1,\Vector_0)$, che si verifica facilmente essere una forma bilineare.
\begin{Definition}
	Con le notazioni del teorema precedente, $\QuadraticForm$ si chiama \Define{forma quadratica associata}[quadratica associata][forma] alla forma bilineare $\BilinearForm$.
\end{Definition}
\begin{Theorem}
	Sia $\LinearSpace$ un $\Field$-spazio vettoriale. Siano
	\begin{itemize}
		\item $\BilinearForm_0: \LinearSpace^2 \rightarrow \Field$ una forma bilineare simmetrica;
		\item $\QuadraticForm_1$ la forma quadratica associata alla forma bilineare $\BilinearForm_0$;
		\item $\BilinearForm_2$ la forma bilineare associata alla forma quadratica $\QuadraticForm_1$;
		\item $\QuadraticForm_0: \LinearSpace \rightarrow \Field$ una forma quadratica;
		\item $\BilinearForm_1$ la forma bilineare associata a $\QuadraticForm_0$;
		\item $\QuadraticForm_2$ la forma quadratica associata a $\BilinearForm_1$.
	\end{itemize}
	Abbiamo $\BilinearForm_0 = \BilinearForm_2$ e $\QuadraticForm_0 = \QuadraticForm_2$.
\end{Theorem}
\Proof Sia $(\Vector_0,\Vector_1) \in \LinearSpace^2$. Abbiamo
$\BilinearForm_2(\Vector_0,\Vector_1) =
\frac{\QuadraticForm_1(\Vector_0 + \Vector_1) - \QuadraticForm_1(\Vector_0) - \QuadraticForm_1(\Vector_1)}{2} =
\frac{\BilinearForm_0(\Vector_0 + \Vector_1, \Vector_0 + \Vector_1) - \BilinearForm_0(\Vector_0, \Vector_0) - \BilinearForm_0(\Vector_1,\Vector_1)}{2} =
\frac{\BilinearForm_0(\Vector_0,\Vector_1) + \BilinearForm_0(\Vector_1,\Vector_0)}{2} = \BilinearForm_0(\Vector_0,\Vector_1)$.
\par Sia $\Vector \in \LinearSpace$. Abbiamo
$\QuadraticForm_2(\Vector) =
\BilinearForm_1(\Vector,\Vector) =
\frac{\QuadraticForm_0(\Vector + \Vector) - \QuadraticForm_0(\Vector) - \QuadraticForm(\Vector)}{2} = 
\frac{4 \QuadraticForm_0(\Vector) - 2 \QuadraticForm_0(\Vector)}{2} =
\QuadraticForm_0(\Vector)$. \EndProof
\begin{Definition}
	Una forma quadratica $\QuadraticForm$ su un $\Field$-spazio vettoriale $\LinearSpace$ reale o complesso si dice
	\begin{itemize}
		\item \Define{semidefinita positiva}[quadratica semidefinita positiva][forma] quando $\ForAll{\Vector \in \LinearSpace}{\QuadraticForm(\Vector) \geq 0}$;
		\item \Define{semidefinita negativa}[quadratica semidefinita negativa][forma] quando $\ForAll{\Vector \in \LinearSpace}{\QuadraticForm(\Vector) \geq 0}$;
		\item \Define{definita positiva}[quadratica definita positiva][forma] quando \`e semidefinita positiva e inoltre $\Implies{\QuadraticForm(\Vector) = 0}{\Vector = 0}$;
		\item \Define{definita negativa}[quadratica definita negativa][forma] quando \`e semidefinita negativa e inoltre $\Implies{\QuadraticForm(\Vector) = 0}{\Vector = 0}$.
	\end{itemize}
	Una forma bilineare $\BilinearForm$ su un $\Field$-spazio vettoriale $\LinearSpace$ reale o complesso si dice \Define{semidefinita positiva}[bilineare semidefinita positiva][forma], \Define{semidefinita negativa}[bilineare semidefinata negativa], \Define{definita positiva}[bilineare definita positiva][forma] o \Define{definita negativa}[bilineare definita negativa][forma] quando la forma quadratica associata \`e, rispettivamente, semidefinita positiva, semidefinita negativa, definita positiva, definita negativa.
\end{Definition}
\begin{Definition}
	Siano $\LinearSpace_1$ e $\LinearSpace_2$ $\mathbb{C}$-spazi vettoriali. Si chiama \Define{applicazione antilineare}[antilineare][applicazione] un'applicazione $\Antilinear: \LinearSpace_1 \rightarrow \LinearSpace_2$ additiva e tale che $\ForAll{(x,\Scalar) \in (\LinearSpace_1,\mathbb{C})}{\Antilinear(\Scalar x) = \overline{\Scalar}\Antilinear(x)}$. Un'applicazione antilineare biettiva si chiama \Define{antisomorfismo}.
\end{Definition}
\begin{Definition}
	Dato un $\mathbb{C}$-spazio vettoriale $\LinearSpace$, sia $\SesquilinearForm: \LinearSpace^2 \rightarrow \mathbb{C}$ tale che \`e lineare nel primo argomento e antilineare nel secondo argomento: $\SesquilinearForm$ si chiama \Define{forma sesquilineare}[sesquilineare][forma] su $\LinearSpace$.
\end{Definition}
\begin{Definition}
	Siano $\Class$ una classe e $f: \Class^2 \rightarrow \mathbb{C}$ tale che $\ForAll{(x,y) \in \Class^2}{f(x,y) = \overline{f(y,x)}}$. Si dice che $f$ \`e un'\Define{applicazione coniugata simmetrica}[coniugata simmetrica][applicazione].
\end{Definition}
\begin{Definition}
	Dato un $\Field$-spazio vettoriale $\LinearSpace$ su cui \`e definita una forma bilineare o sesquilineare $\BilinearOrSesquilinearForm$, siano $\Vector, \VarVector \in \LinearSpace$. Se $\BilinearOrSesquilinearForm(\Vector,\VarVector) = 0$, diciamo che
	\begin{itemize}
		\item $\Vector$ \`e \Define{ortogonale a sinistra}[a sinistra][ortogonalit\`a] a $\VarVector$ rispetto a $\BilinearOrSesquilinearForm$;
		\item $\VarVector$ \`e \Define{ortogonale a destra}[a destra rispetto a una forma bilineare][ortogonalit\`a] a $\Vector$ rispetto a $\BilinearOrSesquilinearForm$.
	\end{itemize}
	Per ogni parte $\Part \subseteq \LinearSpace$, chiamiamo inoltre
	\begin{itemize}
		\item \Define{complemento ortogonale sinistro}[ortogonale sinistro][complemento] di $\Part$, denotata $\LeftOrthogonalComplement{\Part}$ la parte di $\LinearSpace$ di tutti i vettori ortogonali a sinistra a tutti i vettori di $\Part$;
		\item \Define{complemento ortogonale destro}[ortogonale destro][complemento] di $\Part$, denotata $\RightOrthogonalComplement{\Part}$ la parte di $\LinearSpace$ di tutti i vettori ortogonali a sinistra a tutti i vettori di $\Part$.
	\end{itemize}
	Definiamo ancora
	\begin{itemize}
		\item $\LeftOrthogonalComplement{\LinearSpace}$ \Define{radicale sinistro}[sinistro di uno spazio vettoriale][radicale] di $\LinearSpace$;
		\item $\RightOrthogonalComplement{\LinearSpace}$ \Define{radicale destro}[destro di uno spazio vettoriale][radicale] di $\LinearSpace$;
	\end{itemize}
\end{Definition}
\begin{Definition}
	Dato un $\Field$-spazio vettoriale $\LinearSpace$ su cui \`e definita una forma bilineare o sesquilineare $\BilinearOrSesquilinearForm$, se $\ForAll{(\Vector\,\VarVector) \in \LinearSpace^2)}{\Implies{\BilinearOrSesquilinearForm(\Vector,\VarVector) = 0}{\BilinearOrSesquilinearForm(\VarVector,\Vector) = 0}}$, diciamo che $\BilinearOrSesquilinearForm$ \`e riflessiva.
\end{Definition}
\begin{Theorem}
	Sia $\LinearSpace$ un $\Field$-spazio vettoriale. Se $\BilinearOrSesquilinearForm$ \`e una forma bilineare o sesquilineare riflessiva, allora ortogonalit\`a a sinistra e a destra coincidono.
\end{Theorem}
\Proof Segue direttamente dalle definizioni. \EndProof
\begin{Theorem}
	Sia $\LinearSpace$ un $\Field$-spazio vettoriale su cui \`e definita una forma bilineare o sesquilineare $\BilinearOrSesquilinearForm$ e sia $\Part \subseteq \LinearSpace$. $\LeftOrthogonalComplement{\Part}$ \`e un sottospazio vettoriale di $\LinearSpace$
\end{Theorem}
\Proof Si verifica immediatamente che se $\Vector, \VarVector \in \LeftOrthogonalComplement{\Part}$ e $\Scalar \in \Field$, allora $\Vector - \VarVector \in \LeftOrthogonalComplement{\Part}$ e $\Scalar \Vector \in \LeftOrthogonalComplement{\Part}$. \EndProof
\begin{Theorem}
	Sia $\LinearSpace$ un $\Field$-spazio vettoriale su cui \`e definita una forma bilineare o sesquilineare $\BilinearOrSesquilinearForm$ e sia $\Part \subseteq \LinearSpace$. Abbiamo $\Part \subseteq \RightOrthogonalComplement{(\LeftOrthogonalComplement{\Part})}$ e $\Part \subseteq \LeftOrthogonalComplement{(\RightOrthogonalComplement{\Part})}$.
\end{Theorem}
\Proof Abbiamo $\ForAll{\VarVector \in \LeftOrthogonalComplement{\Part}}{\ForAll{\Vector \in \Part}{\BilinearOrSesquilinearForm(\VarVector,\Vector) = 0}}$, da cui $\ForAll{\Vector \in \Part}{\ForAll{\VarVector \in \LeftOrthogonalComplement{\Part}}{\BilinearOrSesquilinearForm(\VarVector,\Vector) = 0}}$ e dunque $\Part \subseteq \RightOrthogonalComplement{(\LeftOrthogonalComplement{\Part})}$
\par La dimostrazione di $\Part \subseteq \RightOrthogonalComplement{(\LeftOrthogonalComplement{\Part})}$ \`e del tutto analoga. \EndProof
\begin{Definition}
	Sia $\LinearSpace$ un $\Field$-spazio vettoriale su cui \`e definita una forma
  bilineare o sesquilineare $\BilinearOrSesquilinearForm$. Diciamo che
  $\Part \subseteq \LinearSpace$ \`e
  \`e
	\begin{itemize}
		\item \Define{ortogonale}[ortogonale][parte] quando
      \[
        \ForAll{(\Vector,\VarVector) \in \Part}
          {\Implies{\Vector \neq \VarVector}
            {\BilinearOrSesquilinearForm(\Vector,\VarVector) = 0}}.
      \]
      Inoltre $\Part$ si dice \Define{completa}[ortogonale completa][parte] se
      \`e massimale;
		\item \Define{ortonormale}[ortonormale][parte] quando
      \[
        \ForAll{(\Vector,\VarVector) \in \Part}
          {\BilinearOrSesquilinearForm(\Vector,\VarVector)
            = \begin{cases}
                0\text{, se }\Vector \neq \VarVector;\\
                1\text{, se }\Vector = \VarVector.
              \end{cases}}
      \]
	\end{itemize}
\end{Definition}

\section{Prodotti interni.}
\label{GeometriaLineare_ProdottiInterni}
\begin{Definition}
	Una forma sesquilineare coniugata simmetrica definita positiva si chiama
  \Define{prodotto hermitiano}[hermitiano][prodotto] o
  \Define{forma hermitiana}[hermiatiana][forma].
  Si chiama
  \Define{prodotto scalare}[scalare][prodotto]
  una forma bilineare simmetrica definita positiva su uno spazio vettoriale
  reale.
  Chiamiamo inoltre
  \Define{prodotto interno}[interno][prodotto]
  un prodotto hermitiano o scalare.
\end{Definition}
\begin{Definition}
  Si chiama
  \Define{spazio prehilbertiano}[prehilbertiano][spazio]
  un $\Field$-spazio vettoriale $\PrehilbertSpace$
  ($\Field \in \lbrace \mathbb{R}, \mathbb{C} \rbrace$)
  munito di prodotto interno $\InnerProduct{\cdot}{\cdot}$.
\end{Definition}
\begin{Theorem}
	Siano $\PrehilbertSpace$ un $\Field$-spazio vettoriale con prodotto interno
  $\InnerProduct{\cdot}{\cdot}$
  di dimensione
  $n \in \mathbb{N}$ e sia
  $\VarHyperplane \subseteq \PrehilbertSpace$.
  $\VarHyperplane$ \`e un iperpiano di $\PrehilbertSpace$ se e solo se
  esiste $\NormalVector \in \PrehilbertSpace$ tale che
  $\NormalVector \cdot \LinearSpace = 0$.
\end{Theorem}
\begin{Theorem}
	Fissato $n \in \mathbb{N}$, l'applicazione
  $\HermitianProduct{\cdot}{\cdot}:
  \mathbb{C}^n \times \mathbb{C}^n \rightarrow \mathbb{C}$
  che a
  $(Z,W) \in \mathbb{C}^n \times \mathbb{C}^n$
  associa
  $\sum_{i \in n} Z_i \Conjugate{W_i}$
  \`e un prodotto hermitiano su $\mathbb{C}^n$.
\end{Theorem}
\Proof Siano $X \in \mathbb{C}^n$ e $\Scalar \in \Field$. Abbiamo
\begin{itemize}
	\item $\sum_{i \in n} (Z_i + X_i) \Conjugate{W_i}
          = \sum_{i \in n} Z_i \Conjugate{W_i}
          + \sum_{i \in n} X_i \Conjugate{W_i}$;
	\item $\sum_{i \in n} Z_i \Conjugate{(W_i + X_i)}
          = \sum_{i \in n} Z_i \Conjugate{W_i}
          + \sum_{i \in n} Z_i \Conjugate{X_i}$;
	\item $\sum_{i \in n} (\Scalar Z_i) \Conjugate{W_i}
          = \Scalar \sum_{i \in n} Z_i \Conjugate{W_i}$;
	\item $\sum_{i \in n} Z_i \Conjugate{\Scalar W_i}
          = \Conjugate{\Scalar} \sum_{i \in n} Z_i \Conjugate{W_i}$;
	\item $\Conjugate{\sum_{i \in n} Z_i \Conjugate{W_i}}
          = \sum_{i \in n} \Conjugate{Z_i} W_i$. \EndProof
\end{itemize}
\begin{Definition}
	Con le notazioni del teorema precedente, $\HermitianProduct{\cdot}{\cdot}$ si
  chiama
  \Define{prodotto hermitiano standard}[hermitiano strandard][prodotto]
  su $\mathbb{C}^n$ ($n \in \mathbb{N}$).
\end{Definition}
\begin{Theorem}
	Fissato $n \in \mathbb{N}$, l'applicazione
  $\ScalarProduct{\cdot}{\cdot}:
    \mathbb{R}^n \times \mathbb{R}^n \rightarrow \mathbb{R}$ che a
  $(X,Y) \in \mathbb{R}^n \times \mathbb{R}^n$ associa
  $\sum_{i \in n} X_i Y_i$ \`e un prodotto hermitiano su $\mathbb{R}^n$.
\end{Theorem}
\Proof Siano $Z \in \mathbb{R}^n$ e $\Scalar \in \Field$. Abbiamo
\begin{itemize}
	\item $\sum_{i \in n} (X_i + Z_i) Y_i
          = \sum_{i \in n} X_i Y_i + \sum_{i \in n} Z_i Y_i$;
	\item $\sum_{i \in n} X_i (Y_i + Z_i)
          = \sum_{i \in n} X_i Y_i + \sum_{i \in n} X_i Z_i$;
	\item $\sum_{i \in n} (\Scalar X_i) Y_i
          = \Scalar \sum_{i \in n} X_i Y_i$;
	\item $\sum_{i \in n} X_i (\Scalar Y_i)
          = \Scalar \sum_{i \in n} X_i Y_i$. \EndProof
\end{itemize}
\begin{Definition}
	Con le notazioni del teorema precedente,
  $\HermitianProduct{\cdot}{\cdot}$ si chiama
  \Define{prodotto hermitiano standard}[hermitiano strandard][prodotto]
  su $\mathbb{C}^n$ ($n \in \mathbb{N}$).
\end{Definition}
\begin{Definition}
	Sia $\NormedSpace$ un $\Field$-spazio vettoriale
  ($\Field \in \lbrace \mathbb{R}, \mathbb{C} \rbrace$) e sia
  $\Norm{\cdot}: \NormedSpace \rightarrow \RealNonNegative$
  un'applicazione tale che
	\begin{itemize}
		\item $\ForAll{(\Scalar,\Vector) \in \Field \times \NormedSpace}
            {\Norm{\Scalar \Vector}
              = \AbsoluteValue{\Scalar}\Vector}$
          (\Define{omogeneit\`a}[di una seminorma][omogeneit\`a]);
		\item $\ForAll{(\Vector,\VarVector) \in \NormedSpace^2}
            {\Norm{\Vector + \VarVector}
              \leq \Norm{\Vector} + \Norm{\VarVector}}$
          (\Define{disuguaglianza triangolare}[triangolare per una seminorma][disuguaglianza]).
	\end{itemize}
	$\Norm{\cdot}$ si chiama \Define{seminorma} su $\NormedSpace$ e
  $(\NormedSpace,\Norm{\cdot})$ si chiama
  \Define{spazio seminormato}[seminormato][spazio].
  Se inoltre $\Implies{\Norm{\Vector} = 0}{\Vector = 0}$,
  allora $\Norm{\cdot}$ si chiama \Define{norma} su $\NormedSpace$ e
  $(\NormedSpace,\Norm{\cdot})$ si chiama
  \Define{$\Field$-spazio normato}[normato][spazio].
  Definiamo inoltre \Define{unitario}[unitario][vettore] un vettore
  $\Vector \in \NormedSpace$ tale che $\Norm{\Vector} = 1$.
\end{Definition}
\begin{Theorem}
	Sia $\InnerProduct{\cdot}{\cdot}$ un prodotto interno definito su un
  $\Field$-spazio vettoriale $\NormedSpace$
  ($\Field \in \lbrace \mathbb{R}, \mathbb{C}$).
  $\Norm{\cdot}: \NormedSpace \rightarrow [0,\infty]$ definita come
  l'applicazione che a $\Vector \in \NormedSpace$ associa
  $\Norm{\Vector} = \sqrt{\InnerProduct{\Vector}{\Vector}}$ \`e una norma.
\end{Theorem}
\Proof Segue direttamente dalle definizioni. \EndProof
\begin{Definition}
	Con le notazioni del teorema precedente, $\Norm{\cdot}$ si chiama
  \Define{norma indotta}[indotta da prodotto interno][norma]
  dal prodotto interno $\InnerProduct{\cdot}{\cdot}$.
\end{Definition}
\begin{Theorem}
	Sia $\NormedSpace$ un $\Field$-spazio normato.
  L'applicazione
  $\Distance: \NormedSpace^2 \rightarrow \RealNonNegative$ che a
  $(\Vector,\VarVector) \in \NormedSpace^2$ associa
  $\Norm{\Vector - \VarVector}$.
  $\Distance$ \`e una distanza.
\end{Theorem}
\Proof Siano $\Vector_0, \Vector_1, \Vector_2 \in \NormedSpace$. Abbiamo
\begin{itemize}
	\item $\Norm{\Vector_0 - \Vector_0}
          = \Norm{0} = \Norm{0 \cdot 0} = 0\Norm{0} = 0$;
	\item $\Norm{\Vector_0 - \Vector_1}
          = \Norm{(-1)(\Vector_1 - \Vector_0)} = \Norm{\Vector_1 - \Vector_0}$;
	\item $\Norm{\Vector_0 - \Vector_1}
          = \Norm{\Vector_0 - \Vector_2 + \Vector_2 - \Vector_1}
          \leq \Norm{\Vector_0 - \Vector_2} + \Norm{\Vector_2 - \Vector_1}$.
\EndProof
\end{itemize}
\begin{Definition}
	Con le notazioni del teorema precedente, $\Distance$ si chiama
  \Define{distanza indotta}[indotta da una norma][distanza] da $\Norm{\cdot}$.
  Se $\Norm{\cdot}$ \`e indotta da un prodotto interno
  $\InnerProduct{\cdot}{\cdot}$, allora diciamo anche che $\Distance$ \`e
  \Define{indotta}[indotta da un prodotto interno][distanza].
\end{Definition}
\begin{Definition}
  Si chiama
  \Define{spazio di Hilbert}[di Hilbert][spazio] o
  \Define{hilbertiano}[hilbertiano][spazio]
  un $\Field$-spazio vettoriale $\HilbertSpace$
  ($\Field \in \lbrace \mathbb{R}, \mathbb{C} \rbrace$)
  tale che
  \begin{itemize}
    \item $\HilbertSpace$ \`e prehilbertiano con prodotto interno
          $\InnerProduct{\cdot}{\cdot}$;
    \item $\HilbertSpace$ completo rispetto alla distanza indotta da
          $\InnerProduct{\cdot}{\cdot}$.
  \end{itemize}
\end{Definition}
\begin{Theorem}
	\TheoremName{Disuguaglianza di Bessel}[di Bessel][disuguaglianza]
  Sia $\LinearSpace$ un $\Field$-spazio vettoriale su cui \`e definito un
  prodotto interno $\InnerProduct{\cdot}{\cdot}$.
  Sia $(\OrthonormalBase_i)_{i \in n}$ ($n \in \mathbb{N}$) una famiglia di
  vettori ortonormali. Per ogni $\Vector \in \LinearSpace$, abbiamo
  \[
    \sum_{i \in n}
      \AbsoluteValue{\InnerProduct{\Vector}{\OrthonormalBase_i}}
      \leq \Norm{\Vector}^2.
  \]
  Inoltre,
  $\VarVector =
    \Vector
    -\sum_{i \in n}\InnerProduct{\Vector}{\OrthonormalBase_i} \OrthonormalBase_i
    \in \OrthogonalComplement{
      \LinearSpan{\bigcup_{i \in n} \lbrace \OrthonormalBase_i \rbrace}}$.
\end{Theorem}
\Proof Abbiamo $0 \leq \Norm{\VarVector}^2 = \InnerProduct{\VarVector}{\VarVector} = \InnerProduct{\Vector}{\Vector} - 2 \sum_{i \in n} \AbsoluteValue{\InnerProduct{\Vector}{\OrthonormalBase_i}} + \sum_{i \in n} \AbsoluteValue{\InnerProduct{\Vector}{\OrthonormalBase_i}} = \Norm{\Vector} - \sum_{i \in n} \AbsoluteValue{\InnerProduct{\Vector}{\OrthonormalBase_i}}$.
\par Abbiamo inoltre $\InnerProduct{\VarVector}{\OrthonormalBase_i} = \InnerProduct{\Vector}{\OrthonormalBase_i} - \sum_{j \in n}\InnerProduct{\Vector}{\OrthonormalBase_j}\InnerProduct{\OrthonormalBase_i}{\OrthonormalBase_j} = \InnerProduct{\Vector}{\OrthonormalBase_i} - \Scalar{\Vector}{\OrthonormalBase_i} = 0$. \EndProof
\begin{Corollary}
	\TheoremName{Disuguaglianza di Cauchy-Schwarz}[di Cauchy-Schwarz][disuguaglianza]
  Sia $\LinearSpace$ un $\Field$-spazio vettoriale su cui \`e definito un
  prodotto interno. Abbiamo
  \[
    \ForAll{(\Vector,\VarVector) \in \LinearSpace^2}
      {\AbsoluteValue{\InnerProduct{\Vector}{\VarVector}}
      \leq \Norm{\Vector}\Norm{\VarVector}}.
  \]
\end{Corollary}
\Proof Siano $\Vector, \VarVector \in \LinearSpace$. Se $\VarVector = 0$, allora
\`e immediato verficare che
$\AbsoluteValue{\InnerProduct{\Vector}{\VarVector}}
\leq \Norm{\Vector}\Norm{\VarVector}$. Se invece $\VarVector \neq 0$, allora il
singoletto $\left \lbrace \frac{\VarVector}{\Norm{\VarVector}} \right \rbrace$
\`e ortonormale e la tesi segue direttamente dalla disuguaglianza di Bessel.
\EndProof
\begin{Theorem}
	Sia $\LinearSpace$ un $\Field$-spazio vettoriale su cui \`e definito un
  prodotto interno $\InnerProduct{\cdot}{\cdot}$. Sia
  $(\OrthonormalBase_i)_{i \in n}$ ($n \in \mathbb{N}$) una famiglia di vettori
  ortonormali. Sono equivalenti le seguenti affermazioni:
	\begin{itemize}
		\item $(\OrthonormalBase_i)_{i \in n}$ \`e una famiglia ortonormale completa;
		\item $\Implies{\ForAll{i \in n}{\InnerProduct{\Vector}{\OrthonormalBase_i} = 0}}{\Vector = 0}$;
		\item $\LinearSpan{\bigcup_{i \in n} \lbrace \OrthonormalBase_i \rbrace} = \LinearSpace$;
		\item $\ForAll{\Vector \in \LinearSpace}{\Vector = \sum_{i \in n} \InnerProduct{\Vector}{\OrthonormalBase_i}\OrthonormalBase_i}$;
		\item \TheoremName{Identit\`a di Parseval}[di Parseval][identit\`a] $\ForAll{(\Vector,\VarVector) \in \LinearSpace^2}{\InnerProduct{\Vector}{\VarVector} = \sum_{i \in n} \InnerProduct{\Vector}{\OrthonormalBase_i}\InnerProduct{\OrthonormalBase_i}{\VarVector}}$;
		\item $\ForAll{\Vector \in \LinearSpace}{\Norm{\Vector}^2 = \sum_{i \in n}\AbsoluteValue{\InnerProduct{\Vector}{\OrthonormalBase_i}}^2}$.
	\end{itemize}
\end{Theorem}
\Proof Assumiamo $(\OrthonormalBase_i)_{i \in n}$ sia una famiglia ortonormale completa e sia $\Vector \in \LinearSpace$ tale che $\ForAll{i \in n}{\InnerProduct{\Vector}{\OrthonormalBase_i} = 0}$. Se $\Vector \neq 0$, allora, ponendo $\OrthonormalBase_n = \frac{\Vector}{\Norm{\Vector}}$, $(\OrthonormalBase_i)_{i \in n + 1}$ \`e una famiglia ortonormale, in contraddizione con la completezza di $(\OrthonormalBase_i)_{i \in n}$. Dunque $\Vector = 0$.
\par Assumiamo $\Implies{\ForAll{i \in n}{\InnerProduct{\Vector}{\OrthonormalBase_i} = 0}}{\Vector = 0}$. Sia $\Vector \in \LinearSpace$. Se $\Vector \notin \LinearSpan{(\OrthonormalBase_i)_{i \in n}}$, allora $\Vector - \sum_{i \in n} \InnerProduct{\Vector}{\OrthonormalBase_i} \neq 0$. Ma dalla disuguaglianza di Bessel deduciamo anche $\ForAll{j \in n}{\InnerProduct{\Vector - \sum_{i \in n} \InnerProduct{\Vector}{\OrthonormalBase_i}}{\OrthonormalBase_j} = 0}$, da cui $\Vector - \sum_{i \in n} \InnerProduct{\Vector}{\OrthonormalBase_i} = 0$. Dunque deve essere $\Vector \in \LinearSpan{(\OrthonormalBase_i)_{i \in n}}$.
\par Assumiamo che $\LinearSpan{\bigcup_{i \in n} \lbrace \OrthonormalBase_i \rbrace} = \LinearSpace$. Sia $\Vector \in \LinearSpace$ e siano $(\Scalar_i)_{i \in n} \in \Field^n$ tali che $\Vector = \sum_{i \in n} \Scalar_i \OrthonormalBase_i$. Per ogni $j \in n$, abbiamo $\InnerProduct{\Vector}{\OrthonormalBase_j} = \sum_{i \in n} \Scalar_i \InnerProduct{\OrthonormalBase_i}{\OrthonormalBase_j} = \Scalar_j$.
\par Assumiamo $\ForAll{\Vector \in \LinearSpace}{\Vector = \sum_{i \in n} \InnerProduct{\Vector}{\OrthonormalBase_i}\OrthonormalBase_i}$.Siano $\Vector, \VarVector \in \LinearSpace$. Abbiamo $\InnerProduct{\Vector}{\VarVector} = \InnerProduct{\sum_{i \in n} \InnerProduct{\Vector}{\OrthonormalBase_i}\OrthonormalBase_i}{\sum_{i \in n} \InnerProduct{\VarVector}{\OrthonormalBase_i}\OrthonormalBase_i} = \sum_{i \in n} \InnerProduct{\Vector}{\OrthonormalBase_i}\InnerProduct{\OrthonormalBase_i}{\VarVector}$.
\par Assumiamo $\ForAll{(\Vector,\VarVector) \in \LinearSpace^2}{\InnerProduct{\Vector}{\VarVector} = \sum_{i \in n} \InnerProduct{\Vector}{\OrthonormalBase_i}\InnerProduct{\OrthonormalBase_i}{\VarVector}}$. Ponendo $\Vector = \VarVector$, otteniamo direttamente $\ForAll{\Vector \in \LinearSpace}{\Norm{\Vector}^2 = \sum_{i \in n}{\AbsoluteValue{\InnerProduct{\Vector}{\OrthonormalBase_i}}}^2}$.
\par Assumiamo infine $\ForAll{\Vector \in \LinearSpace}{\Norm{\Vector}^2 = \sum_{i \in n}\AbsoluteValue{\InnerProduct{\Vector}{\OrthonormalBase_i}}^2}$. Supponiamo $\OrthonormalBase_n \in \LinearSpace$ tale che $(\OrthonormalBase_i)_{i \in n + 1}$ sia ortonormale. Allora $\Norm{\OrthonormalBase_n}^2 = \sum_{i \in n} \AbsoluteValue{\InnerProduct{\OrthonormalBase_n}{\OrthonormalBase_i}}^2 = 0 \neq 1$. Quindi $(\OrthonormalBase_i)_{i \in n + 1}$ non pu\`o essere ortonormale e $(\OrthonormalBase_i)_{i \in n}$ deve essere completo. \EndProof
\begin{Theorem}
	\TheoremName{Ortogonalizzazione di Gram-Schmidt}[di Gram-Schmidt][ortogonalizzazione]
  Sia $\LinearSpace$ un $\Field$-spazio vettoriale finito di dimensione
  $n = \Dimension{\LinearSpace}$ su cui \`e definito un prodotto interno
  $\InnerProduct{\cdot}{\cdot}$. Sia $(\LinearBase_i)_{i \in n}$ una base di
  $\LinearSpace$. Poniamo
	\begin{itemize}
		\item $\OrthogonalBase_0 = \LinearBase_0$;
		\item per ogni $i \in n$ con $i > 0$,
      $\OrthogonalBase_i
        = \LinearBase_i - \sum_{j \in i}
          \InnerProduct{\LinearBase_i}{\OrthogonalBase_j}\OrthogonalBase_j$;
		\item per ogni $i \in n$,
      $\OrthonormalBase_i = \frac{\OrthogonalBase_i}{\Norm{\OrthogonalBase_i}}$.
	\end{itemize}
	Per ogni $m \in n$, $(\OrthogonalBase_i)_{i \in m}$ \`e una base ortogonale di
  $\LinearSpan{\bigcup_{i \in m} \lbrace \LinearBase_i \rbrace}$
  e
  $(\OrthonormalBase_i)_{i \in n}$ una base ortonormale di
  $\LinearSpan{\bigcup_{i \in m} \lbrace \LinearBase_i \rbrace}$.
\end{Theorem}
\Proof Proviamo per induzione su $m$ che $(\OrthogonalBase_i)_{i \in m}$ \`e una base di $\LinearSpace{\bigcup_{i \in m} \lbrace \LinearBase_i \rbrace}$.
\par Se $m = 1$, la tesi \`e immediata. Assumiamola dimostrata per $m = N$ e proviamola per $m = N + 1$.
\par Abbiamo $\LinearBase_N = \OrthogonalBase_N + \sum_{j \in N} \InnerProduct{\LinearBase_{N + 1}}{\OrthogonalBase_i}\OrthogonalBase_j$, dunque $\LinearSpan{\bigcup_{i \in N + 1} \lbrace \LinearBase_i \rbrace} \subseteq \LinearSpan{\bigcup_{i \in N + 1} \lbrace \OrthogonalBase_i \rbrace}$. Considerando le dimensioni degli spazi generati e il numero dei loro generatori, vediamo che deve essere addrittura $\LinearSpan{\bigcup_{i \in N + 1} \lbrace \LinearBase_i \rbrace} = \LinearSpan{\bigcup_{i \in N + 1} \lbrace \OrthogonalBase_i \rbrace}$ e che $(\OrthogonalBase_i)_{i \in N + 1}$ deve essere una base di $\LinearSpan{\bigcup_{i \in N + 1} \lbrace \OrthogonalBase_i \rbrace}$.
\par Si verifica immediatamente che, per ogni $m \in n$, $(\OrthonormalBase_i)_{i \in n}$ \`e una base ortonormale di $\LinearSpan{\bigcup_{i \in n} \lbrace \LinearSpace_i \rbrace}$. \EndProof
\begin{Theorem}
	Sia $\LinearSpace$ un $\Field$-spazio vettoriale finito di dimensione
  $n = \Dimension{\LinearSpace}$ su cui \`e definito un prodotto interno
  $\InnerProduct{\cdot}{\cdot}$ e sia $\VarLinearSpace \subseteq \LinearSpace$
  un sottospazio vettoriale di $\LinearSpace$ di dimensione
  $m = \Dimension{\VarLinearSpace}$. Abbiamo
  $\LinearSpace
    = \VarLinearSpace \oplus \OrthogonalComplement{\VarLinearSpace}$.
\end{Theorem}
\Proof Sia $(\OrthonormalBase_i)_{i \in n}$ una base ortonormale di $\LinearSpace$ tale che $(\OrthonormalBase_i)_{i \in m}$ sia una base ortonormale di $\VarLinearSpace$. Definiamo $\Projection: \LinearSpace \rightarrow \LinearSpace$ che a $\Vector \in \LinearSpace$ associa $\Vector - \sum_{i \in m} \InnerProduct{\Vector}{\OrthonormalBase_i}\OrthonormalBase_i$. \`E immediato verificate che $\Projection$ \`e una proiezione e dunque $\LinearSpace = \Kernel{\Projection} \oplus \Image{\Projection}$. Ma abbiamo appunto $\Kernel{\Projection} = \VarLinearSpace$ e $\Image{\Projection} = \OrthogonalComplement{\VarLinearSpace}$. \EndProof
\begin{Corollary}
	Nelle ipotesi del teorema precedente, $\VarLinearSpace = \OrthogonalComplement{\OrthogonalComplement{\VarLinearSpace}}$.
\end{Corollary}
\Proof Abbiamo gi\`a provato che $\VarLinearSpace \subseteq \OrthogonalComplement{\OrthogonalComplement{\VarLinearSpace}}$. Ma per il teorema precedente, i due spazi sono equidimensionali, e dunque devono coincidere. \EndProof
\begin{Theorem}
	\TheoremName{Teorema di rappresentazione di Riesz}[di rappresentazione di Riesz][teorema]
  Sia $\LinearSpace$ un $\Field$-spazio vettoriale finito su cui \`e definito un
  prodotto interno e sia $\Dual{y} \in \Dual{\LinearSpace}$.
  Esiste uno e un solo vettore $y \in \LinearSpace$ tale che
  $\ForAll{x \in \LinearSpace}{\InnerProduct{x}{y} = \Dual{y}(x)}$.
\end{Theorem}
\Proof Proviamo prima l'esistenza. Se $\Dual{y} = 0$, allora $y = 0$ verifica la propriet\`a richiesta. Supponiamo dunque $\Dual{y} \neq 0$.
\par Sia $\VarLinearSpace = \Kernel{\Dual{y}}$. Poich\'e $\Dual{y} \neq 0$, esiste $y^\perp \in \OrthogonalComplement{\VarLinearSpace}$ unitario. Poniamo $y = \Conjugate{\Dual{y}(y^\perp)}y^\perp$.
\par Sia $x \in \LinearSpace$ e poniamo $x_0 = x - \frac{\Dual{y}(x)}{\Dual{y}(y^\perp)}y^\perp$. Abbiamo allora
\begin{itemize}
	\item $\Dual{y}(x_0) = \Dual{y}(x) - \frac{\Dual{y}(x)}{\Dual{y}(y^\perp)}\Dual{y}(y^\perp) = 0$ e quindi $\InnerProduct{x_0}{y^\perp} = 0$;
	\item $\InnerProduct{x}{y} = \InnerProduct{x_0 + \frac{\Dual{y}(x)}{\Dual{y}(y^\perp)}y^\perp}{\Conjugate{\Dual{y}(y^\perp)}y^\perp} = \Dual{y}(y^\perp)\InnerProduct{x_0}{y^\perp} + \Dual{y}(x) = \Dual{y}{x}$.
\end{itemize}
\par Siano $y_0$ e $y_1$ tali che, per ogni $x \in \LinearSpace$, $\InnerProduct{x}{y_0} = \InnerProduct{x}{y_1} = \Dual{y}(x)$. Allora, ponendo $x = y_0 - y_1$, abbiamo $\Norm{y_0 - y_1}^2 = 0$, da cui $y_0 = y_1$. \EndProof
\begin{Corollary}
	Con le ipotesi del teorema precedente, l'applicazione $\Linear: \Dual{\LinearSpace} \rightarrow \LinearSpace$ che a $\Dual{y} \in \Dual{\Linear}$ associa l'unico vettore $y \in \LinearSpace$ tale che $\ForAll{x \in \LinearSpace}{\Dual{y}(x) = \InnerProduct{x}{y}}$. $\Linear$ \`e un (anti)isomorfismo.
\end{Corollary}
\Proof Siano $\Dual{y}, \Dual{z} \in \Dual{\LinearSpace}$, $y, z \in \LinearSpace$ tali che $y = \Linear(\Dual{y})$ e $z = \Linear(\Dual{z})$ e $\Scalar \in \Field$. Abbiamo, per ogni $x \in \LinearSpace$,
\begin{itemize}
	\item $(\Dual{y} + \Dual{z})(x) = \Dual{y}(x) + \Dual{z}(x) = \InnerProduct{x}{y} + \InnerProduct{x}{z} = \InnerProduct{x}{y + z}$;
	\item $(\Scalar \Dual{y})(x) = \Scalar \Dual{y}(x) = \InnerProduct{x}{\Conjugate{\Scalar}y}$.
\end{itemize}
\par Quindi $\Linear$ \`e un'applicazione (anti)lineare. \`E anche invertibile: la sua inversa \`e l'applicazione che a $y \in \LinearSpace$ associa $\Dual{y} \in \Dual{\LinearSpace}$ tale che $\ForAll{x \in \LinearSpace}{\Dual{y}(x) = \InnerProduct{x}{y}}$. \EndProof
\begin{Corollary}
	Con le notazioni del corollario precedente, per ogni $\Part \subseteq \LinearSpace$, abbiamo $\Dual{y} \in \Annihilator{\Part}$ se e solo se $y = \Linear(\Dual{y}) \in \OrthogonalComplement{\Part}$.
\end{Corollary}
\Proof Segue direttamente dalle definizioni e dal teorema di rappresentazione. \EndProof
\begin{Definition}
  Siano $\PrehilbertSpace$ e $\VarPrehilbertSpace$ spazi prehilbertiani con
  prodotto interni entrambi denotati
  $\InnerProduct{\cdot}{\cdot}$.
  Sia inoltre
  $\Linear: \PrehilbertSpace \rightarrow \VarPrehilbertSpace$
  un'applicazione lineare.
  Definiamo
  \Define{applicazione aggiunta}[aggiunta][applicazione]
  \footnote{Il concetto di applicazione aggiunta \`e molto frequente in tema di
  operatori, motivo per cui tratteremo pi\`u spesso di
  \Define{operatori aggiunti}[aggiunto][operatore] pi\`u che di
  applicazioni aggiunte.}
  di $\Linear$, denotata $\Adjoint{\Linear}$, l'applicazione
  $\Adjoint{\Linear}: \VarPrehilbertSpace \rightarrow \PrehilbertSpace$
  che a $\VarVector \in \VarPrehilbertSpace$ associa il vettore
  $\Adjoint{\Linear}\VarVector$ che rappresenta il funzionale di
  $\Dual{\PrehilbertSpace}$ che a $\Vector \in \PrehilbertSpace$ associa
  $\InnerProduct{\Linear\Vector}{\VarVector}$.
  \par Nel caso in cui $\Linear = \Adjoint{\Linear}$ diciamo che $\Linear$ \`e
  un'\Define{applicazione autoaggiunta}[autoggiunta][applicazione]
  \footnote{Per lo stesso motivo illustrato nella nota precedente, tratteremo
  frequentemente di
  \Define{operatori autoaggiunti}[autoaggiunti][operatori].}
\end{Definition}
\begin{Theorem}
  Siano $\PrehilbertSpace$ e $\VarPrehilbertSpace$ spazi prehilbertiani con
  prodotto interni entrambi denotati
  $\InnerProduct{\cdot}{\cdot}$.
  Sia inoltre
  $\Linear: \PrehilbertSpace \rightarrow \VarPrehilbertSpace$
  un'applicazione lineare.
  Abbiamo, per ogni $\Vector, \VarVector \in \PrehilbertSpace$,
  \[
    \InnerProduct{\Linear\Vector}{\VarVector}
      = \InnerProduct{\Vector}{\Adjoint{\Linear}\VarVector}.
  \]
\end{Theorem}
\Proof Segue direttamente dalla definizione. \EndProof
\begin{Corollary}
  Siano $\PrehilbertSpace_0$, $\PrehilbertSpace_1$ e $\PrehilbertSpace_2$
  $\Field-$spazi prehilbertiani con prodotto interni tutti denotati
  $\InnerProduct{\cdot}{\cdot}$.
  Siano inoltre
  \begin{itemize}
    \item $\Linear_0: \PrehilbertSpace_0 \rightarrow \PrehilbertSpace_1$,
      $\VarLinear_0: \PrehilbertSpace_0 \rightarrow \PrehilbertSpace_1$,
      e $\Linear_1: \PrehilbertSpace_1 \rightarrow \VarPrehilbertSpace_2$
      applicazioni lineari;
    \item $\Scalar \in \Field$.
  \end{itemize}
  Abbiamo
  \begin{itemize}
    \item $\Adjoint{\Adjoint{\Linear_0}} = \Linear_0$;
    \item $\Adjoint{(\Linear_0 + \VarLinear_0)}
            = \Adjoint{\Linear_0} + \Adjoint{\VarLinear_0}$;
    \item $\Adjoint{(\Scalar\Linear_0)} = \Conjugate{\Scalar}\Adjoint{\Linear_0}$;
    \item $\Adjoint{(\Linear_1\Linear_0)}
            = \Adjoint{\Linear_0}\Adjoint{\Linear_1}$.
  \end{itemize}
\end{Corollary}
\Proof Siano $\Vector_0 \in \PrehilbertSpace_0$,
$\Vector_1 \in \PrehilbertSpace_1$ e $\Vector_2 \in \PrehilbertSpace_2$.
\par Abbiamo
\begin{itemize}
  \item $\InnerProduct{\Linear\Vector_0}{\Vector_1}
    = \InnerProduct{\Vector_0}{\Adjoint{\Linear}\Vector_1}
    = \Conjugate{\InnerProduct{\Adjoint{\Linear}\Vector_1}{\Vector_0}}
    = \Conjugate{\InnerProduct{\Vector_1}{\Adjoint{\Adjoint{\Linear}}\Vector_0}}
    = \InnerProduct{\Adjoint{\Adjoint{\Linear}}\Vector_0}{\Vector_1}$,
    dunque, per l'arbitrariet\`a di $\Vector_0$ e $\Vector_1$, deve essere
    $\Linear = \Adjoint{\Adjoint{\Linear}}$;
  \item $\InnerProduct{\Vector_0}{\Adjoint{(\Linear_0 + \VarLinear_0)}\Vector_1}
    = \InnerProduct{(\Linear_0 + \VarLinear_0)\Vector_0}{\Vector_1}
    = \InnerProduct{\Linear_0\Vector_0}{\Vector_1}
      + \InnerProduct{\VarLinear_0\Vector_0}{\Vector_1}
    = \InnerProduct{\Vector_0}{\Adjoint{\Linear_0}\Vector_1}
      + \InnerProduct{\Vector_0}{\Adjoint{\VarLinear_0}\Vector_1}
    = \InnerProduct{\Vector_0}
      {\Adjoint{\Linear_0}\Vector_1 + \Adjoint{\VarLinear_0}\Vector_1}$,
    dunque, per l'arbitrariet\`a di $\Vector_0$ e $\Vector_1$, deve essere
    $\Adjoint{(\Linear_0 + \VarLinear_0)}
      = \Adjoint{\Linear_0} + \Adjoint{\VarLinear_0}$;
  \item $\InnerProduct{\Vector_0}{\Adjoint{(\Scalar\Linear_0)\Vector_1}}
    = \InnerProduct{\Scalar\Linear_0\Vector_0}{\Vector_1}
    = \InnerProduct{\Linear_0\Vector_0}{\Conjugate{\Scalar}\Vector_1}
   =\InnerProduct{\Vector_0}{\Conjugate{\Scalar}\Adjoint{\Linear_0}\Vector_1}$,
    dunque, per l'arbitrariet\`a di $\Vector_0$ e $\Vector_1$, deve essere
    $\Adjoint{(\Scalar\Linear_0)} = \Conjugate{\Scalar}\Adjoint{\Linear_0}$;
  \item $\InnerProduct{\Vector_0}{\Adjoint{(\Linear_1\Linear_0)}\Vector_2}
    = \InnerProduct{\Linear_1\Linear_0\Vector_0}{\Vector_2}
    = \InnerProduct{\Linear_0\Vector_0}{\Adjoint{\Linear_1}\Vector_2}
    =\InnerProduct{\Vector_0}{\Adjoint{\Linear_0}\Adjoint{\Linear_1}\Vector_2}$,
    dunque, per l'arbitrariet\`a di $\Vector_0$ e $\Vector_2$, deve essere
    $\Adjoint{(\Linear_1\Linear_0)} = \Adjoint{\Linear_0}\Adjoint{\Linear_1}$.
    \EndProof
\end{itemize}

\section{Determinanti}
\label{GeometriaLineare_Determinante}
\begin{Theorem}
	Sia $\LinearSpace$ un $\Field$-spazio vettoriale finito di dimensione $n = \Dimension{\LinearSpace}$. Abbiamo, per ogni $k \in \NotZero{\mathbb{N}}$ $\Dimension{\AlternatingForms{(\LinearSpace)_{i \in k}}} = \binom{k}{n}$.
\end{Theorem}
\Proof Fissiamo una base $(\LinearBase_j)_{j \in n}$ di $\LinearSpace$. Ricordiamo che $\Dimension{\MultilinearForms{(\LinearSpace)_{i \in k}}} = n^k$ e che una forma multilineare \`e interamente determinata dai suoi $n^k$ coefficienti, cio\`e dai valori che essa assume in ogni $k$-upla di elementi di $(\LinearBase_j)_{j \in n}$.
\par Osserviamo che, fissato un coefficiente al valore $\Scalar \in \Field$, sono automaticamente fissati tutti i $\Factorial{k}$ coefficienti che coinvolgono gli stessi vettori di $(\LinearBase_i)_{i \in n}$ in ordine diverso, senza ulteriori vincoli. Dunque abbiamo appunto $\Dimension{\AlternatingForms{(\LinearSpace)_{i \in k}}} = \binom{k}{n}$. \EndProof
\begin{Theorem}
	Sia $\LinearSpace$ un $\Field$-spazio vettoriale di dimensione finita
  $n = \Dimension{\LinearSpace}$.
  Siano $\LinearTransformation \in \Endomorphisms{\LinearSpace}$.
  Esiste uno e un solo scalare
  $\Determinant(\LinearTransformation)$ tale che, per ogni
  $\AlternatingForm \in \AlternatingForms{(\LinearSpace)_{i \in n}}$,
  abbiamo
  $\ForAll{(\Vector_i)_{i \in n} \in \LinearSpace^n}
    {\AlternatingForm((\LinearTransformation(\Vector_i))_{i \in n})
  =\Determinant(\LinearTransformation)\AlternatingForm((\Vector_i)_{i \in n})}$.
\end{Theorem}
\Proof Osserviamo che, per ogni $\AlternatingForm \in \AlternatingForms{(\LinearSpace)_{i \in n}}$, $\VarAlternatingForm((\Vector_i)_{i \in n}) = \AlternatingForm((\LinearTransformation(\Vector_i))_{i \in n})$ \`e una forma alternante. Sia l'applicazione $\Linear: \AlternatingForms{(\LinearSpace)_{i \in n}} \rightarrow \AlternatingForms{(\LinearSpace)_{i \in n}}$ che a $\AlternatingForm \in \AlternatingForms{(\LinearSpace)_{i \in n}}$ associa $\VarAlternatingForm \in \AlternatingForms{(\LinearSpace)_{i \in n}}$ tale che $\ForAll{(\Vector_i)_{i \in n}}{\VarAlternatingForm((\Vector_i)_{i \in n}) = ((\LinearTransformation(\Vector_i))_{i \in n})}$. 
\par Si verifica immediatamente che $\Linear$ \`e lineare e, ricordando che $\Dimension{\AlternatingForms{(\LinearSpace)_{i \in n}}} = 1$ deve esistere un unico scalare $\Determinant(\LinearTransformation)$ tale che $\ForAll{(\Vector_i)_{i \in n} \in \LinearSpace^n}{\AlternatingForm((\LinearTransformation(\Vector_i))_{i \in n}) = \Determinant(\LinearTransformation)\AlternatingForm((\Vector_i)_{i \in n})}$. \EndProof
\begin{Definition}
	Con le notazioni del teorema precedente, chiamiamo $\Determinant: \Endomorphisms{\LinearSpace} \rightarrow \Field$ \Define{determinante}[di un'applicazione lineare][determinante]. Diciamo inoltre che un endomorfismo $\LinearTransformation \in \Endomorphisms{\LinearSpace}$ \`e \Define{singolare}[singolare][endomorfismo] quando $\Determinant(\LinearTransformation) = 0$.
\end{Definition}
\begin{Theorem}
  Sia $\LinearSpace$ un $\Field$-spazio vettoriale finito di dimensione
  $n = \Dimension{\LinearSpace}$ e sia
  $\LinearTransformation \in \Endomorphisms{\LinearSpace}$.
  Supponiamo inoltre che
  $\LinearSubspace, \VarLinearSubspace \subseteq \LinearSpace$
  siano sottospazi vettoriali supplementari tali che
  \begin{itemize}
    \item $\LinearSubspace$ sia invariante per $\LinearTransformation$;
    \item $\VarLinearSubspace$ sia invariante puntualmente per
      $\LinearTransformation$.
  \end{itemize}
  Abbiamo
  $\Determinant{\Restricted{\LinearTransformation}{\LinearSubspace}}
  = \Determinant{\LinearTransformation}$.
\end{Theorem}
\Proof Sia $(\LinearBase_k)_{k \in n}$ una base di $\LinearSpace$ e poniamo
$m = \LinearDimension{\LinearSubspace}$.
\par Fissiamo una forma alternante
$\AlternatingForm_0 \in
\AlternatingForms{(\LinearSubspace)_{k \in m}}$.
Sia
$\AlternatingForm \in
\AlternatingForms{(\LinearSpace)_{k \in n}}$
la forma alternante definita come segue:
\begin{itemize}
  \item $\AlternatingForm((\Vector_k)_{k \in n}) = 0$ se i vettori
    $(\Vector_k)_{k \in n}$ sono vettori di $(\LinearBase_k)_{k \in n}$ e almeno due di
    essi sono uguali;
  \item $\AlternatingForm((\Vector_k)_{k \in n})
    = \AlternatingForm_0((\Vector_k)_{k \in m})$ se i vettori
    $(\Vector_k)_{k \in n}$ sono una permutazione dei vettori
    $(\LinearBase_k)_{k \in n}$.
\end{itemize}
\par Abbiamo
\begin{align*}
  \Determinant{\LinearTransformation}
  &= \frac{\AlternatingForm((\LinearTransformation(\LinearBase_k))_{k \in n})}
      {\AlternatingForm((\LinearBase_k)_{k \in n})},\\
  &= \frac{\AlternatingForm_0((\LinearTransformation(\LinearBase_k))_{k \in m})}
      {\AlternatingForm_0((\LinearBase_k)_{k \in m})},\\
  &= \Determinant{\Restricted{\LinearTransformation}{\LinearSubspace}}.
  \text{ \EndProof}
\end{align*}
\begin{Theorem}
	\TheoremName{Teorema di Binet}[di Binet][teorema]
  Sia $\LinearSpace$ un $\Field$-spazio vettoriale finito di dimensione
  $n = \Dimension{\LinearSpace}$ e siano
  $\LinearTransformation, \VarLinearTransformation
    \in \Endomorphisms{\LinearSpace}$.
  Abbiamo
  $\Determinant(\LinearTransformation\VarLinearTransformation)
  = \Determinant(\LinearTransformation)\Determinant(\VarLinearTransformation)$.
\end{Theorem}
\Proof Per ogni
$\AlternatingForm \in \AlternatingForms{(\LinearSpace)_{i \in n}}$
abbiamo
$\ForAll{(\Vector_i)_{i \in n} \in \LinearSpace^n}{
  \AlternatingForm(((\LinearTransformation\VarLinearTransformation)(
    \Vector_i))_{i \in n})
  = \AlternatingForm((\LinearTransformation(\VarLinearTransformation(
    \Vector_i)))_{i \in n})}$. \EndProof
\begin{Corollary}
  Sia $\LinearSpace$ un $\Field$-spazio vettoriale finito di dimensione
  $n = \Dimension{\LinearSpace}$ e sia
  $\LinearTransformation \in \Endomorphisms{\LinearSpace}$.
  Supponiamo inoltre che
  $\LinearSubspace \subseteq \LinearSpace$
  sia un sottospazio vettoriale invariante per $\LinearTransformation$.
  Abbiamo
  $\Determinant(\Restricted{\LinearTransformation}{\LinearSubspace})
  | \Determinant(\LinearTransformation)$.
\end{Corollary}
\Proof Sia 
$(\LinearBase_k)_{k \in \LinearDimension{\LinearSubspace}}$
una base di $\LinearSubspace$:
sia essa estesa ad una base
$(\LinearBase_k)_{k \in n}$.
\par Consideriamo gli endomorfismi di $\LinearSpace$
\begin{itemize}
  \item $\LinearTransformation_0$ tale che
    $\LinearTransformation_0(\LinearBase_k) =
      \begin{cases}
        \LinearTransformation(\LinearBase_k)\text{ se }
          k \in \LinearDimension{\LinearSubspace},\\
        \LinearBase_k\text{ se }
          k \in n \SetMin \LinearDimension{\LinearSubspace};
      \end{cases}$
  \item $\LinearTransformation_1$ tale che
    $\LinearTransformation_1(\LinearBase_k) =
      \begin{cases}
        \LinearBase_k\text{ se }
          k \in \LinearDimension{\LinearSubspace},\\
        \LinearTransformation(\LinearBase_k)\text{ se }
          k \in n \SetMin \LinearDimension{\LinearSubspace}.
      \end{cases}$
\end{itemize}
Abbiamo
$\LinearTransformation = \LinearTransformation_1 \circ \LinearTransformation_0$.
\par Per il teorema di Binet,
$\Determinant{\Restricted{\LinearTransformation}{\LinearSubspace}}
= \Determinant{\LinearTransformation_0} | \Determinant{\LinearTransformation}$.
\EndProof
\begin{Corollary}
  Sia $\LinearSpace$ un $\Field$-spazio vettoriale finito di dimensione
  $n = \Dimension{\LinearSpace}$ e sia
  $\LinearTransformation \in \Endomorphisms{\LinearSpace}$.
  Supponiamo inoltre che
  $\LinearSubspace, \VarLinearSubspace \subseteq \LinearSpace$
  siano sottospazi vettoriali supplementari e invarianti per
  $\LinearTransformation$.
  Abbiamo
  $\Determinant(\Restricted{\LinearTransformation}{\LinearSubspace})
  \Determinant(\Restricted{\LinearTransformation}{\VarLinearSubspace})
  = \Determinant(\LinearTransformation)$.
\end{Corollary}
\Proof Sia
$(\LinearBase_k)_{k \in n}$
una base di $\LinearSpace$ tale che
\begin{itemize}
  \item $(\LinearBase_k)_{k = 0}^{\LinearDimension{\LinearSubspace}}$
    sia una base di $\LinearSubspace$;
  \item $(\LinearBase_k)_{k = \LinearDimension{\LinearSubspace}}^n$
    sia una base di $\VarLinearSubspace$.
\end{itemize}
\par Consideriamo gli endomorfismi di $\LinearSpace$
\begin{itemize}
  \item $\LinearTransformation_0$ tale che
    $\LinearTransformation_0(\LinearBase_k) =
      \begin{cases}
        \LinearTransformation(\LinearBase_k)\text{ se }
          k \in \LinearDimension{\LinearSubspace},\\
        \LinearBase_k\text{ se }
          k \in n \SetMin \LinearDimension{\LinearSubspace};
      \end{cases}$
  \item $\LinearTransformation_1$ tale che
    $\LinearTransformation_1(\LinearBase_k) =
      \begin{cases}
        \LinearBase_k\text{ se }
          k \in \LinearDimension{\LinearSubspace},\\
        \LinearTransformation(\LinearBase_k)\text{ se }
          k \in n \SetMin \LinearDimension{\LinearSubspace}.
      \end{cases}$
\end{itemize}
Abbiamo
$\LinearTransformation = \LinearTransformation_1 \circ \LinearTransformation_0$.
\par Per il teorema di Binet,
$\Determinant{\Restricted{\LinearTransformation}{\LinearSubspace}}
\Determinant{\Restricted{\LinearTransformation}{\VarLinearSubspace}}
= \Determinant{\LinearTransformation_0}\Determinant{\LinearTransformation_1}
= \Determinant{\LinearTransformation}$.
\EndProof
\begin{Corollary}
	Sia $\LinearSpace$ un $\Field$-spazio vettoriale finito di dimensione
  $n = \Dimension{\LinearSpace}$ e sia
  $\LinearTransformation \in \Endomorphisms{\LinearSpace}$.
  $\LinearTransformation$ \`e invertibile se e solo se \`e non singolare.
\end{Corollary}
\Proof Se $\LinearTransformation$ invertibile, allora esiste
$\VarLinearTransformation \in \Endomorphisms{\LinearSpace}$ tale che
$\LinearTransformation\VarLinearTransformation = \Identity$.
Per il teorema di Binet,
$\Determinant{\LinearTransformation}{\VarLinearTransformation} = \Identity$, da
cui $\Determinant{\LinearTransformation} \neq 0$.
\par  Supponiamo ora
$\Determinant{\LinearTransformation} \neq 0$.
Sia $(\LinearBase_i)_{i \in n}$ una base di $\LinearSpace$ e
$\AlternatingForm \in \AlternatingForms{(\LinearSpace)_{i \in n}}$ non nulla.
Abbiamo
$(\Determinant{\LinearTransformation})
(\AlternatingForm((\LinearBase_i)_{i \in n})) \neq 0$, da cui
$\AlternatingForm((\LinearTransformation(\LinearBase_i))_{i \in n}) \neq 0$ e
quindi
$(\LinearTransformation(\LinearBase_i))_{i \in n}$
\`e una famiglia indipendente, da cui $\LinearTransformation$ \`e invertibile.
\EndProof
\begin{Theorem}
	\TheoremName{Formula di Leibniz}[di Leibniz][formula] Sia $\LinearSpace$ un $\Field$-spazio vettoriale finito di dimensione $n = \Dimension{\LinearSpace}$ e sia $\LinearTransformation \in \Endomorphisms{\LinearSpace}$. Sia inoltre $(\LinearBase_i)_{i \in n}$ una base di $\LinearSpace$. Se $\Matrix \in \Field^{n \times n}$ \`e la matrice che rappresenta $\LinearTransformation$ rispetto alla base fissata, allora $\Determinant{\LinearTransformation} = \sum_{\Permutation \in \SymmetricGroup{n}} \PermutationSign{\Permutation} \prod_{i \in n} \Matrix_{\Permutation(i),i}$.
\end{Theorem}
\Proof Sia $\AlternatingForm \in \AlternatingForms{(\LinearSpace)_{i \in n}}$ con $\AlternatingForm \neq 0$. Abbiamo $\Determinant{\LinearTransformation} = \frac{\AlternatingForm{(\LinearTransformation(\LinearBase_i))_{i \in n}}}{\AlternatingForm{(\LinearBase_i)_{i \in n}}} = \frac{\AlternatingForm((\sum_{i \in n}\Matrix_{i,j}\LinearBase_i)_{j \in n})}{\AlternatingForm((\LinearBase_i)_{i \in n})} = \frac{\sum_{\Permutation \in \SymmetricGroup{n}} \Matrix_{\Permutation(i),i} \AlternatingForm((\LinearBase_{\Permutation(i)})_{i \in n})}{\AlternatingForm((\LinearBase_i)_{i \in n})} = \sum_{\Permutation \in \SymmetricGroup{n}} \PermutationSign{\Permutation} \prod_{i \in n} \Matrix_{\Permutation(i),i}$. \EndProof
\begin{Definition}
	Fissato $n \in \mathbb{N}$, per ogni matrice $\Matrix = (\MatrixElement_{i,j})_{(i,j) \in n \times n}\in \Field^{n \times n}$ definiamo il \Define{determinante}[di una matrice][determinante] $\Determinant{\Matrix} = \sum_{\Permutation \in \SymmetricGroup{n}} \PermutationSign{\Permutation} \prod_{i \in n} \MatrixElement_{\Permutation(i),i}$. Denotiamo il determinante $\Determinant{\Matrix}$ anche $\lvert \Matrix \rvert$ o ancora
$$\left \lvert
\begin{array}{ccc}
	\MatrixElement_{0,0}	&	\cdots	&	\MatrixElement_{0,n} \\
	\vdots	&	\ddots	&	\vdots \\
	\MatrixElement_{m,0}	&	\cdots	&	\MatrixElement_{m,n}
\end{array}
\right \rvert$$ 
\end{Definition}
\begin{Theorem}
	Siano $\Field$ un campo e $n \in \NotZero{\mathbb{N}}$. Sia $\AlternatingForm: {\Field^n}^n \rightarrow \Field$ l'applicazione che a $\Matrix \in \Field^{n \times n}$ associa $\Determinant{\Matrix}$. Allora $\AlternatingForm$ \`e l'unica applicazione da $\Field^{n \times n}$ in $\Field$ che sia $n$-lineare alternante e tale che $\AlternatingForm(\Identity) = 1$.
\end{Theorem}
\Proof Il fatto che $\AlternatingForm$ sia $n$-lineare alternante e che $\AlternatingForm(\Identity) = 1$ segue per calcolo diretto sulla formula di Leibniz.
\par L'unicit\`a segue dal fatto che $\Dimension{\AlternatingForms{(\Field^n)_{i \in n}}} = 1$ e da $\AlternatingForm(\Identity) = 1$. \EndProof
\begin{Theorem}
	Sia $\Field$ un campo, $n \in \NotZero{\mathbb{N}}$ e $\Matrix \in \Field^{n \times n}$. Abbiamo $\Determinant{\Matrix} = \Determinant{\Transposed{\Matrix}}$.
\end{Theorem}
\Proof Segue direttamente dall'equazione $\sum_{\Permutation \in \SymmetricGroup{n}} \PermutationSign{\Permutation} \prod_{i \in n} \Matrix_{\Permutation(i),i} = \sum_{\Permutation \in \SymmetricGroup{n}} \PermutationSign{\Permutation} \prod_{i \in n} \Matrix_{i,\Permutation(i)}$. \EndProof
\begin{Definition}
	Siano $\Field$ un campo, $n \in \NotZero{\mathbb{N}}$ e $\Matrix \in \Field^{n \times n}$. Si chiama \Define{minore}[di una matrice][minore] di $\Matrix$ il determinante di una sottomatrice di $\Matrix$. Fissati $(I,J) \in n \times n$, chiamiamo inoltre \Define{complemento algebrico}[algebrico][complemento] o anche \Define{cofattore} lo scalare $(-1)^{i + j} \Determinant \left ( (\Matrix_{i,j})_{\substack{(i,j) \in n \times n\\i \neq I\\j \neq J}} \right )$.
\end{Definition}
\begin{Theorem}
	\TheoremName{Sviluppo di Laplace}[di Laplace][sviluppo] Siano $\Field$ un campo, $n \in \mathbb{N}$ e $\Matrix \in \Field^{n \times n}$. Abbiamo
	\begin{itemize}
		\item se $n = 1$, $\Determinant{\Matrix} = \Matrix_{0,0}$;
		\item se $n > 1$, fissato $I \in n$, $\Determinant{\Matrix} = \sum_{J \in n} (-1)^{I+J} \Matrix_{I,J} \Determinant \left ( (\Matrix_{i,j})_{\substack{(i,j) \in n \times n\\i \neq I\\j \neq J}} \right )$;
		\item se $n > 1$, fissato $J \in n$, $\Determinant{\Matrix} = \sum_{I \in n} (-1)^{I+J} \Matrix_{I,J} \Determinant \left ( (\Matrix_{i,j})_{\substack{(i,j) \in n \times n\\i \neq I\\j \neq J}} \right )$.
	\end{itemize}
\end{Theorem}
\Proof Se $n = 1$ segue direttamente dalle definizioni che $\Determinant{\Matrix} = \Matrix_{0,0}$.
\par Assumiamo $n > 1$. Abbiamo, fissato $I \in n$, $\sum_{J \in n} (-1)^{I + J} \Matrix_{I,J} \Determinant \left ( (\Matrix_{i, j})_{\substack{(i,j) \in n \times n\\i \neq I\\j \neq J}} \right ) = \sum_{J \in n} (-1)^{I + J} \Matrix_{I,J} \sum_{\Permutation \in \SymmetricGroup{n - 1}} \PermutationSign{\Permutation} \prod_{i \in n - 1} \Matrix_{(f \circ \Permutation)(i),g(i)} = \sum_{\Permutation \in \SymmetricGroup{n}} \PermutationSign{\Permutation} \prod_{i \in n} \Matrix_{\Permutation(i),i} = \Determinant{\Matrix}$, dove $f: n - 1 \rightarrow n \SetMin I$ \`e l'applicazione tale che $f(i) = \begin{cases} i\text{, se }i < I\\i + 1\text{, altrimenti,} \end{cases}$ e $g: n - 1 \rightarrow n \SetMin J$ \`e l'applicazione tale che $g(j) = \begin{cases} j\text{, se }j < J\\j + 1\text{, altrimenti.}\end{cases}$
\par L'ultima affermazione segue dalla penultima applicata a $\Transposed{\Matrix}$. \EndProof

\section{Applicazioni trasposte e applicazione aggiunte.}
\label{GeometriaLineare_ApplicazioniTrasposteEApplicazioniAggiunte}
\begin{Definition}
	Sia
  $\Linear: \LinearSpace \rightarrow \VarLinearSpace$
  un'applicazione lineare tra i $\Field$-spazi vettoriali
  $\LinearSpace$ e $\VarLinearSpace$.
  Chiamiamo
  \Define{applicazione trasposta}[lineare trasposta][applicazione] di
  $\Linear$ l'applicazione
  $\Transposed{\Linear}: \Dual{\VarLinearSpace} \rightarrow \Dual{\LinearSpace}$
  che a $\VarFunctional \in \Dual{\VarLinearSpace}$ associa
  $\VarFunctional \circ \Linear \in \Dual{\LinearSpace}$.
\end{Definition}
\begin{Theorem}
	Siano $\LinearSpace$ e $\VarLinearSpace$ $\Field$-spazi vettoriali finiti
  di dimensioni
  $\Dimension{\LinearSpace} = n$
  e
  $\Dimension{\VarLinearSpace} = m$.
  Fissisamo basi
  $(\LinearBase_j)_{j \in n}$
  e
  $(\VarLinearBase_i)_{i \in m}$
  di $\LinearSpace$ e $\VarLinearSpace$ rispettivamente.
  Se $A$ \`e la matrice che rappresenta $\Linear$ rispetto alle basi fissate e
  $B$ la matrice che rappresenta $\Transposed{\Linear}$ rispetto alle basi
  duali, allora $B = \Transposed{A}$.
\end{Theorem}
\Proof Fissiamo $i \in m$. Abbiamo
  $\Transposed{\Linear}(\VarLinearBase^i) = \VarLinearBase^i \circ \Linear$.
\begin{Theorem}
	Siano
	\begin{itemize}
		\item $\LinearSpace$ e $\VarLinearSpace$ $\Field$-spazi vettoriali su cui
          sono definiti prodotti denotati entrambi
          $\InnerProduct{\cdot}{\cdot}$;
		\item $\LinearOperator: \LinearSpace \rightarrow \VarLinearSpace$ lineare.
	\end{itemize}
	Esiste una e una sola applicazione lineare
  $\Adjoint{\LinearOperator}: \VarLinearSpace \rightarrow \LinearSpace$ tale che
  $\ForAll{(x,y) \in \LinearSpace \times \VarLinearSpace}
    {\InnerProduct{\LinearOperator x}{y}
      = \InnerProduct{x}{\Adjoint{\LinearOperator} y}}$.
\end{Theorem}
\Proof Se l'applicazione $\Adjoint{\Linear}$ esiste,
allora essa \`e necessarimente l'applicazione che a $y \in \VarLinearSpace$
associa il vettore di $\LinearSpace$ che, per il teorema di rappresentazione di
Riesz, rappresenta il funzionale su $\LinearSpace$ dato da
$x \mapsto \InnerProduct{\LinearOperator x}{y}$; e dunque, se esiste, essa \`e
unica. Occora allora provare solo la linearit\`a di $\Adjoint{\Linear}$.
\par Siano $y_0, y_1 \in \VarLinearSpace$.
Abbiamo, per ogni $x \in \LinearSpace$,
$\InnerProduct{\LinearOperator x}{y_0 + y_1}
= \InnerProduct{x}{\Adjoint{\LinearOperator}(y_0 + y_1)}$, ma anche
\begin{align*}
  \InnerProduct{\LinearOperator x}{y_0 + y_1}
  &= \InnerProduct{\LinearOperator x}{y_0}
    + \InnerProduct{\LinearOperator x}{y_1},\\
  &= \InnerProduct{x}{\Adjoint{\LinearOperator}y_0}
    + \InnerProduct{x}{\Adjoint{\LinearOperator}y_1},\\
  &= \InnerProduct{x}{\Adjoint{\LinearOperator}y_0
    + \Adjoint{\LinearOperator}y_1}.
\end{align*}
\par Quindi, per il teorema di rappresentazione di Riesz,
$\Adjoint{\LinearOperator}(y_0 + y_1)
= \Adjoint{\LinearOperator}y_0 + \Adjoint{\LinearOperator}y_1$.
\par Siano ora $y \in \VarLinearSpace$ e $\Scalar \in \Field$.
Abbiamo, per ogni $x \in \LinearSpace$,
$\InnerProduct{\LinearOperator x}{\Scalar y}
= \InnerProduct{x}{\Adjoint{\LinearOperator} \Scalar y}$,
ma anche
\begin{align*}
  \InnerProduct{\LinearOperator x}{\Scalar y}
  &= \Conjugate{\Scalar} \InnerProduct{\LinearOperator x}{y},\\
  &= \Conjugate{\Scalar}\InnerProduct{x}{\Adjoint{\LinearOperator}y},\\
  &= \InnerProduct{x}{\Scalar \Adjoint{\LinearOperator} y}.
\end{align*}
\par Quindi, per il teorema di rappresentazione di Riesz,
$\Adjoint{\LinearOperator} \Scalar y = \Scalar \Adjoint{\LinearOperator} y$.
\par $\Adjoint{\LinearOperator}$ \`e pertanto un'applicazione lineare. \EndProof
\begin{Theorem}\label{GeometriaLineare_Teorema_TraspostaAggiunta}
	Siano
	\begin{itemize}
		\item $\LinearSpace$ e $\VarLinearSpace$ $\Field$-spazi vettoriali su cui
          sono definiti prodotti denotati entrambi $\InnerProduct{\cdot}{\cdot}$
		\item $\LinearOperator: \LinearSpace \rightarrow \VarLinearSpace$ lineare;
		\item per ogni $\Vector \in \LinearSpace$,
          $F_{\LinearSpace}: \LinearSpace \rightarrow \Dual{\LinearSpace}$
          che a $\Vector \in \LinearSpace$ associa
          $\Functional \in \Dual{\LinearSpace}$ definito da
          $\Functional: x \mapsto \InnerProduct{x}{\Vector}$;
		\item per ogni $\VarVector \in \LinearSpace$,
      $G_{\VarLinearSpace}: \VarLinearSpace \rightarrow \Dual{\VarLinearSpace}$
          che a $\VarVector \in \VarLinearSpace$ associa
          $\VarFunctional \in \Dual{\VarLinearSpace}$ definito da
          $\VarFunctional: x \mapsto \InnerProduct{x}{\VarVector}$.
	\end{itemize}
	Abbiamo
  $F_{\LinearSpace} \circ \Adjoint{\LinearOperator}
    = \Transposed{\LinearOperator} \circ G_{\LinearSpace}$,
  \Cfr\ figura \label{GeometriaLineare_Figura_TraspostaAggiunta}.
\end{Theorem}
\begin{figure}
	\center
	\begin{tikzcd}[column sep=6em, row sep=6em]
		{\VarLinearSpace} \arrow{r}{\Adjoint{\LinearOperator}} \arrow{d}{G_{\LinearSpace}} & {\LinearSpace} \arrow{d}{F_{\LinearSpace}}\\
		{\Dual{\VarLinearSpace}} \arrow{r}{\Transposed{\LinearOperator}} & {\Dual{\LinearSpace}}
	\end{tikzcd}
	\caption{Enunciato del teorema \ref{GeometriaLineare_Teorema_TraspostaAggiunta}, diagramma 1.}
	\label{GeometriaLineare_Figura_TraspostaAggiunta}
\end{figure}
\Proof Fissiamo $\VarVector \in \VarLinearSpace$ e sia
$\Vector \in \LinearSpace$.
Abbiamo
\begin{itemize}
	\item $(F_{\LinearSpace} \circ \Adjoint{\LinearOperator})(\VarVector)(\Vector)
        = \InnerProduct{\Vector}{\Adjoint{\LinearOperator} \VarVector}
        = \InnerProduct{\LinearOperator \Vector}{\VarVector}$;
	\item $(\Transposed{\LinearOperator} \circ  G_{\VarLinearSpace})
          (\VarVector)(\Vector)
        = G_{\VarLinearSpace}(\VarVector)(\LinearOperator \Vector)
        = \InnerProduct{\LinearOperator \Vector}{\VarVector}$. \EndProof
\end{itemize}
\begin{Definition}
  Sia $\PrehilbertSpace$ uno spazio prehilbertiano con prodotto interno
  $\InnerProduct{\cdot}{\cdot}$.
  Sia inoltre
  $\HermitianOperator: \PrehilbertSpace \rightarrow \PrehilbertSpace$
  un operatore lineare autoaggiunto.
  $\HermitianOperator$ si dice
  \begin{itemize}
    \item \Define{hermitiano}[hermitiano][operatore] quando lo spazio
          prehilbertiano \`e complesso;
    \item \Define{simmetrico}[simmetrico][operatore] quando lo spazio
          prehilbertiano \`e reale.
    \item \Define{normale}[normale][operatore] quando
          $\HermitianOperator\Adjoint{\HermitianOperator}
            = \Adjoint{\HermitianOperator}\HermitianOperator$.
  \end{itemize}
\end{Definition}
\begin{Theorem}
  Sia $\PrehilbertSpace$ uno spazio prehilbertiano con prodotto interno
  $\InnerProduct{\cdot}{\cdot}$.
  Sia inoltre
  $\HermitianOperator: \PrehilbertSpace \rightarrow \PrehilbertSpace$
  un operatore lineare hermitiano.
  $\HermitianOperator$ \`e normale.
\end{Theorem}
\Proof Segue direttamente dalle definizioni. \EndProof


	\chapter{Geometria affine}
	\section{Definizioni e teoremi di base.}\label{DefinizioniETeoremiDiBase}
\begin{Definition}
	Siano $\AffineSpace$ una classe munita di un'azione destra $+$ di uno spazio vettoriale $\LinearSpace$ libera e transitiva: $(\AffineSpace,+)$ si chiama \Define{spazio affine}[affine][spazio] su $\LinearSpace$. Diremo anche che $\LinearSpace$ \`e \Define{associato}[vettoriale associato ad uno spazio affine][spazio] a $\AffineSpace$ e che $\AffineSpace$ \`e un $\LinearSpace$-spazio affine.
\end{Definition}
\begin{Theorem}
	Sia $\AffineSpace$ un $\LinearSpace$-spazio affine. Sia $P \in \AffineSpace$: l'applicazione $f: \LinearSpace \rightarrow \AffineSpace$ che a $\Vector \in \LinearSpace$ associa $P + \Vector$ \`e biettiva.
\end{Theorem}
\Proof L'iniettivit\`a di $f$ \`e conseguenza della libert\`a dell'azione di $\LinearSpace$, la suriettivit\`a della transitivit\`a. \EndProof

\section{Poliedri e politopi.}\label{PoliedriEPolitopi}
\begin{Definition}
	Un \Define{poliedro convesso} \`e l'intersezione di un numero finito di semispazi affini.
\end{Definition}
\begin{Definition}
	Un \Define{politopo} \`e un poliedro convesso limitato.
\end{Definition}
\par \`E importante osservare che non c'\`e accordo su queste definizioni: per altra impostazione, un poliedro convesso viene chiamato invece ``politopo convesso'', mentre un poliedro \`e un politopo convesso (per noi un poliedro convesso) in uno spazio affine di dimensione $3$.
\begin{Theorem}
	Un poliedro convesso \`e convesso.
\end{Theorem}
\Proof I semispazi affini sono convessi e l'intersezione di convessi \`e convessa. \EndProof
%	\begin{Theorem}
%		Ogni problema di programmazione lineare di dimensione $\ProblemDimension$ e con $\ConstraintsNumber$ vincoli lineari pu\`o essere messo nella forma $\max\ \lbrace \CostsVector \cdot x : A\AdmissibleSolution \leq b \rbrace$, dove $\CostsVector,\AdmissibleSolution \in \mathbb{R}^\ProblemDimension$, $A \in \mathbb{R}^{\ConstraintsNumber \times \ProblemDimension}$, $b \in \mathbb{R}^\ConstraintsNumber$ e su $\ConstraintsNumber$ consideriamo l'ordinamento parziale definito da $\Coimplies{(x_i)_{i \in \ConstraintsNumber} \leq (y_i)_{i \in \ConstraintsNumber}}{\ForAll{i \in \ConstraintsNumber}{x_i \leq y_i}}$.
%	\end{Theorem}
%	\Proof Abbiamo gi\`a provato l'equivalenza tra problemi di massimo e di minimo tramite la moltiplifcazione per $-1$ della funzione obiettivo. Basta dunque provare che tutti i vincoli lineari possono essere riscritti nella forma $a \cdot \AdmissibleSolution \leq b$ per opportuni $a \in \mathbb{R}^\ProblemDimension$ e $b \in \mathbb{R}$.
%	\par Supponiamo dato il vincolo lineare $a \cdot \AdmissibleSolution \geq b$ ($a \in \mathbb{R}^\ProblemDimension$, $b \in \mathbb{R}$): esso \`e equivalente al vincolo $-a \cdot \AdmissibleSolution \leq -b$. Supponiamo dato il vincolo $a \cdot \AdmissibleSolution = b$ ($a \in \mathbb{R}^\ProblemDimension$, $b \in \mathbb{R}$): esso \`e equivalente alla congiunzione dei vincoli $a \cdot \AdmissibleSolution \leq b$ e $a \cdot \AdmissibleSolution \geq b$. \EndProof
\begin{Definition}
	Un cono lineare che \`e anche un poliedro convesso si chiama \Define{cono poliedrico}[poliedrico][cono].
\end{Definition}
\begin{Theorem}
	Un cono poliedrico \`e un cono convesso.
\end{Theorem}
\Proof Segue direttamente dalle definizioni. \EndProof
\begin{Theorem}
	$\PolyhedricCone \subseteq \mathbb{R}^n$ \`e un cono poliedrico se solo se esiste $\Matrix \in \mathbb{R}^{m \times n}$ tale $\PolyhedricCone = \lbrace \Vector \in \mathbb{R}^n : \Matrix \Vector \leq 0\rbrace$.
\end{Theorem}
\Proof Segue direttamente dalle definizioni. \EndProof
\begin{Definition}
	Sia $\Polyhedron$ un poliedro convesso nel $\Field$-spazio vettoriale $\LinearSpace$. Definiamo
	\begin{itemize}
		\item \Define{direzione di linealit\`a}[di linealit\`a][direzione] un vettore $\Vector \in \LinearSpace$ tale che $\ForAll{(x,\Scalar) \in \Polyhedron \times \Field}{x + \Scalar\Vector \in \Polyhedron}$;
		\item \Define{direzione di recessione}[di recessione][direzione] un vettore $\Vector \in \LinearSpace$ tale che $\ForAll{(x,\Scalar) \in \Polyhedron \times \Field}{\Implies{\Scalar \geq 0}{x + \Scalar\Vector \in \Polyhedron}}$ (si suppone che su $\Field$ sia definito un ordinamento).
	\end{itemize}
	Denoteremo $\LinealityDirections{\Polyhedron}$ e $\RecessionDirections{\Polyhedron}$ le classi delle direzioni di linealit\`a e di recessione di $\Polyhedron$ rispettivamente.
\end{Definition}
\begin{Theorem}
	Sia $\Polyhedron$ un poliedro convesso nello spazio affine $\mathbb{R}^n$ ($n \in \mathbb{N}$), definito come luogo di tutti e soli gli $X \in \mathbb{R}^n$ tali che $AX \leq B$, con $A \in \mathbb{R}^{m \times n}$ e $B \in \mathbb{R}^m$ ($m \in \mathbb{R}^m$). Abbiamo
	\begin{itemize}
		\item $\RecessionDirections{\Polyhedron} = \lbrace \Vector \in \mathbb{R}^n: A\Vector \leq 0 \rbrace$;
		\item $\LinealityDirections{\Polyhedron} = \lbrace \Vector \in \mathbb{R}^n: A\Vector = 0 \rbrace$.
	\end{itemize}
\end{Theorem}
\Proof Sia $\Vector \in \mathbb{R}^n$ tale che $A\Vector \leq 0$. Siano poi $X \in \Polyhedron$ e $\Scalar \in \mathbb{R}$ con $\Scalar \geq 0$: abbiamo $A(X + \Scalar\Vector) = AX + \Scalar A\Vector \leq AX \leq B$. Quindi $\Vector$ \`e una direzione di recessione.
\par Supponiamo ora che $\Vector$ sia una direzione di recessione: per ogni $X \in \mathbb{R}^n$ e $\Scalar \in \mathbb{R}$ con $\Scalar \geq 0$ abbiamo $A(X + \Scalar\Vector) = AX + \Scalar A \Vector \leq B$. Quindi, per ogni $\Scalar > 0$, $A\Vector \leq \frac{B - AX}{\Scalar}$: ne deduciamo $A\Vector \leq \inf_{\Scalar > 0} \frac{B - AX}{\Scalar} = 0$.
\par La dimostrazione riguardo alle direzioni di linealit\`a e del tutto analoga. \EndProof
\begin{Theorem}
	La proiezione di un poliedro convesso su un sottospazio vettoriale \`e ancora un poliedro convesso.
\end{Theorem}
\Proof Dimostriamo il teorema nello spazio vettoriale $\mathbb{R}^n$ ($n \in \mathbb{N}$) e rispetto al sottospazio vettoriale di tutti i punti aventi le prime $k$ componenti nulle ($k \in \NotZero{n}$), isomorfo a $\mathbb{R}^{n - k}$: non si perde generalit\`a in virt\`u dell'esistenza di una base che consenta di ricondurci a questa situazione tramite l'isomorfismo di coordinate. Denotiamo dunque $\Projection: \mathbb{R}^n \rightarrow \mathbb{R}^{n - k}$ la proiezione di $\mathbb{R}^n$ su $\mathbb{R}^{n - k}$.
\par Supponiamo inizialmente $k = 1$.
\par Sia $\Polyhedron = \lbrace x: M x \leq B \rbrace$, con $M \in \mathbb{R}^{m \times n}$ e $B \in \mathbb{R}^m$ ($m \in \mathbb{N}$). Eseguiamo il seguente algoritmo:
\begin{enumerate}
	\item definiamo i seguenti sottoinsiemi di $m$:
	\begin{itemize}
		\item $m^+ = \lbrace i \in m: M_{i,0} \geq 0 \rbrace$;
		\item $m^0 = \lbrace i \in m: M_{i,0} = 0 \rbrace$;
		\item $m^- = \lbrace i \in m: M_{i,0} \leq 0 \rbrace$;
	\end{itemize}
	\item per ogni coppia $(i,h) \in m^+ \times m^-$,
	\begin{itemize}
		\item eliminiano la prima colonna di $M$ e ne rinumeriamo le colonne;
		\item rimuoviamo le righe $M_i$ e $M_j$ e i corrispondenti elementi $B_i$ e $B_j$ (non rinumeriamo le righe);
		\item aggiungiamo a $M$ la riga $(M_{h,1}M_{i,j} - M_{i,1}M_{h,j})_{j \in n - 1}$ e a $B$ il termine noto corrispondente $M_{h,1}B_i - M_{i,1}B_h$;
	\end{itemize}
	\item rinominiamo la nuova matrice del sistema ottenuta $M'$ e il nuovo vettore dei termini noti $B'$, per distinguerli dagli originali $M$ e $B$.
\end{enumerate}
\par Poniamo $\Polyhedron' = \lbrace x: M' x \leq B' \rbrace$ e dimostriamo che $X \in \Polyhedron$ se e solo se $\Projection(X) \in \Polyhedron'$.
\par Supponiamo $X \in \Polyhedron$. $\Projection(X)$, che il vettore $X$ tolta la prima componente, soddisfa chiaramente tutte le disequazioni di $\Polyhedron'$ che apparivano gi\`a identiche in $\Polyhedron$, cio\`e quelle corrispondenti agli indici di riga in $m^0$.
\par Consideriamo ora le nuove disequazioni di $\Polyhedron'$. Fissiamo $(i,h) \in m^+ \times m^-$ e proviamo che $\Projection(X)$ verifica la disequazione introdotta tramite questa coppia. Dall'$i$-esima disequazione di $\Polyhedron$ deduciamo $X_0 \leq \frac{B_i - \sum_{j \in \NotZero{n}} M_{i,j}X_j}{M_{i,1}}$ e dall'$h$-esima $\frac{B_h - \sum_{j \in \NotZero{n}} M_{h,j}X_j}{M_{h,1}} \leq X_0$, da cui $\frac{(B_h - \sum_{j \in \NotZero{n}} M_{h,j}X_j}{M_{h,1}} \leq \frac{B_i - \sum_{j \in \NotZero{n}} M_{i,j}X_j}{M_{i,1}}$, che si verifica facilmente essere equivalente alla disequazione introdotta tramite la coppia $(i,h)$.
\par Supponiamo ora $Y \in \Polyhedron'$ e proviamo che esiste $X \in \Polyhedron$ tale che $Y = \Projection(X)$. Per tutte le coppie $(i,h) \in m^+ \times m^-$, abbiamo $\frac{(B_h - \sum_{j \in n - 1} M_{h,j}Y_j}{M_{h,1}} \leq \frac{B_i - \sum_{j \in n - 1} M_{i,j}Y_j}{M_{i,1}}$: scegliamo $X_0$ tale che $\max_{h \in m^-} \frac{(B_h - \sum_{j \in n - 1} M_{h,j}Y_j}{M_{h,1}} \leq X_0 \leq \min_{i \in m^+} \frac{B_i - \sum_{j \in n - 1} M_{i,j}Y_j}{M_{i,1}}$ e poniamo, per ogni $k \in \NotZero{n}$, $X_k = Y_k$. $X = (X_k)_{k \in n}$ verifica chiaramente tutte le disequazioni di $\Polyhedron$ e $Y = \Projection(X)$. \EndProof
\begin{Definition}
	L'algoritmo descritto nella dimostrazione del teorema precedente si chiama \Define{eliminazione di Fourier-Motzkin}[di Fourier-Motzkin][eliminazione].
\end{Definition}
\begin{Corollary}
	Un cono finitamente generato \`e un cono poliedrico.
\end{Corollary}
\Proof Ragioniamo in $\mathbb{R}^n$ ($n \in \mathbb{R}$) tramite l'isomorfismo di coordinate. Sia $(\Vector_i)_{i \in m}$ un insieme finito di generatori del cono finitamente generato $\FinitelyGeneratedCone$. Abbiamo $X \in \FinitelyGeneratedCone$ se e solo se $X = \sum_{i \in m} \Scalar_i \Vector_i$, dove $(\Scalar_i)_{i \in m}$ sono tutti scalari positivi o nulli. Queste relazioni, viste come relazioni in $\mathbb{R}^{n + m}$ (consideriamo cio\`e variabili le componententi di $X$, ma anche gli scalari $(\Scalar_i)_{i \in m}$), sono tutte lineari: esse descrivono dunque un cono poliedrico di $\mathbb{R}^{n + m}$. Proiettenado tale cono sullo spazio originale (sono cio\`e variabili solo le componenti di $X$), per il teorema precedente, abbiamo ancora un poliedro. Dunque un $\FinitelyGeneratedCone$ \`e un cono poliedrico. \EndProof
\begin{Definition}
	Fissati $n, m \in \mathbb{N}$ e $\Matrix \in \mathbb{R}^{m \times n}$, sia il cono poliedrico $\PolyhedricCone = \lbrace \Vector \in \mathbb{R}^n : \Matrix \Vector \leq 0 \rbrace$. Definiamo \Define{cono duale}[duale][cono] il cono finitamente generato $\Dual{\PolyhedricCone} = \ConicalHull{\bigcup_{i \in m} \lbrace \Matrix_i \rbrace}$.
\end{Definition}
\begin{Lemma}
	Sia $(\Vector_i)_{i \in I}$ una famiglia di vettori di $\mathbb{R}^n$ ($n \in \mathbb{N}$) e sia $\Vector \in \mathbb{R}^n$. Poniamo $\FinitelyGeneratedCone = \ConicalHull{\bigcup_{i \in I} \lbrace \Vector_i \rbrace}$. Esiste $\Vector_+ \in \FinitelyGeneratedCone$ tale che $\Vector \cdot \Vector_+ > 0$ se e solo se esiste $i \in I$ tale che $\Vector \cdot \Vector_i > 0$.
\end{Lemma}
\Proof Supponiamo per assurdo $\ForAll{i \in I}{\Vector \cdot \Vector_i \leq 0}$ e sia $\Vector_\FinitelyGeneratedCone \in \FinitelyGeneratedCone$. Per definizione di inviluppo conico, esistono scalari positivi o nulli $(\Scalar_i)_{i \in I}$ tali che $\Vector_\FinitelyGeneratedCone = \sum_{i \in I} \Scalar_i\Vector_i$. Abbiamo allora $\Vector \cdot \Vector_\FinitelyGeneratedCone = \sum_{i \in I} \Scalar_i(\Vector \cdot \Vector_i) \leq 0$. \EndProof
\begin{Theorem}
	Sia $\PolyhedricCone$ un cono poliedrico. Abbiamo $\ForAll{(\Vector,\VarVector) \in \PolyhedricCone \times \Dual{\PolyhedricCone}}{\Vector \cdot \VarVector \leq 0}$.
\end{Theorem}
\Proof Segue direttamente dal lemma precedente. \EndProof
\begin{Lemma}
	Fissati $n, m \in \mathbb{N}$ e $\Matrix \in \mathbb{R}^{m \times n}$, sia il cono poliedrico $\PolyhedricCone = \lbrace \Vector \in \mathbb{R}^n : \Matrix \Vector \leq 0 \rbrace$. Esiste $\VarMatrix \in \mathbb{R}^{m \times n}$ tale che $\Dual{\PolyhedricCone} = \lbrace \Vector \in \mathbb{R}^n: \VarMatrix \Vector \leq 0 \rbrace$, tale che $\ForAll{i \in m}{\VarMatrix_i \in \PolyhedricCone}$.
\end{Lemma}
\Proof Ogni cono finitamente generato \`e poliedrico. Esiste dunque $\VarMatrix \in \mathbb{R}^{m \times n}$ tale che $\Dual{\PolyhedricCone} = \lbrace \Vector \in \mathbb{R}^n: \VarMatrix \Vector \leq 0 \rbrace$. Inoltre $\ForAll{i \in m}{\VarMatrix \Matrix_i \leq 0}$. Dunque $\ForAll{i \in m}{\VarMatrix_i \Matrix_i \leq 0}$, da cui $\ForAll{i \in m}{\Matrix \VarMatrix_i \leq 0}$. \EndProof
\begin{Theorem}
	Sia $\PolyhedricCone$ un cono poliedrico. $\PolyhedricCone$ coincide col suo biduale.
\end{Theorem}
\Proof Siano $\Matrix \in \mathbb{R}^{m \times n}$, $\Dual{\Matrix} \in \mathbb{R}^{\Dual{m} \times n}$, $\Dual{\Dual{\Matrix}} \in \mathbb{R}^{\Dual{\Dual{m}} \times n}$ ($m, \Dual{m}, \Dual{\Dual{m}}, n \in \mathbb{N}^2$) tali che
\begin{itemize}
	\item $\PolyhedricCone = \lbrace \Vector \in \mathbb{R}^n: \Matrix \Vector \leq 0 \rbrace$;
	\item $\Dual{\PolyhedricCone} = \lbrace \Vector \in \mathbb{R}^n: \Dual{\Matrix} \Vector \leq 0 \rbrace$ e ogni riga di $\Dual{\Matrix}$ appartiene a $\PolyhedricCone$;
	\item $\Dual{\Dual{\PolyhedricCone}} = \lbrace \Vector \in \mathbb{R}^n: \Dual{\Dual{\Matrix}} \Vector \leq 0 \rbrace$ e ogni riga di $\Dual{\Dual{\Matrix}}$ appartiene a $\PolyhedricCone$.
\par Ora, $\Dual{\Dual{\PolyhedricCone}}$ \`e inviluppo conico delle righe di $\Dual{\Matrix}$, che sono vettori appartenenti a $\PolyhedricCone$, dunque $\Dual{\Dual{\PolyhedricCone}} \subseteq \PolyhedricCone$.
\par D'altra parte, per ogni $i \in \Dual{\Dual{m}}$, abbiamo $\Dual{\Dual{\Matrix}}_i \in \Dual{\PolyhedricCone}$, da cui, per opportuni scalari non negativi $(\Scalar_{i,j})_{(i,j) \in \Dual{\Dual{m}} \times m}$, per ogni $i \in \Dual{\Dual{m}}$, $\Dual{\Dual{\Matrix}}_i = \sum_{j \in m} \Scalar_{i,j} \Matrix_i$. Fissato $\Vector \in \PolyhedricCone$, abbiamo, per ogni $i \in \Dual{\Dual{m}}$, $\Dual{\Dual{\Matrix}}_i \Vector = \sum_{j \in m} \Scalar_{i,j} \Matrix_i \Vector \leq 0$. Dunque $\Vector \in \Dual{\Dual{\PolyhedricCone}}$. \EndProof
\end{itemize}
\begin{Corollary}
	Un cono poliedrico \`e un cono finitamente generato.
\end{Corollary}
\Proof Il duale di un cono poliedrico \`e finitamente generato, ogni cono finitamente generato \`e poliedrico e ogni cono poliedrico coincide col suo biduale. \EndProof


	\chapter{Geometria proiettiva}
	\section{Definizioni e Teoremi di base.}

\begin{Definition}\label{def1}
	Siano $K$ un campo, $V$ un $\mathbb{K}$-spazio vettoriale di dimensione $n$ ($n \in \mathbb{N}$), $\sim$ la relazione d'equivalenza sugli elementi di $V - \lbrace 0 \rbrace$ data da $(x \sim y) \Leftrightarrow ((\exists \lambda \in \mathbb{K}^*)(x = \lambda y))$, dove $\mathbb{K}^* = \mathbb{K} - \lbrace 0 \rbrace$. Il quoziente $(V - \lbrace 0 \rbrace)/\sim$ si denota $\mathbb{P}(V)$ e si chiama \Define{spazio proiettivo}[proiettivo][spazio] associato a $V$. Si definisce \Define{dimensione}[di uno spazio proiettivo][dimensione] di $\mathbb{P}(V)$ il numero $\dim \mathbb{P}(V) = \dim V - 1$.
\end{Definition}
	\par Nel seguito, salvo eccezioni esplicitamente espresse, $\mathbb{K}$ indicher\`a sempre un campo e $V$ e $W$ indicheranno sempre $\mathbb{K}$-spazi vettoriali; $\sim$ sar\`a sempre la relazione sopradefinita; indicheremo sempre con $\pi$ la proiezione canonica di $V$ su $\mathbb{P}(V)$. Dato $v \in V$ indicheremo spesso $\pi(v)$ con $[v]$. $\mathbb{K}^*$ indicher\`a l'insieme $\mathbb{K} - \lbrace 0 \rbrace$. Quanto fissato per $V$ riguardo a $\sim$, $\pi$ e parentesi quadre varr\`a per qualunque spazio vettoriale comunque denotato; eventuali ambiguit\`a saranno risolte caso per caso.
\begin{Definition}\label{def2}
	Se $V = \mathbb{K}^{n + 1}$, con $n \in \mathbb{N}$, indicheremo con $\mathbb{P}^n(\mathbb{K})$ lo spazio proiettivo $\mathbb{P}(V)$ e lo chiameremo \Define{spazio proiettivo standard}[proiettivo standard][spazio] di dimensione $n$ su $\mathbb{K}$.
\end{Definition}
	\par Nelle ipotesi della definizione precedente, indicheremo $\pi((x_0, ..., x_n))$ con $[x_0,...,x_n]$ piuttosto che con $[(x_0, ..., x_n)]$.
\begin{Definition}\label{def3}
	Sia $W \subset V$ un sottospazio vettoriale di $V$. $\pi(W)$ \`e chiamato \Define{sottospazio proiettivo}[proiettivo][sottospazio] di $\mathbb{P}(V)$.
\end{Definition}
\begin{Theorem}\label{th1}
	Sia $W \subset V$ un sottospazio vettoriale di $V$. $W$ \`e saturo per la relazione d'equivalenza $\sim$\footnote{\label{nota_saturo} Qui e nel seguito, quando diremo che un sottospazio vettoriale $W$ \`e saturo per $\sim$, intenderemo con abuso di linguaggio che $W - \lbrace 0 \rbrace$ \`e saturo per $\sim$.} e abbiamo $\pi(W) = \mathbb{P}(W)$.
\end{Theorem}
\Proof Siano $x \in W - \lbrace 0 \rbrace$ e $y \in V \in \lbrace 0 \rbrace$ tali che $x \sim y$. Poich\'e $W$ \`e un sottospazio vettoriale ed $y = \lambda x$ per $\lambda \in \mathbb{K}^*$ opportuno, $y \in W$.
	\par Sia poi $z \in \pi(W)$: questa relazione equivale a $(\exists w)(w \in W \land z = [w])$, a sua volta equivalente a $z \in \mathbb{P}(W)$. \EndProof
	\par Quando scriveremo espressioni quali ``$\mathbb{P}(W) \subseteq \mathbb{P}(V)$ \`e un sottospazio proiettivo'', intenderemo che $W \subseteq V$ \`e un sottospazio vettoriale di $V$ e, di conseguenza, $\mathbb{P}(W)$ \`e un sottospazio proiettivo di $\mathbb{P}(V)$.
\begin{Corollary}\label{cor0.1}
	Siano $\mathbb{P}(W), \mathbb{P}(U) \subseteq \mathbb{P}(V)$ sottospazi proiettivi. Abbiamo $(\mathbb{P}(W) \subseteq \mathbb{P}(U)) \Leftrightarrow (W \subseteq U)$; in particolare, $(\mathbb{P}(W) = \mathbb{P}(U)) \Leftrightarrow (U = W)$. In maniera pi\`u precisa, se $S$ \`e l'insieme dei sottospazi vettoriali di $V$ e $\mathbb{S}$ l'insieme dei sottospazi proiettivi di $\mathbb{P}(V)$ e consideriamo su entrambi gli insiemi l'ordinamento di inclusione, allora l'applicazione $f: S \rightarrow \mathbb{S}$ che a $W \in S$ associa $f(W) = \mathbb{P}(W) \in \mathbb{S}$ \`e un isomorfismo d'insiemi ordinati.
\end{Corollary}
\Proof Sia $W \subseteq U$. $x \in \mathbb{P}(W)$ significa che esiste un elemento $w \in W - \lbrace 0 \rbrace$ tale che $x = [w]$; ma $W \subseteq U$, quindi $w \in U$ e, essendo i sottospazi vettoriali $U$ e $W$ saturi per $\sim$ per il teorema \ref{th1}, allora $x = [w] \in \mathbb{P}(U)$.
	\par Viceversa, sia $\mathbb{P}(W) \subseteq \mathbb{P}(U)$. Sia $w \in W - 0$. Allora $[w] \in \mathbb{P}(W)$, da cui $[w] \in \mathbb{P}(U)$. Per il teorema \ref{th1}, $U$ \`e saturo per $\sim$, dunque $w \in U$. \EndProof
\begin{Definition}\label{def4}
	Sia $\mathbb{P}(W) \subseteq \mathbb{P}(V)$ un sottospazio proiettivo. $\mathbb{P}(W)$ si chiama
	\begin{itemize}
		\item\Define{retta proiettiva}[proiettiva][retta] se $\dim \mathbb{P}(W) = 1$;
		\item\Define{piano proiettivo}[proiettivo][piano] se $\dim \mathbb{P}(W) = 2$;
		\item\Define{iperpiano proiettivo}[proiettivo][iperpiano] se $\dim \mathbb{P}(W) = \dim \mathbb{P}(V) - 1$.
	\end{itemize}
\end{Definition}
\begin{Definition}\label{def5}
	Sia $\phi: V \rightarrow W$ un'applicazione lineare iniettiva e sia $f: \mathbb{P}(V) \rightarrow \mathbb{P}(W)$ l'applicazione ottenuta da $\phi$ per passaggio ai quozienti: $f$ \`e detta \Define{applicazione proiettiva}[proiettiva][applicazione]\footnote{La terminologia qui adottata \`e coerente con l'uso comune di riservare la parola ``trasformazione'' ad applicazioni biettive che hanno ugual dominio e codominio (``endofunzioni''). Nell'eserciziario \cite{1}, tuttavia, quanto appena definito viene chiamato ``trasformazione proiettiva''}.
\end{Definition}
	\par  Data $\phi$ come sopra, indichiamo con $\bar{\phi}$ l'applicazione proiettiva associata. Nelle notazioni della definizione precedente, $f = \bar{\phi}$.
\begin{Theorem}\label{th1.1}
	Sia $\phi: V \rightarrow W$ un'applicazione lineare iniettiva. Abbiamo $\Image \bar{\phi} = \mathbb{P}(\Image \phi)$; in particolare, l'immagine di un'applicazione proiettiva \`e un sottospazio proiettivo del codominio.
\end{Theorem}
\Proof $x \in \Image \bar{\phi}$ equivale a $(\exists y)(y \in \mathbb{P}(V) \land x = \bar{\phi}(y))$, a sua volta equivalente a $(\exists y)(\exists z)(z \in V - \lbrace 0 \rbrace \land y = [z] \land x = \bar{\phi}(y))$ e quindi a $(\exists z)(z \in V - \lbrace 0 \rbrace \land x = [\phi(z)])$ e quest'ultima relazione \`e equivalente a $(\exists u)(u \in \Image \phi - \lbrace 0 \rbrace \land x = [u])$. \EndProof
	\par Osserviamo che non vale un teorema analogo per il nucleo, dato che non esiste un concetto di nucleo per un'applicazione proiettiva.
\begin{Theorem}\label{th2}
	Siano $\phi: V \rightarrow W$ e $\psi: V \rightarrow W$ due applicazioni lineari iniettive tra gli stessi spazi vettoriali $V$ e $W$. Allora vale $(\bar{\phi} = \bar{\psi}) \Leftrightarrow ((\exists \lambda \in \mathbb{K}^*)(\psi = \lambda\phi))$ .
\end{Theorem}
\Proof Supponiamo $\bar{\phi} = \bar{\psi}$. Allora, per ogni $v \in V - \lbrace 0 \rbrace$, $[\phi(v)] = [\psi(v)]$; esiste dunque $\lambda_v \in \mathbb{K}^*$ tale che $\psi(v) = \lambda_v \phi(v)$. Sia $\lambda_0 \in \mathbb{K}^*$ qualsiasi: abbiamo $\psi(0) = \lambda_0 \phi(0)$.
	\par Ora, $\bar{\phi} = \bar{\psi}$, dunque $\Image \bar{\phi} = \Image \bar{\psi}$, da cui, per il teorema \ref{th1.1}, $\mathbb{P}(\Image \phi) = \mathbb{P}(\Image \psi)$: ne deduciamo $\dim \Image \phi = \dim \Image \psi$. Consideriamo dunque le applicazioni $\phi'$ e $\psi'$ ottenute da $\phi$ e $\psi$ rispettivamente restringendo il codominio alle immagini. $\phi'$ e $\psi'$ sono invertibili e risulta, per ogni $v \in V$, $\phi'^{-1} \psi'(v) = \lambda_v \Identity_V(v)$.
	\par Dunque tutti i vettori sono autovettori di $A = \phi'^{-1} \psi'$.
	\par Siano $v, w \in V$. Allora
\begin{alignat*}{2}
	&Av = \lambda_v v,\\
	&Aw = \lambda_w w,\\
	&A(v + w) = \lambda_{v + w}(v + w) &&= \lambda_{v+w} v + \lambda_{v+w} w,\\
	&&&= \lambda_v v + \lambda_w w.
\end{alignat*}
	\par Se $v$ e $w$ sono dipendenti, chiaramente $\lambda_v = \lambda_w = \lambda_{v + w}$; se invece $v$ e $w$ sono indipendenti allora lo stesso risultato vale per unicit\`a delle coordinate.
	\par Ne consegue che $A = \lambda \Identity_V$ per $\lambda \in \mathbb{K}^*$ opportuno e dunque $\psi = \lambda \phi$.
	\par Supponiamo ora viceversa, per un certo $\lambda \in \mathbb{K}^*$, $\psi = \lambda\phi$. Allora, per ogni $v \in V - \lbrace 0 \rbrace$, abbiamo $\bar{\phi}([v]) = [\phi(v)] = [\lambda\phi(v)] = [\psi(v)] = \bar{\psi}([v])$. \EndProof
\begin{Definition}\label{def6}
	Sia $\phi: V \rightarrow W$ un'applicazione lineare. $\bar{\phi}: \mathbb{P}(V) \rightarrow \mathbb{P}(W)$ si dice
	\begin{itemize}
		\item \Define{isomorfismo proiettivo}[proiettivo][isomorfismo] o \Define{omografia} quando $\phi$ \`e un isomorfismo lineare;
		\item \Define{proiettivit\`a} quando $\bar{\phi}$ \`e un isomorfismo proiettivo e $V = W$.
	\end{itemize}
\end{Definition}
\begin{Theorem}\label{th3}
	Siano $\mathbb{P}(V)$ e $\mathbb{P}(W)$ spazi proiettivi. $\mathbb{P}(V)$ e $\mathbb{P}(W)$ sono isomorfi se e solo se $\dim \mathbb{P}(V) = \dim \mathbb{P}(W)$.
\end{Theorem}
\Proof La condizione $\dim \mathbb{P}(V) = \dim \mathbb{P}(W)$ equivale a $\dim V = \dim W$, a sua volta equivalente all'esistenza di un isomorfismo tra $V$ e $W$, corrispondente ad un isomorfismo proiettivo tra $\mathbb{P}(V)$ e $\mathbb{P}(W)$. \EndProof
\begin{Theorem}\label{th4}
	Un'applicazione proiettiva trasforma sottospazi proiettivi in sottospazi proiettivi isomorfi e un'omografia \`e data dalla restrizione dell'applicazione stessa al sottospazio proiettivo in questione, considerata come applicazione sull'immagine.
	\par Pi\`u precisamente, siano $f: \mathbb{P}(V) \rightarrow \mathbb{P}(W)$ un'applicazione proiettiva, $H$ un sottospazio proiettivo di $\mathbb{P}(V)$. Allora
	\begin{itemize}
		\item $f(H)$ \`e sottospazio proiettivo di $\mathbb{P}(W)$;
		\item $\dim H = \dim f(H)$;
		\item $f_{|H}: H \rightarrow f(H)$ \`e un isomorfismo proiettivo.
	\end{itemize}
\end{Theorem}
\Proof Per definizione di applicazione proiettiva esiste un'applicazione lineare iniettiva $\phi: V \rightarrow W$ tale che $f = \bar{\phi}$. Per definizione di sottospazio proiettivo esiste $K \subseteq V$ sottospazio vettoriale tale che $\mathbb{P}(K) = H$.
	\par Segue dalla teoria delle applicazioni lineare che $\phi_{|K}: K \rightarrow \phi(K)$ \`e un isomorfismo tra gli spazi vettoriali $K$ e $\phi(K)$. Se ne deduce un isomorfismo proiettivo $h$ tra $\mathbb{P}(K) = H$ e $\mathbb{P}(\phi(K))$.
	\par Si verifica facilmente che $h = f_{|H}$. Se ne deduce che $f(H) = \Image f_{|H} = \mathbb{P}(\phi(K))$. \EndProof
\begin{Definition}\label{def7}
	Siano $A, B \subseteq \mathbb{P}(V)$. $A$ e $B$ sono \Define{proiettivamente equivalenti}[proiettivamente equivalenti][parti] quando esiste una proiettivit\`a $f$ di $\mathbb{P}(V)$ tale che $f(A) = B$.
\end{Definition}
\begin{Theorem}\label{th5}
	Sia $(W_i)_{i \in I}$ una famiglia di sottospazi vettoriali di $V$. Allora $\bigcap_{i \in I} \mathbb{P}(W_i)$ \`e un sottospazio proiettivo di $\mathbb{P}(V)$ e $\bigcap_{i \in I} \mathbb{P}(W_i) = \mathbb{P}(\bigcap_{i \in I} W_i)$.
\end{Theorem}
\Proof $\bigcap_{i \in I} W_i$ \`e un sottospazio vettoriale di $V$, dunque $\mathbb{P}(\bigcap_{i \in I} W_i)$ \`e un sottospazio proiettivo di $\mathbb{P}(V)$.
	\par $x \in \bigcap_{i \in I} \mathbb{P}(W_i)$ significa che esiste una famiglia di vettori $(w_i)_{i \in I}$ tale che $(\forall i \in I)(w_i \in W_i - \lbrace 0 \rbrace \land x = [w_i])$. Per il teorema \ref{th1}, ogni sottospazio vettoriale $W_i$ \`e saturo\footnote{Cfr. nota \ref{nota_saturo}.} per la relazione d'equivalenza $\sim$ e pertanto \`e possibile scegliere i $w_i$ tutti uguali ad uno stesso elemento $w \in \bigcap_{i \in I} W_i - \lbrace 0 \rbrace$: se ne deduce che $x \in \bigcap_{i \in I} \mathbb{P}(W_i)$ equivale a $x \in \mathbb{P}(\bigcap_{i \in I} W_i)$. \EndProof
\begin{Definition}\label{def8}
	Siano $S_1$ e $S_2$ sottospazi proiettivi di uno stesso spazio proiettivo $\mathbb{P}(V)$. $S_1$ e $S_2$ si dicono
	\begin{itemize}
		\item \Define{incidenti}[incidenti][sottospazi] quando $S_1 \cap S_2 \neq \varnothing$;
		\item \Define{sghembi}[sghembi][sottospazi] altrimenti.
	\end{itemize}
\end{Definition}
\begin{Definition}\label{def9}
	Sia $A \subseteq \mathbb{P}(V)$. Definiamo \Define{sottospazio proiettivo generato da $A$}[proiettivo generato][sottospazio] il sottospazio proiettivo $L(A) = \bigcap_{S \in I} S$, dove $I$ \`e l'insieme di tutti i sottospazi proiettivi $T$ di $\mathbb{P}(V)$ tali che $A \subseteq T$.
	\par Dato $B \subseteq \mathbb{P}(V)$ denotiamo $L(A, B)$ il sottospazio $L(A \cup B)$ e, dati i punti, $P_1, ..., P_n \in \mathbb{P}(V)$ denotiamo $L(P_1, ..., P_n)$ il sottospazio $L(\bigcup_{i = 1, ..., n} \lbrace P_i \rbrace)$.
\end{Definition}
\begin{Theorem}\label{th6}
	Sia $(W_i)_{i \in I}$ una famiglia di sottospazi vettoriali di $V$. Abbiamo $L(\bigcup_{i \in I}(\mathbb{P}(W_i)) = \mathbb{P}(\sum_{i \in I} W_i)$.
\end{Theorem}
\Proof $\sum_{i \in I} W_i$ \`e il pi\`u piccolo sottospazio vettoriale che contiene $\bigcup_{i \in I} W_i$, dunque $\mathbb{P}(\sum_{i \in I} W_i)$ \`e il pi\`u piccolo sottospazio proiettivo che contiene $\bigcup_{i \in I} \mathbb{P}(W_i)$ (corollario \ref{cor0.1}). \EndProof
\begin{Theorem}\label{th7}
	\TheoremName{Formula di Grassman}. Siano $S_1$ e $S_2$ sottospazi proiettivi. Vale la relazione $\dim S_1 + \dim S_2 = \dim L(S_1, S_2) + \dim (S_1 \cap S_2)$.
\end{Theorem}
\Proof Siano $W_1, W_2 \subseteq V$ sottospazi vettoriali tali che $\mathbb{P}(W_1) = S_1$ e $\mathbb{P}(W_2) = S_2$. Dalla formula di Grassman per gli spazi vettoriali sappiamo che $\dim W_1 + \dim W_2 = \dim (W_1 + W_2) + \dim (W_1 \cap W_2)$. Passando alle dimensioni degli spazi proiettivi associati a $W_1$, $W_2$, $W_1 + W_2$ e $W_1 \cap W_2$ e applicando il teorema \ref{th6} si deduce la relazione della tesi. \EndProof
\begin{Definition}\label{def10}
	Il gruppo delle proiettivit\`a di $\mathbb{P}(V)$\footnote{La verifica che le proiettivit\`a costituiscono un gruppo \`e immediata.} viene detto \Define{gruppo lineare proiettivo}[lineare proiettivo][gruppo] e denotato $\mathbb{P}GL(V)$.
\end{Definition}
\begin{Theorem}\label{th8}
	Sia $\alpha: GL(V) \rightarrow \mathbb{P}GL(V)$ che a $\phi$ associa $\bar{\phi}$. $\alpha$ induce un isomorfismo (di gruppi) tra $GL(V)/\sim$ e $\mathbb{P}GL(V)$, dove $\sim$ \`e la relazione su $GL(V)$ definita da $(f \sim g) \Leftrightarrow ((\exists \lambda \in \mathbb{K}^*)(f = \lambda g))$.
\end{Theorem}
\Proof Siano $\phi, \psi \in GL(V)$ e $v \in V - \lbrace 0 \rbrace$. Abbiamo $(\alpha(\phi) \circ \alpha(\psi))(v) = \alpha(\phi)(\alpha(\psi)(v)) = \alpha(\phi)[\psi(v)]= [(\phi \circ \psi)(v)] = \alpha(\phi \circ \psi)(v)$. Dunque $\alpha$ \`e un omomorfismo.
	\par Inoltre $\alpha$ \`e suriettiva per costruzione e per il teorema \ref{th2} la relazione d'equivalenza determinata da $\alpha$ coincide con $\sim$. Dunque $GL(V)/\sim \cong \mathbb{P}GL(V)$. \EndProof
\begin{Corollary}\label{cor1}
	$\mathbb{P}GL(V)$ \`e isomorfo a $\mathbb{P}GL(n + 1, \mathbb{K}) = GL(n+ 1, \mathbb{K})/\sim$, dove $GL(n + 1, \mathbb{K})$ \`e il gruppo delle matrici quadrate di ordine $n + 1$ a coefficienti in $\mathbb{K}$.
\end{Corollary}
\Proof $GL(V) \cong GL(n + 1, \mathbb{K})$, quindi $GL(V)/\sim \cong GL(n + 1, \mathbb{K})/\sim$ e di conseguenza $\mathbb{P}GL(V) \cong GL(n + 1, \mathbb{K})/\sim$. \EndProof
\begin{Definition}\label{def11}
	Sia $\phi: V \rightarrow W$ un'applicazione lineare non nulla. Chiamiamo \Define{applicazione proiettiva degenere}[proiettiva degenere][applicazione] indotta da $\phi$ l'applicazione $f: \mathbb{P}(V) - \mathbb{P}(\Kernel \phi) \rightarrow \mathbb{P}(W)$ ottenuta per passaggio ai quozienti dall'applicazione $\phi_{|V - \Kernel \phi}$.
\end{Definition}
\begin{Theorem}\label{th9}
	Siano $\phi: V \rightarrow W$ e $\psi: V \rightarrow W$ due applicazioni lineari non nulle. Le applicazioni proiettive degeneri $f$ e $g$, indotte da $\phi$ e $\psi$ rispettivamente, sono uguali se e solo se esiste $\lambda \in \mathbb{K}^*$ tale che $\psi = \lambda \phi$.
\end{Theorem}
\Proof Supponiamo $f = g$. Consideriamo le applicazioni ottenute per passaggio al quoziente $\widetilde{\phi}: V/(\Kernel \phi) \rightarrow W$ e $\widetilde{\psi}: V/(\Kernel \psi) \rightarrow W$: queste sono lineari e iniettive.
	\par Da $f = g$ segue che $\mathbb{P}(V) - \mathbb{P}(\Kernel \phi) = \mathbb{P}(V) - \mathbb{P}(\Kernel \psi)$, da cui $\mathbb{P}(\Kernel \phi) = \mathbb{P}(\Kernel \psi)$ e $\Kernel \phi = \Kernel \psi$ (corollario \ref{cor0.1}), cosicch\'e $V/(\Kernel \phi) = V/(\Kernel \psi)$. Denotiamo $\pi'$ la proiezione canonica di $V$ su $V/(\Kernel \phi)$.
	\par Allora $f([v]) = [\phi(v)] = [(\widetilde{\phi} \circ \pi')(v)] = [\widetilde{\phi}(\pi'(v))]$ e $g([v]) = [\psi(v)] = [(\widetilde{\psi} \circ \pi') (v)] = [\widetilde{\psi}(\pi'(v))]$, per ogni $v \in V - \Kernel \phi$, da cui, poich\'e $f = g$, $[\widetilde{\phi}(\pi'(v))] = [\widetilde{\psi}(\pi'(v))]$, vale a dire le applicazioni proiettive indotte da $\widetilde{\phi}$ e $\widetilde{\psi}$ coincidono su tutto il dominio e dunque, per il teorema \ref{th2}, $\widetilde{\psi} = \lambda \widetilde{\phi}$ per $\lambda \in \mathbb{K}^*$ opportuno.
	\par Se ne deduce $\psi = \widetilde{\psi} \circ \pi' = (\lambda \widetilde{\phi}) \circ \pi' = \lambda \phi$.
	\par Viceversa, se $\psi = \lambda \phi$, allora per ogni $v \in V - \Kernel \phi = V - \Kernel \psi$, abbiamo $g([v]) = [\psi(v)] = [\lambda \phi(v)] = [\phi(v)] = f([v])$. \EndProof
\begin{Theorem}\label{th10}
	Siano $S = \mathbb{P}(U)$ e $H = \mathbb{P}(W)$ sottospazi proiettivi di $\mathbb{P}(V)$ tali che $S \cap H = \varnothing$ e $L(S, H) = \mathbb{P}(V)$. Sia $P \in \mathbb{P}(V) - H$. $L(H, P)$ interseca $S$ in uno e un solo punto.
\end{Theorem}
\Proof Sfruttiamo la formula di Grassman:
\begin{align*}
	\dim(L(H,P) \cap S) &= \dim L(H,P) + \dim S - \dim L(L(H,P),S),\\
	&= \dim H + 1 + \dim S - \dim \mathbb{P}(V),\\
	&= \dim L(S,H) + \dim(S \cap H) + 1 - \dim \mathbb{P}(V) = 0.
\end{align*}
	\par Dunque $L(H,P) \cap S$ \`e costituito da un solo punto. $\square$
\begin{Definition}\label{def12}
	Con le notazioni del teorema precedente, l'applicazione $\pi_H: \mathbb{P}(V) - H \rightarrow S$ che associa a $P$ l'unico punto nell'intersezione $L(H,P) \cap S$ si chiama \Define{proiezione} su $S$ di centro $H$.
\end{Definition}
\begin{Theorem}\label{th11}
	L'applicazione $\pi_H$ della definizione precedente \`e un'applicazione proiettiva degenere.
\end{Theorem}
\Proof Usiamo le stesse notazioni della definizione e del teorema precedenti. Per il corollario \ref{cor0.1}, da $S \cap H = \varnothing$ deduciamo $U \cap W = \lbrace 0 \rbrace$ e, usando anche il teorema \ref{th5}, da $L(S, H) = \mathbb{P}(V)$ deduciamo $U + W = V$. Dunque $U \oplus W = V$. Sia $p: V \rightarrow V$ la proiezione su $U$ lungo $W$: abbiamo $\Kernel p = W$.
	\par Sia $\bar{p}: \mathbb{P}(V) - H \rightarrow S$ l'applicazione proiettiva degenere associata a $p$. Proviamo che $\bar{p} = \pi_H$.
	\par Sia $v \in V - W$. Abbiamo $[v] \in \mathbb{P}(V) - H$ e $\bar{p}([v]) = [p(v)] = [u]$, dove $v = u + w$ con $u \in U$ e $w \in W$. Abbiamo $[u] \in \mathbb{P}(U) = S$. Inoltre, $L(H, [v]) = \mathbb{P}(W + <v>)$ e, essendo $u = - w + v$, $[u] \in L(H, [v])$. Dunque $[u]$ \`e l'unico punto a cui \`e ridotto l'insieme $L(H, [v]) \cap S$ - i.e. $\pi_H([v]) = [u]$. \EndProof
\begin{Definition}\label{def13}
	Siano $S_1$ e $S_2$ sottospazi proiettivi di stessa dimensione $k$, $H$ un terzo sottospazio tale che $H \cap S_1 = H \cap S_2 = \varnothing$ e $\dim H = n - k - 1$. La restrizione $f$ a $S_1$ della proiezione su $S_2$ di centro $H$ si chiama \Define{prospettivit\`a} di centro $H$.
\end{Definition}
	\par Nella definizione precedente, la proiezione su $S_2$ di centro $H$ \`e definita in virt\`u del fatto che abbiamo $S_2 \cap H = \varnothing$ e $\dim L(S_2, H) = \dim S_2 + \dim H - \dim(S_2 \cap H) = k + n - k - 1 - (-1) = n$, cosicch\'e $L(S_2, H) = \mathbb{P}(V)$.
\begin{Theorem}\label{th12}
	Con le notazioni della definizione precedente, $f$ \`e un isomorfismo proiettivo, $f^{-1}$ \`e un'altra prospettivit\`a di centro $H$ e le applicazioni $f_{|S_1 \cap S_2}$ e $\Identity_{S_1 \cap S_2}$ hanno lo stesso grafo.
\end{Theorem}
\Proof Siano $\pi_H: \mathbb{P}(V) - H \rightarrow S_2$ la proiezione di centro $H$ su $S_2$ e $p: V \rightarrow V$ l'endomorfismo che induce $\pi_H$ costruito come nella dimostrazione precedente. $S_1 \cap H = \varnothing$ dunque $\pi_H$ \`e effettivamente restringibile a $S_1$ ottenendo $f$.
	\par Siano $U, W, Z \subseteq V$ sottospazi vettoriali tali che $S_2 = \mathbb{P}(U)$, $H = \mathbb{P}(W)$ e $S_1 = \mathbb{P}(Z)$ e consideriamo la restrizione $g$ di $p$ a $Z$ come applicazione sulla sua immagine. Vogliamo provare che
\begin{itemize}
	\item $\Image g = p(Z) = U$;
	\item $g$ \`e un isomorfismo;
	\item $g$ induce $f$.
\end{itemize}
	\par Tutto questo segue subito osservando che $g$ \`e suriettiva per costruzione, iniettiva perch\'e $\Kernel g = Z \cap W = \lbrace 0 \rbrace$, e dunque $g$ \`e un isomorfismo. Cos\`i $\dim p(Z) = \dim Z = \dim U$ ed essendo $p(Z) \subseteq U$ deve essere $p(Z) = U$. Dunque $g$ induce un isomorfismo proiettivo tra $S_1$ e $S_2$ che non pu\`o che coincidere con $f$.
	\par Per dimostrare che $f^{-1}$ \`e un'altra prospettivit\`a di centro $H$ \`e sufficiente provare che $f^{-1}$ \`e la restrizione a $S_2$ della proiezione su $S_1$ di centro $H$: ci\`o segue osservando che $f^{-1}$ \`e indotta da $g^{-1}$.
	\par Il fatto che $f_{|S_1 \cap S_2} = \Identity_{S_1 \cap S_2}$ segue direttamente dalla definizione di $f$. \EndProof

\section{Riferimenti proiettivi e coordinate omogenee.}

\begin{Definition}\label{def14}
	Siano $P_1 = [v_1], ..., P_k = [v_k] \in \mathbb{P}(V)$, con $v_1, ..., v_k \in V$. $P_1, ..., P_k$ si dicono \Define{linearmente indipendenti}[lineare][indipendenza] se e solo se sono linearmente indipendenti $v_1, ..., v_k$. $P_1,... P_k$ si dicono \Define{in posizione generale}[in posizione generale][punti] quando sono linearmente indipendenti o se, per ogni scelta di $n + 1$ punti tra i $P_1, ..., P_k$, questi sono linearmente indipendenti.
\end{Definition}
	\par Si verifica subito che la definizione \`e ben posta.
\begin{Definition}\label{def15}
	Sia $\mathcal{R} = \lbrace P_0, ..., P_n, P_{n + 1} \rbrace$ un insieme ordinato di $n + 2$ punti in posizione generale: $\mathcal{R}$ si chiama \Define{riferimento proiettivo}[proiettivo][riferimento]. $P_0, ..., P_n$ si chiamano \Define{punti fondamentali}[fondamentali][punti] del riferimento, $P_{n + 1}$ \Define{punto unit\`a}[unit\`a][punto].
\end{Definition}
\begin{Theorem}\label{th13}
	Sia $\mathcal{R} = \lbrace P_0, ..., P_n, P_{n + 1} \rbrace$ un riferimento proiettivo di $\mathbb{P}(V)$. Per ogni $u \in V$ tale che $[u] = P_{n + 1}$ esiste una e una sola base $\mathcal{B}_u = \lbrace v_0, ..., v_n \rbrace$ di $V$ tale che
\begin{itemize}
	\item $P_i = [v_i]$ per ogni $i = 0, ..., n$;
	\item $u = \sum_{i = 0}^n v_i$.
\end{itemize}
\end{Theorem}
\Proof Sia $u \in V - \lbrace 0 \rbrace$ tale che $[u] = P_{n + 1}$. Per ogni $i = 0, ..., n$ scegliamo arbitrariamente un vettore $w_i \in V - \lbrace 0 \rbrace$ tale che $[w_i] = P_i$. I $P_0, ..., P_n, P_{n + 1}$ sono in posizione generale, dunque $w_1, ..., w_n, u$ sono linearmente indipendenti e costituiscono una base di $V$. Allora $w_0$ \`e combinazione lineare dei $w_1, ..., w_n, u$: esistono opportuni $\alpha_1, ..., \alpha_n, \beta \in \mathbb{K}$ tali che $w_0 = \sum_{i = 1}^n \alpha_i w_i + \beta u$. Ne risulta $u = \beta^{-1}w_0 - \sum_{i = 1}^n \beta^{-1} \alpha_i w_i$. Ponendo $v_0 = \beta^{-1}w_0$ e $v_i = - \beta^{-1} \alpha_i w_i$ per $i = 1, ..., n$ abbiamo costruito una base $\mathcal{B}_u = \lbrace v_0, ..., v_n \rbrace$ con le propriet\`a volute (l'indipendenza di $v_0, ..., v_n$ segue di nuovo dal fatto che i punti di $\mathcal{R}$ sono in posizione generale).
	\par Supponiamo ora invece $\mathcal{B}_u = \lbrace v_0, ..., v_n \rbrace$ e $\mathcal{B}'_u = \lbrace v'_0, ..., v'_n \rbrace$ siano due basi di $V$ tali che $P_i = [v_i] = [v'_i]$ per ogni $i = 0, ..., n$ e $u = \sum_{i = 0}^n v_i = \sum_{i = 0}^n v'_i$. Abbiamo, per ogni $i = 0, ..., n$, $v'_i = \lambda_i v_i$, per $\lambda_i \in \mathbb{K}^*$ opportuno. Dunque $u = \sum_{i = 0}^n v_i = \sum_{i = 0}^n v'_i = \sum_{i = 0}^n \lambda_i v'_i$. Le coordinate di $u$ rispetto a $\mathbb{B}_u$ sono uniche, dunque necessariamente $\lambda_i = 1$ per ogni $i = 0, ..., n$ e le due basi sono identiche. \EndProof
\begin{Definition}\label{def16}
	Le basi $\mathcal{B}_u$ costruite come descritto nel teorema precedente si chiamano \Define{basi normalizzate}[normalizzate][basi] rispetto al riferimento proiettivo $\mathcal{R}$.
\end{Definition}
\begin{Theorem}\label{th14}
	Nelle ipotesi del teorema precedente, abbiamo $\mathcal{B}_{\lambda u} = \lbrace \lambda v_0, ..., \lambda v_{n + 1} \rbrace$.
\end{Theorem}
\Proof \`E immediato che $P_i = [\lambda v_i]$ per ogni $i = 0, ..., n$. Inoltre $\lambda u = \lambda \sum_{i = 0}^n v_i = \sum_{i = 0}^n \lambda v_i$. \EndProof
\begin{Theorem}\label{th15}
	Siano $\mathcal{R} = \lbrace P_0, ..., P_n, P_{n + 1} \rbrace$ un riferimento proiettivo, $\phi_{\mathcal{B}_u}: V \rightarrow \mathbb{K}^{n + 1}$ e $\phi_{\mathcal{B}_v}: V \rightarrow \mathbb{K}^{n + 1}$ le applicazione coordinate rispetto alle basi normalizzate $\mathcal{B}_u$ e $\mathcal{B}_v$. Le applicazioni proiettive associate sono lo stesso isomorfismo proiettivo $f: \mathbb{P}(V) \rightarrow \mathbb{P}^n(\mathbb{K})$.
\end{Theorem}
\Proof $\phi_{\mathcal{B}_u}$ e $\phi_{\mathcal{B}_v}$ sono isomorfismi, dunque le applicazioni proiettive associate sono isomorfismi proiettivi.
	\par Siano $v_0, ..., v_n$ gli elementi della base normalizzata $\mathcal{B}_u$. $\mathcal{B}_u$ e $\mathcal{B}_v$ sono basi normalizzate rispetto allo stesso riferimento proiettivo, dunque $v = \lambda u$ per $\lambda \in \mathbb{K}^*$ opportuno e, per il teorema \ref{th14}, $\mathcal{B}_v = \lbrace \lambda v_0, ..., \lambda v_n \rbrace$. Pertanto $\phi_{\mathcal{B}_u} = \lambda \phi_{\mathcal{B}_v}$, da cui le due applicazioni inducono la stessa applicazione proiettiva per il teorema \ref{th2}. \EndProof
\begin{Definition}\label{def17}
	Con le notazioni del teorema precedente, dato $P \in \mathbb{P}(V)$, si chiamano \Define{coordinate omogenee}[omogenee][coordinate] rispetto al riferimento proiettivo $\mathcal{R}$ le coordinate di un qualunque rappresentante di $f(P)$.
\end{Definition}
\par Le coordinate omogenee di un punto rispetto ad un riferimento proiettivo dato sono dunque determinate a meno di una costante moltiplicativa non nulla.
\begin{Definition}\label{def18}
	In $\mathbb{P}^n(\mathbb{K})$ il \Define{riferimento proiettivo standard}[proiettivo standard][riferimento] \`e l'insieme ordinato $\lbrace [1, 0, ..., 0], [0, 1, ..., 0], ..., [0, 0, ..., 1], [1, 1, ..., 1] \rbrace$.
\end{Definition}
	\par Una base normalizzata associata al rifermiento proiettivo standard \`e la base canonica di $\mathbb{K}^{n + 1}$.
	\par Se in un fissato riferimento proiettivo $\mathcal{R}$ il punto $P \in \mathbb{P}(V)$ ha coordinate omogenee $[x_0, ..., x_n]$, scriveremo $[P]_{\mathcal{R}} = [x_0, ..., x_n]$, o, pi\`u brevemente, $P = [x_0, ..., x_n]$.
\begin{Theorem}\label{th16}
\TheoremName{Teorema fondamentale delle applicazioni proiettive}. Siano $\mathbb{P}(V)$ e $\mathbb{P}(W)$ due spazi proiettivi su campo $\mathbb{K}$ tali che $\dim \mathbb{P}(V) = n \leq \dim \mathbb{P}(W)$. Sia $\mathcal{R} = \lbrace P_0, ..., P_{n + 1} \rbrace$ un riferimento proiettivo di $\mathbb{P}(V)$ e sia $\mathcal{R}' = \lbrace Q_0, ..., Q_{n + 1} \rbrace$ un riferimento proiettivo di un sottospazio $S$ di dimensione $n$ di $\mathbb{P}(W)$. Allora esiste una e una sol'applicazione proiettiva $f: \mathbb{P}(V) \rightarrow \mathbb{P}(W)$ tale che $f(P_i) = Q_i$ per ogni $i = 0, ..., n + 1$. Se $S = \mathbb{P}(W)$, allora $f$ \`e un isomorfismo proiettivo.
\end{Theorem}
\Proof Siano $\mathcal{B}_u = \lbrace u_0, ..., u_n \rbrace$ e $\mathcal{B}_v = \lbrace v_0, ..., v_n \rbrace$ basi normalizzate associate ai riferimenti proiettivi $\mathcal{R}$ e $\mathcal{R}'$ rispettivamente, con $[u] = P_{n + 1}$ e $[v] = Q_{n + 1}$. \`E noto dall'algebra lineare che esiste una ed una sol'applicazione lineare $\phi: V \rightarrow W$ che trasforma la base $\mathcal{B}_u$ nella base $\mathcal{B}_v$; questa \`e iniettiva e si ha che l'applicazione proiettiva indotta $\bar{\phi}$ \`e tale che $\bar{\phi}(P_i) = [\phi(u_i)] = [v_i] = Q_i$ per ogni $i = 0, ..., n$. Inoltre $\bar{\phi}(P_{n + 1}) = [\phi(u)] = [\phi(\sum_{i = 0}^n u_i)] = [\sum_{i = 0}^n v_i] = [v] = Q_{n+ 1}$. Questo dimostra l'esistenza.
	\par Sia $g: \mathbb{P}(V) \rightarrow \mathbb{P}(W)$ un'altra applicazione proiettiva tale che $g(P_i) = Q_i$ per ogni $i = 0, ..., n + 1$. Supponiamo $g$ sia indotta da un'applicazione lineare iniettiva $\psi: V \rightarrow W$.
	\par Le applicazioni $\phi$ e $\psi$ sono due applicazioni lineari che mandano la stessa base $\mathcal{B}_u$ in due basi normalizzate rispetto allo stesso riferimento proiettivo $\mathcal{R}'$ tali che i loro punti unit\`a, $\phi(u)$ e $\psi(u)$ rispettivamente, verificano la relazione $\phi(u) = \lambda \psi(u)$ per $\lambda \in \mathbb{K}^*$ opportuno. Allora, per il teorema \ref{th14}, abbiamo anche $\phi = \lambda \psi$, da cui, per il teorema \ref{th2}, $f = g$. \EndProof
\begin{Theorem}\label{th17}
	Consideriamo gli spazi proiettivi standard $\mathbb{P}^n(\mathbb{K})$ e $\mathbb{P}^m(\mathbb{K})$, con $n, m \in \mathbb{N}$ e $m \geq n$. Sia $A$ una matrice di dimensione $(m + 1) \times (n + 1)$ e rango massimo $n + 1$. L'applicazione $f_A: \mathbb{P}^n(\mathbb{K}) \rightarrow \mathbb{P}^m(\mathbb{K})$ che a $[x_0, ..., x_n]$ associa $[A (x_0 \cdots x_n)^t]$ \`e un'applicazione proiettiva (le coordinate sono espresse rispetto ai riferimenti proiettivi standard).
\end{Theorem}
\Proof Consideriamo l'applicazione lineare $\phi: \mathbb{K}^{n + 1} \rightarrow \mathbb{K}^{m + 1}$ definita dalla matrice $A$ rispetto alle basi canoniche: essa \`e iniettiva e induce un'applicazione proiettiva $\bar{\phi}: \mathbb{P}^n(\mathbb{K}) \rightarrow \mathbb{P}^m(\mathbb{K})$. \`E immediato che $\bar{\phi}$ coincide con l'applicazione $f$. \EndProof
\begin{Theorem}\label{th18}
	Con le notazioni del teorema precedente, se $B$ \`e un'altra matrice $(m + 1) \times (n + 1)$ di rango $n + 1$, abbiamo $(f_A = f_B) \Leftrightarrow (\exists \lambda \in \mathbb{K}^*)(B = \lambda A)$.
\end{Theorem}
\Proof Per quanto visto nella dimostrazione del teorema precedente, $f_A$ \`e indotta dall'applicazione lineare $\phi: \mathbb{K}^{n + 1} \rightarrow \mathbb{K}^{m + 1}$ definita dalla matrice $A$ e $f_B$ dall'applicazione lineare $\psi: \mathbb{K}^{n + 1} \rightarrow \mathbb{K}^{m + 1}$ definita dalla matrice $B$. $f_A = f_B$, per il teorema \ref{th2}, equivale all'esistenza di un $\lambda \in \mathbb{K}^*$ tale che $\psi = \lambda \phi$, relazione a sua volta equivalente a $B = \lambda A$. $\square$
\begin{Theorem}\label{th19}
	Sia $f: \mathbb{P}^n(\mathbb{K}) \rightarrow \mathbb{P}^m(\mathbb{K})$, con $n, m \in \mathbb{N}$ e $n \leq m$, un'applicazione proiettiva. Allora, mantenendo le notazioni del teorema \ref{th17}, esiste una matrice $A$ di dimensioni $(m + 1) \times (n + 1)$ e rango massimo $n + 1$ tale che $f_A = f$.
\end{Theorem}
\Proof Denotiamo $\mathcal{R} = \lbrace [e_1], ..., [e_{n + 1}], [\sum_{i = 1}^{n + 1} e_i] \rbrace$ il riferimento proiettivo standard. Costruiamo il riferimento proiettivo $f(\mathcal{R})$ di un opportuno sottospazio proiettivo di $\mathbb{P}^m(\mathbb{K})$ e quindi la matrice $A$ di dimensione $(m + 1) \times (n + 1)$ le cui colonne sono i vettori di una base normalizzata associata a $f(\mathcal{R})$. Abbiamo, per costruzione, $f_A([e_i]) = f([e_i])$ per ogni $i = 1, ..., n + 1$; inoltre, sempre per costruzione, $f_A([\sum_{i = 1}^{n + 1} e_i]) = f([\sum_{i = 1}^{n + 1} e_ i])$. Per il teorema \ref{th16}, $f$ ed $f_A$ devono coincidere. $\square$
\begin{Definition}\label{def19}
	Nelle ipotesi del teorema precedente, la matrice $A$ si dice \Define{associata all'applicazione proiettiva $f$}[associata ad un'applicazione proiettiva][matrice] (rispetto ai riferimenti proiettivi standard).
\end{Definition}
\begin{Definition}\label{def20}
	Sia $f: \mathbb{P}(V) \rightarrow \mathbb{P}(W)$ un'applicazione proiettiva. Siano $\mathcal{R}_1$ un riferimento proiettivo di $\mathbb{P}(V)$ e $\mathcal{R}_2$ un riferimento proiettivo di $\mathbb{P}(W)$, $\phi_1$ e $\phi_2$ i corrispondenti isomorfismi proiettivi con $\mathbb{P}^n(\mathbb{K})$ e $\mathbb{P}^m(\mathbb{K})$ rispettivamente. Se $A$ \`e una matrice associata all'applicazione proiettiva $\phi_2 \circ f \circ \phi_1^{-1}$, allora $A$ si dice \Define{associata ad $f$}[associata ad un'applicazione proiettiva][matrice] rispetto ai riferimenti proiettivi $\mathcal{R}_1$ e $\mathcal{R}_2$.
\end{Definition}
	\par La matrice associata ad un'applicazione proiettiva rispetto a due riferimenti proiettivi dati \`e unica a meno di una costante moltiplicativa non nulla per il teorema \ref{th18}.
\begin{Definition}\label{def21}
	Siano $\mathcal{R}_1$ e $\mathcal{R}_2$ due riferimenti proiettivi dello stesso spazio proiettivo $\mathbb{P}(V)$. La matrice associata ad $\Identity_{\mathbb{P}(V)}$ rispetto ai riferimenti $\mathcal{R}_1$ e $\mathcal{R}_2$ si chiama \Define{matrice del cambiamento di riferimento proiettivo}[del cambiamento di riferimento proiettivo][matrice] o \Define{matrice del cambiamento di coordinate omogenee}[del cambio di coordinate omogenee][matrice].
\end{Definition}
\begin{Theorem}\label{th20}
	Sia $S = \mathbb{P}(W)$ un sottospazio proiettivo di $\mathbb{P}(V)$ e siano $\dim S = k$ e $\dim \mathbb{P}(V) = n$. Esistono uno spazio vettoriale $U$ e un'applicazione lineare $\phi: V \rightarrow U$ tale che $S = \mathbb{P}(\Kernel \phi)$. Inoltre, fissato un riferimento proiettivo $\mathcal{R}$, esiste una matrice $A$ di rango $n - k$ tale che $(P = [x_0, ..., x_n] \in S) \Leftrightarrow (A (x_0 \cdots x_n)^t = 0)$.
\end{Theorem}
\Proof Per la prima parte del teorema basta prendere $U = V/W$ e $\phi$ uguale alla proiezione canonica di $V$ su $V/W$: abbiamo allora in effetti $W = \Kernel \phi$, equivalente a $S = \mathbb{P}(\Kernel \phi)$ per il corollario \ref{cor0.1}.
	\par Fissiamo una base normalizzata rispetto ad $\mathcal{R}$ di $V$ e una base di $U$; sia $A$ la matrice associata a $\phi$ rispetto a queste basi.
	\par Sia $x = (x_0, ..., x_n) \in \mathbb{K}^{n+ 1}$. Poich\'e $A$ \`e la matrice associata a $\phi$ rispetto alle basi fissate, abbiamo $A (x_0 \cdots x_n)^t = 0$ se e solo se $x \in \Kernel \phi = W$ e inoltre il rango di $A$ \`e $n - k$. Inoltre $x \in \Kernel \phi = W$ equivale a $[x] \in \mathbb{P}(W) = S$ e le coordinate omogenee di $[x]$ rispetto ad $\mathcal{R}$ sono proprio $[x_0, ..., x_n]$. $\square$
\begin{Definition}\label{def22}
	Con le notazioni del teorema precedente, le equazioni del sistema $AX = 0$, dove $X = (x_0 \cdots x_n)^t$, sono dette \Define{equazioni cartesiane}[cartesiane di un sottospazio proiettivo][equazioni] del sottospazio $S$.
\end{Definition}
\begin{Theorem}\label{th22}
	Siano dati un riferimento proiettivo $\mathcal{R}$ di $\mathbb{P}(V)$ e una base $\mathcal{B}$ di $V$ normalizzata rispetto ad $\mathcal{R}$. Il sottospazio vettoriale $W$ di $V$ \`e descritto dalle equazioni cartesiane $AX = 0$, dove $A$ ha rango $\dim V - \dim W$ e $X = (x_0 \cdots x_n)^t$, se e solo se il sottospazio proiettivo $\mathbb{P}(W)$ di $\mathbb{P}(V)$ \`e descritto dalle stesse equazioni cartesiane $AX = 0$.
\end{Theorem}
\Proof Sia $W$ descritto dalle equazioni cartesiane $AX = 0$. Sia $[w] \in \mathbb{P}(W)$: abbiamo $w \in W$ (teorema \ref{th1}). Supponiamo che $X_0$ sia il vettore delle coordinate di $w$; allora $AX_0 = 0$. Ma $X_0$ coincide con uno dei possibili vettori delle coordinate omogenee di $[w]$, dunque $\mathbb{P}(W)$ \`e contenuto nel sottospazio proiettivo descritto dalle equazioni cartesiane $AX = 0$.
	\par Supponiamo ora invece $[x] \in \mathbb{P}(V)$ e sia $X_1$ un vettore delle coordinate omogenee di $[x]$ tale che $AX_1 = 0$: essendo le coordinate omogenee determinate a meno di moltiplicazione per uno scalare non nullo, possiamo supporre che $X_1$ sia anche il vettore delle coordinate di $x \in V$, per cui $x \in W$ e dunque $[x] \in \mathbb{P}(W)$, e questo conclude la prima parte della dimostrazione.
	\par Supponiamo adesso che $\mathbb{P}(W)$ sia descritto dalle equazioni cartesiane $AX = 0$. Sia $w \in W - \lbrace 0 \rbrace$. Allora $[w] \in \mathbb{P}(W)$ e, se $X_0$ \`e il vettore delle coordinate di $w$, esso \`e anche un vettore delle coordinate omogenee per $[w]$, da cui $AX_0 = 0$. Dunque $W - \lbrace 0 \rbrace$ \`e contenuto nel sottospazio vettoriale descritto dalle equazioni cartesiane $AX = 0$.
	\par Supponiamo ora invece $x \in V - \lbrace 0 \rbrace$ di coordinate $X_1$ tale che $AX_1 = 0$. Allora $X_1$ \`e anche un vettore di coordinate omogenee per $[x]$ e pertanto $[x] \in \mathbb{P}(W)$, da cui, per il teorema \ref{th1}, $x \in W$.
	\par Inoltre, se $x = 0$, \`e immediato che $x$ appartiene sia allo spazio vettoriale descritto da $AX = 0$ che a $W$, e questo conclude la dimostrazione. \EndProof
\begin{Theorem}\label{th23}
	Sia $S = \mathbb{P}(W)$ un sottospazio proiettivo di $\mathbb{P}(V)$, dove $k = \dim S$. Esiste un'applicazione proiettiva $f: \mathbb{P}^k(\mathbb{K}) \rightarrow \mathbb{P}(V)$ tale che $S = \Image f$.
\end{Theorem}
\Proof Fissiamo un riferimento proiettivo $\mathcal{R}$ di $\mathbb{P}(V)$ tale che i primi $k + 1$ punti di $\mathcal{R}$ appartengano ad $S$; sia $\mathcal{B}$ una base di $V$ normalizzata rispetto ad $\mathcal{R}$. Sia $\phi: \mathbb{K}^{k + 1} \rightarrow V$ l'applicazione lineare che a $(x_0 \cdots x_k)^t$ associa l'elemento di coordinate $(x_0, ..., x_k, 0, ..., 0)$ rispetto a $\mathcal{B}$. Abbiamo $\Image \phi = W$. Sia $f: \mathbb{P}^k(\mathbb{K}) \rightarrow \mathbb{P}(V)$ l'applicazione proiettiva indotta da $\phi$: per il corollario \ref{cor0.1}, $\Image f = \mathbb{P}(\Image \phi) = \mathbb{P}(W) = S$. \EndProof
\begin{Definition}\label{def23}
	Con le notazioni del teorema e della dimostrazione precedenti, le equazioni del sistema $AX = Y$ - dove $A$ \`e la matrice associata ad $f$ rispetto ai due riferimenti proiettivi $\mathcal{R}_1$ di $\mathbb{P}^k(\mathbb{K})$ e $\mathcal{R}_2$ di $\mathbb{P}(V)$, $X = (x_0 \cdots x_k)^t$ un vettore le cui componenti prendono il nome di \NIDefine{parametri} ed $Y = (y_0 \cdots y_n)^t$ - vengono dette \Define{equazioni parametriche}[parametriche di un sottospazio proiettivo][equazioni] del sottospazio proiettivo $S$.
\end{Definition}
\begin{Theorem}\label{th24}
	Siano dati un riferimento proiettivo $\mathcal{R}$ di $\mathbb{P}(V)$ e una base $\mathcal{B}$ di $V$ normalizzata rispetto ad $\mathcal{R}$. Il sottospazio vettoriale $W$ di $V$ \`e descritto dalle equazioni parametriche $AX = Y$, dove $A$ ha rango $\dim W$, $X = (x_0 \cdots x_n)^t$ e $Y = (y_0 \cdots y_n)^t$, se e solo se il sottospazio proiettivo $\mathbb{P}(W)$ di $\mathbb{P}(V)$ \`e descritto dalle stesse equazioni parametriche $AX = Y$.
\end{Theorem}
\Proof Sia $W$ descritto dalle equazioni parametriche $AX = Y$. Sia $[w] \in \mathbb{P}(W)$: abbiamo $w \in W$ (teorema \ref{th1}) e supponiamo che $Y_0$ sia il vettore delle coordinate di $w$. Esiste allora un vettore di parametri $X_0$ tali che $AX_0 = Y_0$. Ma $Y_0$ \`e anche un vettore di coordinate omogenee per $[w]$, dunque $\mathbb{P}(W)$ \`e contenuto nel sottospazio proiettivo descritto dalle equazioni parametriche $AX = Y$.
	\par Supponiamo ora invece $[y] \in \mathbb{P}(V)$ e sia $Y_1$ un vettore delle coordinate omogenee per $[y]$ tale che esistano parametri $X_1$ per cui $AX_1 = Y_1$: essendo le coordinate omogenee determinate a meno di moltiplicazione per uno scalare non nullo, possiamo supporre $Y_1$ sia anche il vettore delle coordinate per $x$, e l'equazione resta vera moltiplicando il vettore dei parametri per la stessa costante. Dunque $y \in W$, da cui $[y] \in \mathbb{P}(W)$ e questo conclude la prima parte della dimostrazione.
	\par Supponiamo adesso che $\mathbb{P}(W)$ sia descritto dalle equazioni parametriche $AX = Y$. Sia $w \in W - \lbrace 0 \rbrace$. Allora $[w] \in \mathbb{P}(W)$ e, se $Y_0$ \`e il vettore delle coordinate di $w$, nonch\'e un vettore di coordinate omogenee per $[w]$, esiste un vettore dei parametri $X_0$ tale che $AX_0 = Y_0$. Dunque $W - \lbrace 0 \rbrace$ \`e contenuto nel sottospazio vettoriale descritto dalle equazioni paramtriche $AX = Y$.
	\par Supponiamo ora invece $x \in V - \lbrace 0 \rbrace$ di coordinate $Y_1$ tale che, per un'opportuna scelta di parametri $X_1$, sia verificata l'uguaglianza $AX_1 = Y_1$. Allora $Y_1$ \`e anche un vettore di coordinate omogenee per $[x]$ e pertanto $[x] \in \mathbb{P}(W)$, da cui, per il teorema \ref{th1}, $x \in W$.
	\par Inoltre, se $x = 0$, \`e immediato che $x$ appartiene sia allo spazio vettoriale descritto da $AX = Y$ che a $W$, e questo conclude la dimostrazione. \EndProof

\section{Propriet\`a topologiche di alcuni spazi proiettivi e alcuni confronti tra geometria proiettiva e geometria affine.}

	Considereremo la topologia euclidea su $\mathbb{R}^{n + 1}$ e su $\mathbb{C}^{n + 1}$ e le topologie quoziente che se ne deducono sui rispettivi spazi proiettivi $\mathbb{P}^n(\mathbb{R})$ e $\mathbb{P}^n(\mathbb{C})$.
\begin{Theorem}\label{thtop1}
	Sia $G$ il gruppo delle omotetie di $\mathbb{K}^{n + 1} - \lbrace 0 \rbrace$. Gli spazi topologici $\mathbb{P}^n(\mathbb{K})$ e $\mathbb{K}^{n + 1}/G$ sono identici.
\end{Theorem}
\Proof \`E sufficiente provare l'uguaglianza tra gli insiemi: l'uguaglianza delle strutture topologiche segue dalla definizione di topologia quoziente.
	\par Siano $x, y \in \mathbb{K}^{n + 1}$. Abbiamo $x \sim y$ se e solo se esiste un $\lambda \in \mathbb{K}^*$ tale che $x = \lambda y$, ci\`o che equivale a dire che esite un'omotetia $\rho$ di rapporto $\lambda$ tale che $x = \rho(y)$. \EndProof
\begin{Theorem}\label{thtop2}
	Se $\mathbb{K} = \mathbb{R}$ o $\mathbb{K} = \mathbb{C}$, allora $\mathbb{P}^n(\mathbb{K})$ \`e uno spazio topologico connesso per archi e compatto.
\end{Theorem}
\Proof Per definizione di topologia quoziente, la proiezione canonica $\pi: \mathbb{\mathbb{K}}^{n + 1} \rightarrow \mathbb{P}^n(\mathbb{K})$ \`e un'identificazione, e dunque un'applicazione continua e surgettiva. La restrizione di $\pi$ a $S^n$ se $\mathbb{K} = \mathbb{R}$ o a $S^{2n+1}$ se $\mathbb{K} = \mathbb{C}$ \`e ancora continua e surgettiva. La tesi segue dal fatto che l'immagine tramite un'applicazione continua di spazi connessi per archi \`e connessa per archi e l'immagine tramite un'applicazione continua di spazi compatti \`e compatta. \EndProof
\begin{Theorem}\label{thtop3}
	$\mathbb{P}^n(\mathbb{R})$ \`e omeomorfo a $S^n/H$ e $\mathbb{P}^n(\mathbb{C})$ \`e omeomorfo a $S^{2n + 1}/H$, dove $H$ \`e il gruppo di omeomorfismi dei $2$ elementi $\Id$ e $- \Id$.
\end{Theorem}
\Proof Dimostriamo solo il caso reale: il caso complesso \`e analogo.
	\par Consideriamo l'inclusione canonica $i: S^n \rightarrow \mathbb{R}^{n + 1}$, l'applicazione $r: \mathbb{R}^{n + 1} \rightarrow S^n$ che a $x$ associa $r(x) = \frac{x}{\Vert x \Vert}$, le proiezioni canoniche $\pi_1: S^n \rightarrow S^n/H$ e $\pi_2: \mathbb{R}^{n + 1} - \lbrace 0 \rbrace \rightarrow \mathbb{P}^n(\mathbb{R})$: tutte queste applicazioni sono continue.
\par $\pi_2 \circ i$ e $\pi_1 \circ r$ sono applicazioni continue in quanto composizioni di applicazioni continue e $\pi_1$ e $\pi_2$ sono identificazioni per definizione di topologia quoziente: le applicazioni $f$ e $g$ ottenute per passaggio al quoziente da $\pi_2 \circ i$ e $\pi_1 \circ r$ rispettivamente sono dunque continue.
\par Sia $x \in S^n/H$. Scelto $y \in S^n$ tale che $\pi_1(y) = x$, abbiamo $(g \circ f)(x) = g(f(x)) = g(\pi_2(i(y))) = g(\pi_2(y)) = \pi_1(r(y)) = \pi_1(y) = x$.
\par Sia adesso invece $x \in \mathbb{P}^n(\mathbb{R})$ e scegliamo $y \in S^n \subset \mathbb{R}^{n + 1}$ tale che $\pi_2(y) = x$. Abbiamo $(f \circ g)(x) = f(g(x)) = f(\pi_1(r(y))) = f(\pi_1(y)) = \pi_2(i(y)) = \pi_2(y) = x$.
\par Dunque le applicazioni $f$ e $g$ sono l'una l'inversa dell'altra. \EndProof
\begin{Corollary}\label{cortop1}
	$\mathbb{P}^n(\mathbb{R})$ e $\mathbb{P}^n(\mathbb{C})$ sono spazi di Hausdorff.
\end{Corollary}
\Proof Segue dal teorema precedente, dal fatto che $S^n$ \`e uno spazio di Hausdorff e che $H$ \`e un gruppo finito. \EndProof
\par Forniamo una seconda dimostrazione del corollario precedente per il caso complesso.
\begin{Theorem}\label{thtop4}
	$\mathbb{P}^n(\mathbb{C})$ \`e uno spazio di Hausdorff.
\end{Theorem}
\Proof Per un noto teorema di topologia e per il teorema \ref{thtop1}, $\mathbb{P}^n(\mathbb{C})$ \`e di Hausdorff se e solo se l'insieme $K = \lbrace x \in (\mathbb{C}^{n + 1} - \lbrace 0 \rbrace) \times (\mathbb{C}^{n + 1} - \lbrace 0 \rbrace)\ |\ (\exists y \in \mathbb{C}^{n + 1} - \lbrace 0 \rbrace)(\exists g \in G)(x = (y, g(y))) \rbrace$, dove $G$ \`e il gruppo delle omotetie di $\mathbb{C}^{n + 1} - \lbrace 0 \rbrace$, \`e chiuso.
	\par Sia $\mathcal{M}(\mathbb{C}, n + 1, 2)$ l'insieme delle matrici a coefficienti complessi di dimensione $(n + 1) \times 2$. Sia $f: (\mathbb{C}^{n+ 1} - \lbrace 0 \rbrace) \times (\mathbb{C}^{n + 1} - \lbrace 0 \rbrace) \rightarrow \mathcal{M}(\mathbb{C}, n + 1, 2) - \lbrace 0 \rbrace$ l'applicazione che a $x \in \mathbb{C}^{n + 1} \times \mathbb{C}^{n + 1}$ associa la matrice $f(x)$ la cui prima colonna \`e costituita dal primo vettore di $x$ e la cui seconda colonna \`e costituita dal secondo vettore di $x$. $f$ \`e continua e surgettiva. Consideriamo su $\mathcal{M}(\mathbb{C}, n + 1, 2) - \lbrace 0 \rbrace$ la topologia quoziente rispetto ad $f$. Dunque $K = f^{-1}(f(K))$ \`e chiuso se e solo se $f(K)$ \`e chiuso.
	\par Ora, $f(K)$ coincide con l'insieme delle matrici di rango $1$. Definiamo l'applicazione $D: \mathcal{M}(\mathbb{C}, n + 1, 2) - \lbrace 0 \rbrace \rightarrow \mathbb{C}$ che a $x$ associa il prodotto dei determinanti dei minori $2 \times 2$ di $x$: $D$ \`e continua e $f(K)$ risulta essere l'immagine inversa di $0$, chiuso in $\mathbb{C}$. \EndProof
\begin{Definition}\label{deftop1}
	Per ogni $i = 0, ..., n$, sia $H_i$ l'iperpiano fondamentale di $\mathbb{P}^n(\mathbb{K})$ descritto dall'equazione cartesiana $x_i = 0$ e sia $U_i = \mathbb{P}^n(\mathbb{K}) - H_i$. Consideriamo inoltre le applicazioni invertibili $j_i: \mathbb{K}^n \rightarrow U_i$ che a $(x_1, ..., x_n)$ associa $[y_0, ..., y_n]$, dove per ogni $k = 0, ..., n$ abbiamo $y_k = \begin{cases} x_{k + 1}\text{, se } k < i,\\ 1\text{, se } k = i,\\ x_k \text{, se } k > i. \end{cases}$ Ognuna delle coppie della famiglia $((U_i, j_i^{-1}))_{i = 0, ..., n}$ si chiama \Define{carta affine standard}[affine standard][carta] di $\mathbb{P}^n(\mathbb{K})$\footnote{La dicitura \NIDefine{standard} \`e mia. In ESERCIZIARIO viene chiamata \NIDefine{standard} solo la carta $(U_0, j_0^{-1})$. Non ho trovato altri documenti che chiamassero \NIDefine{standard} una qualche carta.}; la famiglia stessa invece si chiama \Define{atlante standard}[standard][atlante]\footnote{La dicitura \NIDefine{standard} per l'atlante \`e invece comune.}.
\end{Definition}
	\par Le notazioni del teorema precedente saranno mantenute per tutta la durata del documento, anche quando non esplicitamente indicato. \`E immediato osservare che ogni $j_i$ \`e un omeomorfismo.
\begin{Definition}\label{deftop2}
	Sia $H$ un iperpiano di $\mathbb{P}(V)$ e sia $f: \mathbb{P}^n(K) \rightarrow \mathbb{P}(V)$ un isomorfismo proiettivo tale che $f(H_0) = H$\footnote{Esiste per il teorema fondamentale delle applicazioni proiettive.}. Poniamo $U_H = f(U_0)$ e $j_H = f \circ j_0$. La coppia $(U_H,j_H^{-1})$ viene chiamata \Define{carta affine}[affine][carta] di $\mathbb{P}(V)$; dato $P \in U_H$, $P$ si dice \Define{punto proprio}[proprio][punto] rispetto alla carta $(U_H, j_H^{-1})$ e le componenti di $j_H^{-1}(P)$ si chiamano \Define{coordinate affini}[affini][coordinate] di $P$ rispetto alla carta $(U_H, j_H^{-1})$; dato invece $Q \in H$, $Q$ si dice \Define{punto improprio}[improprio][punto] o \Define{punto all'infinito}[all'infinito][punto] rispetto a $(U_H, j_H^{-1})$.
\end{Definition}
	\par A volte una carta $(U_H, j_H^{-1})$ viene indicata per abuso di linguaggio solo con $U_H$, $j_H^{-1}$ o $j_H$.
\begin{Definition}\label{deftop3}
	Uno spazio topologico $X$ si dice una \Define{variet\`a} di dimensione $n$ (o $n$-variet\`a), dove $n \in \mathbb{N}$, quando
	\begin{itemize}
		\item $X$ \`e uno spazio di Hausdorff;
		\item per ogni $x \in X$ esiste un intorno aperto $U$ di $x$ tale che $U$ \`e omeomorfo ad un aperto di $\mathbb{R}^n$ (si dice che $X$ \`e \NIDefine{localmente euclideo});
		\item ogni componente connessa di $X$ soddisfa il secondo assioma di numerabilit\`a.
	\end{itemize}
\end{Definition}
\begin{Definition}\label{deftop4}
	Dati una $n$-variet\`a $X$ e un suo sottoinsieme $U$ omeomorfo ad un aperto $V$ di $\mathbb{R}^n$ tramite l'omeomorfismo $j: V \rightarrow U$, la coppia $(U,j^{-1})$ si chiama \Define{carta}. Se $(U_i)_{i \in I}$ \`e una famiglia di sottoinsiemi di $X$ che ricopre $X$ omeomorfi a un qualche aperto $V_i$ di $\mathbb{R}^n$ tramite gli omeomorfismi $j_i: V_i \rightarrow U_i$, allora chiamiamo \Define{atlante} la famiglia di carte $((U_i,j_i^{-1}))_{i \in I}$.
\end{Definition}
\begin{Theorem}\label{thtop5-1}
	$\mathbb{P}^n(\mathbb{R})$ \`e omeomorfo a $D^n/\sim$ e $\mathbb{P}^n(\mathbb{C})$ \`e omeomorfo a $D^{2n + 1}/\sim$, dove $\sim$ \`e la relazione d'equivalenza che contrae il bordo di $D^n$ ad un punto.
\end{Theorem}
\Proof Dimostriamo il caso reale. Il caso complesso \`e analogo.
\par Sia $f: S^n_+ \rightarrow D^n$, dove $S^n_+ = S^n \cap \lbrace x = (x_0, ..., x_n) \in \mathbb{R}^{n + 1} | x_0 \geq 0 \rbrace$, l'applicazione che a $x = (x_0, ..., x_n) \in S^n_+$ associa $f(x) = (x_1, ..., x_n)$: l'applicazione $f$ \`e evidentemente un omeomorfismo. Dunque $S^n_+/H$ \`e omeomorfo a $D^n/\sim$, dove $H = \lbrace \Identity, - \Identity \rbrace$\footnote{L'omeomorfsimo si ottiene da $f$ per passaggio ai quozienti.}. La tesi segue dal teorema \ref{thtop3}. \EndProof
\begin{Theorem}\label{thtop5}
	$\mathbb{P}^n(\mathbb{R})$ \`e una variet\`a di dimensione $n$ e $\mathbb{P}^n(\mathbb{C})$ \`e una variet\`a di dimensione $2n$.
\end{Theorem}
\Proof Per il corollario \ref{cortop1} e per il teorema \ref{thtop4}, i due spazi sono di Hausdorff. Le costruzioni degli atlanti standard provano che essi sono anche localmente euclidei e individua la dimensione delle variet\`a. Entrambi gli spazi sono connessi, quindi occorre dimostrare che essi verificano il secondo assioma di numerabilit\`a.
\par Per il teorema \ref{thtop5-1} \`e sufficiente dimostrare che $D^n/\sim$ e $D^{2n + 1}/\sim$ ammettono entrambi una base numerabile; dimostreremo solo il primo caso. $D^n$ \`e un sottospazio di $\mathbb{R}^{n + 1}$, quindi ammette a base numerabile: sia $\mathcal{B}$ una base numerabile di $D^n$ e sia $\mathcal{B}'$ la famiglia (numerabile) delle immagini degli elementi di $\mathcal{B}$ in $D^n/\sim$ attraverso la proiezione canonica $\pi: D^n \rightarrow D^n/\sim$.
\par Sia $A$ un aperto di $D^n/\sim$. Allora $\pi^{-1}(A)$ \`e aperto per definizione di topologia quoziente. Se $A$ non contiene il punto $x$ a cui \`e stato contratto il bordo di $D^n$, allora $\pi^{-1}(A)$ \`e unione di elementi della base $\mathcal{B}$ tutti disgiunti dal bordo di $D^n$, e dunque saturi per la contrazione. Pertanto $A = \pi(\pi^{-1}(A))$ \`e unione di insiemi aperti della famiglia $\mathcal{B}'$.
\par Supponiamo ora invece $x \in A$. $D^n/\sim$ \`e di Hausdorff, quindi $\lbrace x \rbrace$ \`e chiuso e dunque $A - \lbrace x \rbrace$ \`e ancora aperto: il caso precedente garantisce che quest'ultimo ammette un ricoprimento numerabile $\mathcal{R}$ con elementi di $\mathcal{B}'$. Inoltre, $x \in A$ implica che il bordo di $D^n$ sia contenuto in $\pi^{-1}(A)$. Essendo $\pi^{-1}(A)$ aperto, esiste un elemento $B \in \mathcal{B}$ tale che $B \subseteq \pi^{-1}(A)$ e $B \cap \partial D^n \neq \varnothing$. La famiglia $\mathcal{R}'$, ottenuta da $\mathcal{R}$ aggiungendovi l'insieme $B$ ricopre $A$ e questo conclude la dimostrazione.
	\par Se $(U_H, j_H^{-1})$ \`e una carta, allora $\mathbb{K}^n$ si immerge in $\mathbb{P}^n(\mathbb{K})$ tramite $i \circ j_H$, dove $i: U_H \rightarrow \mathbb{P}^n(\mathbb{K})$ \`e l'inclusione. Possiamo dunque vedere i punti propri di $\mathbb{P}^n(\mathbb{K})$ per la carta data come punti di $\mathbb{K}^n$ e i punti all'infinito come punti ``aggiunti'' a $\mathbb{K}^n$.
\begin{Theorem}\label{thtop6}
	Sia $\AffineGroup(\mathbb{K}^n)$ il gruppo delle affinit\`a di $\mathbb{K}^n$. Sia $G$ il sottogruppo di $\mathbb{P}GL(\mathbb{K}^{n + 1})$ che lascia fisso un iperpiano arbitrario $H$ di $\mathbb{P}^n(\mathbb{K})$. Allora $\AffineGroup(\mathbb{K}^{n + 1}) \cong G$.
\end{Theorem}
\Proof A meno di un cambio di coordinate, possiamo supporre $H$ descritto dall'equazione cartesiana $x_0 = 0$. Sia $f \in G$ e $A$ una matrice rappresentante $f$ per il riferimento proiettivo canonico. $f$ fissa $H$ dunque tutti gli elementi di $A$ sulla prima riga tranne il primo sono nulli e possiamo scegliere $A$ in modo che il primo elemento sulla prima riga e sulla prima colonna sia $1$. Abbiamo dunque $A$ della forma $A = \left ( \begin{array}{c|c} 1 & \cdots 0 \cdots\\ \hline B & C \end{array} \right )$; $C$ \`e invertibile.
	\par Sia $\tilde{f} = j_H^{-1} \circ f \circ j_H$. Abbiamo, per $X \in \mathbb{K}^n$, $\tilde{f}(X) = j_H^{-1}(A (1 \cdots X \cdots)^t) = j_H^{-1}(1 \cdots CX + B \cdots) = CX + B$, dunque $\tilde{f} \in \AffineGroup(\mathbb{K}^n)$
	\par Consideriamo l'applicazione $\phi: G \rightarrow \AffineGroup(\mathbb{K}^n)$ che a $f \in G$ associa $\tilde{f}$ costruita come appena descritto. Si verifica direttamente che $\phi$ \`e un isomorfismo. \EndProof
\begin{Definition}\label{deftop5}
	Consideriamo $W$ sottospazio affine di $\mathbb{K}^n$ e una carta $(U_H, j_H^{-1})$ di $\mathbb{P}^n(\mathbb{K})$. La \Define{chiusura proiettiva}[proiettiva][chiusura] $\bar{W}$ di $W$ rispetto alla carta $(U_H, j_H^{-1})$ \`e il pi\`u piccolo sottospazio proiettivo di $\mathbb{P}^n(\mathbb{K})$ contenente $j_H(W)$.
\end{Definition}
\begin{Theorem}\label{thtop7}
	Sia $W$ il sottospazio affine di $\mathbb{K}^n$ descritto dalle equazioni cartesiane $AX + B = 0$, dove $X = (x_1, ..., x_n)^t$ \`e il vettore delle incognite, e sia $W'$ il sottospazio proiettivo di $\mathbb{P}^n(\mathbb{K})$ descritto dalle equazioni cartesiane $CX' = 0$, dove $C = \left ( \begin{array}{c|c} B & A \end{array} \right )$ e $X' = (x_0, x_1, ..., x_n)$ \`e il vettore delle incognite. Allora $W' = \bar{W}$, dove $\bar{W}$ \`e la chiusura proiettiva rispetto alla carta affine standard $(U_0,j_0^{-1})$. Inoltre $X' \in \bar{W}$ \`e un punto proprio se e solo se $X' \in j_0(W)$.
\end{Theorem}
\Proof La chiusura proiettiva di $W$ \`e, per il corollario \ref{cor0.1}, il proiettivizzato del pi\`u piccolo sottospazio vettoriale $R$ di $\mathbb{K}^{n + 1}$ contenente l'insieme dei punti $X'$ tali che la loro prima componente \`e $1$ e il vettore delle sue restanti componenti soddisfa $AX + B = 0$, condizioni equivalenti a $CX' = 0$. Quindi, per il teorema \ref{th22}, $W' = \bar{W}$.
	\par Sia $X' \in W'$ un punto proprio. Allora possiamo assumere che la sua prima componente sia $1$; denotiamo $X$ il vettore delle componenti restanti: abbiamo $j_0(X) = X'$. Abbiamo $CX' = 0$, che \`e equivalente a $AX + B = 0$, a sua volta equivalente a $X \in W$. Dunque $X' \in \bar{W}$ \`e un punto proprio se e solo se $X' \in j_0(W)$. \EndProof
\begin{Corollary}\label{cortop2}
	Sia $W$ un sottospazio affine di $\mathbb{K}^n$ e supponiamo fissata una carta affine: abbiamo $\dim \bar{W} = \dim W$.
\end{Corollary}
\Proof A meno di un cambiamento di riferimento proiettivo, possiamo supporre che la carta affine sia la carta standard $(U_0, j_0^{-1})$; manteniamo le notazioni del teorema \ref{thtop7}. La dimensione di $W$ \`e uguale ad $n - k$, dove $k$ \`e il rango della matrice $A$. La dimensione di $\bar{W}$ \`e uguale a $n + 1 - q - 1 = n - q$, dove $q$ \`e il rango della matrice $C$. $W$ \`e uno spazio affine, dunque non vuoto. Per il teorema di Rouch\'e - Capelli $k = q$, da cui segue la tesi del teorema. \EndProof
\begin{Theorem}\label{thtop8}
	Sia $T$ un sottospazio proiettivo di $\mathbb{P}^n(\mathbb{K})$ non interamente contenuto in $H_0$ - l'iperpiano fondamentale di equazione $x_0 = 0$ - descritto dall'equazione cartesiana $CX' = 0$, dove $X' = (x_0, x_1, ..., x_n)$ e consideriamo la carta affine standard $(U_0, j_0^{-1})$. L'insieme $j_0^{-1}(T \cap U_0)$ \`e il sottospazio affine di $\mathbb{K}^n$ descritto dall'equazione cartesiana $AX + B = 0$, dove $B$ \`e la prima colonna di $C$ e $A$ la matrice ottenuta da $C$ togliendone la prima colonna e $X = (x_1, ..., x_n)$.
\end{Theorem}
\Proof Sia $X' \in T \cap U_0$. $X'$ ha prima coordinata non nulla: possiamo suporre che tale coordinata sia $1$; designamo con $X$ il vettore delle restanti componenti. Allora abbiamo $CX' = 0$ se e solo se $AX + B = 0$ (la condizione di uguaglianza dei ranghi di $A$ e di $C$ \`e soddisfatta per ipotesi). La tesi segue constatando che $X = j_0^{-1}(X')$. \EndProof
\begin{Theorem}\label{thtop9}
	Sia $\phi: S \rightarrow \mathbb{S}_0$, dove $S$ \`e l'insieme dei sottospazi affini di $\mathbb{K}^n$ e $\mathbb{S}_0$ l'insieme dei sottospazi proiettivi di $\mathbb{P}^n(\mathbb{K})$ incidenti con $U_0$, l'applicazione che a $W$ associa $\bar{W}$ rispetto alla carta standard $(U_0, j_0^{-1})$. L'applicazione $\phi$ \`e biettiva con inversa $\psi: \mathbb{S} \rightarrow S$ tale che $\psi(T) = j_0^{-1}(T \cap U_0)$ e manda sottospazi affini in sottospazi proiettivi della stessa dimensione.
\end{Theorem}
\Proof Siano $X = (x_1, ..., x_n)$ e $X' = (x_0, x_1, ..., x_n)$. Sia $W \in S$ descritto dall'equazione $AX + B = 0$ e sia $C = \left ( \begin{array}{c|c} B & A \end{array} \right )$. Allora, sfruttando i teoremi \ref{thtop7} e \ref{thtop8}, abbiamo che $\phi(W)$ \`e il sottospazio proiettivo descritto dall'equazione $CX' = 0$ e che $\psi(\phi(W))$ \`e il sottospazio affine descritto da $AX + B = 0$, quindi $\psi \circ \phi = \Identity_S$. Analogamente si dimostra $\phi \circ \psi = \Identity_{\mathbb{S}}$. \EndProof
\begin{Theorem}\label{thtop10}
	$\mathbb{P}^n(\mathbb{C})$ \`e semplicemente connesso per ogni $n \in \mathbb{N}$.
\end{Theorem}
\Proof Se $n = 0$, $\mathbb{P}^n(\mathbb{C})$ \`e ridotto ad un solo punto, necessariamente semplicemente connesso. Supponiamo il teorema vero cambiando $n$ in $n - 1$.
\par Consideriamo l'iperpiano fondamentale $H_0$ definito dall'equazione $x_0 = 0$. Poniamo $A = \mathbb{P}^n(\mathbb{C}) - H_0$ e $B = \mathbb{P}^n(\mathbb{C}) - \lbrace [1, 0, ..., 0]$. Abbiamo evidentemente $A \cup B = \mathbb{P}^n(\mathbb{C})$. Inoltre $A$ \`e semplicemente connesso in quanto omeomorfo tramite $j_0$ a $\mathbb{C}^n$, semplicemente connesso; $B$ \`e anch'esso semplicemente connesso in quanto esso si deforma in $H$ tramite l'applicazione che a $([x_0, ..., x_n], t) \in B \times [0,1]$ associa $[tx_0, ..., x_n]$, ed $H$ \`e omeomorfo a $\mathbb{P}^{n - 1}(\mathbb{C})$, semplicemente connesso per ipotesi. La tesi segue dunque dal teorema di Van Kampen. \EndProof
\begin{Theorem}\label{thtop11}
	$\mathbb{P}^1(\mathbb{R})$ \`e omeomorfo a $S^1$.
\end{Theorem}
\Proof $\mathbb{P}^1(\mathbb{R})$ \`e compatto e di Hausdorff ed \`e dunque omeomorfo alla compattificazione di Alexandroff di $\mathbb{P}^1(\mathbb{R}) - \lbrace [0, 1] \rbrace$, omeomorfo a $\mathbb{R}$ tramite l'applicazione $[x_0, x_1] \mapsto \frac{x_1}{x_0}$.
\par D'altra parte $\mathbb{R}$ \`e omeomorfo per proiezione stereografica a $S^1 - (0,1)$, la cui compattificazione di Alexandroff \`e $S^1$, essendo $S^1$ compatto e di Hausdorff.
\par In conclusione, $\mathbb{P}^1(\mathbb{R})$ \`e omeomorfo a $S^1$. \EndProof
\begin{Theorem}\label{thtop12}
	$\mathbb{P}^0(\mathbb{R})$ \`e semplicemente connesso. Il gruppo fondamentale di $\mathbb{P}^1(\mathbb{R})$ \`e isomorfo a $\mathbb{Z}$, quello di $\mathbb{P}^n(\mathbb{R})$ con $n \geq 2$ \`e isomorfo a $\mathbb{Z}/(2)$.
\end{Theorem}
\Proof La semplice connessione di $\mathbb{P}^0(\mathbb{R})$ \`e evidente.
\par Per il teorema \ref{thtop11}, $\mathbb{P}^1(\mathbb{R})$ \`e omeomorfo alla sfera $S^1$, la quale ha gruppo fondamentale $\mathbb{Z}$.
\par Sia ora $n \geq 2$. Consideriamo il rivestimento $p: S^n \rightarrow \mathbb{P}^n(\mathbb{R})$ e fissiamo due punti $e \in S^n$ e $x_0 \in \mathbb{P}^n(\mathbb{R})$ tali che $p(e) = x_0$. Sia $x \in \mathbb{P}^n(\mathbb{R})$. Il rivestimento $p$ \`e connesso quindi $p^{-1}(x)$ ha cardinalit\`a uguale all'indice del sottogruppo $p_* \pi_1(S^n, e)$ nel gruppo $\pi_1(\mathbb{P}^n(\mathbb{R}), x_0)$. D'altra parte, $S^n$ \`e semplicemente connesso e pertanto l'ordine di $p_* \pi_1(S^n, e)$ \`e $1$. L'ordine di $\pi_1(\mathbb{P}^n(\mathbb{R}), x_0)$ \`e dunque $2$ e $\pi_1(\mathbb{P}^n(\mathbb{R}), x_0)$ \`e necessariamente isomorfo a $\mathbb{Z}/(2)$. \EndProof
\par	Concludiamo questa sezione osservando che i teoremi che sono stati enunciati e dimostrati solo per la carta affine standard $(U_0,j_0^{-1})$ si generalizzano facilmente per carte affini qualsiasi.

\section{Spazio proiettivo duale e dualit\`a.}

\par In questa sezione $\mu$ indica l'isomorfismo canonico dal biduale $V^{**}$ di $V$ in $V$. In questa sezione assumeremo $n$, dimensione dello spazio proiettivo $\mathbb{P}(V)$, \underline{finito}.
\begin{Definition}\label{def24}
	Sia $\mathbb{P}(V)$ lo spazio proiettivo associato allo spazio vettoriale $V$. $\mathbb{P}(V^*)$, dove $V^*$ \`e il duale di $V$, si chiama \Define{spazio proiettivo duale}[proiettivo duale][spazio] di $\mathbb{P}(V)$ e si denota $\mathbb{P}(V)^*$.
\end{Definition}
\begin{Definition}\label{def25}
	Siano $\mathcal{R}$ un riferimento proiettivo di $\mathbb{P}(V)$, $\mathcal{B}$ una base di $V$ normalizzata rispetto ad $\mathcal{R}$. Chiamiamo \Define{riferimento proiettivo duale}[proiettivo duale][riferimento] di $\mathcal{R}$, denotato $\mathcal{R}^*$, il riferimento proiettivo i cui elementi sono le classi d'equivalenza degli elementi della base $\mathcal{B}^*$ duale di $\mathcal{B}$ e la classe d'equivalenza della somma degli elementi della base duale $\mathcal{B}^*$.
\end{Definition}
\begin{Theorem}\label{th25}
	Per ogni $k = -1, ..., n$, siano
	\begin{itemize}
		\item $\mathbb{S}_k$ l'insieme dei sottospazi proiettivi di $\mathbb{P}(V)$ di dimensione $k$, ordinato per inclusione;
		\item $\mathbb{S}_k^*$ l'insieme dei sottospazi proiettivi di $\mathbb{P}(V)^*$ di dimensione $k$, ordinato per la relazione d'ordine opposta all'inclusione;
		\item $\delta_k: \mathbb{S}_k \rightarrow \mathbb{S}_{n - k - 1}^*$ l'applicazione che a $\mathbb{P}(W)$ associa $\mathbb{P}(\Annihilator(W))$.
	\end{itemize}
	Allora, per ogni $k = -1, ..., n$, $\delta_k$ \`e un isomorfismo d'insiemi ordinati.
\end{Theorem}
\Proof Fissiamo $k \in \lbrace -1, ..., n \rbrace$. Consideriamo l'insieme $S_k$ dei sottospazi vettoriali di $V$ di dimensione $k + 1$, ordinato per inclusione, l'insieme $S_k^*$ dei sottospazi vettoriali di $V^*$ di dimensione $k + 1$, ordinato con la relazione d'ordine opposta alla relazione d'inclusione, e l'applicazione $\delta': S_k \rightarrow S_{n - k - 1}^*$ che a $W \in S_k$ associa $\Annihilator W \in S_{n - k - 1}^*$. Definiamo analogamente l'applicazione $\delta'': S_{n - k - 1}^* \rightarrow S_k^{**}$ che a $U \in S_{n - k - 1}^*$ associa $\Annihilator U \in S_k^{**}$, dove $S_k^{**}$ \`e l'insieme dei sottospazi vettoriali di $V^{**}$.
	\par Abbiamo $((\mu \circ \delta'') \circ \delta')(W) = \mu(\Annihilator (\Annihilator W)) = W$ e $(\delta' \circ (\mu \circ \delta''))(U) = \Annihilator (\mu(\Annihilator U)) = U$, dunque $\delta'$ \`e biettiva e la verifica che sia un isomorfismo d'insiemi ordinati \`e immediata. La tesi segue per il corollario \ref{cor0.1}. \EndProof
\begin{Definition}\label{def26}
	Con le notazioni del teorema precedente, sia $\delta: \mathbb{S} \rightarrow \mathbb{S}^*$, dove $\mathbb{S}$ \`e l'insieme dei sottospazi proiettivi di $\mathbb{P}(V)$ e $\mathbb{S}^*$ l'insieme dei sottospazi proiettivi di $\mathbb{P}^*$, l'applicazione definita da $\delta(S) = \delta_{\dim S} (S)$. L'applicazione $\delta$ \`e chiamata \Define{corrispondenza di dualit\`a}[di dualit\`a][corrispondenza].
\end{Definition}
\par Per abuso di linguaggio, denoteremo con la stessa lettera $\delta$ la corrispondenza di dualit\`a di qualunque spazio proiettivo. Questo permetter\`a di un'interpretazione pi\`u intuitiva dei teoremi che seguono.
\begin{Theorem}\label{th25.1}
	Vale la relazione $\delta^2 = \bar{\mu}$. In altri termini, la corrispondenza di dualit\`a $\delta$ \`e un'involuzione se identifichiamo $V$ e il suo biduale $V^{**}$ tramite l'isomorfismo canonico $\mu$, e dunque $\mathbb{P}(V)$ e $\mathbb{P}(V)^{**}$ tramite la proiettivit\`a $\bar{\mu}$.
\end{Theorem}
\Proof Sia $S$ un sottospazio proiettivo proprio di $\mathbb{P}(V)$ corrispondente al sottospazio vettoriale $W$ di $V$. Allora $(\delta \circ \delta)(S) = \delta(\delta(S)) = \delta(\mathbb{P}(\Annihilator W)) = \mathbb{P}(\Annihilator (\Annihilator W)) = \mathbb{P}(\mu(W)) = \bar{\mu}(S)$. \EndProof
\begin{Corollary}\label{corth25.1}
	Sia $S$ un sottospazio proiettivo proprio di $\mathbb{P}(V)$. Se $T = \delta(S)$, allora $\delta(T) = \bar{\mu}(S)$.
\end{Corollary}
\Proof Abbiamo $\delta(T) = \delta^2(S) = \bar{\mu}(S)$. \EndProof
\begin{Theorem}\label{th26}
	Data una famiglia \underline{finita}\footnote{Non so se la dimostrazione che ho scritto valga anche nel caso infinito: ci\`o dipende da se l'annullatore dell'intersezione di una famiglia infinita di sottospazi vettoriali sia uguale alla somma degli annullatori dei singoli sottospazi e da se l'annullatore della somma di una famiglia finita di sottospazi sia uguale all'intersezione degli annullatori dei singoli sottospazi.} $(S_i)_{i \in I}$ sottospazi proiettivi di $\mathbb{P}(V)$ sono verificate le relazioni $\delta(\bigcap_{i \in I} S_i) = L(\bigcup_{i \in I} \delta(S_i))$ e $\delta(L(\bigcup_{i \in I} S_i)) = \bigcap_{i \in I} \delta(S_i)$.
\end{Theorem}
\Proof Siano $(W_i)_{i \in I}$ una famiglia di sottospazi vettoriali di $V$ tali che per ogni $i \in I$ si abbia $S_i = \mathbb{P}(W_i)$. Abbiamo $\delta(\bigcap_{i \in I} S_i) = \delta(\bigcap_{i \in I} \mathbb{P}(W_i)) = \delta(\mathbb{P}(\bigcap_{i \in I} W_i)) = \mathbb{P}(\Annihilator (\bigcap_{i \in I} W_i)) = \mathbb{P}(\sum_{i \in I} \Annihilator W_i) = L(\bigcup_{i \in I} \mathbb{P}(\Annihilator W_i)) = L(\bigcup_{i \in I} \delta(S_i))$. E ancora $\delta(L(\bigcup_{i \in I} S_i)) = \delta(L(\bigcup_{i \in I} \mathbb{P}(W_i))) = \delta(\mathbb{P}(\sum_{i \in I} W_i)) = \mathbb{P}(\Annihilator (\sum_{i \in I} W_i)) = \bigcap_{i \in I} \mathbb{P}(\Annihilator W_i) = \bigcap_{i \in I} \delta(S_i)$. \EndProof
\begin{Definition}\label{def27}
	Sia $\mathcal{P}$ una proposizione che riguarda i sottospazi proiettivi di uno spazio proiettivo $\mathbb{P}(V)$ di dimensione $n$. Chiamiamo \Define{proposizione duale}[duale][proposizione] di $\mathcal{P}$, denotata $\mathcal{P}^*$, la proposizione ottenuta sostituendo i termini ``intersezione'', ``sottospazio generato'', ``contenuto'', ``contenente'', ``dimensione'' coi termini ``sottospazio generato'', ``intersezione'', ``contenente'', ``contenuto'', ``dimensione duale'', dove la ``dimensione duale'' vale $n - k - 1$ se la dimensione vale $k$. Se $\mathcal{P}$ e $\mathcal{P}^*$ sono la stessa proposizione, allora $\mathcal{P}$ viene detta \Define{autoduale}[autoduale][proposizione].
\end{Definition}
\begin{Principle}\label{princ1}
	\TheoremName{Principio di dualit\`a}. Sia $\mathcal{P}$ una proposizione che riguarda i sottospazi proiettivi di uno spazio proiettivo $\mathbb{P}(V)$. $\mathcal{P}$ \`e vera se e solo se $\mathcal{P}^*$ \`e vera.
\end{Principle}
\Proof La dimostrazione metamatematica del principio consiste nell'applicare su $\mathcal{P}$ i teoremi \ref{th25} e \ref{th26}. \EndProof
\begin{Definition}\label{def28}
	Supponiamo fissato un riferimento proiettivo $\mathcal{R}$ su $\mathbb{P}(V)$. Sia $S$ un iperpiano proiettivo di $\mathbb{P}(V)$. Chiameremo \Define{coordinate omogenee dell'iperpiano $S$}[omogenee di un iperpiano][coordinate] le coordinate del punto $\delta(S) \in \mathbb{P}(V)^*$ rispetto al riferimento proiettivo duale $\mathcal{R}^*$\footnote{Identifichiamo $\delta(S)$ con l'unico punto di $\mathbb{P}(V)$ che esso contiene.}.
\end{Definition}
\begin{Theorem}\label{th27}
	Supponiamo fissato un riferimento proiettivo $\mathcal{R}$ di $\mathbb{P}(V)$. Sia $S$ un iperpiano proiettivo di $\mathbb{P}(V)$ descritto dall'equazione cartesiana $\sum_{i = 0}^n a_i x_i = 0$, con $a_i \in \mathbb{K}$. Le coordinate omogenee dell'iperpiano $S$ sono $[a_0, ..., a_n]$.
\end{Theorem}
\Proof Sia $W$ l'iperpiano di $V$ tale che $\mathbb{P}(W) = S$. Allora fissata una base $\mathcal{B}$ di $V$ normalizzata rispetto ad $\mathcal{R}$, per il teorema \ref{th22}, $W$ \`e descritto dalla stessa equazione cartesiana. Consideriamo il duale $V^*$ di $V$ con la base duale $\mathcal{B}^*$: un argomento dimensionale permette di concludere subito che l'annulatore di $W$ \`e dato dalla retta di $V^*$ generata dal vettore di coordinate $(a_0, ..., a_n)$. Dunque $\delta(S)$ ha coordinate omogenee $[a_0, ..., a_n]$. \EndProof
\begin{Definition}\label{def29}
	Un \Define{sistema lineare}[lineare][sistema] $\mathcal{L}$ di $\mathbb{P}(V)$ \`e un qualunque sottospazio proiettivo di $\mathbb{P}(V)^*$. Il \Define{centro}[di un sistema lineare][centro] $S$ di $\mathcal{L}$ \`e definito come $S = \bigcap_{P \in \mathcal{L}} \delta^{-1}(P)$.
\end{Definition}
\begin{Lemma}\label{Lemma1}
	Sia $\mathcal{L}$ un sistema lineare di centro $S$. Allora $\delta(S) = \mathcal{L}$ e $\delta(\mathcal{L}) = \bar{\mu}(S)$.
\end{Lemma}
\Proof Abbiamo $\delta(S) = \delta(\bigcap_{P \in \mathcal{L}} \delta^{-1}(P))$.
\par Sia $\mathcal{S}$ un insieme massimale di punti indipendenti di $\mathcal{L}$. Chiaramente, $\bigcap_{P \in \mathcal{L}} \delta^{-1}(P) \subseteq \bigcap_{P \in \mathcal{S}} \delta^{-1}(P)$.
\par Siano poi $x \in \bigcap_{P \in \mathcal{S}} \delta^{-1}(P)$ e $Q \in \mathcal{L}$: proviamo che $x \in \delta^{-1}(Q)$. Sia $W$ il sottospazio vettoriale di $V^*$ tale che $\mathbb{P}(W) = \mathcal{L}$ e sia $(s_i)_{i \in I}$ una base di $W$ tale che, per ogni $i \in I$, $[s_i] \in \mathcal{S}$. $Q \in \mathcal{L}$, dunque esiste $q \in W$ tale che $[q] = Q$. Inoltre, esiste una famiglia $(\alpha_i)_{i \in I}$ di scalari tale che $q = \sum_{i \in I} \alpha_i s_i$.
\par Sia $y \in V - \lbrace 0 \rbrace$ tale che $[y] = x$: abbiamo, per ogni $P \in \mathcal{L}$, $x \in \delta^{-1}(P)$ se e solo se $\Annihilator {y} \subseteq \tilde{P}$, dove $\tilde{p}$ \`e il sottospazio vettoriale di $V^*$ generato da un qualunque $p$ rappresentante di $P$. Abbiamo dunque, per ogni $i \in I$, $\Annihilator {y} \subseteq \tilde{s_i}$. Inoltre, sia $z \in \Annihilator {y}$: abbiamo $zq = \sum_{i \in I} \alpha_i zs_i = 0$, dunque $\Annihilator {y} \subseteq \tilde{q}$, e quindi $x \in \delta^{-1}(Q)$.
\par Abbiamo quindi provato che $\bigcap_{P \in \mathcal{L}} \delta^{-1}(P) = \bigcap_{P \in \mathcal{S}} \delta^{-1}(P)$.
\par Abbiamo dunque infine $\delta(S) = \delta(\bigcap_{P \in \mathcal{L}} \delta^{-1}(P)) = \delta(\bigcap_{P \in \mathcal{S}} \delta^{-1}(P)) = L((\delta(\delta^{-1}(P)))_{P \in \mathcal{S}}) = L((P)_{P \in \mathcal{S}}) = \mathcal{L}$. \EndProof
\begin{Theorem}\label{th28}
	Dato $S$, sottospazio proiettivo di $\mathbb{P}(V)$, esiste uno e un solo sistema lineare $\mathcal{L}$ di centro $S$.
\end{Theorem}
\Proof Per il Lemma \ref{Lemma1}, $\delta(S)$ ha per centro $S$ e, se $\mathcal{L}$ \`e un sistema lineare di centro $S$, $\delta{\mathcal{L}} = \bar{\mu}(S)$ e dunque $\mathcal{L} = (\delta^{-1} \circ \bar{\mu})(S)$. \EndProof
\begin{Definition}\label{def30}
	Il \Define{sistema lineare}[lineare][sistema] degli iperpiani di centro $S$, denotato $\Lambda_1(S)$, \`e l'unico sistema lineare avente $S$ per centro: $\Lambda_1(S) = \delta(S)$.
\end{Definition}
\begin{Theorem}\label{th29}
	Sia $\mathcal{L}$ un sistema lineare. $\mathcal{L} = \Lambda_1(S)$ se e solo se la famiglia $F = (\delta^{-1}(P))_{P \in \mathcal{L}}$ \`e costituita da tutti e soli gli iperpiani di $\mathbb{P}(V)$ che contengono $S$.
\end{Theorem}
\Proof Supponiamo $\mathcal{L}$ abbia centro $S$. Se $P \in \mathcal{L}$, allora per definizione di centro $S \subseteq \delta^{-1}(P)$; quindi tutti gli iperpiani di $F$ contengono effettivamente $S$. Supponiamo viceversa che $T$ sia un iperpiano di $\mathbb{P}(V)$ contenente $S$; allora $\delta(T)$ \`e un punto di $\mathbb{P}(V)^*$ contenuto in $\delta(S) = \mathcal{L}$ (teorema \ref{th25} e Lemma \ref{Lemma1}), dunque $T = \delta^{-1}(\delta(T)) \in F$. \EndProof
\begin{Definition}\label{def31}
	Sia $S$ un sottospazio proiettivo di $\mathbb{P}(V)$. Se $\dim S = n - 2$, allora il sistema lineare $\Lambda_1(S)$ viene chiamato \Define{fascio di iperpiani}[di iperpiani][fascio] di centro $S$.
\end{Definition}
\begin{Definition}\label{def37}
	Siano $\Lambda_1(S)$, dove $S$ \`e un sottospazio proiettivo di $\mathbb{P}(V)$, il sistema lineare degli iperpiani di centro $S$ e $L$ un iperpiano di $\mathbb{P}(V)$ e consideriamo la carta affine corrispondente data da $(U_L, j_L^{-1})$ (cfr. definizione \ref{deftop2}). La famiglia $(\delta^{-1}(P) \cap U_L)_{P \in \Lambda_1(S)}$ delle intersezioni degli iperpiani di $\Lambda_1(S)$ con la carta $U_L$, che denoteremo $\Lambda_1(S) \cap U_L$, si dice \Define{sistema lineare di iperpiani affini}[lineare di iperpiani affini][sistema]. Il sistema viene detto \Define{improprio}[lineare di iperpiani affini improprio][sistema] se $S \subseteq L$, \Define{proprio}[lineare di iperpiani affini proprio] altrimenti. Se il sistema lineare $\Lambda_1(S)$ \`e un fascio si parla allora di \Define{fascio improprio}[improprio][fascio] o \Define{fascio proprio}[proprio][fascio] rispettivamente.
\end{Definition}
\begin{Theorem}\label{th37}
	Siano $S$ un sottospazio proiettivo di $\mathbb{P}(V)$, $L$ un iperpiano di $\mathbb{P}(V)$. Se il sistema lineare $\Lambda_1(S) \cap U_L$ \`e proprio allora $L$ non \`e un iperpiano del sistema lineare $\Lambda_1(S)$.
\end{Theorem}
\Proof $\Lambda_1(S) \cap U_L$ \`e proprio se e solo se $S$ non \`e sottoinsieme di $L$ per definizione. Ma $S$ \`e per definizione l'intersezione comune di tutti gli iperpiani del sistema lineare. \EndProof
\begin{Theorem}\label{th40}
	Siano $S$ un sottospazio proiettivo di $\mathbb{P}(V)$, $L$ un iperpiano di $\mathbb{P}(V)$. Consideriamo l'applicazione $\phi: \Lambda_1(S) \rightarrow \Lambda_1(S) \cap U_L$ che al punto $P \in \Lambda_1(S)$ associa $\delta^{-1}(P) \cap U_L$. Se $\Lambda_1(S)$ \`e proprio allora $\phi$ \`e biettiva; se $\Lambda_1(S)$ \`e improprio allora $\phi(\delta(L)) = \varnothing$.
\end{Theorem}
\Proof Sia $\Lambda_1(S)$ un sistema lineare proprio. Allora, per definizione, ogni $\delta^{-1}(P)$ interseca $U_L$ e la biettivit\`a di $\phi$ segue dunque dal teorema \ref{thtop9}.
	\par Sia ora invece $\Lambda_1(S)$ improprio. Allora $S \subseteq L$ e quindi $\delta(L) \in \Lambda(S)$ e risulta $\phi(\delta(L)) = \varnothing$. \EndProof

\section{Spazi proiettivi di dimensione $1$: birapporto.}

	\par Per questa sezione, $V$ sar\`a uno spazio vettoriale di dimensione $2$, e di conseguenza $\mathbb{P}(V)$ uno spazio proiettivo di dimensione $1$. Introduciamo un simbolo $\infty$ definito dalla propriet\`a di essere uguale ad ogni rapporto $\frac{\lambda}{0}$, dove $\lambda \in \mathbb{K}^*$.
	\par Per ogni $i = 1, ..., 4$, $P_i$ sar\`a un punto di $\mathbb{P}(V)$ di coordinate omogenee $[\lambda_i, \mu_i]$ rispetto ad un dato riferimento proiettivo $\mathcal{R}$ e supporremo che almeno tre di questi quattro punti siano distinti. Per ogni $i = 1, ..., 4$ ed ogni $j = 1, ..., 4$, la matrice $M^{(i,j)}$ \`e la matrice $2 \times 2$ che ha le coordinate omogenee fissate di $P_i$ nella prima colonna e quelle di $P_j$ nella seconda colonna.
\begin{Definition}\label{def38}
	Chiamiamo \Define{birapporto} $\beta(P_1, P_2, P_3, P_4)$ il rapporto $\frac{\det(M^{(1,4)}) \cdot \det(M^{(3,2)})}{\det(M^{(1,3)}) \cdot \det(M^{(4,2)})} = \frac{(\lambda_1\mu_4 - \lambda_4\mu_1)(\lambda_3\mu_2 - \lambda_2\mu_3)}{(\lambda_1\mu_3 - \lambda_3\mu_1)(\lambda_4\mu_2 - \lambda_2\mu_4)}$.
\end{Definition}
\begin{Theorem}\label{th41}
	Siano $\mathcal{R}$ e $\mathcal{S}$ due riferimenti proiettivi di $\mathbb{P}(V)$. Siano $r_1$ e $r_2$ i birapporti della famiglia $(P_i)_{i = 1,2,3,4}$ relativamente ad $\mathcal{R}$ ed $\mathcal{S}$ rispettivamente. Abbiamo $r_1 = r_2$.
\end{Theorem}
\Proof Sia $A$ una matrice di cambio di riferimento proiettivo da $\mathcal{R}$ ad $\mathcal{S}$. Allora, per ogni $i = 1, ..., 4$, una coppia di coordinate omogenee di $P_i$ rispetto a $\mathcal{S}$ \`e $A \begin{pmatrix} \lambda_i & \mu_i \end{pmatrix}^t$.
	\par Abbiamo dunque $r_2 = \frac{\det(A) \cdot \det(M^{(1,4)}) \cdot \det(A) \cdot \det(M^{(3,2)})}{\det(A) \cdot \det(M^{(1,3)}) \cdot \det(A) \cdot \det(M^{(4,2)})} = r_1$. \EndProof
\begin{Theorem}\label{th42}
	Siano $P_1$, $P_2$ e $P_3$ distinti e supponiamo che $\mathcal{R}$ sia il riferimento proiettivo $(P_i)_{i = 1, 2, 3}$. Abbiamo $\beta(P_1, P_2, P_3, P_4) = \frac{\mu_4}{\lambda_4}$.
\end{Theorem}
\Proof Segue immediatamente dalla definizione \ref{def38}. \EndProof
\begin{Theorem}\label{th43}
	Siano $P_1$, $P_2$ e $P_3$ distinti, sia $P_4 \neq P_2$ e supponiamo $\mathcal{R} = (P_i)_{i = 1, 2, 3}$. Allora $\beta(P_1, P_2, P_3, P_4)$ \`e la coordinata affine di $P_4$ nella carta affine $\mathbb{P}(V) - \lbrace P_2 \rbrace$ rispetto al riferimento affine $\lbrace P_1, P_3 \rbrace$.
\end{Theorem}
\Proof Sia $f: \mathbb{P}(\mathbb{K}) \rightarrow \mathbb{P}(V)$ la proiettivit\`a tale che $f([1,0]) = P_1$, $f([0,1]) = P_2$, $f([1,1]) = P_3$. Consideriamo la carta affine $(U, j^{-1})$, dove $U = \mathbb{P}(V) - \lbrace P_2 \rbrace$ e $j = f \circ j_0$.
	\par La coordinata affine di $P_4$ nella carta affine $U$ rispetto al riferimento affine $\lbrace P_1, P_3 \rbrace$ \`e dunque $j^{-1}(P_4) = j_0^{-1}(f^{-1}(P_4)) = j_0^{-1}([\lambda_4, \mu_4]) = \frac{\mu_4}{\lambda_4} = \beta(P_1, P_2, P_3, P_4)$. \EndProof
\begin{Theorem}\label{th44}
	Siano $P_1$, $P_2$ e $P_3$ distinti e sia $x \in \mathbb{K} \cup \lbrace \infty \rbrace$. Esiste uno e un solo punto $Q \in \mathbb{P}(V)$ tale che $\beta(P_1, P_2, P_3, Q) = x$.
\end{Theorem}
\Proof Ragioniamo nel riferimento proiettivo $\mathcal{R} = (P_i)_{i = 1,2,3}$. Sia $Q$ di coordinate $[\xi, \eta]$.
	\par Abbiamo $\beta(P_1, P_2, P_3, Q) = x$ se e solo se $x = \frac{\eta}{\xi}$. Per l'esistenza, basta scegliere $\xi = 1$ e $\eta = \xi x = x$ quando $x \in \mathbb{K}$ e scegliere $\xi = 0$, $\eta = 1$ quando invece $x = \infty$.
	\par Per l'unicit\`a, sia $Q'$ di coordinate $[\alpha, \beta]$ che verifica l'equazione $\beta(P_1, P_2, P_3, Q') = x$. Se $x \in \mathbb{K} - \lbrace 0 \rbrace$, allora abbiamo $\frac{\eta}{\xi} = \frac{\beta}{\alpha}$, equivalente a $\frac{\eta}{\beta} = \frac{\xi}{\alpha}$, da cui $Q = Q'$; se invece $x = 0$ o $x = \infty$, allora necessariamente abbiamo $\eta = \beta = 0$ nel primo caso e $\xi = \alpha = 0$ nel seconodo, e dunque di nuovo $Q = Q'$. \EndProof
\begin{Theorem}\label{th45}
	Siano $\mathbb{P}(V)$ e $\mathbb{P}(W)$ rette proiettive. Siano $(P_i)_{i = 1,2,3,4}$ e $(Q_i)_{i = 1,2,3,4}$ due quaterne di punti tali che
	\begin{itemize}
		\item $P_1, P_2, P_3, P_4 \in \mathbb{P}(V)$;\\
		\item $Q_1, Q_2, Q_3, Q_4 \in \mathbb{P}(V)$;\\
		\item $P_1$, $P_2$ e $P_3$ siano distinti;\\
		\item $Q_1$, $Q_2$ e $Q_3$ siano distinti.
	\end{itemize}
	Esiste un isomorfismo proiettivo $f: \mathbb{P}(V) \rightarrow \mathbb{P}(W)$ tale che, per ogni $i = 1, ..., 4$, $f(P_i) = Q_i$ se e solo se $\beta(P_1,P_2,P_3,P_4) = \beta(Q_1, Q_2, Q_3, Q_4)$.
\end{Theorem}
\Proof Consideriamo i riferimenti proiettivi $\mathcal{R} = (P_i)_{i = 1,2,3}$ e $\mathcal{S} = (Q_i)_{i = 1,2,3}$; siano $\phi: \mathbb{P}(V) \rightarrow \mathbb{P}(\mathbb{K})$ e $\psi: \mathbb{P}(W) \rightarrow \mathbb{P}(\mathbb{K})$ gli isomorfismi coordinati.
	\par Se esiste l'isomorfismo proiettivo $f$, sia $A$ una matrice associata ad $f$ rispetto ai riferimenti proiettivi $\mathcal{R}$ ed $\mathcal{S}$. Allora la stessa dimostrazione del teorema \ref{th41} prova l'uguaglianza dei due birapporti.
	\par Supponiamo adesso che i due birapporti siano uguali. Esiste un unico isomorfismo proiettivo $f: \mathbb{P}(V) \rightarrow \mathbb{P}(W)$ che trasforma $\mathcal{R}$ in $\mathcal{S}$ (teorema fondamentale delle applicazioni proiettive): occorre dunque verificare che $f(P_4) = Q_4$.
	\par Per l'implicazione gi\`a dimostrata, $\beta(Q_1, Q_2, Q_3, f(P_4)) = \beta(P_1, P_2, P_3, P_4)$. D'altra parte, $\beta(Q_1,Q_2,Q_3,Q_4) = \beta(P_1,P_2,P_3,P_4)$ per ipotesi; dunque, per il teorema \ref{th44}, $f(P_4) = Q_4$. \EndProof

\section{Quadriche.}
	\par In questa sezione identificheremo ogni matrice con l'applicazione lineare da essa definita rispetto alle basi canoniche: scriveremo in questo senso di nuclei e immagini di matrici.
\subsection{Classificazione affine e metrica.}

\begin{Definition}\label{def39}
	Una \Define{quadrica affine}[affine][quadrica] (o anche pi\`u brevemente \NIDefine{quadrica}) in uno spazio affine $\mathbb{K}^n$ \`e il luogo degli zeri di un polinomio di secondo grado. Nel caso in cui $n = 2$ la quadrica affine viene anche detta \Define{conica affine}[affine][conica].
\end{Definition}
\begin{Theorem}\label{th46}
	Se $\mathbb{K}$ \`e il campo $\mathbb{R}$ o $\mathbb{C}$ munito della topologia euclidea, allora una quadrica \`e un sottospazio chiuso di $\mathbb{K}^n$.
\end{Theorem}
\Proof Segue direttamente dal fatto che i polinomi sono continui e che $0$ \`e chiuso in $\mathbb{K}$. \EndProof
\par Osserviamo inoltre che tutte le quadriche sono chiuse per definizione in $\mathbb{K}^n$ munito della topologia di Zariski.
\begin{Definition}\label{def40}
	Due quadriche affini $\mathcal{Q}$ e $\mathcal{Q}'$ di $\mathbb{K}^n$ si dicono \Define{affinemente equivalenti}[affinemente equivalenti][quadriche] quando esiste un'affinit\`a $f: \mathbb{K}^n \rightarrow \mathbb{K}^n$ tale che $f(\mathcal{Q}) = \mathcal{Q}'$.
\end{Definition}
\begin{Definition}\label{def41}
	Due quadriche $\mathcal{Q}$ e $\mathcal{Q}'$ di $\mathbb{K}^n$ si dicono \Define{metricamente equivalenti}[metricamente equivalenti][quadriche] quando esiste un'isometria $f: \mathbb{K}^n \rightarrow \mathbb{K}^n$ tale che $f(\mathcal{Q}) = \mathcal{Q}'$.
\end{Definition}
	\par Nel seguito, supponiamo dato un polinomio di secondo grado $p \in \mathbb{K}[x_1, ..., x_n]$. Scriveremo $p$ nella forma $p = X^tAX + 2 B^t X + C$, dove $X = \begin{pmatrix} x_1 \cdots x_n \end{pmatrix}^t$, $A$ \`e una matrice simmetrica di dimensione $n \times n$\footnote{$A$ \`e la matrice tale che il suo elemento di posto $(i,j)$, con $i, j = 1, ..., n$, sia \begin{itemize} \item la met\`a del coefficiente di $x_ix_j$ se $i \neq j$; \item il coefficiente di $x_i^2$ se $i = j$. \end{itemize}}, $B$ una matrice di dimensione $n \times 1$ e $C \in \mathbb{K}$. La quadrica associata a $p$ sar\`a denotata $\mathcal{Q}$.
\begin{Definition}\label{def42}
	Data la quadrica $\mathcal{Q}$, se esiste un punto $N$ tale che, detta $f$ la simmetria centrale di centro $N$, $\mathcal{Q}$ \`e invariante rispetto ad $f$, allora $\mathcal{Q}$ si dice una \Define{quadrica a centro}[a centro][quadrica]. Il punto $N$ viene detto \Define{centro di simmetria}[di simmetria di una quadrica affine][centro] per $\mathcal{Q}$.
\end{Definition}
\begin{Theorem}\label{th47}
	Sia $\mathcal{Q}$ una quadrica. L'origine $O$ \`e un centro di simmetria per $\mathcal{Q}$ se e solo se $B = 0$.
\end{Theorem}
\Proof Se $B = 0$, allora \`e immediato che $\mathcal{Q}$ \`e invariante per la simmetria centrale di centro $O$ - i.e. l'applicazione che a $X \in \mathbb{K}^n$ associa $-X$.
	\par Viceversa, se $O$ \`e un centro di simmetria per $\mathcal{Q}$, allora, per ogni $X \in \mathbb{K}^n$, $X^tAX + 2 B^t X + C = 0$ se e solo se $(-X)^tA(-X) + 2 B^t (-X) + C = 0$, la qual cosa implica, sottraendo membro a membro la seconda equazione dalla prima, $4 B^t X = 0$, da cui $B = 0$ (esiste necessariamente un $X \neq 0$ che verifica $X^tAX + 2B^tX + C = 0$: il primo membro è un polinomio di secondo grado in $x_1$ dunque, fissato il valore delle altre incognite, esiste sempre almeno un valore assegnabile a $x_1$ affinch\'e $X$ verifichi l'equazione). \EndProof
\begin{Theorem}\label{th48}
	Il punto $N$ \`e un centro per la quadrica $\mathcal{Q}$ se e sole se $AN = -B$. In particolare, $\mathcal{Q}$ \`e a centro se e solo se $A$ e $C = \left ( \begin{array}{c|c} B & A \end{array} \right )$ hanno lo stesso rango.
\end{Theorem}
\Proof La quadrica $\mathcal{Q}$ ha il punto $N$ per centro di simmetria se e solo se la quadrica $\mathcal{Q}'$ luogo degli zeri di $p' = (X + N)^tA(X + N) + 2 B^t(X + N) + C$ (ottenuta da $\mathcal{Q}$ tramite la traslazione che trasforma $N$ in $0$) ha per centro di simmetria l'origine $O$.
	\par Abbiamo $p' = X^tAX + N^tAX + X^tAN + N^tAN + 2B^tX + 2B^tN + C$. Per il teorema \ref{th47}, $N$ \`e dunque centro di simmetria per $\mathcal{Q}$ se e solo se $2B^t + 2N^tA = 0$, equivalente a $AN = -B$. La tesi segue dunque dal teorema di Rouch\'e-Capelli. \EndProof
\begin{Corollary}\label{cor5}
	La propriet\`a di una quadrica di essere a centro o meno \`e invariante rispetto al gruppo delle isometrie o delle affinit\`a.
\end{Corollary}
\Proof Basta provare il corollario nel caso delle affinit\`a. Sia dunque $f(X) = MX + T$ un'affinit\`a. Allora $f(\mathcal{Q})$ sar\`a la quadrica associata al polinomio $p' = (X - T)^tM^{-1, t}AM^{-1}(X - T) + 2B^tM^{-1}(X - T) + C = X^tM^{-1,t}AM^{-1}X - 2 T^tM^{-1,t}AM^{-1}X + 2B^tM^{-1}X + T^tM^{-1}AM^{-1}T - 2B^tM^{-1}T + C$. Ora, per il teorema \ref{th48}, $\mathcal{Q}$ \`e a centro se e solo se $\left ( \begin{array}{c|c} B & A \end{array} \right )$ ha lo stesso rango di $A$ e $\mathcal{Q}'$ \`e a centro se e solo se $\left ( \begin{array}{c|c} 2M^{-1,t}B - M^{-1,t}AM^{-1}T & M^{-1,t}AM^{-1} \end{array} \right )$ ha lo stesso rango di $M^{-1,t}AM^{-1}$.
	\par Assumiamo vera la prima condizione sul rango. Essa equivale all'esistenza di $Y \in \mathbb{K}^n$ tale che $B = AY$ e quindi, posto $Z = MY$, avendo allora $2M^{-1,t}B - M^{-1,t}AM^{-1}T = 2M^{-1,t}AM^{-1}Z - M^{-1,t}AM^{-1}T = M^{-1,t}AM^{-1}(2Z - T)$ essa implica la seconda condizione sul rango.
	\par Assumiamo infine vera la seconda condizione sul rango. Analogamente, essa equivale all'esistenza d'un $Y \in \mathbb{K}^n$ tale che $2M^{-1,t}B - M^{-1,t}AM^{-1}T = M^{-1,t}AM^{-1}Y$. Abbiamo allora $B = 2^{-1}M^t \cdot M^{-1,t}AM^{-1}(Y + T) = A(2^{-1}M^{-1}(Y + T))$ e questo conclude la dimostrazione. \EndProof
\begin{Theorem}\label{th49}
	Abbiamo $\mathbb{K}^n \cong \Kernel A \oplus \Image A$.
\end{Theorem}
\Proof Si considera la successione esatta $0 \rightarrow \Kernel A \rightarrow \mathbb{K}^n \rightarrow \Image A \rightarrow 0$. Considerando una base del nucleo di $A$ e completandola a base di $\mathbb{K}^n$ si costruisce facilmente una retrazione lineare dell'inclusione, da cui la tesi. \EndProof
\begin{Corollary}\label{cor6}
	Esiste una matrice non singolare $M$ tale che $M^{-1}AM = \left ( \begin{array}{c|c} \bar{A} & 0\\ \hline 0 & 0 \end{array} \right )$, dove $\bar{A}$ \`e una matrice di dimensione $k \times k$, con $k$ uguale al rango di $A$.
\end{Corollary}
\Proof \`E sufficiente scegliere per $M$ la matrice di cambio di base dalla base canonica ad una base i cui primi $k$ elementi siano una base di $\Image A$ e i restanti elementi una base di $\Kernel A$. \EndProof
	\par Nel seguito $M$ e $\bar{A}$ denoteranno sempre matrici con le propriet\`a del corollario precedente. Inoltre, per il resto di questa sottosezione supporremo $\mathbb{K} = \mathbb{R}$.
\begin{Definition}\label{def43}
	Due matrici $M^{(1)}$ e $M^{(2)}$ di dimensione $n \times n$ si dicono
		\begin{itemize}
			\item \Define{congruenti}[congruenti][matrici] se esiste una matrice non singolare $S$ tale che $M^{(1)} = S^tM^{(2)}S$;
			\item \Define{simili}[simili][matrici] se esiste una matrice non singolare $S$ tale che $M^{(1)} = S^{-1}M^{(2)}S$.
		\end{itemize}
\end{Definition}
\begin{Theorem}\label{th50}
	(\TheoremName{Teorema di Sylvester.}) Ogni matrice simmetrica \`e congruente a una matrice diagonale i cui elementi non nulli sono solo $-1$ e $1$ e ogni matrice diagonale che verifica tale propriet\`a ha lo stesso numero di elementi uguali a $1$ e lo stesso numero di elementi uguali a $-1$.
\end{Theorem}
\Proof Omessa. \EndProof
\begin{Definition}\label{def44}
	Sia $M^{(0)}$ una matrice simmetrica e sia $D$ una matrice a cui $M^{(0)}$ \`e congruente e del tipo descritto dal teorema precedente. Il numero di elementi uguali a $1$, a $-1$, a $0$ sulla diagonale di $D$ si chiama, rispettivamente, \Define{indice di positivit\`a}[di positivit\`a di una matrice simmetrica][indice], \Define{indice di negativit\`a}[di negativit\`a di una matrice simmetrica][indice] e \Define{indice di nullit\`a}[di nullit\`a di una matrice simmetrica][indice] di $M^{(0)}$. La terna $(i_+, i_-, i_0)$, le cui componenti sono i tre indici appena definiti, \`e detta \Define{segnatura}[di una matrice simmetrica][segnatura].
\end{Definition}
	\par Nel seguito, $i_+$, $i_-$, $i_0$ denoteranno gli indici di positivit\`a, negativit\`a e nullit\`a di $A$; $D$ sar\`a la matrice diagonale i cui elementi sulla diagonale, nell'ordine, sono $i_+$ volte $1$, $i_-$ volte $-1$ e $i_0$ volte $0$; $S$ sar\`a la matrice non singolare tale che $S^{t}AS = D$.
\begin{Theorem}\label{th51}
	Abbiamo $i_0 = \dim \Kernel A$ e $i_+ + i_- = \dim \Image A$.
\end{Theorem}
\Proof Omessa. \EndProof
\begin{Theorem}\label{th51bis}
	$i_+$, $i_-$ e $i_0$ sono rispettivamente uguali al numero di autovalori positivi, negativi e nulli di $A$.
\end{Theorem}
\Proof Omessa.\EndProof
\begin{Theorem}\label{th51tris}
	(\TheoremName{Criterio di Cartesio per polinomi con tutte le radici reali.}) Il numero di variazioni di segno nei termini di un polinomio ordinato \`e uguale al numero delle sue radici positive.
\end{Theorem}
\Proof Omessa.\EndProof
\begin{Definition}\label{def45}
	Una matrice $U$ si dice \Define{ortogonale}[ortogonale][matrice] quando $U^t = U^{-1}$.
\end{Definition}
	\par Notiamo che la matrice $M$ gi\`a definita pu\`o essere scelta ortogonale (basta scegliere nella dimostrazione del teorema che garantisce l'esistenza di $M$ e $\bar{A}$ una base di arrivo ortonormale, cosa possibile in quanto $\Kernel A$ e $\Image A$ sono spazi ortogonali): assumeremo di aver compiuto questa scelta.
\begin{Theorem}\label{th52}
	(\TheoremName{Teorema spettrale in dimensione finita}.) Ogni matrice simmetrica \`e simile a una matrice diagonale per mezzo di matrici ortogonali. Pi\`u precisamente, se $M^{(0)}$ \`e una matrice simmetrica esistono una matrice ortogonale $U$ e una matrice diagonale $D^{(0)}$ tali che $M^{(0)} = U^{-1}D^{(0)}U$.
\end{Theorem}
\Proof Omessa. \EndProof
\begin{Theorem}\label{th55}
	(\TheoremName{Classificazione affine delle quadriche reali.}) $\mathcal{Q}$ \`e affinemente equivalente ad una e una sola delle seguenti quadriche:
	\begin{itemize}
		\item $\sum_{i = 1}^{i_+} x_i^2 - \sum_{i = i_+ + 1}^{i_+ + i_-} x_i^2 + c = 0$, dove $c \in \lbrace 0, 1 \rbrace$ (quadriche a centro);
		\item $\sum_{i = 1}^{i_+} x_i^2 - \sum_{i = i_+ + 1}^{i_+ + i_-} x_i^2 + \sum_{i = i_+ + i_- + 1}^n x_i = 0$ (quadriche non a centro).
	\end{itemize}
\end{Theorem}
\Proof Sia $\mathcal{Q}$ una quadrica a centro e sia $N$ un suo centro. Considerando la traslazione che porta $N$ in $O$, il trasformato $\mathcal{Q}'$ di $\mathcal{Q}$ \`e il luogo degli zeri di $p' = X^tAX + (2B^t + 2N^tA)X + 2B^tN + C + N^tAN = X^tAX + 2B^tN + C + N^tAN$. Poniamo $C' = 2B^tN + C + N^tAN = B^tN + C$, cosicch\'e $p' = X^tAX + C'$. Eseguiamo l'ulteriore trasformazione $X \mapsto S^{-1}X$, cosicch\'e $\mathcal{Q}'$ viene trasformata in $\mathcal{Q}''$ definita dall'equazione $X^tS^tASX + C = 0$. Se $C' = 0$ abbiamo finito, altrimenti possiamo assumere $C' > 0$. Un'omotetia $X \mapsto \sqrt{C'^{-1}}X$ e una moltiplicazione di ambi i membri per $C'^{-1}$ trasforma $\mathcal{Q}''$ in $\mathcal{Q}'''$ definita da $X^tDX + 1 = 0$.
	\par Sia ora $\mathcal{Q}$ non a centro. Effettuiamo la trasformazione $X \mapsto M^{-1}X$: $\mathcal{Q}$ viene trasformata in $\mathcal{Q}'$ definita da $X^tM^tAMX + 2 B^tMX + C = 0$. Definiamo $B_1$ come il vettore delle prime $k$ coordinate di $B^tM$, dove $k$ \`e il rango di $A$, e $B_2$ quello delle restanti; definiamo allo stesso modo $X_1$ e $X_2$ a partire da $X$. Allora l'equazione di $\mathcal{Q}'$ diventa $X_1^t\bar{A}X_1 + 2B_1^tX_1 + 2B_2^tX_2 + C = 0$.
	\par Osserviamo che $\left ( \begin{array}{c|c} B_1 & \bar{A} \end{array} \right )$ e $\bar{A}$ hanno lo stesso rango perch\'e $\bar{A}$ \`e invertibile. Dunque $\mathcal{Q}_0'$ di equazione $X_1^t\bar{A}X_1 + 2B_1^tX_1 = 0$ \`e una quadrica che ha per unico centro il punto $N_1 = - \bar{A}^{-1}B_1$. Applicando quanto gi\`a dimostrato a quest'ultima, troviamo che un'opportuna isometria $X \rightarrow U^{-1}X$ che lascia fisso puntualmente tutto il sottospazio generato dagli ultimi $n - k$ vettori della base canonica trasforma $\mathcal{Q}'$ in $\mathcal{Q}''$ definita da $X_1^t\bar{U}^t \left ( \begin{array}{c|c} I_{i_+} & 0\\ \hline 0 & I_{i_-} \end{array} \right )\bar{U}X_1 + 2B_2^tX_2 - B_1^t \bar{A}^{-1} B_1 + C = 0$, dove $\bar{U}$ \`e il minore principale di $U$ di ordine $k$. La trasformazione lineare $X \mapsto T^{-1}X$, dove $T = \left ( \begin{array}{c|c} I_{k} & 0\\ \hline 0 & \tilde{T}\end{array} \right )$, con $\tilde{T}$ tale che $\tilde{T}^t B_2 = ( \cdots 1 \cdots)^t$, porta $\mathcal{Q}''$ in $\mathcal{Q}'''$ di equazione $X_1^t\bar{U}^t \left ( \begin{array}{c|c} I_{i_+} & 0\\ \hline 0 & I_{i_-} \end{array} \right ) \bar{U}X_1 + (\cdots 1 \cdots) - B_1^t \bar{A}^{-1} B_1 + C = 0$. Infine, una traslazione $X \mapsto X + v$, dove le prime $k$ coordinate di $v$ sono nulle e le restanti sono tutte uguali a $(n - k)^{-1}(B_1^t \bar{A}^{-1} B_1 - C)$ trasforma $\mathcal{Q}'''$ in $\mathcal{Q}^{(IV)}$ descritta da un'equazione del secondo tipo dell'enunciato.
	\par L'unicit\`a della forma canonica segue dall'invarianza della segnatura di $A$ rispetto al gruppo delle affinit\`a.\EndProof
\begin{Corollary}\label{cor7}
	Ogni quadrica a centro ammette un solo centro.
\end{Corollary}
\Proof Immediato per il teorema appena dimostrato e per il corollario \ref{cor6}. \EndProof
\begin{Theorem}\label{th55}
	(\TheoremName{Classificazione metrica delle quadriche reali.}) $\mathcal{Q}$ \`e metricamente equivalente ad una delle seguenti quadriche (identificate con una delle equazioni che le descrivono):
	\begin{itemize}
		\item $\sum_{i = 1}^n a_ix_i^2 + c = 0$, dove gli $a_i$ sono numeri reali e $c \in \lbrace 0, 1 \rbrace$ (quadriche a centro);
		\item $\sum_{i = 1}^{r} a_ix_i^2 + \sum_{i = r}^n b_ix_i = 0$, dove $r$ \`e il rango di $A$ e gli $a_i$ e i $b_i$ sono numeri reali (quadriche non a centro).
	\end{itemize}
\end{Theorem}
\Proof Analoga alla dimostrazione della classificazione affine, salvo che non si fa uso delle trasformazioni che non sono isometrie e al posto del teorema di Sylvester si usa il teorema spettrale (per cui gli $a_i$ sono tutti gli autovalori non nulli di $A$). L'unicit\`a segue stavolta dall'invarianza degli autovalori di $A$ rispetto al gruppo ortogonale. \EndProof

\subsection{Classificazione proiettiva.}

\begin{Definition}\label{def46}
	Un polinomio $p \in \mathbb{K}[x_0, ..., x_n]$ si dice \Define{omogeneo}[omogeneo][polinomio] di grado $k$ se ogni suo termine ha lo stesso grado $k$. Una \Define{forma quadratica}[quadratica][forma] $F$ \`e un polinomio omogeneo di grado $2$.
\end{Definition}
	\par Nel seguito, $F$ denoter\`a sempre una forma quadratica assegnata, $X$ il vettore $(x_0, ..., x_n)$ e $A$ sar\`a la matrice simmetrica (necessariamente unica) tale che $F(X) = X^tAX$.
\begin{Theorem}\label{th56}
	Sia $p \in \mathbb{K}[x_0, ..., x_n]$ un polinomio omogeneo di grado $k$ e sia $X \in \mathbb{K}^{n + 1}$ tale che $p(X) = 0$. Allora, per ogni $\lambda \in \mathbb{K}^*$, $p(\lambda X) = 0$.
\end{Theorem}
\Proof Abbiamo, per ogni $Y \in \mathbb{K}^{n + 1}$, $p(\lambda Y) = \lambda^k p(Y) = 0$. \EndProof
	\par Osserviamo che, dato un polinomio omogeneo $p$ di grado $k > 1$, l'applicazione di sostituzione $\phi: \mathbb{K}^{n + 1} \rightarrow \mathbb{K}$ che a $X \in \mathbb{K}^{n + 1}$ associa $p(X)$ non \`e compatibile con la relazione d'equivalenza $\sim$ che definisce $\mathbb{P}^n(\mathbb{K})$; tuttavia, il luogo degli zeri di $\phi$ \`e saturo per $\sim$ e possiamo dunque parlare di ``luogo degli zeri di $p$'' in $\mathbb{P}^n(\mathbb{K})$, intendendo con quest'espressione il proiettivizzato del luogo degli zeri di $p$ in $\mathbb{K}^{n + 1}$.
\begin{Definition}\label{def47}
	Una \Define{quadrica proiettiva}[proiettiva][quadrica] di $\mathbb{P}^n(\mathbb{K})$ \`e il luogo degli zeri di una qualche forma quadratica $F$. Una \Define{conica proiettiva}[proiettiva][conica] \`e una quadrica proiettiva di $\mathbb{P}^2(\mathbb{K})$.
\end{Definition}
	\par Nel seguito, $\mathcal{Q}$ denoter\`a la quadrica proiettiva descritta da $F(X) = 0$.
\begin{Definition}\label{def48}
	Due quadriche proiettive $\mathcal{Q}$ e $\mathcal{Q}'$ di $\mathbb{P}^n(\mathbb{K})$ si dicono \Define{proiettivamente equivalenti}[proiettivamente equivalenti][quadriche] quando esiste una proiettivit\`a $f: \mathbb{P}^n(\mathbb{K}) \rightarrow \mathbb{P}^n(\mathbb{K})$ tale che $f(\mathcal{Q}) = \mathcal{Q}'$.
\end{Definition}
\begin{Definition}\label{def50}
	$\mathcal{Q}$ si dice \Define{degenere}[degenere][quadrica] quando $A$ \`e singolare.
\end{Definition}
\begin{Definition}\label{def51}
	Sia $P \in \mathbb{P}^n(\mathbb{K})$. Si chiama \Define{spazio polare}[polare][spazio] di $P$ rispetto a $\mathcal{Q}$, denotato $\PolarSpace_\mathcal{Q}(P)$, il sottospazio di equazione $P^tAX = 0$\footnote{Si verifica facilmente che se la stessa quadrica \`e definita tramite due forme quadratiche diverse, e dunque da due matrici diverse, allora lo spazio polare associato ad uno stesso punto \`e lo stesso - i.e. che la definizione \`e ben posta.}. $P$ si dice \Define{singolare}[singolare rispetto a una quadrica][punto] rispetto a $\mathcal{Q}$ se $P^tA = AP = 0$ - i.e. se $\PolarSpace_\mathcal{Q}(P) = \mathbb{P}^n(\mathbb{K})$.
\end{Definition}
	\par Nel seguito, considereremo solo sottospazi polari e singolarit\`a relativi a $\mathcal{Q}$ senza ulteriori specificazioni.
\begin{Theorem}\label{th56bis}
	L'insieme dei punti singolari di $\mathcal{Q}$ \`e il sottospazio proiettivo $\mathbb{P}(\Kernel A)$ e ha dimensione non maggiore di $n - 1$.
\end{Theorem}
\Proof Per definizione di punto singolare, tale insieme non \`e altro che $\mathbb{P}(\Kernel A)$ ed \`e dunque un sottospazio proiettivo di dimensione $\dim (\Kernel A) - 1$. Affinch\'e $F$ sia una forma quadratica \`e necessario che la matrice $A$ abbia sulla sua diagonale almeno un elemento non nullo, per cui $\Rank A \geq 1$ e $\dim \Kernel A = n + 1 - \Rank A \leq n$. \EndProof
\begin{Theorem}\label{th57}
	Se $P \in \mathbb{P}^n(\mathbb{K})$ \`e non singolare, allora $\PolarSpace(P)$ \`e un iperpiano. Se invece $P$ \`e singolare allora $P \in \mathcal{Q}$.
\end{Theorem}
\Proof $\dim \PolarSpace(P) = n - 1$ equivale a $\Rank P^tA = 1$ e questo avviene se e solo se $P \notin \Kernel A$, equivalente alla condizione che $P$ sia non singolare.
\par La seconda affermazione segue direttamente dalla definizione di punto singolare. \EndProof
\begin{Theorem}\label{th58}
	$P \in \mathbb{P}^n(\mathbb{K})$ appartiene a $\mathcal{Q}$ se e solo se $P \in \PolarSpace(P)$.
\end{Theorem}
\Proof Segue immediatamente dalla definizione. \EndProof
\begin{Theorem}\label{th59}
	Siano $P_1, P_2 \in \mathbb{P}^n(\mathbb{K})$. Allora $P_1 \in \PolarSpace(P_2)$ se e solo se $P_2 \in \PolarSpace(P_1)$.
\end{Theorem}
\Proof Sia $P_1 \in \PolarSpace(P_2)$. Allora $P_1$ soddisfa l'equazione $P_2^tAP_1 = 0$, equivalente a $P_1^tAP_2 = 0$, cio\`e a $P_2 \in \PolarSpace(P_1)$.\EndProof
\begin{Theorem}\label{th57}
	Sia $\PolarSpace: \mathbb{P}^n(\mathbb{K}) - \mathbb{P}(\Kernel A) \rightarrow \mathbb{P}^n(\mathbb{K})^*$, l'applicazione che ad ogni punto non singolare $P \in \mathbb{P}^n(\mathbb{K})$ associa il funzionale $X \mapsto P^tAX$  (a cui corrisponde un iperpiano di $\mathbb{P}^n(\mathbb{K})$ per la corrispondenza di dualit\`a) \`e un'applicazione proiettiva degenere. In particolare, se $\mathcal{Q}$ \`e non degenere, allora $\PolarSpace$ \`e un isomorfismo proiettivo.
\end{Theorem}
\Proof L'applicazione $\phi: \mathbb{K}^{n + 1} \rightarrow \mathbb{K}^{n+1,*}$ che a $P \in \mathbb{K}^{n + 1}$ associa il funzionale che a $X \mapsto P^tAX$ \`e un'applicazione lineare di nucleo $\Kernel A$. Il teorema segue dall'osservazione che $\PolarSpace$ \`e l'applicazione proiettiva degenere dedotta da $\phi$. \EndProof
\begin{Definition}\label{def52}
	Sia $\mathcal{Q}$ non degenere e sia $\pi$ un iperpiano di $\mathbb{P}^n(\mathbb{K})$. Il punto $P = \PolarSpace^{-1}(\delta(\pi))$, dove $\PolarSpace$ \`e la funzione del teorema precedente, si chiama \Define{polo} di $\pi$ rispetto a $\mathcal{Q}$.
\end{Definition}
	\par Nel seguito, considereremo solo poli relativi a $\mathcal{Q}$ senza ulteriori specificazioni.
\begin{Definition}\label{def53}
	Siano $P_1, P_2 \in \mathbb{P}^n(\mathbb{K})$. $P_1$ e $P_2$ si dicono \Define{coniugati}[coniugati rispetto a una quadrica][punti] rispetto a $\mathcal{Q}$ se e solo se $P_1^tAP_2 = 0$. Una famiglia $(P_i)_{i = 0, ..., n}$ di punti di $\mathbb{P}^n(\mathbb{K})$ determina i vertici di un \Define{simplesso autopolare}[autopolare][simplesso] quando i suoi elementi sono linearmente indipendenti e per ogni $(i, j) \in \lbrace 0, ..., n \rbrace^2$ si ha che $P_i$ e $P_j$ sono coniugati.
\end{Definition}
	\par Per $n = 2$ si parla di \Define{triangolo autopolare}[autopolare][triangolo], per $n = 3$ di \Define{tetraedro autopolare}[autopolare][tetraedro].
\begin{Theorem}\label{th61}
	Sia $(P_i)_{i = 0, ..., n}$ una famiglia di punti di $\mathbb{P}^n(\mathbb{K})$, vertici di un simplesso autopolare. Per ogni $i = 0, ..., n$, abbiamo che se $P_i$ \`e non singolare, allora $\PolarSpace(P_i) = L((P_j)_{j \in \lbrace 0, ..., n \rbrace - \lbrace i \rbrace})$.
\end{Theorem}
\Proof $P_i$ \`e non singolare, dunque $\PolarSpace(P_i)$ \`e un iperpiano, univocamente determinato da una famiglia di $n$ punti indipendenti che gli appartengono: basta dunque provare che per ogni $j \in \lbrace 0, ..., n \rbrace - \lbrace i \rbrace$ abbiamo $P_j \in \PolarSpace(P_i)$, ma questo segue direttamente dalla definizione di simplesso autopolare. \EndProof
\begin{Theorem}\label{th62}
	Per $n \geq 1$, esiste un simplesso autopolare.
\end{Theorem}
\Proof Procediamo per induzione su $n$.
	\par Se $n = 1$. Sia $P_0 \in \mathbb{P}^1(\mathbb{K})$ tale che $P_0 \notin \mathcal{Q}$\footnote{Esiste certamente un punto con questa propriet\`a: basta scegliere un punto opportuno del riferimento proiettivo canonico.}. Necessariamente, $P_0$ \`e non singolare e il suo iperpiano polare \`e un punto $P_1 \neq P_0$: $P_0$ e $P_1$ sono punti linearmente indipendenti e coniugati, quindi vertici di un simplesso autopolare.
	\par Supponiamo il teorema vero per $n = k$ ($k \in \mathbb{N} - \lbrace 0 \rbrace$) e proviamolo per $n = k + 1$. Sia $P_0 \in \mathbb{P}^{k + 1}(\mathbb{K})$ tale che $P_0 \notin \mathcal{Q}$. Necessariamente, $P_0$ \`e non singolare. La traccia di $\mathcal{Q}$ su $\PolarSpace(P_0)$ \`e una quadrica in un sottospazio proiettivo isomorfo a $\mathbb{P}^k(\mathbb{K})$: esiste dunque una famiglia di punti $(P_i)_{i = 1, ..., n}$ tale che i $P_i$ sono tutti indipendenti, coniugati e appartenenti a $\PolarSpace(P_0)$. Quindi la famiglia $(P_i)_{i = 0, ..., n}$ determina un simplesso autopolare. \EndProof
\begin{Theorem}\label{th63}
	Se $n \geq 0$, allora $\mathcal{Q}$ \`e proiettivamente equivalente ad una quadrica proiettiva $\mathcal{Q}'$ descritta da $X^tDX$, dove $D$ \`e una matrice diagonale.
\end{Theorem}
\Proof La tesi \`e ovvia per $n = 0$. Supponiamo $n \geq 1$.
	\par Sia $\mathcal{R} = (P_i)_{i = 0, ..., n + 1}$ un riferimento proiettivo i cui punti fondamentali costituiscono i vertici di un simplesso autopolare e sia $M$ la matrice di cambio di riferimento proiettivo dal riferimento canonico a $\mathcal{R}$. Allora $\mathcal{Q}$ \`e proiettivamente equivalente a $\mathcal{Q}'$ di equazione $X^tM^tAMX = 0$ e $D = M^tAM$ \`e una matrice simmetrica avente nel posto $(i,j)$ ($i,j \in \lbrace 1, ..., n + 2 \rbrace$) l'elemento $P_{i - 1}^tAP_{j - 1}$, il quale \`e nullo quando $i \neq j$\footnote{La condizione \`e sufficiente, ma non necessaria: $P_i^tAP_i$ \`e nullo se $P_i$ \`e singolare.}. \EndProof
\begin{Theorem}\label{th64}
	\TheoremName{(Classificazione proiettiva delle quadriche complesse.)} Sia $\mathbb{K} = \mathbb{C}$. $\mathcal{Q}$ \`e proiettivamente equivalente ad una e una sola quadrica della famiglia di quadriche $X^tD_kX = 0$, dove $D_k = \left ( \begin{array}{c|c} \Identity_k & 0\\\hline 0 & 0 \end{array} \right )$, con $k \in [1, n] \cap \mathbb{N}$.
\end{Theorem}
\Proof Riprendiamo le notazioni del teorema \ref{th63} e della sua dimostrazione. Gli elementi diagonali di $D$ sono dati da $P_{i - 1}^tAP_{i - 1}$, per $i = 1, ..., n + 2$: denotiamoli $\lambda_i$. A meno di permutazione degli indici $i$ (corrispondente ad una trasformazione proiettiva), possiamo assumere che i $\lambda_i$ non nulli siano tutti e soli quelli con $i \leq k$ ($k < n + 1$). Abbiamo dunque, per $i = 0, ..., k$, $1 = \lambda_i^{-1} P_i^tAP_i = (\lambda_i^{-\frac{1}{2}}P_i)^tA(\lambda_i^{-\frac{1}{2}} P_i)$, da cui, dato che $\lambda_i P_i$ e $P_i$ sono rappresentanti dello stesso punto in $\mathbb{P}^n(\mathbb{C})$, $D$ pu\`o essere scelta della forma enunciata dal teorema.
	\par L'unicit\`a della forma canonica cos\`i individuata segue dal fatto che il numero $k$ \`e necessariamente uguale al rango di $A$, invariante per trasformazione proiettiva. \EndProof
\begin{Theorem}\label{th65}
	\TheoremName{(Classificazione proiettiva delle quadriche reali.)} Sia $\mathbb{K} = \mathbb{R}$. $\mathcal{Q}$ \`e proiettivamente equivalente ad una e una sola quadrica della famiglia di quadriche $X^tD_{k_1, k_2}X = 0$, dove $D_{k_1, k_2} = \left ( \begin{array}{c|c|c} \Identity_{k_1} & 0 & 0\\\hline 0 & - Id_{k_2} & 0\\\hline 0& 0& 0\end{array} \right )$, con $k_1, k_2 \in [0, n + 1] \cap \mathbb{N}$ e $k_1 + k_2 \leq n + 1$.
\end{Theorem}
\Proof Analoga a quella del teorema precedente, permutando stavolta i vertici del simplesso autopolare in modo tale che sulla diagonale di $D$ appaiano prima gli autovalori positivi, poi quelli negativi, e con la differenza che in $\mathbb{R}$ non esistono le radici degli autovalori negativi, per cui, posto $k_1$ e $k_2$ uguali al numero di autovalori positivi e negativi rispettivamente, per gli $i = k_1 + 1, ..., k_1 + k_2$ sfrutteremo invece la relazione $- 1 = \left | \lambda_i \right |^{-1} P_i^tAP_i = (\left | \lambda_i \right | ^{-\frac{1}{2}} P_i^t)A(\left | \lambda_i \right | ^{-\frac{1}{2}} P_i)$.
	\par L'unicit\`a deriva stavolta dall'invarianza della segnatura di $A$ per trasformazioni proiettive. \EndProof
\begin{Theorem}\label{th66}
	Sia $(U_0,j_0^{-1})$ la carta affine canonica di $\mathbb{K}^n$ e sia $\mathcal{Q}$ un sottoinsieme di $\mathbb{K}^n$. $\mathcal{Q}$ \`e una quadrica affine di equazione $X^tAX + 2B^tX + C = 0$ (rispetto alla base canonica), dove $A$ \`e una matrice simmetrica $n \times n$, $B$ un vettore colonna di $n$ componenti, $C \in \mathbb{K}$, $X = (x_1, ..., x_n)$, se e solo se $j_0(\mathcal{Q}) = \bar{\mathcal{Q}} \cap U_0$, dove $\bar{\mathcal{Q}}$ \`e la quadrica proiettiva descritta da $\bar{X}^t\bar{A}\bar{X} = 0$ (nel riferimento proiettivo canonico), dove $\bar{X} = (x_0, ..., x_n)$, $\bar{A} = \left ( \begin{array}{c|c} C & B^t\\\hline B & A \end{array} \right )$.
\end{Theorem}
\Proof Sia $\mathcal{Q}$ una quadrica affine di equazione $X^tAX + 2 B^tX + C = 0$, con $A$, $B$, $C$ oggetti della forma descritta dall'enunciato. Sia $j_0(X) = Y$. Abbiamo $X \in \mathcal{Q}$ se e solo $j_0^{-1}(Y)^tAj_0^{-1}(Y) + 2 B^tj_0^{-1}(Y) + C = 0$, la qual cosa, scegliendo $(1 \cdots X \cdots)^t$ per rappresentante di $Y$, \`e evidentemente equivalente a $Y^t\bar{A}Y = 0$ e dunque a $j_0(X) \in \bar{\mathcal{Q}}$.
	\par Il viceversa \`e analogo. \EndProof
\begin{Definition}\label{def54}
	Sia $\mathcal{Q}$ una quadrica affine di $\mathbb{K}^n$. La quadrica proiettiva costruita come nel teorema precedente si chiama \Define{chiusura proiettiva}[proiettiva][chiusura] di $\mathcal{Q}$, mentre la quadrica proiettiva $\mathcal{Q}_\infty = \bar{\mathcal{Q}} \cap H_0$ si chiama \Define{quadrica all'infinito}[all'infinito][quadrica].
\end{Definition}
	\par Nel seguito, $\mathcal{Q}$ sar\`a sempre una quadrica affine di $\mathbb{K}^n$ di equazione $X^tAX + 2B^tX + C = 0$ rispetto alla base canonica, dove $A$ \`e una matrice simmetrica $n \times n$, $B$ un vettore di $n$ componenti, $C$ una costante e $X = (x_1, ..., x_n)$. $\bar{\mathcal{Q}}$ sar\`a la chiusura proiettiva di $\mathcal{Q}$ di equazione $\bar{X}^t\bar{A}\bar{X} = 0$ rispetto al riferimento proiettivo canonico, dove $\bar{X} = (x_0, ..., x_n)$ e $\bar{A} = \left ( \begin{array}{c|c} C & B^t\\\hline B & A \end{array} \right )$.
	\par Osserviamo che $\bar{\mathcal{Q}}$ non \`e l'unica quadrica proiettiva che contiene $j_0(\mathcal{Q})$, n\'e la pi\`u piccola in un qualche senso. Consideriamo come modello del piano proiettivo la sfera $S^2$ con i punti antipodali identificati. Visualizziamo il piano affine in cui giace $\mathcal{Q}$ nel piano $x = 1$ dello spazio affine $\mathbb{R}^3$: allora la chiusura proiettiva della quadrica $\mathcal{Q}$ \`e data dall'intersezione delle rette individuate dall'origine e da ciascuno dei punti di $\mathcal{Q}$ con la sfera $S^2$. Supponiamo ora che $\mathcal{Q}$ sia una circonferenza centra nell'origine (del piano affine): in $\mathbb{R}^3$ esistono almeno due quadriche a centro che passano per la detta circonferenza e per l'origine (per esempio un cono e un iperboloide a una falda entrambi di rotazione), la cui traccia su $S^2$ individua due quadriche distinte.
\begin{Theorem}\label{th74}
	Siano $\mathcal{Q}$ e $\mathcal{Q}'$ due quadriche affini di $\mathbb{K}^n$. $\mathcal{Q}$ e $\mathcal{Q}'$ sono affinemente equivalenti se e solo se le loro chiusure proiettive, $\bar{\mathcal{Q}}$ e $\bar{\mathcal{Q}'}$ rispettivamente, sono proiettivamente equivalenti.
\end{Theorem}
\Proof Su tutti gli spazi affini consideriamo la base canonica e su tutti gli spazi proiettivi il riferimento proiettivo canonico. Per semplicit\`a, non specificheremo le dimensioni di ogni matrice coinvolta nella dimostrazione: tali dimensioni si possono facilmente inferire dal contesto.
\par Sia $\phi(X) = MX + v$ un'affinit\`a tale che $\phi(\mathcal{Q}) = \mathcal{Q}'$. L'equazione di $\mathcal{Q}'$ \`e allora $X^tM^{-1}AM^{-1}X + 2B^tM^{-1}X + C = 0$. La chiusura proiettiva $\bar{\mathcal{Q}'}$ \`e dunque definita dalla matrice $\left ( \begin{array}{c|c} C & B^tM^{-1}\\\hline M^{-1,t}B & M^{-1}AM^{-1} \end{array} \right )$.
\par Sia $M_P$ la matrice $M_P = \left ( \begin{array}{c|c} 1 & \cdots 0 \cdots\\\hline v & M \end{array} \right )$. Posto $f(\bar{X}) = M_P(\bar{X})$, abbiamo $f(\mathcal{Q}) = \bar{X}^tM_P^{-1,t}AM_P^{-1}\bar{X}$, che non \`e altro che l'equazione di $\bar{\mathcal{Q}'}$.\EndProof



	\part{Analisi}
	\chapter{Campo reale}
	\section{Sezioni di Dedekind.}
\label{CampoReale_SezioniDiDedekind}
\begin{Definition}
  Chiamiamo
  \Define{sezione di Dedekind}[di Dedekind][sezione]
  una partizione $(A,B)$ di $\mathbb{Q}$ in due parti tale che
  \[
    \ForAll{x \in A}{\ForAll{y \in B}{x < y}}.
  \]
  Denotiamo la classe di tutte le sezioni di Dedekind $\DedekindCuts$.
  \par Definiamo su $\DedekindCuts$ le operazioni
  \begin{itemize}
    \item somma che a $(A,B), (C,D) \in \DedekindCuts$ associa la sezione
      \[
        (A,B) + (C,D)
          = (\lbrace x | \Exists{a \in A}{\Exists{c \in C}{x = a + c}} \rbrace,
          \lbrace y | \Exists{b \in B}{\Exists{d \in D}{y = b + d}} \rbrace);
      \]
    \item prodotto che a $(A,B), (C,D) \in \DedekindCuts$ associa la sezione
      \[
        (A,B) \cdot (C,D)
          = (\lbrace x | \Exists{a \in A}{\Exists{c \in C}{x = ac}} \rbrace,
          \lbrace y | \Exists{b \in B}{\Exists{d \in D}{y = bd}} \rbrace).
      \]
  \end{itemize}
  Definiamo inoltre su $\DedekindCuts$ un ordinamento $\leq$ tale che
  \[
    \ForAll{(A,B) \in \DedekindCuts}{
      \ForAll{(C,D) \in \DedekindCuts}{
        \Implies{A \subseteq C}{(A,B) \leq (C,D)}}}.
  \]
\end{Definition}
\begin{Theorem}
  $\leq$ \`e un ordinamento totale su $\DedekindCuts$.
\end{Theorem}
\Proof Siano $(A,B), (C,D) \in \DedekindCuts$.
\par Supponiamo
$\Not{\Or{A \subseteq C}{C \subseteq A}}$.
\begin{Theorem}
  \label{CampoReale_Campo}
  $\DedekindCuts$ \`e un campo.
\end{Theorem}
\Proof Segue direttamente dalle definizioni. \EndProof
\begin{Theorem}
  \label{CampoReale_Archimedeo}
\end{Theorem}
\begin{Theorem}
  Abbiamo $\DedekindCuts = \mathbb{R}$, a meno di isomorfismo di campi.
\end{Theorem}
\Proof Segue dai teoremi \ref{CampoReale_Campo}, \ref{CampoReale_Archimede}

\section{Rappresentazione in base.}
\label{CampoReale_RappesentazioneInBase}
\begin{Theorem}
	Sia $\RealNumberBase \in \mathbb{N} \SetMin 2$. Per ogni $x \in \mathbb{R}$, esistono unici
	\begin{itemize}
		\item un intero $\RealNumberExponent \in \mathbb{Z}$;
		\item una successione $(a_n)_{n \in \mathbb{N}} \in \RealNumberBase^\mathbb{N}$;
	\end{itemize}
	tali che
	\begin{itemize}
		\item \TheoremName{condizione di normalizzazione}[di normalizzazione della rappresentazione di base][condizione] $a_0 \neq 0$;
		\item $(a_n)_{n \in \mathbb{N}}$ non converge a $\RealNumberBase - 1$;
		\item $x = \Sign{x} \RealNumberBase^\RealNumberExponent \sum_{n \in \mathbb{N}} a_n \RealNumberBase^{- (n + 1)}$.
	\end{itemize}
\end{Theorem}
\begin{Definition}
	\label{CampoReale_RappresentazioneInBase}
	Con le notazioni del teorema precedente, chiamiamo \Define{rappresentazione in base}[in base][rappresentazione] $\RealNumberBase$ di $x$ il dato
	\begin{itemize}
		\item del segno $\Sign{x}$;
		\item della quantit\`a $\RealNumberBase$, detta \Define{base della rappresentazione}[di una rappresentazione numerica][base];
		\item della quantit\`a $\RealNumberExponent$, detta \Define{esponente della rappresentazione}[di una rappresentazione numerica][esponente];
		\item della successione $(a_n)_{n \in \mathbb{N}}$, i cui elementi sono detti \Define{cifre della rappresentazione}[di una rappresentazione numerica][cifra].
	\end{itemize}
	Inoltre, chiamiamo \Define{mantissa della rappresentazione}[mantissa] la quantit\`a $\sum_{n \in \mathbb{N}} a_n \RealNumberBase^{- (n + 1)}$. Se la successione delle cifre della rappresentazione \`e definitivamente nullo e $a_k$ \`e l'ultima cifra non nulla, diciamo che le cifre della rappresentazione sono in numero $k$.
\end{Definition}


	\chapter{Spazi vetttoriali complessi}
	\section{Norme}
\label{SpaziVettorialiComplessi_Norme}
\begin{Theorem}
  Siano
  \begin{itemize}
    \item $n \in \NotZero{\mathbb{N}}$;
    \item $p \in \mathbb{R}$ tale che $p \geq 1$.
  \end{itemize}
  L'applicazione $\Norm{\cdot}[p]: \mathbb{C}^n \rightarrow \mathbb{R}$
  definita per ogni $x \in \NormedSpace$ da
  \[
    \Norm{x}[p] = \sqrt[p]{\sum_{k \in n} \AbsoluteValue{x_k}^p}
  \]
  \`e una norma.
\end{Theorem}
\Proof Tutte le propriet\`a che definiscono una norma si verificano
immediatamente per $\Norm{\cdot}[p]$ ad eccezione della disuguaglianza
triangolare: procediamo duqnue a dimostrare la validit\`a anche di questa
disuguaglianza.
\par 

\begin{Theorem}
  Siano
  \begin{itemize}
    \item $n \in \NotZero{\mathbb{N}}$;
    \item $p \in \mathbb{R}$ tale che $p \geq 1$.
  \end{itemize}
  L'applicazione $\Norm{\cdot}[\infty]: \mathbb{C}^n \rightarrow \mathbb{R}$
  definita per ogni $x \in \NormedSpace$ da
  \[
    \Norm{x}[\infty] = \max_{k \in n} \AbsoluteValue{x_k}
  \]
  \`e una norma.
\end{Theorem}
\begin{Theorem}
  Sia $\Norm{\cdot}$ una norma su un $\mathbb{C}$-spazio vettoriale
  $\NormedSpace$ di dimensione finita $n \in \mathbb{N}$.
  $\Norm{\cdot}$ \`e un'applicazione continua rispetto alla topologia
  di $\NormedSpace$ indotta da $\Norm{\cdot}[1]$.
\end{Theorem}
\Proof Fissiamo $\epsilon > 0$.
\par Per la disuguaglianza triangolare abbiamo
\begin{align*}
  &\Norm{x} - \Norm{y} \leq \Norm{x - y},\\
  &\Norm{y} - \Norm{x} \leq \Norm{y - x} = \Norm{x - y},
\end{align*}
da cui
\[
  \AbsoluteValue{\Norm{x} - \Norm{y}} \leq \Norm{x - y}.
\]
\par Abbiamo inoltre, denotati $(e_i)_{i \in n}$ i vettori della base canonica
di $\mathbb{C}^n$,
\begin{align*}
  \Norm{x - y}
  &= \Norm{\sum_{i \in n} (x_i - y_i) e_i},\\
  &\leq \sum_{i \in n} \Norm{(x_i - y_i) e_i},\\
  &= \sum_{i \in n} \AbsoluteValue{x_i - y_i} \Norm{e_i},\\
  &\leq \sum_{i \in n} \AbsoluteValue{x_i - y_i} \max_{i \in n} \Norm{e_i},\\
  &\leq \Norm{x - y}[1] \max_{i \in n} \Norm{e_i}.
\end{align*}
\par Poniamo ora
$\delta = \frac{\epsilon}{\max_{i \in n} \Norm{e_i}}$ e supponiamo
$\Norm{x - y}[1] < \delta$. Abbiamo infine
\begin{align*}
  \AbsoluteValue{\Norm{x} - \Norm{y}}
  &\leq \Norm{x - y},\\
  &\leq \Norm{x - y}[1] \max_{i \in n} \Norm{e_i},\\
  &= \Norm{x - y}[1] \frac{\epsilon}{\delta},\\
  &< \epsilon.\text{ \EndProof}
\end{align*}
\begin{Theorem}
  \TheoremName{Teorema di equivalenza delle norme}[di equivalenza delle norme][teorema]
  Fissato $n \in \NotZero{\mathbb{N}}$, siano $\Norm{\cdot}$ e $\Norm{\cdot}'$ due norme
  su $\mathbb{C}^n$. Valgono i seguenti fatti:
  \begin{itemize}
    \item esistono costanti $m, M \in \RealNonNegative$ tali che, per ogni
      $x \in \mathbb{C}^n$,
      \[
        m\Norm{x}' \leq \Norm{x} \leq M\Norm{x}';
      \]
    \item $\Norm{\cdot}$ e $\Norm{\cdot}'$ sono norme equivalenti.
  \end{itemize}
\end{Theorem}
\Proof Assumiamo in un primo momento $\Norm{\cdot}' = \Norm{\cdot}[1]$.
\par Se $x = 0$, allora la tesi \`e immediata.
\par Supponiamo $x \neq 0$ e poniamo
$y = \frac{x}{\Norm{x}[1]}$. Abbiamo $y \in \NSphere{n}$, dove $\NSphere{n}$ \`e
la sfera unitaria rispetto alla norma $\Norm{\cdot}[1]$.
\par L'insieme $\NSphere{n}$ \`e limitato per definizione e chiuso in quanto
controimmagine di $1$ rispetto alla norma $\Norm{\cdot}[1]$, che \`e
un'applicazione continua: $\NSphere{n}$ \`e dunque compatto.
Ne consegue, per il teorema precedente e per il teorema di Weierstrass,
 che la norma $\Norm{\cdot}$ ammette minimo $m \in \RealNonNegative$ e massimo
$M \in \RealNonNegative$ su $\NSphere{n}$.
\par Abbiamo allora
\[
  m \leq \Norm{y} \leq M,
\]
da cui,
\[
  m\Norm{x}[1] \leq \Norm{x} \leq M\Norm{x}[1].
\]
\par Ora, per $\Norm{\cdot}'$ arbitraria, basta osservare che per opportune
costanti $m', M' \in \mathbb{R}$ abbiamo 
\[
  m'\Norm{x}[1] \leq \Norm{x}' \leq M'\Norm{x}[1],
\]
da cui, dividendo per $m'$ e combinando con la coppia di diseguaglianze
precedente,
\[
  \Norm{x} \leq M\Norm{x}[1] \leq \frac{M}{m'}\Norm{x}';
\]
analogamente, dividendo per $M'$ e combinando,
\[
  \frac{m}{M'} \Norm{x}' \leq m\Norm{x}[1] \leq \Norm{x};
\]
e quindi
\[
  \frac{m}{M'} \Norm{x}' \leq m\Norm{x}[1] \leq \Norm{x}
  \leq M\Norm{x}[1] \leq \frac{M}{m'}\Norm{x}'.
\]
\par Sia $\Open$ un aperto della topologia indotta da $\Norm{\cdot}$.
Per ogni $x \in \Open$ esiste $r > 0$ tale che
\[
  \OpenBall{x}{r} = \lbrace y \in \mathbb{C}^n | \Norm{x - y} < r \rbrace
    \subseteq \Open.
\]
\par Per quanto gi\`a provato, esiste $M \in \RealNonNegative$ tale che
$\Norm{x - y} \leq M\Norm{x - y}'$.
\par Poniamo inoltre
\[
  \OpenBall{x}{r}' = \lbrace y \in \mathbb{C}^n |
    \Norm{x - y}' < \frac{r}{M} \rbrace.
\]
\par Per ogni $y \in \OpenBall{x}{r}'$, abbiamo
\begin{align*}
  \Norm{x - y}
  &\leq M\Norm{x - y}',\\
  &< M \frac{r}{M},\\
  &= r.
\end{align*}
\par Ne deduciamo
$\OpenBall{x}{r}' \subseteq \OpenBall{x}{r} \subseteq \Open$.
Dunque $\Open$ \`e aperto rispetto alla topologia indotta da $\Norm{\cdot}'$.
Quindi la topologia indotta da $\Norm{\cdot}'$ \`e pi\`u fine rispetto alla
topologia indotta da $\Norm{\cdot}$. Analogamente si prova che la topologia
indotta da $\Norm{\cdot}$ \`e pi\`u fine di quella indotta da $\Norm{\cdot}'$.
E quindi infini le due topologie sono equivalenti. \EndProof
\begin{Corollary}
  Fissati $p, q \in \NotZero{\mathbb{N}}$, siano
  \begin{itemize}
    \item $\Norm{\cdot}[a]$ e $\Norm{\cdot}[b]$ norme su $\mathbb{C}^p$;
    \item $\Norm{\cdot}[A]$ e $\Norm{\cdot}[B]$ norme su $\mathbb{C}^q$;
    \item $f: \mathbb{C}^p \rightarrow \mathbb{C}^q$ un'applicazione
      lipschitziana rispetto alle distanze indotte da $\Norm{\cdot}[a]$ e
      $\Norm{\cdot}[A]$.
  \end{itemize}
  $f$ \`e lipschitziana anche rispetto alle distanze indotta da
  $\Norm{\cdot}[b]$ e $\Norm{\cdot}[B]$.
\end{Corollary}
\Proof Sia $L > 0$ tale che, che per ogni $x, y \in \mathbb{C}^p$ abbiamo
$\Norm{f(y) - f(x)}[A] \leq L \Norm{y - x}[a]$.
\par Per il teorema di equivalenza delle norme,
\begin{itemize}
  \item esiste $M$ tale che, per ogni
    $x, y \in \mathbb{C}^q$, $\Norm{y - x}[a] \leq M \Norm{y - x}[b]$;
  \item esiste $m$ tale che, per ogni
    $X, Y \in \mathbb{C}^p$, $m \Norm{Y - X}[B] \leq \Norm{Y - X}[A]$.
\end{itemize}
\par Ne deduciamo
\begin{align*}
  \Norm{f(y) - f(x)}[B]
  &\leq m^{-1} \Norm{f(y) - f(x)}[A],\\
  &\leq m^{-1} L \Norm{y - x}[a],\\
  &\leq m^{-1} L M \Norm{y - x}[b]. \text{ \EndProof}
\end{align*}

\section{Teorema spettrale}
\label{SpaziVettorialiComplessi_TeoremaSpettrale}
\begin{Theorem}
  Sia $\LinearOperator$ un operatore lineare su uno spazio prehilbertiano
  complesso $\PrehilbertSpace$ con prodotto hermitiano
  $\HermitianProduct{\cdot}{\cdot}$.
  Abbiamo
  \[
    \Coimplies{\Eigenvalue \in \Spectrum{\LinearOperator}}
    {\Conjugate{\Eigenvalue} \in \Spectrum{\Adjoint{\LinearOperator}}}.
  \]
\end{Theorem}
\Proof $\Eigenvalue \in \mathbb{C}$ non \`e un autovalore di $\LinearOperator$
se e solo se $\LinearOperator - \Eigenvalue\Identity$ \`e invertibile,
vale a dire
se e solo se esiste un altro operatore $\VarLinearOperator$ tale che
\[
  \VarLinearOperator(\LinearOperator - \Eigenvalue\Identity) =
  (\LinearOperator - \Eigenvalue\Identity)\VarLinearOperator =
  \Identity;
\]
ci\`o equivale a
\[
  (\Adjoint{\LinearOperator} - \Conjugate{\Eigenvalue}\Identity)
    \Adjoint{\VarLinearOperator} =
  \Adjoint{\VarLinearOperator}
    (\Adjoint{\LinearOperator} - \Conjugate{\Eigenvalue}\Identity) =
  \Identity;
\]
e dunque a
$\Conjugate{\Eigenvalue} \notin \Spectrum{\Adjoint{\LinearOperator}}$.
\EndProof
\begin{Theorem}
  Sia $\NormalOperator$ un operatore normale su uno spazio prehilbertiano
  complesso $\PrehilbertSpace$ con prodotto hermitiano
  $\HermitianProduct{\cdot}{\cdot}$.
  Sia inoltre $\Eigenvalue \in \mathbb{C}$ un autovalore di $\NormalOperator$
  relativo all'autovettore $\Eigenvector \in \PrehilbertSpace$.
  Allora $\Adjoint{\NormalOperator} \Eigenvector
  = \Conjugate{\Eigenvalue} \Eigenvector$.
\end{Theorem}
\Proof Abbiamo
\begin{align*}
\Norm{\Adjoint{\NormalOperator}\Eigenvector-\Conjugate{\Eigenvalue}\Eigenvector}
  &= \HermitianProduct{\Adjoint{\NormalOperator}\Eigenvector}
        {\Adjoint{\NormalOperator}\Eigenvector}
    - \HermitianProduct{\Adjoint{\NormalOperator}\Eigenvector}
        {\Conjugate{\Eigenvalue}\Eigenvector}
    - \HermitianProduct{\Conjugate{\Eigenvalue}\Eigenvector}
        {\Adjoint{\NormalOperator}\Eigenvector}
    + \HermitianProduct{\Conjugate{\Eigenvalue}\Eigenvector}
        {\Conjugate{\Eigenvalue}\Eigenvector},\\
  &= \HermitianProduct{\NormalOperator\Adjoint{\NormalOperator}\Eigenvector}
        {\Eigenvector}
    - \HermitianProduct{\Adjoint{\NormalOperator}\NormalOperator\Eigenvector}
        {\Eigenvector}
    - \HermitianProduct{\Eigenvalue\Eigenvector}
        {\Eigenvalue\Eigenvector}
    + \HermitianProduct{\Eigenvalue\Eigenvector}
        {\Eigenvalue\Eigenvector},\\
  &= \HermitianProduct{\NormalOperator\Adjoint{\NormalOperator}\Eigenvector
        - \Adjoint{\NormalOperator}\NormalOperator\Eigenvector}
        {\Eigenvector},\\
  &= 0.\text{ \EndProof}
\end{align*}
\begin{Corollary}
  Sia $\LinearOperator$ un operatore lineare su uno spazio prehilbertiano
  complesso $\PrehilbertSpace$ con prodotto hermitiano
  $\HermitianProduct{\cdot}{\cdot}$. Tutti gli autovalori di
  $\HermitianOperator$ sono reali.
\end{Corollary}
\Proof $\HermitianOperator$ \`e hermitiano dunque normale.
\par Quindi un autovettore ha per autovalore sia $\Eigenvalue$ che
$\Conjugate{\Eigenvalue}$ e per l'unicit\`a dell'autovettore deve essere
$\Eigenvalue = \Conjugate{\Eigenvalue} \in \mathbb{R}$. \EndProof
\begin{Corollary}
  Sia $\NormalOperator$ un operatore normale su uno spazio prehilbertiano
  complesso $\PrehilbertSpace$ con prodotto hermitiano
  $\HermitianProduct{\cdot}{\cdot}$.
  Gli autospazi di $\NormalOperator$ sono fra loro ortogonali.
\end{Corollary}
\Proof Siano $\Eigenvalue, \VarEigenvalue \in \Spectrum{\NormalOperator}$,
con $\Eigenvalue \neq \VarEigenvalue$.
Siano inoltre $\Eigenvector$ e $\VarEigenvector$ autovettori di
$\NormalOperator$ relativi a $\Eigenvalue$ e $\VarEigenvalue$
rispettivamente.
\par Supponiamo, senza perdita di generalit\`a, che il prodotto interno sia
un prodotto hermitiano.
\par Abbiamo
\begin{align*}
  \Eigenvalue \HermitianProduct{\Eigenvector}{\VarEigenvector}
  &= \HermitianProduct{\Eigenvalue\Eigenvector}{\VarEigenvector},\\
  &= \HermitianProduct{\NormalOperator\Eigenvector}{\VarEigenvector},\\
  &= \HermitianProduct{\Eigenvector}{\Adjoint{\NormalOperator}\VarEigenvector},\\
  &= \HermitianProduct{\Eigenvector}{\Conjugate{\VarEigenvalue}\VarEigenvector},\\
  &= \VarEigenvalue\HermitianProduct{\Eigenvector}{\VarEigenvector}.
\end{align*}
\par Da cui necessariamente
$\HermitianProduct{\Eigenvector}{\VarEigenvector}$. \EndProof
\begin{Theorem}
  Sia $\NormalOperator$ un operatore normale su uno spazio prehilbertiano
  complesso $\PrehilbertSpace$ con prodotto hermitiano
  $\HermitianProduct{\cdot}{\cdot}$.
  Gli autospazi di $\NormalOperator$ sono tra loro ortogonali.
\end{Theorem}
\Proof Siano $\Eigenvalue, \VarEigenvalue \in \Spectrum{\NormalOperator}$ e
$\Eigenvector, \VarEigenvector \in \PrehilbertSpace$ tali che
$\NormalOperator\Eigenvector = \Eigenvalue\Eigenvector$
e
$\NormalOperator\VarEigenvector = \VarEigenvalue\VarEigenvector$.
\par Abbiamo
\begin{align*}
  \Eigenvalue\HermitianProduct{\Eigenvector}{\VarEigenvector}
  &= \HermitianProduct{\Eigenvalue\Eigenvector}{\VarEigenvector},\\
  &= \HermitianProduct{\NormalOperator\Eigenvector}{\VarEigenvector},\\
  &=\HermitianProduct{\Eigenvector}{\Adjoint{\NormalOperator}\VarEigenvector},\\
 &=\HermitianProduct{\Eigenvector}{\Conjugate{\VarEigenvalue}\VarEigenvector},\\
  &= \VarEigenvalue\HermitianProduct{\Eigenvector}{\VarEigenvector}.
\end{align*}
\par Dunque
$\Or{\Eigenvalue = \VarEigenvalue}
{\HermitianProduct{\Eigenvector}{\VarEigenvector}}$. \EndProof
\begin{Theorem}
  Sia $\NormalOperator$ un operatore normale su uno spazio prehilbertiano
  complesso $\PrehilbertSpace$ con prodotto hermitiano
  $\HermitianProduct{\cdot}{\cdot}$.
  Sia inoltre $\Eigenvector$ autovettore di $\NormalOperator$:
  lo spazio $\OrthogonalComplement{\LinearSpan{\Eigenvector}}$ \`e 
  invariante rispetto a $\NormalOperator$.
\end{Theorem}
\Proof Sia $\Vector \in \OrthogonalComplement{\LinearSpan{\Eigenvector}}$ e
sia $\Eigenvalue \in \Spectrum{\NormalOperator}$ l'autovalore associato a
$\Eigenvector$.
Abbiamo
\begin{align*}
  \HermitianProduct{\NormalOperator\Vector}{\Eigenvector}
  &= \HermitianProduct{\NormalOperator\Vector}{\Eigenvector},\\
  &= \HermitianProduct{\Vector}{\Adjoint{\NormalOperator}\Eigenvector},\\
  &= \HermitianProduct{\Vector}{\Conjugate{\Eigenvalue}\Eigenvector},\\
  &= \HermitianProduct{\Eigenvalue\Vector}{\Eigenvector},\\
  &= \Eigenvalue\HermitianProduct{\Vector}{\Eigenvector},\\
  &= 0.\text{ \EndProof}
\end{align*}
\begin{Corollary}
  \TheoremName{Teorema spettrale}[spettrale][teorema] Sia $\NormalOperator$ un
  operatore normale su uno spazio prehilbertiano complesso $\PrehilbertSpace$
  con prodotto hermitiano $\HermitianProduct{\cdot}{\cdot}$.
  Esiste una base $(\Eigenvector_k)_{k \in n}$ di $\PrehilbertSpace$ costituita
  da autovettori ortonormali di $\NormalOperator$.
  Inoltre,
  \begin{itemize}
    \item la matrice che rappresenta $\NormalOperator$ \`e diagonale, con tutti
      gli autovalori di $\NormalOperator$ sulla diagonale presenti tante volte
      quanta \`e la molteplicit\`a geometrica di ciascuno di essi;
    \item per ogni autovalore di $\NormalOperator$, la molteplicit\`a algebrica
      e la molteplicit\`a geometrica coincidono.
  \end{itemize}
\end{Corollary}
\Proof Procediamo per induzione sulla dimensione di $\PrehilbertSpace$.
Se $\LinearDimension{\PrehilbertSpace} = 1$ il teorema \`e immediatamente
dimostrato.
\par Supponiamo il teorema dimostrato per
$\LinearDimension{\PrehilbertSpace} = N$
e proviamolo per
$\LinearDimension{\PrehilbertSpace} = N + 1$.
\par Sia $\Eigenvalue \in \Spectrum{\NormalOperator}$ e
$\Eigenvector \in \PrehilbertSpace$
autovettore associato a $\Eigenvalue$. Per il teorema precedente,
$\NormalOperator$ si riduce su $\LinearSpan{\Eigenvector}$ e
$\OrthogonalComplement{\LinearSpan{\Eigenvector}}$.
Ma per ipotesi induttiva,
$\OrthogonalComplement{\LinearSpan{\Eigenvector}}$
ammette una base di autovettori ortonormali di
$\Restricted{\NormalOperator}{\OrthogonalComplement{\LinearSpan{\Eigenvector}}}$
e dunque di $\NormalOperator$. Estendiamo tale base col vettore
$\frac{\Eigenvector}{\Norm{\Eigenvector}}$ e, se necessario, ortogonalizziamo
l'insieme dei vettori della base che sono autovettori di $\NormalOperator$
con autovalore $\Eigenvalue$: la base cos\`i ottenuta \`e ortonormale.
\par Ne segue immediatamente la diagonalit\`a della matrice che rappresenta
$\NormalOperator$ rispetto ad una base s\`i costituita con gli autovalori
presenti sulla diagonale tante volte quanta \`e la loro molteplicit\`a
geometrica. Infine, l'uguaglianza delle molteplicit\`a geometriche e algebriche
segue dal calcolo diretto del polinomio caratteristico a partire dalla suddetta
matrice diagonale. \EndProof
\begin{Definition}
  Con le notazioni del teorema precedente diciamo che
  $(\Eigenvector_k)_{k \in n}$ \`e una
  \Define{base spettrale}[spettrale][base].
\end{Definition}


	\chapter{Teoria delle successioni e delle serie}
	\section{Successioni in spazi vettoriali normati.}
\label{TeoriaDelleSuccessioniEDelleSerie_SuccessioniInSpaziVettorialiNormati}
\begin{Definition}
	Sia $\NormedSpace$ un $\Field$-spazio vettoriale normato.
	Sia $(a_n)_{n \in \mathbb{N}} \in \NormedSpace^\mathbb{N}$. $(a_n)_{n \in \mathbb{N}}$ si dice
	\begin{itemize}
		\item \Define{divergente}[divergente][successione] se $\ForAll{M \in \mathbb{R}}{\Exists{N \in \mathbb{N}}{\Implies{n > N}{\Norm{a_n} > M}}}$, e in tal caso scriviamo anche $\lim_{n \rightarrow \infty} a_n = \infty$;
		\item \Define{indeterminata}[indeterminata][successione] se non \`e n\'e convergente n\'e divergente.
	\end{itemize}
	Chiamiamo la propriet\`a di $(a_n)_{n \in \mathbb{N}}$ di convergere, divergere o non avere limite \Define{comportamento}[di una successione][comportamento] di $(a_n)_{n \in \mathbb{N}}$
\end{Definition}
\begin{Definition}
	Sia $\NormedSpace$ un $\Field$-spazio vettoriale normato.
	Sia $(a_n)_{n \in \mathbb{N}} \in \NormedSpace^\mathbb{N}$ e sia la sottosuccessione $(a_{f(n)})_{n \in \mathbb{N}}$, con $f: \mathbb{N} \rightarrow \mathbb{N}$ crescente. Se $(a_n)_{n \in \mathbb{N}}$ diverge, allora diverge anche $(a_{f(n)})_{n \in \mathbb{N}}$.
\end{Definition}
\Proof Segue direttamente dalle definizioni. \EndProof
\begin{Definition}
	Siano
	\begin{itemize}
		\item $(\TopologicalSpace,\Topology)$ uno spazio topologico;
		\item $\NormedSpace$ un $\Field$-spazio vettoriale normato;
		\item $\Part \subseteq \TopologicalSpace$;
		\item $f: \Part \rightarrow \VarTopologicalSpace$;
		\item $\AccumulationPoint$ punto di accumulazione per $\Part$.
	\end{itemize}
	Se $\ForAll{M > 0}{\Exists{\Neighborhood_\AccumulationPoint \in \Neighborhoods{\AccumulationPoint} \cap \Topology}{\Implies{x \in \Neighborhood_\AccumulationPoint \cap \Part}{\Norm{f(x)} > M}}}$, allora diciamo che $f(x)$ \Define{diverge}[di una funzione][divergenza], in simboli $f(x) \rightarrow \infty$, per $x$ che tende a $\AccumulationPoint$, in simboli $x \rightarrow \AccumulationPoint$; diciamo anche che $\infty$ \`e il \Define{limite}[infinito di una funzione][limite], denotato $\lim_{x \rightarrow \AccumulationPoint} f(x) = \infty$. In tal caso diciamo inoltre che la retta $x = \AccumulationPoint$ \`e \Define{asintoto verticale}[verticale][asintoto] del grafico della funzione $y = f(x)$ nel piano $\OxyPlane$.
\end{Definition}
\begin{Theorem}
	Siano
	\begin{itemize}
		\item $(\TopologicalSpace,\Topology)$ spazio topologico che verifica il primo assioma di numerabilit\`a;
		\item $\NormedSpace$ un $\Field$-spazio vettoriale normato;
		\item una successione $(a_n)_{n \in \mathbb{N}} \in \TopologicalSpace^\mathbb{N}$ convergente a $\AccumulationPoint \in \TopologicalSpace$;
		\item $f: \bigcup_{n \in \mathbb{N}} \lbrace a_n \rbrace \rightarrow \NormedSpace$.
	\end{itemize}
	Se \`e vero l'assioma della scelta, $(f(a_n))_{n \in \mathbb{N}}$ diverge se e solo se $f(x)$ diverge per $x \rightarrow \AccumulationPoint$.
\end{Theorem}
\Proof Per prima cosa osserviamo che $\AccumulationPoint$ \`e di accumulazione per il dominio di $f$.
\par Supponiamo $f(x) \rightarrow \infty$ per $x \rightarrow \AccumulationPoint$. Sia $M > 0$: per opportuno $\Neighborhood_\AccumulationPoint \in \Neighborhoods{\AccumulationPoint} \cap \Topology$, abbiamo, per ogni $x \in \Neighborhood_\AccumulationPoint \cap \bigcup_{n \in \mathbb{N}} \lbrace a_n \rbrace$, $\Norm{f(x)} > M$.  Ma $a_n \rightarrow \AccumulationPoint$ per $n \rightarrow \infty$ e quindi esiste $N \in \mathbb{N}$ tale che $\Implies{n > N}{a_n \in \Neighborhood_\AccumulationPoint}$, da cui $\Implies{n > N}{\Norm{f(a_n)} > M}$. Ne deduciamo $f(a_n) \rightarrow \infty$ per $n \rightarrow \infty$.
\par Supponiamo ora $\lim_{n \rightarrow \infty} f(a_n) = \infty$. Supponiamo inoltre che $f(x)$ non diverga per $x \rightarrow \AccumulationPoint$. Allora esiste $M > 0$ tale che $\ForAll{\Neighborhood_\AccumulationPoint \in \Neighborhoods{\AccumulationPoint} \cap \Topology}{\Exists{x \in \Neighborhood_\AccumulationPoint \cap \bigcup_{m \in \mathbb{N}} \lbrace a_m \rbrace}{\Norm{f(x)} \leq M}}$. Sia $(\LocalBase_n)_{n \in \mathbb{N}}$ un sistema fondamentale di intorni aperti di $\AccumulationPoint$. Abbiamo $\ForAll{n \in \mathbb{N}}{\Exists{m \in \mathbb{N}}{\And{a_m \in \LocalBase_n}{\Norm{f(a)} \leq M}}}$. D'altra parte, lo stesso ragionamento pu\`o essere ripetuto per ogni sottosuccesione di $(a_n)_{n \in \mathbb{N}}$ del tipo $(a_{n + k})_{n \in \mathbb{N}}$ per qualsiasi $k \in \mathbb{N}$ e sempre con lo stesso $M > 0$ e quindi $\ForAll{n \in \mathbb{N}}{\ForAll{N \in \mathbb{N}}{\Exists{m \in \mathbb{N} \SetMin N}{\And{a_m \in \LocalBase_n}{\Norm{f(a_m)} \leq M}}}}$.
\par Possiamo quindi definire la sottosuccessione $(a_{g(n)})_{n \in \mathbb{N}} \in \TopologicalSpace^{\mathbb{N}}$, con $g: \mathbb{N} \rightarrow \mathbb{N}$ applicazione strettamente crescente, tale che $\ForAll{n \in \mathbb{N}}{\Norm{f(a_{g(n)})} \leq M}$. Allora
\begin{itemize}
	\item $(f(a_{g(n)}))_{n \in \mathbb{N}}$ diverge in quanto sottosuccessione di $(f(a_n))_{n \in \mathbb{N}}$;
	\item $(f(a_{g(n)}))_{n \in \mathbb{N}}$ non diverge perch\'e $\ForAll{n \in \mathbb{N}}{\Norm{f(a_{g(n)})} \leq M}$.
\end{itemize}
Ma questo \`e assurdo, dunque $f(x)$ deve divergere per $x \rightarrow \AccumulationPoint$. \EndProof
\begin{Definition}
	Sia $\NormedSpace$ un $\Field$-spazio vettoriale normato.
	Chiamiamo \Define{successione aritmetica}[arimetica][successione] o anche \Define{progressione aritmetica}[aritmetica][progessione] una successione $(a_n)_{n \in \mathbb{N}} \in \NormedSpace^\mathbb{N}$ tale che per opportuna costante $K \in \NormedSpace$, $\ForAll{n \in \mathbb{N}}{a_{n + 1} - a_n = K}$. La costante $K$ si chiama \Define{ragione}[di una progressione aritemetica][ragione] della progressione aritmetica.
\end{Definition}
\begin{Theorem}
	Tutte le progressioni aritmetiche divergono.
\end{Theorem}
\Proof Per definizione, le progressioni aritmetiche non sono di Cauchy. \EndProof

\section{Successioni complesse.}\label{SuccessioniComplesse}
\begin{Theorem}
	Siano $(a_n)_{n \in \mathbb{N}}, (b_n)_{n \in \mathbb{N}} \in \mathbb{C}^\mathbb{N}$ convergenti e $\Scalar \in \NotZero{\mathbb{C}}$. Abbiamo
	\begin{itemize}
		\item $(a_n + b_n)_{n \in \mathbb{N}}$ converge e $\lim_{n \rightarrow \infty} (a_n + b_n) = \lim_{n \rightarrow \infty} a_n + \lim_{n \rightarrow \infty} b_n$;
		\item $(a_n b_n)_{n \in \mathbb{N}}$ converge e $\lim_{n \rightarrow \infty} (a_n b_n) = \left ( \lim_{n \rightarrow \infty} a_n \right ) \left ( \lim_{n \rightarrow \infty} b_n \right )$;
		\item $(\Scalar a_n)_{n \in \mathbb{N}}$ converge e $\lim_{n \rightarrow \infty} \Scalar a_n = \Scalar \lim_{n \rightarrow \infty} a_n$;
		\item se $\ForAll{n \in \mathbb{N}}{a_n \neq 0}$ e $\lim_{n \rightarrow \infty} a_n \neq 0$, allora $(a_n^{-1})_{n \in \mathbb{N}}$ converge e $\lim_{n \rightarrow \infty} a_n^{-1} = \left ( \lim_{n \rightarrow \infty} a_n \right )^{-1}$;
	\end{itemize}
\end{Theorem}
\Proof Fissiamo $\epsilon > 0$ e poniamo $A = \lim_{n \rightarrow \infty} a_n$ e $B = \lim_{n \rightarrow \infty} b_n$. Abbiamo
\begin{itemize}
	\item per $N \in \mathbb{N}$ opportuno $\Implies{n > N}{\And{\AbsoluteValue{a_n - A} < \frac{\epsilon}{2}}{\AbsoluteValue{b_n - B} < \frac{\epsilon}{2}}}$, da cui, per $n > N$, $\AbsoluteValue{(a_n + b_n) - (A + B)} \leq \AbsoluteValue{a_n - A} + \AbsoluteValue{b_n - B} < \epsilon$;
	\item fissato $K \in \mathbb{R}$ tale che $\ForAll{n \in \mathbb{N}}{\AbsoluteValue{b_n} < K}$, per $N \in \mathbb{N}$ opportuno $\Implies{n > N}{\And{\AbsoluteValue{a_n - A} < \frac{\epsilon}{K + \AbsoluteValue{A}}}{\AbsoluteValue{b_n - B} < \frac{\epsilon}{K + \AbsoluteValue{A}}}}$, da cui, per $n > N$, $\AbsoluteValue{a_nb_n - AB} = \AbsoluteValue{a_nb_n - Ab_n + Ab_n - AB} \leq \AbsoluteValue{a_n - A}\AbsoluteValue{b_n} + \AbsoluteValue{A}\AbsoluteValue{b_n - B} < \frac{\epsilon}{K + \AbsoluteValue{A}} (K + \AbsoluteValue{A}) = \epsilon$;
	\item per $N \in \mathbb{N}$ opportuno $\Implies{n > N}{\AbsoluteValue{a_n - A} < \AbsoluteValue{\Scalar}^{-1}\epsilon}$, da cui, per $n > N$, $\AbsoluteValue{\Scalar a_n - \Scalar A} = \AbsoluteValue{\Scalar}\AbsoluteValue{a_n - A} < \epsilon$;
	\item fissato $H \in \mathbb{R}$ tale che $\ForAll{n \in \mathbb{N}}{\AbsoluteValue{a_n}} < H$, per $N \in \mathbb{N}$ opportuno $\Implies{n > N}{\AbsoluteValue{a_n - A} < \epsilon H \AbsoluteValue{A}}$, da cui $\AbsoluteValue{\frac{1}{a_n} - \frac{1}{A}} = \frac{\AbsoluteValue{A - a_n}}{\AbsoluteValue{a_n} \AbsoluteValue{A}} < \frac{\epsilon H \AbsoluteValue{A}}{H \AbsoluteValue{A}} = \epsilon$. \EndProof
\end{itemize}
\begin{Corollary}
	Le successioni convergenti di $\mathbb{C}^\mathbb{N}$ costituiscono un sottospazio vettoriale di $\mathbb{C}^\mathbb{N}$.
\end{Corollary}
\Proof Segue direttamente dalle definizioni e dal teorema precedente. \EndProof
\begin{Definition}
	Chiamiamo \Define{spazio delle successioni complesse convergenti}[delle successioni complesse convergenti][spazio] il sottospazio spazio vettoriale di $\mathbb{C}^\mathbb{N}$ di tutte le successioni convergenti, denotato $\ConvergentComplexSequences$.
\end{Definition}
\begin{Theorem}
	Sia $(a_n)_{n \in \mathbb{N}} \in \mathbb{C}^\mathbb{N}$. Allora
	\begin{itemize}
		\item $(a_n)_{n \in \mathbb{N}}$ converge se e solo se $(\RealPart{a_n})_{n \in \mathbb{N}}$ e $(\ImaginaryPart{a_n})_{n \in \mathbb{N}}$ convergono e in tal caso $\lim_{n \rightarrow \infty} a_n = \lim_{n \rightarrow \infty} \RealPart{a_n} + i \lim_{n \rightarrow \infty} \ImaginaryPart{a_n}$;
		\item se $(a_n)_{n \in \mathbb{N}}$ converge se e solo se $(\Conjugate{a_n})_{n \in \mathbb{N}}$ converge e in tal caso $\lim_{n \rightarrow \infty} \Conjugate{a_n} = \Conjugate{\lim_{n \rightarrow \infty} a_n}$;
		\item se $(a_n)_{n \in \mathbb{N}}$ converge, $(\AbsoluteValue{a_n})_{n \in \mathbb{N}}$ converge e abbiamo $\lim_{n \rightarrow \infty} \AbsoluteValue{a_n} = \AbsoluteValue{\lim_{n \rightarrow \infty} a_n}$.
	\end{itemize}
\end{Theorem}
\Proof Abbiamo gi\`a provato che se $(\RealPart{a_n})_{n \in \mathbb{N}}$ e $(\ImaginaryPart{a_n})_{n \in \mathbb{N}}$ convergono, allora converge anche $(a_n)_{n \in \mathbb{N}} = (\RealPart{a_n} + i \ImaginaryPart{a_n})_{n \in \mathbb{N}}$ e il suo limite \`e proprio $\lim_{n \rightarrow \infty} (\RealPart{a_n} + i \ImaginaryPart{a_n})$.
\par Assumiamo viceversa che $a_n$ converga e poniamo $L = \lim_{n \rightarrow \infty} a_n$. Fissiamo $\epsilon > 0$. Abbiamo, per opportuno $N \in \mathbb{N}$ e per ogni $n > N$, $\AbsoluteValue{a_n - L} < \epsilon$, da cui $\sqrt{(\RealPart{a_n - L})^2 + (\ImaginaryPart{a_n - L})^2} < \epsilon$ e quindi $\sqrt{(\RealPart{a_n - L})^2} = \AbsoluteValue{\RealPart{a_n - L}} = \AbsoluteValue{\RealPart{a_n} - \RealPart{L}} < \epsilon$ e analogamente $\AbsoluteValue{\ImaginaryPart{a_n} - \ImaginaryPart{L}} < \epsilon$. Pertanto $(\RealPart{a_n})_{n \in \mathbb{N}}$ e $(\ImaginaryPart{a_n} - L)_{n \in \mathbb{N}}$ convergono a $\RealPart{L}$ e $\ImaginaryPart{L}$ rispettivamente.
\par Abbiamo provato che se $(a_n)_{n \in \mathbb{N}}$ converge a $L \in \mathbb{C}$, allora convergono $(\RealPart{a_n})_{n \in \mathbb{N}}$ e $(\ImaginaryPart{a_n})_{n \in \mathbb{N}}$ a $\RealPart{L}$ e $\ImaginaryPart{L}$ e dunque $(\RealPart{a_n} - i \ImaginaryPart{a_n})_{n \in \mathbb{N}} = (\Conjugate{a_n})_{n \in \mathbb{N}}$ converge a $\RealPart{L} - i \ImaginaryPart{L} = \Conjugate{L}$.
\par Assumiamo $(\Conjugate{a_n})_{n \in \mathbb{N}}$ convega a $\Conjugate{L}$. Ne deduciamo che $(a_n)_{n \in \mathbb{N}} = (\Conjugate{\Conjugate{a_n}})_{n \in \mathbb{N}}$ converge a $L = \Conjugate{\Conjugate{L}}$.
\par Assumiamo infine che $(a_n)_{n \in \mathbb{N}}$ converga a $L$. Fissiamo $\epsilon > 0$ e $N \in \mathbb{N}$ tale che per ogni $n > N$ abbiamo $\AbsoluteValue{a_n - L} < \epsilon$. Abbiamo inoltre $\AbsoluteValue{\AbsoluteValue{a_n} - \AbsoluteValue{L}} = \AbsoluteValue{a_n} - \AbsoluteValue{L}$ o $\AbsoluteValue{\AbsoluteValue{L} - \AbsoluteValue{a_n}}$: in entrambi casi, ne deduciamo $\AbsoluteValue{\AbsoluteValue{a_n} - \AbsoluteValue{L}} \leq \AbsoluteValue{a_n - L} < \epsilon$. E quindi $(\AbsoluteValue{a_n})_{n \in \mathbb{N}}$ converge a $\AbsoluteValue{L}$. \EndProof
\begin{Theorem}
	\TheoremName{Teorema di permanenza del segno} Sia $(a_n)_{n \in \mathbb{N}}$ una successione complessa. Se $(a_n)_{n \in \mathbb{N}}$ converge a un limite $L \neq 0$, allora esistono $\delta > 0$ e $N \in \mathbb{N}$ tale che $\Implies{n > N}{\AbsoluteValue{a_n} \geq \delta}$.
\end{Theorem}
\Proof Fissiamo $\epsilon \in \mathbb{R}$ tale che $0 < \epsilon < \AbsoluteValue{L}$ e poniamo $\delta = \AbsoluteValue{L} - \epsilon$: chiaramente $\delta > 0$.
\par Inoltre, fissato $N \in \mathbb{N}$ tale che $\AbsoluteValue{\AbsoluteValue{a_n} - \AbsoluteValue{L}} < \epsilon$, per $n > N$,  abbiamo $\AbsoluteValue{a_n} \geq \AbsoluteValue{L} - \epsilon = \delta$. \EndProof
\begin{Definition}
	Chiamiamo \Define{successione geometrica}[arimetica][successione] o anche \Define{progressione geometrica}[geometrica][progessione] una successione complessa $(a_n)_{n \in \mathbb{N}}$ tale che per opportuna costatne $K \in \mathbb{C}$, ${\ForAll{n \in \mathbb{N}}{\frac{a_{n + 1}}{a_n} = K}}$. La costante $K$ si chiama \Define{ragione}[di una progressione geometrica][ragione] della progressione geometrica.
\end{Definition}

\section{Successioni asintoticamente equivalenti.}
\label{TeoriaDelleSuccessioniEDelleSerie_SuccessioniAsintoticamenteEquivalenti}
\begin{Definition}
	Siano $(a_n)_{n \in \mathbb{N}}, (b_n)_{n \in \mathbb{N}} \in \mathbb{C}^\mathbb{N} - \SetMin \mathbb{C}^{(\mathbb{N})}$. $(a_n)_{n \in \mathbb{N}}$ e $(b_n)_{n \in \mathbb{N}}$ si dicono \Define{asintoticamente equivalenti}[asintoticamente equivalenti][successioni], denotato $(a_n)_{n \in \mathbb{N}} \AsymptoticallyEquivalent (b_n)_{n \in \mathbb{N}}$, quando la successione $(\frac{a_n}{b_n})_{n \in \mathbb{N}}$ (supponiamo senza perdita di generalit\`a $\ForAll{n \in \mathbb{N}}{b_n \neq 0}$: in caso contrario omettiamo l'$n$-esimo rapporto e rinumeriamo la successione) converge e $\lim_{n \rightarrow \infty} \frac{a_n}{b_n} = 1$.
\end{Definition}
\begin{Theorem}
	L'equivalenza asintotica \`e una relazione di equivalenza su $\mathbb{C}^\mathbb{N} \SetMin \mathbb{C}^{(\mathbb{N})}$.
\end{Theorem}
\Proof La dimostrazione della riflessivit\`a e della simmetria di $\AsymptoticallyEquivalent$ \`e immediata.
\par Siano $(a_n)_{n \in \mathbb{N}}, (b_n)_{n \in \mathbb{N}}, (c_n)_{n \in \mathbb{N}} \in \mathbb{C}^\mathbb{N} \SetMin \mathbb{C}^{(\mathbb{N})}$ e assumiamo $(a_n)_{n \in \mathbb{N}} \AsymptoticallyEquivalent (b_n)_{n \in \mathbb{N}}$ e $(b_n)_{n \in \mathbb{N}} \AsymptoticallyEquivalent (c_n)_{n \in \mathbb{N}}$.
\par Abbiamo, per ogni $n \in \mathbb{N}$ tale che $b_n \neq 0$ e $c_n \neq 0$, $\frac{a_n}{c_n} = \frac{a_n}{b_n}\frac{b_n}{c_n}$ e quindi $\left ( \frac{a_n}{c_n} \right )$ converge a $1$, da cui $(a_n)_{n \in \mathbb{N}} \AsymptoticallyEquivalent (c_n)_{n \in \mathbb{N}}$. \EndProof

\section{Successioni reali.}\label{SuccessioniReali}
\begin{Definition}
	Sia $(a_n)_{n \in \mathbb{N}} \in \mathbb{R}^\mathbb{N}$.
  $(a_n)_{n \in \mathbb{N}}$ si dice
	\begin{itemize}
		\item
      \Define{a termini positivi}[a termini positivi][successione]
      quando $\ForAll{n \in \mathbb{N}}{a_n \geq 0}$;
		\item
      \Define{a termini strettamente positivi}[a termini positivi][successione]
      quando $\ForAll{n \in \mathbb{N}}{a_n > 0}$;
		\item
      \Define{a termini negativi}[a termini negativi][successione]
      quando $\ForAll{n \in \mathbb{N}}{a_n \leq 0}$;
		\item
      \Define{a termini strettamente negativi}[a termini negativi][successione]
      quando $\ForAll{n \in \mathbb{N}}{a_n < 0}$;
		\item
      \Define{divergente positivamente}[divergente positivamente][successione]
      o anche
      \Define{divergente a $+\infty$}[divergente a $+\infty$][successione] 
      quando
      \[
        \ForAll{M \in \mathbb{R}}{
          \Exists{N \in \mathbb{N}}{
            \Implies{n > N}{a_n > M}}},
      \]
      e in tal caso scriviamo anche
      $\lim_{n \rightarrow \infty} a_n = + \infty$;
		\item
      \Define{divergente negativamente}[divergente negativamente][successione]
      o anche
      \Define{divergente a $-\infty$}[divergente a $-\infty$][successione]
      quando
      \[
        \ForAll{M \in \mathbb{R}}{
          \Exists{N \in \mathbb{N}}{
            \Implies{n > N}{
              a_n < M}}},
      \]
      e in tal caso scriviamo anche
      $\lim_{n \rightarrow \infty} a_n = - \infty$.
	\end{itemize}
\end{Definition}
\begin{Theorem}
	Sia $(a_n)_{n \in \mathbb{N}} \in \mathbb{R}^\mathbb{N}$.
  $(a_n)_{n \in \mathbb{N}}$ diverge positivamente se e solo se
  $(- a_n)_{n \in \mathbb{N}}$ diverge negativamente.
\end{Theorem}
\Proof $(a_n)_{n \in \mathbb{N}}$ diverge positivamente se e solo se
$\ForAll{M > 0}{\Exists{N > 0}{\Implies{n > N}{a_n > M}}}$,
equivalente a
$\ForAll{M > 0}{\Exists{N > 0}{\Implies{n > N}{- a_n < - M}}}$,
cio\`e alla divergenza negativa di $(- a_n)_{n \in \mathbb{N}}$. \EndProof
\begin{Theorem}
	Sia $(a_n)_{n \in \mathbb{N}} \in \mathbb{R}^\mathbb{N}$ divergente
  positivamente. Ogni sottosuccessione $(a_{f(n)})_{n \in \mathbb{N}}$,
  con $f: \mathbb{N} \rightarrow \mathbb{N}$ applicazione crescente,
  diverge anch'essa positivamente.
\end{Theorem}
\Proof Sia $M > 0$. Esiste $N \in \mathbb{N}$ tale che
$\Implies{n > N}{a_n > M}$. A maggior ragione,
$\Implies{n > N}{a_{f(n)} > M}$ e dunque $(a_{f(n)})_{n \in \mathbb{N}}$
diverge positivamente. \EndProof
\begin{Corollary}
	Sia $(a_n)_{n \in \mathbb{N}} \in \mathbb{R}^\mathbb{N}$ divergente
  negativamente. Ogni sottosuccessione $(a_{f(n)})_{n \in \mathbb{N}}$,
  con $f: \mathbb{N} \rightarrow \mathbb{N}$ applicazione crescente,
  diverge anch'essa negativamente.
\end{Corollary}
\Proof $(a_n)_{n \in \mathbb{N}}$ diverge negativamente se e solo se
$(- a_n)_{n \in \mathbb{N}}$ diverge positivamente, che implica la divergenza
positiva di $(- a_{f(n)})_{n \in \mathbb{N}}$, equivalente alla divergenza
negativa di $(a_{f(n)})_{n \in \mathbb{N}}$. \EndProof
\begin{Theorem}
	Sia $(a_n)_{n \in \mathbb{N}} \in \mathbb{R}^\mathbb{N}$ una successione
  crescente.
  Se $(a_n)_{n \in \\mathbb{N}}$ \`e limitata, allora \`e convergente;
  altrimenti \`e divergente.
\end{Theorem}
\Proof Consideriamo $L = \sup_{n \in \mathbb{N}} a_n$: segue direttamente dalle
definizioni che
\begin{itemize}
	\item se $L$ \`e finito, $(a_n)_{n \in \mathbb{N}}$ converge a $L$;
	\item se $L$ \`e infinito, $(a_n)_{n \in \mathbb{N}}$ diverge. \EndProof
\end{itemize}
\begin{Theorem}
	Sia $(a_n)_{n \in \mathbb{N}} \in \mathbb{R}^\mathbb{N}$ una successione
  decrescente.
  $(a_n)_{n \in \mathbb{N}} \in \mathbb{R}^\mathbb{N}$ converge.
\end{Theorem} 
\Proof Consideriamo $L = \inf_{n \in \mathbb{N}} a_n$: segue direttamente dalle
definizioni che
\begin{itemize}
	\item se $L$ \`e finito, $(a_n)_{n \in \mathbb{N}}$ converge a $L$;
	\item se $L$ \`e infinito, $(a_n)_{n \in \mathbb{N}}$ diverge. \EndProof
\end{itemize}
\begin{Theorem}
	Le progressioni geometriche $(a^n)_{n \in \mathbb{N}}$ tali che
  $\AbsoluteValue{a} < 1$ convergono a $0$, mentre le progressioni geometriche
  tali che $\AbsoluteValue{a} > 1$ divergono.
\end{Theorem}
\Proof Quando $\AbsoluteValue{a} < 1$,
$(\AbsoluteValue{a}^n)_{n \in \mathbb{N}}$ \`e una successione decrescente;
quando invece $\AbsoluteValue{a} > 1$,
$(\AbsoluteValue{a}^n)_{n \in \mathbb{N}}$ \`e una successione crescente.
\begin{Theorem}
	\TheoremName{Teorema del confronto o dei carabinieri}[del confronto o dei carabinieri][teorema]
	Siano
  $(a_n)_{n \in \mathbb{N}},
  (b_n)_{n \in \mathbb{N}},
  (c_n)_{n \in \mathbb{N}} \in \mathbb{R}^\mathbb{N}$.
  Se abbiamo
	\begin{itemize}
		\item $\ForAll{n \in \mathbb{N}}{a_n \leq b_n \leq c_n}$;
		\item $\lim_{n \rightarrow \infty} a_n = \lim_{n \rightarrow \infty} c_n$;
	\end{itemize}
	allora
  $\lim_{n \rightarrow \infty} a_n =
  \lim_{n \rightarrow \infty} b_n =
  \lim_{n \rightarrow \infty} c_n$.
\end{Theorem}
\Proof Fissiamo $\epsilon > 0$ e poniamo
$L = \lim_{n \rightarrow \infty} a_n = \lim_{n \rightarrow \infty} c_n$.
Per $N \in \mathbb{N}$ abbastanza grande, abbiamo che $n > N$ implica
\begin{itemize}
	\item $L - \epsilon < a_n < L + \epsilon$;
	\item $L - \epsilon < c_n < L + \epsilon$.
\end{itemize}
Abbiamo altres\`i
$L - \epsilon < a_n \leq b_n \leq c_n < L + \epsilon$.
Dunque $\lim_{n \rightarrow \infty} b_n = L$.
\begin{Definition}
  Sia $(a_n)_{n \in \mathbb{N}} \in \mathbb{R}^\mathbb{N}$.
  Definiamo
  \begin{itemize}
    \item \Define{limite superiore}[superiore][limite]
      di $(a_n)_{n \in \mathbb{N}}$ il limite della successione
      $(\sup_{k \geq n} a_k)_{n \in \mathbb{N}}$,
      denotato $\limsup_{n \rightarrow \infty} a_n$;
    \item \Define{limite inferiore}[inferiore][limite]
      di $(a_n)_{n \in \mathbb{N}}$ il limite della successione
      $(\inf_{k \geq n} a_k)_{n \in \mathbb{N}}$,
      denotato $\liminf_{n \rightarrow \infty} a_n$.
  \end{itemize}
\end{Definition}
\begin{Theorem}
  Sia $(a_n)_{n \in \mathbb{N}} \in \mathbb{R}^\mathbb{N}$.
  Abbiamo
  \begin{align*}
    \limsup_{n \rightarrow \infty} a_n
    &= \inf_{n \in \mathbb{N}} \sup_{k \geq n} a_k,\\
    \liminf_{n \rightarrow \infty} a_n
    &= \sup_{n \in \mathbb{N}} \inf_{k \geq n} a_k,\\
    \liminf_{n \rightarrow \infty} a_n
    &\leq \limsup_{n \in \rightarrow \infty} a_n.
  \end{align*}
\end{Theorem}
\Proof La prima relazione Segue direttamente dal fatto che la successione
$(\sup_{k \in n} a_k)_{n \in \mathbb{N}}$
\`e decrescente.
\par La seconda relazione segue dal fatto che la successione
$(\inf_{k \in n} a_k)_{n \in \mathbb{N}}$ \`e
crescente.
\par Dimostriamo la terza relazione. Per ogni $n \in \mathbb{N}$, abbiamo
$\inf_{k \geq n} a_k \leq \sup_{k \in n} a_k$,
da cui
$\sup_{n \in \mathbb{N}} \inf_{k \geq n} a_k \leq \sup_{k \in n} a_k$
e quindi
$\sup_{n \in \mathbb{N}} \inf_{k \geq n} a_k
\leq \inf_{n \in \mathbb{N}} \sup_{k \in n} a_k$,
equivalente proprio a
$\liminf_{n \rightarrow \infty} a_n
\leq \limsup_{n \in \rightarrow \infty} a_n$. \EndProof
\begin{Theorem}
  Sia $(a_n)_{n \in \mathbb{N}} \in \mathbb{R}^\mathbb{N}$.
  La successione $(a_n)_{n \in \mathbb{N}}$
  \begin{itemize}
    \item converge a limite finito $L \in \mathbb{R}$ se e solo se
    $\liminf_{n \rightarrow \infty} a_n
      = \limsup_{n \rightarrow \infty} a_n = L$;
    \item diverge positivamente se e solo se
    $\liminf_{n \rightarrow \infty} a_n
      = \limsup_{n \rightarrow \infty} a_n = + \infty$;
    \item diverge negativamente se e solo se
    $\liminf_{n \rightarrow \infty} a_n
      = \limsup_{n \rightarrow \infty} a_n = - \infty$.
  \end{itemize}
\end{Theorem}
\Proof Supponiamo $(a_n)_{n \in \mathbb{N}}$ converga a $L \in \mathbb{R}$.
Fissato $\epsilon > 0$, esiste $n \in \mathbb{N}$ tale che
$\Implies{k \geq n}{L \leq a_k < L + \epsilon}$.
Ne deduciamo
$L \leq \sup_{k \geq n} a_k \leq L + \epsilon$.
Per l'arbitrariet\`a di $\epsilon$, deve essere
$\limsup_{n \rightarrow \infty} a_k = L$.
Analogamente
$\liminf_{n \rightarrow \infty} a_k = L$.
\par Supponiamo
$\liminf_{n \rightarrow \infty} a_n
  = \limsup_{n \rightarrow \infty} a_n = L$.
Poich\'e, per ogni $n \in \mathbb{N}$,
$\inf_{k \geq n} a_k \leq a_n \leq \sup_{k \in n} a_k$,
dal teorema del confronto segue
$\lim_{n \rightarrow \infty} a_n = L$.
\par Supponiamo $(a_n)_{n \in \mathbb{N}}$ diverga positivamente.
Fissato $M > 0$, esiste $n \in \mathbb{N}$ tale che
$\Implies{k \geq n}{a_k > M}$.
Ne deduciamo
$\inf_{k \geq n}  a_k > M$
e
$\sup_{k \geq n}  a_k > M$.
Per l'arbitrariet\`a di $M$, deve essere
$\liminf_{n \rightarrow \infty} a_n
  = \limsup_{n \rightarrow \infty} a_n = + \infty$.
\par Supponiamo
$\liminf_{n \rightarrow \infty} a_n
  = \limsup_{n \rightarrow \infty} a_n = + \infty$.
Per ogni $n \in \mathbb{N}$, abbiamo
$\inf_{k \geq n} a_k \leq a_n$, dunque
$\liminf_{n \rightarrow \infty} a_n \leq \lim_{n \rightarrow \infty} a_n$
e quindi
$\lim_{n \rightarrow \infty} a_n = + \infty$.
\par Supponiamo $(a_n)_{n \in \mathbb{N}}$ diverga negativamente.
Fissato $M > 0$, esiste $n \in \mathbb{N}$ tale che
$\Implies{k \geq n}{a_k < - M}$.
Ne deduciamo
$\inf_{k \geq n}  a_k < - M$
e
$\sup_{k \geq n}  a_k < - M$.
Per l'arbitrariet\`a di $M$, deve essere
$\liminf_{n \rightarrow \infty} a_n
  = \limsup_{n \rightarrow \infty} a_n = - \infty$.
\par Supponiamo
$\liminf_{n \rightarrow \infty} a_n
  = \limsup_{n \rightarrow \infty} a_n = - \infty$.
Per ogni $n \in \mathbb{N}$, abbiamo
$\sup_{k \geq n} a_k \geq a_n$, dunque
$\limsup_{n \rightarrow \infty} a_n \geq \lim_{n \rightarrow \infty} a_n$
e quindi
$\lim_{n \rightarrow \infty} a_n = - \infty$. \EndProof
\begin{Definition}
	Siano
	\begin{itemize}
		\item $(\VarTopologicalSpace,\VarTopology)$ spazio topologico;
		\item $\Part \subseteq \mathbb{R}$ tale che $\sup{\Part} = + \infty$;
		\item $f: \Part \rightarrow \VarTopologicalSpace$;
		\item $\Limit \in \VarTopologicalSpace$.
	\end{itemize}
	Se $\ForAll{\Neighborhood_\Limit \in \Neighborhoods{\Limit} \cap \VarTopology}{\Exists{N > 0}{\Implies{x > N}{f(x) \in \Neighborhood_\Limit}}}$, allora diciamo che $f(x)$ \Define{converge}[di una funzione a $+\infty$][convergenza] a $\Limit$, in simboli $f(x) \rightarrow \Limit$, per $x$ che tende a $+\infty$, in simboli $x \rightarrow +\infty$. Chiamiamo $\Limit$ \Define{limite}[di una funzione a $+\infty$][limite] di $f(x)$ per $x \rightarrow \infty$, denotato $\lim_{x \rightarrow +\infty} f(x) = \Limit$. In tal caso diciamo inoltre che la retta $y = \Limit$ \`e \Define{asintoto orizzontale positivo}[orizzontale positivo][asintoto] del grafico della funzione $y = f(x)$ nel piano $\OxyPlane$.
\end{Definition}
\begin{Definition}
	Siano
	\begin{itemize}
		\item $(\VarTopologicalSpace,\VarTopology)$ spazio topologico;
		\item $\Part \subseteq \mathbb{R}$ tale che $\inf{\Part} = - \infty$;
		\item $f: \Part \rightarrow \VarTopologicalSpace$;
		\item $\Limit \in \VarTopologicalSpace$.
	\end{itemize}
	Se $\ForAll{\Neighborhood_\Limit \in \Neighborhoods{\Limit} \cap \VarTopology}{\Exists{N > 0}{\Implies{x < -N}{f(x) \in \Neighborhood_\Limit}}}$, allora diciamo che $f(x)$ \Define{converge}[di una funzione a $-\infty$][convergenza] a $\Limit$, in simboli $f(x) \rightarrow \Limit$, per $x$ che tende a $-\infty$, in simboli $x \rightarrow -\infty$. Chiamiamo $\Limit$ \Define{limite}[di una funzione a $-\infty$][limite] di $f(x)$ per $x \rightarrow \infty$, denotato $\lim_{x \rightarrow -\infty} f(x) = \Limit$. In tal caso diciamo inoltre che la retta $y = \Limit$ \`e \Define{asintoto orizzontale negativo}[orizzontale negativo][asintoto] del grafico della funzione $y = f(x)$ nel piano $\OxyPlane$.
\end{Definition}
\begin{Theorem}
	Siano
	\begin{itemize}
		\item $(\VarTopologicalSpace,\VarTopology)$ spazio topologico;
		\item $\Part \subseteq \mathbb{R}$ tale che $\sup{\Part} = - \infty$;
		\item $f: \Part \rightarrow \VarTopologicalSpace$;
		\item $\Limit \in \VarTopologicalSpace$.
	\end{itemize}
	Abbiamo $\lim_{x \rightarrow +\infty} f(x) = \Limit$ per qualche $\Limit \in \VarTopologicalSpace$ se e solo se $\lim_{x \rightarrow -\infty} f(-x) = \Limit$,
\end{Theorem}
\Proof $\lim_{x \rightarrow +infty} f(x) = \Limit$ equivale a $\ForAll{\Neighborhood_\Limit \in \Neighborhoods{\Limit} \cap \VarTopology}{\Exists{N > 0}{\Implies{x > N}{f(x) \in \Neighborhood_\Limit}}}$, a sua volta equivalente a $\ForAll{\Neighborhood_\Limit \in \Neighborhoods{\Limit} \cap \VarTopology}{\Exists{N > 0}{\Implies{x < - N}{f(-x) \in \Neighborhood_\Limit}}}$ e quindi a $\lim_{x \rightarrow -\infty} f(-x) = \Limit$. \EndProof
\begin{Theorem}
	Siano
	\begin{itemize}
		\item $(\VarTopologicalSpace,\VarTopology)$ spazio topologico;
		\item una successione $(a_n)_{n \in \mathbb{N}} \in \mathbb{R}^\mathbb{N}$ divergente positivamente;
		\item $f: \bigcup_{n \in \mathbb{N}} \lbrace a_n \rbrace \rightarrow \VarTopologicalSpace$.
	\end{itemize}
	Se \`e vero l'assioma della scelta, $(f(a_n))_{n \in \mathbb{N}}$ converge a $\Limit$ se e solo se $f(x)$ converge a $\Limit$ per $x \rightarrow +\infty$.
\end{Theorem}
\Proof Supponiamo $f(x) \rightarrow \Limit$ per $x \rightarrow +\infty$. Sia $\Neighborhood_\Limit \in \Neighborhoods{\Limit} \cap \VarTopology$: per opportuno $N > 0$, abbiamo $\Implies{x > N}{f(x) \in \Neighborhood_\Limit}$. Ma $a_n \rightarrow +\infty$ per $n \rightarrow \infty$ e quindi esiste $N_0 \in \mathbb{N}$ tale che $\Implies{n > N_0}{a_n > N}$, da cui $\Implies{n > N_0}{f(a_n) \in \Neighborhood_\Limit}$. Ne deduciamo $f(a_n) \rightarrow \Limit$ per $n \rightarrow \infty$.
\par Supponiamo ora $\lim_{n \rightarrow \infty} f(a_n) = \Limit$. Supponiamo inoltre che $f(x)$ non converga a $\Limit$ per $x \rightarrow \infty$. Allora esiste $\Neighborhood_\Limit \in \Neighborhoods{\Limit} \cap \VarTopology$ tale che $\ForAll{N > 0}{\Exists{m \in \mathbb{N}}{\Implies{a_m > N}{f(a_m) \notin \Neighborhood_\Limit}}}$.
\par Possiamo quindi definire la sottosuccessione $(a_{g(n)})_{n \in \mathbb{N}} \in \mathbb{R}^{\mathbb{N}}$, con $g: \mathbb{N} \rightarrow \mathbb{N}$ applicazione strettamente crescente, tale che $\ForAll{n \in \mathbb{N}}{f(a_{g(n)}) \notin \Neighborhood_\Limit}$. Allora
\begin{itemize}
	\item $(f(a_{g(n)}))_{n \in \mathbb{N}}$ converge a $\Limit$ in quanto sottosuccessione di $(f(a_n))_{n \in \mathbb{N}}$;
	\item $(f(a_{g(n)}))_{n \in \mathbb{N}}$ non converge a $\Limit$ perch\'e $\ForAll{n \in \mathbb{N}}{f(a_{g(n)}) \notin \Neighborhood_\Limit}$.
\end{itemize}
Ma questo \`e assurdo, dunque $f(x)$ deve converge a $\Limit$ per $x \rightarrow +\infty$. \EndProof
\begin{Corollary}
	Siano
	\begin{itemize}
		\item $(\VarTopologicalSpace,\VarTopology)$ spazio topologico;
		\item una successione $(a_n)_{n \in \mathbb{N}} \in \mathbb{R}^\mathbb{N}$ divergente negativamente;
		\item $f: \bigcup_{n \in \mathbb{N}} \lbrace a_n \rbrace \rightarrow \VarTopologicalSpace$.
	\end{itemize}
	Se \`e vero l'assioma della scelta, $(f(a_n))_{n \in \mathbb{N}}$ converge a $\Limit$ se e solo se $f(x)$ converge a $\Limit$ per $x \rightarrow -\infty$.
\end{Corollary}
\Proof Segue dal teorema precedente, moltiplicando per $(-1)$ la successione $a_n$ e sostituendo a $f(x)$ la funzione $f(-x)$. \EndProof
\begin{Definition}
	Siano
	\begin{itemize}
		\item $\Part \subseteq \mathbb{R}$ tale che $\sup{\Part} = + \infty$;
		\item $f: \Part \rightarrow \mathbb{R}$.
	\end{itemize}
	Se $\ForAll{M > 0}{\Exists{N > 0}{\Implies{x > N}{f(x) > M}}}$, allora diciamo che $f(x)$ \Define{diverge}[positia di una funzione a $+\infty$][divergenza] a $+\infty$, in simboli $f(x) \rightarrow +\infty$, per $x$ che tende a $+\infty$, in simboli $x \rightarrow +\infty$; diciamo anche che $+\infty$ \`e il \Define{limite}[infinito positivo di una funzione a $+\infty$][limite], denotato $\lim_{x \rightarrow +\infty} f(x) = +\infty$.
	\par Se invece $\ForAll{M > 0}{\Exists{N > 0}{\Implies{x > N}{f(x) < - M}}}$, allora diciamo che $f(x)$ \Define{diverge}[negativa di una funzione a $+\infty$][divergenza] a $-\infty$, in simboli $f(x) \rightarrow -\infty$, per $x$ che tende a $+\infty$, in simboli $x \rightarrow +\infty$; diciamo anche che $-\infty$ \`e il \Define{limite}[infinito negativo di una funzione a $+\infty$][limite], denotato $\lim_{x \rightarrow +\infty} f(x) = -\infty$.
\end{Definition}
\begin{Definition}
	Siano
	\begin{itemize}
		\item $\Part \subseteq \mathbb{R}$ tale che $\sup{\Part} = - \infty$;
		\item $f: \Part \rightarrow \mathbb{R}$.
	\end{itemize}
	Se $\ForAll{M > 0}{\Exists{N > 0}{\Implies{x < - N}{f(x) > M}}}$, allora diciamo che $f(x)$ \Define{diverge}[positiva di una funzione a $-\infty$][divergenza] a $+\infty$, in simboli $f(x) \rightarrow +\infty$, per $x$ che tende a $-\infty$, in simboli $x \rightarrow -\infty$; diciamo anche che $+\infty$ \`e il \Define{limite}[infinito positivo di una funzione a $-\infty$][limite], denotato $\lim_{x \rightarrow -\infty} f(x) = +\infty$.
	\par Se invece $\ForAll{M > 0}{\Exists{N > 0}{\Implies{x < - N}{f(x) < - M}}}$, allora diciamo che $f(x)$ \Define{diverge}[negativa di una funzione a $-\infty$][divergenza] a $-\infty$, in simboli $f(x) \rightarrow -\infty$, per $x$ che tende a $-\infty$, in simboli $x \rightarrow -\infty$; diciamo anche che $-\infty$ \`e il \Define{limite}[infinito negativo di una funzione a $-\infty$][limite], denotato $\lim_{x \rightarrow -\infty} f(x) = -\infty$.
\end{Definition}
\begin{Theorem}
	Siano
	\begin{itemize}
		\item $\Part \subseteq \mathbb{R}$ tale che $\sup{\Part} = - \infty$;
		\item $f: \Part \rightarrow \mathbb{R}$.
	\end{itemize}
	Sono equivalenti
	\begin{itemize}
		\item $\lim_{x \rightarrow +\infty} f(x) = +\infty$;
		\item $\lim_{x \rightarrow -\infty} f(-x) = +\infty$;
		\item $\lim_{x \rightarrow +\infty} -f(x) = -\infty$;
		\item $\lim_{x \rightarrow -\infty} f(-x) = -\infty$.
	\end{itemize}
\end{Theorem}
\Proof $\lim_{x \rightarrow +infty} f(x) = +\infty$ equivale a $\ForAll{M > 0}{\Exists{N > 0}{\Implies{x > N}{f(x) < M}}}$. Quest'ultima affermazione \`e equivalente a
\begin{itemize}
	\item $\ForAll{M > 0}{\Exists{N > 0}{\Implies{x < - N}{f(-x) > M}}}$ e quindi a $\lim_{x \rightarrow -\infty} f(-x) = +\infty$;
	\item $\ForAll{M > 0}{\Exists{N > 0}{\Implies{x > N}{- f(x) < - M}}}$ e quindi a $\lim_{x \rightarrow +\infty} - f(x) = -\infty$;
	\item $\ForAll{M > 0}{\Exists{N > 0}{\Implies{x > N}{- f(-x) < - M}}}$ e quindi a $\lim_{x \rightarrow -\infty} - f(x) = -\infty$. \EndProof
\end{itemize}
\begin{Theorem}
	Siano
	\begin{itemize}
		\item una successione $(a_n)_{n \in \mathbb{N}} \in \mathbb{R}^\mathbb{N}$ divergente positivamente;
		\item $f: \bigcup_{n \in \mathbb{N}} \lbrace a_n \rbrace \rightarrow \mathbb{R}$.
	\end{itemize}
	Se \`e vero l'assioma della scelta, $(f(a_n))_{n \in \mathbb{N}}$ diverge positivamente se e solo se $f(x)$ diverge positivamente per $x \rightarrow +\infty$.
\end{Theorem}
\Proof Supponiamo $f(x) \rightarrow +\infty$ per $x \rightarrow +\infty$. Sia $M > 0$: per opportuno $N > 0$, abbiamo $\Implies{x > N}{f(x) > M}$. Ma $a_n \rightarrow +\infty$ per $n \rightarrow \infty$ e quindi esiste $N_0 \in \mathbb{N}$ tale che $\Implies{n > N_0}{a_n > N}$, da cui $\Implies{n > N_0}{f(a_n) < M}$. Ne deduciamo $f(a_n) \rightarrow +\infty$ per $n \rightarrow +\infty$.
\par Supponiamo ora $\lim_{n \rightarrow \infty} f(a_n) = +\infty$. Supponiamo inoltre che $f(x)$ non diverga positivamente per $x \rightarrow \infty$. Allora esiste $M > 0$ tale che $\ForAll{N > 0}{\Exists{m \in \mathbb{N}}{\Implies{a_m > N}{f(a_m) \leq M}}}$.
\par Possiamo quindi definire la sottosuccessione $(a_{g(n)})_{n \in \mathbb{N}} \in \mathbb{R}^{\mathbb{N}}$, con $g: \mathbb{N} \rightarrow \mathbb{N}$ applicazione strettamente crescente, tale che $\ForAll{n \in \mathbb{N}}{f(a_{g(n)}) \leq M}$. Allora
\begin{itemize}
	\item $(f(a_{g(n)}))_{n \in \mathbb{N}}$ diverge positivamente in quanto sottosuccessione di $(f(a_n))_{n \in \mathbb{N}}$;
	\item $(f(a_{g(n)}))_{n \in \mathbb{N}}$ non diverge positivamente perch\'e $\ForAll{n \in \mathbb{N}}{f(a_{g(n)}) \leq M}$.
\end{itemize}
Ma questo \`e assurdo, dunque $f(x)$ diverge positivamente per $x \rightarrow +\infty$. \EndProof
\begin{Corollary}
	Siano
	\begin{itemize}
		\item una successione $(a_n)_{n \in \mathbb{N}} \in \mathbb{R}^\mathbb{N}$ divergente negativamente;
		\item $f: \bigcup_{n \in \mathbb{N}} \lbrace a_n \rbrace \rightarrow \mathbb{R}$.
	\end{itemize}
	Se \`e vero l'assioma della scelta, $(f(a_n))_{n \in \mathbb{N}}$ diverge positivamente se e solo se $f(x)$ diverge positivamente per $x \rightarrow -\infty$.
\end{Corollary}
\Proof Segue dal teorema precedente, moltiplicando per $(-1)$ la successione $a_n$ e sostituendo a $f(x)$ la funzione $f(-x)$. \EndProof
\begin{Corollary}
	Siano
	\begin{itemize}
		\item una successione $(a_n)_{n \in \mathbb{N}} \in \mathbb{R}^\mathbb{N}$ divergente positivamente;
		\item $f: \bigcup_{n \in \mathbb{N}} \lbrace a_n \rbrace \rightarrow \mathbb{R}$.
	\end{itemize}
	Se \`e vero l'assioma della scelta, $(f(a_n))_{n \in \mathbb{N}}$ diverge negativamente se e solo se $f(x)$ diverge negativamente per $x \rightarrow +\infty$.
\end{Corollary}
\Proof Segue dal teorema precedente, sostituendo a $f(x)$ la funzione $-f(x)$. \EndProof
\begin{Corollary}
	Siano
	\begin{itemize}
		\item una successione $(a_n)_{n \in \mathbb{N}} \in \mathbb{R}^\mathbb{N}$ divergente negativamente;
		\item $f: \bigcup_{n \in \mathbb{N}} \lbrace a_n \rbrace \rightarrow \mathbb{R}$.
	\end{itemize}
	Se \`e vero l'assioma della scelta, $(f(a_n))_{n \in \mathbb{N}}$ diverge negativamente se e solo se $f(x)$ diverge negativamente per $x \rightarrow -\infty$.
\end{Corollary}
\Proof Segue dal teorema precedente, sostituendo a $f(x)$ la funzione $-f(-x)$. \EndProof
\begin{Definition}
  Dati $x, n \in \mathbb{N}$
  \begin{itemize}
    \item \Define{fattoriale crescente}[crescente][fattoriale]
      di $x$ con $n$ fattori, denotato
      $\RisingFactorialBar{x}{n}$ o anche
      $\RisingFactorialPochammer{x}{n}$ o ancora
      $\RisingFactorialSupscript{x}{n}$,
      la quantit\`a
      $\RisingFactorialBar{x}{n} =
      \begin{cases}
        0,&\text{ se }n = 0,\\
        \prod_{k \in n} (x + k),&\text{ se }n = 0;
      \end{cases}$
    \item \Define{fattoriale decrescente}[decrescente][fattoriale]
      di $x$ con $n$ fattori, denotato
      $\FallingFactorialBar{x}{n}$ o anche
      $\FallingFactorialPochammer{x}{n}$,
      la quantit\`a
      $\FallingFactorialBar{x}{n} =
      \begin{cases}
        0,&\text{ se }n = 0,\\
        \prod_{k \in n} (x - k),&\text{ se }n = 0;
      \end{cases}$.
  \end{itemize}
  Si osservi che la notazione $\FactorialPochammer{x}{n}$, detta
  \Define{simbolo di Pochammer}[di Pochammer][simbolo],
  \`e utilizzata sia per il fattoriale crescente che per quello
  decrescente.
\end{Definition}
\begin{Theorem}
  \TheoremName{Formula di Faulhaber}[di Faulhaber][formula]
  Siano $n \in \mathbb{N}$ e $p \in \NotZero{\mathbb{N}}$. Abbiamo
  \[
    \sum_{k = 1}^n =
    \frac{n^{p + 1}}{p + 1} + \frac{n^p}{2}
    + \sum_{k = 2}^p \frac{\BernoulliNumber_k}{k!} \FallingFactorial{p}{k - 1}
      n^{p - k + 1}.
  \]
\end{Theorem}
\begin{Theorem}
  Sia $n \in \mathbb{N}$. Abbiamo
  \begin{itemize}
    \item $\sum_{k \in n} k^2 = \frac{n(n - 1)(2n - 1)}{6}$;
    \item $\sum_{k \in n} k^3 = \frac{n^2(n - 1)^2}{4}$.
  \end{itemize}
\end{Theorem}

\section{Serie in spazi vettoriali normati.}
\label{SuccessioniESerie_SerieInSpaziVettorialiNormati}
\begin{Definition}
	Sia $\NormedSpace$ un $\Field$-spazio vettoriale normato.
	Data una successione $(a_i)_{i \in \mathbb{N}} \in \mathbb{\NormedSpace}^\mathbb{N}$, diciamo che la successione $\left ( \sum_{i = 0}^N a_i \right )_{N \in \mathbb{N}}$, denotata pi\`u brevemente $\sum_{i \in \mathbb{N}} a_i$ o anche $\sum_{i = 0}^\infty a_i$, \`e una \Define{serie}[in uno spazio vettoriale normato][serie] in $\NormedSpace$; chiamiamo inoltre
	\begin{itemize}
		\item \Define{somme parziali}[parziale][somma] gli elementi di tale successione;
		\item \Define{somma della serie}[di una serie][somma] il limite della serie, cio\`e della succcessione delle somme parziali;
		\item \Define{termini della serie}[di una serie][termine] gli elementi della successione originale $(a_i)_{i \in \mathbb{N}}$;
		\item \Define{successione associata a $\sum_{i \in \mathbb{N}} a_i$}[associata a una serie][successione] o \Define{successione dei termini di $\sum_{i \in \mathbb{N}} a_i$}[dei termini][successione] la successione originale $(a_i)_{i \in \mathbb{N}}$.
	\end{itemize}
	Definiamo la serie \Define{complessa}[complessa][serie], \Define{reale}[reale][serie] ecc, a seconda che la successione dei suoi termini sia una successione complessa, reale ecc. rispettivamente.
\end{Definition}
\begin{Definition}
	Chiamiamo \Define{$\Field$-spazio di Banach}[di Banach][spazio] un $\Field$-spazio vettoriale normato completo.
\end{Definition}
\begin{Definition}
	Sia $\BanachSpace$ un $\Field$-spazio di Banach.
	Sia $\sum_{i \in \mathbb{N}} a_i$ una serie in $\BanachSpace$. $\sum_{i \in \mathbb{N}} a_i$ si dice \Define{assolutamente convegente}[assolutamente convergente][serie] quando $\sum_{i \in \mathbb{N}} \Norm{a_i}$ converge.
\end{Definition}
\begin{Theorem}
	\TheoremName{Criterio di convergenza di Cauchy}[di convergenza di Cauchy][criterio]
	Siano $\BanachSpace$ un $\Field$-spazio di Banach e $\sum_{n \in \mathbb{N}} a_n$ una serie in $\BanachSpace$. $\sum_{i \in \mathbb{N}} a_i$ \`e convergente se e solo se $\ForAll{\epsilon > 0}{\Exists{N \in \mathbb{N}}{\ForAll{p \in \mathbb{N}}{\AbsoluteValue{\sum_{n = 0}^p a_{N + n}} < \epsilon}}}$.
\end{Theorem}
\Proof Segue direttamente dalle definizioni. \EndProof
\begin{Corollary}
	Siano $\BanachSpace$ un $\Field$-spazio vettoriale normato e sia $\sum_{n \in \mathbb{N}} a_n$ una serie in $\BanachSpace$. $(a_n)_{n \in \mathbb{N}}$ \`e una successione infinitesima.
\end{Corollary}
\Proof Segue direttamente dal criterio di convergenza di Cauchy. \EndProof
\begin{Corollary}
	Sia $\sum_{n \in \mathbb{N}} a_n$ una serie assolutamente convergente: $\sum_{n \in \mathbb{N}} a_n$ converge anche semplicmente.
\end{Corollary}
\Proof Fissiamo $\epsilon > 0$. Esiste $N \in \mathbb{N}$ tale che, per ogni $p \in \mathbb{N}$, $\Norm{\sum_{n = 0}^p \Norm{a_{N + p}}} < \epsilon$, da cui $\sum_{n = 0}^p \Norm{a_{N + p}} < \epsilon$ e quindi $\Norm{\sum_{n = 0}^p a_{N + p}} < \epsilon$. Per il criterio di convergenza di Cauchy, $\sum_{n \in \mathbb{N}} a_n$ converge. \EndProof
\begin{Definition}
	Chiamiamo \Define{serie aritmetica}[aritmetica][serie] la serie i cui termini costituiscono una progressione aritmetica.
\end{Definition}
\begin{Theorem}
	Sia $\sum_{n = 0}^\infty a_n$ una serie aritmetica. Abbiamo
	\begin{itemize}
		\item fissato $N \in \mathbb{N}$, $\sum_{n = 0}^N a_n = \frac{(N + 1)(a_0 + a_N)}{2}$;
		\item $\sum_{n = 0}^\infty a_n$ \`e divergente.
	\end{itemize}
\end{Theorem}
\Proof Sia $K = a_1 - a_0$ la ragione della progressione aritmetica. Abbiamo $\ForAll{n \in \mathbb{N}}{a_n = a_0 + nK}$. Inoltre, per ogni $q \in N + 1$, $a_q + a_{N - q} = a_0 + qK + a_0 + NK - qK = 2 a_0 + NK = a_0 + a_N$.
\par Ora, se $N$ \`e pari, allora $\sum_{n = 0}^N a_n = a_{\frac{N}{2}} + \sum_{n = 0}^{\frac{N}{2} - 1} (a_0 + a_N) = a_0 + \frac{N}{2}K + \frac{N}{2} (a_0 + a_N) = \frac{a_0 + a_n}{2} + \frac{N}{2}(a_0 + a_N) = \frac{(N + 1)(a_0 + a_N)}{2}$.
\par Se invece $N$ \`e dispari, allora $\sum_{n = 0}^N a_n = \sum_{n = 0}^{\frac{N + 1}{2}} a_n = \frac{N + 1}{2}(a_0 + a_n)$.
\par Infine, la serie \`e divergente per il criterio di convergenza di Cauchy. \EndProof
\begin{Corollary}
	Sia $\sum_{n = 1}^\infty a_n$ una serie aritmetica. Abbiamo, fissato $N \in \mathbb{N}$, $\sum_{n = 1}^N a_n = \frac{N(a_1 + a_N)}{2}$.
\end{Corollary}
\Proof Abbiamo $\sum_{n = 1}^\infty a_n = \sum_{n = 0}^\infty a_{n + 1}$, da cui $\sum_{n = 1}^N a_n = \sum_{n = 0}^{N - 1} a_{n + 1} = \frac{N(a_1 + a_N)}{2}$. \EndProof

\section{Serie complesse.}\label{SerieComplesse}
\begin{Definition}
	Chiamiamo \Define{serie geometrica}[geometrica][serie] la serie i cui termini costituiscono una progressione geometrica.
\end{Definition}
\begin{Theorem}
	Sia $\sum_{n = 0}^\infty a^n$ una serie geometrica di ragione $a$. Abbiamo
	\begin{itemize}
		\item fissato $N \in \mathbb{N}$, abbiamo $\sum_{n = 0}^N a^n = \frac{1 - a^{N + 1}}{1 - a}$;
		\item $\sum_{n = 0}^\infty a^n$ \`e convergente per $\AbsoluteValue{a} < 1$ e abbiamo $\sum_{n = 0}^\infty a^n = \frac{1}{1 - a}$.
		\item $\sum_{n = 0}^\infty a^n$ \`e divergente per $\AbsoluteValue{a} > 1$.
	\end{itemize}
\end{Theorem}
\Proof Abbiamo $(1 - a) \sum_{n = 0}^N a^N = 1 - a^N$, da cui $\sum_{n = 0}^N a^N = \frac{1 - a^N}{1 - a}$.
\par Se $\AbsoluteValue{a} < 1$, allora $\lim_{n \rightarrow \infty} a^n = 0$ e dunque, per $N \rightarrow \infty$, $\frac{1 - a^N}{1 - a} \rightarrow \frac{1}{1 - a}$, da cui $\sum_{n = 0}^\infty$ converge a $\frac{1}{1 - a}$.
\par Se invece $\AbsoluteValue{a} > 1$, allora $(\AbsoluteValue{a^n})_{n \in \mathbb{N}}$ diverge e quindi diverge anche $\left ( \frac{1 - a^N}{1 - a} \right )_{N \in \mathbb{N}} = \left ( \sum_{n = 0}^N a_n \right )_{N \in \mathbb{N}}$. Dunque $\sum_{n = 0}^\infty a_n$ diverge. \EndProof
\begin{Definition}
	Sia $\sum_{n = 0}^\infty a_n$ una serie complessa tale che esista una successione $(b_n)_{n \in \mathbb{N}}$ tale che $\ForAll{n \in \mathbb{N}}{a_n = b_n - b_{n + 1}}$. $\sum_{n = 0}^\infty a_n$ si chiama \Define{serie telescopica}[telescopica][serie].
\end{Definition}
\begin{Theorem}
	Sia $\sum_{n = 0}^\infty a_n$ una serie complessa telescopica e $(b_n)_{n \in \mathbb{N}} \in \mathbb{C}^\mathbb{N}$ tale che $\ForAll{n \in \mathbb{N}}{a_n = b_n - b_{n + 1}}$. Per ogni $N \in \mathbb{N}$, abbiamo $\sum_{n = 0}^N a_n = b_0 - b_N$.
\end{Theorem}
\Proof Segue direttamente per induzione. \EndProof

\section{Serie reali.}\label{SerieReali}
\begin{Definition}
	$\sum_{n = 1}^\infty \frac{1}{n(n + 1)}$ si chiama \Define{serie di Mengoli}[di Mengoli][serie].
\end{Definition}
\begin{Theorem}
	La serie di Mengoli \`e una serie telescopica convergente a $1$.
\end{Theorem}
\Proof Abbiamo $\frac{1}{n(n + 1)} = \frac{1}{n} - \frac{1}{n + 1}$. Abbiamo pertanto, per ogni $N \in \mathbb{N}$ tale che $N > 2$, $\sum_{n = 1}^N \frac{1}{n(n + 1)} = 1 - \frac{1}{N + 1}$; quindi $\sum_{n = 0}^N \frac{1}{n(n + 1)} = 1$. \EndProof
\begin{Definition}
	Chiamiamo \Define{serie armonica}[armonica][serie] la serie $\sum_{n \in \NotZero{\mathbb{N}}} n^{-1}$. Fissato $\alpha > 0$, chiamiamo \Define{serie armonica generalizzata}[armonica generalizzata][serie] la serie $\sum_{n \in \NotZero{\mathbb{N}}} n^{-\alpha}$.
\end{Definition}
\begin{Theorem}
	La serie armonica \`e divergente.
\end{Theorem}
\Proof Fissiamo $N \in \NotZero{\mathbb{N}}$. Abbiamo $\sum_{n = 1}^{2N} n^{-1} - \sum_{n = 1}^N n^{-1} = \sum_{n = N + 1}^{2N} n^{-1} \geq N (2N)^{-1} = 2^{-1}$. Quindi la serie diverge per il criterio di convergenza di Cauchy. \EndProof
\begin{Theorem}
	\TheoremName{Criterio del confronto} Siano $\sum_{i \in \mathbb{N}} a_i$ e $\sum_{i \in \mathbb{N}} b_n$ serie reali a termini positivi o nulli tali che $\ForAll{i \in \mathbb{N}}{a_i \leq b_i}$. Abbiamo
	\begin{itemize}
		\item se $\sum_{i \in \mathbb{N}} b_i$ converge, allora converge anche $\sum_{i \in \mathbb{N}} a_i$;
		\item se $\sum_{i \in \mathbb{N}} a_i$ diverge, allora diverge anche $\sum_{i \in \mathbb{N}} b_i$.
	\end{itemize}
\end{Theorem}
\Proof Supponiamo $\sum_{i \in \mathbb{N}} b_i$ converga a $B$. Allora, per ogni $N \in \mathbb{N}$, $\sum_{i \in N} a_i \leq \sum_{i \in N} b_i \leq B$. Dunque la successione delle somme parziali della serie $\sum_{i \in \mathbb{N}} a_i$ \`e crescente e limitata, e pertanto convergente.
\par La seconda affermazione \`e equivalente alla prima. \EndProof
\begin{Theorem}
	La serie armonica generalizzata $\sum_{n = 1}^\infty n^{-\alpha}$ \`e
	\begin{itemize}
		\item convergente per $\alpha > 1$;
		\item divergente per $\alpha \leq 1$.
	\end{itemize}
\end{Theorem}
\Proof Il risultato \`e gi\`a stato dimostrato per $\alpha = 1$.
\par Assumiamo $\alpha < 1$. Per ogni $n \in \NotZero{\mathbb{N}}$, abbiamo $n^{-\alpha} > n^{-1}$, da cui la divergenza di $\sum_{n = 1}^\infty n^{-\alpha}$ per il criterio del confronto.
\par Assumiamo $\alpha = 2$. Abbiamo, per ogni $n \in \mathbb{N}$ tale che $n \geq 2$, $\frac{1}{n^2} \leq \frac{1}{n(n - 1)}$. Quindi, per il criterio del confronto con la serie di Mengoli, $\sum_{n = 2}^\infty n^{-2}$ e dunque anche $\sum_{n = 1}^\infty n^{-2}$ converge.
\par Assumiamo adesso $\alpha > 2$. Per ogni $n \in \NotZero{\mathbb{N}}$, abbiamo $n^{- alpha} \leq n^{- 2}$ e quindi, per il criterio del confronto, $\sum_{n = 1}^\infty a^{-alpha}$ converge.
\begin{Theorem}
	\TheoremName{Criterio del rapporto} Sia $\sum_{i \in \mathbb{N}} a_i$ una serie a termini positivi. Se esistono $\lambda < 1$ e $N \in \mathbb{N}$ tali che $\Implies{i > N}{\frac{a_{i + 1}}{a_i} \leq \lambda}$, allora $\sum_{i \in \mathbb{N}} a_i$ converge.
\end{Theorem}
\Proof Siano $\lambda$ e $N$ come nelle ipotesi del teorema. Abbiamo, per $i > N$, $a_i = a_i \prod_{k = N}^{i - 1}\frac{a_{i + 1}}{a_i} \leq a_N \lambda^{i - N}$, che converge per il criterio del confronto con la serie armonica generalizzata. \EndProof
\begin{Theorem}
	\TheoremName{Criterio del confronto asintotico} Siano $\sum_{i \in \mathbb{N}} a_i$ una serie reale a termini non negativi e $\sum_{i \in \mathbb{N}} b_i$ una serie reale a termini positivi. Abbiamo
	\begin{itemize}
		\item se $\sum_{i \in \mathbb{N}} b_i$ converge e $(a_ib_i^{-1})_{i \in \mathbb{N}}$ converge, allora $\sum_{i \in \mathbb{N}} a_i$ converge;
		\item se $\sum_{i \in \mathbb{N}} b_i$ diverge e $(a_ib_i^{-1})_{i \in \mathbb{N}}$ converge a un limite strettamente positivo o diverge, allora $\sum_{i \in \mathbb{N}} a_i$ diverge.
	\end{itemize}
\end{Theorem}
\Proof Assumiamo che $(a_ib_i^{-1})_{i \in \mathbb{N}}$ converga e poniamo $L = \lim_{i \rightarrow \infty} a_ib_i^{-1}$.
\par Se $L \neq 0$, fissiamo $\epsilon \in ]0,L[$. Per $N \in \mathbb{N}$ opportuno, abbiamo, per ogni intero $i > N$, $\AbsoluteValue{a_ib_i^{-1} - L} < \epsilon$, equivalente a $(L - \epsilon)b_i < a_i < (L + \epsilon)b_i$. Ora, se $\sum_{i \in mathbb{N}} b_i$ converge, allora converge anche $\sum_{i \in \mathbb{N}} (L + \epsilon)b_i$ e quindi, per il criterio del confronto anche $\sum_{i \in \mathbb{N}} a_i$. Se invece $\sum_{i \in \mathbb{N}} b_i$ diverge, allora diverge anche $\sum_{i \in \mathbb{N}} (L - \epsilon)b_i$ e quindi, per il criterio del confronto, anche $\sum_{i \in \mathbb{N}} a_i$.
\par Se invece $L = 0$, allora, fissato $\epsilon > 0$, esiste $N \in \mathbb{N}$ tale che per ogni $i > N$ abbiamo $a_ib_i^{-1} < \epsilon$, equivalente a $a_i < b_i \epsilon$. Dunque, se $\sum_{i \in \mathbb{N}} b_i$ convervge, la serie $\sum_{i = N + 1}^\infty a_i$ converge per il criterio del confronto e quindi anche $\sum_{i \in \mathbb{N}} a_i$.
\par Infine, se $(a_ib_i^{-1})_{i \in \mathbb{N}}$ diverge, allora, fissato $M > 0$, esiste $N \in \mathbb{N}$ tale che per ogni $i > N$ abbiamo $a_ib_i^{-1} > M$, equivalente a $a_i > b_iM$. Dunque, se $\sum_{i \in \mathbb{N}} b_i$ diverge, la serie $\sum_{i = N + 1}^\infty a_i$ diverge per il criterio del confronto e quindi anche $\sum_{i \in \mathbb{N}} a_i$.\EndProof

\section{Serie di potenze.}\label{SerieDiPotenze}
\begin{Theorem}
	Sia $\sum_{n = 0}^\infty a_n X^n \in \PowerSeries{\mathbb{C}}{X}$ e sia $z \in \mathbb{C}$. Abbiamo
	\begin{itemize}
		\item se $\sum_{n = 0}^\infty a_n z^n$ converge, allora per ogni $w \in \mathbb{C}$ tale che $\AbsoluteValue{w} < \AbsoluteValue{z}$, $\sum_{n = 0}^\infty a_n w^n$ converge assolutamente;
		\item se $\sum_{n = 0}^\infty a_n z^n$ diverge, allora per ogni $w \in \mathbb{C}$ tale che $\AbsoluteValue{w} > \AbsoluteValue{z}$, diverge anche $\sum_{n = 0}^\infty a_n w^n$.
	\end{itemize}
\end{Theorem}
\Proof Supponiamo $\sum_{n = 0}^\infty a_n z^n$ converga. Siano $N \in \mathbb{N}$ e $M > 0$ tali che $\ForAll{n > N}{\AbsoluteValue{a_n z^n} < M}$. Abbiamo $\AbsoluteValue{a_n w^n} = \AbsoluteValue{a_n} \AbsoluteValue{w}^n \frac{\AbsoluteValue{z}^n}{\AbsoluteValue{z}^n} = \AbsoluteValue{a_n z^n} \AbsoluteValue{\frac{w}{z}}^n < M \AbsoluteValue{\frac{w}{z}}^n$. La convergenza assoluta della serie $\sum_{n = N + 1}^\infty a_n w^n$, e dunque della serie $\sum_{n = 0}^\infty a_n w^n$ segue dunque per confronto con la serie geometrica $\sum_{n = N + 1}^\infty M \AbsoluteValue{\frac{w}{z}}^n$ di ragione $\AbsoluteValue{\frac{w}{z}} < 1$.
\par La seconda parte del teorema pu\`o essere dimostrata in maniera analoga o anche per assurdo osservando che se $\sum_{n = 0}^\infty a_n w^n$ convergesse, allora per la prima parte del teorema deve converge anche $\sum_{n = 0}^\infty a_n z^n$. \EndProof
\begin{Definition}
	Sia $\sum_{n = 0}^\infty a_n X^n \in \PowerSeries{\mathbb{C}}{X}$ e sia $z \in \mathbb{C}$. Chiamiamo
	\begin{itemize}
		\item \Define{raggio di convergenza}[di convergenza][raggio] di $\sum_{n = 0}^\infty a_n X^n$ l'estremo superiore dei moduli di tutti i punti $z \in \mathbb{C}$ tali che $\sum_{n = 0}^\infty a_n z^n$ converge;
		\item \Define{cerchio di convergenza}[di convergenza][cerchio] di $\sum_{n = 0}^\infty a_n X^n$ la palla aperta $\Open{0}{\RadiusOfConvergence}$, dove $\RadiusOfConvergence$ \`e il raggio di convergenza di $\sum_{n = 0}^\infty a_n X^n$.
	\end{itemize}
\end{Definition}
\begin{Theorem}
	\TheoremName{Teorema di Cauchy-Hadamard}[di Cauchy-Hadamard][teorema]
	Sia $\sum_{n = 0}^\infty a_n X^n \in \PowerSeries{\mathbb{C}}{X}$ e denotiamo $\RadiusOfConvergence$ il suo raggio di convergenza. Abbiamo
	\[
		\RadiusOfConvergence^{-1}
    = \limsup_{n \rightarrow \infty} \AbsoluteValue{a_n}^\frac{1}{n}.
	\]
\end{Theorem}
\begin{Theorem}
	\label{SuccessioniESerie_ConvergenzaDerivataFormale}
	Sia $\sum_{n = 0}^\infty a_n X^n \in \PowerSeries{\mathbb{C}}{X}$ e sia $z \in \mathbb{C}$. Se $z$ appartiene al cerchio di convergenza di  $\sum_{n = 0}^\infty a_n X^n$, allora appartiene anche al cerchio di convergenza della derivata formale $\sum_{n = 1}^\infty n a_n X^{n - 1} \in \PowerSeries{\mathbb{C}}{X}$.
\end{Theorem}

\section{Notazione asintotica (o di Landau).}
\label{TeoriaDelleSuccessioniEDelleSerie_NotazioneAsintotica}
\begin{Definition}
	\TheoremName{Notazione asintotica (o di Landau)} Siano $(a_n)_{n \in \mathbb{N}}, (b_n)_{n \in \mathbb{N}} \in \mathbb{\NormedSpace}^\mathbb{N}$, dove $\NormedSpace$ \`e un $\Field$-spazio vettoriale normato. Diciamo che
	\begin{itemize}
		\item $(a_n)_{n \in \mathbb{N}}$ \`e un \Define{O-grande} di $(b_n)_{n \in \mathbb{N}}$, denotato $\IsBigO{(a_n)_{n \in \mathbb{N}}}{(b_n)_{n \in \mathbb{N}}}$, quando $$\Exists{C \in \RealPositive}{\Exists{N \in \mathbb{N}}{\Implies{n > N}{\Norm{a_n} \leq C \Norm{b_n}}}};$$
		\item $(a_n)_{n \in \mathbb{N}}$ \`e un \Define{$\Omega$-grande} di $(b_n)_{n \in \mathbb{N}}$, denotato $\IsBigOmega{(a_n)_{n \in \mathbb{N}}}{(b_n)_{n \in \mathbb{N}}}$, quando $$\Exists{C \in \RealPositive}{\Exists{N \in \mathbb{N}}{\Implies{n > N}{\Norm{a_n} \geq C \Norm{b_n}}}};$$
		\item $(a_n)_{n \in \mathbb{N}}$ \`e un \Define{$\Theta$-grande} di $(b_n)_{n \in \mathbb{N}}$, denotato $\IsBigTheta{(a_n)_{n \in \mathbb{N}}}{(b_n)_{n \in \mathbb{N}}}$, quando $\IsBigO{(a_n)_{n \in \mathbb{N}}}{(b_n)_{n \in \mathbb{N}}}$ e $\IsBigOmega{(a_n)_{n \in \mathbb{N}}}{(b_n)_{n \in \mathbb{N}}}$;
		\item $(a_n)_{n \in \mathbb{N}}$ \`e un \Define{o-piccolo} di $(b_n)_{n \in \mathbb{N}}$, denotato $\IsLittleo{(a_n)_{n \in \mathbb{N}}}{(b_n)_{n \in \mathbb{N}}}$, quando $$\ForAll{C \in \RealPositive}{\Exists{N \in \mathbb{N}}{\Implies{n > N}{\Norm{a_n} \leq C \Norm{b_n}}}}.$$
	\end{itemize}
	\par Siano ora
	\begin{itemize}
		\item $(\TopologicalSpace,\Topology)$ uno spazio topologico;
		\item $\NormedSpace$ un $\Field$-spazio vettoriale normato;
		\item $\Part \subseteq \TopologicalSpace$;
		\item $f: \Part \rightarrow \NormedSpace$ e $g: \Part \rightarrow \NormedSpace$;
		\item $\AccumulationPoint$ punto di accumulazione per $\Part$.
	\end{itemize}
	Diciamo che
	\begin{itemize}
		\item $f(x)$ \`e un \Define{O-grande} di $g(x)$, denotato $\IsBigO{f(x)}{g(x)}$, per $x \rightarrow p$ quando $$\Exists{C \in \RealPositive}{\Exists{\Neighborhood \in \Neighborhoods{\AccumulationPoint}}{\Implies{x \in \Neighborhood}{\Norm{f(x)} \leq C \Norm{g(x)}}}};$$
		\item $f(x)$ \`e un \Define{$\Omega$-grande} di $g(x)$, denotato $\IsBigOmega{f(x)}{g(x)}$, per $x \rightarrow p$ quando $$\Exists{C \in \RealPositive}{\Exists{\Neighborhood \in \Neighborhoods{\AccumulationPoint}}{\Implies{x \in \Neighborhood}{\Norm{f(x)} \geq C \Norm{g(x)}}}};$$
		\item $f(x)$ \`e un \Define{$\Theta$-grande} di $g(x)$, denotato $\IsBigTheta{f(x)}{g(x)}$, per $x \rightarrow p$ quando, sempre per $x \rightarrow p$, $\BigO{f(x)}{g(x)}$ e $\BigOmega{f(x)}{g(x)}$;
		\item $f(x)$ \`e un \Define{o-piccolo} di $g(x)$, denotato $\IsLittleo{f(x)}{g(x)}$, per $x \rightarrow p$ quando $$\ForAll{C \in \RealPositive}{\Exists{\Neighborhood \in \Neighborhoods{\AccumulationPoint}}{\Implies{x \in \Neighborhood}{\Norm{f(x)} \leq C \Norm{g(x)}}}}.$$
	\end{itemize}
	\par Siano ora
	\begin{itemize}
		\item $\NormedSpace$ un $\Field$-spazio vettoriale normato;
		\item $\Part \subseteq \mathbb{R}$ tale che $\sup{\Part} = + \infty$;
		\item $f: \Part \rightarrow \NormedSpace$ e $g: \Part \rightarrow \NormedSpace$.
	\end{itemize}
	Diciamo che
	\begin{itemize}
		\item $f(x)$ \`e un \Define{O-grande} di $g(x)$, denotato $\IsBigO{f(x)}{g(x)}$, per $x \rightarrow + \infty$ quando $$\Exists{C \in \RealPositive}{\Exists{M \in \RealPositive}{\Implies{x > M}{\Norm{f(x)} \leq C \Norm{g(x)}}}};$$
		\item $f(x)$ \`e un \Define{$\Omega$-grande} di $g(x)$, denotato $\IsBigOmega{f(x)}{g(x)}$, per $x \rightarrow + \infty$ quando $$\Exists{C \in \RealPositive}{\Exists{\Neighborhood \in \Neighborhoods{\AccumulationPoint}}{\Implies{x \in \Neighborhood}{\Norm{f(x)} \geq C \Norm{g(x)}}}};$$
		\item $f(x)$ \`e un \Define{$\Theta$-grande} di $g(x)$, denotato $\IsBigTheta{f(x)}{g(x)}$, per $x \rightarrow + \infty$ quando, sempre per $x \rightarrow p$, $\BigO{f(x)}{g(x)}$ e $\BigOmega{f(x)}{g(x)}$;
		\item $f(x)$ \`e un \Define{o-piccolo} di $g(x)$, denotato $\IsLittleo{f(x)}{g(x)}$, per $x \rightarrow + \infty$ quando $$\ForAll{C \in \RealPositive}{\Exists{M \in \RealPositive}{\Implies{x > M}{\Norm{f(x)} \leq C \Norm{g(x)}}}}.$$
	\end{itemize}
	\par Siano ora
	\begin{itemize}
		\item $\NormedSpace$ un $\Field$-spazio vettoriale normato;
		\item $\Part \subseteq \mathbb{R}$ tale che $\inf{\Part} = - \infty$;
		\item $f: \Part \rightarrow \NormedSpace$ e $g: \Part \rightarrow \NormedSpace$.
	\end{itemize}
	Diciamo che
	\begin{itemize}
		\item $f(x)$ \`e un \Define{O-grande} di $g(x)$, denotato $\IsBigO{f(x)}{g(x)}$, per $x \rightarrow - \infty$ quando $$\Exists{C \in \RealPositive}{\Exists{M \in \RealPositive}{\Implies{x < - M}{\Norm{f(x)} \leq C \Norm{g(x)}}}};$$
		\item $f(x)$ \`e un \Define{$\Omega$-grande} di $g(x)$, denotato $\IsBigOmega{f(x)}{g(x)}$, per $x \rightarrow - \infty$ quando $$\Exists{C \in \RealPositive}{\Exists{\Neighborhood \in \Neighborhoods{\AccumulationPoint}}{\Implies{x \in \Neighborhood}{\Norm{f(x)} \geq C \Norm{g(x)}}}};$$
		\item $f(x)$ \`e un \Define{$\Theta$-grande} di $g(x)$, denotato $\IsBigTheta{f(x)}{g(x)}$, per $x \rightarrow - \infty$ quando, sempre per $x \rightarrow p$, $\BigO{f(x)}{g(x)}$ e $\BigOmega{f(x)}{g(x)}$;
		\item $f(x)$ \`e un \Define{o-piccolo} di $g(x)$, denotato $\IsLittleo{f(x)}{g(x)}$, per $x \rightarrow - \infty$ quando $$\ForAll{C \in \RealPositive}{\Exists{M \in \RealPositive}{\Implies{x - M}{\Norm{f(x)} \leq C \Norm{g(x)}}}}.$$
	\end{itemize}
\end{Definition}
\begin{Theorem}
	Sia $(a_n)_{n \in \mathbb{N}} \in \mathbb{\NormedSpace}^\mathbb{N}$, dove $\NormedSpace$ \`e un $\Field$-spazio vettoriale normato. Allota $\IsBigO{(a_n)_{n \in \mathbb{N}}}{1}$ se e solo se $(a_n)_{n \in \mathbb{N}}$ \`e limitata.
\end{Theorem}
\Proof Se $(a_n)_{n \in \mathbb{N}}$ \`e limitata, allora abbiamo $\ForAll{n \in \mathbb{N}}{\Norm{a_n} \leq \sup_{n \in \mathbb{N}} \Norm{a_n} \cdot 1}$.
\par Se invece esiste $C$ tale che per qualche $N \in \mathbb{N}$ abbiamo $\Implies{n > N}{\Norm{a_n} \leq C \cdot 1}$, allora $\ForAll{n \in \mathbb{N}}{a_n \leq \max \lbrace (a_n)_{n \leq N}, C \rbrace}$. \EndProof
\begin{Theorem}
	Siano $(a_n)_{n \in \mathbb{N}}, (b_n)_{n \in \mathbb{N}} \in \NormedSpace^{\mathbb{N}}$, dove $\NormedSpace$ \`e un $\Field$-spazio vettoriale normato e $\ForAll{n \in \mathbb{N}}{b_n \neq 0}$. $\IsLittleo{(a_n)_{n \in \mathbb{N}}}{(b_n)_{n \in \mathbb{N}}}$ se e solo se $\lim_{n \rightarrow \infty} \frac{\Norm{a_n}}{\Norm{b_n}} = 0$.
\end{Theorem}
\Proof Segue direttamente dalle definizioni. \EndProof
\begin{Corollary}
	\par Siano
	\begin{itemize}
		\item $(\TopologicalSpace,\Topology)$ uno spazio topologico che verifica il primo assioma di numerabilit\`a;
		\item $\NormedSpace$ un $\Field$-spazio vettoriale normato;
		\item una successione $(a_n)_{n \in \mathbb{N}} \in \TopologicalSpace^\mathbb{N}$ convergente a $\AccumulationPoint \in \TopologicalSpace$;
		\item $f: \bigcup_{n \in \mathbb{N}} \lbrace a_n \rbrace \rightarrow \NormedSpace$ e $g: \bigcup_{n \in \mathbb{N}} \lbrace a_n \rbrace \rightarrow \NormedSpace$.
	\end{itemize}
	Se \`e vero l'assioma della scelta, $\IsLittleo{(f(a_n))_{n \in mathbb{N}}}{(g(b_n))_{n \in \mathbb{N}}}$ se e solo se $\IsLittleo{f(x)}{g(x)}$.
\end{Corollary}
\Proof Abbiamo $\lim_{n \rightarrow \infty} \frac{\Norm{f(a_n)}}{\Norm{(b_n)}} = 0$ se e solo se $\lim_{x \rightarrow \AccumulationPoint} \frac{\Norm{f(x)}}{g(x)} = 0$. \EndProof
\begin{Corollary}
	\par Siano
	\begin{itemize}
		\item $\NormedSpace$ un $\Field$-spazio vettoriale normato;
		\item una successione $(a_n)_{n \in \mathbb{N}} \in \mathbb{R}^\mathbb{N}$ divergente negativamente o positivamente;
		\item $f: \bigcup_{n \in \mathbb{N}} \lbrace a_n \rbrace \rightarrow \NormedSpace$ e $g: \bigcup_{n \in \mathbb{N}} \lbrace a_n \rbrace \rightarrow \NormedSpace$.
	\end{itemize}
	Se \`e vero l'assioma della scelta, $\IsLittleo{(f(a_n))_{n \in mathbb{N}}}{(g(b_n))_{n \in \mathbb{N}}}$ se e solo se $\IsLittleo{f(x)}{g(x)}$.
\end{Corollary}
\Proof Abbiamo $\lim_{n \rightarrow \infty} \frac{\Norm{f(a_n)}}{\Norm{(b_n)}} = 0$ se e solo se $\lim_{x \rightarrow \pm \infty} \frac{\Norm{f(x)}}{g(x)} = 0$, dove il pi\`o o il meno dipende dal tipo di divergenza di $(a_n)_{n \in \mathbb{N}}$. \EndProof
\begin{Theorem}
	Siano $(a_n)_{n \in \mathbb{N}}, (b_n)_{n \in \mathbb{N}}, (c_n)_{n \in \mathbb{N}} \in \mathbb{\NormedSpace}^\mathbb{N}$, dove $\NormedSpace$ \`e un $\Field$-spazio vettoriale normato. Se $\IsLittleo{(b_n)_{n \in \mathbb{N}}}{(a_n)_{n \in \mathbb{N}}}$ e $\IsLittleo{(c_n)_{n \in \mathbb{N}}}{(b_n)_{n \in \mathbb{N}}}$, allora $\IsLittleo{(c_n){n \in \mathbb{N}}}{(a_n)_{n \in \mathbb{N}}}$.
\end{Theorem}
\Proof Fissiamo $C \in \RealPositive$ e sia $N \in \mathbb{N}$ tale che $n > N$ implichi
\begin{itemize}
	\item $\Norm{b_n} \leq \sqrt{C} \Norm{a_n}$;
	\item $\Norm{c_n} \leq \sqrt{C} \Norm{b_n}$.
\end{itemize}
\par Abbiamo, per $n > N$, $\Norm{c_n} \leq \sqrt{C} \Norm{b_n} \leq \sqrt{C} \sqrt{C} \Norm{a_n} = C \Norm{a_n}$. \EndProof
\begin{Theorem}
	Siano $(a_n)_{n \in \mathbb{N}}, (b_n)_{n \in \mathbb{N}}, (c_n)_{n \in \mathbb{N}} \in \mathbb{\NormedSpace}^\mathbb{N}$, dove $\NormedSpace$ \`e un $\Field$-spazio vettoriale normato. Se $\IsLittleo{(c_n)_{n \in \mathbb{N}}}{(a_n + b_n)_{n \in \mathbb{N}}}$, allora $\IsLittleo{(c_n){n \in \mathbb{N}}}{(a_n)_{n \in \mathbb{N}}}$.
\end{Theorem}
\Proof Fissiamo $C \in \RealPositive$ e sia $N \in \mathbb{N}$ tale che $n > N$ implichi $\Norm{c_n} \leq C \Norm{a_n + b_n}$.
\par Abbiamo, per $n > N$, $\Norm{c_n} \leq C \Norm{a_n + b_n} \leq C ( \Norm{a_n} + \Norm{b_n} ) \leq C \Norm{a_n}$. \EndProof
\begin{Theorem}
	Siano $(a_n)_{n \in \mathbb{N}}, (b_n)_{n \in \mathbb{N}} \in \mathbb{\NormedSpace}^\mathbb{N}$ e $f: \NormedSpace \rightarrow \VarNormedSpace$ un'applicazione continua tale che $\ForAll{(\Scalar,\Vector) \in \Field \times \NormedSpace}{f(\Scalar \Vector) = \Scalar f(\Vector)}$, dove $\NormedSpace$ e $\VarNormedSpace$ sono $\Field$-spazi vettoriali normati. Se $\IsLittleo{(a_n)_{n \in \mathbb{N}}}{(b_n)_{n \in \mathbb{N}}}$, allora $\IsLittleo{(f(a_n))_{n \in \mathbb{N}}}{(b_n)_{n \in \mathbb{N}}}$.
\end{Theorem}
\Proof Se $(b_n)_{n \in \mathbb{N}}$ \`e definitivamente nulla, allora necessariamente \`e definitivamente nulla anche $(a_n)_{n \in \mathbb{N}}$ e quindi $(f(a_n))_{n \in \mathbb{N}}$, da cui certamente $\IsLittleo{(f(a_n))_{n \in \mathbb{N}}}{(b_n)_{n \in \mathbb{N}}}$.
\par Assumiamo dunque che $(b_n)_{n \in \mathbb{N}}$ non sia definitivamente nulla. Troncando eventualmente le successioni $(a_n)_{n \in \mathbb{N}}$ e $(b_n)_{n \in \mathbb{N}}$, possiamo assumere senza perdita di generalit\`a $\ForAll{n \in \mathbb{N}}{b_n \neq 0}$.
\par Abbiamo $\frac{\Norm{f(a_n)}}{\Norm{b_n}} = \Norm{f \left ( \frac{a_n}{\Norm{b_n}} \right )}$. Poich\'e $\Norm{\frac{a_n}{\Norm{b_n}}} = \frac{\Norm{a_n}}{\Norm{b_n}}$, abbiamo $\lim_{n \rightarrow \infty} \Norm{\frac{a_n}{\Norm{b_n}}} = 0$ e dunque $\lim_{n \rightarrow \infty} \frac{a_n}{\Norm{b_n}} = 0$, da cui $\lim_{n \rightarrow \infty} \frac{\Norm{f(a_n)}}{\Norm{b_n}} = 0$ e quindi $\IsLittleo{(f(a_n))_{n \in \mathbb{N}}}{(b_n)_{n \in \mathbb{N}}}$. \EndProof
\begin{Theorem}
	Siano $(a_n)_{n \in \mathbb{N}}, (b_n)_{n \in \mathbb{N}} \in \NormedSpace^\mathbb{N}$, dove $\NormedSpace$ \`e uno spazio vettoriale normato. Se $\IsLittleo{(a_n)_{n \in \mathbb{N}}}{(b_n)_{n \in \mathbb{N}}}$, allora $\IsBigO{(a_n)_{n \in \mathbb{N}}}{(b_n)_{n \in \mathbb{N}}}$.
\end{Theorem}
\Proof Segue direttamente dalle definizioni. \EndProof
\begin{Definition}
	Sia $(a_n)_{n \in \mathbb{N}} \in \NormedSpace^\mathbb{N}$, dove $\NormedSpace$ \`e uno spazio vettoriale normato. Chiamiamo $(a_n)_{n \in \mathbb{N}}$
	\begin{itemize}
		\item \Define{successione infinita}[infinita][successione] o \Define{infinito}[infinito (successione)] se diverge;
		\item \Define{successione infinitesima}[infinitesima][successione] o \Define{infinitesimo} se converge e $\lim_{n \rightarrow \infty} a_n = 0$.
	\end{itemize}
\end{Definition}
\begin{Definition}
	Siano $(a_n)_{n \in \mathbb{N}}, (b_n)_{n \in \mathbb{N}} \in \NormedSpace^\mathbb{N}$, dove $\NormedSpace$ \`e uno spazio vettoriale normato, due successioni infinite. Assumiamo, senza perdita di generalit\`a $\ForAll{n \in \mathbb{N}}{b_n \neq 0}$: in caso contrario possiamo troncare le successioni $(a_n)_{n \in \mathbb{N}}$ e $(b_n)_{n \in \mathbb{N}}$ in modo che tale condizione sia verificata. Diciamo che 
	\begin{itemize}
		\item $(a_n)_{n \in \mathbb{N}}$ \`e un \Define{infinito di ordine inferiore}[di ordine inferiore][infinito] rispetto a $(b_n)_{n \in \mathbb{N}}$ quando $\left ( \frac{a_n}{b_n} \right )$ converge e $\lim_{n \rightarrow \infty} \frac{a_n}{b_n} = 0$;
		\item $(a_n)_{n \in \mathbb{N}}$ e $(b_n)_{n \in \mathbb{N}}$ sono \Define{infiniti dello stesso ordine}[dello stesso ordine][infinito] quando $\left ( \frac{a_n}{b_n} \right )$ converge e $\lim_{n \rightarrow \infty} \frac{a_n}{b_n} \neq 0$;
		\item $(a_n)_{n \in \mathbb{N}}$ \`e un \Define{infinito di ordine superiore}[di ordine superiore][infinito] rispetto a $(b_n)_{n \in \mathbb{N}}$ quando $\left ( \frac{a_n}{b_n} \right )$ diverge;
		\item $(a_n)_{n \in \mathbb{N}}$ e $(b_n)_{n \in \mathbb{N}}$ sono \Define{infiniti non confrontabili}[non confrontabile][infinito] quando $\left ( \frac{a_n}{b_n} \right )$ \`e indeterminata.
	\end{itemize}
\end{Definition}
\begin{Definition}
	Siano $(a_n)_{n \in \mathbb{N}}, (b_n)_{n \in \mathbb{N}} \in \NormedSpace^\mathbb{N}$, dove $\NormedSpace$ \`e uno spazio vettoriale normato, due successioni infinitesime non definitivamente nulle. Assumiamo, senza perdita di generalit\`a $\ForAll{n \in \mathbb{N}}{b_n \neq 0}$: in caso contrario possiamo troncare le successioni $(a_n)_{n \in \mathbb{N}}$ e $(b_n)_{n \in \mathbb{N}}$ in modo che tale condizione sia verificata. Diciamo che 
	\begin{itemize}
		\item $(a_n)_{n \in \mathbb{N}}$ \`e un \Define{infinitesimo di ordine inferiore}[di ordine inferiore][infinitesimo] rispetto a $(b_n)_{n \in \mathbb{N}}$ quando $\left ( \frac{a_n}{b_n} \right )$ diverge;
		\item $(a_n)_{n \in \mathbb{N}}$ e $(b_n)_{n \in \mathbb{N}}$ sono \Define{infinitesimi dello stesso ordine}[dello stesso ordine][infinitesimo] quando $\left ( \frac{a_n}{b_n} \right )$ converge e $\lim_{n \rightarrow \infty} \frac{a_n}{b_n} \neq 0$;
		\item $(a_n)_{n \in \mathbb{N}}$ \`e un \Define{infinitesimo di ordine superiore}[di ordine superiore][infinitesimo] rispetto a $(b_n)_{n \in \mathbb{N}}$ quando $\left ( \frac{a_n}{b_n} \right )$ converge e $\lim_{n \rightarrow \infty} \frac{a_n}{b_n} = 0$;
		\item $(a_n)_{n \in \mathbb{N}}$ e $(b_n)_{n \in \mathbb{N}}$ sono \Define{infinitesimi non confrontabili}[non confrontabili][infinitesimi] quando $\left ( \frac{a_n}{b_n} \right )$ \`e indeterminata.
	\end{itemize}
\end{Definition}
\begin{Theorem}
	Siano $(a_n)_{n \in \mathbb{N}}, (b_n)_{n \in \mathbb{N}} \in \NormedSpace^\mathbb{N}$, dove $\NormedSpace$ \`e uno spazio vettoriale normato, infiniti. $(a_n)_{n \in \mathbb{N}}$ \`e un infinito di ordine inferiore rispetto a $(b_n)_{n \in \mathbb{N}}$ se e solo se $\IsLittleo{(a_n)_{n \in \mathbb{N}}}{(b_n)_{n \in \mathbb{N}}}$.
\end{Theorem}
\Proof Il fatto che $(a_n)_{n \in \mathbb{N}}$ sia un infinito di ordine inferiore rispetto a $(b_n)_{n \in \mathbb{N}}$ equivale a $\lim_{n \rightarrow \infty} \frac{a_n}{b_n} = 0$, che per definizione significa che per ogni $\epsilon > 0$, esiste $N \in \mathbb{N}$ tale che per ogni $n > N$ abbiamo $\Norm{\frac{a_n}{b_n}} = \frac{\Norm{a_n}}{\Norm{b_n}} < \epsilon$, o equivalentemente $\Norm{a_n} < \epsilon \Norm{b_n}$. Il teorema segue dal fatto che $\Norm{a_n} < \epsilon \Norm{b_n}$ implica $\Norm{a_n} \leq \epsilon \Norm{b_n}$ e che $\Norm{a_n} \leq \epsilon{b_n}$ implica $\ForAll{\epsilon' > \epsilon}{\Norm{a_n} < \epsilon \Norm{b_n}}$. \EndProof
\begin{Theorem}
	Siano $(a_n)_{n \in \mathbb{N}}, (b_n)_{n \in \mathbb{N}} \in \NormedSpace^\mathbb{N}$, dove $\NormedSpace$ \`e uno spazio vettoriale normato, infinitesimi. $(a_n)_{n \in \mathbb{N}}$ \`e un infinitesimo di ordine superiore rispetto a $(b_n)_{n \in \mathbb{N}}$ se e solo se $\IsLittleo{(a_n)_{n \in \mathbb{N}}}{(b_n)_{n \in \mathbb{N}}}$.
\end{Theorem}
\Proof Il fatto che $(a_n)_{n \in \mathbb{N}}$ sia un infinitesimo di ordine superiore rispetto a $(b_n)_{n \in \mathbb{N}}$ equivale a $\lim_{n \rightarrow \infty} \frac{a_n}{b_n} = 0$, che per definizione significa che per ogni $\epsilon > 0$, esiste $N \in \mathbb{N}$ tale che per ogni $n > N$ abbiamo $\Norm{\frac{a_n}{b_n}} = \frac{\Norm{a_n}}{\Norm{b_n}} < \epsilon$, o equivalentemente $\Norm{a_n} < \epsilon \Norm{b_n}$.  Il teorema segue dal fatto che $\Norm{a_n} < \epsilon \Norm{b_n}$ implica $\Norm{a_n} \leq \epsilon \Norm{b_n}$ e che $\Norm{a_n} \leq \epsilon{b_n}$ implica $\ForAll{\epsilon' > \epsilon}{\Norm{a_n} < \epsilon \Norm{b_n}}$. \EndProof
\begin{Theorem}
	Siano $(a_n)_{n \in \mathbb{N}}, (b_n)_{n \in \mathbb{N}} \in \NormedSpace^\mathbb{N}$, dove $\NormedSpace$ \`e uno spazio vettoriale normato, infiniti. $(a_n)_{n \in \mathbb{N}}$ \`e un infinito di ordine superiore rispetto a $(b_n)_{n \in \mathbb{N}}$ se e solo se $(a_n^{-1})_{n \in \mathbb{N}}$ \`e un infinitesimo di ordine superiore a $(b_n^{-1})_{n \in \mathbb{N}}$.
\end{Theorem}
\Proof Segue dal fatto che per ogni $\epsilon > 0$ e $n \in \mathbb{N}$, $\Norm{\frac{a_n}{b_n}} < \epsilon$ equivale a $\Norm{\frac{b_n}{a_n}} > \epsilon^{-1}$. \EndProof
\begin{Theorem}
	\TheoremName{Principio di sostituzione degli infiniti} Siano $(A_n)_{n \in \mathbb{N}}, (B_n)_{n \in \mathbb{N}}, (a_n)_{n \in \mathbb{N}}, (b_n)_{n \in \mathbb{N}} \in \NormedSpace^\mathbb{N}$, dove $\NormedSpace$ \`e uno spazio vettoriale normato, infiniti. Se $\IsLittleo{(a_n)_{n \in \mathbb{N}}}{(A_n)_{n \in \mathbb{N}}}$ e $\IsLittleo{(b_n)_{n \in \mathbb{N}}}{(B_n)_{n \in \mathbb{N}}}$ e se $\left ( \frac{A_n}{B_n} \right )_{n \in \mathbb{N}}$ converge allora $\left ( \frac{A_n + a_n}{B_n + b_n} \right )$ converge a $\lim_{n \rightarrow \infty} \frac{A_n}{B_n}$.
\end{Theorem}
\Proof Basta osservare che perr ogni $n \in \mathbb{N}$, $\frac{A_n + a_n}{B_n + b_n} = \frac{A_n \left ( 1 + \frac{a_n}{A_n} \right )}{B_n \left ( 1 + \frac{b_n}{B_n} \right )}$. \EndProof
\begin{Theorem}
	\TheoremName{Principio di sostituzione degli infinitesimi} Siano $(A_n)_{n \in \mathbb{N}}, (B_n)_{n \in \mathbb{N}}, (a_n)_{n \in \mathbb{N}}, (b_n)_{n \in \mathbb{N}} \in \NormedSpace^\mathbb{N}$, dove $\NormedSpace$ \`e uno spazio vettoriale normato, infinitesimi. Se $\IsLittleo{(a_n)_{n \in \mathbb{N}}}{(A_n)_{n \in \mathbb{N}}}$ e $\IsLittleo{(b_n)_{n \in \mathbb{N}}}{(B_n)_{n \in \mathbb{N}}}$ e se $\left ( \frac{A_n}{B_n} \right )_{n \in \mathbb{N}}$ converge allora $\left ( \frac{A_n + a_n}{B_n + b_n} \right )$ converge a $\lim_{n \rightarrow \infty} \frac{A_n}{B_n}$.
\end{Theorem}
\Proof Basta osservare che perr ogni $n \in \mathbb{N}$, $\frac{A_n + a_n}{B_n + b_n} = \frac{A_n \left ( 1 + \frac{a_n}{A_n} \right )}{B_n \left ( 1 + \frac{b_n}{B_n} \right )}$. \EndProof


	\chapter{Calcolo differenziale}
	\section{Definizioni e teoremi di base}\label{DefinizioniETeoremiDiBase}
\begin{Definition}
	Siano
	\begin{itemize}
		\item $\BanachSpace$ e $\VarBanachSpace$ $\Field$-spazi di Banach;
		\item $\Open \subseteq \BanachSpace$ aperto;
		\item $f: \Open \rightarrow \VarBanachSpace$;
		\item $x \in \Open$.
	\end{itemize}
	Se esiste $\phi: \BanachSpace \rightarrow \VarBanachSpace$ operatore lineare continuo tale che $f(x + h) = f(x) + \phi(h) + \Littleo{h}$, allora diciamo che
	\begin{itemize}
		\item $f$ \`e \Define{differenziabile secondo Fr\'echet}[differenziabile secondo Fr\'echet][applicazione] o anche solo \Define{differenziabile}[differenziabile][applicazione] in $x$;
		\item $\phi$ \`e la \Define{derivata di Fr\'echet}[di Fr\'echet][derivata] o, molto pi\`u comunemente, il \Define{differenziale} o ancora il \Define{differenziale totale}[totale][differenziale] di $f$ nel punto $x$, denotato $df_x$ o $D_x f$, omettendo la $x$ ogni qual volta il punto $x$ \`e evidente dal contesto;
		\item la matrice che rappresenta $\phi$ rispetto a basi fissate si chiama \Define{matrice jacobiana}[jacobiana][matrice].
	\end{itemize}
\end{Definition}
\begin{Theorem}
	Con le notazioni della definizione precedente, la derivata di Fr\'echet, se esiste, \`e unica.
\end{Theorem}
\Proof Siano $\phi$ e $\psi$ entrambe derivate di Fr\'echet in $x$. Fissiamo $\epsilon > 0$. Per definizione di derivata di Fr\'echet, esiste $\delta > 0$, tale che per ogni $h \in \BanachSpace$ tale che $\Norm{h} < \delta$ abbiamo
\begin{itemize}
	\item $\Norm{f(x + h) - f(x) - \phi(h)} \leq \frac{\epsilon}{2}\Norm{h}$;
	\item $\Norm{f(x + h) - f(x) - \psi(h)} \leq \frac{\epsilon}{2}\Norm{h}$.
\end{itemize}
Da cui $\Norm{(\phi - \psi)h} = \Norm{(f(x + h) - f(x) - \psi(h)) - (f(x + h) - f(x) - \phi(h)} \leq \Norm{f(x + h) - f(x) - psi(h)} + \Norm{f(x + h) - f(x) - \phi(h)} \leq \epsilon \Norm{h}$. E quindi $\delta \Norm{\phi - \psi} = \delta \sup_{\Norm{\Vector} \leq 1} \Norm{(\phi - \psi)\Vector} = \sup_{\Norm{\Vector} \leq \delta} \Norm{(\phi - \psi) \Vector} \leq \sup_{\Norm{\Vector} \leq \delta} \epsilon \Norm{\Vector} \leq \epsilon \delta$. Pertanto $\Norm{\phi - \psi} \leq \epsilon$.
\par Ne deduciamo $\Norm{\phi - \psi} = 0$ e dunque $\phi = \psi$. \EndProof
\begin{Definition}
	Siano $(\BanachSpace_i)_{i \in n}$ ($n \in \mathbb{N}$) e $\VarBanachSpace$ $\Field$-spazi di Banach. Sia $f: \bigoplus_{i \in n} \BanachSpace_i \rightarrow \VarBanachSpace$ e sia $(x_i)_{i \in n} \in \bigoplus_{i \in n} \BanachSpace_i$. Fissiamo $j \in n$: se $f_j: \BanachSpace_j \rightarrow \VarBanachSpace$ che a $x \in \BanachSpace_j$ associa $f((y_i)_{i \in n}) \in \VarBanachSpace$, con $y_i = \begin{cases} x\text{ se }i = j,\\x_i\text{ altimenti,}\end{cases}$ \`e differenziabile secondo Fr\'echet, chiamiamo $\TotalDifferential{f_j}$ \Define{$j$-esimo differenziale parziale}[parziale][differenziale] o \Define{$j$-esima derivata parziale di Fr\'echet}[parziale di Fr\'echet][derivata], denotato $\PartialDifferential{f}{j}$.
\end{Definition}
\begin{Theorem}
	Con le notazioni della definizione precedente, abbiamo $\phi = \PartialDifferential{f}{j}$ se e solo se, per ogni $h \in \bigoplus_{i \in n} \BanachSpace_i$, $f(x + \Projection(h)) = f(x) + \phi(\Projection(h)) + \Littleo{h}$, dove $\Projection: \bigoplus_{i \in n} \BanachSpace_i \rightarrow \bigoplus_{i \in n} \BanachSpace_i$ \`e una proiezione su $\BanachSpace_j$.
\end{Theorem}
\Proof Segue direttamente dalle definizioni. \EndProof
\begin{Theorem}
	Siano $(\BanachSpace_i)_{i \in n}$ ($n \in \mathbb{N}$) e $\VarBanachSpace$ $\Field$-spazi di Banach. Sia $\Open \subseteq \bigoplus_{i \in n}$ aperto, $f: \Open \BanachSpace_i \rightarrow \VarBanachSpace$ e sia $(x_i)_{i \in n} \in \bigoplus_{i \in n} \BanachSpace_i$. Se $f$ \`e differenziabile in $(x_i)_{i \in n}$, allora
	\begin{itemize}
		\item esistono anche tutti i differenziali parziali $(\PartialDifferential{f}{i})_{i \in n}$;
		\item per ogni famiglia di proiezioni $(\Projection_i)_{i \in n}$ di $\bigoplus_{i \in n} \BanachSpace_i$ tali che, per ogni $h \in \bigoplus_{i \in n} \BanachSpace_i$, abbiamo $h = \sum_{i \in n} \Projection_i(h)$, abbiamo $\ForAll{i \in n}{\PartialDifferential{f}{i} = \TotalDifferential \circ \Projection_i}$;
		\item abbiamo $\TotalDifferential{f} = \bigoplus_{i \in n} \PartialDifferential{f}{i}$.
	\end{itemize}
\end{Theorem}
\Proof Definiamo una famiglia di proiezioni $(\Projection_i)_{i \in n}$ di $\bigoplus_{i \in n} \BanachSpace_i$ tali che, per ogni $h \in \bigoplus_{i \in n} \BanachSpace_i$, abbiamo $h = \sum_{i \in n} \Projection_i(h)$.
\par Abbiamo allora,
$f(x + h) = f(x) + \TotalDifferential(h) + \Littleo{h}$, da cui
$f(x + \Projection_i(h)) = f(x) + \TotalDifferential((\Projection_i(h)) + \Littleo{h}$
e quindi $\TotalDifferential \circ \Projection_i$ \`e l'$i$-esimo differenziale parziale $\PartialDifferential{f}{i}$ di $f$.
\par Abbiamo infine, per ogni $h \in \bigoplus_{i \in n} \BanachSpace_i$, $\left ( \bigoplus_{i \in n} \PartialDifferential{f}{i} \right )(h) = \sum_{i \in n} (\TotalDifferential \circ \Projection_i)(h) = \TotalDifferential \left ( \sum_{i \in n} \Projection_i(h) \right ) = \TotalDifferential(h)$. \EndProof
\begin{Theorem}
	Siano
	\begin{itemize}
		\item $(\BanachSpace_i)_{i \in 3}$ $\Field$-spazi di Banach;
		\item $\Open \subseteq \BanachSpace_0$ aperto;
		\item $f: \Open \rightarrow \BanachSpace_1$;
		\item $\VarOpen \subseteq \BanachSpace_1$ aperto tale che $f(\Open) \subseteq \VarOpen$;
		\item $g: \VarOpen \rightarrow \BanachSpace_2$;
		\item $x \in \Open$.
	\end{itemize}
	Se $f$ \`e differenziabile in $x$ e $g$ \`e differenziabile in $f(x)$, allora $(f \circ g)$ \`e differenziabile in $x$ e abbiamo $\TotalDifferential{(g \circ f)}[x] = \TotalDifferential{g}[f(x)] \circ \TotalDifferential{f}[x]$.
\end{Theorem}
\Proof Abbiamo
\begin{align*}
(g \circ f)(x + h) &= g(f(x) + \TotalDifferential{f}[x](h) + \Littleo{h}),\\
&= g(f(x) + \TotalDifferential{f}[x](h) + \Littleo{h}),\\
&= g(f(x)) + \TotalDifferential{g}[f(x)](\TotalDifferential{f}[x](h) + \Littleo{h}) + \Littleo{\TotalDifferential{f}[x](h) + \Littleo{h}},\\
&= g(f(x)) + \TotalDifferential{g}[f(x)](\TotalDifferential{f}[x](h)) + \TotalDifferential{g}[f(x)](\Littleo{h}) + \Littleo{\TotalDifferential{f}[x](h) + \Littleo{h}}.
\end{align*}
\par Abbiamo $\TotalDifferential{g}[f(x)](\Littleo{h}) = \Littleo{h}$ perch\'e il differenziale totale \`e un operatore lineare continuo. Di conseguenza, $\TotalDifferential{g}[f(x)](\Littleo{h}) + \Littleo{\TotalDifferential{f}[x](h) + \Littleo{h}} = \Littleo{h}$.
\par Quindi $(g \circ f)$ \`e differenziabile in $x$ e il suo differenziale \`e $\TotalDifferential{(g \circ f)}[x] = \TotalDifferential{g}[f(x)] \circ \TotalDifferential{f}[x]$. \EndProof
\begin{Definition}
	Siano
	\begin{itemize}
		\item $\BanachSpace$ e $\VarBanachSpace$ $\Field$-spazi di Banach;
		\item $\Open \subseteq \BanachSpace$ aperto;
		\item $f: \Open \rightarrow \VarBanachSpace$;
		\item $x \in \Open$;
		\item $\Vector \in \BanachSpace$.
	\end{itemize}
	Se esiste $\lim_{t \rightarrow 0} \frac{f(x + t\Vector) - f(x)}{t}$ ($t \in \Field$), definiamo tale limite \Define{derivata direzionale}[direzionale][derivata] di $f$ in $x$ lungo $\Vector$, denotata $\frac{\partial f}{\partial \Vector}(x)$ o $D_{\Vector} f(x)$ o ancora $f_{\Vector}(x)$. Inoltre, se $\BanachSpace = \Field^n$ per $n \in \mathbb{N}$ e $\Vector$ \`e l'$i$-esimo vettore della base canonica di $\Field^n$, chiamiamo $\PartialDerivative{f}{\Vector}$ \Define{$i$-esima derivata parziale}[parziale][derivata] di $f$; quando denotiamo con un simbolo diverso ogni componente di $\Field^n$, denotiamo anche la derivata parziale anche sostituendo a $\Vector$ il simbolo in questione: per esempio per esprimere le tre derivate parziali di $f(x,y,z)$ (supponendo che esistano) useremo le notazioni $\PartialDerivative{f}{x}$, $\PartialDerivative{f}{y}$, $\PartialDerivative{f}{z}$. Inoltre, se $\Dimension{\BanachSpace} = 1$, anzich\'e $\PartialDerivative{f}{\Vector}$ o $\PartialDerivative{f}{x}$ scriveremo $\frac{df}{dx}$ o $f'(x)$ o ancora $\dot{f}(x)$.
\end{Definition}
\begin{Definition}
	Siano
	\begin{itemize}
		\item $\BanachSpace$ e $\VarBanachSpace$ $\Field$-spazi di Banach;
		\item $\Open \subseteq \BanachSpace$ aperto;
		\item $\VarOpen \subseteq \Open$ aperto;
		\item $f: \Open \rightarrow \VarBanachSpace$.
		\item $\Vector \in \BanachSpace$.
	\end{itemize}
	Sia $\Part$ la parte di $\Open$ tale che per ogni $x \in \VarOpen$, $f$ \`e derivabile lungo $\Vector$ in $x$: chiamiamo $g: \Part \rightarrow \VarBanachSpace$ \Define{funzione derivata lungo $\Vector$}[derivata lungo un vettore][funzione] la funzione che a $x \in \Part$ associa $\DirectionalDerivative{f}{\Vector}(x)$. Inoltre, se per opportuna parte $\Part' \subseteq \Open$, la funzione $h: \Part' \rightarrow \VarBanachSpace$ si ottiene derivando $f$ a turno lungo i vettori di una famiglia di vettori $(\Vector_i)_{i \in n}$ ($n \in \mathbb{N}$), allora $h$ si chiama \Define{derivata di ordine $n$}[di una derivata][ordine].
\end{Definition}
\begin{Theorem}
	Siano
	\begin{itemize}
		\item $\BanachSpace$ e $\VarBanachSpace$ $\Field$-spazi di Banach;
		\item $\Linear: \BanachSpace \rightarrow \VarBanachSpace$ un'applicazione lineare;
		\item $x \in \BanachSpace$.
	\end{itemize}
	Abbiamo $\TotalDifferential{\Linear} = \Linear$.
\end{Theorem}
\Proof Abbiamo $\Linear(x + h) = \Linear(x) + \Linear(h) + 0$ e $\IsLittleo{0}{h}$. \EndProof
\begin{Corollary}
	Siano
	\begin{itemize}
		\item $\BanachSpace$ un $\Field$-spazio di Banach;
		\item $x \in \BanachSpace$.
	\end{itemize}
	Abbiamo $\TotalDifferential{\Identity[\BanachSpace]} = \Identity[\BanachSpace]$.
\end{Corollary}
\Proof $\Identity[\BanachSpace]$ \`e un automorifsmo lineare di $\BanachSpace$. \EndProof
\begin{Theorem}
	Siano
	\begin{itemize}
		\item $\BanachSpace$ un $\Field$-spazio di Banach tale che $\Dimension{\BanachSpace} = 1$;
		\item $\Open \subseteq \BanachSpace$ aperto;
		\item $f: \Open \rightarrow \Field$;
		\item $x \in \Open$.
	\end{itemize}
	$f$ \`e differenziabile in $x$ se e solo se \`e derivabile in $x$ e abbiamo $\Differential{f} = \Derivative{f}(x) \Differential{x}$.
\end{Theorem}
\Proof Sia $\Vector$ un vettore unitario di $\BanachSpace$: esso ne costituisce da solo una base normale. $\lim_{t \rightarrow 0} \frac{f(x + t\Vector) - f(x)}{t} = \Derivative{f}(x)$ equivale a $\lim_{t \rightarrow 0} \frac{\Norm{f(x + t\Vector) - f(x) - t f'(x)}}{\Norm{t\Vector}}$

\section{Regole di differenziazione.}
\label{CalcoloDifferenziale_RegoleDiDifferenziazione}

\section{Punti stazionari.}
\label{CalcoloDifferenziale_PuntiStazionari}

\section{Funzioni reali variabili reale}
\label{CalcoloDifferenziale_FunzioniRealiDiVariabiliReale}
\begin{Definition}
	Una funzione $f$ si dice
	\begin{itemize}
		\item \Define{reale}[reale][funzione] quando $\Cod{f} \subseteq \mathbb{R}$;
		\item \Define{di variabile reale}[di variabile reale] quando $\Dom{f} \subseteq \mathbb{R}$.
	\end{itemize}
\end{Definition}
\begin{Definition}
	Data la funzione reale di variabile reale $f$, per ogni $n \in \mathbb{N}$, chiamiamo la sua derivata di ordine $n$, se esiste, \Define{derivata $n$-esima}[$n$-esima][derivata].
\end{Definition}
\begin{Definition}
	Siano
	\begin{itemize}
		\item $n, m \in \NotZero{\mathbb{N}}$;
		\item $\Open \subseteq \mathbb{R}^n$ aperto;
		\item $\Part \subseteq \mathbb{R}^m$.
	\end{itemize}
	Per ogni $k \in \mathbb{N}$, definiamo $\CClass{k}[\Open][\Part]$ come la classe di tutte le funzioni $f$ tali che
	\begin{itemize}
		\item $f: \Open \rightarrow \Part$;
		\item $f$ \`e continua;
		\item $f$ ammette tutte le derivate parziali di ordine minore o uguale a $k$ e esse sono tutte continue.
	\end{itemize}
	Poniamo inoltre $\CClass{\infty}[\Open][\Part] = \bigcap_{n \in \mathbb{N}} \CClass{k}[\Open][\Part]$. Ometteremo spesso l'indicazione esplicita del dominio $\Open$ e del codominio $\Part$ quando essi saranno chiari dal contesto.
\end{Definition}
\begin{Definition}
	Sia $f$ funzione reale di variabile reale avente come dominio un intervallo e derivabile $N \in \mathbb{N}$ volte in un punto $x_0 \in \Dom{f}$, dove $\Dom{f}$ \`e un intervallo, definiamo
	\begin{itemize}
		\item \Define{polinomio di Taylor}[di Taylor][polinomio] di grado $N \in \mathbb{N}$ il polinomio $\sum_{n = 0}^N \frac{f^{(n)}(x)}{n!} (x - x_0)^n$, al variare di $x \in \Dom{f}$;
		\item nel caso $f$ sia derivabile in $x_0$ infinite volte, \Define{serie di Taylor}[di Taylor][serie] la serie $\sum_{n = 0}^\infty \frac{f^{(n)}(x)}{n!} (x - x_0)^n$, al variare di $x \in \Dom{f}$;
		\item nel caso $f$ sia derivabile in $x_0 = 0$ infinite volte, \Define{serie di Maclaurin}[di Maclaurin][serie] la serie $\sum_{n = 0}^\infty \frac{f^{(n)}(x)}{n!} (x - x_0)^n$, al variare di $x \in \Dom{f}$.
	\end{itemize}
\end{Definition}
\begin{Theorem}
	\TheoremName{Teorema di Taylor}[di Taylor][teorema]
	Sia $f$ funzione reale di variabile reale avente come dominio un intervallo e derivabile $N \in \mathbb{N}$ volte in un punto $x_0 \in \Dom{f}$. Abbiamo, per $x \in \Dom{f}$,
\[
	f(x) = \sum_{n = 0}^N \frac{f^{(n)}(x_0)}{n!} (x - x_0)^n + \Littleo{(x - x_0)^N}.
\]
\end{Theorem}
\begin{Definition}
	Con le notazioni del teorema precedente, $\Littleo{(x - x_0)^k}$ si chiama \Define{resto di Peano}[di Peano][resto].
\end{Definition}
\begin{Theorem}
	\TheoremName{Resto di Lagrange}[di Lagrange][resto] 
	Sia $f$ funzione reale di variabile reale avente come dominio un intervallo e derivabile $N + 1$ ($N \in \mathbb{N}$) volte in un punto $x_0 \in \Dom{f}$. Abbiamo, per $x \in \Dom{f}$, $f(x) = \sum_{n = 0}^N \frac{f^{(n)}(x_0)}{n!} (x - x_0)^n + \frac{f^{N + 1}(\xi)}{(N + 1)!}(x - x_0)^{N + 1}$, per opportuno $\xi \in [x,x_0] \cup [x_0,x]$. 
\end{Theorem}
\begin{Theorem}
	\TheoremName{Teorema di Taylor}[di Taylor][teorema]
	Sia $f$ funzione reale di variabile pi\`u variabili reali avente come dominio un intervallo di $\mathbb{R}^m$ $(m \in \NotZero{\mathbb{N}})$ e differenziabile $N \in \mathbb{N}$ volte in un punto $x_0 \in \Dom{f}$. Abbiamo, per $x \in \Dom{f}$,
\[
	f(x) = \sum_{\NaturalDegree{\alpha} \leq N} \frac{\HigherPartialDerivative{\alpha}{f}(x_0)}{\alpha!} (x - x_0)^\alpha + \sum_{\NaturalDegree{\alpha} = N}\Littleo{(x - x_0)^\alpha}.
\]
\end{Theorem}
\begin{Definition}
  Siano
  \begin{itemize}
    \item $\Open \subseteq \mathbb{R}$ aperto;
    \item $f \in \CClass{\infty}[\Open][\mathbb{R}]$.
  \end{itemize}
  $f$ si dice
  \Define{analitica}[analitica][funzione]
  quando in ogni punto $x \in \Open$ lo sviluppo di Taylor di $f$ attorno ad
  $x$ converge a $f(x)$.
\end{Definition}

\section{Equazioni differenziali.}
\label{CalcoloDifferenziale_EquazioniAlleDerivateParzialiEEquazioniDifferenziali}
\begin{Definition}
	Un'\Define{equazione funzionale}[funzionale][equazion] \`e un'equazione dove le incognite sono funzioni.
\end{Definition}
\begin{Definition}
	Un'\Define{equazione differenziale}[differenziale] \`e un'equazione funzionale che coinvolge almeno una derivata di una funzione incognita. L'equazione si chiama
	\begin{itemize}
		\item \Define{equazione differenziale ordinaria}[differenziale ordinaria][equazione] quando le funzioni incognite sono funzioni in una sola variabile;
		\item \Define{equazione alle derivate parziali}[alle derivate parziali][equazione] quando le funzioni incognite sono funzioni in pi\`u variabili.
	\end{itemize}
	Chiamiamo \Define{ordine}[di un'equazione differenziale] dell'equazione il massimo degli ordini delle derivate presenti nell'equazione stessa.
\end{Definition}
\begin{Definition}
	Un'\Define{equazione differenziale autonoma} \`e un'equazione differenziale ordinaria che non dipende esplicitamente dalla variabile indipendente. Vale a dire che \`e della forma $f((x^{(i)})_{i \in n})) = 0$, dove $n \in \mathbb{N}$, $f$ \`e una funzione qualsiasi e $x(t)$ \`e la funzione incognita.
\end{Definition}


	\chapter{Teoria della misura}
	\section{Definizioni e teoremi di base.}
\label{Misura_DefinizioniETeoremiDiBase}
\begin{Definition}
	Un'\Define{algebra di parti}[di parti][algebra]\footnote{Il termine in uso
  solitamente \`e
  \Define{algebra di insiemi}[di insiemi][algebra], ma preferiamo il termine
  ``algebra di parti'' perch\'e consente facili estensioni al caso in cui le
  parti in questione siano parti di una classe, e non solo di un insieme, come
  \`e invece il caso per l'usuale definizione di algebra di insiemi.}
  $\AlgebraOfParts$ su una classe $\MeasureSpace$ \`e una parte di
  $\PowerSet{\MeasureSpace}$ tale che
  \begin{itemize}
    \item $\emptyset \in \AlgebraOfParts$;
    \item $\AlgebraOfParts$ sia chiuso per passaggio al complementare;
    \item $\AlgebraOfParts$ sia chiuso per unione finita.
  \end{itemize}
\end{Definition}
\begin{Theorem}
  Un'algebra $\AlgebraOfParts$ su $\MeasureSpace$ \`e chiusa per
  intersezione finita.
\end{Theorem}
\Proof Siano $\MeasurablePart, \VarMeasurablePart \in \AlgebraOfParts$.
Abbiamo
$\MeasurablePart \cap \VarMeasurablePart
= \Complement{\Complement{\MeasurablePart} \cup \Complement{\VarMeasurablePart}}
\in \AlgebraOfParts$. \EndProof
\begin{Definition}
  Una
  \Define{$\sigma$-algebra} su una classe $\MeasureSpace$ \`e un'algebra
  di parti su $\MeasureSpace$ chiusa per unione numerabile.
\end{Definition}
\begin{Theorem}
  Una $\sigma$-algebra $\SigmaAlgebra$ su $\MeasureSpace$ \`e chiusa per
  intersezione numerabile.
\end{Theorem}
\Proof Sia
$(\MeasurablePart_n)_{n \in \mathbb{N}} \in \SigmaAlgebra^\mathbb{N}$.
Abbiamo
$\bigcap_{n \in \mathbb{N}} \MeasurablePart_n
= \Complement{\bigcup_{n \in \mathbb{N}} \Complement{\MeasurablePart_n}}
\in \SigmaAlgebra$. \EndProof
\begin{Theorem}
  Sia $(\AlgebraOfParts_i)_{i \in I}$ una famiglia di algebre di parti
  sulla stessa classe $\MeasureSpace$.
  $\bigcap_{i \in I} \AlgebraOfParts$ \`e un'algebra di parti su
  $\MeasureSpace$.
\end{Theorem}
\Proof Segue immediatamente dalle definizioni. \EndProof
\begin{Corollary}
  Sia $(\SigmaAlgebra_i)_{i \in I}$ una famiglia di $\sigma$-algebre sulla
  stessa classe $\MeasureSpace$.
  $\bigcap_{i \in I} \SigmaAlgebra_i$ \`e una $\sigma$-algebra su
  $\MeasureSpace$.
\end{Corollary}
\Proof Segue immediatamente dalle definizioni. \EndProof
\begin{Definition}
  Sia $\MeasureSpace$ una classe e sia $S \subseteq \PowerSet{\MeasureSpace}$.
  Chiamiamo
  \begin{itemize}
    \item \Define{algebra di parti generata}[di parti generata][algebra]
      da $S$ in $\MeasureSpace$ la pi\`u piccola algebra di parti
      $\AlgebraOfParts$ di $\MeasureSpace$ tale che
      $S \subseteq \AlgebraOfParts$;
    \item \Define{$\sigma$-algebra generata}[generata][$\sigma$-algebra]
      da $S$ in $\MeasureSpace$ la pi\`u piccola $\sigma$-algebra
      $\SigmaAlgebra$ di $\MeasureSpace$ tale che
      $S \subseteq \SigmaAlgebra$, denotata
      $\GeneratedSigmaAlgebra{S}$.
  \end{itemize}
\end{Definition}
\begin{Definition}
  Sia $(\TopologicalSpace,\Topology)$ uno spazio topologico. Chiamiamo
  $\GeneratedSigmaAlgebra{\Topology}$
  \Define{$\sigma$-algebra di Borel}[di Borel][$\sigma$-algebra]
  e chiamiamo gli elementi di
  $\GeneratedSigmaAlgebra{\Topology}$
  \Define{boreliani}[boreliano].
\end{Definition}
\begin{Definition}
	Uno \Define{spazio misurabile}[misurabile][spazio]
  $(\MeasureSpace,\SigmaAlgebra)$
  \`e una coppia in cui
  \begin{itemize}
    \item $\MeasureSpace$ \`e una classe;
    \item $\SigmaAlgebra$ una $\sigma$-algebra su $\MeasureSpace$.
  \end{itemize}
\end{Definition}
\begin{Definition}
  Sia $(\TopologicalSpace,\Topology)$ uno spazio topologico. Chiamiamo
  $(\TopologicalSpace,\GeneratedSigmaAlgebra{\Topology})$
  \Define{spazio boreliano}[boreliano][spazio].
\end{Definition}
\begin{Definition}
	Siano
  \begin{itemize}
    \item $(\MeasurableSpace,\SigmaAlgebra)$ e
      $(\VarMeasurableSpace,\VarSigmaAlgebra)$ spazi misurabili;
    \item un'applicazione $\MeasurableApplication: \MeasurableSpace \rightarrow
      \VarMeasurableSpace$.
  \end{itemize}
  Se $\ForAll{\VarMeasurablePart \in \VarSigmaAlgebra}
  {\MeasurableApplication^{-1}(\VarMeasurablePart) \in \SigmaAlgebra}$, allora
  si dice che $\MeasurableApplication$ \`e
  \Define{misurabile}[misurabile][applicazione]
  Se inoltre $\MeasurableSpace$ e $\VarMeasurableSpace$ sono spazi boreliani,
  allora
  $\MeasurableApplication$ si dice
  \Define{boreliana}[boreliana][applicazione].
\end{Definition}
\begin{Theorem}
  Con le notazioni del teorema precedente,
  se $\MeasurableApplication$ \`e costante, allora \`e misurabile.
\end{Theorem}
\Proof Sia $c \in \Image{\MeasurableApplication}$.
\par Sia $\VarMeasurablePart \in \VarSigmaAlgebra$.
\par Ora,
\begin{itemize}
  \item se $c \notin \VarMeasurablePart$, allora
    $\MeasurableApplication^{-1} (\VarMeasurablePart)
    = \emptyset \in \SigmaAlgebra$;
  \item se $c \in \VarMeasurablePart$, allora
    $\MeasurableApplication^{-1} (\VarMeasurablePart)
    = \MeasurableSpace \in \SigmaAlgebra$. \EndProof
\end{itemize}
\begin{Theorem}
  Siano
  \begin{itemize}
    \item $(\MeasureSpace_0,\SigmaAlgebra_0)$,
      $(\MeasureSpace_1,\SigmaAlgebra_1)$
      e $(\MeasureSpace_2,\SigmaAlgebra_2)$
      spazi misurabili;
    \item $\MeasurableApplication_0:
      \MeasureSpace_0 \rightarrow \MeasureSpace_1$
      un'applicazione misurabile;
    \item $\MeasurableApplication_1:
      \MeasureSpace_1 \rightarrow \MeasureSpace_2$
      un'applicazione misurabile.
  \end{itemize}
  L'applicazione
  $\MeasurableApplication
  = \MeasurableApplication_1 \circ \MeasurableApplication_0$ \`e
  misurabile.
\end{Theorem}
\Proof Sia $\MeasurablePart \in \SigmaAlgebra_2$. Per la misurabilit\`a
di $\MeasurableApplication_1$, $\MeasurableApplication_1^{-1}(\MeasurablePart)$
\`e misurabile. Per la misurabilit\`a di $\MeasurableApplication_0$,
$\MeasurableApplication^{-1}(\MeasurablePart) =
\MeasurableApplication_0^{-1} (\MeasurableApplication_1^{-1} (\MeasurablePart))$
\`e misurabile. \EndProof
\begin{Corollary}
  Le applicazioni misurabili costituiscono una cateogria i cui oggetti sono gli
  spazi misurabili. Denoteremo tale categoria $\CatMeasurableSpace$.
\end{Corollary}
\Proof Segue direttamente dal teorema precedente. \EndProof
\begin{Definition}
  Data una classe $\MeasureSpace$ e
  $\SigmaAlgebra  \subseteq \PowerSet{\MeasureSpace}$,
  un'applicazione
  $\Measure: \SigmaAlgebra \rightarrow \mathbb{R}$
  si dice
  \begin{itemize}
    \item \Define{additiva}[additiva][applicazione] se
      \begin{itemize}
        \item $\SigmaAlgebra$ \`e un'algebra di parti su
          $\MeasureSpace$;
        \item per ogni coppia
            $(\MeasurablePart,\VarMeasurablePart) \in \SigmaAlgebra^2$
            di parti disgiunte abbiamo
            $\Measure(\MeasurablePart \cup \VarMeasurablePart) =
            \Measure(\MeasurablePart) + \Measure(\VarMeasurablePart)$;
      \end{itemize}
    \item \Define{$\sigma$-additiva}[$\sigma$-additiva][applicazione] se
      \begin{itemize}
        \item $\SigmaAlgebra$ \`e una $\sigma$-algebra su
          $\MeasureSpace$;
        \item per ogni famiglia
          $(\MeasurablePart_n)_{n \in \mathbb{N}} \in \SigmaAlgebra^\mathbb{N}$
          di parti disgiunte due a due abbiamo
          $\Measure \left ( \bigcup_{n \in \mathbb{N}} \MeasurablePart_n\right )
            = \sum_{n \in \mathbb{N}} \Measure(\MeasurablePart_n)$.
      \end{itemize}
  \end{itemize}
  Nell'ipotesi che
  \begin{itemize}
    \item $\SigmaAlgebra$ sia una $\sigma$-algebra;
    \item $\Measure$ sia $\sigma$-additiva;
    \item $\Cod{\Measure} = [0,+\infty]$;
  \end{itemize}
  definiamo
  \begin{itemize}
    \item \Define{misura} sullo spazio misurabile
      $(\MeasureSpace,\SigmaAlgebra)$
      l'applicazione $\Measure$;
    \item \Define{spazio di misura}[di misura][spazio] la terna
      $(\MeasureSpace,\SigmaAlgebra,\Measure)$.
  \end{itemize}
  Inoltre, dato $\MeasurablePart \in \SigmaAlgebra$, diciamo che
  \begin{itemize}
    \item $\MeasurablePart$ contiene \Define{quasi tutti}[tutti][quasi]
      gli elementi di $\MeasureSpace$ se
      $\Measure(\MeasurablePart) = \Measure(\MeasureSpace)$;
    \item $\MeasurablePart$ contiene \Define{quasi nessun}[nessuno][quasi]
      elemento di $\MeasureSpace$ se
      $\Measure(\MeasurablePart) = 0$; $\MeasurablePart$ si dice in tal caso
      \Define{trascurabile}[trascurabile][parte].
  \end{itemize}
\end{Definition}
\begin{Definition}
  Sia $(\MeasurableSpace,\SigmaAlgebra)$ uno spazio di Borel. Una misura
  $\Measure$ definita su uno spazio misurabile $(\MeasureSpace,\SigmaAlgebra)$
  si chiama
  \Define{misura di Borel}[di Borel][misura].
\end{Definition}
\begin{Theorem}
  Dati
  \begin{itemize}
    \item una classe $\MeasureSpace$,
    \item $\SigmaAlgebra  \subseteq \PowerSet{\MeasureSpace}$,
    \item un'applicazione
          $\Measure: \SigmaAlgebra \rightarrow \mathbb{R}$
          additiva;
  \end{itemize}
  abbiamo, se $\emptyset \in \SigmaAlgebra$,
  $\Measure(\emptyset) = 0$
\end{Theorem}
\Proof Abbiamo
\begin{align*}
  \Measure(\emptyset)
  &= \Measure(\emptyset \cup \emptyset),\\
  &= \Measure(\emptyset) + \Measure(\emptyset),
\end{align*}
da cui $\Measure(\emptyset) = 0$. \EndProof
\begin{Theorem}
	Dati uno spazio misurabile
  $(\MeasureSpace,\SigmaAlgebra)$ e una funzione
  $\Measure: \SigmaAlgebra \rightarrow [0; +\infty]$ sono equivalenti i
  seguenti fatti:
	\begin{itemize}
		\item $\Measure$ \`e una misura;
		\item se
      $(\MeasurablePart_n)_{n \in \mathbb{N}} \in \SigmaAlgebra^\mathbb{N}$
      \`e una successione crescente, allora
      $\lim_{n \rightarrow \infty} \Measure(\MeasurablePart_n)
      = \Measure(\bigcup_{n \in \mathbb{N}} \MeasurablePart_n)$;
		\item se
      $(\MeasurablePart_n)_{n \in \mathbb{N}} \in \SigmaAlgebra^\mathbb{N}$
      \`e una successione decrescente e
      $\Exists{n \in \mathbb{N}}{\MeasurablePart_n < + \infty}$, allora
      $\lim_{n \rightarrow \infty} \Measure(\MeasurablePart_n)
      = \Measure(\bigcap_{n \in \mathbb{N}} \MeasurablePart_n)$.
	\end{itemize}
\end{Theorem}
\begin{Theorem}
  Siano
  \begin{itemize}
    \item $(\MeasureSpace,\SigmaAlgebra,\Measure)$ uno spazio di misura;
    \item $(\MeasurableSpace,\VarSigmaAlgebra)$ uno spazio misurabile;
    \item $\MeasurableApplication: \MeasureSpace \rightarrow \MeasurableSpace$
      un'applicazione misurabile.
  \end{itemize}
  L'applicazione
  $\VarMeasure: \VarSigmaAlgebra \rightarrow [0;+\infty]$
  che a $\VarMeasurablePart \in \VarSigmaAlgebra$ associa
  $(\Measure \circ \MeasurableApplication^{-1}) (\VarMeasurablePart)$ \`e
  una misura sullo spazio misurabile $(\MeasurableSpace,\VarSigmaAlgebra)$.
\end{Theorem}
\Proof Sia
$(\VarMeasurablePart_n)_{n \in \mathbb{N}} \in \VarSigmaAlgebra^\mathbb{N}$.
Abbiamo
\begin{align*}
  \VarMeasure \left ( \bigcup_{n \in \mathbb{N}} \VarMeasurablePart_n \right )
  &= \Measure \left (
    \MeasurableApplication^{-1} \left (
    \bigcup_{n \in \mathbb{N}} \VarMeasurablePart_n \right ) \right ),\\
  &= \Measure \left (
    \bigcup_{n \in \mathbb{N}}
      \MeasurableApplication^{-1} (\VarMeasurablePart_n) \right ),\\
  &= \sum_{n \in \mathbb{N}} \Measure (
      \MeasurableApplication^{-1} (\VarMeasurablePart_n) ),\\
  &= \sum_{n \in \mathbb{N}} \VarMeasure (\VarMeasurablePart_n).\text{ \EndProof}
\end{align*}
\begin{Definition}
  Con le notazioni del teorema precedente, chiamiamo $\VarMeasure$
  \Define{misura immagine}[immagine][misura] di $\Measure$ tramite
  $\MeasurableApplication$.
  Inoltre, se $\phi(x)$ \`e una formula che definisce una parte di
  $\MeasurableSpace$ in $\VarSigmaAlgebra$ secondo lo schema di comprensione,
  allora denotiamo $\Measure (\phi(\MeasurableApplication))$ la quantit\`a
  $\VarMeasure(\lbrace x \in \MeasurableSpace | \phi(x) \rbrace)$.
\end{Definition}

\section{Sottospazi misurabili e sottospazi di misura.}
\label{Misura_SottospaziMisurabiliESottospaziDiMisura}
\begin{Theorem}
  Sia $(\MeasurableSpace,\SigmaAlgebra)$ uno spazio misurabile e sia
  $\MeasurableSpace_0 \in \SigmaAlgebra$. Posto
  \[
    \SigmaAlgebra_0 =
    \lbrace \MeasurablePart_0 \in \PowerSet{\MeasurableSpace_0} |
    \Exists{\MeasurablePart \in \SigmaAlgebra}
    {\MeasurablePart_0 = \MeasurablePart \cap \MeasurableSpace_0} \rbrace,
  \]
  e
  $(\MeasurableSpace_0,\SigmaAlgebra_0)$ \`e uno spazio misurabile.
\end{Theorem}
\Proof Poich\'e $\emptyset = \emptyset \cap \MeasurableSpace_0$,
$\emptyset \in \SigmaAlgebra_0$.
\par Sia $\MeasurablePart_0 \in \SigmaAlgebra_0$. Per opportuno
$\MeasurablePart \in \SigmaAlgebra$, abbiamo
$\MeasurablePart_0 = \MeasurablePart \cap \SigmaAlgebra_0$.
Poich\'e $\Complement{\MeasurablePart} \in \SigmaAlgebra$, anche
\begin{align*}
        \Complement{\MeasurablePart} \cap \MeasurableSpace_0,
        &= \MeasurableSpace_0 \SetMin \MeasurablePart,\\
        &=\MeasurableSpace_0 \SetMin (\MeasurablePart \cap \MeasurableSpace_0)
        \in \SigmaAlgebra_0.
\end{align*}
\par Sia
$(\MeasurablePart_{0,n})_{n \in \mathbb{N}}
\in \SigmaAlgebra_0^\mathbb{N}$.
Esiste
$(\MeasurablePart_n)_{n \in \mathbb{N}}
\in \SigmaAlgebra^\mathbb{N}$ tale che
\[
  \ForAll{n \in \mathbb{N}}
  {\MeasurablePart_{0,n} = \MeasurablePart_n \cap \MeasurableSpace_0}.
\]
Abbiamo
\begin{align*}
  \bigcup_{n \in \mathbb{N}} \MeasurablePart_{0,n}
  &= \bigcup_{n \in \mathbb{N}} (\MeasurablePart_n \cap \MeasurableSpace_0),\\
  &= \left ( \bigcup_{n \in \mathbb{N}} \MeasurablePart_n \right )
    \cap \MeasurableSpace_0
  \in \SigmaAlgebra_0.\text{ \EndProof}
\end{align*}

\section{Assoluta continuit\`a.}
\label{Misura_AssolutaContinuita}
\begin{Definition}
  Siano $\Measure$ e $\VarMeasure$ due misure definite sullo stesso spazio
  misurabile $(\MeasurableSpace,\SigmaAlgebra)$.
  $\Measure$ \`e
  \Define{assolutamente continua}[assolutamente continua][misura]
  rispetto a $\VarMeasure$, denotato
  $\AbsoluteContinuousMeasure{\Measure}{\VarMeasure}$
  quando
  $\ForAll{\MeasurablePart \in \SigmaAlgebra}{
    \Implies{\VarMeasure(\MeasurablePart) = 0}
    {\Measure(\MeasurablePart) = 0}}$.
  Qualora si abbia simultaneamente
  $\AbsoluteContinuousMeasure{\Measure}{\VarMeasure}$
  e
  $\AbsoluteContinuousMeasure{\VarMeasure}{\Measure}$
  si dice che
  $\Measure$ e $\VarMeasure$ sono misure
  \Define{equivalenti}[equivalenti][misure].
\end{Definition}

\section{Misura di Lebesgue.}
\label{Misura_MisuraDiLebesgue}
\begin{Theorem}
  Le parti misurabili secondo Lebesgue costituiscono una $\sigma$-algebra.
\end{Theorem}
\begin{Definition}
  La $\sigma$-algebra di tutti le parti misurabili secondo Lebesgue, d'ora in
  poi denotata $\LebesgueSigmaAlgebra$, si chiama
  \Define{$\sigma$-algebra di Lebesgue}[di Lebesgue][$\sigma$-algebra].
\end{Definition}
\begin{Definition}
  Una misura si dice \Define{completa}[completa][misura] se ogni parte di una
  parte trascurabile \`e misurabile.
\end{Definition}
\begin{Theorem}
  La misura di Lebesgue coindice con la misura di Borel su tutti gli elementi
  della $\sigma$-algebra di Borel.
\end{Theorem}
\begin{Theorem}
  La misura di Lebesgue \`e il completamento della misura di Borel.
\end{Theorem}

\section{Misure definite da densit\`a.}
\label{Misura_MisureDefiniteDaDensita}
\begin{Definition}
	Siano dati uno spazio misurabile $(\MeasurableSpace,\SigmaAlgebra)$ munito
  della misura $\Measure$ e un'applicazione misurabile
  $\DensityFunction: \MeasurableSpace \rightarrow [0,+\infty]$. Chiamiamo
  $\VarMeasure: \SigmaAlgebra \rightarrow [0, +\infty]$ tale che
  $\VarMeasure(A) = \int_A f d\Measure$
  \Define{misura definita dalla densit\`a $\DensityFunction$ rispetto a
  $\Measure$}:
  esprimiamo ci\`o anche con la notazione
  $\VarMeasure = \DensityDefinedProbability{\DensityFunction}{\Measure}$
\end{Definition}
\begin{Theorem}
	Una misura definita da densit\`a \`e effettivamente una misura.
\end{Theorem}
\begin{Theorem}
	Siano $\VarMeasure$ e $\Measure$ due misure su
  $(\MeasurableSpace,\SigmaAlgebra)$ e supponiamo esistano
  $\DensityFunction$ e $\VarDensityFunction$
  ($\DensityFunction, \VarDensityFunction:
  \MeasurableSpace \rightarrow [0,+\infty]$) applicazioni misurabili tali che
  $\VarMeasure
    = \DensityDefinedProbability{\DensityFunction}{\Measure}
    = \DensityDefinedProbability{\VarDensityFunction}{\VarMeasure}$, allora
  $\DensityFunction$ e $\VarDensityFunction$ sono equivalenti.
\end{Theorem}
\begin{Definition}
	Con le notazioni del teorema precedene, diciamo che $\DensityFunction$ \`e
  \Define{una versione della densit\`a di $\VarMeasure$ rispetto a $\Measure$} e
  denotiamo $\DensityFunction = \DensityVersion{\VarMeasure}{\Measure}$.
\end{Definition}
\par Osserviamo che la simbologia introdotta dalla definizione precedente \`e un
abuso di notazione:
$\DensityVersion{\VarMeasure}{\Measure}$ \`e in realt\`a una classe di
equivalenza di funzioni e non una sola funzione e sarebbe dunque pi\`u giusto
scrivere
$\DensityFunction \in \DensityVersion{\VarMeasure}{\Measure}$.


  \chapter{Teoria degli integrali}
  \section{Definizioni e teoremi di base.}
\label{Integrali_DefinizioniETeoremiDiBase}
\begin{Theorem}
\label{Misura_th_ApplicazioneCaratteristica}
  Siano
  \begin{itemize}
    \item $(\MeasureSpace,\SigmaAlgebra,\Measure)$ uno spazio di misura;
    \item $\MeasurablePart \in \SigmaAlgebra$;
    \item $\CharacteristicApplication{\MeasurablePart}:
      \MeasurableSpace \rightarrow \mathbb{R}$
      l'applicazione caratteristica
      di $\MeasurablePart$;
    \item $c \in \mathbb{R}$.
  \end{itemize}
  L'applicazione $c \CharacteristicApplication{\MeasurablePart}$ \`e misurabile.
\end{Theorem}
\Proof Se $c = 0$, allora $c \CharacteristicApplication{\MeasurablePart}$ \`e
costante e dunque misurabile.
\par Assumiamo $c \neq 0$ e sia $\VarMeasurablePart \subseteq \mathbb{R}$
misurabile.
\par Ora,
\begin{itemize}
  \item se $0, c \notin \VarMeasurablePart$, allora
    $\CharacteristicApplication{\MeasurablePart}^{-1} (\VarMeasurablePart)
    = \emptyset \in \SigmaAlgebra$;
  \item se $0 \notin \VarMeasurablePart$ e $c \in \VarMeasurablePart$, allora
    $\CharacteristicApplication{\MeasurablePart}^{-1} (\VarMeasurablePart)
    = \MeasurablePart \in \SigmaAlgebra$;
  \item se $0 \in \VarMeasurablePart$ e $c \notin \VarMeasurablePart$, allora
    $\CharacteristicApplication{\MeasurablePart}^{-1} (\VarMeasurablePart)
    = \Complement{\MeasurablePart} \in \SigmaAlgebra$;
  \item se $0, c \in \VarMeasurablePart$, allora
    $\CharacteristicApplication{\MeasurablePart}^{-1} (\VarMeasurablePart)
    = \MeasureSpace \in \SigmaAlgebra$. \EndProof
\end{itemize}
\begin{Definition}
  Siano
  \begin{itemize}
    \item $(\MeasureSpace,\SigmaAlgebra,\Measure)$ uno spazio di misura;
    \item $\MeasurablePart \in \SigmaAlgebra$.
  \end{itemize}
  L'\Define{integrale}[di un'applicazione caratteristica][integrale]
  dell'applicazione caratteristica
  $\CharacteristicApplication{\MeasurablePart}:
  \MeasurableSpace \rightarrow \mathbb{R}$
  di $\MeasurablePart$, denotato
  $\int \CharacteristicApplication{\MeasurablePart} d\Measure$, \`e 
  $\int \CharacteristicApplication{\MeasurablePart} d\Measure
  = \Measure(\MeasurablePart)$.
\end{Definition}
\begin{Definition}
  Siano
  \begin{itemize}
    \item $(\MeasureSpace,\SigmaAlgebra,\Measure)$ uno spazio di misura;
    \item $N \in \mathbb{N}$;
    \item $(\MeasurablePart_n)_{n \in N} \in \SigmaAlgebra^N$;
    \item $(\Scalar_n)_{n \in N} \in (\NotZero{\mathbb{C}})^N$
    \item $\MeasurableApplication
      = \sum_{n \in N} \Scalar_n\CharacteristicApplication{\MeasurablePart_n}$.
  \end{itemize}
  Diciamo che $\MeasurableApplication$ \`e
  un'\Define{applicazione semplice}[semplice][applicazione]
  o
  una \Define{funzione semplice}[semplice][funzione].
\end{Definition}
\begin{Theorem}
  Con le notazioni della definizione precedente,
  assumendo $\MeasurableApplication$ reale,
  l'applicazione semplice
  $\MeasurableApplication$ \`e misurabile.
\end{Theorem}
\Proof Per ogni $c \in \Image{\MeasurableApplication}$, definiamo
$S_c
= \lbrace X \in \PowerSet{N} | \sum_{n \in N} \Scalar_n = c \rbrace$.
\par Sia $\VarMeasurablePart \in \LebesgueSigmaAlgebra$.
\par Abbiamo
\begin{align*}
  \MeasurableApplication^{-1}(\VarMeasurablePart)
  &= \MeasurableApplication^{-1}
    \left ( \VarMeasurablePart \cap \Image{\MeasurableApplication} \right ),\\
  &= \MeasurableApplication^{-1}
    \left ( \bigcup_{c \in \VarMeasurablePart \cap \Image{\MeasurableApplication}}
    \lbrace c \rbrace \right ),\\
  &= \bigcup_{c \in \VarMeasurablePart \cap \Image{\MeasurableApplication}}
    \MeasurableApplication^{-1}(\lbrace c \rbrace).
\end{align*}
\par Basta dunque provare
\[
  \ForAll{c \in \Image{\MeasurableApplication}}
  {\MeasurableApplication^{-1}(c) \in \SigmaAlgebra}.
\]
\par Abbiamo
$\MeasurableApplication^{-1}(\lbrace c \rbrace)
= \bigcup_{X \in S_C} \bigcap_{n \in X} \MeasurablePart_n \in \SigmaAlgebra$.
\EndProof
\begin{Theorem}
  Con le notazioni della definizione precedente,
  assumendo $\MeasurableApplication$ reale,
  se
  \begin{itemize}
    \item $M \in \mathbb{N}$;
    \item $(\VarMeasurablePart_m)_{m \in M} \in \SigmaAlgebra^M$;
    \item $(\VarScalar_m)_{m \in M} \in (\NotZero{\mathbb{R}})^M$;
    \item $\MeasurableApplication
      = \sum_{m \in M}
        \VarScalar_m\CharacteristicApplication{\VarMeasurablePart_m}$,
  \end{itemize}
  allora
  $\sum_{n \in N} \Scalar_n \Measure{\MeasurablePart_n}
    = \sum_{m \in M} \VarScalar_m \Measure{\VarMeasurablePart_m}$.
\end{Theorem}
\Proof Procediamo per doppia induzione su $N$ e $M$.
\par Siano $N = M = 1$.
\par Abbiamo 
\begin{align*}
  \MeasurableApplication
  &= \Scalar_0 \CharacteristicApplication{\MeasurablePart_0},
  &= \VarScalar_0 \CharacteristicApplication{\VarMeasurablePart_0},
\end{align*}
da cui
\begin{align*}
  \Image{\MeasurableApplication}
  &= \lbrace \Scalar_0 \rbrace,
  &= \lbrace \VarScalar_0 \rbrace,
\end{align*}
e quindi $\Scalar_0 = \VarScalar_0$.
\par Abbiamo inoltre
\begin{align*}
  \MeasurableApplication^{-1}(\Scalar_0)
  &= \MeasurablePart_0,\\
  &= \VarMeasurablePart_0.
\end{align*}
Dunque
$\MeasurablePart_0 = \VarMeasurablePart_0$
e infine
$\Scalar_0 \Measure{\MeasurablePart_0}
= \VarScalar_0 \Measure{\VarMeasurablePart_0}$.
%INCOMPLETA
\begin{Definition}
  Con le notazioni della definizione precedente,
  assumendo $\MeasurableApplication$ reale,
  l'\Define{integrale}[di un'applicazione semplice][integrale]
  dell'applicazione semplice 
  $\MeasurableApplication$
  $\int \MeasurableApplication d\Measure$, \`e 
  $\int \MeasurableApplication d\Measure
    = \sum_{n \in N} \Scalar_n
      \int \CharacteristicApplication{\MeasurablePart_n}d\Measure
    = \sum_{n \in N} \Scalar_n \MeasurablePart_n$.
\end{Definition}
\begin{Theorem}
  Siano
  \begin{itemize}
    \item $(\MeasureSpace,\SigmaAlgebra,\Measure)$ uno spazio di misura;
    \item $(\VarMeasureSpace,\VarSigmaAlgebra)$ uno spazio misurabile;
    \item $\MeasurableApplication: \MeasureSpace \rightarrow \VarMeasureSpace$
      un'applicazione misurabile;
    \item $\VarMeasure$ la misura immagine di $\Measure$ tramite
      $\MeasurableApplication$;
    \item $\VarMeasurableApplication:
      \VarMeasureSpace \rightarrow \mathbb{R}$
      un'applicazione misurabile.
  \end{itemize}
  $\VarMeasurableApplication$ \`e integrabile rispetto a
  $\VarMeasure$ se e solo se
  $\VarMeasurableApplication \circ \MeasurableApplication$
  \`e integrabile rispetto a
  $\Measure$ e in tal caso vale
  $\int \VarMeasurableApplication d\VarMeasure
    = \int (\VarMeasurableApplication \circ \MeasurableApplication)  d\Measure$.
\end{Theorem}

\section{Momenti.}
\label{Integrali_Momenti}
\begin{Definition}
	Sia $f: \mathbb{R} \rightarrow \mathbb{R}$ una funzione continua e $c \in \mathbb{R}$. Per ogni $n \in \mathbb{N}$ definiamo \Define{momento semplice}[di una funzione reale continua][momento] o anche solamente \Define{momento}[di una funzione reale continua][momento] di origine $c$ e ordine $n$ di $f$ la quantit\`a
\[
	\int_{-\infty}^{+\infty} (x - c)^n f(x) dx.
\]
	Qualora non venga specificata l'origine $c$, si sottointende $c = 0$.
	Chiamiamo inoltre
	\begin{itemize}
		\item \Define{media}[di una funzione reale continua][media] il primo momento di $f$;
		\item \Define{momento centrale}[centrale][momento] di $f$ il momento semplice di $f$ di ordine $n$ e di origine la media di $f$;
		\item \Define{varianza} il secondo momento di $f$ di origine la media di $f$.
	\end{itemize}
\end{Definition}


	\chapter{Studio di alcune funzioni notevoli}
	\section{Serie di potenze e polinomi}
\label{FunzioniNotevoli_SerieDiPotenzePolinomi}
\begin{Theorem}
	Sia $n \in \mathbb{N}$, sia $\Monomial{m}(x) = x^n \in \RingAdjunction{\mathbb{C}}{x}$. $\Monomial{m}$ \`e derivabile ovunque in $\mathbb{C}$ e abbiamo
	\[
		\Derivative{\Monomial}(x) =
		\begin{cases}
			x^{n - 1}&\text{ se }n > 0,\\
			0&\text{ altrimenti}.
		\end{cases}
	\]
\end{Theorem}
\Proof Per $n > 1$, abbiamo
\begin{align*}
	\Derivative{\Polynomial}
	&= \lim_{h \rightarrow 0} \frac{(x + h)^n - x^n}{h},\\
	&= \lim_{h \rightarrow 0} \frac{\sum_{k = 1}^n \binom{n}{k} x^{n - k}h^k}{h},\\
	&= \lim_{h \rightarrow 0} \left ( \sum_{k = 1}^n \binom{n}{k} x^{n - k}h^{k - 1} \right ),\\
	&= n x^{n - 1}.
\end{align*}
\par Se invece $n = 0$, allora $\Monomial{m}$ \`e costante e dunque $\Derivative{\Monomial{m}} = 0$. \EndProof
\begin{Corollary}
	Fissato $n \in \mathbb{N}$, sia $\Polynomial(x) = \sum_{k \in n} a_k x^k \in \RingAdjunction{C}{x}$. $\Polynomial$ \`e derivabile ovunque in $\mathbb{C}$ e abbiamo
	\[
		\Derivative{\Polynomial}(x) =
		\begin{cases}
			\sum_{k \in \NotZero{n - 1}} k a_k x^{n - 1}&\text{ se }n > 0,\\
			0&\text{ altrimenti}.
		\end{cases}
	\]
\end{Corollary}
\Proof Segue direttamente dal teorema precedente per la linearit\`a dell'operatore di derivazione. \EndProof
\begin{Corollary}
	Sia $s(X) = \sum_{k \in \mathbb{N}} a_k X^k \in \PowerSeries{\mathbb{C}}{X}$. Se $\xi \in \mathbb{C}$ appartiene al cerchio di convergenza, allora $s(X)$ \`e derivabile in $\xi$ e
	\[
	\Derivative{s}(\xi) = \sum_{k \in \NotZero{\mathbb{N}}} n a_k \xi^{k - 1},
	\]
	vale a dire, il valore della derivata di $s$ in $\xi$ coincide col valore della derivata formale di $s$ in $\xi$.
\end{Corollary}
\Proof Segue direttamente dal teorema \ref{SuccessioniESerie_ConvergenzaDerivataFormale} e dal teorema precedente. \EndProof
\begin{Theorem}
	Siano
  \begin{itemize}
    \item $n \in \NotZero{\mathbb{N}}$;
    \item $s(X) = \sum_{k \in \mathbb{N}} a_k X^k
                \in \PowerSeries{\mathbb{C}}{X}$;
    \item $\RadiusOfConvergence \in \RealNonNegative$ il raggio di convergenza
      di $s(X)$;
    \item $\Matrix \in \mathbb{C}^{n \times n}$.
  \end{itemize}
  Se $\SpectralRadius{\Matrix} < \RadiusOfConvergence$, allora la serie
  $\sum_{k \in \mathbb{N}} a_k \Matrix^k$ \`e convergente.
\end{Theorem}

\section{Funzione esponenziale.}
\label{FunzioniNotevoli_FunzioneEsponenziale}
\begin{Theorem}
	Sia
  $\Exponential(X) =
  \sum_{n \in \mathbb{N}} \frac{X^n}{n!}
  \in \PowerSeries{\mathbb{C}}{X}$. 
  $\Exponential(X)$ converge ovunque in $\mathbb{C}$.
\end{Theorem}
\begin{Definition}
	La funzione $\Exponential: \mathbb{C} \rightarrow \mathbb{C}$ che a
  $z \in \mathbb{C}$ associa la somma della serie
  $\Exponential(z) \in \mathbb{C}$ si chiama
  \Define{esponenziale}.
  Chiamiamo $\Exponential(1)$
  \Define{numero di Nepero}[di Nepero][numero],
  denotato $e$.
  Denoteremo il valore della funzione esponenziale in un punto
  $z \in \mathbb{C}$ come $\Exponential(z)$ o anche $e^z$.
\end{Definition}
\begin{Theorem}
	L'esponziale verifica la relazione
	\[
		\ForAll{(z,w) \in \mathbb{C}^2}{\Exponential(z + w) = \Exponential(z)\Exponential(w)}.
	\]
\end{Theorem}
\Proof Sia $(z,w) \in \mathbb{C}^2$. Abbiamo
\begin{align*}
	\Exponential(z + w)
	&= \sum_{n \in \mathbb{N}} \frac{(z + w)^n}{n!},\\
	&= \sum_{n \in \mathbb{N}} \sum_{k = 0}^n \frac{\binom{n}{k} z^k w^{n - k}}{n!},\\
	&= \sum_{n \in \mathbb{N}} \sum_{k = 0}^n \frac{n! z^k w^{n - k}}{)n - k)!k!n!},\\
	&= \sum_{n \in \mathbb{N}} \sum_{k = 0}^n \frac{z^k}{k!} \frac{w^(n - k)}{n - k!},\\
	&= \left ( \sum_{n \in \mathbb{N}} \frac{z^n}{n!} \right ) \left ( \sum_{n \in \mathbb{N}} \frac{w^n}{n!} \right ),\\
	&= \Exponential(z)\Exponential(w).\text{ \EndProof}
\end{align*}
\begin{Theorem}
  Abbiamo
  \[
    \frac{d\Exponential}{dz}(z) = \Exponential(z).
  \]
\end{Theorem}
\Proof Derivando la serie formale che definisce l'esponenziale otteniamo
\begin{align*}
  \frac{d\Exponential}{dz}(z)
    &= \sum_{n \in \NotZero{\mathbb{N}}} n \frac{z^{n - 1}}{n!},\\
    &= \sum_{n \in \NotZero{\mathbb{N}}} \frac{z^{n - 1}}{(n - 1)!},\\
    &= \sum_{n \in \mathbb{N}} \frac{z^n}{n!},\\
    &= \Exponential(z).\text{ \EndProof}
\end{align*}
\begin{Theorem}
	Sia
  $\Exponential(X) =
  \sum_{n \in \mathbb{N}} \frac{X^n}{n!}
  \in \PowerSeries{\mathbb{C}}{X}$. 
  Per ogni $n \in \NotZero{\mathbb{N}}$ e per ogni
  $\Matrix \in \mathbb{C}^{n \times n}$,
  $\Exponential(\Matrix)$ converge.
\end{Theorem}
\Proof Segue direttamente dal fatto il raggio spettrale di una matrice \`e
finito, mentre il raggio di convergenza di $\Exponential(X)$ \`e $+\infty$.
\EndProof
\begin{Definition}
	Fissato $n \in \NotZero{\mathbb{N}}$, la funzione
  $\Exponential: \mathbb{C}^{n \times n} \rightarrow \mathbb{C}^{n \times n}$
  che a
  $\Matrix \in \mathbb{C}^{n \times n}$ associa la somma della serie
  $\Exponential(\Matrix) \in \mathbb{C}^{n \times n}$
  si chiama
  \Define{esponenziale di matrice}[di matrice][esponenziale].
  Denoteremo il valore della funzione esponenziale di matrice in un punto
  $\Matrix \in \mathbb{C}^{n \times n}$ come $\Exponential(\Matrix)$ o
  anche $e^\Matrix$.
\end{Definition}
\begin{Theorem}
  Siano $A, B \in \mathbb{C}^{n \times n}$ ($n \in \NotZero{\mathbb{N}}$) tali che $AB = BA$.
  Allora $\Exponential(A + B) = \Exponential(B) \Exponential(C)$.
\end{Theorem}
\Proof Abbiamo, per il teorema del binomio di Newton,
\begin{align*}
  \Exponential(A + B)
  &= \sum_{k \in \mathbb{N}} \frac{(A + B)^k}{k!},\\
  &= \sum_{k \in \mathbb{N}} \frac{\sum_{l = 0}^k \binom{k}{l} A^lB^{k - l}}{k!},\\
  &= \sum_{k \in \mathbb{N}} \frac{\sum_{l = 0}^k \frac{k!}{l!(k - l)!} A^lB^{k - l}}{k!},\\
  &= \sum_{k \in \mathbb{N}} \sum_{l = 0}^k \frac{A^l}{l!}\frac{B^{k - l}}{(k - l)!},\\
  &= \left ( \sum_{l \in \mathbb{N}} \frac{A^l}{l!} \right )
      \left ( \sum_{k \in \mathbb{N}} \frac{B^k}{k!} \right ),\\
  &= \Exponential(A) \Exponential(B).\text{ \EndProof}
\end{align*}
\begin{Theorem}
  Siano
  \begin{itemize}
    \item $n \in \NotZero{\mathbb{C}}$;
    \item $\Matrix \in \mathbb{C}^{n \times n}$;
    \item $\Norm{\cdot}$ una norma di matrice su $\mathbb{C}^{n \times n}$.
  \end{itemize}
  Vale la disuguaglianza
  \[
    \Norm{\Exponential{\Matrix}} \leq \Exponential{\Norm{\Matrix}}.
  \]
\end{Theorem}
\Proof Abbiamo
\begin{align*}
  \Norm{\Exponential{\Matrix}}
  &= \Norm{\sum_{k \in \mathbb{N}} \frac{\Matrix^k}{k!}},\\
  &\leq \sum_{k \in \mathbb{N}} \frac{\Norm{\Matrix}^k}{k!},\\
  &= \Exponential{\Norm{\Matrix}}.\text{ \EndProof}
\end{align*}
\begin{Theorem}
  Siano
  \begin{itemize}
    \item $n \in \NotZero{\mathbb{N}}$;
    \item $\Matrix \in \mathbb{C}^{n \times n}$;
    \item $f: \mathbb{C} \rightarrow \mathbb{C}^{n \times n}$ tale che
      $f(z) = \Exponential(z\Matrix)$.
  \end{itemize}
  Abbiamo $f'(z) = \Matrix \Exponential(z \Matrix)$.
\end{Theorem}
\Proof Abbiamo
\begin{align*}
  f'(z)
    &= \sum_{k \in \NotZero{\mathbb{N}}} k \frac{z^{k - 1} \Matrix^k}{k!},\\
    &= \Matrix \sum_{k \in \NotZero{\mathbb{N}}} \frac{z^{k - 1}\Matrix^{k - 1}}{(k - 1)!},\\
    &= \Matrix \sum_{k \in \mathbb{N}} \frac{z^k \Matrix^k}{k!},\\
    &= \Matrix \Exponential(z \Matrix).\text{ \EndProof}
\end{align*}

\section{Funzioni trigonometriche.}
\label{FunzioniNotevoli_FunzioniTrigonometriche}
\begin{Theorem}
	Siano
	\begin{itemize}
		\item $\cos(X)
          = \sum_{n \in \mathbb{N}} (-1)^n \frac{X^{2n}}{(2n)!}
          \in \PowerSeries{\mathbb{C}}{X}$;
		\item $\sin(X)
          = \sum_{n \in \mathbb{N}} (-1)^n \frac{X^{2n + 1}}{(2n + 1)!}
          \in \PowerSeries{\mathbb{C}}{X}$.
	\end{itemize}
	Abbiamo
	\begin{itemize}
		\item $\cos(X)$ converge ovunque in $\mathbb{C}$;
		\item $\sin(X)$ converge ovunque in $\mathbb{C}$.
	\end{itemize}
\end{Theorem}
\begin{Definition}
	Le funzioni $\cos$ e $\sin$ del teorema precedente si chiamano
  \Define{coseno} e \Define{seno} rispettivamente: le denoteremo sempre
  $\cos$ e $\sin$.
\end{Definition}
\par \`E molto comune l'abuso di notazione secondo cui, per
$z, \alpha \in \mathbb{C}$, denotiamo $\cos^\alpha{z}$ e
$\sin^\alpha{z}$ le quantit\`a $(\cos{z})^\alpha$ e $(\sin{z})^\alpha$
rispettivamente. Analoghe considerazioni varranno per le ulteriori
funzioni trigonometriche che introdurremo.
\par Noi adotteremo sempre questo abuso di linguaggio, salvo esplicite
indicazioni contrarie.
\begin{Theorem}
	Sia $z \in \mathbb{C}$. Abbiamo
  $e^z
  = e^\RealPart{z} (\cos \ImaginaryPart{z} + i \sin \ImaginaryPart{z})$.
\end{Theorem}
\Proof Abbiamo
$e^z
= e^{\RealPart{z} + i \ImaginaryPart{z}}
= e^{\RealPart{z}} e^{i \ImaginaryPart{z}}$.
\par Esaminiamo il fattore $e^{i \ImaginaryPart{z}}$: abbiamo
\begin{align*}
	e^{i \ImaginaryPart{z}}
	&= \sum_{n \in \mathbb{N}} \frac{i^n \ImaginaryPart{z}^n}{n!},\\
	&= \sum_{n \in \mathbb{N}}
	\left ( \frac{i^{4n} \ImaginaryPart{z}^{4n}}{(4n)!}
	+ \frac{i^{4n + 1} \ImaginaryPart{z}^{4n + 1}}{(4n + 1)!}
	+ \frac{i^{4n + 2} \ImaginaryPart{z}^{4n + 2}}{(4n + 2)!}
	+ \frac{i^{4n + 3} \ImaginaryPart{z}^{4n + 3}}{(4n + 3)!} \right ),\\
	&= \sum_{n \in \mathbb{N}}
	\left ( \frac{\ImaginaryPart{z}^{4n}}{(4n)!}
	+ \frac{i \ImaginaryPart{z}^{4n + 1}}{(4n + 1)!}
	+ \frac{- \ImaginaryPart{z}^{4n + 2}}{(4n + 2)!}
	+ \frac{- i \ImaginaryPart{z}^{4n + 3}}{(4n + 3)!} \right ),\\
	&= \left (
      \sum_{n \in \mathbb{N}} (-1)^n \frac{\ImaginaryPart{z}^{2n}}{(2n)!}
      \right )
      + i
      \left (
      \sum_{n \in \mathbb{N}}
        (-1)^n \frac{\ImaginaryPart{z}^{2n + 1}}{(2n + 1)!}
      \right ),\\
	&= \cos{\ImaginaryPart{z}} + i \sin{\ImaginaryPart{z}}.
  \text{ \EndProof}
\end{align*}
\begin{Corollary}
	Sia $z \in \mathbb{C}$. Abbiamo $\cos^2{z} + \sin^2{z} = 1$.
\end{Corollary}
\Proof Abbiamo $e^0 = 1$, ma anche
\begin{align*}
	e^0
	&= e^{z - z},\\
	&= e^z e^{-z},\\
	&= e^{\RealPart{z}}(\cos \ImaginaryPart{z} + i \sin \ImaginaryPart{z})
	  e^{- \RealPart{z}}(\cos \ImaginaryPart{z} - i \sin \ImaginaryPart{z})
	&= \cos^2{x} - i^2 \sin^2{x},\\
	&= \cos^2{x} + \sin^2{x}.
\end{align*}
\par Ne consegue $\cos^2{x} + \sin^2{x} = 1$. \EndProof
\begin{Definition}
  Definiamo \Define{tangente} la funzione
  $\tan: \mathbb{C}
    \SetMin \bigcup_{k \in \mathbb{Z}} \lbrace k\pi \rbrace
    \rightarrow \mathbb{C}$
  tale che $\tan{z} = \frac{\sin{z}}{\cos{z}}$.
CONTROLLARE DOMINIO: gli zeri di cos sono tutti reali?
\end{Definition}
\begin{Theorem}
  Siano $x, y \in \mathbb{R}$. Abbiamo
  \begin{itemize}
    \item $\cos{x + y} = \cos{x}\cos{y} - \sin{x}\sin{y}$;
    \item $\sin{x + y} = \sin{x}\cos{y} + \cos{x}\sin{y}$;
    \item $\tan{x + y} = \frac{\tan{x} + \tan{y}}{1 - \tan{x}\tan{y}}$.
  \end{itemize}
\end{Theorem}
\begin{Corollary}
  \TheoremName{Formule di duplicazione}[di duplicazione][formule]
  Sia $x \in \mathbb{R}$. Abbiamo
  \begin{itemize}
    \item $\cos{2x}
            = \cos^2{x} - \sin^2{x}
            = 1 - 2 \sin^2{x}
            = \cos^2{x} - 1$;
    \item $\sin{2x} = 2 \sin{x} \cos{x}$;
    \item $\tan{2x} = \frac{2 \tan{x}}{1 - \tan^2{x}}$.
  \end{itemize}
\end{Corollary}
\begin{Corollary}
  \TheoremName{Formule di bisezione}[di bisezione][formule]
  \begin{itemize}
    \item $\cos{\frac{x}{2}}
            = \pm \sqrt{\frac{1 + \cos{x}}{2}}$;
    \item $\sin{\frac{x}{2}}
            = \pm \sqrt{\frac{1 - \cos{x}}{2}}$;
    \item $\tan{\frac{x}{2}}
            = \pm \sqrt
                {\frac{1 - \cos{x}}{1 + \cos{x}}}
            = \frac{\sin{x}}{1 + \cos{x}}
            = \frac{1 - \cos{x}}{\sin{x}}$.
  \end{itemize}
\end{Corollary}
\begin{Theorem}
  \TheoremName{Formule di prostaferesi}[di prostaferesi][formule]
  Siano $x, y \in \mathbb{R}$, abbiamo
  \begin{itemize}
    \item $\sin x + \sin y
            = 2 \sin \frac{x + y}{2} \cos \frac{x - y}{2}$;
    \item $\sin x - \sin y
            = 2 \sin \frac{x - y}{2} \cos \frac{x + y}{2}$;
    \item $\cos x + \cos y
            = 2 \cos \frac{x + y}{2} \cos \frac{x - y}{2}$;
    \item $\cos x - \cos y
            = - 2 \sin \frac{x + y}{2} \sin \frac{x - y}{2}$;
  \end{itemize}
\end{Theorem}
\begin{Theorem}
  \TheoremName{Formule di Werner}[di Werner][formule]
  Siano $x, y \in \mathbb{R}$, abbiamo
  \begin{itemize}
    \item $\sin{x}\cos{y} = \frac{\sin(x + y) + \sin(x - y)}{2}$;
    \item $\cos{x}\cos{y} = \frac{\cos(x + y) + \cos(x - y)}{2}$;
    \item $\sin{x}\sin{y} = \frac{\cos(x - y) - \cos(x + y)}{2}$.
  \end{itemize}
\end{Theorem}
\begin{Theorem}
  Abbiamo
  \begin{itemize}
    \item $\frac{d\cos}{dz}(z) = - \sin z$;
    \item $\frac{d\sin}{dz}(z) = \cos z$;
    \item $\frac{d\tan}{dz}(z) = 1 - \tan^2 z = \frac{1}{\cos^2 z}$.
  \end{itemize}
\end{Theorem}
\Proof Derivando le serie formali che definiscono coseno e
seno otteniamo
\begin{align*}
  \frac{d\cos}{dz}(z)
    &= \sum_{n \in \NotZero{\mathbb{N}}} (-1)^n 2n \frac{z^{2n - 1}}{(2n)!},\\
    &= \sum_{n \in \NotZero{\mathbb{N}}} (-1)^n \frac{z^{2n - 1}}{(2n - 1)!},\\
    &= \sum_{n \in \mathbb{N}} (-1)^{n + 1} \frac{z^{2n + 1}}{(2n + 1)!},\\
    &= - \sum_{n \in \mathbb{N}} (-1)^n \frac{z^{2n + 1}}{(2n + 1)!},\\
    &= - \sin z.
\end{align*}
e
\begin{align*}
  \frac{d\sin}{dz}(z)
    &= \sum_{n \in \mathbb{N}} (-1)^n (2n + 1)\frac{z^{2n}}{(2n + 1)!},\\
    &= \sum_{n \in \mathbb{N}} (-1)^n \frac{z^{2n}}{(2n)!},\\
    &= \cos z.
\end{align*}
\par Usando infine la formula della derivata di un rapporto di funzioni,
\begin{align*}
  \frac{d\tan}{dz}(z)
    &= \frac{\cos z \cos z + \sin z \sin z}{\cosh^2 z},\\
    &= 1 - \tan z,\\
    &= \frac{1}{\cos^2 z}.\text{ \EndProof}
\end{align*}

\section{Funzioni iperboliche.}
\label{FunzioniNotevoli_FunzioniIperboliche}
\begin{Theorem}
	Siano
	\begin{itemize}
		\item $\cosh(X)
          = \sum_{n \in \mathbb{N}} \frac{X^{2n}}{(2n)!}
          \in \PowerSeries{\mathbb{C}}{X}$;
		\item $\sinh(X)
          = \sum_{n \in \mathbb{N}} \frac{X^{2n + 1}}{(2n + 1)!}
          \in \PowerSeries{\mathbb{C}}{X}$.
	\end{itemize}
	Abbiamo
	\begin{itemize}
		\item $\cosh(X)$ converge ovunque in $\mathbb{C}$;
		\item $\sinh(X)$ converge ovunque in $\mathbb{C}$.
	\end{itemize}
\end{Theorem}
\begin{Definition}
	Le funzioni $\cosh$ e $\sinh$ del teorema precedente si chiamano
  \Define{coseno iperbolico}[iperbolico][coseno] e
  \Define{seno}[iperbolico][seno] rispettivamente: le denoteremo sempre
  $\cosh$ e $\sinh$.
\end{Definition}
\par \`E molto comune l'abuso di notazione secondo cui, per
$z, \alpha \in \mathbb{C}$, denotiamo $\cosh^\alpha{z}$ e
$\sinh^\alpha{z}$ le quantit\`a $(\cosh{z})^\alpha$ e $(\sinh z)^\alpha$
rispettivamente. Analoghe considerazioni varranno per le ulteriori
funzioni iperboliche che introdurremo.
\par Noi adotteremo sempre questo abuso di linguaggio, salvo esplicite
indicazioni contrarie.
\begin{Theorem}
  Sia $z \in \mathbb{C}$. Abbiamo
  \begin{itemize}
    \item $\cosh z
            = \cos iz
            = \frac{e^z + e^{-z}}{2}$;
    \item $\sinh z
            = - i \sin iz
            = \frac{e^z - e^{-z}}{2}$.
  \end{itemize}
\end{Theorem}
\Proof Per il coseno iperbolico abbiamo
\begin{align*}
  \cosh z
  &= \sum_{n \in \mathbb{N}} \frac{z^{2n}}{(2n)!},\\
  &= \sum_{n \in \mathbb{N}} (-1)^{n} \frac{(iz)^{2n}}{(2n)!},\\
  &= \cos iz.
\end{align*}
\par Inoltre
\begin{align*}
  \frac{e^z + e^{-z}}{2}
  &= \frac{1}{2}
      \sum_{n \in \mathbb{N}} \frac{z^n + (-z)^n}{n!},\\
  &= \frac{1}{2}
      \sum_{n \in \mathbb{N}} \frac{2z^{2n}}{(2n)!},\\
  &= \sum_{n \in \mathbb{N}} \frac{z^{2n}}{(2n)!},\\
  &= \cosh z.
\end{align*}
\par Per il seno iperbolico abbiamo
\begin{align*}
  \sinh z
  &= \sum_{n \in \mathbb{N}} \frac{z^{2n + 1}}{(2n + 1)!},\\
  &= \sum_{n \in \mathbb{N}} (-1)^{n} i \frac{(iz)^{2n + 1}}{(2n + 1)!},\\
  &= - i \sin iz.
\end{align*}
\par Inoltre
\begin{align*}
  \frac{e^z - e^{-z}}{2}
  &= \frac{1}{2}
      \sum_{n \in \mathbb{N}} \frac{z^n - (-z)^n}{n!},\\
  &= \frac{1}{2}
      \sum_{n \in \mathbb{N}} \frac{2z^{2n + 1}}{(2n + 1)!},\\
  &= \sum_{n \in \mathbb{N}} \frac{z^{2n + 1}}{(2n + 1)!},\\
  &= \sinh z.\text{ \EndProof}
\end{align*}
\begin{Corollary}
	Sia $z \in \mathbb{C}$. Abbiamo $\cosh^2 z - \sinh^2 z = 1$.
\end{Corollary}
\Proof Abbiamo
\begin{align*}
\cosh^2 z - \sinh^2 z
&= \left ( \frac{e^z + e^{-z}}{2} \right)^2
  - \left ( \frac{e^z - e^{-z}}{2} \right)^2,\\
&= \frac{e^{2z} + 2 e^{z-z} + e^{-2z} - e^{2z} + 2 e^{z-z} - e^{-2z}}
    {4}
&= e^0,\\
&= 1.
\end{align*}
\begin{Definition}
  Definiamo \Define{tangente iperbolica}[iperbolica][tangente] la
  funzione
  $\tanh: \mathbb{C}
    \rightarrow \mathbb{C}$
  tale che $\tanh{z} = \frac{\sinh{z}}{\cosh{z}}$.
CONTROLLARE DOMINIO: quali sono gli zeri di cosh?
\end{Definition}
\begin{Theorem}
  Abbiamo
  \begin{itemize}
    \item $\frac{d\cosh}{dz}(z) = \sinh z$;
    \item $\frac{d\sinh}{dz}(z) = \cosh z$;
    \item $\frac{d\tanh}{dz}(z) = 1 - \tanh^2 z = \frac{1}{\cosh^2 z}$.
  \end{itemize}
\end{Theorem}
\Proof Derivando le serie formali che definiscono coseno iperbolico e
seno iperbolico, otteniamo
\begin{align*}
  \frac{d\cosh}{dz}(z)
    &= \sum_{n \in \NotZero{\mathbb{N}}} 2n \frac{z^{2n - 1}}{(2n)!},\\
    &= \sum_{n \in \NotZero{\mathbb{N}}} \frac{z^{2n - 1}}{(2n - 1)!},\\
    &= \sum_{n \in \mathbb{N}} \frac{z^{2n + 1}}{(2n + 1)!},\\
    &= \sinh z.
\end{align*}
e
\begin{align*}
  \frac{d\sinh}{dz}(z)
    &= \sum_{n \in \mathbb{N}} (2n + 1)\frac{z^{2n}}{(2n + 1)!},\\
    &= \sum_{n \in \mathbb{N}} \frac{z^{2n}}{(2n)!},\\
    &= \cosh z.
\end{align*}
\par Usando infine la formula della derivata di un rapporto di funzioni,
\begin{align*}
  \frac{d\tanh}{dz}(z)
    &= \frac{\cosh z \cosh z - \sinh z \sinh z}{\cosh^2 z},\\
    &= 1 - \tanh z,\\
    &= \frac{1}{\cosh^2 z}.\text{ \EndProof}
\end{align*}


  \chapter{Analisi complessa}
  \section{Definizioni e teoremi di base.}
\label{AnalisiComplessa_DefinizioniETeoremiDiBase}
\begin{Definition}
  Siano
  \begin{itemize}
    \item $\Open \subseteq \mathbb{C}$ aperto;
    \item $z_0 \in \Open$;
    \item $\Holomorphic: \Open \rightarrow \mathbb{C}$.
  \end{itemize}
  $f$ si dice
  \begin{itemize}
    \item \Define{olomorfa}[olomorfa][funzione]
      quando \`e derivabile in ogni punto di $\Open$;
    \item \Define{olomorfa in $z_0$}[olomorfa in un punto][funzione]
      quando \`e olomorfa in un intorno di $z_0$.
  \end{itemize}
\end{Definition}
\begin{Definition}
  Siano
  \begin{itemize}
    \item $\Open \subseteq \mathbb{C}$ aperto;
    \item $\Singularity \in \Open$;
    \item $\Holomorphic: \Open \SetMin \lbrace \Singularity \rbrace
            \rightarrow \mathbb{C}$ olomorfa.
  \end{itemize}
  $\Singularity$ \`e una
  \Define{singolarit\`a isolata}[isolata][singolarit\`a]
  di $\Holomorphic$.
  $\Singularity$ \`e
  \begin{itemize}
    \item una \Define{singolarit\`a eliminabile}[elimibabile][singolarit\`a] se
      $\Holomorphic(z)$ converge per $z \rightarrow \Singularity$;
    \item un \Define{polo}[di una funzione olomorfa][polo]
      se
      $\lim_{z \rightarrow \Singularity}
        \AbsoluteValue{\Holomorphic(z)} = + \infty$;
    \item una \Define{singolarit\`a essenziale}[essenziale][singolarit\`a]
      se non esiste il limite di $\AbsoluteValue{\Holomorphic(z)}$ per
      $z \rightarrow \Singularity$.
  \end{itemize}
\end{Definition}
\begin{Definition}
  Siano
  \begin{itemize}
    \item $\Open \subseteq \mathbb{C}$ aperto;
    \item $\Part \subseteq \Open$ costituito da punti isolati;
    \item $\Meromorphic: \Open \SetMin \Part \rightarrow \mathbb{C}$;
    \item $z_0 \in \Open$.
  \end{itemize}
  $f$ si dice
  \begin{itemize}
    \item \Define{meromorfa}[meromorfa][funzione]
      quando \`e olomorfa in ogni punto di $\Open$ tranne che nei
      punti di $\Part$ che sono tutti poli;
    \item \Define{meromorfa in $z_0$}[meromorfa in un punto][funzione]
      quando \`e meromorfa in un intorno di $z_0$.
  \end{itemize}
\end{Definition}
\begin{Theorem}
  Siano
  \begin{itemize}
    \item $\Open \subseteq \mathbb{C}$ aperto;
    \item $\Holomorphic: \Open \rightarrow \mathbb{C}$.
  \end{itemize}
  $\Holomorphic$ \`e olomorfa se e solo se \`e analitica.
\end{Theorem}
\begin{Theorem}
  Siano
  \begin{itemize}
    \item $n \in \NotZero{\mathbb{N}}$;
    \item $\Matrix \in \mathbb{C}^{n \times n}$ diagonalizzabile;
    \item $\Analytical: \mathbb{C} \rightarrow \mathbb{C}$ analitica.
  \end{itemize}
  $\Eigenvalue \in \Spectrum{\Matrix}$ se e solo se
  $\Analytical(\Eigenvalue) \in \Spectrum{\Analytical(\Matrix)}$.
\end{Theorem}
\Proof Sia $\SpectralBaseChange$ una matrice che diagonalizza $\Matrix$.
$\SpectralBaseChange$ diagonalizza anche $\Analytical{\Matrix}$:
\begin{align*}
  \SpectralBaseChange\Analytical(\Matrix)\SpectralBaseChange^{-1}
  &=\SpectralBaseChange
  \left (
    \sum_{k \in \mathbb{N}} \frac{\Analytical^{(k)}(\Matrix)\Matrix^k}{k!}
  \right )
  \SpectralBaseChange^{-1},\\
  &=
  \left (
    \sum_{k \in \mathbb{N}} \frac{\Analytical^{(k)}(
    (\SpectralBaseChange\Matrix\SpectralBaseChange^{-1})^k)}
    {k!}
  \right ),\\
  &= \Analytical(\SpectralBaseChange\Matrix\SpectralBaseChange^{-1}).
\end{align*}
\par Ne consegue che gli elementi diagonali di
$\SpectralBaseChange\Analytical(\Matrix)\SpectralBaseChange^{-1}$,
vale a dire gli autovalori di $\Analytical(\Matrix)$, sono esattamente le
immagini degli elementi diagonali di
$\SpectralBaseChange\Matrix\SpectralBaseChange^{-1}$.
vale a dire degli autovalori di $\Matrix$. \EndProof


  \chapter{Teoria spettrale}
  \section{Autovettori e autovalori.}\label{AutovettoriEAutovalori}
\begin{Definition}
	Sia $\LinearTransformation: \LinearSpace \rightarrow \LinearSpace$ un
  endomorfismo del $\Field$-spazio vettoriali $\LinearSpace$.
  Se $\Eigenvector \in \LinearSpace \SetMin \lbrace 0 \rbrace$ \`e tale che per
  $\Eigenvalue \in \Field$ opportuno abbiamo
  \[
    \LinearTransformation(\Eigenvector) = \Eigenvalue \Eigenvector,
  \]
  allora chiamiamo $\Eigenvalue$ \Define{autovalore} di
  $\LinearTransformation$ e $\Eigenvector$ \Define{autovettore} di
  $\LinearTransformation$ relativo a $\Eigenvalue$.
  Chiamiamo inoltre
  \Define{spettro}[di una trasformazione lineare][spettro]
  di $\LinearTransformation$, denotato $\Spectrum{\LinearTransformation}$,
  la classe di tutti gli autovalori di $\LinearTransformation$ e
  \Define{raggio spettrale}[spettrale][raggio] di $\LinearTransformation$,
  denotato $\SpectralRadius{\LinearTransformation}$, la quantit\`a
  $\sup_{\Eigenvalue \in \Spectrum{\LinearTransformation}} \Eigenvalue$.
\end{Definition}
\begin{Definition}
	Sia $\LinearTransformation: \LinearSpace \rightarrow \LinearSpace$ un
  endomorfismo del $\Field$-spazio vettoriali $\LinearSpace$.
  Se $\GeneralizedEigenvector \in \LinearSpace \SetMin \lbrace 0 \rbrace$ \`e
  tale che per
  $\Eigenvalue \in \Field$ e $m \in \mathbb{N}$ opportuno abbiamo
  $(\LinearTransformation - \Eigenvalue \Identity)\GeneralizedEigenvector = 0$,
  allora chiamiamo
  $\GeneralizedEigenvector$
  \Define{autovettore generalizzato}[generalizzato][autovettore] di
  $\LinearTransformation$ relativo a $\Eigenvalue$ di
  \Define{rango}[di un autovettore generalizzato][rango] $m$.
\end{Definition}
\begin{Theorem}
	Sia $\LinearTransformation: \LinearSpace \rightarrow \LinearSpace$ un
  endomorfismo del $\Field$-spazio vettoriali $\LinearSpace$.
  Un autovettore di $\LinearTransformation$ \`e un autovettore generalizzato
  di $\LinearTransformation$ di rango $1$.
\end{Theorem}
\Proof Segue direttamente dalle definizioni. \EndProof
\begin{Theorem}
	Sia $\LinearTransformation: \LinearSpace \rightarrow \LinearSpace$ un
  endomorfismo del $\Field$-spazio vettoriale $\LinearSpace$ e sia
  $\Eigenvalue \in \Field$ un autovalore di $\LinearTransformation$.
  $\LinearSpace_\Eigenvalue
    = \lbrace \Vector \in \LinearSpace | \LinearTransformation(\Vector)
    = \Eigenvalue\Vector \rbrace$
  \`e un sottospazio vettoriale di $\LinearSpace$.
\end{Theorem}
\Proof Abbiamo
$\LinearSpace_\Eigenvalue
  = \Kernel ( \LinearTransformation - \Eigenvalue )$. \EndProof
\begin{Definition}
	Con le notazioni del teorema precedente, $\LinearSpace_\Eigenvalue$ si chiama
\Define{autospazio} di $\Linear$ relativo a $\Eigenvalue$. Inoltre definiamo
$\Dimension{\LinearSpace_\Eigenvalue}$
\Define{molteplicit\`a geometrica}[geometrica di un autovalore][molteplicit\`a]
di $\Eigenvalue$.
\end{Definition}
\begin{Theorem}
	Sia $\LinearTransformation: \LinearSpace \rightarrow \LinearSpace$ un
  endomorfismo del $\Field$-spazio vettoriale $\LinearSpace$.
  $\Image{(\LinearTransformation - \Eigenvalue\Identity}$
  \`e un sottospazio vettoriale invariante per $\LinearTransformation$.
\end{Theorem}
\Proof Siano
\begin{itemize}
  \item $\Vector \in \Image{(\LinearTransformation - \Eigenvalue\Identity}$;
  \item $\VarVector \in \LinearSpace$ tale che
    $(\LinearTransformation - \Eigenvalue\Identity)\VarVector = \Vector$.
\end{itemize}
\par Abbiamo
\begin{align*}
  \LinearTransformation\Vector
  &=\LinearTransformation(\LinearTransformation - \Eigenvalue\Identity)\VarVector,\\
  &= \LinearTransformation^2\VarVector -\Eigenvalue\VarVector,\\
  &=(\LinearTransformation - \Eigenvalue\Identity)\LinearTransformation\VarVector.
  \text{ \EndProof}
\end{align*}
\begin{Theorem}
	Sia $\LinearTransformation: \LinearSpace \rightarrow \LinearSpace$ un
  endomorfismo del $\Field$-spazio vettoriale finito $\LinearSpace$ di
  dimensione $n = \Dimension{\LinearSpace}$.
	Abbiamo
  $\CharacteristicPolynomial(\lambda)
    = \Determinant(\LinearTransformation - \lambda \Identity)
    \in \RingAdjunction{\Field}{\lambda}$.
  Inoltre $\PolynomialDegree{\CharacteristicPolynomial} = n$;
\end{Theorem}
\Proof Fissata una base $(\LinearBase_i)_{i \in n}$ di $\LinearSpace$, se $\Matrix \in \Field^{n \times n}$ rappresenta $\LinearTransformation$ rispetto a $(\LinearBase_i)_{i \in n}$, allora, posto $\VarMatrix = \Matrix - \lambda \Identity$, abbiamo $\CharacteristicPolynomial(\lambda) = \Determinant(\LinearTransformation - \lambda \Identity) = \Determinant(\VarMatrix) = \sum_{\Permutation \in \SymmetricGroup{n}} \PermutationSign{\Permutation} \prod_{i \in n} \VarMatrix_{\Permutation(i),i} \in \RingAdjunction{\Field}{\lambda}$.
\par Abbiamo $\LeadingTerm{\CharacteristicPolynomial} = \LeadingTerm{\prod_{i \in n} (\Matrix_{i,i} - \lambda)} = (- 1)^n \lambda^n$ e quindi $\PolynomialDegree{\CharacteristicPolynomial} = n$. \EndProof
\begin{Definition}
	Con le notazioni del teorema precedente, chiamiamo $\CharacteristicPolynomial(\lambda) = \Determinant(\LinearTransformation - \lambda \Identity)$ \Define{polinomio caratteristico}[caratteristico di una trasformazione lineare][polinomio].
\end{Definition}
\begin{Theorem}
	Sia $\LinearTransformation: \LinearSpace \rightarrow \LinearSpace$ un endomorfismo del $\Field$-spazio vettoriale finito $\LinearSpace$ di dimensione $n = \Dimension{\LinearSpace}$.
	Gli autovalori di $\LinearTransformation$ sono tutte e sole le radici del polinomio caratteristico $\CharacteristicPolynomial(\lambda)$.
\end{Theorem}
\Proof $\lambda \in \Field$ \`e autovalore di $\LinearTransformation$ se e solo se, per opportuno $\lambda \in \NotZero{\LinearSpace}$, abbiamo $\LinearTransformation \Eigenvector = \lambda \Eigenvector$, equivalente a $(\LinearTransformation - \lambda \Identity) \Eigenvector = 0$ e dunque a $\Kernel(\LinearTransformation - \lambda \Identity) \neq \lbrace 0 \rbrace$, condizione verificata se e solo se $\Determinant(\LinearTransformation - \lambda \Identity) = 0$. \EndProof
\begin{Corollary}
	Sia
  $\LinearTransformation: \LinearSpace \rightarrow \LinearSpace$
  un endomorfismo del $\Field$-spazio vettoriale finito $\LinearSpace$ di
  dimensione $n = \Dimension{\LinearSpace}$, con $\Field$ campo algebricamente
  chiuso.
  $\Spectrum{\LinearTransformation}$ \`e un insieme finito di cardinalit\`a non
  maggiore di $n$.
\end{Corollary}
\Proof Segue direttamente dall'ipotesi che $\Field$ sia algebricamente chiuso.
\EndProof
\begin{Definition}
	Sia $\LinearTransformation: \LinearSpace \rightarrow \LinearSpace$ un
  endomorfismo del $\Field$-spazio vettoriale finito $\LinearSpace$ di
  dimensione $n = \Dimension{\LinearSpace}$.
	Sia $\Eigenvalue \in \Spectrum{\LinearTransformation}$, chiamiamo
  \Define{molteplicit\`a algebrica}[algebrica di un autovalore][molteplicit\`a]
  la molteplicit\`a di $\Eigenvalue$ come radice del polinomio caratteristico di
  $\LinearTransformation$.
\end{Definition}
\begin{Theorem}
	Sia $\LinearTransformation: \LinearSpace \rightarrow \LinearSpace$ un
  endomorfismo del $\Field$-spazio vettoriale finito $\LinearSpace$ di
  dimensione $n = \Dimension{\LinearSpace}$ e sia
  $\Eigenvalue \in \Spectrum{\LinearTransformation}$.
  La molteplicit\`a geometrica di $\Eigenvalue$ non \`e maggiore della sua 
  molteplicit\`a algebrica.
\end{Theorem}
\Proof Sia $\Eigenspace$ l'autospazio relativo a $\Eigenvalue$: esso \`e
invariante rispetto a $\LinearTransformation - \Eigenvalue\Identity$,
essendone il nucleo.
\par Il polinomio caratteristico di
$\Restricted{\LinearTransformation}{\Eigenspace}$
\`e
$\Determinant
(\Restricted{\LinearTransformation}{\Eigenspace} - \Eigenvalue\Identity)
= \Determinant(\Restricted{\LinearTransformation - \Eigenvalue\Identity}
{\Eigenspace)}$
e dunque
$\Determinant
(\Restricted{\LinearTransformation}{\Eigenspace} - \Eigenvalue\Identity)
| \Determinant(\LinearTransformation - \Eigenvalue\Identity)$.
\par La molteplicit\`a geometrica di $\Eigenvalue$ \`e
$\LinearDimension{\Eigenspace}$. Inoltre
$\Restricted{\LinearTransformation}{\Eigenspace} = \Eigenvalue\Identity$
e dunque il polinomio caratteristico di
$\Restricted{\LinearTransformation}{\Eigenspace}$
\`e
$\CharacteristicPolynomial(x)
= (\Eigenvalue - x)^{\LinearDimension{\Eigenspace}}$:
pertanto la molteplicit\`a algebrica di $\Eigenvalue$ per
$\LinearTransformation$ \`e almeno $\LinearDimension{\Eigenspace}$.
\EndProof
\begin{Theorem}
	Sia $\LinearTransformation: \LinearSpace \rightarrow \LinearSpace$ un
  endomorfismo del $\Field$-spazio vettoriale finito $\LinearSpace$ di
  dimensione $n = \Dimension{\LinearSpace}$ con $\Field$ campo algebricamente
  chiuso. Allora $\LinearTransformation$ ammette una base
  $(\GeneralizedEigenvector_k)_{k \in n}$ tale che
  \begin{itemize}
    \item per ogni $k \in n$, $\GeneralizedEigenvector_k$ \`e un autovettore
      generalizzato;
    \item per ogni autovalore $\Eigenvalue \in \Spectrum{\LinearTransformation}$
      $(\GeneralizedEigenvector_k)_{k\in n}$ contiene tanti autovettori relativi
      a $\Eigenvalue$ quanta \`e la molteplicit\`a geometrica di $\Eigenvalue$;
    \item per $k, q \in n$ tali che
      \begin{itemize}
        \item $\GeneralizedEigenvector_k$ e $\GeneralizedEigenvector_q$
              sono autovettori;
        \item per ogni $l \in q \SetMin (k + 1)$, $\GeneralizedEigenvector_l$ non
              \`e un autovettore;
      \end{itemize}
      abbiamo, per ogni $l \in q \SetMin (k +  1)$,
      $(\LinearTransformation - \Eigenvalue \Identity)\GeneralizedEigenvector_l
      = \GeneralizedEigenvector_{l - 1}$.
  \end{itemize}
  La matrice che rappresenta $\LinearTransformation$ rispetto a questa base ha
  la forma
  \[
    J =
    \lmatrix
      \begin{array}{cccc}
        J_0 &     &         &\\
            & J_1 &         &\\
            &     & \ddots  &\\
            &     &         & J_s
      \end{array}
    \rmatrix,
  \]
  dove $s$ \`e la somma delle molteplicit\`a geometriche degli autovalori di
  $\LinearTransformation$ e, per ogni $k \in s + 1$, abbiamo
  \[
    J_k =
    \lmatrix
      \begin{array}{cccc}
        \Eigenvalue & 1       &             &\\
                    & \ddots  & \ddots      &\\
                    &         & \Eigenvalue & 1\\
                    &         &             & \Eigenvalue
      \end{array}
    \rmatrix
  \]
  per qualche $\Eigenvalue \in \Spectrum{\LinearTransformation}$.
\end{Theorem}
\Proof Procediamo per induzione su $\LinearDimension{\LinearSpace}$. Il caso
$\LinearDimension{\LinearSpace} = 1$ \`e immediato. Supponiamo dunque il teorema
provato per ogni $N < n$ con $N \in n \in \mathbb{N} \SetMin \lbrace 1 \rbrace$
e dimostriamo il caso $\LinearDimension{\LinearSpace} = n$.
\par Sia $\Eigenvalue \in \Spectrum{\LinearTransformation}$.
\par Lo spazio
$\Image{(\LinearTransformation - \Eigenvalue\Identity)}$
\`e invariante per $\LinearTransformation$ e, poich\'e $\Eigenvalue$ \`e
autovalore,
posto
$m = \LinearDimension{(\Image{(\LinearTransformation - \Eigenvalue\Identity)}}$,
abbiamo
$m =
\LinearDimension{\LinearSpace} -
\LinearDimension{(\Kernel{(\LinearTransformation - \Eigenvalue\Identity)})} < n$.
Dunque, per ipotesi induttiva, esiste una base
$(\GeneralizedEigenvector_k)_{k \in m}$
di
$\Image{(\LinearTransformation - \Eigenvalue\Identity)}$
che verifica le ipotesi del teorema.
\par Sia ora
$s = \LinearDimension{(\Image{(\LinearTransformation - \Eigenvalue\Identity)}
\cap \Kernel{(\LinearTransformation - \Eigenvalue\Identity)})}$.
\par Necessariamente,
$(\GeneralizedEigenvector_k)_{k \in m}$
contiene $s$ autovettori di $\LinearTransformation$ a cui corrispondono
$s$ catene di autovettori generalizzati che si trasformano l'uno nell'altro
tramite l'applicazione $\LinearTransformation - \Eigenvalue\Identity$:
definiamo i vettori
$(\GeneralizedEigenvector_k)_{k \in (m + s) \SetMin m}$,
controimmagini tramite
$\LinearTransformation - \Eigenvalue\Identity$ dell'autovettore generalizzato
inziale di ogni catena. Dato che gli autovettori generalizzati
iniziali sono linearmente indipendenti, devono esserlo anche le loro
controimmagini; inoltre, tali controimmagini sono essi stessi autovettori
generalizzati e, dato che
$(\GeneralizedEigenvector_k)_{k \in m}$
\`e una base di
$\Image{(\LinearTransformation - \Eigenvalue\Identity)}$,
essi non appartengono a 
$\Image{(\LinearTransformation - \Eigenvalue\Identity)}$.
\par Poniamo infine
$(\GeneralizedEigenvector_k)_{k \in n \SetMin (m + s)}$
autovettori linearmente indipendenti dell'autospazio
$\Kernel{(\LinearTransformation - \Eigenvalue\Identity)}$
che non giacciono in
$\Image{(\LinearTransformation - \Eigenvalue\Identity)}
\cap \Kernel{(\LinearTransformation - \Eigenvalue\Identity)}$.
$(\GeneralizedEigenvector_k)_{k \in n \SetMin (m + s)}$
sono indipendenti da
$(\GeneralizedEigenvector_k)_{k \in m}$
perch\'e nessuno di essi appartiene a 
$\Image{(\LinearTransformation - \Eigenvalue\Identity)}$;
sono inoltre indipendenti anche da
$(\GeneralizedEigenvector_k)_{k \in (m + s) \SetMin m}$
perch\'e nessuno di questi ultimi vettori \`e autovettore di
$\LinearTransformation$ per $\Eigenvalue$.
\par $(\GeneralizedEigenvector_k)_{k \in n}$ \`e dunque una base di
$\LinearTransformation$ interamente costituita da autovettori generalizzati:
gli autovettori di $\LinearTransformation$ relativi a $\Eigenvalue$ sono
$s + (n - (m + s)) = n - m =
\LinearDimension{(\Kernel{(\LinearTransformation - \Eigenvalue\Identity)})}$.
\`e sufficiente riordinarla per ottenere una base con le propriet\`a
dell'enunciato.
MANCA LA PARTE SULLA MOLTEPLICITA GEOMETRICA DEGLI AUTOVALORI
\par La parte matriciale del teorema segue direttamente per induzione.
\begin{Definition}
  Con le notazioni del teorema precedente, diciamo che
  $(\GeneralizedEigenvector_k)_{k \in n}$ \`e una
  \Define{base di Jordan}[di Jordan][base]; inoltre chiamiamo
  \Define{catena di Jordan}[di Jordan][catena]
  ogni sottosuccessione
  $(\GeneralizedEigenvector_k)_{k = a}^b$ ($a, b \in n$) tale che
  \begin{itemize}
    \item $a = 0$ o
      $\LinearTransformation \GeneralizedEigenvector_{a - 1}
        \neq \GeneralizedEigenvector_a$;
    \item $\ForAll{k \in (b + 1) \SetMin a}
      {\LinearTransformation \GeneralizedEigenvector_k
        = \GeneralizedEigenvector_{k + 1}}$.
  \end{itemize}
  \par Si dice che la matrice $J$
  \`e in
  \Define{forma canonica di Jordan}[canonica di Jordan][forma]
  \footnote{Altri testi definiscono la forma canonica di Jordan ponendo la
  diagonale degli $1$ sotto la diagonale principale anzich\'e sopra o
  ponendo condizione sull'ordinamento degli autovalori sulla diagonale. Le
  definizioni si equivalgono tutte a meno di premutazioni degli elementi
  della base $(\GeneralizedEigenvector_k)_{k \in n}$.} e le sottomatrici
  $(J_k)_{k \in s + 1}$ si chiamano
  \Define{blocchi di Jordan}[di Jordan][blocco].
\end{Definition}
\begin{Theorem}
	Sia $\LinearTransformation: \LinearSpace \rightarrow \LinearSpace$ un
  endomorfismo del $\Field$-spazio vettoriale finito $\LinearSpace$ di
  dimensione $n = \Dimension{\LinearSpace}$ con $\Field$ campo algebricamente
  chiuso. Allora $\LinearTransformation$ \`e rappresentato da un'unica matrice
  in forma canonica di Jordan a meno di permutazione dei blocchi di Jordan ed
  essa \`e tale che per ogni autovalore
  $\Eigenvalue \in \Spectrum{\LinearTransformation}$ esistono esattamente
  tanti blocchi di Jordan con $\Eigenvalue$ sulla diagonale quanta \`e la
  molteplicit\`a geometrica di $\Eigenvalue$.
\end{Theorem}
\Proof Il teorema precedente dimostra l'esistenza.
\par Supponendo $J$ sia una forma canonica di Jordan di $\LinearTransformation$.
E' chiaro che le colonne che contengono solo l'autovalore corrispondono ad
autovettori.
\begin{Theorem}
	Sia $\LinearTransformation: \LinearSpace \rightarrow \LinearSpace$ un
  endomorfismo del $\Field$-spazio vettoriale finito $\LinearSpace$ di
  dimensione $n = \Dimension{\LinearSpace}$.
  $\PolynomialIdeal
    = \lbrace \Polynomial \in \RingAdjunction{\Field}{x}
    | \Polynomial(\LinearTransformation) = 0$
  \`e un ideale principale.
\end{Theorem}
\begin{Definition}
  Con le notazioni del teorema precedente, il generatore monico
  $\LinearMinimalPolynomial$ dell'ideale $\PolynomialIdeal$ si chiama
  \Define{polinomio minimo}[minimo di una trasformazione lineare][polinomio]
  di $\LinearTransformation$.
\end{Definition}
\begin{Theorem}
  \TheoremName{Teorema di Hamilton-Cayley}[di Hamilton-Cayley][teorema]
	Sia $\LinearTransformation: \LinearSpace \rightarrow \LinearSpace$ un
  endomorfismo del $\Field$-spazio vettoriale finito $\LinearSpace$ di
  dimensione $n = \Dimension{\LinearSpace}$. Il polinomio caratteristico
  $\CharacteristicPolynomial$ di $\LinearTransformation$ valutato in
  $\LinearTransformation$ \`e nullo:
  $\CharacteristicPolynomial(\LinearTransformation) = 0$.
\end{Theorem}
\begin{Definition}
  Sia $\Matrix \in \Field^{n \times n}$ ($n \in \mathbb{N}$).
  Chiamiamo
  \Define{autovettore destro}[destro][autovettore]
  di $\Matrix$ un autovettore dell'endomorfismo lineare
  $\Vector \mapsto \Matrix\Vector$ di $\Field^n$.
  Chiamiamo
  \Define{autovettore sinistro}[sinistro][autovettore]
  di $\Matrix$ un autovettore dell'endomorfismo lineare
  $\HermitianTransposed{\Vector} \mapsto \HermitianTransposed{\Vector}\Matrix$
  di $\Dual{\Field^n}$.
\end{Definition}
\begin{Theorem}
  Sia $\Matrix \in \Field^{n \times n}$ ($n \in \mathbb{N}$).
  $\Eigenvector \in \Field^n$ \`e
  \begin{itemize}
    \item autovettore destro di $\Matrix$ per $\Eigenvalue$ se e solo se
      $\Matrix\Eigenvector = \Eigenvalue\Eigenvector$;
    \item autovettore destro di $\Matrix$ per $\Eigenvalue$ se e solo se
      $\HermitianTransposed{\Vector} \Matrix
      = \Eigenvalue\HermitianTransposed{\Vector}$.
  \end{itemize}
\end{Theorem}
\begin{Theorem}
	Sia $\LinearTransformation: \LinearSpace \rightarrow \LinearSpace$ un
  endomorfismo lineare. Abbiamo
  $\Spectrum{\LinearTransformation} =
  \Spectrum{\Transposed{\LinearTransformation}}$, inoltre ogni autovalore
  $\Eigenvalue \in \Spectrum{\LinearTransformation}$ ha la stessa molteplicit\`a
  algebrica sia come autovalore di $\LinearTransformation$ che come autovalore
  di $\Transposed{\LinearTransformation}$.
\end{Theorem}
\Proof Segue direttamente dal fatto che gli autovalori sono le radici del
polinomio caratteristico e che il determinante di un'applicazione \`e uguale
al determinante dell'applicazione trasposta.
\EndProof
\begin{Theorem}
	Sia $\LinearTransformation: \LinearSpace \rightarrow \LinearSpace$ un
  endomorfismo nilpotente del $\Field$-spazio vettoriale $\LinearSpace$.
  Abbiamo $\Spectrum{\LinearTransformation} = \lbrace 0 \rbrace$.
\end{Theorem}
\Proof Se $n \in \NotZero{\mathbb{N}}$ \`e tale che
$\LinearTransformation^n = 0$ e $\Eigenvalue$ \`e un autovalore di
$\LinearTransformation$ corrispondente ad un autovettore $\Eigenvector$,
allora abbiamo
$\LinearTransformation^n \Eigenvector = \Eigenvalue^n \Eigenvector$,
ma anche $\LinearTransformation^n \Eigenvector = 0$, da cui
$\Eigenvalue^n = 0$ e quindi $\Eigenvalue = 0$. \EndProof
\begin{Theorem}
  Fissato un campo $\Field$, siano
  \begin{itemize}
    \item siano $n, p \in \NotZero{\mathbb{N}}$;
    \item $A \in \Field^{n \times n}$;
    \item $B \in \Field^{p \times p}$;
    \item $\Eigenvalue \in \Spectrum{A}$;
    \item $\VarEigenvalue \in \Spectrum{B}$;
    \item $\Vector \in \NotZero{\Field^n}$ tale che
      $A\Vector = \Eigenvalue\Vector$;
    \item $\VarVector \in \NotZero{\Field^n}$ tale che
      $B\VarVector = \VarEigenvalue\VarVector$.
  \end{itemize}
  Abbiamo
  $(A \KroneckerProduct B)(\Vector \KroneckerProduct \VarVector)
  = \Eigenvalue\VarEigenvalue (\Vector \KroneckerProduct \VarVector)$.
\end{Theorem}
\Proof Abbiamo
\begin{align*}
  (A \KroneckerProduct B)(\Vector \KroneckerProduct \VarVector)
  &= A\Vector \KroneckerProduct B\VarVector,\\
  &= \Eigenvalue\Vector \KroneckerProduct \VarEigenvalue\VarVector,\\
  &= \Eigenvalue\VarEigenvalue (\Vector \KroneckerProduct \VarVector).
  \text{ \EndProof}
\end{align*}


  \chapter{Equazioni differenziali}
  \section{Definizioni e teoremi di base.}
\label{EquazioniDifferenziali_DefinizioniETeoremiDiBase}
\begin{Definition}
  Chiamiamo
  \Define{equazione differenziale}[differenziale][equazione]
  un'equazione avente per incognite una funzione $f$ ed alcune sue derivate,
  messe in relazione anche eventualmente con gli argomenti di $f$.
  L'
  \Define{ordine}[di un'equazione differenziale][equazione]
  dell'equazione \`e il massimo degli ordini delle derivate di $f$ che
  compaiono nell'equazione stessa.
  Inoltre,
  \begin{itemize}
    \item se nell'equazione appaiono solo derivate totali, diciamo che l'equazione
      differenziale \`e
      \Define{ordinaria}[differenziale ordinaria][equazione];
    \item se nell'equazione appaiono derivate parziali, diciamo che l'equazione
      differenziale \`e
      \Define{alle derivate parziali}[differenziale alle derivate parziali][equazione].
  \end{itemize}
\end{Definition}

\section{Problema di Cauchy.}
\label{EquazioniDifferenziali_ProblemaDiCauchy}
\begin{Definition}
	Dato un intervallo $I = [t_0,t_{\text{max}}]$, con
  $t_{\text{max}} \in \mathbb{R} \cup \lbrace + \infty \rbrace$ e
  $t_0 < t_{\text{max}}$, chiamiamo
  \Define{problema di Cauchy}[di Cauchy][problema] o
  \Define{problema ai valori iniziali}[ai valori iniziali][problema] su $I$ il
  sistema
	\[
	\begin{cases}
		y' = f(t,y),\\
		y(t_0) = y_0,
	\end{cases}
	\]
	dove
	\begin{itemize}
		\item $y: I \rightarrow \mathbb{R}^n$;
		\item $f: I \times \mathbb{R}^n \rightarrow \mathbb{R}^n$.
	\end{itemize}
\end{Definition}
\par Assumeremo sempre nel seguito $f \in \CClass{1}[I]$ in modo che siano verificate le ipotesi del teorema seguente.
\begin{Theorem}
	\TheoremName{Teorema di esistenza e unicit\`a di Cauchy-Lipschitz}[di esistenza e unicit\`a di Cauchy-Lipschitz][teorema]
	Sia un problema di Cauchy
	\[
		\begin{cases}
			y' = f(t,y),\\
			y(t_0) = y_0.
		\end{cases}
	\]
	Assumiamo inoltre $f$ lipschitziana nella seconda variabile. Allora esiste una e una sola soluzione del problema di Cauchy.
\end{Theorem}
\begin{Theorem}
	Il problema di Cauchy
	\[
		\begin{cases}
			y' = f(t,y),\\
			y(t_0) = y_0.
		\end{cases}
	\]
	equivale all'equazione integrale $y(t) = y_0 + \int_{t_0}^t f(\tau,y(\tau)) d\tau$.
\end{Theorem}
\Proof Da $y' = f(t,y)$ deduciamo $dy = f(t,y)dt$. Integrando sull'intervallo $[t_0,t]$ e rinominando $\tau$ la variabile $t$ in modo da evitare ambiguit\`a, abbiamo $\int_{y_0}^y dy = \int_{t_0}^t f(\tau,y)d\tau$, equivalente a $y = y_0 + \int_{t_0}^t f(\tau,y)d\tau$.
\par Viceversa, da $y = y_0 + \int_{t_0}^t f(\tau,y)d\tau$ deduciamo
\begin{itemize}
	\item $y' = f(t,y)$ per il teorema fondamentale del calcolo integrale;
	\item $y(t_0) = y_0 + \int_{t_0}^{t_0} f(\tau,y)d\tau = y_0$. \EndProof
\end{itemize}
\begin{Theorem}
  Siano
  \begin{itemize}
    \item $n \in \NotZero{\mathbb{N}}$;
    \item $\Matrix \in \mathbb{C}^{n \times n}$;
    \item $y_0 \in \mathbb{C}^n$.
  \end{itemize}
  Il problema di Cauchy
	\[
		\begin{cases}
			y' = f(t,y),\\
			y(t_0) = y_0.
		\end{cases}
	\]
  ammette l'unica soluzione
  $y = \Exponential(t \Matrix)y_0$.
\end{Theorem}
\Proof Segue per verifica diretta. \EndProof

\section{Equazioni differenziali lineari.}
\label{EquazioniDifferenziali_EquazioniDifferenzialiLineari}
\subsection{Equazioni lineari del secondo ordine.}
\label{EquazioniDifferenziali_EquazioniLineariDelSecondoOrdine}
\begin{Theorem}
  Siano
  \begin{itemize}
    \item $a, b, c \in \mathbb{C}$;
    \item $\lambda_1 \in \mathbb{R}$ una soluzione di
        $a\lambda^2 + b\lambda + c$, se esiste;
    \item $\lambda_2 \in \mathbb{R} \SetMin \lbrace \lambda_1 \rbrace$
        un'ulteriore soluzione di $a\lambda^2 + b\lambda + c$, se esiste;
    \item $\alpha = \RealPart{\tilde{\lambda}}$ e
        $\beta = \ImaginaryPart{\tilde{\lambda}}$, dove
        $\tilde{\lambda} \in \mathbb{C} \SetMin \mathbb{R}$
        \`e una soluzione di
        $a\lambda^2 + b\lambda + c$, se esiste.
  \end{itemize}
  L'equazione in $x(t)$
  \[
    a\ddot{x} + b\dot{x} + cx = 0
  \]
  ammette le seguenti soluzioni particolari:
  \begin{itemize}
    \item $e^{\lambda_1 t}$ e $e^{\lambda_2 t}$
      se $a\lambda^2 + b\lambda + c = 0$
      ammette due soluzioni reali distinte;
    \item $e^{\lambda_1 t}$ e $e^{\lambda_1 t}$
      se $a\lambda^2 + b\lambda + c = 0$
      ammette un'unica soluzione reale;
    \item $e^{\alpha t}\cos \beta t$ e
      $e^{\alpha t}\sin \beta t$
      se $a\lambda^2 + b\lambda + c = 0$
      ammette soluzioni coniugate complesse,
  \end{itemize}
\end{Theorem}
\Proof Supponiamo $\lambda_1$ sia una soluzione reale di
$a\lambda^2 + b\lambda + c = 0$. Abbiamo
\begin{align*}
  a \frac{d^2 e^{\lambda_1 t}}{dt^2}
  + b \frac{d e^{\lambda_1 t}}{dt}
  + c e^{\lambda_1 t}
  &= a \lambda_1^2 e^{\lambda_1 t}
  + b \lambda_1 e^{\lambda_1 t}
  + c e^{\lambda_1 t},\\
  &= e^{\lambda_1 t} (a \lambda_1^2 + b \lambda_1 + c),\\
  &= 0.
\end{align*}
\par Supponiamo ora $\lambda_1$ sia l'unica soluzione reale di
$a\lambda^2 + b\lambda + c = 0$.
Abbiamo dunque
$\lambda_1 = - \frac{b}{2a}$
e
$b^2 - 4 ac = 0$. Inoltre 
\begin{align*}
  a \frac{d^2 t e^{\lambda_1 t}}{dt^2}
  + b \frac{d t e^{\lambda_1 t}}{dt}
  + c t e^{\lambda_1 t}
  &= a \lambda_1 e^{\lambda_1 t}
  + a \lambda_1 e^{\lambda_1 t}
  + a \lambda_1^2 t e^{\lambda_1 t}
  + b e^{\lambda_1 t}
  + b \lambda_1 t e^{\lambda_1 t}
  + c t e^{\lambda_1 t},\\
  &= e^{\lambda_1 t}
    (at \lambda_1^2 + 2a \lambda_1 + bt \lambda_1 + b + c t),\\
  &= e^{\lambda_1 t}
    (a t \frac{b^2}{4 a^2} - 2a \frac{b}{2a} - bt \frac{b}{2 a} + b + ct),\\
  &= t e^{\lambda_1 t}
    (\frac{b^2}{4a} -  \frac{b^2}{2a} + c),\\
  &= t e^{\lambda_1 t}
    \frac{4ac - b^2}{4a}
  &= 0.
\end{align*}
\par Supponiamo infine
$a\lambda^2 + b\lambda + c = 0$
ammetta soluzioni coniugate complesse distinte.
Poniamo $z = e^{\tilde{\lambda}}$. Abbiamo
\begin{align*}
  a \frac{d^2 e^{\tilde{\lambda}_1 t}}{dt^2}
  + b \frac{d e^{\tilde{\lambda}_1 t}}{dt}
  + c e^{\tilde{\lambda}_1 t}
  &= a \tilde{\lambda}_1^2 e^{\tilde{\lambda}_1 t}
  + b \tilde{\lambda}_1 e^{\tilde{\lambda}_1 t}
  + c e^{\tilde{\lambda}_1 t},\\
  &= e^{\tilde{\lambda}_1 t} (a \tilde{\lambda}_1^2 + b \tilde{\lambda}_1 + c),\\
  &= 0.
\end{align*}
Ne consegue la tesi separando parte reale e immaginaria. \EndProof

\subsection{Equazioni alle derivate parziali lineari.}
\label{EquazioniDifferenziali_EquazioniAlleDerivateParzialiLineari}
\begin{Definition}
  Siano
  \begin{itemize}
    \item $\OpenConnected \subseteq \mathbb{R}^2$ un aperto connesso;
    \item $u: \Closure{\OpenConnected} \rightarrow \mathbb{R}$ incognita;
    \item $f: \OpenConnected \rightarrow \mathbb{R}$ nota e
      sufficientemente regolare;
    \item $A, B, C, D, E, F: \OpenConnected \rightarrow \mathbb{R}$;
    \item $L$ un'operatore differenziale lineare sulle funzioni di classe
      $\CClass{2}[\Closure{\OpenConnected}][\mathbb{R}]$ definito da
      \[
        L[u]
        = A \frac{\partial^2 u}{\partial x^2}
        + B \frac{\partial^2 u}{\partial x \partial y}
        + C \frac{\partial^2 u}{\partial y^2}
        + D \frac{\partial u}{\partial x}
        + E \frac{\partial u}{\partial y}
        + Fu.
      \]
  \end{itemize}
  L'equazione $L[u] = f$ si dice
  \begin{itemize}
    \item \Define{ellittica}[differenziale ellittica][equazione]
      quando $\ForAll{(x,y) \in \OpenConnected}{B^2 - 4AC < 0}$;
    \item \Define{parabolica}[differenziale parabolica][equazione]
      quando $\ForAll{(x,y) \in \OpenConnected}{B^2 - 4AC = 0}$;
    \item \Define{iperbolica}[differenziale iperbolica][equazione]
      quando $\ForAll{(x,y) \in \OpenConnected}{B^2 - 4AC > 0}$.
  \end{itemize}
\end{Definition}
\begin{Theorem}
  Con le notazioni della definizione precedente, $L[u] = f$ \`e
  ellittica, parabolica o iperbolica se e solo se, per ogni
  $(x,y) \in \mathbb{R}^2$, la conica $A\xi^2 + B\xi\eta + C\eta^2 = 0$
  nel piano $\xi\eta$ \`e un'ellisse, una parabola o un'iperbole
  rispettivamente.
\end{Theorem}
\Proof Segue direttamente dalle definizioni e dal teorema di classificazione
delle coniche. \EndProof



	\chapter{Sistemi dinamici}
	\section{Definizioni e teoremi di base}
\label{SistemiDinamici_DefinizioniETeoremiDiBase}
\begin{Definition}
	Un \Define{sistema dinamico}[dinamico][sistema] \`e una tripla $(\Times,\States,\DynSysFunction)$ tale che
	\begin{itemize}
		\item $\Times$, detto \Define{spazio dei parametri}[dei parametri][spazio], \`e un monoide;
		\item $\States$, detto \Define{spazio delle fasi}[delle fasi][spazio] o \Define{spazio degli stati}[degli stati][spazio] \`e una classe non vuota;
		\item $\DynSysFunction$, detta \Define{funzione di evoluzione}[di evoluzione][funzione], una funzione $\DynSysFunction: \DynSysDomain \rightarrow (\Times,\States) \rightarrow \States$ tale che
		\begin{itemize}
			\item $\ForAll{\State \in \States}{\Exists{\Time \in \Times}{(\Time,\State) \in \DynSysDomain}}$;
			\item $\ForAll{\State \in \States}{\DynSysFunction(\Time,\State) = \State}$;
			\item $\ForAll{(\State,\Time_0,\Time_1) \in \States \times \Times \times \Times}{\Implies{\And{(\Time_0,\State) \in \DynSysDomain}{(\Time_0 + \Time_1,\State) \in \DynSysDomain}}{\DynSysFunction(\Time_1,\DynSysFunction(\Time_0,\State)) = \DynSysFunction(\Time_0 + \Time_1, \State)}}$
		\end{itemize}
	\end{itemize}
	Chiamiamo inoltre
	\begin{itemize}
		\item \Define{flusso integrale}[integrale][flusso] del sistema la famiglia di applicazioni $(\DynSysFlux_\Time)_{\Time \in Times}$ tali che per ogni $\Time \in \Times$  l'applicazione $\DynSysFlux_t: \States \rightarrow \States$ sia l'applicazione che a $\State \in \States$ associa $\DynSysFlux(\Time,\State)$;
	\end{itemize}
\end{Definition}

\section{Definizioni e teoremi di base}
\label{SistemiDinamici_DefinizioniETeoremiDiBase}
\begin{Definition}
	Chiamiamo \Define{sistema dinamico continuo}[dinamico continuo][sistema] un'equazione differenziale del primo ordine autonoma $\dot{X} = \VectorField(X)$, dove
	\begin{itemize}
		\item $n \in \mathbb{N}$;
		\item $\Open \subseteq \mathbb{R}^n$ aperto;
		\item $\VectorField: \Open \rightarrow \mathbb{R}^n$ un campo vettoriale di classe almeno $\CClass{1}$.
	\end{itemize}
	Definiamo inoltre \Define{orbita}[di un sistema dinamico continuo][orbita] una soluzione dell'equazione differenziale.
\end{Definition}
\begin{Definition}
	Con le notazioni della definizione precedente, chiamiamo \Define{integrale primo}[primo][integrale] una funzione $\PrimeIntegral: \Open \rightarrow \mathbb{R}$ tale che
	\begin{itemize}
		\item $\PrimeIntegral$ \`e differenziabile;
		\item se $X$ \`e un'orbita del sistema dinamico, allora $\ForAll{t \in W}{\PrimeIntegral(X(t)) = \PrimeIntegral(X(0))}$.
	\end{itemize}
\end{Definition}
\begin{Theorem}
	Data un'equazione differenziale ordinaria di ordine $n \in \NotZero{\mathbb{N}}$ in $x$ esiste un sistema dinamico continuo di $n$ funzioni incognite $\dot{X} = \VectorField(X)$ tale che ogni sua soluzione $(X_i)_{i \in n + 1}$ sia tale che $X_0$ risolva l'equazione differenziale ordinaria originale.
\end{Theorem}
\Proof Definiamo $X = (X_i)_{i \in n + 1}$ tale che $\ForAll{i \in n + 1}{X_i = x^{(i)}}$. Il sistema dinamico \`e allora dato dall'equazione differenziale ordinaria originale e dalle equazioni $\dot{X_i} = X_{i + 1}$ al variare di $i \in n$. \EndProof
\begin{Theorem}
	Data un'equazione differenziale ordinaria non autonoma del primo ordine in $x$, esiste un sistema dinamico continuo di $2$ funzioni incognite $\dot{X} = \VectorField(X)$ tale che ogni sua soluzione $(X_i)_{i \in 2}$ sia tale che $X_0$ risolva l'equzione differenziale ordinaria originale.
\end{Theorem}
\Proof Definiamo $X = (X_i)_{i \in 2}$ tale che $X_0 = x$ e $X_1 = t$, dove $t$ \`e la variabile indipendente. Il sistema dinamico \`e allora dato dall'equzione differenziale ordinaria originale e dall'equazione $\dot{X_1} = 1$. \EndProof
\begin{Definition}
	Chiamiamo \Define{sistima dinamico gradiente}[dinamico gradiente][sistema] \`e un sistema dinamico continuo $\dot{X} = \VectorField(X)$ tale che esista un'applicazione $U: \Dom{\VectorField} \rightarrow \Cod{\VectorField}$ per cui $\VectorField = - \Gradient{U}$. Chiamiamo la funzione $-U$ \Define{potenziale} del sistema.
\end{Definition}
\section{Equilibri e stabilit\`a.}
\label{SistemiDinamici_EquilibriEStabilita}
\begin{Definition}
	Dato un sistema dinamico $\dot{X} = \VectorField(X)$, chiamiamo \Define{punto di equilibrio}[di equilibrio][punto] o \Define{configurazione di equilibrio}[di equilibrio][configurazione] un punto $\Equilibrium$ tale che $\VectorField(\Equilibrium) = 0$.
\end{Definition}
\begin{Definition}
	Dato un sistema dinamico $\dot{X} = \VectorField(X)$, un punto di equilibrio $\StableEquilibrium$ si dice \Define{stabile}[d'equilibrio stabile][punto] quando per ogni intorno $\Neighborhood_0$ di $\StableEquilibrium$ esiste un altro intorno $\Neighborhood_1$ di $\StableEquilibrium$ tale che $\Implies{X(0) \in \Neighborhood_0}{\ForAll{t \in \RealNonNegative}{X(t) \in \Neighborhood_1}}$. Se $\StableEquilibrium$ non \`e stabile, allo si dice \Define{instabile}[d'equilibrio instabile][punto].
\end{Definition}
\begin{Definition}
	Dato un sistema dinamico $\dot{X} = \VectorField(X)$, un punto $\AttractivePoint$ si dice \Define{attrattivo}[attrattivo][punto] quando esiste un intorno $\Neighborhood$ di $\AttractivePoint$ tale che $\Implies{X(0) \in \Neighborhood}{\lim_{t \rightarrow + \infty} X(t) = \AttractivePoint}$. Chiamiamo inoltre \Define{bacino d'attrazione}[d'attrazione][bacino] l'insieme $\BasinOfAttraction$ di tutti i punti $X_0$ tali che $\Implies{X(0) \in \Neighborhood}{\lim_{t \rightarrow + \infty} X(t) = \AttractivePoint}$.
\end{Definition}
\begin{Definition}
	Dato un sistema dinamico $\dot{X} = \VectorField(X)$, chiamiamo un punto di equilibrio $\AsymptoticallyStableEquilibrium$ \Define{asintoticamente stabile}[d'equilibrio asintoticamente stabile][punto] se\`e un punto di equilibrio stabile e se \`e attrattivo.
\end{Definition}

\subsection{Sistemi newtoniani.}
\label{SistemiDinamici_SistemiNewtoniani}
\begin{Definition}
	Chiamiamo \Define{sistema newtoniano}[newtoniano][sistema] o \Define{sistema undimensionale}[dinamico unidimensionale][sistema] un sistema dinamico della forma
	$$\begin{cases}
		\dot{x} = v,\\
		\dot{v} = \frac{\VectorField(x)}{\Mass}\\
		x(0) = x_0,\\
		v(0) = v_0,
	\end{cases}$$
	dove $\Mass = 1$. 
\end{Definition}
\par Questo tipo di sistema si chiama newtoniano in quanto descrive il moto di un punto materiale di massa $\Mass$ in un campo di forze date da $\VectorField$. Lo studio di un sistema dinamico newtoniano avviene normalmente, come indicato anche dalla nostra definizione, ponedo $\Mass = 1$, ma preferiamo mantenere la variabile $\Mass$ generica in modo da poter applicare direttamente tutte le definizioni e tutti i teoremi riguardo ai sistemi newtoniani a contesti fisici.
\par Nel seguito di questa sottosezione assumeremo sempre dato un sistema newtoniano descritto con le notazioni della definizione precedente.
\begin{Definition}
	Chiamiamo \Define{energia del sistema newtoniano}[di un sistema newtoniano][energia] la funzione $\Energy(x,\dot{x}) = \frac{1}{2}m\dot{x}^2 + m\Potential(x)$.
\end{Definition}
\begin{Theorem}
	$\Energy(x,\dot{x})$ \`e un integrale primo del moto.
\end{Theorem}
\Proof Abbiamo $\TotalDerivative{\Energy}{t}(x,\dot{x}) = \frac{1}{2}2m\dot{x}\ddot{x} + m\Derivative{\Potential}{x}\dot{x} = \dot{x} \left ( m\ddot{x} + m\Derivative{\Potential}{x} \right ) = \dot{x} (\VectorField(x) - m\Potential(x)) = 0$. \EndProof
\begin{Theorem}
	Il moto \`e possibile solo nei punti $x$ tali che $\Energy - \PotentialEnergy(x) \geq 0$.
\end{Theorem}
\Proof Abbiamo $\Energy - \PotentialEnergy(x) = \frac{1}{2}m\dot{x}^2 \geq 0$. \EndProof
\begin{Definition}
	Chiamiamo \Define{punto di arresto del moto}[di arresto del moto][punto] o \Define{punto di inversione del moto}[di inversione del moto][punto] i punti $x$ di un sistema newtoniano tali che $\Potential(x) = \Energy$.
\end{Definition}
\begin{Theorem}
	Sia un sistema newtoniano $\ddot{x} = - \Potential(x)$ per qualche potenziale $\Potential \in \CClass{2}$ e sia $\Energy$ l'energia del sistema per una qualche condizione iniziale fissata $x(0) = x_0$, $\dot{x}(0) = v_0$. Abbiamo, se $v_0 \neq 0$, $t(x) = \Sign{v_0} \sqrt{\frac{\Mass}{2}} \int_{x_0}^x \frac{d\tilde{x}}{\sqrt{\Energy - \Potential(\tilde{x})}}$, dove $\Sign{v_0}$ \`e il segno di $v_0$, considerando il sistema su un intervallo di tempo sufficientemente piccolo perch\'e $\dot{x}$ non cambi segno.
\end{Theorem}
\Proof Dalla definizione dell'energia $\Energy = \frac{1}{2} \Mass \dot{x}^2 + \Potential(x)$ deduciamo $\ddot{x} = \pm \sqrt{\frac{2}{m} ( \Energy - \Potential(x) )}$. Tra i due segni, quello corretto \`e quello di $v_0$ per la continuit\`a di $\dot{x}$. Integrando per parti abbiamo allora la tesi. \EndProof
\begin{Theorem}
	Sia un sistema newtoniano con potenziale $\PotentialEnergy \in \CClass{2}$: $\Mass \ddot{x} = - \PotentialEnergy'(x)$. Supponiamo $\Energy$ sia l'energia del sistema e che $x_1, x_2$ siano due punti tali che
	\begin{itemize}
		\item $\PotentialEnergy(x_1) = \PotentialEnergy(x_2) = \Energy$;
		\item $\ForAll{x \in (x_1,x_2)}{\PotentialEnergy{x} < \Energy}$;
		\item $\PotentialEnergy'(x_1) \neq 0$;
		\item $\PotentialEnergy'(x_2) \neq 0$.
	\end{itemize}
	Se $x(0) \in (x_1,x_2)$, allora il moto \`e periodico con periodo $\Period = \sqrt{2\Mass} \int_{x_1}^{x_2} \frac{dx}{\Energy - \PotentialEnergy(x)}$.
\end{Theorem}
\begin{Definition}
	Sia un sistema newtoniano $\ddot{x} = - \PotentialEnergy'(x)$, per qualche potenziale $\PotentialEnergy \in \CClass{2}$ e qualche condizione iniziale $x(0) = x_0$. Se esiste $\bar{x}$ tale che $\lim_{x \rightarrow \bar{x}} t(x) = + \infty$, allora la soluzione \`e un \Define{moto a meta asintotica}[a meta asintotica][moto], con \Define{meta asintotica}[asintotica][meta] $\bar{x}$
\end{Definition}
\begin{Theorem}
	Sia un sistema newtoniano con potenziale $\PotentialEnergy \in \CClass{2}$: $\ddot{x} = - \PotentialEnergy'(x)$. Supponiamo $\Energy$ sia l'energia del sistema e che $\bar{x}$ sia un punto tale che
	\begin{itemize}
		\item $\PotentialEnergy(\bar{x}) = \Energy$;
		\item $\ForAll{x \in [x(0),\bar{x})}{\PotentialEnergy{x} < \Energy}$;
		\item $\PotentialEnergy'(\bar{x}) = 0$.
	\end{itemize}
	Il moto \`e a meta asintotica con meta asintotica $\bar{x}$.
\end{Theorem}
\begin{Definition}
	Dati un sistema dinamico $\dot{X} = \VectorField(X)$ e un punto di equilibrio instabile $\UnstableEquilibrium$, chiamiamo \Define{sepratrice} la curva di livello della funzione energia $\Energy(x,t)$ corrispondente al valore $\Energy(\UnstableEquilibrium,0)$.
\end{Definition}
\begin{Theorem}
	Sia un sistema newtoniano con potenziale $\PotentialEnergy \in \CClass{2}$: $\ddot{x} = - \PotentialEnergy'(x)$. Se $\StableEquilibrium$ \`e un minimo isolato di $\PotentialEnergy$, allora \`e un punto di equilibrio stabile.
\end{Theorem}
\begin{Theorem}
	Sia un sistema newtoniano con potenziale $\PotentialEnergy \in \CClass{2}$: $\ddot{x} = - \PotentialEnergy'(x)$: non esistono equilibri asintoticamente stabili.
\end{Theorem}
\Proof Assumiamo esista un equilibrio asintoticamente stabile $\AsymptoticallyStableEquilibrium$. Sia $X$ orbita del sistema dinamico tale che $\lim_{t \rightarrow + \infty} \AsymptoticallyStableEquilibrium$. Allora $\lim_{t \rightarrow + \infty} \Energy(X(t),t) = \PotentialEnergy(\AsymptoticallyStableEquilibrium$. Ma l'energia \`e un integrale primo, quindi deve essere $\ForAll{t \in \RealNonNegative}{X(t) = \AsymptoticallyStableEquilibrium}$. \EndProof
\begin{Theorem}
	Un'orbita di un sistema newtoniano giace interamente su una curva di livello della funzione energia.
\end{Theorem}
\Proof Segue direttamente dal fatto che l'energia \`e un integrale primo. \EndProof
\begin{Definition}
	Chiamiamo \Define{piano delle fasi}[delle fasi][piano] di un sistema newtoniano il piano $x\dot{x}$.
\end{Definition}
\begin{Theorem}
	Le curve di livello della funzione energia di un sistema newtoniano nel piano delle fasi sono simmetriche rispetto all'asse $x$.
\end{Theorem}
\Proof Segue direttamente da $\Energy(x,\dot{x}) = \Energy(x,-\dot{x})$. \EndProof
\begin{Theorem}
	Le curve di livello della funzione energia di un sistema newtonianosono regolari ovunque tranne che nei punti di equilibrio del sistema.
\end{Theorem}
\Proof Consideriamo una curva di livello della funzione energia corrispondente al valore $\Energy$.  La funzione $G(x,\dot{x}) = \frac{1}{2}m\dot{x}^2 + \PotentialEnergy(x) - \Energy$ si annulla in tutti e soli i punti della curva di livello. I punti singolari sono tutti e soli i punti tali che $\frac{\partial G}{\partial \dot{x}} = 0$, equivalenti $\dot{x} = 0$, e $\frac{\partial{G}}{\partial{x}}$, equivalente a $\PotentialEnergy'(x) = 0$. Quindi i punti singolari sono punti di equilibrio. \EndProof
\begin{Theorem}
	Le orbite di un sistema newtoniano vengono percorse
	\begin{itemize}
		\item da sinistra verso destra nel semipiano $\dot{x} > 0$; 
		\item da destra verso sinistra nel semipiano $\dot{x} < 0$;
		\item in senso orario se sono orbite chiuse.
	\end{itemize}
\end{Theorem}
\Proof Se $\dot{x} > 0$ allora $x$ cresce, se $\dot{x} < 0$ allora $x$ decresce. Se l'orbita \`e chiusa, allora \`e simmetrica rispetto all'asse $x$ e le considerazioni precedenti implicano che essa viene percorsa in senso orario. \EndProof

\section{Analisi qualitativa.}
\label{SistemiDinamici_AnalisiQualitativa}


\section{Problema di Cauchy.}
\label{SistemiDinamici_ProblemaDiCauchy}
\begin{Definition}
	Sia data un'equazione differenziale
\end{Definition}


	\chapter{Analisi funzionale}
	\section{Successioni di funzioni.}
\label{AnalisiFunzionale_SuccessioniDiFunzioni}
\begin{Definition}
	Siano $(f_n)_{n \in \mathbb{N}}$ e $f$ funzioni tutte definite sugli stessi domini e codomini e tale che $\Cod{f}$ sia uno spazio topologico. $(f_n)_{n \in \mathbb{N}}$ \Define{converge puntualmente}[puntuale][convergenza] a $f$ quando $\ForAll{x \in \Dom{f}}{\lim_{n \rightarrow \infty} f_n(x) = f(x)}$.
\end{Definition}
\begin{Definition}
	Siano $(f_n)_{n \in \mathbb{N}}$ e $f$ funzioni tutte definite sugli stessi domini e codomini e tale che $\Cod{f}$ sia uno spazio metrico con distanza $\Distance$. $(f_n)_{n \in \mathbb{N}}$ \Define{converge uniformemente}[uniformemente][convergenza] a $f$ quando $\ForAll{\epsilon > 0}{\Exists{N \in \mathbb{N}}{\ForAll{n \in \mathbb{N}}{\Implies{n \geq N}{\ForAll{x \in \Dom{f}}{\Distance(f_n(x),f(x)) < \epsilon}}}}}$.
\end{Definition}

\section{Operatori lineari limitati.}
\label{AnalisiFunzionale_OperatoriLineariLimitati}
\begin{Definition}
	Sia $\BoundedLinearOperator: \LinearTopologicalSpace \rightarrow \VarLinearTopologicalSpace$ un operatore lineare. $\BoundedLinearOperator$ si dice \Define{limitato}[lineare limitato][operatore] quando trasforma parti limitate in parti limitate. Denoteremo lo spazio di tutte gli operatori lineari da $\LinearTopologicalSpace$ in $\VarLinearTopologicalSpace$ con $\BoundedLinearOperators{\LinearTopologicalSpace}{\VarLinearTopologicalSpace}$.
\end{Definition}
\begin{Theorem}
	Sia $\BoundedLinearOperator: \LinearTopologicalSpace \rightarrow \VarLinearTopologicalSpace$ un operatore lineare. Se $\LinearTopologicalSpace$ \`e finito, allora $\BoundedLinearOperator$ \`e limitato.
\end{Theorem}
\begin{Theorem}
	Sia $\LinearOperator: \NormedSpace \rightarrow \VarNormedSpace$ un operatore lineare tra gli spazi vettoriali normati $\NormedSpace$ e $\VarNormedSpace$. $\LinearOperator$ \`e limitato se e solo se \`e continuo.
\end{Theorem}
\begin{Definition}
	Fissato $n \in \mathbb{N}$, una \Define{norma di matrice}[di matrice][norma] \`e una norma $\Norm{\cdot}$ sullo spazio vettoriale $\mathbb{C}^n$ che \`e inoltre submoltiplicativa: $\ForAll{(\Matrix,\VarMatrix) \in \mathbb{C}^n \times \mathbb{C}^n}{\Norm{\Matrix\VarMatrix} \leq \Norm{\Matrix}\Norm{\VarMatrix}}$.
\end{Definition}
\begin{Theorem}
	Siano $\NormedSpace$ e $\VarNormedSpace$ spazi vettoriali normali. L'applicazione $\Norm{\cdot}: \BoundedLinearOperators{\NormedSpace}{\VarNormedSpace} \rightarrow \RealNonNegative$ che a $\BoundedLinearOperator \in \BoundedLinearOperators{\NormedSpace}{\VarNormedSpace}$ associa $\min \lbrace M \in \RealNonNegative | \ForAll{\Vector \in \NormedSpace}{\Norm{\LinearOperator} \leq M \Norm{\Vector}} \rbrace \in \RealNonNegative$ \`e una norma sullo spazio vettoriale $\BoundedLinearOperators{\NormedSpace}{\VarNormedSpace}$.
\end{Theorem}
\begin{Definition}
	Con le notazioni del teorema precedente, definiamo $\Norm{\BoundedLinearOperator}$ \Define{norma operatore}[di un operatore lineare limitato][norma] di $\BoundedLinearOperator$.
\end{Definition}
\par Ogni volta che tratteremo uno spazio di operatori lineare limitati come uno spazio vettoriale normato, si tratter\`a sempre della norma appena definita, salvo avvisi contrari.
\begin{Theorem}
	Le norme operatore sono norme di matrici.
\end{Theorem}
\begin{Definition}
	Una norma di matrice si dice \Define{indotta}[di matrice indotta][norma] quando \`e una norma operatore.
\end{Definition}
\begin{Theorem}
	Siano $\NormedSpace$ e $\VarNormedSpace$ spazi vettoriali normati.
  Sia
  $\BoundedLinearOperator \in
    \BoundedLinearOperators{\NormedSpace}{\VarNormedSpace}$.
  Abbiamo
	\begin{itemize}
		\item $\Norm{\BoundedLinearOperator} = \sup_{\substack{\Vector \in \NormedSpace\\\Norm{\Vector} \leq 1}} \Norm{\BoundedLinearOperator \Vector}$;
		\item $\Norm{\BoundedLinearOperator} = \sup_{\substack{\Vector \in \NormedSpace\\\Norm{\Vector} < 1}} \Norm{\BoundedLinearOperator \Vector}$;
		\item $\Norm{\BoundedLinearOperator} = \sup_{\substack{\Vector \in \NormedSpace\\\Norm{\Vector} = 1}} \Norm{\BoundedLinearOperator \Vector}$;
		\item $\Norm{\BoundedLinearOperator} = \sup_{\substack{\Vector \in \NormedSpace\\\Vector \neq 0}} \frac{\Norm{\BoundedLinearOperator \Vector}}{\Norm{\Vector}}$.
	\end{itemize}
\end{Theorem}
\begin{Corollary}
	Siano $\NormedSpace$ e $\VarNormedSpace$ spazi vettoriali normati.
  Sia
  $\BoundedLinearOperator \in
    \BoundedLinearOperators{\NormedSpace}{\VarNormedSpace}$.
  Abbiamo
  \[
    \ForAll{\Vector \in \NormedSpace}{
      \Norm{\BoundedLinearOperator \Vector} \leq
      \Norm{\BoundedLinearOperator} \Norm{\Vector}}.
  \]
\end{Corollary}
\Proof Se $\Vector = 0$ la tesi \`e immediata. Assumiamo $\Vector \neq 0$.
\par Abbiamo
\begin{align*}
  \Norm{\BoundedLinearOperator \Vector}
  &= \Norm{\BoundedLinearOperator
    \Norm{\Vector} \frac{\Vector}{\Norm{\Vector}}},\\
  &= \Norm{\Vector}
    \Norm{\BoundedLinearOperator \frac{\Vector}{\Norm{\Vector}}},\\
  &\leq \Norm{\Vector} \Norm{\BoundedLinearOperator}.\text{ \EndProof}
\end{align*}
\begin{Theorem}
  Fissato $n \in \NotZero{\mathbb{N}}$, abbiamo, per ogni
  $\Matrix \in \mathbb{C}^{n \times n}$
  \begin{itemize}
    \item $\Norm{\Matrix}[1]
      = \max_{j \in n} \sum_{k \in n} \AbsoluteValue{\Matrix_k^j}$;
    \item $\Norm{\Matrix}[2]
      = \sqrt{\SpectralRadius{\HermitianTransposed{\Matrix}\Matrix}}$;
    \item $\Norm{\Matrix}[\infty]
      = \max_{k \in n} \sum_{j \in n} \AbsoluteValue{\Matrix_k^j}$.
  \end{itemize}
\end{Theorem}
\begin{Theorem}
	Siano $\NormedSpace$ e $\VarNormedSpace$ spazi vettoriali normati finiti. Sia $\BoundedLinearOperator \in \BoundedLinearOperators{\NormedSpace}{\VarNormedSpace}$. Per ogni $\epsilon > 0$, esiste una norma indotta di $\BoundedLinearOperators{\NormedSpace}{\VarNormedSpace}$ tale che $\SpectralRadius{\BoundedLinearOperator} \leq \Norm{\BoundedLinearOperator} \leq \SpectralRadius{\BoundedLinearOperator} + \epsilon$.
\end{Theorem}


	\part{Analisi numerica}
  \chapter{Analisi degli errori}
  \section{Definizioni e teoremi di base.}
\label{AnalisiDegliErrori_DefinizioniETeoremiDiBase}
\begin{Definition}
  Sia $\Distance$ una distanza su uno spazio metrico $\MetricSpace$.
	Sia $x \in \MetricSpace$ e supponiamo di voler approssimare $x$ con un altro
  punto $\tilde{x} \in \MetricSpace$.
  Chiamiamo la quantit\`a $\AbsoluteError = \Distance{\tilde{x}}{x}$
  \Define{errore assoluto}[assoluto][errore]
  o anche semplicemente
  \Define{errore},
  o ancora
  \Define{perturbazione}.
	\par Se la distanza $\Distance$ \`e indotta da una norma $\Norm{\cdot}$
  e $\Norm{x} \neq 0$, allora definiamo
  \Define{errore relativo}[relativo][errore]
  o
  \Define{perturbazione relativa}[relativa][perturbazione]
  la quantit\`a
  $\RelativeError = \frac{\Norm{\tilde{x} - x}}{\Norm{x}}$.
  \par Nel caso particolare in cui
  $\MetricSpace = \mathbb{R}$, si definiscono altres\`i
  \NIDefine{errore assoluto} e \NIDefine{errore relativo} le quantit\`a
  $\AbsoluteError = \tilde{x} - x$
  e
  $\RelativeError = \frac{\tilde{x} - x}{x}$
  rispettivamente.
\end{Definition}
\begin{Definition}
  Chiamiamo
  \Define{errore al primo ordine}[al primo ordine][errore]
  l'approssimazione di un errore $\Error$ ottenuta trascurando,
  in un'espressione di $\Error$ in dipendenza di altri errori,
  tutti gli errori di grado maggiore di $1$.
\end{Definition}
\par L'analisi al primo ordine \`e molto utile nella pratica perch\'e, nelle
applicazioni, si suppongono sempre gli errori molto piccoli, e dunque i prodotti
di errori sono ancora pi\`u piccoli, impercettibili dagli strumenti di
approssimazione, siano essi strumenti di misura, di calcolo o altri.
\begin{Definition}
  Siano
  \begin{itemize}
    \item $n \in \mathbb{N}$;
    \item $f: \mathbb{R}^n \rightarrow \mathbb{R}$;
    \item $X \in \MetricSpace$;
    \item $Y = f(X)$.
  \end{itemize}
  Fissata una perturbazione $\Perturbation{X} \in \mathbb{R}^n$ di $X$,
  chiamiamo
  \Define{errore in avanti}[in avanti][errore]
  la perturbazione $\Perturbation{Y}$ di $Y$ unica soluzione dell'equazione
  $f(X + \Perturbation{X}) = Y + \Perturbation{Y}$.
  Fissata una perturbazione $\Perturbation{Y} \in \mathbb{R}$ di $Y$,
  chiamiamo
  \Define{errore all'indietro}[all'indietro][errore]
  l'estremo inferiore della classe di tutte le perturbazioni
  $\Perturbation{X}$ soluzioni dell'equazione
  $f(X + \Perturbation{X}) = Y + \Perturbation{Y}$.
\end{Definition}
\begin{Definition}
  Siano
  \begin{itemize}
    \item $n \in \NotZero{\mathbb{N}}$;
    \item $x, \tilde{x} \in \mathbb{R}^n$;
    \item $f: \mathbb{R}^n \rightarrow \mathbb{R}$;
    \item $\tilde{f}: \mathbb{R}^n \rightarrow \mathbb{R}$.
  \end{itemize}
  Supponiamo $f(x) \neq 0$.
  Chiamiamo
  \begin{itemize}
    \item \Define{errore inerente}[inerente][errore]
      l'errore relativo $\InherentError$ di $f(x)$ approssimato con
      $f(\tilde{x})$:
      \[
        \InherentError = \frac{f(\tilde{x}) - f(x)}{f(x)};
      \]
    \item \Define{errore algoritmico}[algoritmico][errore]
      l'errore relativo $\AlgorithmicError$ di $f(x)$ approssimato con
      $\tilde{f(x)}$, restituito come dato in uscita da un algoritmo
      fissato per il calcolo della funzione $f$ passando in ingresso $x$:
      \[
        \AlgorithmicError = \frac{\tilde{f(x)} - f(x)}{f(x)};
      \]
    \item \Define{errore analitico}[analitico][errore]
      l'errore relativo $\AnalyticalError$ di $f(x)$ approssimato con
      $\tilde{f}(x)$:
      \[
        \AnalyticalError = \frac{\tilde{f}(x) - f(x)}{f(x)};
      \]
    \item \Define{errore totale}[totale][errore]
      l'errore relativo $\TotalError$ di $f(x)$ approssimato con
      $\overline{\tilde{f}(\tilde{x})}$, restituito come dato in uscita da un
      algoritmo fissato per il calcolo della funzione $\tilde{f}$ passando in 
      ingresso $\tilde{x}$:
      \[
        \TotalError = \frac{\overline{\tilde{f}(\tilde{x})} - f(x)}{f(x)}.
      \]
  \end{itemize}
\end{Definition}
\begin{Theorem}
  Supponendo
  \begin{itemize}
    \item $n \in \NotZero{\mathbb{N}}$
    \item $x, \tilde{x} \in \mathbb{R}^n$;
    \item $f: \mathbb{R}^n \rightarrow \mathbb{R}$;
    \item $\tilde{f}: \mathbb{R}^n \rightarrow \mathbb{R}$.
  \end{itemize}
  abbiamo, al primo ordine,
  \[
    \TotalError = \InherentError + \AlgorithmicError + \AnalyticalError,
  \]
  dove
  \begin{itemize}
    \item $\TotalError$ 
  \end{itemize}
\end{Theorem}
\Proof Abbiamo, al primo ordine,
\begin{align*}
  \TotalError
  &= \frac{\overline{\tilde{f}(\tilde{x})} - f(x)}{f(x)},\\
  &= \frac{\overline{\tilde{f}(\tilde{x})}}{f(x)} - 1,\\
  &= \frac{\overline{\tilde{f}(\tilde{x})}}{\tilde{f}(\tilde{x})}
      \frac{\tilde{f}(\tilde{x})}{f(\tilde{x})}
      \frac{f(\tilde{x})}{f(x)} - 1,\\
  &= (\AnalyticalError + 1)
      (\AlgorithmicError + 1)
      (\InherentError + 1) - 1,\\
  &= \AnalyticalError + \AlgorithmicError + \InherentError.\text{ \EndProof}
\end{align*}

\section{Coefficienti di amplificazione.}
\label{AnalisiDegliErrori_CoefficientiDiAmplificazione}

\section{Condizionamento.}
\label{AnalisiDegliErrori_Condizionamento.}
\begin{Definition}
  Siano
  \begin{itemize}
    \item $(\NormedSpace,\Norm{\cdot})$ uno spazio normato;
    \item $(\MetricSpace,\Distance)$ uno spazio metrico;
    \item $f: \NormedSpace \rightarrow \NormedSpace$;
    \item $x \in \NormedSpace$.
  \end{itemize}
  Definiamo
  \Define{numero di condizionamento}[di condizionamento][numero]
  di $f$ in $x$ la quantit\`a
  $\ConditionNumber
  = \limsup_{\Norm{\delta x} \rightarrow 0}
    \frac{\Distance{f(x + \delta x)}{f(x)}}{\Norm{\delta x}}$.
\end{Definition}
\par Nelle applicazioni, spesso abbiamo $\NormedSpace = \MetricSpace$ e
$\Distance$ \`e la distanza indotta da $\Norm{\cdot}$, per cui il numero di
condizionamento sopra definito diventa
$\ConditionNumber
= \limsup_{\Norm{\delta x} \rightarrow 0}
  \frac{\Norm{f(x + \delta x) - f(x)}}{\Norm{\delta x}}$.


	\chapter{Aritmetica in virgola mobile}
	\section{Definizioni e teoremi di base.}
\label{AritmeticaInVirgolaMobile_DefinizioniETeoremiDiBase}
\begin{Definition}
	\label{AritmeticaInVirgolaMobile_DefinizioneNumeriInVirgolaMobile}
	Siano
	\begin{itemize}
		\item $\FloatingPointNumberExponent_{\min},\FloatingPointNumberExponent_{\max} \in \mathbb{Z}$ con $\RealNumberExponent_{\min} \leq \RealNumberExponent_{\max}$;
		\item $\FloatingPointNumberDigits \in \NotZero{\mathbb{N}}$.
	\end{itemize}
	Sono \Define{numeri in virgola mobile}[in virgola mobile][numero] in base $\FloatingPointNumberBase$ di \Define{precisione di macchina}[di macchina][precisione] $\MachinePrecision = \FloatingPointNumberBase^{1 - \FloatingPointNumberDigits}$ gli $x \in \mathbb{Q}$ tali che $x = 0$ o la rappresentazione di $x$ si scrive con al pi\`u $\FloatingPointNumberDigits$ cifre e con esponente $\FloatingPointNumberExponent \in [\FloatingPointNumberExponent_{\min},\FloatingPointNumberExponent_{\max}]$.
\end{Definition}
\begin{Definition}
	\label{AritmeticaInVirgolaMobile_ApprossimazioneInVirgolaMobile}
	Siano
	\begin{itemize}
		\item $x \in \mathbb{R}$;
		\item $\FloatingPointNumberBase \in \NotZero{\mathbb{N}}$;
		\item $\FloatingPointNumberExponent \in \mathbb{Z}$;
		\item $(a_n)_{n \in \mathbb{N}} \in \FloatingPointNumberBase^\mathbb{N}$;
	\end{itemize}
	tali che $x = \Sign{x} \FloatingPointNumberBase^\FloatingPointNumberExponent \sum_{n \in \mathbb{N}} a_n \FloatingPointNumberBase^{- (n + 1)}$. Ora,
	\begin{itemize}
		\item se $\FloatingPointNumberExponent \in [\FloatingPointNumberExponent_{\min},\FloatingPointNumberExponent_{\max}]$, chiamiamo \Define{approssimazione in virgola mobile}[in virgola mobile][approssimazione] di $x$ in base $\FloatingPointNumberBase$ con $\FloatingPointNumberDigits$ \Define{cifre significative}[significativa][cifra] il numero in virgola mobile $\tilde{x} = \Sign{x} \FloatingPointNumberBase^\FloatingPointNumberExponent \sum_{n \in \FloatingPointNumberDigits} a_n \FloatingPointNumberBase^{- (n + 1)}$; in questo caso, chiamiamo inoltre \Define{errore di approssimazione in virgola mobile}[di approssimazione in virgola mobile][errore] l'errore dovuto all'approssimazione di $x$ con $\tilde{x}$;
		\item se $\FloatingPointNumberExponent < \FloatingPointNumberExponent_{\min}$, allora non associamo nessuna approssimazione ad $x$: chiamiamo questa circostanza \Define{\English{underflow}};
		\item se $\FloatingPointNumberExponent > \FloatingPointNumberExponent_{\max}$, allora non associamo nessuna approssimazione ad $x$: chiamiamo questa circostanza \Define{\English{overflow}} (\Define{traboccamento}).
	\end{itemize}
\end{Definition}
\begin{Theorem}
	Con le notazioni della definizione precedente, se approssimiamo un numero reale per cui si verifichi \English{underflow} con $0$ commettiamo un errore relativo del $100\%$.
\end{Theorem}
\Proof Segue per calcolo diretto. \EndProof
\begin{Theorem}
	Con le notazioni della definizione \ref{AritmeticaInVirgolaMobile_ApprossimazioneInVirgolaMobile}, abbiamo, nel caso $\FloatingPointNumberExponent \in [\FloatingPointNumberExponent_{\min},\FloatingPointNumberExponent_{\max}]$, denotata $\MachinePrecision$ la precisione di macchina,
	\begin{itemize}
		\item $\AbsoluteValue{\frac{x - \tilde{x}}{x}} < \MachinePrecision$, cio\`e l'errore relativo di approssimazione in virgola mobile \`e limitato dalla precisione di macchina;
		\item $\AbsoluteValue{\frac{x - \tilde{x}}{\tilde{x}}} < \MachinePrecision$.
	\end{itemize}
\end{Theorem}


	\chapter{Metodi numerici per equazioni}
	\section{Metodo di bisezione.}
\label{MetodiNumericiPerLaRisoluzioneDiEquazioni_MetodoDiBisezioni}

\section{Metodi del punto fisso.}
\label{MetodiNumericiPerLaRisoluzioneDiEquazioni_MetodiDelPuntoFisso}
\begin{Definition}
	Sia $f: I \rightarrow \mathbb{C}$ ($I \subseteq \mathbb{C}$) una funzione. Chiamiamo \Define{metodo del punto fisso}[del punto fisso][metodo] il dato di
	\begin{itemize}
		\item una funzione $g: J \rightarrow J$ ($J\subseteq \mathbb{C}$) tale che $f(x) = 0$ se e solo se $x = g(x)$;
		\item un punto $x_0 \in J$, detto \Define{punto iniziale}[iniziale di un metodo del punto fisso][punto].
	\end{itemize}
	Diciamo che il metodo \Define{converge}[di un metodo del punto fisso][convergenza] a un punto $\xi \in \mathbb{C}$ quando converge a $\xi$ la successione $(x_n)_{n \in \mathbb{N}}$, definita dalla relazione $\ForAll{k \in \mathbb{N}}{x_{n + 1} = g(x_n)}$.
\end{Definition}
\begin{Theorem}
	Con le notazioni della definizione precedente, se $g$ \`e continua e il metodo converge a $\xi$, allora $f(\xi) = 0$.
\end{Theorem}
\Proof Abbiamo $\xi = \lim_{n \rightarrow \infty} x_{n + 1} = \lim_{n \rightarrow \infty} g(x_n) = g \left ( \lim_{n \rightarrow \infty} x_n \right ) = g(\xi)$, che \`e equivalente a $f(\xi) = 0$. \EndProof
\subsection{Metodo delle secanti.}
\label{MetodiNumericiPerLaRisoluzioneDiEquazioni_MetodoDelleSecanti}

\subsection{Metodo di Newton o delle tangenti.}
\label{MetodiNumericiPerEquazioni_MetodoDiNewton}
\begin{Definition}
	Sia $f: I \rightarrow \mathbb{C}$ ($I \subseteq \mathbb{R}^n$) una funzione differenziabile. Chiamiamo \Define{metodo di Newton}[di Newton][metodo] o \Define{metodo delle tangenti}[delle tangenti][metodo] un metodo del punto fisso definito da un punto iniziale qualsiasi e della funzione $g(x) = x - (\Jacobian(x))^{-1} f(x)$, dove $\Jacobian$ \`e lo Jacobiano di $f$.
\end{Definition}
\begin{listing}
	\insertcode{octave}{"Metodi_numerici_per_equazioni/MetodoDiNewton.m"}
	\caption{Implementazione del metodo di Newton in \LanguageName{octave}.}
\end{listing}



	\chapter{Metodi numerici per sistemi lineari}
	\section{Definizioni e teoremi di base.}
\label{MetodiNumericiPerSistemiLineari_DefinizioniETeoremiDiBase}
\begin{Theorem}
  Siano
  \begin{itemize}
    \item $n \in \NotZero{\mathbb{N}}$;
    \item $\Vector, \VarVector \in \mathbb{C}^n$.
  \end{itemize}
  Il costo computazionale del prodotto
  $\Transposed{\Vector}\VarVector$ \`e
  $2n - 1 = \BigO{n}$ operazioni aritmetiche.
\end{Theorem}
\Proof Abbiamo
$\Transposed{\Vector}\VarVector
= \Vector_k\VarVector_k$: il calcolo prevede dunque
\begin{itemize}
  \item $n$ operazioni di coniugazione, che non consideriamo operazioni
    aritmetiche perch\'e possiamo effettuarle con un semplice cambio di
    segno se rappresentiamo le componenti di $\Vector$ in forma cartesiana
    o polare;
  \item $n$ moltiplicazioni;
  \item $n - 1$ somme. \EndProof
\end{itemize}
\begin{Corollary}
  Siano
  \begin{itemize}
    \item $n \in \NotZero{\mathbb{N}}$;
    \item $\Matrix, \VarMatrix \in \mathbb{C}^{n \times n}$.
  \end{itemize}
  Il costo computazionale del prodotto
  $\Matrix\VarMatrix$ \`e
  $2n^3 - n^2 = \BigO{n^3}$ operazioni aritmetiche.
\end{Corollary}
\Proof Il calcolo del prodotto
$\Matrix\VarMatrix$
comporta il calcolo di $n^2$ elementi calcolati come prodotto scalare
di vettori di $n$ elementi. \EndProof

\section{Condizionamento.}
\label{MetodiNumericiPerLaRisoluzioneDiSistemiLineari_Condizionamento}
\begin{Definition}
  Sia $\Matrix \in \mathbb{N}$, chiamiamo
  \Define{numero di condizionamento}[di condizionamento di una matrice][numero]
  la quantit\`a $\MatrixConditionNumber = \Norm{\Matrix}\Norm{\Matrix^{-1}}$.
\end{Definition}
\begin{Theorem}
  Fissato $n \in \NotZero{\mathbb{N}}$
  e denotato $\Set$ l'insieme di tutte le matrici invertibili di
  $\mathbb{C}^{n \times n}$, sia
  $f: \Set \times \mathbb{C}^n \rightarrow \mathbb{C}^n$
  l'applicazione che a
  $(A,B) \in \Set \times \mathbb{C}^n$
  associa $X = A^{-1}B$, cio\`e l'unica soluzione del sistema lineare
  $AX = B$.
  Al primo ordine, il numero di condizionamento di $f$ in
  $(A,B) \in \Set \times \mathbb{C}^n$
  verifica
  \[
    \ConditionNumber \leq \MatrixConditionNumber,
  \]
  dove $\MatrixConditionNumber$ \`e il numero di condizionamento di $A$.
\end{Theorem}
\Proof Siano $\Perturbation{A}$ e $\Perturbation{B}$ le perturbazioni di $A$ e
$B$: esse perturbano la soluzione originale $X \in \mathbb{C}^n$ del sistema
$AX = B$.
\par Abbiamo allora
\[
  (A + \Perturbation{A})(X + \Perturbation{X}) = (B + \Perturbation{B},
\]
equivalente, al primo ordine, a
\[
  AX + A\Perturbation{X} + \Perturbation{A}X = B + \Perturbation{B},
\]
e dunque a
\[
  A\Perturbation{X} + \Perturbation{A}X = \Perturbation{B},
\]
equivalente a
\[
  \Perturbation{X}
  = A^{-1}(\Perturbation{B} - \Perturbation{A}X).
\]
Ne deduciamo
\begin{align*}
  \Norm{\Perturbation{X}}
  &\leq \Norm{A^{-1}(\Perturbation{B} - \Perturbation{A}X)},\\
  &\leq \Norm{A^{-1}}\Norm{\Perturbation{B} - \Perturbation{A}X)},\\
  &\leq \Norm{A^{-1}}(\Norm{\Perturbation{B} + \Perturbation{A}X}),\\
  &\leq \Norm{A^{-1}}\Norm{\Perturbation{B}}.
\end{align*}
\par Inoltre da $AX = B$, deduciamo $\Norm{X} \geq \frac{\Norm{B}}{\Norm{A}}$.
\par Abbiamo infine
\begin{align*}
  \frac{\Perturbation{X}}{\Norm{X}}
  &\leq \Norm{A^{-1}}\Norm{\Perturbation{B}} \cdot \frac{\Norm{A}}{\Norm{B}},\\
  &= \Norm{A}\Norm{A^{-1}},\\
  &= \MatrixConditionNumber.\text{ \EndProof}
\end{align*}

\section{Sistemi Triangolari.}
\label{MetodiNumericiPerSistemiLineari_SistemiTriangolari}
\begin{Definition}
  Sia il sistema
  $\Matrix X = B$ con
  \begin{itemize}
    \item $n, m \in \NotZero{\mathbb{N}^2}$;
    \item $\Matrix \in \mathbb{C}^{n \times m}$ a gradini;
    \item $X \in \mathbb{C}^m$ incognito;
    \item $B \in \mathbb{C}^n$ noto.
  \end{itemize}
  Si chiama
  \Define{sostituzione all'indietro}[all'indietro][sostituzione]
  l'algoritmo seguente:
  \begin{enumerate}
    \item\label{GradiniSuperiore_0} fissiamo arbitrariamente
      $X \in \mathbb{C}^m$;
    \item\label{GradiniSuperiore_1} poniamo $k = n - 1$;
    \item\label{GradiniSuperiore_2} poniamo $q$ uguale al minimo indice
      $r$ tale che $\Matrix_k^r \neq 0$ e
      $X_q = (T_k^q)^{-1}B_k - \sum_{r = q + 1}^{n - 1} T_k^r X_r$;
    \item\label{GradiniSuperiore_3} decrementiamo $k$ di $1$;
    \item\label{GradiniSuperiore_4} se $k \geq 0$ torniamo al punto
      \ref{GradiniSuperiore_1}, altrimenti l'algoritmo termina.
  \end{enumerate}
\end{Definition}
\begin{listing}
	\insertcode{octave}{"Metodi_numerici_per_sistemi_lineari/SostituzioneAllIndietro.m"}
	\caption{Implementazione della sostituzione all'indietro in \LanguageName{octave}.}
\end{listing}
\begin{Theorem}
  Con le notazioni della definizione precedente, la sostituzione
  all'indietro risolve il sistema $\Matrix X = B$
  L'algoritmo costa $\BigO{n^2}$ operazioni aritmetiche.
\end{Theorem}
\Proof Verifichiamo la correttezza dell'algoritmo per induzione. Il caso $n = 1$
\`e immediato. Per $n > 1$, basta osservare che i primi $n - 1$ cicli
dell'algoritmo sono gli stessi che risolvono il sistema proiettato sulle ultime
$n - 1$ componenti e che l'ultimo ciclo risolve con successo l'equazione
definita dalla prima riga del sistema.
\par Abbiamo, ad ogni ciclo,
\begin{itemize}
  \item $n - 1 - (q + 1) + 1 + 1 = n - q$ somme;
  \item $n - 1 - (q + 1) + 1 = n - q - 1$ moltiplicazioni;
  \item $1$ divisione.
\end{itemize}
\par Ad ogni ciclo, $q$ \`e al pi\`u $k$; quindi in totale le operazioni
sono al pi\`u
\begin{align*}
  \sum_{k = 0}^{n - 1} 2n - 2k
  &= 2n^2 - 2 \frac{(n - 1)n}{2},\\
  &= 2n^2 - n^2 + n,\\
  &= n^2 + n,\\
  &= \BigO{n^2}.\text{ \EndProof}
\end{align*}

\section{Matrici elementari.}
\label{MetodiNumericiPerLaRisoluzioneDiSistemiLineari_MatriciElementari}
\begin{Definition}
  Chiamiamo
  \Define{matrice elementare}[elementare][matrice]
  una matrice
  $\ElementaryMatrix \in \mathbb{C}^{n \times n}$ ($n \in \NotZero{\mathbb{N}}$)
  tale che
  $\ElementaryMatrix
  = \Identity - \Scalar\Vector\HermitianTransposed{\VarVector}$, con
  \begin{itemize}
    \item $\Scalar \in \mathbb{C}^{n \times n}$;
    \item $\Vector, \VarVector \in \mathbb{C}^{n \times n}$.
  \end{itemize}
\end{Definition}
\begin{Theorem}
  Con le notazioni della definizione precedente, $\ElementaryMatrix$ \`e
  invertibile se e solo se
  $\Scalar\HermitianTransposed{\VarVector}\Vector \neq 1$.
  Quando esistente, $\ElementaryMatrix^{-1}$ \`e una matrice elementare data da
  $\ElementaryMatrix^{-1}
  = \Identity - \VarScalar\Vector\HermitianTransposed{\VarVector}$, con
  $\VarScalar
  = - \frac{\sigma}{\sigma\HermitianTransposed{\VarVector}\Vector - 1}$.
\end{Theorem}
\Proof Supponiamo $\ElementaryMatrix$ invertibile. Abbiamo
\begin{align*}
  \MatrixElement\Vector
  &= \Vector - \Scalar\Vector\HermitianTransposed{\VarVector}\Vector,\\
  &= (1 - \Scalar\HermitianTransposed{\VarVector}\Vector)\Vector.
\end{align*}
Dunque deve essere $1 - \Scalar\HermitianTransposed{\VarVector}\Vector \neq 0$.
\par Supponiamo ora invece al contrario $\ElementaryMatrix$ non invertibile.
Allora esiste $x \in \NotZero{\mathbb{C}^n}$ tale che $\ElementaryMatrix x = 0$,
vale a dire
$x - \Scalar\Vector\HermitianTransposed{\VarVector} x
(1 - \Scalar\Vector\HermitianTransposed{\VarVector}) x
= 0$
e dunque
$\Scalar\Vector\HermitianTransposed{\VarVector} = 1$.
\par La parte restante del teorema si dimostra facilmente per calcolo diretto.
\EndProof
\begin{Lemma}
  \label{MetodiNumericiPerLaRisoluzioneDiSistemiLineari_LemmaRango1}
  Siano
  \begin{itemize}
    \item $n \in \NotZero{\mathbb{N}}$;
    \item $\Matrix \in \mathbb{C}^{n \times n}$;
    \item $\Vector, \VarVector \in \mathbb{C}^n$.
  \end{itemize}
  Se
  $\Matrix = \Vector\HermitianTransposed{\VarVector}$,
  allora
  \begin{itemize}
    \item $\MatrixRank{\Matrix} = 1$;
    \item $\Norm{\Matrix}[2] = \Norm{\Vector}[2] \Norm{\VarVector}[2]$.
  \end{itemize}
\end{Lemma}
\Proof Poich\'e
$\Matrix = \Vector\HermitianTransposed{\VarVector}$,
per ogni $k \in \NotZero{n}$ abbiamo la seguente relazione sulle colonne di
$\Matrix$:
$\Matrix^k = \frac{\VarVector_k}{\VarVector_0} \Matrix^0$.
Dunque $\MatrixRank{\Matrix} = 1$.
\par Siano ora $\UnitaryMatrix$ e $\VarUnitaryMatrix$ matrici unitarie le cui
prime colonne sono
$\UnitaryMatrix^0 = \frac{\Vector}{\Norm{\Vector}}$
e
$\VarUnitaryMatrix^0 = \frac{\VarVector}{\Norm{\VarVector}}$
rispettivamente.
\par Abbiamo
$\Norm{\UnitaryMatrix^{-1}\Matrix\VarUnitaryMatrix}[2]
= \Norm{\Matrix}[2]$
e
$\UnitaryMatrix^{-1}\Matrix\VarUnitaryMatrix$ \`e una matrice interamente
nulla tranne che nell'elemento della prima riga e della prima colonna
che \`e uguale a
$\Norm{\Vector}[2] \Norm{\VarVector}[2]$,
da cui
$\Norm{\Matrix} =
\Norm{\Vector}[2] \Norm{\VarVector}[2]$. \EndProof
\begin{Theorem}
  La risoluzione del sistema
  $\ElementaryMatrix X = B$ con
  \begin{itemize}
    \item $n \in \NotZero{\mathbb{N}}$;
    \item $\ElementaryMatrix \in \mathbb{C}^{n \times n}$ elementare;
    \item $X \in \mathbb{C}^n$ incognito;
    \item $B \in \mathbb{C}^n$ noto;
  \end{itemize}
  pu\`o essere effettuata con
  \begin{itemize}
    \item $3n$ addizioni;
    \item $3n + 1$ moltiplicazioni;
    \item $1$ divisione;
  \end{itemize}
  per un totale di $\BigO{n}$ operazioni aritmetiche.
\end{Theorem}
\begin{Corollary}
  La risoluzione del sistema
  $\Matrix X = B$ con
  \begin{itemize}
    \item $n \in \NotZero{\mathbb{N}}$;
    \item $\VarMatrix \in \mathbb{C}^{n \times n}$ prodotto di matrici
      elementari invertibili;
    \item $\TriangularMatrix \in \mathbb{C}^{n \times n}$ trriangolare
      superiore.
    \item $\Matrix = \VarMatrix\TriangularMatrix$;
    \item $X \in \mathbb{C}^n$ incognito;
    \item $B \in \mathbb{C}^n$ noto;
  \end{itemize}
  costa $\BigO{n^2}$ operazioni aritmetiche.
\end{Corollary}
\Proof Si risolve prima il sistema $\VarMatrix X = B$ ottenendo una soluzione
$Y$ con $\BigO{n}$ operazioni aritmetiche; dunque si risolve il sistema
$\TriangularMatrix X = Y$ con $\BigO{n^2}$ operazioni aritmetiche,
ottenendo la soluzione del sistema originale con un totale di $\BigO{n^2}$
operazioni aritmetiche. \EndProof
\begin{Theorem}
  Siano
  \begin{itemize}
    \item $n \in \NotZero{\mathbb{N}}$;
    \item $\VarVarVector, \VarVarVarVector \in \NotZero{\mathbb{C}^n}$.
  \end{itemize}
  Esiste una matrice elementare non singolare $\ElementaryMatrix$ tale che
  $\ElementaryMatrix\VarVarVector = \VarVarVarVector$.
\end{Theorem}
\Proof Siano
\begin{itemize}
  \item $\Vector = \VarVarVector - \VarVarVarVector$;
  \item $\VarVector \in \mathbb{C}^n$ tale che esso non sia ortogonale n\'e a
    $\VarVarVector$ n\'e a $\VarVarVarVector$
    \footnote{Una tale scelta \`e sempre possibile:
    \begin{itemize}
      \item se $\HermitianTransposed{\VarVarVector}{\VarVarVarVector} = 0$,
        allora si pu\`o scegliere
        $\VarVector = \VarVarVector + \VarVarVarVector$;
      \item se $\HermitianTransposed{\VarVarVector}{\VarVarVarVector} \neq 0$,
        allora si pu\`o scegliere
        $\VarVector = \VarVarVector$.
    \end{itemize}};
  \item $\Scalar = (\HermitianTransposed{\VarVector}\VarVarVector)^{-1}$.
\end{itemize}
\par Poniamo
$\ElementaryMatrix
= \Identity - \Scalar\Vector\HermitianTransposed{\VarVector}$.
\par Abbiamo
\begin{align*}
  \ElementaryMatrix \VarVarVector
  &= \VarVarVector
    - (\HermitianTransposed{\VarVector}\VarVarVector)^{-1}
    (\VarVarVector - \VarVarVarVector)
    \HermitianTransposed{\VarVector}\VarVarVector,\\
  &= \VarVarVector
    - (\HermitianTransposed{\VarVector}\VarVarVector)^{-1}
    \VarVarVector
    \HermitianTransposed{\VarVector}\VarVarVector
    + (\HermitianTransposed{\VarVector}\VarVarVector)^{-1}
    \VarVarVarVector
    \HermitianTransposed{\VarVector}\VarVarVector,\\
  &= \VarVarVarVector.
\end{align*}
\par Inoltre,
\begin{align*}
  \Scalar\HermitianTransposed{\VarVector}\Vector
  &= (\HermitianTransposed{\VarVector}\VarVarVector)^{-1}
    \HermitianTransposed{\VarVector}
    (\VarVarVector - \VarVarVarVector),\\
  &= (\HermitianTransposed{\VarVector}\VarVarVector)^{-1}
    \HermitianTransposed{\VarVector}
    \VarVarVector
    - (\HermitianTransposed{\VarVector}\VarVarVector)^{-1}
    \HermitianTransposed{\VarVector}
    \VarVarVarVector,\\
  &= 1
    - (\HermitianTransposed{\VarVector}\VarVarVector)^{-1}
    \HermitianTransposed{\VarVector}
    \VarVarVarVector,\\
  &\neq 1,
\end{align*}
da cui l'invertibilit\`a di $\ElementaryMatrix$. \EndProof
\begin{Theorem}
  \label{th_FattorizzazioneMatriciElementari}
  Siano
  \begin{itemize}
    \item $n, m \in \NotZero{\mathbb{N}}$;
    \item $\Matrix \in \mathbb{C}^{n \times m}$.
  \end{itemize}
  Il seguente algoritmo costruisce una fattorizzazione
  $\Matrix = A \TriangularMatrix$ dove $A$ \`e prodotto di matrici
  elementari e $\TriangularMatrix$ \`e triangolare superiore:
  \begin{enumerate}
    \item\label{FattorizzazioneMatriciElementari_1} si ponga $l = 0$;
    \item\label{FattorizzazioneMatriciElementari_2} si definisca
      $\VarMatrix = (\Matrix_k^j)_{\substack{k = l\\j= l}}^{\substack{k = n - 1}{j = m - 1}}$;
    \item\label{FattorizzazioneMatriciElementari_3} si fissi una matrice
      elementare non singolare $\ElementaryMatrix_{(l,0)}$ tale che
      $\ElementaryMatrix_{(l,0)}\VarMatrix = \CanonicalBase_1^{(l)}$, dove
      $\CanonicalBase_1^{(l)}$ \`e il $l$-esimo vettore della base
      canonica di $\mathbb{C}^{m - l}$;
    \item\label{FattorizzazioneMatriciElementari_4} si ponga
      $(\ElementaryMatrix_{(l)})_k^j =
      \begin{cases}
        1&\text{ se }k = j < l,\\
        (\ElementaryMatrix_{(l,0)})_{k - l}^{j - l}&\text{ se }\And{k \geq l}{j \geq l},\\
        0&\text{ altrimenti},
      \end{cases}$
      vale a dire
      \[
        \ElementaryMatrix_{(l)} =
        \lmatrix
        \begin{array}{c|c}
          \Identity &\\
          \hline
          & \ElementaryMatrix_{(l,0)}
        \end{array}
        \rmatrix;
      \]
    \item\label{FattorizzazioneMatriciElementari_5} si ridefinisca
      $\Matrix = \ElementaryMatrix_{(l)}\Matrix$;
    \item\label{FattorizzazioneMatriciElementari_6} si incrementi
      $l$ di $1$;
    \item\label{FattorizzazioneMatriciElementari_7} se $l < m$, si torni al
      punto \ref{FattorizzazioneMatriciElementari_2};
    \item\label{FattorizzazioneMatriciElementari_8} si ponga
      $A = \ElementaryMatrix_{(m - 1)}^{-1}\ElementaryMatrix_{(m - 2)}^{-1}
      \cdots
      \ElementaryMatrix_{(1)}^{-1}\ElementaryMatrix_{(0)}^{-1}$;
    \item\label{FattorizzazioneMatriciElementari_9} si ponga
      $\TriangularMatrix = \Matrix$;
    \item\label{FattorizzazioneMatriciElementari_10} l'algoritmo termina restituendo
      $A$ e $\TriangularMatrix$.
  \end{enumerate}
\end{Theorem}
\Proof Si dimostra facilmente per induzione che $\TriangularMatrix$ \`e triangolare.
\par Per dimstrare che $A$ \`e prodotto di matrici elementari \`e sufficiente notare che
ogni $\ElementaryMatrix_{(l)}$ al variare di $l \in m$ \`e elementare. In effetti,
$\ElementaryMatrix_{(l,0)}$ \`e elementare per costruzione: esistono dunque uno scalare
$\Scalar \in \mathbb{C}$ e vettori $\Vector,\VarVector \in \mathbb{C}^{m - l}$  tali che
$\ElementaryMatrix_{(l,0)} = \Identity - \Scalar\Vector\VarVector$.
Per calcolo diretto si verifica che
$\ElementaryMatrix_{(l,0)} = \Identity - \Scalar\tilde{\Vector}\tilde{\VarVector}$, dove
\begin{itemize}
  \item la matrice $\Identity$ \`e stavolta l'identit\`a di $\mathbb{C}^m$;
  \item $\tilde{\Vector} \in \mathbb{C}^m$ \`e tale che
    $\tilde{\Vector}_k =
    \begin{cases}
      0&\text{ se }k < l,\\
      \Vector_{k - l}&\text{ altrimenti};
    \end{cases}$
  \item $\tilde{\VarVector} \in \mathbb{C}^m$ \`e tale che
    $\tilde{\Vector}_k =
    \begin{cases}
      0&\text{ se }k < l,\\
      \VarVector_{k - l}&\text{ altrimenti}.\text{ \EndProof}
    \end{cases}$
\end{itemize}

\section{Fattorizzazione $LU$.}
\label{MetodiNumericiPerLaRisoluzioneDiSistemiLineari_FattorizzazioneLU}
\begin{Definition}
  Chiamiamo
  \Define{matrice elementare di Gauss}[elementare di Gauss][matrice]
  una matrice elementare
  $\GaussElementaryMatrix \in \mathbb{C}^{n \times n}$
  ($n \in \NotZero{\mathbb{N}}$)
  tale che
  $\GaussElementaryMatrix
  = \Identity - \Vector\HermitianTransposed{\CanonicalBase_0}$, con
  \begin{itemize}
    \item $\Scalar \in \mathbb{C}^{n \times n}$;
    \item $\Vector \in \mathbb{C}^{n \times n}$, dove $\Vector_0 = 0$;
    \item $\CanonicalBase_0$ primo vettore della base canonica.
  \end{itemize}
  Una tale matrice ha la forma
  \[
    \GaussElementaryMatrix =
    \lmatrix
    \begin{array}{cccc}
      1                 &   &         &   \\
      - \Vector_1       & 1 &         &   \\
      \vdots            &   & \ddots  &   \\
      - \Vector_{n - 1} &   &         & 1 \\
    \end{array}
    \rmatrix.
  \]
\end{Definition}
\begin{Theorem}
  Con le notazioni della definizione precedente,
  $\GaussElementaryMatrix$ \`e invertibile e la sua inversa \`e
  \[
    \GaussElementaryMatrix =
    \lmatrix
    \begin{array}{cccc}
      1                 &   &         &   \\
      \Vector_1         & 1 &         &   \\
      \vdots            &   & \ddots  &   \\
      \Vector_{n - 1}   &   &         & 1 \\
    \end{array}
    \rmatrix.
  \]
  L'inversa \`e calcolabile con $0$ operazioni aritmetiche.
\end{Theorem}
\Proof Segue per calcolo diretto e per il fatto che l'inversa \`e calcolabile
con $n - 1$ cambi di segno, che non sono operazioni aritmetiche. \EndProof
\begin{Theorem}
  Siano
  \begin{itemize}
    \item $n \in \NotZero{\mathbb{N}}$;
    \item $\VarVarVector \in \NotZero{\mathbb{C}^n}$ tale che
      $\VarVarVector_0 \neq 0$.
  \end{itemize}
  Sia
  $\GaussElementaryMatrix
  = \Identity - \Vector\HermitianTransposed{\CanonicalBase^0}$,
  con
  \begin{itemize}
    \item $e^0$ primo vettore della base canonica;
    \item $\Vector_k =
            \begin{cases}
              1&\text{ se }k = 0,\\
              \frac{\VarVarVector_k}{\VarVarVector_0}&\text{ se }k > 0,
          \end{cases}$
  \end{itemize}
  vale a dire
  \[
    \GaussElementaryMatrix =
    \lmatrix
      \begin{array}{cccc}
        1 &&&\\
        - \frac{\VarVarVector_{2}}{\VarVarVector_0}         & \ddots & &\\
        \vdots            & & \ddots &  \\
        - \frac{\VarVarVector_{n - 1}}{\VarVarVector_0}   &   & & 1 \\
      \end{array}
    \rmatrix.
  \]
  $\GaussElementaryMatrix$ \`e l'unica matrice elementare di Gauss tale che
  \begin{itemize}
    \item $\GaussElementaryMatrix\VarVarVector_0 = \VarVarVector_0$;
    \item $\ForAll{k \in \NotZero{n}}{(\GaussElementaryMatrix\VarVarVector)_k = 0}$.
  \end{itemize}
\end{Theorem}
\Proof Si verifica direttamente che $\GaussElementaryMatrix$ verifica le relazioni richieste.
\par Supponiamo ora
$\GaussElementaryMatrix' = \Identity - \VarVector \HermitianTransposed{\CanonicalBase^0}$
con $\VarVector \in \mathbb{C}^n$.
\par Se $\GaussElementaryMatrix'$ verifica le relazioni richieste, abbiamo allora,
\[
  (\Identity - \VarVector \HermitianTransposed{\CanonicalBase^0})\VarVarVector
  = \VarVarVector_0 \CanonicalBase^0,
\]
a sua volta equivalente a
\begin{align*}
  \VarVector
  &= \frac{\VarVarVector - \VarVarVector_0 \CanonicalBase^0}
    {\HermitianTransposed{\CanonicalBase^0}\VarVarVector},\\
  &= \frac{\VarVarVector}{\VarVarVector_0} - \CanonicalBase_0,\\
  &= \VarVector.\text{ \EndProof}
\end{align*}
\begin{Theorem}
  Siano
  \begin{itemize}
    \item $n \in \NotZero{\mathbb{N}}$;
    \item $\Vector \in \mathbb{C}^n$, con $\Vector_0 = 0$;
    \item $\CanonicalBase_0 \in \mathbb{C}^n$ primo vettore della base canonica;
    \item $\GaussElementaryMatrix = \Identity - \Vector\HermitianTransposed{\CanonicalBase_0}$
      matrice elementare di Gauss;
    \item $\PermutationMatrix \in \mathbb{C}^{n \times n}$ matrice di permutazione tale che
      la permutazione $\Permutation \in \SymmetricGroup{n}$ associata alle sue righe lascia
      fisso $0$.
  \end{itemize}
  $\PermutationMatrix\GaussElementaryMatrix\PermutationMatrix^{-1}$
  \`e una matrice elementare di Guass.
\end{Theorem}
\Proof Abbiamo, per $(k,j) \in n \times n$,
$(\PermutationMatrix\Vector\HermitianTransposed{\CanonicalBase_0})_k^j
= \begin{cases}
  \Vector_{\Permutation(k)}&\text{ se }j = 0,\\
  0&\text{ altrimenti}.
\end{cases}$
\par Poich\'e il prodotto a destra per $\PermutationMatrix$ lascia
invariata la prima riga del secondo fattore, il prodotto a sinistra
per $\PermutationMatrix$ lascia invariata la prima colonna del primo fattore.
Ne consegue
$(\PermutationMatrix\Vector\HermitianTransposed{\CanonicalBase_0}\PermutationMatrix^{-1})_k^j
= \begin{cases}
  \Vector_{\Permutation(k)}&\text{ se }j = 0,\\
  0&\text{ altrimenti}.
\end{cases}$
\par Dunque
$\PermutationMatrix\GaussElementaryMatrix\PermutationMatrix^{-1}$
\`e una matrice elementare di Guass. \EndProof
\begin{Definition}
  Sia $\Matrix \in \mathbb{C}^{n \times n}$ ($n \in \mathbb{N}$). Chiamiamo
  \Define{fattorizzazione $LU$}[$LU$][fattorizzazione]
  o
  \Define{decomposizione $LU$}[$LU$][decompisionzione]
  di $\Matrix$ una coppia di matrici
  $(L,U) \in \mathbb{C}^{n \times n} \times \mathbb{C}^{n \times n}$
  tali che
  \begin{itemize}
    \item $\Matrix = LU$;
    \item $L$ \`e triangolare inferiore con tutti gli elementi diagonali uguali
      a $1$;
    \item $U$ \`e triangolare superiore.
  \end{itemize}
  Chiamiamo inoltre
  \Define{fattorizzazione $LUP$}[$LUP$][fattorizzazione]
  o
  \Define{decomposizione $LUP$}[$LUP$][decomposizione]
  di $\Matrix$ una terna di matrici
  $(L,U,P) \in
  \mathbb{C}^{n \times n} \times \mathbb{C}^{n \times n} \times \mathbb{C}^{n \times n}$
  tali che
  \begin{itemize}
    \item $P \Matrix = LU$;
    \item $L$ \`e triangolare inferiore con tutti gli elementi diagonali uguali
      a $1$;
    \item $U$ \`e triangolare superiore;
    \item $P$ \`e una matrice di permutazione.
  \end{itemize}
\end{Definition}
\begin{Definition}
  Con le notazioni del teorema \ref{th_FattorizzazioneMatriciElementari},
  l'algoritmo ottenuto dall'algoritmo del teorema \ref{th_FattorizzazioneMatriciElementari}
  scegliendo come matrici $(\ElementaryMatrix_{(l,0)})_{l \in m}$ matrici elementari di Gauss
  si chiama
  \Define{metodo di Gauss}[di Gauss][metodo].
\end{Definition}
\begin{listing}
	\insertcode{octave}{"Metodi_numerici_per_sistemi_lineari/FattorizzazioneLU.m"}
	\caption{Implementazione del metodo di Gauss in \LanguageName{octave}.}
\end{listing}
\begin{Theorem}
  Con le notazioni del teorema \ref{th_FattorizzazioneMatriciElementari}, il metodo di Gauss
  restituisce una fattorizzazione $LU$ di $\Matrix$.
\end{Theorem}
\Proof $L$ \`e prodotto di matrici triangolari inferiori con tutti gli elementi diagonali
uguali a $1$, dunque \`e trinagolare inferiore con tutti gli elementi diagonali uguali a $1$
essa stessa. \EndProof
\begin{Theorem}
  Se $\Matrix \in \mathbb{C}^{n \times n}$ ($n \in \NotZero{\mathbb{N}}$) ha tutte le
  sottomatrici principali invertibili, allora ammette una fattorizzazione $LU$.
\end{Theorem}
\begin{Theorem}
  \label{th_FattorizzazioneLUP}
  Siano
  \begin{itemize}
    \item $n, m \in \NotZero{\mathbb{N}}$;
    \item $\Matrix \in \mathbb{C}^{n \times m}$.
  \end{itemize}
  Il seguente algoritmo costruisce una fattorizzazione $LUP$ di
  $\Matrix$
  \begin{enumerate}
    \item\label{FattorizzazioneLUP_1} si ponga $l = 0$, $L = \Identity[n]$;
    \item\label{FattorizzazioneLUP_2} si definisca
      $\VarMatrix = (\Matrix_k^j)_{\substack{k = l\\j= l}}^{\substack{k = n - 1}{j = m - 1}}$;
    \item\label{FattorizzazioneLUP_2_1} se possibile, si scelga una matrice di permutazione
      $\PermutationMatrix_{(l)} \in \mathbb{C}^{n \times n}$ che permuti la riga $l$ di
      $\Matrix$ con una riga $r \geq l$ tale che $\Matrix_r^l \neq 0$; se non \`e possibile
      si vada al punto \ref{FattorizzazioneLUP_9};
    \item\label{FattorizzazioneLUP_2_2} si ridefinisca
      $\Matrix = \PermutationMatrix_{(l)} \Matrix$;
    \item\label{FattorizzazioneLUP_2_2} si ridefinisca
      $L = \Transposed{\PermutationMatrix_{(l)}} \Matrix \PermutationMatrix$;;
    \item\label{FattorizzazioneLUP_3} sia $\GaussElementaryMatrix_{(l,0)}$ la matrice elementare
      di Gauss tale che
      $\GaussElementaryMatrix_{(l,0)}\VarMatrix = \CanonicalBase_1^{(l)}$, dove
      $\CanonicalBase_1^{(l)}$ \`e il $l$-esimo vettore della base
      canonica di $\mathbb{C}^{m - l}$;
    \item\label{FattorizzazioneLUP_4} si ponga
      $(\ElementaryMatrix_{(l)})_k^j =
      \begin{cases}
        1&\text{ se }k = j < l,\\
        (\GaussElementaryMatrix_{(l,0)})_{k - l}^{j - l}&\text{ se }\And{k \geq l}{j \geq l},\\
        0&\text{ altrimenti},
      \end{cases}$
      vale a dire
      \[
        \ElementaryMatrix_{(l)} =
        \lmatrix
        \begin{array}{c|c}
          \Identity &\\
          \hline
          & \GaussElementaryMatrix_{(l,0)}
        \end{array}
        \rmatrix;
      \]
    \item\label{FattorizzazioneLUP_5} si ridefinisca
      $\Matrix = \ElementaryMatrix_{(l)}\Matrix$;
    \item\label{FattorizzazioneLUP_6} si incrementi
      $l$ di $1$;
    \item\label{FattorizzazioneLUP_6_1} si ridefinisca
      $L = L \ElementaryMatrix^{-1}$;
    \item\label{FattorizzazioneLUP_7} se $l < m$, si torni al
      punto \ref{FattorizzazioneLUP_2};
    \item\label{FattorizzazioneLUP_9} si ponga
      $U = \Matrix$;
    \item\label{FattorizzazioneLUP_9_1} si ponga
      $\PermutationMatrix = \PermutationMatrix_{(0)} \cdot ... \cdot \PermutationMatrix_{(m)}$;
    \item\label{FattorizzazioneLUP_10} l'algoritmo termina restituendo
      $L$, $U$ e $P$.
  \end{enumerate}
\end{Theorem}
\begin{listing}
	\insertcode{octave}{"Metodi_numerici_per_sistemi_lineari/FattorizzazioneLUP.m"}
	\caption{Implementazione di un algoritmo per il calcolo della decomposizione $LUP$ in
    \LanguageName{octave}.}
\end{listing}
\par In molte applicazioni, siamo interessati ad utilizzare esclusivamente la matrice $U$ di
una decomposizione $LUP$: per esempio essa pu\`o essere usata per calcolare rapidamente il
rango della matrice di partenza $\Matrix$, oppure si pu\`o ridurre a scalini la matrice completa
di un sistema per perlo poi risolvere tramite sostituzione all'indietro. In tal caso, l'algoritmo
seguente consente di ottenere la matrice $U$ senza impiegare tempo di calcolo ulteriore per le
matrici $L$ e $P$.
\begin{Definition}
  \label{def_EliminazioneGaussiana}
  Siano
  \begin{itemize}
    \item $n, m \in \NotZero{\mathbb{N}}$;
    \item $\Matrix \in \mathbb{C}^{n \times m}$.
  \end{itemize}
  Chiamiamo
  \Definition{eliminazione gaussiana}[gaussiana][eliminazione]
  il seguente algoritmo
  \begin{enumerate}
    \item\label{EliminazioneGaussiana_1} si ponga $l = 0$, $L = \Identity[n]$;
    \item\label{EliminazioneGaussiana_2} si definisca
      $\VarMatrix = (\Matrix_k^j)_{\substack{k = l\\j= l}}^{\substack{k = n - 1}{j = m - 1}}$;
    \item\label{EliminazioneGaussiana_2_1} se possibile, si scambino la riga $l$ di $\Matrix$
      con una riga $r \geq l$ di $\Matrix$ in modo tale che, dopo lo scambio, $\Matrix_l^l \neq 0$;
      se non \`e possibile si vada al punto \ref{EliminazioneGaussiana_9};
    \item\label{EliminazioneGaussiana_3} sia $\GaussElementaryMatrix_{(l,0)}$ la matrice elementare
      di Gauss tale che
      $\GaussElementaryMatrix_{(l,0)}\VarMatrix = \CanonicalBase_1^{(l)}$, dove
      $\CanonicalBase_1^{(l)}$ \`e il $l$-esimo vettore della base
      canonica di $\mathbb{C}^{m - l}$;
    \item\label{EliminazioneGaussiana_4} si ponga
      $(\ElementaryMatrix_{(l)})_k^j =
      \begin{cases}
        1&\text{ se }k = j < l,\\
        (\GaussElementaryMatrix_{(l,0)})_{k - l}^{j - l}&\text{ se }\And{k \geq l}{j \geq l},\\
        0&\text{ altrimenti},
      \end{cases}$
      vale a dire
      \[
        \ElementaryMatrix_{(l)} =
        \lmatrix
        \begin{array}{c|c}
          \Identity &\\
          \hline
          & \GaussElementaryMatrix_{(l,0)}
        \end{array}
        \rmatrix;
      \]
    \item\label{EliminazioneGaussiana_5} si ridefinisca
      $\Matrix = \ElementaryMatrix_{(l)}\Matrix$;
    \item\label{EliminazioneGaussiana_6} si incrementi
      $l$ di $1$;
    \item\label{EliminazioneGaussiana_7} se $l < m$, si torni al
      punto \ref{EliminazioneGaussiana_2};
      $U$.
  \end{enumerate}
\end{Definition}
\begin{listing}
	\insertcode{octave}{"Metodi_numerici_per_sistemi_lineari/EliminazioneGaussiana.m"}
	\caption{Implementazione dell'eliminazione gaussiana in \LanguageName{octave}.}
\end{listing}
\begin{Theorem}
  Con le notazioni della definizione precedente, l'eliminazione di Gauss restituisce la stessa
  matrice triangolare superiore $U$ dell'algoritmo del teorema \ref{th_FattorizzazioneLUP}.
\end{Theorem}
\Proof Segue immediatamente dalle definizioni degli algoritmi. \EndProof

\section{Fattorizzazione QR.}
\label{MetodiNumericiPerLaRisoluzioneDiSistemiLineari_FattorizzazioneQR}
\begin{Definition}
  Chiamiamo
  \Define{matrice elementare di Householder}[elementare di Householder][matrice]
  o anche
  \Define{riflettore di Householder}[di Householder][riflettore]
  una matrice elementare $\HouseholderElementaryMatrix$ hermitiana e unitaria.
\end{Definition}
\begin{Theorem}
  Una matrice elementare di Householder $\HouseholderElementaryMatrix$ \`e
  l'inversa di s\'e stessa.
\end{Theorem}
\Proof Abbiamo
$\HouseholderElementaryMatrix^{-1}
= \HermitianTransposed{\HouseholderElementaryMatrix}$
per unitariet\`a e
$\HermitianTransposed{\HouseholderElementaryMatrix}
= \HouseholderElementaryMatrix$
per hermitianit\`a. \EndProof
\begin{Theorem}
  Una matrice $\HouseholderElementaryMatrix$ di dimensione
  $n \in \NotZero{\mathbb{N}}$ e diversa dall'identit\`a \`e una matrice elementare di
  Householder se e solo se
  \[
    \HouseholderElementaryMatrix
    = \Identity
    - \frac{2}{\HermitianTransposed{\Vector}\Vector}
    \Vector\HermitianTransposed{\Vector},
  \]
  per $\Vector \in \mathbb{C}^n$ opportuno.
\end{Theorem}
\Proof Supponiamo $\HouseholderElementaryMatrix$ sia una matrice elementare di Householder.
\par Abbiamo, per opportuni $\Vector, \VarVector \in \mathbb{C}^n$ e $\Scalar \in \mathbb{C}$,
$\HouseholderElementaryMatrix = \Identity - \Scalar\Vector\HermitianTransposed{\VarVector}$.
Per l'hermitianit\`a di $\HouseholderElementaryMatrix$ e di $\Identity$, deve essere
hermitiana anche $\Scalar\Vector\HermitianTransposed{\VarVector}$:
\[
  \Scalar\Vector\HermitianTransposed{\VarVector}
  = \Conjugate{\Scalar}\VarVector\HermitianTransposed{\Vector}.
\]
Moltiplicando a destra per $\VarVector$, otteniamo
\[
  \Scalar\Norm{\VarVector}_2^2\Vector
  = \Conjugate{\Scalar}\VarVector\HermitianTransposed{\Vector}\VarVector;
\]
quindi $\VarVector$ \`e un multiplo di $\Vector$: a meno di modificare la costante
$\Scalar$ possiamo assumere senza perdita di generalit\`a $\Vector = \VarVector$.
\par Per l'unitariet\`a di $\HouseholderElementaryMatrix$, abbiamo
\begin{align*}
  \Identity
  &= \HouseholderElementaryMatrix\HouseholderElementaryMatrix^{-1},\\
  &= \HouseholderElementaryMatrix\HouseholderElementaryMatrix,\\
  &= (\Identity - \Scalar\Vector\HermitianTransposed{\Vector})
      (\Identity - \Scalar\Vector\HermitianTransposed{\Vector}),\\
  &= \Identity
      - 2 \Scalar\Vector\HermitianTransposed{\Vector}
      + \Scalar^2\Vector\HermitianTransposed{\Vector}\Vector\HermitianTransposed{\Vector}.
\end{align*}
Ci\`o \`e equivalente a
\[
  \Scalar(\Scalar\HermitianTransposed{\Vector}\Vector - 2)\Vector\HermitianTransposed{\Vector})
  = 0.
\]
\par Ora, se $\HouseholderElementaryMatrix \neq \Identity$, allora necessariamente
$\Scalar \neq 0$ e $\HermitianTransposed{\Vector}{\Vector} \neq 0$. Quindi infine
\[
  \Scalar = \frac{2}{\HermitianTransposed{\Vector}\Vector}.
\]
\par Assumiamo ora invece che $\HouseholderElementaryMatrix$ sia della forma
\[
  \HouseholderElementaryMatrix
  = \Identity
  - \frac{2}{\HermitianTransposed{\Vector}\Vector}
  \Vector\HermitianTransposed{\Vector},
\]
per $\Vector \in \mathbb{C}^n$ opportuno.
Segue per calcolo diretto che $\HouseholderElementaryMatrix$ \`e hermitiana e unitaria. \EndProof
\begin{Theorem}
  \label{th_Householder}
  Siano
  \begin{itemize}
    \item $n \in \NotZero{\mathbb{N}}$;
    \item $\VarVarVector \in \NotZero{\mathbb{C}^n}$.
  \end{itemize}
  Sia
  $\HouseholderElementaryMatrix
  = \Identity - \Scalar \Vector\HermitianTransposed{\Vector}$,
  con
  \[
    \Scalar = (\Norm{\VarVarVector}[2]^2
      + \AbsoluteValue{\VarVarVector_0} \Norm{\VarVarVector}[2])^{-1}
  \]
  e
  \[
    \Vector_k =
    \begin{cases}
      \left ( 1 + \frac{\Norm{\VarVarVector}[2]}{\AbsoluteValue{\VarVarVector_0}} \right )
        \VarVarVector_0&\text{ se }k = 0\text{ e }\VarVarVector_0 \neq 0,\\
      \Norm{\VarVarVector}[2]&\text{ se }k = 0\text{ e }\VarVarVector_0 = 0,\\
      \VarVarVector_k&\text{ se }k \neq 0.
    \end{cases}
  \]
  $\HouseholderElementaryMatrix$ \`e una matrice elementare di Householder tale che
  \begin{itemize}
    \item $\HouseholderElementaryMatrix\VarVarVector_0 \neq 0$;
    \item $\ForAll{k \in \NotZero{n}}{(\HouseholderElementaryMatrix\VarVarVector)_k = 0}$.
  \end{itemize}
  Inoltre $\HouseholderElementaryMatrix$ \`e calcolabile con
  $2n + 6 = \BigO{n}$ operazioni aritmetiche e estrazioni di radici, al caso pessimo.
\end{Theorem}
\Proof Dimostriamo che $\HouseholderElementaryMatrix$ \`e una matrice di Householder.
Abbiamo, se $\VarVarVector_0 \neq 0$
\begin{align*}
  \frac{2}{\HermitianTransposed{\Vector}\Vector}
  &= 2 \left ( \Norm{\VarVarVector}[2]^2
    + 2 \Norm{\VarVarVector}[2]\AbsoluteValue{\VarVarVector_0}
    + \Norm{\VarVarVector}[2]^2 \right )^{-1},\\
  &= \Scalar.
\end{align*}
\par Se invece $\VarVarVector_0 \neq 0$,
\begin{align*}
  \frac{2}{\HermitianTransposed{\Vector}\Vector}
  &= 2 \left ( \Norm{\VarVarVector}[2]^2
    + \Norm{\VarVarVector}[2]^2 \right )^{-1},\\
  &= \Scalar.
\end{align*}
\par Quindi $\HouseholderElementaryMatrix$ \`e una matrice elementare di Householder.
\par Sia ora $k \in n$. Abbiamo
\begin{align*}
  (\HouseholderElementaryMatrix\VarVarVector)_k 
  &= \sum_{j \in n} \HouseholderElementaryMatrix_k^j \VarVarVector_j,\\
  &= \VarVarVector_k - \Scalar \Vector_k \sum_{j \in n}
    \Conjugate{\Vector_j}
    \VarVarVector_j.
\end{align*}
\par Assumiamo in un primo momento $\VarVarVector_0 \neq 0$.
\par Supponiamo $k \neq 0$: abbiamo
\begin{align*}
  (\HouseholderElementaryMatrix\VarVarVector)_k
  &= \VarVarVector_k - \Scalar \VarVarVector_k \left (
    \Conjugate{
      \left ( 1 + \frac{\Norm{\VarVarVector}[2]}{\AbsoluteValue{\VarVarVector_0}} \right )
        \VarVarVector_0}
    \VarVarVector_0
    + \sum_{j = 1}^{n - 1}
    \Conjugate{\VarVarVector_j}
    \VarVarVector_j \right ),\\
  &= \VarVarVector_k - \Scalar \VarVarVector_k \left (
      \Norm{\VarVarVector}[2] \AbsoluteValue{\VarVarVector_0}
    + \Norm{\VarVarVector}[2]^2 \right ),\\
  &= 0.
\end{align*}
\par Se invece $k = 0$,
\begin{align*}
  (\HouseholderElementaryMatrix\VarVarVector)_0
  &= \VarVarVector_0 - \Scalar
    \left ( 1 + \frac{\Norm{\VarVarVector}[2]}{\AbsoluteValue{\VarVarVector_0}} \right )
    \VarVarVector_0 \left (
    \Conjugate{
      \left ( 1 + \frac{\Norm{\VarVarVector}[2]}{\AbsoluteValue{\VarVarVector_0}} \right )
        \VarVarVector_0}
    \VarVarVector_0
    + \sum_{j = 1}^{n - 1}
    \Conjugate{\VarVarVector_j}
    \VarVarVector_j \right ),\\
  &= \VarVarVector_0 - \Scalar
    \left( 1 + \frac{\Norm{\VarVarVector}[2]}{\AbsoluteValue{\VarVarVector_0}} \right )
      \VarVarVector_0 \left (
      \Norm{\VarVarVector}[2] \AbsoluteValue{\VarVarVector_0}
    + \Norm{\VarVarVector}[2]^2 \right ),\\
  &= - \frac{\Norm{\VarVarVector}[2]}{\AbsoluteValue{\VarVarVector_0}} \VarVarVector_0.
\end{align*}
\par Assumiamo ora invece $\VarVarVector_0 = 0$.
\par Supponiamo $k \neq 0$, abbiamo
\begin{align*}
  (\HouseholderElementaryMatrix\VarVarVector)_k
  &= \VarVarVector_k - \Scalar \VarVarVector_k
    \sum_{j \in n}
    \Conjugate{\VarVarVector_j}
    \VarVarVector_j,\\
  &= \VarVarVector_k - \Scalar \VarVarVector_k
    \Norm{\VarVarVector}[2]^2,\\
  &= 0.
\end{align*}
\par Se invece $k = 0$,
\begin{align*}
  (\HouseholderElementaryMatrix\VarVarVector)_0
  &= - \Scalar \Norm{\VarVarVector}[2]
    \sum_{j \in n}
    \Conjugate{\VarVarVector_j}
    \VarVarVector_j,\\
  &= - \Scalar \Norm{\VarVarVector}[2]^3,\\
  &= - \Norm{\VarVarVector}[2].
\end{align*}
\par Possiamo calcolare $\HouseholderElementaryMatrix$ con le seguenti operazioni:
\begin{itemize}
  \item calcolo di $\Norm{\VarVarVector}[2]^2$, che costa $2n - 1$
  \item calcolo di $\Norm{\VarVarVector}[2]$, ottenibile dal dato precedente con un'estrazione
    di radice quadrata;
  \item calcolo di $\Scalar$ a partire dai dati precedenti effettuando una moltiplicazione,
    una somma e un'inversione;
  \item calcolo di $\Vector$ che, al caso pessimo ($\VarVector \neq 0$), \`e calcolabile
    utilizzando i dati precedenti tramite una divisione, una somma e una moltiplicazione.
\end{itemize}
Abbiamo in totale, al caso pessimo,
$2n - 1 + 1 + 3 + 3 = 2n + 6 = \BigO{n}$. \EndProof
\begin{Definition}
  Sia $\Matrix \in \mathbb{C}^{n \times n}$ ($n \in \mathbb{N}$). Chiamiamo
  \Define{fattorizzazione $QR$}[$QR$][fattorizzazione]
  o
  \Define{decomposizione $QR$}[$QR$][decomposizione]
  di $\Matrix$ una coppia di matrici
  $(Q,R) \in \mathbb{C}^{n \times n} \times \mathbb{C}^{n \times n}$
  tali che
  \begin{itemize}
    \item $\Matrix = QR$;
    \item $Q$ \`e unitaria;
    \item $R$ \`e triangolare superiore.
  \end{itemize}
\end{Definition}
\begin{Definition}
  Con le notazioni del teorema \ref{th_FattorizzazioneMatriciElementari},
  l'algoritmo ottenuto dall'algoritmo del teorema \ref{th_FattorizzazioneMatriciElementari}
  scegliendo come matrici $(\ElementaryMatrix_{(l,0)})_{l \in m}$ matrici elementari di Householder
  si chiama
  \Define{metodo di Householder}[di Householder][metodo].
\end{Definition}
\begin{listing}
	\insertcode{octave}{"Metodi_numerici_per_sistemi_lineari/FattorizzazioneQR.m"}
	\caption{Implementazione del metodo di Householder in \LanguageName{octave}.}
\end{listing}
\begin{Theorem}
  Con le notazioni della definizione precedente, il metodo di Householder restituisce
  una fattorizzazione $QR$ di $\Matrix$. Il costo del metodo \`e $\BigO{n^3}$.
\end{Theorem}
\Proof $Q$ \`e prodotto di matrici diagonali a blocchi del tipo
$\lmatrix
\begin{array}{c|c}
  \Identity &\\
  \hline
  &\HouseholderElementaryMatrix
\end{array}
\rmatrix$, dove l'identit\`a $\Identity$ pu\`o anche mancare (pi\`u precisamente manca per la
prima matrice elementare calcolata dal metodo) e $\HouseholderElementaryMatrix$ \`e una matrice
elementare di Householder. $Q$ \`e quindi prodotto di matrici unitarie, dunque unitaria.
$R$ \`e triangolare superiore per il teorema \ref{th_FattorizzazioneMatriciElementari}.
\par Il metodo prevede il calcolo di $n$ matrici a blocchi $B$ del tipo sopra specificato,
col blocco di Householder di dimensione decrescente $q \in \NotZero{(n + 1)}$,
ciascuna di esse moltiplicate per
una matrice $M \in \mathbb{C}^{n \times n}$. Tali prodotti $P \in \mathbb{C}^{n \times n}$ sono
calcolabili nel modo seguente:
\[
    P_k^j =
    \begin{cases}
      M_k^j&\text{ se }k < n - q,\\
      0&\text{ se }\And{k > n - q}{j = n - q},\\
      M_k^j - \Scalar \Vector_k \sum_{r = k}^n \Conjugate{\Vector_r} \Matrix_r^j&\text{ se }
        \And{k > n - q}{j > n - q},\\
      - \Norm{\VarVarVector}[2]&\text{ se }\And{k = j = n - q}{M_k^j = 0},\\
      - \frac{\Norm{\VarVarVector}[2]}{\AbsoluteValue{M_k^j}}M_k^j&\text{ se }\And{k = j = n - q}{M_k^j \neq 0},
    \end{cases}
\]
dove $\Scalar$ e $\Vector$ sono rispettivamente lo scalare e il vettore che intervengono
nella costruzione della matrice di Householder, mentre
$\VarVector = (M_k^{n - q})_{k = n - q}^n$.
\begin{Corollary}
  Qualsiasi $\Matrix \in \mathbb{C}^{n \times n}$ ($n \in \NotZero{\mathbb{N}}$) ammette una
  fattorizzazione $QR$.
\end{Corollary}
\Proof Per il teorema \ref{th_Householder}, per ogni $l \in m$ \`e sempre possibile scegliere
$\ElementaryMatrix_{(l,0)}$ di Householder al passo \ref{FattorizzazioneMatriciElementari_3}
dell'algoritmo del teorema \ref{th_FattorizzazioneMatriciElementari}. \EndProof
\begin{Theorem}
  \label{th_OrtonormalizzazioneQR}
  Siano
  \begin{itemize}
    \item $n, m \in \NotZero{\mathbb{N}}$;
    \item $\Matrix \in \mathbb{C}^{n \times m}$;
    \item $Q \in \mathbb{C}^{n \times n}$ e $R \in \mathbb{C}^{n \times m}$
      tali che $\Matrix = QR$ sia una fattorizzazione $QR$;
    \item $p$ tale che $R_p$ sia una riga nulla;
    \item $Q' = (Q_k^j)_{\substack{k \in n\\j \in p}}$;
    \item $R' = (R_k^j)_{\substack{k \in p\\j \in m}}$.
  \end{itemize}
  Abbiamo,
  \begin{itemize}
    \item $\ForAll{(k,j) \in p \times m}{R_k^j = 0}$;
    \item $\Matrix = Q'R'$.
  \end{itemize}
  Inoltre, se $p = m$,
  \begin{itemize}
    \item $R'$ \`e invertibile;
    \item $\Image{\Matrix} = \Image{Q'}$.
  \end{itemize}
\end{Theorem}
\Proof La prima relazione segue direttamente dalla definizione del metodo
di Householder. Per la seconda relazione, abbiamo, per ogni
$(k,j) \in n \times m$,
\begin{align*}
  \Matrix_k^j
  &= \sum_{l \in n} Q_k^l R_l^j,\\
  &= \sum_{l \in p} Q_k^l R_l^j,\\
  &= \sum_{l \in p} {Q'}_k^l {R'}_l^j.
\end{align*}
\par Inoltre, se $p = m$,
$R$ \`e invertibile perch\'e, essendo triangolare, i suoi
autovalori sono gli elementi diagonali, nessundo dei quali \`e nullo
per definizione del metodo di Householder.
\par Ora, sia $Y \in \Image{\Matrix}$ e sia $X \in \mathbb{C}^n$ tale che
$Y = \Matrix X$. Abbiamo
\begin{align*}
  Q' \cdot R'X
  &= Q'R' \cdot X,\\
  &= \Matrix X,\\
  &= Y.
\end{align*}
\par Se invece $Y \in \Image{\Matrix}$ e $X \in \mathbb{C}^n$ \`e tale che
$Y = Q'X$, abbiamo
\begin{align*}
  \Matrix \cdot R'^{-1}X
  &= Q'R' \cdot R'^{-1}X,\\
  &= QX,\\
  &= Y.\text{ \EndProof}
\end{align*}
\begin{listing}
	\insertcode{octave}{"Metodi_numerici_per_sistemi_lineari/OrtonormalizzaColonne.m"}
	\caption{Implementazione di un algoritmo in \LanguageName{octave} per ortonormalizzare
  le colonne di una matrice.}
\end{listing}
\begin{Definition}
  Con le notazioni del teorema precedente,
  chiamiamo ancora
  \Define{fattorizzazione $QR$}[$QR$][fattorizzazione]
  o
  \Define{decomposizione $QR$}[$QR$][decomposizione]
  di $\Matrix$ la coppia di matrici $(Q',R')$.
\end{Definition}
\begin{Theorem}
  \label{th_FattorizzazioneQRparziale}
  Siano
  \begin{itemize}
    \item $n, m \in \NotZero{\mathbb{N}}$;
    \item $p \in \NotZero{n}$;
    \item $\Matrix \in \mathbb{C}^{n \times m}$;
    \item $Q \in \mathbb{C}^{n \times n}$ e $R \in \mathbb{C}^{n \times m}$
      tali che $\Matrix = QR$ sia una fattorizzazione $QR$;
    \item $\Matrix' = (\Matrix_k^j)_{\substack{k \in n\\j \in p}}$;
    \item $Q' = (Q_k^j)_{\substack{k \in n\\j \in p}}$;
    \item $R' = (R_k^j)_{\substack{k \in p\\j \in p}}$.
  \end{itemize}
  Abbiamo, $\Matrix' = Q'R'$.
\end{Theorem}
\Proof Abbiamo, per ogni
$(k,j) \in n \times p$,
\begin{align*}
  \Matrix_k^j
  &= \sum_{l \in n} Q_k^l R_l^j,\\
  &= \sum_{l \in p} Q_k^l R_l^j,\\
  &= \sum_{l \in p} {Q'}_k^l {R'}_l^j.\text{ \EndProof}
\end{align*}
\begin{Definition}
  Sia $\Ring$ un anello e sia
  $\PhaseMatrix \in \Ring^{n \times n}$ ($n \in \NotZero{\mathbb{N}}$)
  diagonale e unitaria: $\PhaseMatrix$ si dice
  \Define{matrice di fase}[di fase][matrice].
\end{Definition}
\begin{Theorem}
  Sia $\TriangularMatrix \in \mathbb{C}^{n \times n}$ ($n \in \NotZero{\mathbb{N}}$) una 
  matrice triangolare superiore unitaria. $\TriangularMatrix$ \`e una matrice di fase.
\end{Theorem}
\Proof Procediamo per induzione su $n$. Se $n = 1$ la dimostrazione \`e
immediata.
\par Supponiamo il teorema provato per $n = N$ e proviamolo per $n = N + 1$.
Abbiamo
\begin{align*}
  \HermitianTransposed{\TriangularMatrix^0} \TriangularMatrix^0
  &= \sum_{l \in n} \Conjugate{\TriangularMatrix_l^0} \TriangularMatrix_l^0,\\
  &= \Conjugate{\TriangularMatrix_0^0} \TriangularMatrix_0^0,\\
  &= 1.
\end{align*}
Dunque $\TriangularMatrix_0^0 \neq 0$.
\par Per ogni $j \in \NotZero{n}$, abbiamo
\begin{align*}
  \HermitianTransposed{\TriangularMatrix^0} \TriangularMatrix^j
  &= \sum_{l \in n} \Conjugate{\TriangularMatrix_l^0} \TriangularMatrix_l^j,\\
  &= \Conjugate{\TriangularMatrix_0^0} \TriangularMatrix_0^j,\\
  &= 0.
\end{align*}
Dunque necessariamente $\TriangularMatrix_0^j = 0$.
\par D'altra parte, la sottomatrice
$(\TriangularMatrix_k^j)_{(k,j) \in (\NotZero{n}) \times (\NotZero{n})}$
\`e ancora unitaria e triangolare superiore: per ipotesi induttiva \`e diagonale.
\par Dunque \`e diagonale $\TriangularMatrix$. \EndProof
\begin{Theorem}
  Siano
  \begin{itemize}
    \item $n \in \NotZero{\mathbb{N}}$;
    \item $\Matrix \in \mathbb{C}^{n \times n}$ invertibile;
    \item $Q, R \in \mathbb{C}^{n \times n}$ una fattorizzazione $QR$ di $\Matrix = QR$;
    \item $Q', R' \in \mathbb{C}^{n \times n}$ una fattorizzazione $QR$ di $\Matrix = Q'R'$.
  \end{itemize}
  Esiste una matrice di fase $\DiagonalMatrix$ tale che
  $Q' = Q\PhaseMatrix$ e
  $R' = \PhaseMatrix^{-1} R$.
\end{Theorem}
\Proof Poniamo $\PhaseMatrix = Q^{-1}Q'$: $\PhaseMatrix$ \`e unitaria.
\par Inoltre, da $QR = Q'R'$, deduciamo $\PhaseMatrix = RR'^{-1}$, dunque
$\PhaseMatrix$ \`e anche triangolare superiore.
\par Quindi $\PhaseMatrix$ \`e unitaria e diagonale. Segue infine da
verifica diretta che
$Q' = Q\PhaseMatrix$ e
$R' = \PhaseMatrix^{-1} R$. \EndProof

\section{Forma di Hessenberg.}
\label{MetodiNumericiPerSistemiLineari_FormaDiHessenberg}
\begin{Definition}
	Siano $\Ring$ un anello, $n, m \in \mathbb{N}$ e $\Matrix \in \Ring^{n \times m}$.
  Diciamo che $\Matrix$ \`e
	\begin{itemize}
		\item \Define{matrice di Hessenberg superiore}[di Hessenberg superiore][matrice] quando
      $\ForAll{(i,j) \in n \times m}{\Implies{i > j + 1}{\Matrix_i^j = 0}}$:
      nel caso quadrato abbiamo
\[
\lmatrix
\begin{array}{cccc}
\Matrix_0^0 & \cdots & \cdots & \Matrix_{n - 1}^{n - 1}\\
\Matrix_1^0 & \Matrix_1^1 & \ddots & \Matrix_{n - 1}^{n - 1}\\
& \ddots & \ddots & \vdots\\
& & \Matrix_{n - 1}^{n - 2} & \Matrix_{n - 1}^{n - 1}
\end{array}
\rmatrix;
\]
		\item \Define{matrice di Hessenberg inferiore}[di Hessenberg inferiore][matrice] quando
      $\ForAll{(i,j) \in n \times m}{\Implies{i < j - 1}{\Matrix_i^j = 0}}$:
      nel caso quadrato abbiamo
\[
\lmatrix
\begin{array}{cccc}
\Matrix_0^0 & \Matrix_0^1 &\\
\vdots & \ddots & \ddots &\\
\vdots & \ddots & \Matrix_{n - 2}^{n - 2} & \Matrix_{n - 2}^{n - 1}\\
\Matrix_{n - 1}^0 & \cdots & \Matrix_{n - 1}^{n - 2} & \Matrix_{n - 1}^{n - 1}\\
\end{array}
\rmatrix.
\]
  \end{itemize}
\end{Definition}
\begin{Definition}
  Siano
  \begin{itemize}
    \item $\Matrix \in \mathbb{C}^{n \times n}$;
    \item $\UnitaryMatrix \in \mathbb{C}^{n \times n}$ unitaria;
    \item $\HessenbergForm \in \mathbb{C}^{n \times n}$ matrice di Hessenberg
      superiore;
  \end{itemize}
  tali che $\HermitianTransposed{\UnitaryMatrix} \Matrix \UnitaryMatrix = \HessenbergForm$.
  Si dice che $\HessenbergForm$ \`e una
  \Define{forma di Hessenberg}[di Hessenberg][forma] di $\Matrix$.
\end{Definition}
\begin{Theorem}
  Siano
  \begin{itemize}
    \item $n, m \in \NotZero{\mathbb{N}}$;
    \item $\Matrix \in \mathbb{C}^{n \times m}$.
  \end{itemize}
  Il seguente algoritmo costruisce una fattorizzazione
  $\Matrix = A \HessenbergForm$ dove $A$ \`e prodotto di matrici
  elementari e $\HessenbergForm$ \`e triangolare superiore:
  \begin{enumerate}
    \item\label{FormaDiHessenberg_1} si ponga $l = 0$;
    \item\label{FormaDiHessenberg_2} si definisca
      $\VarMatrix = (\Matrix_k^j)_{\substack{k = l\\j= l}}^{\substack{k = n - 1}{j = m - 1}}$;
    \item\label{FormaDiHessenberg_3} si fissi una matrice
      elementare di Householder $\HouseholderElementaryMatrix_{(l,0)}$ tale che
      $\HouseholderElementaryMatrix_{(l,0)}\VarMatrix = \CanonicalBase_1^{(l + 1)}$, dove
      $\CanonicalBase_1^{(l + 1)}$ \`e il $(l + 1)$-esimo vettore della base
      canonica di $\mathbb{C}^{m - l - 1}$;
    \item\label{FormaDiHessenberg_4} si ponga
      $(\HouseholderElementaryMatrix_{(l)})_k^j =
      \begin{cases}
        1&\text{ se }k = j < l,\\
        (\HouseholderElementaryMatrix_{(l,0)})_{k - l}^{j - l}&\text{ se }\And{k \geq l}{j \geq l},\\
        0&\text{ altrimenti},
      \end{cases}$
      vale a dire
      \[
        \HouseholderElementaryMatrix_{(l)} =
        \lmatrix
        \begin{array}{c|c}
          \Identity &\\
          \hline
          & \HouseholderElementaryMatrix_{(l,0)}
        \end{array}
        \rmatrix;
      \]
    \item\label{FormaDiHessenberg_5} si ridefinisca
      $\Matrix = \HouseholderElementaryMatrix_{(l)}\Matrix\HouseholderElementaryMatrix_{(l)}$;
    \item\label{FormaDiHessenberg_6} si incrementi
      $l$ di $1$;
    \item\label{FormaDiHessenberg_7} se $l < m - 1$, si torni al
      punto \ref{FormaDiHessenberg_2};
    \item\label{FormaDiHessenberg_8} si ponga
      $A = \HouseholderElementaryMatrix_{(m - 2)}^{-1}\HouseholderElementaryMatrix_{(m - 3)}^{-1}
      \cdots
      \HouseholderElementaryMatrix_{(1)}^{-1}\HouseholderElementaryMatrix_{(0)}^{-1}$;
    \item\label{FormaDiHessenberg_9} si ponga
      $\HessenbergForm = \Matrix$;
    \item\label{FormaDiHessenberg_10} l'algoritmo termina restituendo
      $A$ e $\HessenbergForm$.
  \end{enumerate}
\end{Theorem}
\Proof 
\begin{listing}
	\insertcode{octave}{"Metodi_numerici_per_sistemi_lineari/HessenbergViaHouseholder.m"}
	\caption{Implementazione di un algoritmo in \LanguageName{octave} per il calcolo di una
  forma di Hessenberg tramite matrici elementari di Householder.}
\end{listing}
\begin{Definition}
  Chiamiamo
  \Define{rotazione di Givens}[di Givens][rotazione]
  una matrice unitaria
  $\GivensRotation \in \mathbb{C}^{n \times n}$
  ($n \in \NotZero{\mathbb{N}}$) tale che, per opportuni
  \begin{itemize}
    \item $p \in \NotZero{n}$;
    \item $c, s \in \mathbb{C}$;
  \end{itemize}
  \[
    \GivensRotation_k^j =
    \begin{cases}
      1&\text{ se }\And{k = j}{\And{k \neq p}{k \neq p + 1}},\\
      c&\text{ se }k = j = p,\\
      s&\text{ se }\And{k = p}{j = p + 1},\\
      - \Conjugate{s}&\text{ se }\And{k = p + 1}{j = p},\\
      \Conjugate{c}&\text{ se }k = j = p + 1,\\
      0&\text{ altrimenti},
    \end{cases}
  \]
  vale a dire
  \[
    \GivensRotation =
    \lmatrix
    \begin{array}{cccccccccc}
      1 &&&&&&&\\
      & \ddots &&&&&&\\
      && 1 &&&&&\\
      &&& c & s &&&\\
      &&& - \Conjugate{s} & \Conjugate{c} &&&\\
      &&&&& 1 &&\\
      &&&&&& \ddots &\\
      &&&&&&& 1
    \end{array}
    \rmatrix.
  \]
\end{Definition}
\begin{listing}
	\insertcode{octave}{"Metodi_numerici_per_sistemi_lineari/HessenbergViaGivens.m"}
	\caption{Implementazione di un algoritmo in \LanguageName{octave} per il calcolo di una
  forma di Hessenberg tramite rotazioni di Givens.}
\end{listing}
\begin{listing}
	\insertcode{octave}{"Metodi_numerici_per_sistemi_lineari/BulgeChasing.m"}
	\caption{Implementazione del \English{bulge chasing} in \LanguageName{octave}.}
\end{listing}

\section{Forma normale di Schur}
\label{MetodiNumericiPerSistemiLineari_FormaNormaleDiSchur}
\begin{Definition}
  Siano
  \begin{itemize}
    \item $\Matrix \in \mathbb{C}^{n \times n}$;
    \item $\UnitaryMatrix \in \mathbb{C}^{n \times n}$ unitaria;
    \item $\SchurNormalForm \in \mathbb{C}^{n \times n}$ triangolare
      superiore;
  \end{itemize}
  tali che $\HermitianTransposed{\UnitaryMatrix} \Matrix \UnitaryMatrix = \SchurNormalForm$.
  Si dice che $\SchurNormalForm$ \`e una
  \Define{forma normale di Schur}[normale di Schur][forma] di $\Matrix$.
\end{Definition}
\begin{Theorem}
  Sia $\Matrix \in \mathbb{C}^{n \times n}$. Il seguente algoritmo restituisce una forma normale
  di Schur $\SchurNormalForm$ di $\Matrix$ e una matrice unitaria $\UnitaryMatrix$ tale che
  $\HermitianTransposed{\UnitaryMatrix} \Matrix \UnitaryMatrix = \SchurNormalForm$:
  \begin{enumerate}
    \item\label{Schur_1} se $n = 1$ l'algoritmo termina restituendo $\SchurNormalForm = \Matrix$
      e $\UnitaryMatrix = \Identity$;
    \item\label{Schur_2} si fissino un autovalore $\Eigenvalue \in \Spectrum{\Matrix}$ e un
      autovettore $\Eigenvector$ di $\Matrix$ relativo ad $\Eigenvalue$;
    \item\label{Schur_3} sia $\HouseholderElementaryMatrix$ una matrice elementare di
      Householder tale che $\HouseholderElementaryMatrix \Eigenvector$
    \item\label{Schur_4} definiamo
      $\VarMatrix = \HermitianTransposed{\HouseholderElementaryMatrix} \Matrix
                    \HouseholderElementaryMatrix$;
    \item\label{Schur_5} applichiamo l'intero algoritmo alla sottomatrice
      $(\VarMatrix_k^j)_{\substack{k = 1, ..., n - 1}{j = 1, ..., n -1}}$ ottenendo
      la matrice triangolare superiore $\VarSchurNormalForm$ e la matrice unitaria
      $\VarUnitaryMatrix$;
    \item\label{Schur_6} definiamo
      \[
        \SchurNormalForm =
        \lmatrix
        \begin{array}{c|c}
          \Eigenvalue & (\VarMatrix_0^j)_{j = 1, ..., n - 1} \VarUnitaryMatrix\\
          \hline
            & \VarSchurNormalForm
        \end{array}
        \rmatrix
      \]
    \item\label{Schur_7} definiamo
      \[
        \UnitaryMatrix =
        \lmatrix
        \begin{array}{c|c}
          1 &\\
          \hline
            & \VarUnitaryMatrix
        \end{array}
        \rmatrix
      \]
    \item\label{Schur_8} l'algoritmo termina restituendo $\SchurNormalForm$ e
      $\VarUnitaryMatrix$.
  \end{enumerate}
\end{Theorem}
\begin{listing}
	\insertcode{octave}{"Metodi_numerici_per_sistemi_lineari/Schur.m"}
	\caption{Implementazione di un algoritmo in \LanguageName{octave} per il calcolo di una
  forma normale di Schur.}
\end{listing}

\section{Metodi iterativi.}
\label{MetodiNumericiPerSistemiLineari_MetodiIterativi}
\begin{Theorem}
  \label{th_MetodiIterativiPerSistemiLineari}
  Siano
  \begin{itemize}
    \item $n \in \NotZero{\mathbb{N}}$;
    \item $\Matrix \in \mathbb{C}^{n \times n}$;
    \item $C \in \mathbb{C}^n$;
    \item $A, B \in \mathbb{C}^{n \times n}$ tali che
      \begin{itemize}
        \item $A$ \`e invertibile;
        \item $\Matrix = A - B$;
      \end{itemize}
    \item $X_0 \in \mathbb{C}$.
  \end{itemize}
  Poniamo
  \begin{itemize}
    \item $\IterationMatrix = A^{-1}B$;
    \item $V = A^{-1} C$.
  \end{itemize}
  La successione $(X_k)_{k \in \mathbb{N}}$ definita, per $k > 0$, da
  $X_k = \IterationMatrix X_{k - 1} + V$, quando convergente,
  converge a $\tilde{X}$ soluzione di $\Matrix X = C$.
\end{Theorem}
\Proof Abbiamo
\begin{align*}
  X_k
  &= \IterationMatrix X_{k - 1} + V,\\
  &= A^{-1}B X_{k - 1} + A^{-1} C,
\end{align*}
equivalente a
\[
  A X_k = B X_{k - 1} + C,
\]
e dunque a
\[
  A X_k - B X_{k - 1} = C.
\]
\par Al limite abbiamo
\[
  A \tilde{X} - B \tilde{X} = C,
\]
equivalente a
\[
  \Matrix \tilde{X} = C.\text{ \EndProof}
\]
\begin{Definition}
  Con le notazioni del teorema precedente, definiamo
  \Define{metodo iterativo stazionario}[iterativo stazionario per sistemi lineari][metodo]
  un algoritmo di risoluzione del sistema $\Matrix X = C$ che consiste
  nel calcolare ricorsivamente un numero finito di  elementi della
  successione $(X_k)_{k \in \mathbb{N}}$ sino all'elemento di indice $N$ e
  restituisce $X_N$ come approssimazione di una soluzione.
  Inoltre, il metodo si dice
  \Define{convergente}[di un metodo iterativo stazionario][convergenza]
  a $\tilde{X}$
  se \`e convergente a $\tilde{X}$ la successione
  $(X_k)_{k \in \mathbb{N}}$.
  Infine, chiamiamo
  \Define{matrice di iterazione}[di iterazione][matrice]
  la matrice $\IterationMatrix$.
\end{Definition}
\begin{Theorem}
  Con le notazioni del teorema \ref{th_MetodiIterativiPersistemiLineari},
  fissata una norma di matrice $\Norm{\cdot}$ su
  $\mathbb{C}^{n \times n}$, $\Norm{\IterationMatrix} < 1$ se e solo
  se
  \begin{itemize}
    \item il sistema $\Matrix X = C$ ammette un'unica soluzione $\tilde{X}$;
    \item il metodo iterativo converge a $\tilde{X}$.
  \end{itemize}
\end{Theorem}
\subsection{Metodo di Richardson.}
\label{MetodiNumericiPerSistemiLineari_MetodoDiRichardson}
\begin{Definition}
  Con le notazioni del teorema \ref{th_MetodiIterativiPerSistemiLineari},
  fissato $\alpha \in \mathbb{C}$
  chiamiamo
  \Define{metodo di Richardson}[di Richardson][metodo]
  il metodo iterativo definito da
  \begin{itemize}
    \item $A = \alpha^{-1} \Identity$;
    \item $B = \alpha^{-1} \Identity - \Matrix$.
  \end{itemize}
\end{Definition}
\begin{Theorem}
  Con le notazioni della definizione precedente abbiamo, per ogni $k \in \mathbb{N}$,
  $X_{k + 1} = X_k - \alpha(\Matrix X_k - C)$.
\end{Theorem}
\Proof Abbiamo
\begin{align*}
  X_{k + 1}
  &= \IterationMatrix X_k + V,\\
  &= A^{-1} B X_k + A^{-1} C,\\
  &= \alpha \Identity (\alpha^{-1} \Identity - \Matrix) X_k
    + \alpha \Identity C,\\
  &= (\Identity - \alpha \Matrix) X_k + \alpha \Identity C,\\
  &= X_k - \alpha(\Matrix X_k - C).
\end{align*}
\begin{Theorem}
  Con le notazioni della definizione precendente, se $\Matrix$ \`e definita
  positiva il metodo di Richardson converge per $\alpha = \Norm{\Matrix}^{-1}$,
  per qualsiasi norma indotta $\Norm{\cdot}$.
\end{Theorem}
\begin{listing}
	\insertcode{octave}{"Metodi_numerici_per_sistemi_lineari/MetodoDiRichardson.m"}
	\caption{Implementazione del metodo di Richardson in linguaggio \LanguageName{octave}.}
\end{listing}

\subsection{Metodo di Jacobi.}
\label{MetodiNumericiPerSistemiLineari_MetodoDiJacobi}
\begin{Definition}
  Con le notazioni del teorema \ref{th_MetodiIterativiPerSistemiLineari},
  posto
  \begin{itemize}
    \item $\TriangularMatrix$ la matrice triangolare superiore tale che,
      per $(k,j) \in n \times n$,
      $\TriangularMatrix =
        \begin{cases}
          - \Matrix_k^j\text{ se }k < j,\\
          0\text{ altrimenti}.
        \end{cases}$;
    \item $\DiagonalMatrix$ la matrice triangolare superiore tale che,
      per $(k,j) \in n \times n$,
      $\DiagonalMatrix =
        \begin{cases}
          \Matrix_k^j\text{ se }k = j,\\
          0\text{ altrimenti}.
        \end{cases}$;
    \item $\VarTriangularMatrix$ la matrice triangolare inferiore tale che,
      per $(k,j) \in n \times n$,
      $\VarTriangularMatrix =
        \begin{cases}
          - \Matrix_k^j\text{ se }k > j,\\
          0\text{ altrimenti}.
        \end{cases}$;
  \end{itemize}
  chiamiamo
  \Define{metodo di Jacobi}[di Jacobi][metodo]
  il metodo iterativo definito da
  \begin{itemize}
    \item $A = \DiagonalMatrix$;
    \item $B = \TriangularMatrix + \VarTriangularMatrix$.
  \end{itemize}
\end{Definition}
\begin{listing}
	\insertcode{octave}{"Metodi_numerici_per_sistemi_lineari/MetodoDiJacobi.m"}
	\caption{Implementazione del metodo di Jacobi in linguaggio \LanguageName{octave}.}
\end{listing}
\begin{Theorem}
  Con le notazioni della definizione precedente,
  se \`e verificata una delle proposizioni seguente, allora
  il metodo di Jacobi \`e convergente:
  \begin{itemize}
    \item $\Matrix$ \`e fortemente dominante diagonale;
    \item $\HermitianTransposed{\Matrix}$ \`e fortemente dominante diagonale;
    \item $\Matrix$ \`e irriducibilmente dominante diagonale;
    \item $\HermitianTransposed{\Matrix}$ \`e irriducibilmente dominante diagonale.
  \end{itemize}
\end{Theorem}

\subsection{Metodo di Gauss-Seidel.}
\label{MetodiNumericiPerSistemiLineari_MetodoDiGaussSeidel}
\begin{Definition}
  Con le notazioni del teorema \ref{th_MetodiIterativiPerSistemiLineari},
  posto
  \begin{itemize}
    \item $\TriangularMatrix$ la matrice triangolare superiore tale che,
      per $(k,j) \in n \times n$,
      $\TriangularMatrix =
        \begin{cases}
          - \Matrix_k^j\text{ se }k < j,\\
          0\text{ altrimenti}.
        \end{cases}$;
    \item $\DiagonalMatrix$ la matrice triangolare superiore tale che,
      per $(k,j) \in n \times n$,
      $\DiagonalMatrix =
        \begin{cases}
          \Matrix_k^j\text{ se }k = j,\\
          0\text{ altrimenti}.
        \end{cases}$;
    \item $\VarTriangularMatrix$ la matrice triangolare inferiore tale che,
      per $(k,j) \in n \times n$,
      $\VarTriangularMatrix =
        \begin{cases}
          - \Matrix_k^j\text{ se }k > j,\\
          0\text{ altrimenti}.
        \end{cases}$;
  \end{itemize}
  chiamiamo
  \Define{metodo di Jacobi}[di Jacobi][metodo]
  il metodo iterativo definito da
  \begin{itemize}
    \item $A = \DiagonalMatrix - \TriangularMatrix$;
    \item $B = \VarTriangularMatrix$.
  \end{itemize}
\end{Definition}
\begin{listing}
	\insertcode{octave}{"Metodi_numerici_per_sistemi_lineari/MetodoDiGaussSeidel.m"}
	\caption{Implementazione del metodo di Gauss-Seidel in linguaggio \LanguageName{octave}.}
\end{listing}
\begin{Theorem}
  Con le notazioni della definizione precedente,
  se \`e verificata una delle proposizioni seguente, allora
  il metodo di Jacobi \`e convergente:
  \begin{itemize}
    \item $\Matrix$ \`e fortemente dominante diagonale;
    \item $\HermitianTransposed{\Matrix}$ \`e fortemente dominante diagonale;
    \item $\Matrix$ \`e irriducibilmente dominante diagonale;
    \item $\HermitianTransposed{\Matrix}$ \`e irriducibilmente dominante diagonale.
  \end{itemize}
\end{Theorem}



	\chapter{Metodi numerici per problemi agli autovalori}
	\section{Teoremi di Gershgorin.}
\label{MetodiNumericiPerProblemiAgliAutovalori_Gershgorin}
\begin{Definition}
	Data una matrice
  $\Matrix \in \mathbb{C}{n \times n}$,
  con
  $n \in \NotZero{\mathbb{N}}$, definiamo
  \Define{$k$-esimo cerchio di Gershgorin}[di Gershgorin][cerchio],
  con
  $k \in n$, il cerchio
  \[
    \GershgorinDisk{k}[\Matrix]
    = \lbrace z \in \mathbb{C} |
      \AbsoluteValue{z - \Matrix_k^k} \leq
      \sum_{j = 0,\\j \neq i}^{n - 1} \AbsoluteValue{\Matrix_k^j}.
  \]
\end{Definition}
\begin{Theorem}
	\TheoremName{Primo teorema di Gershgorin}[primo di Gershgorin][teorema]
  Data una matrice
  $\Matrix \in \mathbb{C}^{n \times n}$
  ($n \in \NotZero{\mathbb{N}}$),
  abbiamo
  $\Spectrum{\Matrix} \subseteq \bigcup_{k \in n} \GershgorinDisk{k}$.
\end{Theorem}
\Proof Fissato $\Eigenvalue \in \Spectrum{\Matrix}$, sia
$\Eigenvector$ un autovettore relativo a $\Eigenvalue$.
Sia $K \in n$ tale che $\ForAll{k \in n}{x_K \geq x_k}$.
\par Da
$\Matrix \Eigenvector = \Eigenvalue \Eigenvector$
deduciamo, per ogni $k \in n$,
$\sum_{j = 1}^n \Matrix_K^j \Eigenvector_j = \Eigenvalue \Eigenvector_K$,
da cui
$\sum_{j = 0,\\j \neq K}^{n - 1} \Matrix_K^j \Eigenvector_j
= \Eigenvalue \Eigenvector_K - \Matrix_K^K \Eigenvector_K$.
Poich\'e certamente $\Eigenvector_K \neq 0$, deduciamo ancora
$\sum_{j = 0,\\j \neq K}^{n - 1}
\Matrix_K^j \frac{\Eigenvector_j}{\Eigenvector_K}
= \Eigenvalue - \Matrix_K^K$.
La tesi segue dal fatto che il modulo del primo membro \`e maggiorato da
$\sum_{j = 0,\\j \neq K}^{n - 1} \AbsoluteValue{\Matrix_K^j}$
che \`e proprio il raggio del $K$-esimo cerchio di Gershgorin. \EndProof
\begin{Theorem}
	\TheoremName{Secondo teorema di Gershgorin}[secondo di Gershgorin][teorema]
  Data una matrice $\Matrix \in \mathbb{C}{n \times n}$, con
  $n \in \NotZero{\mathbb{N}}$,
  se esiste $K \in \mathbb{N}$ tale che
  \[
    \left ( \bigcup_{k = 0}^K \GershgorinDisk{k} \right ) \cap \left ( \bigcup_{k = K}^{n - 1} \GershgorinDisk{k} \right ) = \emptyset,
  \]
  allora
  \[
    \Cardinality{\Spectrum{\Matrix} \cap \left ( \bigcup_{k = 0}^K \GershgorinDisk{k} \right )} = K.
  \]
\end{Theorem}
\Proof Sia $\DiagonalMatrix$ la matrice diagonale tale che per ogni
$k \in n$ si abbia
$\DiagonalMatrix_k^k = \Matrix_k^k$.
Definiamo l'applicazione
$\phi: [0,1] \rightarrow \mathbb{C}{n \times n}$ tale che
$\phi(t) = (1 - t)\DiagonalMatrix + t\Matrix$ e osserviamo che,
poich\'e gli autovalori di una matrice dipendono con continuit\`a dai
coefficienti della matrice stessa, gli autovalori di $\phi(t)$ dipendono con
continuit\`a dal parametro $t$.
\par I primi $K$ cerchi di Gershgorin della matrice $\phi(0)$ contengono
effettivamente esattamente $K$ autovalori.
Per definizione di $\phi$, per ogni $k \in n$ e ogni $t \in [0,1]$, il
$k$-esimo cerchio di Gershgorin ha lo stesso centro e raggio pi\`u piccolo
del $k$-esimo centro di Gershgorin di $\Matrix$.
Pertanto, per ogni $t \in [0,1]$, i primi $K$ cerchi di Gershgorin di
$\phi(t)$ sono disgiunti dai restanti cerchi: dovendo gli autovalori di
$\phi(t)$ muoversi con continuit\`a al variare di $t$ rimanendo sempre
nell'unione dei cerchi di Gershgorin, essi non possono lasciare l'unione
dei primi $K$ cerchi per entrare nei cerchi restanti o viceversa.
Quindi il numero di autovalori nei primi $K$ cerchi resta costantemente $K$.
\EndProof
\begin{Theorem}
	\TheoremName{Terzo teorema di Gershgorin}[terzo di Gershgorin][teorema]
  Data una matrice $\Matrix \in \mathbb{C}{n \times n}$, se
	\begin{itemize}
		\item $\Matrix$ \`e irriducibile;
		\item $\Eigenvalue$ \`e un autovalore di $\Matrix$;
		\item $\Implies{\Eigenvalue \in \GershgorinDisk{k}}{\Boundary\GershgorinDisk{k}}$;
	\end{itemize}
	allora $\Eigenvalue \in \bigcap_{k = 1}^n \Boundary{\GershgorinDisk{k}}$.
\end{Theorem}
\Proof Riprendiamo la dimostrazione del primo teorema di Gershgorin.
Abbiamo visto che
\begin{align*}
  \sum_{\substack{j = 0\\j \neq K}}^{n - 1} \AbsoluteValue{\Matrix_K^j}
  &\geq \sum_{\substack{j = 0\\j \neq K}}^{n - 1} \AbsoluteValue{\Matrix_K^j}
    \frac{\AbsoluteValue{\Eigenvector_j}}{\AbsoluteValue{\Eigenvector_K}},\\
  &\geq \AbsoluteValue{\sum_{\substack{j = 0\\j \neq K}}^{n - 1} \Matrix_K^j
    \frac{\Eigenvector_j}{\Eigenvector_K}},\\
  &= \AbsoluteValue{\Eigenvalue - \Matrix_K^K}.
\end{align*}
Dunque $\Eigenvalue \in \GershgorinDisk{K}$.
Allora $\Eigenvalue \in \Boundary\GershgorinDisk{K}$ e abbiamo
$\sum_{j = 1,\\j \neq K}^{n - 1} \AbsoluteValue{\Matrix_K^j}
= \sum_{j = 1,\\j \neq K}^{n - 1} \AbsoluteValue{\Matrix_K^j}
\frac{\AbsoluteValue{\Eigenvector_j}}{\AbsoluteValue{\Eigenvector_K}}$.
Dunque, per ogni $j \in n - \SetMin \lbrace K \rbrace$, abbiamo
$\Implies{\Matrix_K^j \neq 0}{\Eigenvector_j = \Eigenvector_K}$.
\par Assumiamo $J \in n$ tale che $\Matrix_K^J = 0$. Poich\'e $\Matrix$ \`e
irriducibile, il grafo associato a $\Matrix$ \`e fortemente connesso, il
ch\'e implica che esiste una successione finita $(a_r)_{r \in m}$
($m \in \mathbb{N}$) di elementi di $n$ tale che
	\begin{itemize}
		\item $a_0 = J$;
		\item $a_m = K$;
		\item per ogni $r < m$, $\Matrix_{a_r}^{a_{r + 1}} \neq 0$.
	\end{itemize}
Ne deduciamo, per ogni $r < m$,
$\Eigenvector_{a_r} = \Eigenvector_{a_{r + 1}}$ e dunque
$\Eigenvector_J = \Eigenvector_K$. Dunque $\Eigenvector$ ha tutte le
componenti uguali e $\Eigenvalue$ appartiene al bordo di tutti i cerchi di
Gershgorin. \EndProof

\section{Perturbazioni di matrici.}
\label{CalcoloScientifico_PerturbazioniDiMatrici}
\begin{Theorem}
  Sia $n \in \NotZero{\mathbb{N}}$.
  Per ogni $k \in n$, l'applicazione
  $\Eigenvalue_k: \mathbb{C}^{n \times n} \rightarrow \mathbb{C}$
  che a $\Matrix \in \mathbb{C}^{n \times n}$ associa il $k$-esimo
  elemento della successione crescente degli $n$ autovalori di $\Matrix$
  (contati con molteplicit\`a algebrica) \`e continua.
\end{Theorem}
\begin{Theorem}
  Con le notazioni del teorema precedente, fissato $k \in n$,
  il numero di condizionamento $\ConditionNumber$ di $\Eigenvalue_k$ in
  $\Matrix \in \mathbb{N}$ verifica, al primo ordine, la relazione
  \[
    \ConditionNumber \leq
    \frac{\Norm{\VarEigenvector}\Norm{\Eigenvector}}{
      \AbsoluteValue{\HermitianTransposed{\VarEigenvector}\Eigenvector}},
  \]
  dove $\Eigenvector$ e $\VarEigenvector$ sono autovettori destro e sinistro
  rispettivamente relativi a $\Eigenvalue = \Eigenvalue_k(\Matrix)$.
\end{Theorem}
\Proof Abbiamo $\Matrix \Eigenvector = \Eigenvalue \Eigenvector$.
Supponendo $\Matrix$ perturbata da $\Perturbation{\Matrix}$, anche
$\Eigenvector$ e $\Eigenvalue$ risultano perturbati da
$\Perturbation{\Eigenvector}$ e
$\Perturbation{\Eigenvalue}$ rispettivamente. Abbiamo
\[
  (\Matrix + \Perturbation{\Matrix})(\Eigenvector + \Perturbation{\Eigenvector})
  = (\Eigenvalue + \Perturbation{\Eigenvalue})
    (\Eigenvector + \Perturbation{\Eigenvector}).
\]
Sviluppando e conservando solo i termini al primo ordine,
\[
  \Matrix\Eigenvector
  + \Matrix \Perturbation{\Eigenvector}
  + \Perturbation{\Matrix}\Eigenvector
  = \Eigenvalue\Eigenvector
  + \Eigenvalue\Perturbation{\Eigenvector}
  + \Perturbation{\Eigenvalue}\Eigenvector,
\]
equivalente a
\[
  \Matrix \Perturbation{\Eigenvector}
  + \Perturbation{\Matrix}\Eigenvector
  = \Eigenvalue\Perturbation{\Eigenvector}
  + \Perturbation{\Eigenvalue}\Eigenvector.
\]
\par Moltiplichiamo a sinistra per $\HermitianTransposed{\VarEigenvector}$:
\[
  \HermitianTransposed{\VarEigenvector} \Matrix \Perturbation{\Eigenvector}
  + \HermitianTransposed{\VarEigenvector} \Perturbation{\Matrix}\Eigenvector
  = \HermitianTransposed{\VarEigenvector} \Eigenvalue\Perturbation{\Eigenvector}
  +\HermitianTransposed{\VarEigenvector} \Perturbation{\Eigenvalue}\Eigenvector.
\]
equivalente a 
\[
  \HermitianTransposed{\VarEigenvector} \Perturbation{\Matrix}\Eigenvector
  =\HermitianTransposed{\VarEigenvector} \Perturbation{\Eigenvalue}\Eigenvector.
\]
Utilizzando la disuguaglianza di Cauchy-Schwarz abbiamo
\begin{align*}
  \AbsoluteValue{\Perturbation{\Eigenvalue}}
    \AbsoluteValue{\HermitianTransposed{\VarEigenvector}\Vector}
  &= \AbsoluteValue{\HermitianTransposed{\VarEigenvector}
      \Perturbation{\Eigenvalue}\Eigenvector},\\
  &= \AbsoluteValue{\HermitianTransposed{\VarEigenvector}
      \Perturbation{\Matrix}\Eigenvector},\\
  &\leq \Norm{\VarEigenvector}
      \Norm{\Perturbation{\Matrix}\Eigenvector},\\
  &\leq \Norm{\VarEigenvector}
      \Norm{\Perturbation{\Matrix}}\Norm{\Eigenvector},
\end{align*}
equivalente a
$\frac{\AbsoluteValue{\Perturbation{\Eigenvalue}}}{
  \Norm{\Perturbation{\Matrix}}}
\leq \frac{\Norm{\VarEigenvector}\Norm{\Eigenvector}}{
  \AbsoluteValue{\HermitianTransposed{\VarEigenvector}{\Eigenvector}}}$,
da cui la tesi. \EndProof
\begin{Corollary}
  Con le notazioni del teorema precedente, se $\Matrix$ \`e normale, allora
$\ConditionNumber \leq 1$.
\end{Corollary}
\Proof Se $\Matrix$ \`e normale, allora autovettori sinistri e destri
coincidono. Scegliendo un autovettore $\Eigenvector$ relativo a $\Eigenvalue$.
$\ConditionNumber \leq
\frac{\Norm{\Eigenvector}\Norm{\Eigenvector}}{
  \AbsoluteValue{\HermitianTransposed{\Eigenvector}\Eigenvector}} = 1$.
\EndProof
\begin{Theorem}
  \TheoremName{Teorema di Hirsch}[di Hirsch][teorema]
  Siano
  \begin{itemize}
    \item $n \in \NotZero{\mathbb{N}}$;
    \item $\Norm{\cdot}$ una norma su $\mathbb{C}^n$;
    \item $\Matrix \in \mathbb{C}^{n \times n}$;
    \item $\Eigenvalue \in \Spectrum{\Matrix}$.
  \end{itemize}
  Abbiamo $\AbsoluteValue{\Eigenvalue} \leq \Norm{\Matrix}$.
\end{Theorem}
\Proof Sia $\Eigenvector$ autovettore relativo a $\Eigenvalue$.
$\Norm{\Matrix}
= \max_{\Vector \neq 0} \frac{\Matrix \Vector}{\Norm{\Vector}} 
\geq \frac{\Matrix\Eigenvector}{\Eigenvector}
\geq \frac{\Eigenvalue\Eigenvector}{\Eigenvector}
= \AbsoluteValue{\Eigenvalue}$. \EndProof
\begin{Definition}
  Fissato $n \in \NotZero{\mathbb{N}}$, una norma su $\mathbb{C}^n$ si dice
  \Define{assoluta}[assoluta][norma]
  quando
  \[
    \ForAll{(z_k)_{k \in n} \in \mathbb{C}}{
      \Norm{(z_k)_{k \in n}} = \Norm{(\AbsoluteValue{(z_k)})_{k \in n}}}.
  \]
\end{Definition}
\begin{Theorem}
  Tutte le $p$-norme sono assolute.
\end{Theorem}
\Proof Segue direttamente dalle definizioni. \EndProof
\begin{Lemma}
  Siano
  \begin{itemize}
    \item $n \in \NotZero{\mathbb{N}}$;
    \item $\DiagonalMatrix \in \mathbb{C}^{n \times n}$ diagonale;
    \item $\Norm{\cdot}[p]$ una $p$-norma su $\mathbb{C}^n$.
  \end{itemize}
  Abbiamo
  $\Norm{\DiagonalMatrix}[p]
  = \max_{k \in n} \AbsoluteValue{\DiagonalMatrix_k^k}$.
\end{Lemma}
\Proof Denotata con $(\CanonicalBase_k)_{k \in n}$ la base canonica
di $\mathbb{C}^n$, abbiamo, per ogni $k \in n$,
$\DiagonalMatrix \CanonicalBase_k =
\DiagonalMatrix_k^k \CanonicalBase_k$.
\par Abbiamo, per ogni $k \in n$,
\begin{align*}
  \Norm{\DiagonalMatrix}[p]
  &= \max_{\Vector \in \NotZero{\mathbb{C}}}
      \frac{\Norm{\DiagonalMatrix \Vector}[p]}{\Norm{\Vector}[p]},\\
  &\geq \frac{\Norm{\DiagonalMatrix \CanonicalBase_k}[p]}
          {\Norm{\CanonicalBase_k}[p]},\\
  &= \AbsoluteValue{\DiagonalMatrix_k^k}.
\end{align*}
Dunque
$\Norm{\DiagonalMatrix}[p]
\geq \max_{k \in n} \AbsoluteValue{\DiagonalMatrix_k^k}$.
\par D'altra parte, per un qualunque vettore unitario $\Vector$, abbiamo
\begin{align*}
  \Norm{\DiagonalMatrix \Vector}[p]
  &= \left ( \sum_{k \in n}
    \AbsoluteValue{\DiagonalMatrix_k^k \Vector_k}^p \right )^{\frac{1}{p}},\\
  &\leq \max_{k \in n} \DiagonalMatrix_k^k \left ( \sum_{k \in n}
    \AbsoluteValue{\Vector_k}^p \right )^{\frac{1}{p}},\\
  &= \max_{k \in n} \DiagonalMatrix_k^k \Norm{\Vector}[p],\\
  &= \max_{k \in n} \DiagonalMatrix_k^k.\text{ \EndProof}
\end{align*}
\begin{Theorem}
  \TheoremName{Teorema di Bauer-Fike}[di Bauer-Fike][teorema]
  Siano
  \begin{itemize}
    \item $n \in \NotZero{\mathbb{N}}$;
    \item $\Norm{\cdot}[p]$ una $p$-norma su $\mathbb{C}^{n \times n}$;
    \item $\Matrix \in \mathbb{C}^{n \times n}$ diagonalizzabile;
    \item $V, \DiagonalMatrix \in \mathbb{C}^{n \times n}$ tali che
      $V^{-1} \Matrix V = \DiagonalMatrix$;
    \item $\Perturbation{\Matrix} \in \mathbb{C}^{n \times n}$;
    \item $\VarEigenvalue \in \Spectrum(\Matrix + \Perturbation{\Matrix})$.
  \end{itemize}
  Abbiamo
  \[
    \min_{\Eigenvalue \in \Spectrum{\Matrix}}
    \AbsoluteValue{\VarEigenvalue - \Eigenvalue}
    \leq \VarMatrixConditionNumber \Norm{\Perturbation{\Matrix}}[p],
  \]
  dove $\VarMatrixConditionNumber$ \`e il numero di condizionamento della matrice
  $V$.
\end{Theorem}
\Proof Se $\VarEigenvalue \in \Spectrum{\Matrix}$, allora 
$\min_{\Eigenvalue \in \Spectrum{\Matrix}}
\AbsoluteValue{\VarEigenvalue - \Eigenvalue} = 0$ e la tesi \`e immediata.
Supponiamo dunque nel seguito, $\VarEigenvalue \notin \Spectrum{\Matrix}$.
\par Sia $\VarEigenvector$ autovettore di $\Matrix + \Perturbation{\Matrix}$
relativo a $\VarEigenvalue$:
\[
  (\Matrix + \Perturbation{\Matrix} - \VarEigenvalue\Identity) \VarEigenvector
  = 0.
\]
\par Effettuando un cambio di base tramite la matrice $V$,
\[
  (V^{-1}\Matrix V
  + V^{-1} \Perturbation{\Matrix} V
  - \VarEigenvalue V^{-1} \Identity V) V \VarEigenvector = 0,
\]
equivalente a
\[
  (\DiagonalMatrix
  + V^{-1} \Perturbation{\Matrix} V
  - \VarEigenvalue \Identity) V \VarEigenvector = 0.
\]
Poich\'e
$\VarEigenvalue \notin \Spectrum{\Matrix} = \Spectrum{\DiagonalMatrix}$,
$\DiagonalMatrix - \VarEigenvalue\Identity$ \`e invertibile. Moltiplichiamo a
sinistra l'equazione precedente per
$(\DiagonalMatrix - \VarEigenvalue\Identity)^{-1}$:
\[
  (\Identity
  + (\DiagonalMatrix - \VarEigenvalue\Identity)^{-1}
    V^{-1} \Perturbation{\Matrix} V) V \VarEigenvector = 0.
\]
Dunque
$-1 \in \Spectrum{
(\DiagonalMatrix - \VarEigenvalue\Identity)^{-1} V^{-1}
\Perturbation{\Matrix} V}$.
\par Per il teorema di Hirsch,
\begin{align*}
  1
  &\leq \Norm{\DiagonalMatrix - \VarEigenvalue\Identity)^{-1} V^{-1}
  \Perturbation{\Matrix} V}[p],\\
  &\leq \Norm{(\DiagonalMatrix - \VarEigenvalue\Identity)^{-1}}[p]
  \Norm{V^{-1}}[p] \Norm{\Perturbation{\Matrix}}[p] \Norm{V}[p],\\
  &\leq \Norm{(\DiagonalMatrix - \VarEigenvalue\Identity)^{-1}}[p]
  \VarMatrixConditionNumber \Norm{\Perturbation{\Matrix}}[p].
\end{align*}
\par D'altra parte, $(\DiagonalMatrix - \VarEigenvalue \Identity)^{-1}$ \`e
diagonale, dunque, per il lemma precedente,
$\Norm{(\DiagonalMatrix - \VarEigenvalue\Identity)^{-1}}[p]
= \max_{\Eigenvalue \in \Spectrum{\Matrix}} \frac{1}{
  \AbsoluteValue{\VarEigenvalue - \Eigenvalue}}
= \frac{1}{\min_{\Eigenvalue \in \Spectrum{\Matrix}}
  \AbsoluteValue{\VarEigenvalue - \Eigenvalue}}$.
E dunque infine
\[
  \min_{\Eigenvalue \in \Spectrum{\Matrix}} 
  \AbsoluteValue{\VarEigenvalue - \Eigenvalue}
  \leq \VarMatrixConditionNumber \Norm{\Perturbation{\Matrix}}[p].\text{ \EndProof}
\]
\begin{Corollary}
  Con le notazioni del teorema precedente, se $\Matrix$ \`e normale, allora
  \[
    \min_{\Eigenvalue \in \Spectrum{\Matrix}}
    \AbsoluteValue{\VarEigenvalue - \Eigenvalue}
    \leq \Norm{\Perturbation{\Matrix}}[2].
  \]
\end{Corollary}
\Proof Segue direttamente dal teorema precedente e dal fatto che $\Matrix$ \`e
diagonalizzabile per mezzo di matrici unitarie. \EndProof
\begin{Theorem}
  L'errore all'indietro $\BackwardError$ rispetto alla norma $\Norm{\cdot}[2]$
  di un'applicazione che a una matrice
  $\Matrix \in \mathbb{C}^{n \times n}$ ($n \in \mathbb{N}$) associa una coppia
  $(\Eigenvalue,\Eigenvector) \in \mathbb{C} \times (\NotZero{\mathbb{C}^n})$
  approssimazione di una coppia
  $(\VarEigenvalue,\VarEigenvector) \in
    \mathbb{C} \times (\NotZero{\mathbb{C}^n})$
  tale che $\Matrix\VarEigenvector = \VarEigenvalue\VarEigenvector$
  \`e
  \[
    \BackwardError
    = \frac{\Norm{\Matrix\Eigenvector - \Eigenvalue\Eigenvector}[2]}
      {\Norm{\Eigenvector}[2]}
  \]
\end{Theorem}
\Proof Sia $\Perturbation{\Matrix} \in \mathbb{C}^{n \times n}$.
Supponiamo
\[
  (\Matrix + \Perturbation{\Matrix})\Eigenvector = \Eigenvalue\Eigenvector;
\]
ci\`o \`e equivalente a
\[
  \Perturbation{\Matrix}\Eigenvector
  = \Eigenvalue\Eigenvector - \Matrix\Eigenvector.
\]
Passando alle norme, abbiamo
\[
  \Norm{\Perturbation{\Matrix}\Eigenvector}[2]
  = \Norm{\Eigenvalue\Eigenvector - \Matrix\Eigenvector}[2],
\]
da cui
\begin{align*}
  \frac{\Norm{\Matrix\Eigenvector - \Eigenvalue\Eigenvector}[2]}
    {\Norm{\Eigenvector}[2]}
  &= \frac{\Norm{\Perturbation{\Matrix}\Eigenvector}[2]}
    {\Norm{\Eigenvector}[2]},
  \leq \Norm{\Perturbation{\Matrix}}[2].
\end{align*}
Dunque
$\BackwardError
\geq \frac{\Norm{\Matrix\Eigenvector - \Eigenvalue\Eigenvector}[2]}
{\Norm{\Eigenvector}[2]}$.
\par D'altra parte, poniamo
\begin{itemize}
  \item $\Vector = \Matrix\Eigenvector - \Eigenvalue\Eigenvector$;
  \item $\Perturbation{\Matrix}
    = - \frac{\Vector\HermitianTransposed{\Eigenvector}}
        {\Norm{\Eigenvector}[2]^2}$.
\end{itemize}
\par Abbiamo
\begin{align*}
  (\Matrix + \Perturbation{\Matrix})\Eigenvector
  &= \Matrix\Eigenvector
    - \frac{\Vector\HermitianTransposed{\Eigenvector}}
      {\Norm{\Eigenvector}[2]^2} \Eigenvector,\\
  &= \Matrix\Eigenvector - \Vector,\\
  &= \Matrix\Eigenvector - \Matrix\Eigenvector + \Eigenvalue\Eigenvector,\\
  &= \Eigenvalue\Eigenvector.
\end{align*}
Per il lemma
\ref{MetodiNumericiPerLaRisoluzioneDiSistemiLineari_LemmaRango1},
abbiamo inoltre
\begin{align*}
  \Norm{\Perturbation{\Matrix}}[2]
  &= \Norm{- \frac{\Vector}{\Norm{\Eigenvector}[2]^2}}[2] \Norm{\Eigenvector}[2],\\
  &= \frac{\Norm{\Vector}[2]}{\Norm{\Eigenvector}[2]^2} \Norm{\Eigenvector}[2],\\
  &= \frac{\Norm{\Vector}[2]}{\Norm{\Eigenvector}[2]},\\
  &= \frac{\Norm{\Matrix\Eigenvector - \Eigenvalue\Eigenvector}[2]}
    {\Norm{\Eigenvector}[2]}.\text{ \EndProof}
\end{align*}

\section{Metodo $QR$.}
\label{CalcoloScientifico_MetodoQR}
\begin{Definition}
  Fissato $n \in \NotZero{\mathbb{N}}$, siano
  \begin{itemize}
    \item $\Matrix \in \mathbb{C}^{n \times n}$;
    \item $\VarVector \in \mathbb{C}^n$.
  \end{itemize}
  Si fissi inoltre una condizione di arresto $\ArrestCondition$.
  Si chiama
  \Define{metodo delle potenze}[delle potenze][metodo]
  l'algoritmo seguente, avente $\Matrix$ e $\VarVector$
  come dati di ingresso:
  \begin{enumerate}
    \item\label{MetodoDellePotenze_1} si ponga $j = 0$,
      $\VarVector_0 = \VarVector$ e
      $\Eigenvalue_0^{(0)} =
        \HermitianTransposed{\VarVector_0}\Matrix\VarVector_{0}$;
    \item\label{MetodoDellePotenze_2} si ponga
      $\tilde{\VarVector}_{j + 1} = \Matrix \VarVector_j$;
    \item\label{MetodoDellePotenze_3} si ponga
      $\VarVector_{j + 1}
      = \frac{\tilde{\VarVector}_{j + 1}}{\Norm{\tilde{\VarVector}_{j + 1}}}$;
    \item\label{MetodoDellePotenze_4} si ponga $\Eigenvalue_0^{(j + 1)}
      = \HermitianTransposed{\VarVector_{j + 1}}\Matrix\VarVector_{j + 1}$;
    \item\label{MetodoDellePotenze_5} se
      $\ArrestCondition$ \`e falsa
      si incrementi $j$ di $1$ e si torni al punto
      \ref{MetodoDellePotenze_2}, altrimenti l'algoritmo termina
      restituendo la coppia $(\Eigenvalue_0^{(j + 1)},\VarVector_{j + 1})$.
  \end{enumerate}
\end{Definition}
\begin{listing}
	\insertcode{octave}{"Metodi_numerici_per_problemi_agli_autovalori/MetodoDellePotenze.m"}
	\caption{Implementazione del metodo delle potenze in \LanguageName{octave}.}
\end{listing}
\par Il passo \ref{MetodoDellePotenze_3} dell'algoritmo sopradescritto ha
lo scopo di evitare il traboccamento dei moduli dei vettori
$(\VarVector_j)_{j \in \mathbb{N}}$.
\begin{Theorem}
  Siano
  \begin{itemize}
    \item $n \in \mathbb{N}$;
    \item $\Matrix \in \mathbb{C}^{n \times n}$ diagonalizzabile;
    \item $(\Eigenvalue_k)_{k \in n}$ gli autovalori di $\Matrix$ ordinati in ordine
      decrescente di modulo e arbitrariamente a parit\`a di modulo;
    \item $(\SpectralBase_k)_{k \in n}$ una base spettrale di $\Matrix$ tale
      che $\ForAll{k \in n}{\Matrix \SpectralBase_k = \Eigenvalue_k \SpectralBase_k}$;
    \item $\VarVector \in \mathbb{C}^n$;
    \item $(\Scalar_k)_{k \in n} \in \mathbb{C}$ le coordinate di $\VarVector$
      rispetto alla base $(\SpectralBase_k)_{k \in n}$.
  \end{itemize}
  Supponiamo inoltre
  \begin{itemize}
    \item $\Eigenvalue_0$ autovalore semplice non nullo;
    \item $\VarVector$ non ortogonale a $\Eigenvector_0$.
  \end{itemize}
  Allora il metodo delle potenze applicato a $\Matrix$ e a $\VarVector$
  converge a $(\Eigenvalue_0,\SpectralBase_0)$.
  Abbiamo inoltre
  $\AbsoluteValue{\Eigenvalue_0^{(j + 1)} - \Eigenvalue_0}
  = \BigO{\AbsoluteValue{\frac{\Eigenvalue_1}{\Eigenvalue_0}}}$.
\end{Theorem}
\Proof Abbiamo, per ogni $j \in \mathbb{N}$
\begin{align*}
  \Matrix^j \VarVector
  &= \sum_{k \in n} \Scalar_k \Matrix^j \SpectralBase_k,\\
  &= \sum_{k \in n} \Scalar_k \Eigenvalue_k^j \SpectralBase_k,\\
  &= \Eigenvalue_0^j \left ( \Scalar_0 \SpectralBase_0
    +  \sum_{k \in n} \Scalar_k
                      \left ( \frac{\Eigenvalue_k}{\Eigenvalue_0} \right )^j
                      \SpectralBase_k \right ).
\end{align*}
\par Per $j \rightarrow \infty$ abbiamo allora
\[
  \Eigenvalue^{-j}\Matrix^j \VarVector \rightarrow \Scalar_0 \SpectralBase_0,
\]
e dunque
$\frac{\Matrix^j \VarVector}{\Norm{\Matrix^j \VarVector}}$ converge a un vettore
unitario $\VarSpectralBase_0$ multiplo del vettore, anch'esso unitario,
$\SpectralBase_0$.
\par Abbiamo inoltre, sempre per $j \rightarrow \infty$,
\[
  \Eigenvalue_0^{(j)}
  \rightarrow \HermitianTransposed{\VarSpectralBase_0}\Matrix\VarSpectralBase_0
  = \Eigenvalue_0.
\]
MANCA LA DIMOSTRAZIONE RIGUARDO ALLA VELOCITA DI CONVERGENZA \EndProof
\par Numericamente, la condizione di non ortogonalit\`a ha scarsa importanza
poich\'e gli algoritmi vengono generalmente implementati in aritmetica in
virgola mobile, che introduce errori che perturbano rapidamente un eventuale
sfortunana ortogonalit\`a.
\begin{Definition}
  Fissato $n \in \NotZero{\mathbb{N}}$, siano
  \begin{itemize}
    \item $\Matrix \in \mathbb{C}^{n \times n}$;
    \item $\VarVector \in \mathbb{C}^n$;
    \item $\VarEigenvalue \in \mathbb{C}$.
  \end{itemize}
  Si fissi inoltre una condizione di arresto $\ArrestCondition$.
  Si chiama
  \Define{iterazione inversa}[inversa][iterazione]
  o anche
  \Define{metodo di Willandt}[di Willandt][metodo]
  l'algoritmo seguente, avente $\Matrix$, $\VarVector$ e $\VarEigenvalue$
  come dati di ingresso:
  \begin{enumerate}
    \item\label{IterazioneInversa_1} si ponga $j = 0$,
      $\VarVector_0 = \VarVector$ e
      $\Eigenvalue_0^{(0)} =
        \HermitianTransposed{\VarVector_0}\Matrix\VarVector_{0}$;
    \item\label{IterazioneInversa_2} si ponga
      $\tilde{\VarVector}_{j + 1}
      = (\Matrix - \VarEigenvalue\Identity)^{-1} \VarVector_j$;
    \item\label{IterazioneInversa_3} si ponga
      $\VarVector_{j + 1}
      = \frac{\tilde{\VarVector}_{j + 1}}{\Norm{\VarVector_{j + 1}}}$;
    \item\label{IterazioneInversa_4} si ponga $\Eigenvalue_0^{(j + 1)}
      = \HermitianTransposed{\VarVector_{j + 1}}\Matrix\VarVector_{j + 1}$;
    \item\label{IterazioneInversa_5} se
      $\ArrestCondition$ \`e falsa
      si incrementi $j$ di $1$ e si torni al punto
      \ref{MetodoDellePotenze_2}, altrimenti l'algoritmo termina
      restituendo la coppia $(\Eigenvalue_0^{(j + 1)},\VarVector_{j + 1})$.
  \end{enumerate}
\end{Definition}
\begin{listing}
	\insertcode{octave}{"Metodi_numerici_per_problemi_agli_autovalori/MetodoDiIterazioneInversa.m"}
	\caption{Implementazione dell'iterazione inversa in \LanguageName{octave}.}
\end{listing}
\begin{Theorem}
  Siano
  \begin{itemize}
    \item $n \in \mathbb{N}$;
    \item $\Matrix \in \mathbb{C}^{n \times n}$ diagonalizzabile;
    \item $(\Eigenvalue_k)_{k \in n}$ gli autovalori di $\Matrix$ ordinati in ordine
      decrescente di modulo e arbitrariamente a parit\`a di modulo;
    \item $(\SpectralBase_k)_{k \in n}$ una base spettrale di $\Matrix$ tale
      che $\ForAll{k \in n}{\Matrix \SpectralBase_k = \Eigenvalue_k \SpectralBase_k}$;
    \item $\VarVector \in \mathbb{C}^n$;
    \item $\VarEigenvalue \in \mathbb{C}$;
    \item $(\Scalar_k)_{k \in n} \in \mathbb{C}$ le coordinate di $\VarVector$
      rispetto alla base $(\SpectralBase_k)_{k \in n}$.
  \end{itemize}
  Supponiamo inoltre
  \begin{itemize}
    \item $q \in n$ tale che
    \begin{itemize}
      \item $\AbsoluteValue{\Eigenvalue_q - \VarEigenvalue}
            = \min_{k \in n} \AbsoluteValue{\Eigenvalue_k - \VarEigenvalue}$;
      \item $\ForAll{k \in n \SetMin \lbrace q \rbrace}{
              \AbsoluteValue{\Eigenvalue_k - \VarEigenvalue} >
              \AbsoluteValue{\Eigenvalue_q - \VarEigenvalue}}$;
    \end{itemize}
    \item $\VarVector$ non ortogonale a $\Eigenvector_q$.
  \end{itemize}
  Allora l'iterazione inversa applicata a
  $\Matrix$, $\VarVector$, $\VarEigenvalue$
  converge a $(\Eigenvalue_q,\SpectralBase_q)$.
  Abbiamo inoltre
  $\AbsoluteValue{\Eigenvalue_q^{(j + 1)} - \Eigenvalue_q}
  = \BigO{\AbsoluteValue{\frac{\Eigenvalue_r}{\Eigenvalue_q}}}$,
  dove $\Eigenvalue_q$ \`e uno dei secondi autovalori pi\`u vicini
  a $\VarEigenvalue$.
\end{Theorem}
\Proof Osserviamo che, se $\SpectralBaseChange$ diagonalizza
$\Matrix$, allora $\SpectralBaseChange$ diagonalizza anche
$(\Matrix - \VarEigenvalue\Identity)^{-1}$:
\begin{align*}
  \SpectralBaseChange
  (\Matrix - \VarEigenvalue\Identity)^{-1}
  \SpectralBaseChange^{-1}
  &= (\SpectralBaseChange
  (\Matrix - \VarEigenvalue\Identity)
  \SpectralBaseChange^{-1})^{-1},\\
  = (\SpectralBaseChange\Matrix\SpectralBaseChange^{-1}
  - \VarEigenvalue\Identity)^{-1}.
\end{align*}
\par $\Eigenvalue$ \`e autovalore di $\Matrix$ se e solo se
$(\Eigenvalue - \VarEigenvalue)^{-1}$ \`e autovalore di
$(\Matrix - \VarEigenvalue\Identity)^{-1}$.
\par Applicare l'iterazione inversa a $\Matrix$ equivale, per quanto riguarda il
vettore restituito, ad applicare il metodo delle potenze a
$(\Matrix - \VarEigenvalue\Identity)^{-1}$: pi\`u precisamente, entrambi gli
algoritmi restituiscono lo stesso autovettore normale $\Eigenvector_q$, per
opportuno $q \in n$.
\par $\Eigenvector_q$ \`e l'autovettore corrispondente all'autovalore
$\Eigenvalue_q$ di $\Matrix$. Inoltre
$\AbsoluteValue{\Eigenvalue_q - \VarEigenvalue)}^{-1}
= \max_{k \in n} \AbsoluteValue{\Eigenvalue_k - \VarEigenvalue)}^{-1}$,
equivalente a
$\AbsoluteValue{\Eigenvalue_q - \VarEigenvalue)}
= \min_{k \in n} \AbsoluteValue{\Eigenvalue_k - \VarEigenvalue)}$.
\par Segue per verifica diretta che il passo \label{IterazioneInversa_4} calcola
effettivamente $\Eigenvalue_q$.
\par Abbiamo infine, riguardo alla velocit\`a di convergenza,
\begin{listing}
	\insertcode{octave}{"Metodi_numerici_per_problemi_agli_autovalori/MetodoDelQuozienteDiRayleigh.m"}
	\caption{Implementazione del quoziente di Rayleigh in \LanguageName{octave}.}
\end{listing}
\begin{figure}
	\includegraphics[width=0.8\textwidth]{Metodi_numerici_per_problemi_agli_autovalori/Immagine_ConfrontoInversaRayleigh.png}
	\centering
	\caption[Confronto tra le velocit\`a di convergenza del metodo di iterazione inversa e del
  metodo di Rayleigh per una matrice simmetrica $10 \times 10$.]
	{Confronto tra le velocit\`a di convergenza del metodo di iterazione inversa e del
  metodo di Rayleigh per la matrice simmetrica riportata nel codice
  \ref{Mtr_ConfrontoInversaRayleigh}.}
  \label{Imm_ConfrontoInversaRayleigh}
\end{figure}
\begin{listing}
	\inputminted[fontsize=\small]{octave}{Metodi_numerici_per_problemi_agli_autovalori/tests/Hermitiana_UnicoAutovaloreDiModuloMassimo}
	\caption[Matrice usata per effettuare il confronto del grafico di figura
  \ref{Imm_ConfrontoInversaRayleigh}.]
	{Matrice usata per effettuare il confronto del grafico di figura
  \ref{Imm_ConfrontoInversaRayleigh} (formato di memorizzazione di \LanguageName{octave}).}
  \label{Mtr_ConfrontoInversaRayleigh}
\end{listing}
\begin{Theorem}
  Siano
  \begin{itemize}
    \item $n \in \mathbb{N}$;
    \item $\Matrix \in \mathbb{C}^{n \times n}$ diagonalizzabile;
    \item $(\Eigenvalue_k)_{k \in n}$ gli autovalori di $\Matrix$ ordinati in ordine
      decrescente di modulo e arbitrariamente a parit\`a di modulo;
    \item $(\SpectralBase_k)_{k \in n}$ una base spettrale di $\Matrix$ tale
      che $\ForAll{k \in n}{\Matrix \SpectralBase_k = \Eigenvalue_k \SpectralBase_k}$;
    \item $\VarVector \in \mathbb{C}^n$;
    \item $(\Scalar_k)_{k \in n} \in \mathbb{C}$ le coordinate di $\VarVector$
      rispetto alla base $(\SpectralBase_k)_{k \in n}$;
    \item $\Analytical: \mathbb{C} \rightarrow \mathbb{C}$ analitica.
  \end{itemize}
  Supponiamo inoltre
  \begin{itemize}
    \item $q \in n$ tale che
    \begin{itemize}
      \item $\AbsoluteValue{\Analytical(\Eigenvalue_q)}
            = \min_{\Eigenvalue \in \Spectrum{\Matrix}}
                \AbsoluteValue{\Analytical(\Eigenvalue)}$;
      \item $\ForAll{\Eigenvalue \in \Spectrum{\Matrix} \SetMin
                                \lbrace \Eigenvalue_q \rbrace}{
              \AbsoluteValue{\Analytical(\Eigenvalue_k)} >
              \AbsoluteValue{\Analytical(\Eigenvalue_q)}}$;
    \end{itemize}
    \item $\VarVector$ non ortogonale a $\Eigenvector_q$.
  \end{itemize}
  Allora il metodo delle potenze applicato a
  $\Analytical{\Matrix}$ e $\VarVector$, modificato sostituendo i passi
  \ref{MetodoDellePotenze_4} e \ref{MetodoDellePotenze_5} con
  i passi analoghi dell'algoritmo applicato a $\Matrix$ e $\VarVector_0$,
  converge a $(\Eigenvalue_q,\SpectralBase_q)$.
  Abbiamo inoltre
  $\AbsoluteValue{\Eigenvalue_q^{(j + 1)} - \Eigenvalue_q}
  = \BigO{\AbsoluteValue{\frac{\Eigenvalue_r}{\Eigenvalue_q}}}$,
  dove $\Eigenvalue_q$ \`e uno dei secondi autovalori pi\`u vicini
  a $\VarEigenvalue$.
\end{Theorem}
\Proof Analoga a quella del teorema precedente. \EndProof
\begin{Theorem}
  Siano
  \begin{itemize}
    \item $n, m \in \NotZero{\mathbb{N}}$;
    \item $\Matrix \in \mathbb{C}^{n \times m}$ una matrice con colonne
      ortogonali.
  \end{itemize}
  Allora $\Matrix\HermitianTransposed{\Matrix}$ \`e una proiezione ortogonale.
\end{Theorem}
\Proof Abbiamo
\begin{align*}
  (\Matrix\HermitianTransposed{\Matrix})^2
  &= \Matrix\HermitianTransposed{\Matrix}\Matrix\HermitianTransposed{\Matrix},\\
  &= \Matrix\HermitianTransposed{\Matrix}.
\end{align*}
Dunque $\Matrix\HermitianTransposed{\Matrix}$ \`e una proiezione.
\par Sia $X \in \mathbb{C}^{n \times m}$. Abbiamo
\begin{align*}
  \HermitianTransposed(\Matrix\HermitianTransposed{\Matrix} X)
    (X - \Matrix\HermitianTransposed{\Matrix} X)
  &= \HermitianTransposed{X} \Matrix\HermitianTransposed{\Matrix} X
    - \HermitianTransposed{X} \Matrix\HermitianTransposed{\Matrix}
      \Matrix\HermitianTransposed{\Matrix} X,\\
  &= \HermitianTransposed{X} \Matrix\HermitianTransposed{\Matrix} X
    - \HermitianTransposed{X} \Matrix\HermitianTransposed{\Matrix}  X,\\
  &= 0.
\end{align*}
Dunque $\Matrix\HermitianTransposed{\Matrix}$ \`e una proiezione ortogonale. \EndProof
\begin{Definition}
  Fissato $n \in \NotZero{\mathbb{N}}$, siano
  \begin{itemize}
    \item $p \in \NotZero{n}$;
    \item $\Matrix \in \mathbb{C}^{n \times n}$;
    \item $\VarMatrix \in \mathbb{C}^{n \times p}$.
  \end{itemize}
  Si fissi inoltre una condizione di arresto $\ArrestCondition$.
  Si chiama
  \Define{metodo di iterazioni di sottospazi}[di iterazioni di sottospazi][metodo]
  l'algoritmo seguente, avente $\Matrix$, $\VarMatrix$
  come dati di ingresso:
  \begin{enumerate}
    \item\label{MetodoDiIterazioneDiSottospazi_1} si ponga
      \begin{itemize}
        \item $Q_0$ uguale alla matrice unitaria di una qualche
              decomposizione $QR$ di $\VarMatrix$;
        \item $\Lambda_0^{(0)} =
                \HermitianTransposed{Q_0}\Matrix Q_{0}$;
        \item $j = 1$;
      \end{itemize}
    \item\label{MetodoDiIterazioneDiSottospazi_2} si ponga
      $Q_j$ uguale alla matrice unitaria di una qualche
      decomposizione $QR$ di $\Matrix Q_{j - 1}$ corrispondente
      a una matrice triangolare superiore quadrata $R_j$;
    \item\label{MetodoDiIterazioneDiSottospazi_3} si costruisca
      $\Lambda^{(j)} \in \mathbb{C}^p$ tale che per ogni $k \in p$
      $\Lambda^{(j)}_k$ \`e il $k$-esimo autovalore di
      $\HermitianTransposed{Q_j}\Matrix Q_{j}$, dove
      gli autovalori sono ordinati in ordine decrescente di modulo e arbitrariamente
      a parit\`a di modulo;
    \item\label{MetodoDiIterazioneDiSottospazi_5} se
      $\ArrestCondition$ \`e falsa
      si incrementi $j$ di $1$ e si torni al punto
      \ref{MetodoDiIterazioneDiSottospazi_2}, altrimenti l'algoritmo termina
      restituendo $\Lambda^{(j)}$.
  \end{enumerate}
\end{Definition}
\begin{listing}
	\insertcode{octave}{"Metodi_numerici_per_problemi_agli_autovalori/MetodoDiIterazioneDiSottospazi.m"}
	\caption{Implementazione del metodo di iterazioni di sottospazi in
    \LanguageName{octave}.}
\end{listing}
\begin{Theorem}
  Siano
  \begin{itemize}
    \item $n \in \NotZero{\mathbb{N}}$;
    \item $p \in \NotZero{n}$;
    \item $\Matrix \in \mathbb{C}^{n \times n}$ diagonalizzabile;
    \item $(\Eigenvalue_k)_{k \in n}$ gli autovalori di $\Matrix$ ordinati in ordine
      decrescente di modulo e arbitrariamente a parit\`a di modulo;
    \item $\SpectralBaseChange \in \mathbb{C}^{n \times n}$ una matrice spettrale
      per $\Matrix$ i cui autovettori sono ordinati in modo tale che
      $\ForAll{j \in n}{\Matrix \SpectralBaseChange^j = \Eigenvalue_j \SpectralBaseChange^j}$;
    \item $\VarMatrix \in \mathbb{C}^{n \times p}$;
    \item $A \in \mathbb{C}^{p \times p}$ e $B \in \mathbb{C}^{{n - p} \times p}$
      tali che
      $\SpectralBaseChange^{-1} \VarMatrix =
        \lmatrix
        \begin{array}{c}
          A\\
          \hline
          B
        \end{array}
        \rmatrix$.
  \end{itemize}
  Supponiamo inoltre
  \begin{itemize}
    \item $A$ \`e invertibile;
    \item $\AbsoluteValue{\Eigenvalue_{p - 1}} > \AbsoluteValue{\Eigenvalue_p}$.
  \end{itemize}
  Allora
  \begin{itemize}
    \item $(\Image{Q_j})$ converge a $\Image{(\SpectralBaseChange^j)_{j \in p}}$:
      pi\`u precisamente abbiamo
      \[
        \ForAll{\Vector \in \Image{(\SpectralBaseChange^j)_{j \in p}}}{
        \Norm{\Vector - Q_j\HermitianTransposed{Q_j} \Vector}[2]
        = \BigO{\AbsoluteValue{\frac{\Eigenvalue_p}{\Eigenvalue_{p - 1}}}^j}};
      \]
    \item il metodo di iterazioni di sottospazi applicato a
      $\Matrix$ e a $\VarMatrix$
      converge al vettore
      $\Lambda \in \mathbb{C}^p$ tale che, per ogni $k \in p$,
      $\Lambda_k = \Eigenvalue_k$;
    \item $\Norm{\Lambda^{(j)} - \Lambda}
        = \BigO{\AbsoluteValue{\frac{\Eigenvalue_p}{\Eigenvalue_{p - 1}}}^j}$.
  \end{itemize}
\end{Theorem}
\Proof Fissiamo $j \in \mathbb{N}$:
abbiamo, per il teorema \ref{th_OrtonormalizzazioneQR},
$\Image{Q_{j + 1}} = \Image{\Matrix \VarMatrix_j}$.
Ne deduciamo per induzione, per ogni $j \in \mathbb{N}$,
$\Image{Q_j} = \Image{\Matrix^j \VarMatrix}$.
\par Fissiamo di nuovo $j \in \mathbb{N}$: abbiamo, posto
\begin{itemize}
  \item $\DiagonalMatrix = \SpectralBaseChange^{-1} \Matrix \SpectralBaseChange$;
  \item $\DiagonalMatrix_0 =
    ((\SpectralBaseChange^{-1} \Matrix \SpectralBaseChange)_k^j)_{
    \substack{k \in p\\j \in p}}$;
  \item $\DiagonalMatrix_1 =
    ((\SpectralBaseChange^{-1} \Matrix \SpectralBaseChange)_k^j)_{
    \substack{k \in n \SetMin p\\j \in p}}$;
\end{itemize}
\begin{align*}
  \Matrix^j \VarMatrix
  &= \SpectralBaseChange \DiagonalMatrix^j \SpectralBaseChange^{-1} \VarMatrix,\\
  &= \SpectralBaseChange \DiagonalMatrix^j
    \lmatrix
    \begin{array}{c}
      A\\
      \hline
      B
    \end{array}
    \rmatrix,\\
  &= \SpectralBaseChange
    \lmatrix
    \begin{array}{c}
      \DiagonalMatrix_0^j A\\
      \hline
      \DiagonalMatrix_1^j B
    \end{array}
    \rmatrix.
\end{align*}
\par Ora, $\SpectralBaseChange$ e $\DiagonalMatrix_0^j A$ sono invertibili,
dunque $\Rank{Q_j} = p$.
\par Sia ora $\Vector$ combinazione lineare dei primi $p$ autovettori:
per ogni $j \in \mathbb{N}$ abbiamo
\begin{align*}
  \Vector
  &\in \Image{
  \SpectralBaseChange
  \lmatrix
  \begin{array}{c}
    \Identity[p]\\
    \hline
    0
  \end{array}
  \rmatrix},\\
  &= \Image{
  \SpectralBaseChange
  \lmatrix
  \begin{array}{c}
    \Identity[p]\\
    \hline
    0
  \end{array}
  \rmatrix
  \Eigenvalue_{p - 1}^{-j} \DiagonalMatrix_0^j A},\\
  &= \Image{
  \SpectralBaseChange
  \lmatrix
  \begin{array}{c}
    \Eigenvalue_{p - 1}^{-j} \DiagonalMatrix_0^j A\\
    \hline
    0
  \end{array}
  \rmatrix}.
\end{align*}
\par Siano ora
\begin{itemize}
  \item $X \in \mathbb{C}^p$ tale che
  $\Vector = 
    \SpectralBaseChange
    \lmatrix
    \begin{array}{c}
      \Identity[p]\\
      \hline
        0
      \end{array}
    \rmatrix X$;
  \item per ogni $j \in \mathbb{N}$, $X_j \in \mathbb{C}^p$ tale che
  $\Vector = 
    \SpectralBaseChange
    \lmatrix
    \begin{array}{c}
      \Eigenvalue_{p - 1}^{-j} \DiagonalMatrix_0^j A\\
      \hline
        0
      \end{array}
    \rmatrix X_j$.
\end{itemize}
Per ogni $j \in \mathbb{N}$, $X_j$ e $X$ sono collegati dalla relazione
$\lmatrix
\begin{array}{c}
  \Identity[p]\\
  \hline
  0
\end{array}
\rmatrix X
=
\lmatrix
\begin{array}{c}
  \Eigenvalue_{p - 1}^{-j} \DiagonalMatrix_0^j A\\
  \hline
    0
  \end{array}
\rmatrix X_j$,
da cui
$X = \Eigenvalue_{p - 1}^{-j} \DiagonalMatrix_0^j A X_j$
e quindi
\[
  X_j = \Eigenvalue_{p - 1}^j A^{-1} \DiagonalMatrix_0^{-j}.
\]
Guardando le norme $2$, otteniamo
\begin{align*}
  \Norm{X_j}[2]
  &\leq \AbsoluteValue{\Eigenvalue_p}^j \Norm{A^{-1}} \AbsoluteValue{\Eigenvalue_0}^{-j},\\
  &= \AbsoluteValue{\frac{\Eigenvalue_{p - 1}}{\Eigenvalue_0}}^j \Norm{A^{-1}},\\
  &\leq \Norm{A^{-1}}.
\end{align*}
\par Ora,
\[
  \Vector = 
  \SpectralBaseChange
  \lmatrix
  \begin{array}{c}
    \Eigenvalue_{p - 1}^{-j} \DiagonalMatrix_0^j A\\
    \hline
    \Eigenvalue_{p - 1}^{-j} \DiagonalMatrix_1^j A
  \end{array}
  \rmatrix
  X_j
  - \SpectralBaseChange
  \lmatrix
  \begin{array}{c}
    0\\
    \hline
    \Eigenvalue_{p - 1}^{-j} \DiagonalMatrix_1^j A
  \end{array}
  \rmatrix
  X_j.
\]
Il primo termine appartiene a $\Image{Q_j}$, il secondo dunque appartiene a un qualche
complemento di $\Image{Q_j}$ in $\mathbb{C}^{n}$: poich\'e la proiezione ortogonale
si $\Image{Q_j}$ \`e la proiezione per cui il vettore che giace nel complementare ha
norma minima deve essere
\begin{align*}
  \Norm{\Vector - Q_j\HermitianTransposed{Q_j} \Vector}[2]
  &\leq \Norm{\SpectralBaseChange
    \lmatrix
    \begin{array}{c}
      0\\
      \hline
      \Eigenvalue_{p - 1}^{-j} \DiagonalMatrix_1^j A
    \end{array}
    \rmatrix
    X_j}[2],\\
  &\leq \Norm{\SpectralBaseChange}[2]
    \Norm{\lmatrix
    \begin{array}{c}
      0\\
      \hline
      \Eigenvalue_{p - 1}^{-j} \DiagonalMatrix_1^j A
    \end{array}
    \rmatrix}[2]
    \Norm{X_j}[2],\\
  &= \Norm{\SpectralBaseChange}[2]
    \Norm{\Eigenvalue_{p - 1}^{-j} \DiagonalMatrix_1^j A}[2]
    \Norm{X_j}[2],\\
  &\leq \Norm{\SpectralBaseChange}[2]
    \AbsoluteValue{\Eigenvalue_{p - 1}^{-j}} \AbsoluteValue{\Eigenvalue_p}^j
    \Norm{A}[2]
    \Norm{X_j}[2],\\
  &\leq \Norm{\SpectralBaseChange}[2]
    \AbsoluteValue{\frac{\Eigenvalue_p}{\Eigenvalue_{p - 1}}}^j
    \Norm{A}[2]
    \Norm{A^{-1}}[2].
\end{align*}
Dunque
$\Norm{\Vector - Q_j\HermitianTransposed{Q_j} \Vector}[2]
= \BigO{\AbsoluteValue{\frac{\Eigenvalue_p}{\Eigenvalue_{p - 1}}}^j}$.
\par Ora, poniamo
\begin{itemize}
  \item $k \in p$;
  \item $\Eigenvector \in \mathbb{C}^n$ autovettore di $\Matrix$ relativo a $\Eigenvalue_k$;
  \item $\VarEigenvector_j = \HermitianTransposed{Q_j} \Eigenvector$;
  \item $\Vector_j = \Eigenvector
          - Q_j\HermitianTransposed{Q_j} \Eigenvector$.
\end{itemize}
Abbiamo
\begin{align*}
  \Norm{
    \HermitianTransposed{Q_j} \Matrix Q_j \VarEigenvector_j
    - \Eigenvalue_k \VarEigenvector_j
    }[2]
  &= \Norm{
    \HermitianTransposed{Q_j} \Matrix Q_j
    \HermitianTransposed{Q_j} \Eigenvector
    - \Eigenvalue_k \HermitianTransposed{Q_j} \Eigenvector
    }[2],\\
  &= \Norm{
    \HermitianTransposed{Q_j} \Matrix (\Eigenvector + \Vector_j)
    - \Eigenvalue_k \HermitianTransposed{Q_j} \Eigenvector
    }[2],\\
  &= \Norm{
    \HermitianTransposed{Q_j} (\Matrix \Eigenvector
    - \Eigenvalue_k \Eigenvector)
    + \HermitianTransposed{Q_j} \Matrix \Vector_j
    }[2],\\
  &= \Norm{\HermitianTransposed{Q_j} \Matrix \Vector_j}[2],\\
  &\leq \Norm{Q_j}[2]
   \Norm{\Matrix}[2]
   \Norm{\Vector_j}[2].
\end{align*}
\par Ora $\ForAll{j \in \mathbb{N}}{\Norm{Q_j} = 1}$ e
$\Norm{\Vector_j}
= \BigO{\AbsoluteValue{\frac{\Eigenvalue_p}{\Eigenvalue_{p - 1}}}^j}$,
dunque
\[
  \Norm{
  \HermitianTransposed{Q_j} \Matrix Q_j \VarEigenvector_j
  - \Eigenvalue_k \VarEigenvector_j
  }[2]
  = \BigO{\AbsoluteValue{\frac{\Eigenvalue_p}{\Eigenvalue_{p - 1}}}^j}.
\]
\par Ne consegue che, fissato $j \in \mathbb{N}$ e posto
$Q_j = \HermitianTransposed{Q_j} \Matrix Q_j$,
l'errore all'indietro dell'applicazione che a
$Q_j$ associa la coppia autovettore-autovalore
$(\VarEigenvector_j,\Eigenvalue_k)$ approssimata \`e anch'esso
$\BigO{\AbsoluteValue{\frac{\Eigenvalue_p}{\Eigenvalue_{p - 1}}}^j}$ e dunque esiste
una perturbazione $\Perturbation{\Matrix_j}$ tale che, per opportuna costante $C \in \mathbb{C}$, abbiamo
\begin{itemize}
  \item $\Norm{\Perturbation{\Matrix_j}}
    \leq C \AbsoluteValue{\frac{\Eigenvalue_p}{\Eigenvalue_{p - 1}}}^j$;
  \item $(\VarEigenvector_j,\Eigenvalue_k)$ \`e una coppia autovettore-autovalore esatta
    per la matrice $\Matrix_j + \Perturbation{\Matrix_j}$.
\end{itemize}
\par Per il teorema di Bauer-Fike, la matrice $\Matrix_j$ ha un autovalore
$\VarEigenvalue_k$ tale che
\begin{align*}
  \AbsoluteValue{\VarEigenvalue_k - \Eigenvalue_k}
  &\leq \Norm{\Perturbation{\Matrix_j}},\\
  &\leq C \AbsoluteValue{\frac{\Eigenvalue_p}{\Eigenvalue_{p - 1}}}^j.
\end{align*}
\par Dunque accanto alla successione $(\Matrix_j)_{j \in \mathbb{N}}$, abbiamo
un'altra succesione $(\VarEigenvalue_k^{(j)})_{j \in \mathbb{N}}$ tale che
$\AbsoluteValue{\VarEigenvalue_k^{(j)} - \Eigenvalue_k}
= \BigO{\AbsoluteValue{\frac{\Eigenvalue_p}{\Eigenvalue_{p - 1}}}^j}$.
\par Ne segue la tesi. \EndProof
\begin{Corollary}
  Siano
  \begin{itemize}
    \item $n \in \NotZero{\mathbb{N}}$;
    \item $\Matrix \in \mathbb{C}^{n \times n}$ diagonalizzabile;
    \item $(\Eigenvalue_k)_{k \in n}$ gli autovalori di $\Matrix$ ordinati in ordine
      decrescente di modulo;
    \item $\SpectralBaseChange \in \mathbb{C}^{n \times n}$ una matrice spettrale
      per $\Matrix$ i cui autovettori sono ordinati in modo tale che
      $\ForAll{j \in n}{\Matrix \SpectralBaseChange^j = \Eigenvalue_j \SpectralBaseChange^j}$.
  \end{itemize}
  Supponiamo inoltre che tutti i minori di testa di
  $\SpectralBaseChange^{-1}$
  siano invertibili.
  Il metodo di iterazione di sottospazi converge a una matrice
  $\SchurNormalForm$ 
  a gradini tale che, per ogni $p \in \NotZero{n}$, se
  $\AbsoluteValue{\Eigenvalue_{p - 1}} \neq \AbsoluteValue{\Eigenvalue_p}$,
  allora per ogni $(k,j) \in n \times p \SetMin p \times p$,
  $\SchurNormalForm_k^j = 0$.
  In particolare, se tutti gli autovalori di $\Matrix$ hanno moduli distinti,
  $\SchurNormalForm$ \`e una forma normale di Schur di $\Matrix$.
\end{Corollary}
\begin{listing}
	\insertcode{octave}{"Metodi_numerici_per_problemi_agli_autovalori/MetodoDiIterazioneDiSottospaziPerFormaDiSchur.m"}
	\caption{Implementazione del metodo di iterazioni di sottospazi 
    per il calcolo di una forma normale di Schur in \LanguageName{octave}.}
\end{listing}
\Proof Applichiamo il metodo di iterazione di sottospazi a $\Matrix$ e
$\VarMatrix = \Identity$ (manteniamo le stesse notazioni del teorema
precedente): otteniamo una successione di matrici
$(Q_j)_{j \in \mathbb{N}}$
le cui immagini convergono al sottospazio invariante
$\Image{\SpectralBaseChange} = \mathbb{C}^n$.
\par Osserviamo che per definizione dell'algoritmo e
per il teorema \ref{th_FattorizzazioneQRParziale}, per ogni
$p \in n$ la successione
$(Q^{(p)}_j)_{j \in \mathbb{N}}$
generata dal metodo di iterazione di sottospazi a partire da $\Matrix$ e
$Q^{(p)} = (\Identity^J)_{J \in p}$, coincide con la successione
$((Q^J_j)_{J \in p})_{j \in \mathbb{N}}$ ottenuta da
$(Q_j)_{j \in \mathbb{N}}$ estraendo da ogni termine
la sottomatrice delle prime $p$ colonne.
\par Ne deduciamo che, per ogni $p \in n$ tale che
$\AbsoluteValue{\Eigenvalue_{p - 1}} \neq \AbsoluteValue{\Eigenvalue_p}$,
le immagini delle matrici
$(Q^{(p)}_j)_{j \in \mathbb{N}}$
convergono al sottospazio invariante
$\Image{(\SpectralBaseChange^J)_{J \in p}}$.
\par Ne segue la tesi. \EndProof
\begin{Lemma}
  Con le notazioni della definizione del metodo di iterazione di sottospazi,
  per ogni $j \in \mathbb{N}$ e posto
  \begin{itemize}
    \item $\VarMatrix_0 = \Identity$;
    \item $\Matrix_j = \HermitianTransposed{Q_j} \Matrix Q_j$'
  \end{itemize}
  abbiamo
  $\Matrix_{j + 1} = R_j Q_j$
  a meno di coniugio per matrici di fase.
\end{Lemma}
\Proof Per ogni $j \in \mathbb{N}$ abbiamo, ricordando che
$Q_j \HermitianTransposed{Q_j}$ \`e una proiezione ortogonale
\begin{align*}
  \Matrix_{j + 1}
  &= \HermitianTransposed{Q_{j + 1}} \Matrix Q_{j + 1},\\
\end{align*}
\begin{Theorem}
  Fissato $n \in \NotZero{\mathbb{N}}$, sia
  $\Matrix \in \mathbb{C}^{n \times n}$.
  Si fissi inoltre una condizione di arresto $\ArrestCondition$.
  L'algoritmo seguente, avente $\Matrix$
  come dato di ingresso, fornisce una forma normale di Schur di $\Matrix$:
  \begin{enumerate}
    \item\label{ProtometodoQR_1} si pongano $Q, R \in \mathbb{C}^{n \times n}$
      tali che $\Matrix = QR$ sia una fattorizzazione $QR$ di $\Matrix$;
    \item\label{ProtometodoQR_2} si ridefinisca
      $\Matrix = RQ$;
    \item\label{ProtometodoQR_3} se
      $\ArrestCondition$ \`e falsa
      si torni al punto
      \ref{ProtometodoQR_1}, altrimenti l'algoritmo termina
      restituendo $\Matrix$.
  \end{enumerate}
\end{Theorem}
\begin{listing}
	\insertcode{octave}{"Metodi_numerici_per_problemi_agli_autovalori/ProtometodoQR.m"}
	\caption{Implementazione del metodo di iterazioni di sottospazi 
    per il calcolo di una forma normale di Schur in \LanguageName{octave}.}
\end{listing}
\Proof 


	\part{Variet\`a}
	\chapter{Variet\`a topologiche}
	\section{Definizioni e teoremi di base.}
\label{VarietaTopologiche_DefinizioniETeoremiDiBase}
\begin{Definition}
	Sia $(\TopologicalManifold,\Topology)$ uno spazio topologico che verifichi le seguenti condizioni:
	\begin{itemize}
		\item $\TopologicalManifold$ sia uno spazio di Hausdorff che verifica il secondo assioma di numerabilit\`a;
		\item esiste $n \in \mathbb{N}$ tale che per ogni punto $p \in \TopologicalManifold$ esiste un omeomorfismo $\Chart: \Open \rightarrow \VarOpen$, detto \Define{carta}\footnote{La definizione di carta varia tra diversi autori: alcuni chiamiano carta $\Open$, altri $\VarOpen$, altri ancora la coppia $(\Open,\Chart)$. La nostra scelta di definizione di carta si giustifica col fatto che essa contiene tutte le informazioni che vanno sotto il nome di carta: in effetti abbiamo $\Open = \Dom{\Chart}$ e $\VarOpen = \Cod{\Chart}$.}, tale che $\Open \in \Neighborhoods{p} \cap \Topology$ e $\VarOpen$ sia un aperto di $\mathbb{R}^n$.
	\end{itemize}
	$\TopologicalManifold$ si chiama \Define{$n$-variet\`a topologica}[topologica][variet\`a].
\end{Definition}
\begin{Definition}
	Sia $\TopologicalManifold$ una variet\`a topologica. Per ogni $p \in \TopologicalManifold$ sia $\Chart_p$ una carta. Chiamiamo $\bigcup_{p \in \TopologicalManifold} \lbrace \Chart_p \rbrace$ \Define{atlante}.
\end{Definition}
\begin{Definition}
	Sia $\TopologicalManifold$ una variet\`a topologica e siano $\Chart$ e $\VarChart$ due carte. Chiamiamo \Define{funzione di transizione}[di transizione][funzione] la funzione $\TransitionFunction: \Chart(\Dom{\Chart} \cap \Dom{\VarChart}) \rightarrow \VarChart(\Dom{\Chart} \cap \Dom{\VarChart})$ che a $x \in \Chart(\Dom{\Chart} \cap \Dom{\VarChart}$ associa $(\VarChart \circ \Chart^{-1})(x) \in \VarChart(\Dom{\Chart} \cap \Dom{\VarChart})$.
\end{Definition}
\begin{Theorem}
	Sia $\TopologicalManifold$ una $n$-variet\`a ed un $m$-variet\`a ($n, m \in \mathbb{N}$). Abbiamo $n = m$.
\end{Theorem}
\begin{Definition}
	Con le notazioni del teorema precedente, chiamiamo $n$ \Define{dimensione}[di una variet\`a][dimensione] di $\TopologicalManifold$.
\end{Definition}


	\chapter{Variet\`a algebriche}
	\section{Definizioni e teoremi di base.}
\label{VarietaAlgebriche_DefinizioniETeoremiDiBase}
\begin{Definition}
	Sia $\PolynomialIdeal$ un ideale polinomiale di $\RingAdjunction{\Field}{(x_i)_{i \in n}}$ ($n \in \mathbb{N}$), con $\Field$ campo algebricamente chiuso. Chiamiamo $\AlgebraicManifold{\PolynomialIdeal} = \lbrace (X_i)_{i \in n} \in \Field^n | \ForAll{\Polynomial \in \PolynomialIdeal}{\Polynomial((X_i)_{i \in n})} \rbrace$ \Define{variet\`a algebrica}[algebrica][variet\`a].
\end{Definition}
\begin{Theorem}
	Sia $(\PolynomialIdeal_j)_{j \in m}$ ($m \in \NotZero{\mathbb{N}}$) una famiglia di ideali polinomiali di $\RingAdjunction{\Field}{(x_i)_{i \in n}}$ ($n \in \NotZero{\mathbb{N}}$). Abbiamo
	\begin{itemize}
		\item $\AlgebraicManifold{\sum_{j \in m} \PolynomialIdeal_j} = \bigcap_{j \in m} \AlgebraicManifold{\PolynomialIdeal_J}$;
		\item $\AlgebraicManifold{\prod_{j \in m} \PolynomialIdeal_j} = \bigcup_{j \in m} \AlgebraicManifold{\PolynomialIdeal_j}$.
	\end{itemize}
\end{Theorem}
\Proof Sia $(X_i)_{i \in n} \in \AlgebraicManifold{\sum_{j \in m} \PolynomialIdeal_j}$. Poich\'e $\ForAll{J \in m}{\PolynomialIdeal_J \subseteq \sum_{j \in m} \PolynomialIdeal_j}$, $(X_i)_{i \in n} \in \bigcap_{j \in m} \AlgebraicManifold{\PolynomialIdeal_j}$.
\par Supponiamo ora $(X_i)_{i \in n} \in \bigcap_{j \in m} \AlgebraicManifold{\PolynomialIdeal_j}$. Ogni elemento di $\sum_{j \in m} \PolynomialIdeal_j$ \`e della forma $\sum_{j \in m} \Polynomial_j$, con $\ForAll{j \in m}{\Polynomial_j \in \PolynomialIdeal_j}$: per un elemento di tale forma, abbiamo $\left ( \sum_{j \in m} \PolynomialIdeal_j \right )((X_i)_{i \in n}) = \sum_{j \in m} \PolynomialIdeal_j((X_i)_{i \in n}) = 0$.
\par Dunque $\AlgebraicManifold{\sum_{j \in m} \PolynomialIdeal_j} = \bigcap_{j \in m} \AlgebraicManifold{\PolynomialIdeal_j}$.
\par Supponiamo ora $(X_i)_{i \in n} \in \bigcup_{j \in m} \AlgebraicManifold{\PolynomialIdeal_j}$. Ogni elemento di $\prod_{j \in m} \PolynomialIdeal_j$ \`e della forma $\sum_{k \in l} \prod_{j \in m} \Polynomial_{k,j}$, con $l \in \NotZero{\mathbb{N}}$ e $\ForAll{(k,j) \in l \times m}{\Polynomial_{k,j} \in \PolynomialIdeal_j}$: per un elemento di tale forma, abbiamo
$\left ( \sum_{k \in l} \prod_{j \in m} \Polynomial_{k,j} \right )((X_i)_{i \in n}) =
\sum_{k \in l} \prod_{j \in m} \Polynomial_{k,j}((X_i)_{i \in n}) = 0$.
\par Supponiamo ora invece $(X_i)_{i \in n} \notin \bigcup_{j \in m} \AlgebraicManifold{\PolynomialIdeal_j}$. Allora, data una qualsiasi famiglia $(\Polynomial_j)_{j \in m}$, con, per ogni $j \in m$, $\Polynomial_j \in \PolynomialIdeal_j$, abbiamo $\prod_{j \in m} \Polynomial_j \in \prod_{j \in m} \PolynomialIdeal_j$, ma anche $\left ( \prod_{j \in m} \Polynomial_j \right ) ((X_i)_{i \in n}) = \prod_{j \in m} \Polynomial_j((X_i)_{i \in n}) \neq 0$. Dunque $(X_i)_{i \in n} \notin \AlgebraicManifold{\prod_{j \in m} \PolynomialIdeal_j}$.
\par Abbiamo quindi $\AlgebraicManifold{\prod_{j \in m} \PolynomialIdeal_j} = \bigcup_{j \in m} \AlgebraicManifold{\PolynomialIdeal_j}$. \EndProof


	\chapter{Variet\`a differenziali}
	\section{Definizioni e teoremi di base.}
\label{VarietaDifferenziali_DefinizioniETeoremiDiBase}


  \part{Probabilit\`a e statistica}
	\chapter{Probabilit\`a}
	\section{Definizioni e teoremi di base.}
\label{Probabilita_DefinizioniETeoremiDiBase}
\begin{Definition}
  Si chiama \Define{spazio di probabilit\`a}[di probabilit\`a][spazio] uno
  spazio di misura $(\SamplesSpace,\Events,\Probability)$ tale che
  $\Probability(\SamplesSpace) = 1$. Chiamiamo
  \begin{itemize}
    \item \Define{evento} un elemento di $\Events$;
    \item \Define{probabilit\`a} la funzione $\Probability$.
  \end{itemize}
  Inoltre, dato un evento $\Event \in \Events$, diciamo che
  \begin{itemize}
    \item $\Event$ \`e \Define{certo}[certo][evento] se
      $\Event = \SamplesSpace$;
    \item $\Event$ \`e \Define{impossibile}[impossible][evento] se
      $\Event = \emptyset$;
    \item $\Event$ \`e \Define{quasi certo}[certo][quasi] se
      $\Probability(\Event) = 1$;
    \item $\Event$ \`e \Define{quasi impossibile}[impossibile][quasi] se
      $\Probability(\Event) = 0$.
  \end{itemize}
\end{Definition}

\section{Probabilit\`a condizionata.}
\label{Probabilita_ProbabilitaCondizionata}
\begin{Theorem}
  Sia $(\ProbabilitySpace,\Events,\Probability)$ uno spazio di probabilit\`a e
  sia $\Event \in \Events$.
  Denotiamo
  L'applicazione $\VarProbability: \Events \rightarrow [0,1]$ che a
  $\VarEvent \in \Events$ associa
  $\frac{\Probability(\VarEvent \cap \Event)}{\Probability(\Event)}$ \`e una
  probabilit\`a sullo spazio misurabile $(\ProbabilitySpace,\Events)$.
\end{Theorem}
\Proof Abbiamo
$\VarProbability(\ProbabilitySpace)
= \frac{\Probability(\ProbabilitySpace \cap \Event)}{\Probability(\Event)}
= \frac{\Probability(\Event)}{\Probability(\Event)} = 1$.
\par Sia $(\Event_n)_{n \in \mathbb{N}}$ una famiglia numerabile di eventi due
a due disgiunti.
Abbiamo
\begin{align*}
  \VarProbability \left ( \bigcup_{n \in \mathbb{N}} \Event_n \right )
  &= \frac{\Probability
    \left ( \left ( \bigcup_{n \in \mathbb{N}} \Event_n \right )
      \cap \Event \right )}{
    \Probability(\Event)},\\
  &= \frac{\Probability
    \left ( \bigcup_{n \in \mathbb{N}} (\Event_n \cap \Event ) \right )}{
    \Probability(\Event)},\\
  &= \frac{\sum_{n \in \mathbb{N}}\Probability(\Event_n \cap \Event)}{
    \Probability(\Event)},\\
  &= \sum_{n \in \mathbb{N}}\VarProbability(\Event_n).\text{ \EndProof}
\end{align*}
\begin{Definition}
  Nelle ipotesi del teorema precedente, chiamiamo
  \Define{probabilit\`a condizionata}[condizionata][probabilit\`a]
  a $\Event$ la probabilit\`a $\VarProbability$.
  Denoteremo
  $\Probability(\VarEvent|\Event)$ la probabilit\`a condizionata a
  $\Event$ di $\VarEvent \in \Events$.
\end{Definition}
\begin{Theorem}
  \TheoremName{Teorema di Bayes}[di Bayes][teorema]
  Sia $(\ProbabilitySpace,\Events,\Probability)$ uno spazio di probabilit\`a e
  siano $\Event, \VarEvent \in \Events$ con
  $\Probability(\VarEvent) \neq 0$.
  Abbiamo
  \[
    \Probability(\Event|\VarEvent)
    = \frac{\Probability(\VarEvent|\Event) \Probability(\Event)}
      {\Probability(\VarEvent)}.
  \]
\end{Theorem}
\Proof Supponiamo $\Probability(\Event) \neq 0$. Abbiamo
\begin{align*}
  \Probability(\Event|\VarEvent)
  &= \frac{\Probability(\Event \cap \VarEvent)}{\Probability(\VarEvent)},\\
  &= \frac{\Probability(\VarEvent|\Event)\Probability(\Event)}
        {\Probability(\VarEvent)}.
\end{align*}
\par Supponiamo invece $\Probability(\Event) = 0$. Allora
$\Probability(\Event|\VarEvent)
= \frac{\Probability(\Event \cap \VarEvent)}{\Probability(\VarEvent)}
= 0$. \EndProof

\section{Variabili aleatorie.}
\label{Probabilita_VariabiliAleatorie}
\begin{Definition}
  Siano
  \begin{itemize}
    \item $(\SamplesSpace,\Events,\Probability)$ uno spazio di
      probabilit\`a;
    \item $(\MeasureSpace,\SigmaAlgebra)$ uno spazio di misura.
  \end{itemize}
  Chiamiamo
  \Define{variabile $(\MeasureSpace,\SigmaAlgebra)$-aleatoria}[aleatoria][variabile] o
  \Define{$(\MeasureSpace,\SigmaAlgebra)$-casuale}[casuale][variabile] o ancora
  \Define{$(\MeasureSpace,\SigmaAlgebra)$-stocastica}[stocastica][variabile]
  un'applicazione misurabile
  $\RandomVariable: \SamplesSpace \rightarrow \MeasureSpace$.
\end{Definition}
\par  Ometteremo la specifica di $\MeasureSpace$ o $\SigmaAlgebra$ ogni
qualvolta essi saranno chiari dal contesto: in particolare, assumeremo molto
spesso
$\MeasureSpace = \mathbb{R}$
e $\SigmaAlgebra$ uguale alla $\sigma$-algebra di Borel di $\mathbb{R}$ munito
della topologia euclidea.
\begin{Definition}
  Siano
  \begin{itemize}
    \item $(\SamplesSpace,\Events,\Probability)$ uno spazio di
      probabilit\`a;
    \item $(\MeasureSpace,\SigmaAlgebra)$ uno spazio di misura;
    \item $\RandomVariable: \SamplesSpace \rightarrow \MeasureSpace$ una
      variabile aleatoria.
  \end{itemize}
  Chiamiamo
  \Define{distribuzione di probabilit\`a}[di probabilit\`a][distribuzione]
  o
  \Define{legge di probabilit\`a}[di probabilit\`a][legge]
  la misura immagine di $\Measure$ tramite $\RandomVariable$.
\end{Definition}

\section{Valore atteso e speranza condizionale.}\label{ValoreAttesoESperanzaCondizionale}
\begin{Definition}
	Si definisce
  \Define{valore atteso}[atteso][valore]
  della variabile aleatoria integrabile $\RandomVariable$ la quantit\`a
  $\ExpectationValue{X} = \int_\SamplesSpace X(\omega) d\Probability{\omega}$.
\end{Definition}
\begin{Definition}
	Due variabili aleatorie $\RandomVariable$ e $\VarRandomVariable$ su uno stesso
  spazio di probabilit\`a $(\ProbabilitySpace,\Events,\Probability)$ si dicono
  \Define{equivalenti}[aleatorie equivalenti][variabili]
  quando sono quasi certamente uguali:
  $\Probability(X = Y) = 1$.
\end{Definition}
\begin{Definition}
	Sia $X$ una variabile aleatoria integrabile per lo spazio $(\SamplesSpace,\SigmaAlgebra_1,\Probability)$ e sia $\SigmaAlgebra_2$ un'altra $\sigma$-algebra tale che $\SigmaAlgebra_2 \subseteq \SigmaAlgebra_1$. Qualsiasi variabile aleatoria $Y$ $\SigmaAlgebra_2$-misurabile tale che per ogni $A \in \SigmaAlgebra_2$ si abbia $\int_A Xd\Probability = \int_A Yd\Probability$ si chiama \Define{speranza condizionale di $X$ rispetto a $\SigmaAlgebra_2$} e si denota $Y = \ConditionalExpectationValue{X}{\SigmaAlgebra_2}$.
\end{Definition}
\begin{Theorem}
	Date una variabile aleatoria integrablie $X$ per lo spazio $(\SamplesSpace,\SigmaAlgebra_1,\Probability)$ e una $\sigma$-algebra $\SigmaAlgebra_2 \subseteq \SigmaAlgebra_1$ esiste una speranza condizionale di $X$ rispetto a $\SigmaAlgebra_2$ e tutte le speranze condizionali sono equivalenti.
\end{Theorem}
\begin{Definition}
	Siano $X$ una variabile aleatoria integrabile da $(\SamplesSpace_1,\SigmaAlgebra_1,\Probability)$ e $Y: (\SamplesSpace_1,\SigmaAlgebra_1) \rightarrow (\SamplesSpace_2,\SigmaAlgebra_2)$ una variabile aleatoria. Si chiama \Define{speranza condizionale di $X$ rispetto ad $Y$} la quantit\`a $\ConditionalExpectationValue{X}{Y} = \ConditionalExpectationValue{X}{Y^{-1}(\SigmaAlgebra_2)}$.
\end{Definition}
\begin{Theorem}
	La speranza condizionale gode delle seguenti propriet\`a:
	\begin{itemize}
		\item $\ExpectationValue{X} = \ExpectationValue{\ConditionalExpectationValue{X}{\SigmaAlgebra_2}}$;
		\item $\ConditionalExpectationValue{aX+Y}{\SigmaAlgebra_2} = a\ConditionalExpectationValue{X}{\SigmaAlgebra_2} + \ConditionalExpectationValue{Y}{\SigmaAlgebra_2}$;
		\item se $X \leq Y$ quasi certamente, allora $\ConditionalExpectationValue{X}{\SigmaAlgebra_2} \leq \ConditionalExpectationValue{Y}{\SigmaAlgebra_2}$;
		\item se $X$ \`e $\SigmaAlgebra_2$-misurabile, allora $\ConditionalExpectationValue{X}{\SigmaAlgebra_2} = X$;
		\item se $X$ \`e indipendente da $\SigmaAlgebra_2$, allora $\ConditionalExpectationValue{X}{\SigmaAlgebra_2} = \ExpectationValue{X}$;
		\item se $\SigmaAlgebra_3 \subseteq \SigmaAlgebra_2$, allora $\ConditionalExpectationValue{\ConditionalExpectationValue{X}{\SigmaAlgebra_2}}{\SigmaAlgebra_3} = \ConditionalExpectationValue{X}{\SigmaAlgebra_3}$;
		\item se $Y$ \`e $\SigmaAlgebra_2$-misurabile e limitata, allora $\ConditionalExpectationValue{XY}{\SigmaAlgebra_2} = Y\ConditionalExpectationValue{X}{\SigmaAlgebra_2}$;
		\item se $(X_n)_{n \in \mathbb{N}}$ \`e una successione crescente di variabili aleatorie convergente quasi certamente ad $X$, allora $(\ConditionalExpectationValue{X_n}{\SigmaAlgebra_2})_{n \in \mathbb{N}}$ converge quasi certamente a $\ConditionalExpectationValue{X}{\SigmaAlgebra_2}$.
	\end{itemize}
\end{Theorem}

\section{Martingale e concetti correlati.}\label{Martingale}
\par D'ora in poi supporremo sempre dato uno spazio di probabilit\`a $(\SamplesSpace,\SigmaAlgebra,\Probability)$, dove $\SamplesSpace$ \`e lo spazio dei campioni, $\SigmaAlgebra$ la $\sigma$-algebra degli eventi e $\Probability$ la misura di probabilit\`a.
\begin{Definition}
	Una \Define{filtrazione} \`e una famiglia crescente\footnote{Per la relazione d'inclusione.} $(\SigmaAlgebra_t)_{t \in T}$, con $T \subseteq \mathbb{R}^+$, di sotto $\sigma$-algebre di $\SigmaAlgebra$. $(\SamplesSpace,\SigmaAlgebra,\SigmaAlgebra_t,\Probability)$ si chiama allora \Define{spazio di probabilit\`a filtrato}.
\end{Definition}
\par Si intrepreta normalmente $\SigmaAlgebra_t$ come la $\sigma$-algebra degli eventi noti al tempo $t$. Supporremo di seguito assegnato un insieme $T \subset \mathbb{N}$.
\begin{Definition}
	Chiamiamo \Define{processo stocastico} una famiglia di variabili aleatorie $(X_t)_{t \in T}$, dove ogni $X_t$ \`e $\SigmaAlgebra_t$-misurabile.
\end{Definition}
\par D'ora in poi, supporremo $T \subseteq \mathbb{N}$ tale che se $x \in T$ e $y < x$ (con $y \in \mathbb{N}$), allora $y \in T$.
\begin{Definition}
	Un processo stocastico $(X_t)_{t \in T}$ si dice
	\begin{itemize}
		\item \Define{adattato} se, per ogni $t \in T$, $X_t$ \`e $\SigmaAlgebra_t$-misurabile;
		\item \Define{prevedibile} se, per ogni $t \in T$, $X_t$ \`e $\SigmaAlgebra_{t - 1}$-misurabile.
	\end{itemize}
\end{Definition}
\par Un processo stocastico prevedibile \`e adattabile.
\begin{Definition}
	Un processo adattato $(M_t)_{t \in T}$ (reale o vettoriale reale) si chiama
	\begin{itemize}
		\item \Define{martingala} quando per ogni $t > 0$ si ha $\ConditionalExpectationValue{M_t}{\SigmaAlgebra_{t - 1}} = M_{t - 1}$;
		\item \Define{supermartingala} quando per ogni $t > 0$ si ha $\ConditionalExpectationValue{M_t}{\SigmaAlgebra_{t - 1}} \leq M_{t - 1}$;
		\item \Define{submartingala} quando per ogni $t > 0$ si ha $\ConditionalExpectationValue{M_t}{\SigmaAlgebra_{t - 1}} \geq M_{t - 1}$.
	\end{itemize}
\end{Definition}
\begin{Lemma}
	Un processo stocastivo vettoriale \`e una [super,sub]martingala se e solo se lo sono le sue componenti.
\end{Lemma}
\begin{Theorem}
	Sia $(M_t)_{t \in T}$ un processo stocastico. Valgono le seguenti propriet\`a:
	\begin{itemize}
		\item $(M_t)_{t \in T}$ \`e una martingala se e solo se, per ogni $t \in T$, $(\forall j \in \mathbb{N})(\ConditionalExpectationValue{M_{t + j}}{\SigmaAlgebra_t} = M_t)$;
		\item se $(M_t)_{t \in T}$ \`e una martingala, allora $(\forall t \in \mathbb{N})(\ExpectationValue{M_t}=\ExpectationValue{M_0})$;
		\item se $(M_t)_{t \in T}$ \`e una martingala e lo \`e anche $(N_t)_{t \in T}$, allora lo \`e anche $(M_t + N_t)_{t \in T}$.
	\end{itemize}
\end{Theorem}
\Proof Vediamo la prima propriet\`a. Sia $(M_t)_{t \in T}$ una martingala. Per $j = 0$ abbiamo $\ConditionalExpectationValue{M_{t + j}}{\SigmaAlgebra_t} = \ConditionalExpectationValue{M_t}{\SigmaAlgebra_t} = M_t$. Consideriamo ora $j$ arbitrario: assumendo l'uguaglianza della tesi vera per $j$, per $j + 1$ abbiamo $\ConditionalExpectationValue{M_{t + j + 1}}{\SigmaAlgebra_t} = \ConditionalExpectationValue{\ConditionalExpectationValue{M_{t + j + 1}}{\SigmaAlgebra_{t + j}}}{\SigmaAlgebra_t} = \ConditionalExpectationValue{M_{t + j}}{\SigmaAlgebra_t} = M_t$.
\par Viceversa supponiamo ora vera l'uguaglianza della prima propriet\`a per ogni $j$. Abbiamo $\ConditionalExpectationValue{M_t}{\SigmaAlgebra_{t - 1}} = \ConditionalExpectationValue{M_{t - 1 + 1}}{\SigmaAlgebra_{t - 1}} = M_{t - 1}$. Inoltre, $M_t$ \`e $\SigmaAlgebra_t$-misurabile e dunque $(M_t)_{t \in T}$ \`e adattabile.
\par Passiamo alla seconda propriet\`a. Procediamo per induzione su $t$. Il passo base \`e immediato, vediamo il passo induttivo: supponiamo dunque $\ExpectationValue{M_t} = \ExpectationValue{M_0}$. Abbiamo $\ExpectationValue{M_{t + 1}} = \ExpectationValue{\ConditionalExpectationValue{M_{t + 1}}{\SigmaAlgebra_t}} = \ExpectationValue{M_t} = \ExpectationValue{M_0}$.
\par Infine la terza propriet\`a. \`E chiaro che la somma di processi adattabili \`e adattabile e che, per ogni $t \in T$, $\ConditionalExpectationValue{M_t + N_t}{\SigmaAlgebra_{t - 1}} = \ConditionalExpectationValue{M_t}{\SigmaAlgebra_{t - 1}} + \ConditionalExpectationValue{N_t}{\SigmaAlgebra_{t - 1}} = M_{t - 1} + N_{t - 1}$. \EndProof
\begin{Corollary}
	Valogono propriet\`a analoghe a quelle del teorema precedente per le supermartingale e le submartingale.
\end{Corollary}
\Proof Analoga a quella del teorema precedente. \EndProof
\par Dato un processo stocastico $(X_t)_{t \in T}$, poniamo, per ogni $t > 0$,  $\Delta X_t = X_t - X_{t - 1}$.
\begin{Definition}
	Sia $(M_t)_{t \in T}$ una martingala e sia $(H_t)_{t \in T}$ un processo stocastico. Si chiama \Define{trasformata di $(M_t)_{t \in T}$ tramite $(H_t)_{t \in T}$} il processo stocastico $(X_t)_{t \in T}$ definito da
	\begin{align}
		&X_0 = H_0M_0,\nonumber\\
		&X_t = H_0M_0 + \sum_{j = 1}^t H_j\Delta M_j.\nonumber
	\end{align}
\end{Definition}
\begin{Theorem}
	Mantenendo le notazione della definizione precedente, se $(H_t)_{t \in T}$ \`e prevedibile, allora la trasformata di $(M_t)_{t \in T}$ tramite $(H_t)_{t \in T}$ \`e una martingala\footnote{Il teorema qui enunciato e dimostrato \`e un caso specifico di un teorema pi\`u generale: se si rilasciano le ipotesi che abbiamo imposto sull'insieme $T$, \`e necessario richiedere inoltre che $(H_t)_{t \in T}$ sia limitato.}.
\end{Theorem}
\Proof \`E chiaro che $(X_t)_{t \in T}$ \`e adattabile.
\par Abbiamo, per ogni $t > 0$,
\begin{align}
	\ConditionalExpectationValue{\Delta X_t}{\SigmaAlgebra_{t - 1}} &= \ConditionalExpectationValue{H_t (M_t - M_{t - 1})}{\SigmaAlgebra_{t -1}},\nonumber\\
	&= H_t\ConditionalExpectationValue{M_t}{\SigmaAlgebra_{t - 1}} - H_t\ConditionalExpectationValue{M_{t - 1}}{\SigmaAlgebra_{t - 1}},\nonumber\\
	&= H_tM_{t - 1} - H_t M_{t - 1} = 0.\nonumber
\end{align}
\par Dunque $\ConditionalExpectationValue{X_t}{\SigmaAlgebra_{t - 1}} = \ConditionalExpectationValue{X_{t - 1}}{\SigmaAlgebra_{t - 1}} = X_{t - 1}$, cio\`e che prova che $(X_t)_{t \in T}$ \`e una martingala. \EndProof.
\begin{Theorem}
	Il processo stocastico $(M_t)_{t \in T}$ \`e una martingala se e solo se per ogni processo stocastico prevedibile $(H_t)_{t \in T}$ si ha, per ogni $t > 0$,  $\ExpectationValue{\sum_{j = 1}^t H_j \Delta M_j} = 0$.
\end{Theorem}
\Proof Supponiamo $(M_t)_{t \in T}$ sia una martingala. Sia $(H_t)_{t \in T}$ un processo stocastico prevedibile. Definiamo $K_t = \begin{cases} 0\text{, se } t = 0,\\H_t\text{, altrimenti}. \end{cases}$. Allora $(K_t)_{t \ in T}$ \`e prevedibile e la trasformata $(X_t)_{t \in T}$ di $(M_t)_{t \in T}$ tramite $(K_t)_{t \in T}$ \`e una martingala. Dunque, per ogni $t \in \mathbb{N}$,  $\ExpectationValue{\sum_{j = 1}^t H_j \Delta M_j} = \ExpectationValue{X_t} = \ExpectationValue{X_0} = 0$.
\par Viceversa, supponiamo ora,  per ogni $t > 0$ e per ogni processo processo stocastico prevedibile $(H_t)_{t \ in T}$, $\ExpectationValue{\sum_{j = 1}^t H_j \Delta M_j} = 0$. Fissato $k > 0$, sia $A \in \SigmaAlgebra_{k - 1}$ e poniamo $H_t = \begin{cases} 0\text{, se }t \neq k,\\\Indicator{A}\text{, se }t = k. \end{cases}$: il processo stocastico $(H_t)_{t \in T}$ \`e prevedibile. Pertanto, abbiamo $\ExpectationValue{\sum_{j = 1}^k H_j \Delta M_j} = \ExpectationValue{H_k (M_k - M_{k - 1})} = 0$, cio\`e $\int_A M_k d\Probability = \int_A M_{k - 1} d\Probability$. Per l'arbitariet\`a di $A$, abbiamo $\ConditionalExpectationValue{M_k}{\SigmaAlgebra_{k - 1}} = M_{k - 1}$, cio\`e $(M_t)_{t \in T}$ \`e una martingala. \EndProof


  \chapter{Studio di alcune variabili aleatorie notevoli.}
  \section{Variabili aleatorie discrete}
\label{VariabiliAleatorieNotevoli_VariabiliAleatorieDiscrete}
\section{Distribuzione uniforme.}
\label{VariabiliAleatorieNotevoli_DistribuzioneUniforme}

\section{Distribuzione binomiale.}
\label{VariabiliAleatorieNotevoli_DistribuzioneBinomiale}
\begin{Definition}
  Siano
  \begin{itemize}
    \item $(\SamplesSpace,\Events,\Probability)$ uno spazio di probabilit\`a;
    \item $\BernoulliProbability \in [0,1]$;
    \item $\BernoulliCoprobability = 1 - \BernoulliProbability$;
    \item $\BernoulliTrials \in \mathbb{N}$;
    \item $\RandomVariable: \SamplesSpace \rightarrow \mathbb{R}$
      una variabile aleatoria.
  \end{itemize}
  Diciamo che $\RandomVariable$ ha distribuzione di probabilit\`a
  \Define{binomiale}[di probabilit\`a binomiale][distribuzione]
  di parametri $\BernoulliTrials$ e $\BernoulliProbability$
  quando la sua distribuzione \`e
  \[
    \Probability(X = k) =
    \begin{cases}
      \binom{n}{k}
      \BernoulliProbability^k \BernoulliCoprobability^{n - k}\text{ se }
      k \in n + 1,\\
      0\text{ altrimenti}.
    \end{cases}
  \]
  Una variabile aleatoria con distribuzione di probabilit\`a binomiale si chiama
  \Define{variabile aleatoria binomiale}[aleatoria binomiale][variabile].
  \par Inoltre, se $n = 1$, allora chiamiamo la distribuzione
  \Define{bernoulliana}[di probabilit\`a bernoulliana][distribuzione]
  di parametro $\BernoulliProbability$ e analogamente la variabile aleatoria
  \Define{bernoulliana}[aleatoria bernoulliana][variabile].
\end{Definition}
\begin{Theorem}
  Sia $\RandomVariable$ una variabile aleatoria bernoulliana di parametro
  $\BernoulliProbability$. Abbiamo
  \[
    \ExpectedValue{\RandomVariable} = \BernoulliProbability.
  \]
\end{Theorem}
\Proof Denotate
\begin{itemize}
  \item $\ProbabilityDistribution{\RandomVariable}$ la distribuzione di
probabilit\`a di $\RandomVariable$;
  \item $\BernoulliCoprobability = 1 - \BernoulliProbability$;
\end{itemize}
abbiamo
\begin{align*}
  \ExpectedValue{\RandomVariable}
  &= \int \RandomVariable d\ProbabilityDistribution{\RandomVariable},\\
  &= 0 \cdot \BernoulliProbability^0 \BernoulliCoprobability^{1 - 0}
    + 1 \cdot \BernoulliProbability^1 \BernoulliCoprobability^{1 - 1},\\
  &= \BernoulliProbability.\text{ \EndProof}
\end{align*} 
\begin{Theorem}
  Una variabile aleatoria $\RandomVariable$ ha distribuzione di probabilit\`a
  binomiale di parametri
  $\BernoulliTrials$ e
  $\BernoulliProbability$
  se e solo se esistono variabili aleatorie bernoulliane indipendenti
  $(\RandomVariable_i)_{i \in \BernoulliTrials}$
  di parametro
  $\BernoulliProbability$
  definite sullo stesso spazio di probabilit\`a tali che
  $\RandomVariable = \sum_{i \in BernoulliTrials} \RandomVariable_i$
\end{Theorem}
\begin{Corollary}
  Sia $\RandomVariable$ una variabile aleatoria binomiale di parametri
  $\BernoulliTrials$ e $\BernoulliProbability$. Abbiamo
  \[
    \ExpectedValue{\RandomVariable} = \BernoulliTrials\BernoulliProbability.
  \]
\end{Corollary}
\Proof Con le notazioni del teorema precedente, abbiamo
$\ExpectedValue{\RandomVariable}
= \sum_{i \in \BernoulliTrials} \ExpectedValue{\RandomVariable_i}
= \BernoulliTrials\BernoulliProbability$. \EndProof

\section{Distribuzione poissoniana.}
\label{VariabiliAleatorieNotevoli_DistribuzionePoissoniana}

\section{Distribuzione geometrica.}
\label{VariabiliAleatorieNotevoli_DistribuzioneGeometrica}


\section{Variabili aleatorie continue}
\label{VariabiliAleatorieNotevoli_VariabiliAleatorieContinue}
\section{Distribuzione gaussiana.}
\label{VariabiliAleatorieNotevoli_DistribuzioneGaussiana}



  \chapter{Statistica.}
  \section{Teorema di Perron-Frobenius.}
\label{CalcoloScientifico_TeoremaDiPerronFrobenius}
\begin{Theorem}
  \TheoremName{Teorema di Perron-Frobenius}[di Perron-Frobenius][teorema]
  Siano
  \begin{itemize}
    \item $n \in \NotZero{\mathbb{N}}$;
    \item $\Matrix \in \mathbb{C}^{n \times n}$ non negativa.
  \end{itemize}
  Allora
  \begin{itemize}
    \item $\SpectralRadius{\Matrix} \in \Spectrum{\Matrix}$;
    \item esiste un autovettore sinistro $\VarEigenvector$ tale che
      $\ForAll{k \in n}{\VarEigenvector \geq 0}$;
    \item esiste un autovettore destro $\Eigenvector$ tale che
      $\ForAll{k \in n}{\Eigenvector \geq 0}$.
  \end{itemize}
  Inoltre,
  \begin{itemize}
    \item se $\Matrix$ \`e irriducibile, allora
      \begin{itemize}
        \item $\SpectralRadius{\Matrix}$ \`e un autovalore semplice;
        \item esiste un autovettore sinistro $\VarEigenvector$ tale che
          $\ForAll{k \in n}{\VarEigenvector > 0}$;
        \item esiste un autovettore destro $\Eigenvector$ tale che
          $\ForAll{k \in n}{\Eigenvector > 0}$;
      \end{itemize}
      \item se $\Matrix$ \`e positiva, allora
        $\SpectralRadius{\Matrix}$
        \`e l'unico autovalore di modulo massimo di $\Matrix$.
  \end{itemize}
\end{Theorem}

\section{\English{Page rank}.}
\label{CalcoloScientifico_PageRank}
Il modello denominato
\Define{\English{page rank}}
\`e un modello nato per classificare l'importanza delle pagine
\English{web}, da cui il nome, ma che pu\`o essere riproposto per ogni
rete. Noi lo presenteremo nel suo contesto originale delle reti.
\par Sia $\PagesNumber$ il numero totale di tutte le pagine di una rete data.
Identificheremo ogni pagina con un numero naturale da $0$ a
$\PagesNumber - 1$, dove $\PagesNumber$ \`e il numero delle pagine stesse.
\par Esso viene costruito a partire da un grafo orientato
$\Graph = (\PagesNumber,\Edges)$ dove
$(\Node,\VarNode) \in \PagesNumber \times \PagesNumber$ appartiene a $\Edges$
se e solo se la pagina $\Node$ contiene un collegamento ipertestuale
alla pagina $\VarNode$.
\par Denotiamo $\AdjacencyMatrix$ la matrice di adiacenza di $\Graph$.
\par Lo scopo del modello di \English{page rank} \`e determinare un
\Define{vettore di pesi}[dei pesi (\English{page rank})][vettore]
$\Weights$: la componente $\Weights_k$ indica l'importanza della pagina $k$.
\par Definiamo i vettori
\begin{itemize}
  \item $\AllOnes = \Transposed{(1 \cdots 1)}$;
  \item $\PageDegree = \AdjacencyMatrix \AllOnes$;
  \item $\MatrixPageDegree \in \mathbb{R}^{\PagesNumber \times \PagesNumber}$
    tale che
    \[
      \ForAll{(j,k) \in \PagesNumber \times \PagesNumber}{
        \MatrixPageDegree_j^k =
        \begin{cases}
          0&\text{ se } j \neq k,\\
          \PageDegree_k&\text{ se } j = k;
        \end{cases}}
    \]
  \item $\Matrix = \MatrixPageDegree^{-1}\AdjacencyMatrix$.
\end{itemize}
La componente $\PageDegree_k$ ($k \in \PagesNumber$) indica il grado uscente
della pagina $k$.
\par Vorremmo costruire un modello per cui ogni pagina trasmette equamente
la sua importanza a tutte le pagine da esse puntata e l'importanza di una pagina
\`e esattamente la somma delle importanze cos\`i trasmesse tramite i
collegamenti.
\par Come vedremo, \`e utile ipotizzare che non esistano pagine pendenti
nella rete in esame; questa ipotesi \`e facilmente realizzabile con una delle due strategie seguenti, nessuna delle quali cambia sostanzialmente
il senso del modello della rete:
\begin{itemize}
  \item aggiungiamo ad ogni pagina pendente un collegamento a tutte
    le altre pagine;
  \item intoduciamo una pagina aggiuntiva $\PagesNumber$ che punta a tutte 
  le altre pagine ed \`e puntata da tutte le altre pagine:
  \[
    \ForAll{k > 0}{\AdjacencyMatrix_\PagesNumber^k
    = \AdjacencyMatrix_k^\PagesNumber = 1}.
  \]
\end{itemize}
\par Assumeremo dunque d'ora in poi il seguente assioma.
\begin{Axiom}
  La rete non contiene nodi pendenti.
\end{Axiom}
\begin{Theorem}
  $\Weights$ \`e autovettore sinistro di $\Matrix$.
\end{Theorem}
\Proof Per ogni pagina $k \in \PagesNumber$, abbiamo
\[
  \Weights_k = \sum_{j \in \PagesNumber} \AdjacencyMatrix_j^k \frac{\Weights_j}{\PageDegree_j},
\]
equivalente a
\[
  \Weights = \MatrixPageDegree^{-1}\AdjacencyMatrix\Weights,
\]
vale a dire
\[
  \Transposed{\Weights} = \Transposed{\Weights}\Matrix.\text{ \EndProof}
\]
\begin{Theorem}
  Abbiamo
  \begin{itemize}
    \item $1$ \`e autovalore di $\Matrix$ relvativo a $\AllOnes$;
    \item $\SpectralRadius{\Matrix} = 1$.
  \end{itemize}
\end{Theorem}
\Proof Abbiamo
$\Matrix\AllOnes
= \MatrixPageDegree^{-1}\AdjacencyMatrix\AllOnes
= \MatrixPageDegree^{-1}\PageDegree
= \AllOnes$.
\par Inoltre,
$\Norm{\Matrix}[\infty]
= \max_{k \in \PagesNumber} \sum_{j \in \PagesNumber}
  \AbsoluteValue{\Matrix_k^j}
= \max_{k \in \PagesNumber} \sum_{j \in \PagesNumber}
  \Matrix_k^j
= \max_{k \in \PagesNumber} \sum_{j \in \PagesNumber}
  \frac{\AdjacencyMatrix_k^j}{\PageDegree_k}
= 1$,
e quindi, se $\Eigenvector \in \mathbb{C}^\PagesNumber$ \`e autovettore
di $\Matrix$ relativo a $\Eigenvalue$, allora
$\AbsoluteValue{\Eigenvalue}\Norm{\Eigenvector}[\infty]
= \Norm{\Eigenvalue\Eigenvector}[\infty]
= \Norm{\Matrix\Eigenvector}[\infty]
\leq \Norm{\Matrix}[\infty]\Norm{\Eigenvector}[\infty]$,
da cui
$\AbsoluteValue{\Eigenvalue}
\leq \Norm{\Matrix}[\infty] = 1$. \EndProof
\begin{Theorem}
  Se $\Matrix$ ha un unico autovalore di modulo massimo, qualsiasi successione
  $(\Weights^{(k)})_{k \in \mathbb{N}}$ tale che
  \begin{itemize}
    \item $\Weights^{(0)}$ non \`e ortogonale a $\AllOnes$;
    \item per ogni $k > 0$,
      $\Weights^{(k)} = \Transposed{\Matrix}\Eigenvector^{(k - 1)}$;
    \item se $\Weights^{(0)}$ non ha componenti negative, allora
      $\ForAll{k \in \mathbb{N}}{
      \Norm{\Weights^{(k)}}[1] = \Norm{\Weights^{(0)}}[1]}$;
  \end{itemize}
  converge a un autovettore sinistro di $\Matrix$, cio\`e a $\Weights$
  a meno di prodotto per scalare.
\end{Theorem}
\Proof Riscriviamo il metodo delle potenze applicandolo alla matrice
$\Transposed{\Matrix}$ e al vettore $\Weights^{(0)}$ con una qualsiasi
tolleranza $\Tollerance$:
\begin{enumerate}
  \item\label{PageRank_MetodoDellePotenze_1} si ponga $j = 0$,
    $\Weights_0 = \Weights^{(0)}$,
    $\Eigenvalue_0^{(0)} =
    = \Transposed{\Weights_0}\Matrix\Weights_{0}$;
  \item\label{PageRank_MetodoDellePotenze_2} si ponga
    $\tilde{\Weights}_{j + 1} = \Matrix \Weights_j$;
  \item\label{PageRank_MetodoDellePotenze_3} si ponga
    $\Weights_{j + 1}
    = \frac{\tilde{\Weights}_{j + 1}}{\Norm{\tilde{\Weights_{j + 1}}}}$;
  \item\label{PageRank_MetodoDellePotenze_4} si ponga $\Eigenvalue_0^{(j + 1)}
    = \Transposed{\Weights_{j + 1}}\Matrix\Weights_{j + 1}$;
  \item\label{PageRank_MetodoDellePotenze_5} se
    $\AbsoluteValue{\Eigenvalue_0^{(j + 1)} - \Eigenvalue_0^{(j)}}
    > \Tollerance$
    si incrementi $j$ di $1$ e si torni al punto
    \ref{PageRank_MetodoDellePotenze_2}, altrimenti l'algoritmo termina
    restituendo la coppia $(\Eigenvalue_0^{(j + 1)},\Weights_{j + 1})$.
\end{enumerate}
\par Per ogni $k \in \mathbb{N}$, abbiamo
$\Weights_k = \Weights^{(k)}$ a meno di moltiplicazione per scalare.
\par Supponiamo $\Weights^{(0)}$ non abbia nessuna componente negativa.
Si dimostra immediatamente per induzione, utilizzando anche la non negativit\`a
di $\Matrix$, che per ogni $k \in \mathbb{N}$,
$\Norm{\Weights^{(k)}}[1]
= \Transposed{\Weights^{(k)}}\AllOnes
= \Transposed{\Weights^{(0)}}\AllOnes
\Norm{\Weights^{(0)}}[1]$. \EndProof
\par Con riferimento a quest'ultimo teorema, grazie alla costanza della
norma dei vettori della successione
$(\Weights^{(k)})_{k \in \mathbb{N}}$
\`e possibile calcolare direttamente l'autovettore sinistro $\Weights$
senza il passo di normalizzazione \ref{PageRank_MetodoDellePotenze_3}
evitando qualsiasi rischio di traboccamento dei moduli.
\par Definiamo adesso
\begin{itemize}
  \item $\gamma \in (0,1)$;
  \item $\Vector \in \mathbb{R}^\PagesNumber$ tale che
    $\ForAll{k \in \PagesNumber}{\Vector_k > 0}$ e
    $\Norm{\Vector}[1] = 1$;
  \item $A = \gamma\Matrix + (1 - \gamma)\AllOnes\Transposed{\Vector}$.
\end{itemize}
\begin{Theorem}
  La matrice $A$ \`e positiva.
\end{Theorem}
\Proof Fissiamo $(k,j) \in \PagesNumber \times \PagesNumber$:
$A_k^j$ \`e combinazione convessa di $\Matrix_k^j$ e
$\sum_{k \in \PagesNumber} \Vector_k$ con coefficienti non nulli,
dunque \`e strettamente positivo. \EndProof
\begin{Theorem}
  $\AllOnes$ \`e autovettore di $A$ relativo all'autovalore $1$.
\end{Theorem}
\Proof Abbiamo
$A\AllOnes
= \gamma\Matrix\AllOnes + (1 - \gamma)(\AllOnes\Transposed{\Vector})\AllOnes
= \gamma\AllOnes
  + (1 - \gamma)\AllOnes(\Transposed{\Vector}\AllOnes)
= \gamma\AllOnes + (1 - \gamma)\AllOnes
= \AllOnes$. \EndProof
\begin{Theorem}
  \TheoremName{Teorema di Brauer}[di Brauer][teorema]
  Siano
  \begin{itemize}
    \item $n \in \NotZero{\mathbb{N}}$;
    \item $J \in \mathbb{C}^{n \times n}$;
    \item $\Eigenvector$ autovettore di $J$ relativo all'autovalore
      $\Eigenvalue \in \Spectrum{J}$;
    \item $\VarVector \in \mathbb{C}^n$;
    \item $K = J + \Eigenvector\HermitianTransposed{\VarVector}$.
  \end{itemize}
  Abbiamo
  \[
    \Spectrum{K}
    = \Spectrum{J}
      \SetMin \lbrace \Eigenvalue \rbrace
      \cup \lbrace \Eigenvalue + \HermitianTransposed{\VarVector}\Eigenvector \rbrace.
  \]
\end{Theorem}
\Proof Sia $\UnitaryMatrix \in \mathbb{C}^n$ unitaria tale che
$\SchurNormalForm = \HermitianTransposed{\UnitaryMatrix} J \UnitaryMatrix$
sia una forma normale di Schur di $J$ tale che
$\SchurNormalForm_0^0 = \Eigenvalue$.
\par Denotando $\CanonicalBase^0$ il primo vettore della base canonica, abbiamo
\begin{align}
  \HermitianTransposed{\UnitaryMatrix} K \UnitaryMatrix
  &= \HermitianTransposed{\UnitaryMatrix} J \UnitaryMatrix
    + \HermitianTransposed{\UnitaryMatrix}
      \Eigenvector\HermitianTransposed{\VarVector}
      \UnitaryMatrix,\\
  &= \SchurNormalForm
    + \CanonicalBase^0\HermitianTransposed{\VarVector}
      \UnitaryMatrix.
\end{align}
\par Sia ora $k \in n$. Supponiamo $k = 0$: abbiamo
\[
  (\CanonicalBase^0\HermitianTransposed{\VarVector}\UnitaryMatrix)_0
  = \HermitianTransposed{\VarVector}\UnitaryMatrix.
\]
Se invece $k \neq 0$, allora, per ogni $j \in n$
\[
  (\CanonicalBase^0\HermitianTransposed{\VarVector}\UnitaryMatrix)_k
  = 0.
\]
Dunque tutti gli elementi diagonali della matrice triangolare superiore
$\HermitianTransposed{\UnitaryMatrix} K \UnitaryMatrix$,
forma normale di Schur,
coincidono con quelli di $\SchurNormalForm$ tranne
$(\HermitianTransposed{\UnitaryMatrix} J \UnitaryMatrix)_0^0$
che viene sostituito da
\[
  \Eigenvalue + \HermitianTransposed{\VarVector}\UnitaryMatrix\CanonicalBase^0
  = \Eigenvalue + \HermitianTransposed{\VarVector}\Eigenvector.
  \text{ \EndProof}
\]
\begin{Theorem}
  $1$ \`e autovalore semplice e di modulo massimo di $A$.
\end{Theorem}
\Proof Ricordiamo che
$A = \gamma\Matrix + (1 - \gamma)\AllOnes\Transposed{\Vector}$.
$\AllOnes$ \`e autovettore relativo a $1$ di $\Matrix$ e $1$ \`e autovalore
semplice e di modulo massimo di $\Matrix$ per il teorema di Perron-Frobenius.
Ne deduciamo che
$\AllOnes$ \`e autovettore relativo a $\gamma$ di $\gamma\Matrix$ e
$\gamma$ \`e autovalore semplice e di modulo massimo di $\gamma\Matrix$
per il teorema di Perron-Frobenius.
\par Abbiamo
$\gamma + (1 - \gamma)\Transposed{\AllOnes}\Vector
= \gamma + (1 - \gamma)\Norm{\Vector}[1] = 1$.
Dunque, per il teorema di Bauer, $1$ \`e autovalore semplice e di modulo
massimo di $A$. \EndProof

\section{Vibrazioni.}
\label{CalcoloScientifico_Vibrazioni.}
\subsubsection{Vibrazioni di corde elastiche.}
\label{CalcoloScientifico_VibrazioniDiCordeElastiche}
\par Consideriamo una successione finita di
$N \in \NotZero{\mathbb{N}}$ punti materiali
$(\PointMass_k)_{k \in N}$ di massa $(\Mass_k)_{k \in N}$
e coordinate $(X_k)_{k \in N}$ (nel piano e nello spazio indifferentemente)
tali che
\begin{itemize}
  \item i punti $\PointMass_0$ e $\PointMass_{N - 1}$ siano vincolati
    nella loro posizione al tempo iniziale $t = 0$;
  \item per ogni $k \in N - 1$, $\PointMass_k$ e $\PointMass_{k + 1}$
    sono collegati da una corda elastica di costante elastica
    $\ElasticConstant_k$;
  \item per ogni $k \in \NotZero{(N - 1)}$, $\PointMass_k$ \`e soggetto
    ad una forza d'attrito dinamico con coefficiente $\Friction_k$.
\end{itemize}
\par Sia $k \in \NotZero{(N - 1)}$. Per la seconda legge di Newton, abbiamo
\[
  \Mass_k X_k = - \ElasticConstant X_k - \Friction_k \dot{X_k}.
\]
Inoltre, abbiamo le condizioni al bordo
\[
  \begin{cases}
    \ForAll{t \in \RealNonNegative}{X_0(t) = X_0(0)},\\
    \ForAll{t \in \RealNonNegative}{X_{N - 1}(t) = X_{N - 1}(0)},\\
    \dot{X_0} = \dot{X_{N - 1}} = 0.
  \end{cases}
\]
\par Poniamo ora
\begin{itemize}
  \item $M \in \mathbb{R}^{N \times N}$ uguale alla matrice diagonale i cui
    elementi sono la successione delle masse:
    \[
      M =
      \lmatrix
      \begin{array}{ccc}
        \Mass_0 & &\\
        & \ddots &\\
        & & \Mass_{N - 1}
      \end{array}
      \rmatrix.
    \]
  \item $R \in \mathbb{R}^{N \times N}$ uguale alla matrice diagonale i cui
    elementi sono la successione dei coefficienti di attrito:
    \[
      R =
      \lmatrix
      \begin{array}{ccc}
        \Friction_0 & &\\
        & \ddots &\\
        & & \Friction_{N - 1}
      \end{array}
      \rmatrix.
    \]
  \item $K \in \mathbb{R}^{N \times N}$ uguale alla matrice tridiagonale i
    cui elementi sono
    $K_k^j =
    \begin{cases}
      - \ElasticConstant_k&\text{ se }k = j + 1,\\
      \ElasticConstant_k + \ElasticConstant_{k + 1}&\text{ se }k = j,\\
      - \ElasticConstant_{k + 1}&\text{ se }k = j - 1,
    \end{cases}$
    vale a dire
    \[
      K =
      \lmatrix
      \begin{array}{cccccc}
        \ElasticConstant_0 + \ElasticConstant_1 & - \ElasticConstant_1 &&&&\\
        - \ElasticConstant_0 & \ddots & \ddots &&&\\
        & \ddots & \ddots & \ddots & &\\
        && \ddots & \ddots & - \ElasticConstant\\
        &&& - \ElasticConstant & \ElasticConstant + \ElasticConstant
      \end{array}
      \rmatrix.
    \]
\end{itemize}

\subsubsection{Vibrazioni di membrane elastiche.}
\label{CalcoloScientifico_VibrazioniDiMembraneElastiche}



	\part{Fisica}
  \chapter{Meccanica classica}
	\section{Definizioni e teoremi sparsi}
\label{MeccanicaClassica_DefinizioniETeoremiSparsi}
\begin{Definition}
	Sia $\Vector \in \mathbb{R}^3$ un vettore applicato in un punto $P$. Chiamiamo \Define{momento}[di un vettore][momento] $\Moment$ di $\Vector$ rispetto a un punto $\Pole$ detto \Define{polo}[di un momento][polo] il prodotto vettoriale $\Moment = \VectorProduct{\vec{r}}{\Vector}$, dove $\vec{r} = P - \Pole$.
\end{Definition}
\begin{Definition}
	Fissato un qualche polo, chiamiamo
	\begin{itemize}
		\item \Define{momento delle forze}[delle forze][momento] di un punto materiale il momento della forza risultante su quel punto;
		\item \Define{momento angolare}[angolare][momento] di un punto materiale il momento della quantit\`a di moto in quel punto.
	\end{itemize}
\end{Definition}
\begin{Theorem}
	Dati un punto materiale e un qualche polo, il momento delle forze $\ForceMoment$ del punto materiale \`e la derivata del momento angolare $\AngularMoment$.
\end{Theorem}
\Proof Segue direttamente dal fatto che la forza risultante sul punto materiale sia la derivata della sua quantit\`a di moto. \EndProof
\begin{Axiom}
	\TheoremName{Legge di Newton}[di Newton][legge]\TheoremName{Secondo principio della dinamica}[secondo della dinamica][principio] Un punto materiale soggetto ad una forza $\vec{\Force}$ riceve un'accelerazione $\vec{\Acceleration}$ proporizionale alla forza stessa.
\end{Axiom}
\begin{Definition}
	Dato un punto materiale $\PointMass$, chiamiamo \Define{massa inerziale}[inerziale][massa] il rapporto tra il modulo di una forza esercitata su $\PointMass$ e il modulo dell'accelerazione che ne consegue.
\end{Definition}
\begin{Axiom}
	\TheoremName{Legge di grativazione universale}[di gravitazione universale][legge] Due punti materiali separati da una distanza $\vec{r}$ sono soggetti ciascuno ad una forza diretta lungo la retta congiungente i due punti, attrattiva e di modulo inversamente proporzionale al quadrato della distanza che le separa.
\end{Axiom}
\begin{Definition}
	Chiamiamo \Define{gravitazionale}[gravitazionale][forza], la forza descritta nell'assioma precedente.
\end{Definition}
\begin{Definition}
	Dato un punto materiale $\PointMass$,
	\begin{itemize}
		\item \Define{massa gravitazionale attiva}[gravitazionale attiva][massa] di $\PointMass$ la sua capacit\`a di attrarre altri punti materiali;
		\item \Define{massa gravitazionale passiva}[gravitazionale passiva][massa] di $\PointMass$ la sua capacit\`a di essere attratti da altri punti materiali.
	\end{itemize}
\end{Definition}
\begin{Theorem}
	Massa gravitazionale attiva e passiva sono equivalenti.
\end{Theorem}
\begin{Axiom}
	\TheoremName{Principio di equivalenza delle masse}[di equivalenza delle masse][principio] La massa inerziale e la massa gravitazionale di un corpo sono uguali.
\end{Axiom}
\begin{Theorem}
	L'accelerazione di caduta libera su un corpo sferico non dipende da nessuna caratteristica del grave.
\end{Theorem}
\Proof Siano $\Mass$ e $\VarMass$ le masse del grave e del corpo su cui si studia la caduta libera rispettivamente. Sia $R$ il raggio del corpo sferico.
\par Abbiamo, denotata $\GravityAcceleration$ l'accelerazione di gravit\`a e considerata l'altezza del grave trascurabile rispetto ad $R$, $\Mass \GravityAcceleration = \NewtonConstant \frac{\Mass\VarMass}{R}$, da cui, per il principio di equivalenza delle masse, $\GravityAcceleration = \NewtonConstant \frac{\Mass\VarMass}{R}$. \EndProof


	\part{Altri argomenti di matematica applicata}
	\part{Storia e filosofia della matematica}
	\chapter{Storia della matematica}
	\section{La matematica greca.}\label{LaMatematicaGreca}
\subsection{Caratteri generali della matematica greca.}\label{CaratteriGeneraliDellaMatematicaGreca}
\par In questa sottosezione discutiamo la matematica greca nel periodo che va dal VI secolo a.c. al VI secolo d.c. In particolare distinguiamo due periodi:
\begin{itemize}
	\item il \Define{periodo classico}: dal VI secolo a.c. al 323 a.c. (morte di Alessandro Magno):
	\item il \Define{periodo ellenistico}: dal 323 a.c. fino al VI secolo d.c.
\end{itemize}
\par Nel periodo classico si distinguono nettamente due tipi di approccio alla matematica: un approccio \`e pragmatico, sorge dai problemi della vita quotidiana di commercianti, artigiani e funzionari; l'altro \`e filosofico e nasce da speculazioni filosofiche.
\par Ben poco \`e noto dell'approccio pragmatico. Sembrano distinguirsi in particolare due classi di matematici pragmatici:
\begin{itemize}
	\item i \Define{``calcolatori''}: si tratta di figure che svolgono calcoli aritmetici per professione, servendosi di una tavola con funzioni simili a quelle del pi\`u moderno abaco. Si pensa che essi abbiano tratto le loro conoscenze dalla Mesopotamia (erano in uso da tempi pi\`u antichi le stesse tavole) e dall'Egitto (il loro modo di trattare le frazioni ha somiglianze col metodo egizio). Hanno verosimilmente influenzato i pitagorici;
	\item i \Define{``misuratori''}: si occupano di calcolare aree e volumi. Essi hanno una tradizione scritta, le cui tracce si ritrovano per\`o pi\`u nel Vicino Oriente che in Grecia: i loro testi somigliano molto a ricette di cucina, indicando istruzioni con una sintassi precisa, imperativi e sempre tramite esempi numerici concreti, mai attraverso formule generali; le istruzioni sono generalmente accompagnate da diagrammi esplicativi. Si pensa che a svolgere la funzione di ``misuratore'' siano inizialmente stranieri arrivati in Grecia portando con loro il proprio sapere matematico: successivamente da essi i greci imparano a loro volta queste competenze. \`E ragionevole pensare che i ``misuratori'' si organizzino in corporazioni che cercano di mantenere segrete le loro tecniche e che essi siano di estrazione sociale piuttosto bassa.
\end{itemize}
\par Sia ``calcolatori'' che ``misuratori'' sviluppano necessariamente una certa attitudine teorica, quasi certamente inconsapevole, i primi notando nei loro calcoli alcuni schemi ricorrenti, i secondi riadattanto ogni ``ricetta'' ai particolari casi numerici che sono chiamati a risolvere. Non sviluppano mai tuttavia concetti fondamentali della matematica quali il concetto di definizione, di teorema o di dimostrazione.
\par Per quanto riguarda l'approccio filosofico invece distinguiamo diverse scuole; in ciascuna di esse la materia di studio \`e la filosofia (all'epoca concentrata sulla comprensione dell'Essere) e la matematica si configura come un ramo della filosofia (da usare come strumento per la comprensione dell'Essere):
\begin{itemize}
	\item la \Define{scuola ionica}: il principale esponente \`e Talete di Mileto\footnote{Una citt\`a della Ionia, regione dell'Anatolia da cui prende il nome la scuola.} che vive verosimilmente nel VI secolo a.c. Talete \`e pi\`u importante come iniziatore della filosofia che della matematica, tuttavia gli si attribuiscono teoremi sui triangoli simili, che si ritiene egli usi per calcolare l'altezza delle piramidi;
	\item la \Define{scuola pitagorica}: fondata a Crotone da Pitagora (VI secolo a.c.), probabile allievo di Talete. La scuola \`e attiva in ambito matematico per il VI e V secolo, occupandosi prevalentamente di aritmetica, sia pure arrangiando ciottoli in forme geometriche (da qui espressioni ancora in uso oggi come ``al quadrato'' e ``al cubo''). La scuola pitagorica \`e la prima ad iniziare lo studio matematico della musica. La matematica pitagorica ha una forte valenza mistica-religiosa: i seguaci di Pitagora credono che ogni cosa in Natura sia numero (nel senso di numero naturale) e che la Natura si possa dunque spiegare in termini di numeri e rapporti fra numeri: \`e dalla scoperta dell'incommensurabilit\`a della diagonale di un quadrato col lato, ad opera di Ippaso da Metaponto (V secolo a.c.), che si pone per la prima volta il problema della legittimit\`a dei numeri -- per i pitagorici, e per tutta la matematica classica (dell'approccio filosofico), solo i numeri razionali positivi sono legittimi;
	\item la \Define{scuola eleatica}: il principale esponente di questa scuola \`e Zenone di Elea (V secolo a.c.), probabilmente originalmente un pitagorico come il suo maestro Parmenide. Egli esamina per la prima volta il concetto di ``continuo'', principalmente attraverso paradossi (il pi\`u famoso dei quali \`e probabilmente quello di Achille e la tartaruga), cio\`e ragionamenti che sembrano convincenti ma si scontrano con la realt\`a del mondo sensible: non avendo sviluppato il concetto di limite matematico, Zenone ne conclude che \`e il mondo sensibile ad essere fallace, rafforzando le gi\`a solide basi del pensiero che sfocia nella scuola platonica;
	\item la \Define{scuola sofista}: si sviluppa ad Atene nel V secolo a.c. Dal punto di vista matematico, i sofisti sono i primi a formulare problemi di costruzione geometrica, ponendo di fatto il requisito che queste costruzioni siano sempre eseguite con solamente riga e compasso. Lavorano principalmente al problema della duplicazione del cubo, della quadratura del cerchio e della trisezione dell'angolo;
	\item la \Define{scuola platonica}: fa riferimento a Platone (IV - III secolo a.c.) e succede ai sofisti come scuola dominante ad Atene. Ha una forte impronta pitagorica e il platonismo \`e la filosofia che pi\`u nettamente distingue mondo sensibile e mondo ideale: stabilisce che l'uomo intellettualmente elevato deve occuparsi di comprendere le idee, in quanto vere ed immmutabili, mentre le cose sono materia da lasciare all'uomo volgare; \`e all'Accademia platonica che si fa pi\`u drastica la distinzione tra approccio filosofico -- riservato ai cittadini nel senso ateniese del termine, cio\`e alla classe dirigente o aristocratica -- e l'approccio pragmatico -- tipico delle attiviti\`a considerate volgari e indegne di un uomo elevato, come ad esempio il commercio. La matematica assume il ruolo di attivit\`a iniziatica allo studio delle idee ed \`e proprio cos\`i che si fa spazio per la prima volta il concetto di dimostrazione in forma consapevole: il filosofo-matematico platonico fissa delle ipotesi che considera evidenti (cio\`e non possono essere un inganno del mondo sensibile) e attraverso un ragionamento deduttivo arriva ad una nuova verit\`a, non necessariamente evidente. Questa operazione di deduzione riesce pi\`u naturale in ambito geometrico, sia per le influenze pitagoriche che per la presa di distanze dall'approccio pragmatico basato sui numeri: \`e l'inizio del primato matematico della geometria sull'aritmetica, che rimarr\`a tale per secoli, sino alla nascita dell'algebra e la sua successiva integrazione con la geometria attraverso i lavori di Descartes e Fermat. I platonici sono i primi a studiare le sezioni coniche e aprono il loro insieme di numeri legittimi a quei numeri irrazionali che possono essere costruiti geometricamente;
	\item la \Define{scuola di Eudosso}: Eudosso (IV secolo a.c.) \`e considerato uno dei pi\`u grandi matematici dell'Antichit\`a. Discepolo di Platone, egli introduce una nuova teoria delle proporzioni e delle grandezze per legittimizzare gli irrazionali incontrati nel corso di costruzioni geometriche: la sua idea \`e quella di aggirare il concetto di numero -- non si assegna dunque pi\`u un numero al lato di un quadrato calcolando poi un rapporto irrazionale della diagonale sul lato; si considera invece direttamente il rapporto, che \`e uguagliabile ad altri rapporti, senza per\`o mai passare dal concetto di numero. \`E il passo decisivo che cancella i numeri dalla geometria, sino come gi\`a detto all'avvento della geometria cartesiana; si continuer\`a tuttavia a studiare la teoria dei numeri (naturali e razionali positivi), su base deduttiva come per la geometria, ma separatamente da questa (si vedano gli \textit{Elementi} di Euclide);
	\item la \Define{scuola aristotelica}: Aristotele (IV secolo a.c.) \`e un filosofo noto per il suo approccio enciclopedico, ed anche in matematica notiamo questa sua caratteristica. \`E il primo filosofo a precisare il concetto di definizione (anche in senso matematico) e a distinguire il concetto di definizione dalla propriet\`a di esistenza: non basta definire una cosa per dimostrare che essa esiste. Egli stabilisce come criterio di esistenza di una figura (ricordiamo il primato della geometria: esso \`e il motivo per cui la questione \`e affrontata solo in ambito geometrico) la sua costruibilit\`a con riga e compasso (eredit\`a sofista): ad esempio, trovare una procedura di costruzione del quadrato implica l'esistenza del quadrato. Aristotele \`e anche il primo a definire il concetto di assioma: egli distingue tra \Define{assiomi} -- verit\`a generali evidenti come le leggi logiche -- e \Define{postulati} -- affermazioni non necessariamente evidenti e meno generali, adatte come punto di partenze per una particolare scienza, e la cui verit\`a pu\`o apparire successivamente nello sviluppo della teoria dalle conseguenze che se ne deducono. Euclide riprende la distinzione aristotelica tra assioma e postulato, ma in seguito i due concetti verranno trattati come identici perdendo la distinzione.
\end{itemize}
\par Euclide (III secolo a.C.) e Apollonio (III secolo a.C. - II secolo a.C.) cronologicamente appartengono al periodo ellenistico, tuttavia per il contenuto delle loro opere, che approfondiremo pi\`u tardi, sono considerati matematici classici.
\par Il periodo ellenistico si caratterizza per una sintesi dell'approccio pragmatico e di quello filosofico, che sfocia in ci\`o che potremmo chiamare la matematica applicata: \`e in effetti soprattutto in periodo ellenistico che si sviluppano scienze come l'astronomia, l'ottica o la meccanica, sebbene questi argomenti fossero gi\`a stati esaminati anche in periodo classico. I matematici ellenistici si curano meno delle giustificazioni teoriche del loro lavoro e usano per esempio i numeri irrazionali senza esitazioni; inoltre rivalutano il primato della geometria, dando pi\`u spazio ai numeri. Di tutto ci\`o sono emblematici i lavori di Archimede (III secolo a.C.) -- che si occupa, fra l'altro, di calcolare aree e volumi, di leve, di ottica (problemi pragmatici) sfruttando i metodi della matematica degli \textit{Elementi} di Euclide -- di Tolomeo (II secolo d.C.), il quale pu\`o essere considerato il padre della trigonometria (una materia geometrica) e non manca di costruire le prime tavole trigonometriche (con valori numerici), e di Diofanto (IV secolo a.C. - III secolo a.C.), che introduce per la prima volta un'algebra simbolica e studia sistematicamente equazioni che chiamiamo oggi appunto diofantee (con metodi deduttivi).

\subsection{Euclide e gli \textit{Elementi}.}\label{EuclideEGliElementi}
\paragraph{Biografia di Euclide.} Ben poco \`e noto della vita di Euclide. Secondo il \textit{Commento al primo libro degli Elementi} di Proclo (412-485), Euclide sarebbe vissuto ad Alessandria d'Egitto nel III secolo a.c. La sua collocazione cronologica si basa su alcune citazioni di Euclide ad opera di Archimede che dimostrebbero che Euclide \`e antecedente ad Archimede, ma queste citazioni sono imprecise ed Archimede potrebbe anche citare invece Eudosso; altra prova addotta \`e un anedotto che coinvolge Euclide e Tolomeo I (il sovrano chiese se esistesse un modo pi\`u semplice per imparare la geometria, Euclide rispose che ``non vi sono vie regie per la geometria''), ma lo stesso anedotto viene raccontato anche con Menecmo e Alessandro Magno come protagonisti. 
\par Ad di l\`a dell'attendibilit\`a del discorso di Proclo, si ritiene comunque che Euclide sia vissuto all'incirca nel III secolo, in ragione della fondazione della biblioteca di Alessandria all'inizio del secolo che potrebbe giustificare la chiamata di un uomo come Euclide nell'importante centro culturale e dello stile del suo lavoro: geometrico, sintetico e deduttivo, in perfetto accordo con la filosofia platonica; enciclopedico, come Aristotele. Inoltre alcuni degli argomenti da lui studiati sono tipici dell'epoca ellenistica, come le sezioni coniche o l'ottica.
\paragraph{Le opere.} L'opera principale di Euclide consiste naturalmente nei suoi famosi \textit{Elementi}, ma egli scrisse anche molte altre opere che trattavano di sezioni coniche, statica, musica, ottica, astronomia e altri argomenti di geometria pi\`u avanzati. Le opere che ci sono pervenute, oltre agli \textit{Elementi}, sono
\begin{itemize}
	\item i \textit{Dati}: una raccolta di proposizioni che permette di determinare quando, dati certi elementi, \`e possibile dedurne (costruirne) altri;
	\item l'\textit{Ottica}: un trattato sulla visione dell'occhio;
	\item la \textit{Catottrica}\footnote{Vi sono dubbi sull'autenticit\`a.}: un trattato sugli specchi;
	\item i \textit{Fenomeni}: un trattato di astronomia;
	\item la \textit{Section canonis}: un trattato sulla musica (teoria armonica).
\end{itemize}
\par Analizziamo i suoi \textit{Elementi}.
\paragraph{Gli \textit{Elementi}: il genere.} Il genere degli \textit{Elementi}, vale a dire una raccolta di risultati ordinata, sistematica e completa del sapere matematico sino ad allora nota, non \`e una novit\`a, ma essi si impongono subito come modello di esposizione e come manuale. Il metodo \`e \Define{sintetico}: i teoremi presuppongono ipotesi da cui si ricavano deduttivamente teoremi. Solo i risultati vengono esposti: non viene mai spiegato come il teorema e, soprattutto, la sua dimostrazione sono stati trovati.
\paragraph{Gli \textit{Elementi}: contenuti.} Gli \textit{Elementi} trattano argomenti di geometria e di teoria dei numeri. Pi\`u precisamente:
\begin{itemize}
	\item i libri I -- VI trattano di geometria piana:
	\begin{itemize}
		\item libro I: geometria del triangolo e del parallelogramma;
		\item libro II: sezioni di segmenti, aree associate ad esse e quadratura di poligoni;
		\item libro III: geometria del cerchio, tangenti;
		\item libro IV: poligoni regolari;
		\item libro V: grandezze e proporzioni;
		\item libro VI: similitudini di figure piane;
	\end{itemize}
	\item i libri VII -- X trattano di teoria dei numeri:
	\begin{itemize}
		\item libro VII: frazioni, massimo comun divisore, minimo comune multiplo;
		\item libro VIII -- IX: progressioni geometriche, numeri primi, numeri perfetti;
		\item libro X: classificazione dei numeri irrazionali;
	\end{itemize}
	\item i libri XI -- XII trattano di geometria solida:
	\begin{itemize}
		\item libro XI: figure solide fondamentali, parallelepipedi;
		\item libro XII: piramidi, prismi, coni, cilindri, sfere;
		\item libro XIII: sezione aurea e solidi platonici;
	\end{itemize}
	\item i libri XIV -- XV sono considerati apocrifi e trattano di solidi platonici inscritti in sfere o in altri solidi platonici.
\end{itemize}
\paragraph{Gli \textit{Elementi}: la forma.}
\par Come vedremo pi\`u avanti, ci sono diverse difficolt\`a nello stabili quale sia il testo originale di Euclide, ma, secondo Proclo i teoremi erano tutti esposti in una forma canonica che presentava, in ordine, i seguenti elementi:
\begin{itemize}
	\item la \Define{protasi}: l'enunciato;
	\item l'\Define{ectesi}: la descrizione della figura geometrica su cui si lavora con la denominazione delle parti coinvolte nella protasi;
	\item il \Define{diorismo}: la denominazione di altre parti della figura che non erano esplicitamente coinvolte dalla protasi;
	\item la \Define{costruzione}: la costruzione di nuovi elementi all'interno della figura, con eventuali nuove denominazioni;
	\item la \Define{dimostrazione}: il ragionamento logico deduttivo, basato sulla figura, che permette di mostrare la verit\`a della protasi;
	\item il \Define{sumperasma}: la conclusione della dimostrazione, che ribadisce la protasi stavolta con un valore di dimostrata verit\`a.
\end{itemize}
\par Nonostante la forma sistematica dell'opera, non mancano difficolt\`a logiche e interpretative: per esempio, nella prima proposizione che mostra la procedura di costruzione di un triangolo equilatero Euclide usa il fatto non dimostrato che due circonferenze tracciate agli estremi di un segmento con lo stesso segmento per raggio si intersecano in due punti; nella quarta proposizione si fa riferimento al concetto non definito di movimento rigido.
\paragraph{Gli \textit{Elementi}: le difficolt\`a nello stabilire il testo originale di Euclide.} \`E sempre difficile stabilire il testo originale di un documento antico: esso \`e facilmente soggetto a errori di copiatura, perdite di parti intere, attribuzioni sbagliate ecc. Nel caso di un testo matematico come gli \textit{Elementi} si verifica anche la presenza di numerose alterazioni volontarie:
\begin{itemize}
	\item aggiunta o rimozione di materiale (soprattutto aggiunta): esse sono giustificate dal desiderio di rendere il testo pi\`u chiaro e completo o meno prolisso e pi\`u essenziale;
	\item riordinamento delle proposizione: lo scopo \`e quello di migliorare il ragionamento deduttivo;
	\item sostituzione di dimostrazione: pu\`o essere l'effetto della volont\`a di una maggior generalit\`a o di evitare particolari metodi di dimostrazione, in ogni caso di migliorare la dimostrazione.
\end{itemize}
\paragraph{Gli \textit{Elementi}: la Storia del testo.} Si ritiene che esistano due edizioni fondamentali nel Mondo Antico: l'edizione euclidea di et\`a ellenistica e un'edizione del IV secolo d.c. ad opera di Teone d'Alessandria. Circolavano inoltre gli importanti commenti di Erone (I secolo d.c.) e Pappo (III-IV secolo d.c.), che possono aver avuto una certa influenza sull'edizione teonina. Si pensa che il testo che ci \`e pervenuto sia principalmente quello di Teone.
\par A partire dal IX secolo gli arabi traducono gli \textit{Elementi} contribuendo alla loro diffusione e conservazione, ma la tradizione araba presenta testi molto pi\`u essenziali, in cui mancano numerose definizioni e diversi lemmi.
\par Nell'Europa medievale incontriamo sia traduzioni dal greco al latino che traduzioni dall'arabo al latino. Le pi\`u importanti di queste sono quelle di Adelardo di Bath (1075-1160) basate sia su fonti greche che su fonti arabe: su di esse (ma non solo) lavora Campano di Novara (m. 1296) per scrivere la propria traduzione degli \textit{Elementi} (1259), che costituisce un modello di riferimento sino al Cinquecento. Campano non mira a una ricostruzione del contenuto originale, bens\`i alla stesura di un testo matematicamente coerente, chiaro e corretto: non esita dunque ad allontanarsi dal testo di Euclide ogni qual volta egli lo ritenga necessario.
\par Nella seconda met\`a del Quattrocento viene inventata la stampa. Venezia era particolarmente interessata al reperimento, alla traduzione e alla diffusione dei testi antichi e si trova allora in una posizione geografica e storica ideale per queste attivit\`a; diventa dunque rapidamente uno dei principali centri di stampa di Europa. \`E proprio a Venezia che Erhardt Ratdolt (1142 - 1528) pubblica nel 1482 la prima edizione a stampa (\textit{editio princeps}) degli \textit{Elementi}: Ratdolt non \`e un matematico, il suo interesse per i testi scientifici nasce da una collaborazione col matematico Regiomontano (1436 - 1476), e dunque sceglie l'edizione di Campano perch\'e la pi\`u diffusa, senza porsi problemi filologici n\'e matematici. Notevole la presenza delle figure.
\par Nel 1505 viene stampata da Zamberti (1473 - 1543) una nuova edizione degli \textit{Elementi} in spirito filologico: Zamberti tenta di restare quanto pi\`u vicino possibile al suo codice greco. Il risultato \`e un testo vicino al testo originale (anche se bisogna tener conto che il codice greco in questione era pessimo), ma privo di qualsiasi sensibilit\`a geometrica. Ne scaturisce un dualismo tra l'edizione campiana e quella zambertiana che perdura per tutto il Cinquecento, con tentativi di sintesi o addirittura di accostamento delle due versioni.
\par Nel 1543 Tartaglia (1499 - 1557) pubblica la prima edizione a stampa in lingua volgare: \`e la prima edizione integrale pervenutaci, anche se almeno frammenti erano gi\`a stati pubblicati, per esempio nel \textit{Summa de arithmetica, geometria, proportioni et proportionalita} (1494) di Luca Pacioli (1445 circa - 1517), il quale si dice abbia pubblicato una traduzione integrale degli \textit{Elementi} che per\`o non ci \`e pervenuta. Tartaglia, sebbene sia solo un maestro di abaco, \`e un matematico e in quanto tale compie un'operazione di sintesi consapevole delle due edizioni campana e zambertiana: cerca dunque di modificare il meno possibile il testo, ma lo modifica comunque quando questo \`e necessario per ottenere un testo matematicamente coerente, in particolare con l'aggiunta di commenti propri e suggerimenti di applicazioni pratiche; ad esempio Tartaglia giustifica la sua scelta tra il termine campiano ``colonna rotonda'' e quello zambertiano ``cilindro'' (sceglie ``cilindro'' in quanto termine originale, mentre il termine ``colonna rotonda'' si riferische alle colonne architettoniche le quali per\`o non sono cilindriche bens\'i hanno ``un puoco di panzetta'').
\par La miglior sintesi tra matematica e filologia \`e dovuta a Commandino (1509 - 1575) che traduce gli \textit{Elementi} sia in latino (1572) che in volgare (1575). In particolare, Commandino esamina la questione della paternit\`a del testo, cio\`e se esso sia opera di Euclide o di Teone e in quale misura: la sua conclusione \`e che Euclide diede sia le proposizione che le dimostrazioni, ma che Teone riscrisse le dimostrazioni nel modo in cui egli le insegnava ai suoi discepoli (prima di Commandino c'era chi si era a spinto a sostenere che le dimostrazioni fossero tutte originali di Teone). Notevole \`e anche la scelta grafica nuova di Commandino: egli non rappresenta pi\`u i solidi come schiacciati nel piano, come veniva fatto nei codici greci, arabi, nonch\'e nelle edizioni campane e zambertiane, bens\`i egli li disegna secondo le contemporanee tecniche prospettiche; la scelta \`e accettabile anche da un punto di vista filologico in virt\`u della \textit{scaenographica}, citata da Gemino (I secolo a.c.), una tecnica che rappresenta gli oggetti come appaiono alla vista, ma di cui non si conoscono i dettagli.
\par L'edizione di Commandino rimane quella di riferimento sino al 1818 circa quando viene individuato nel manoscritto Vat.Gr.190 un testo corrispondente all'edizione euclidea con aggiunte dall'edizione teonina e per questo verosimilmente il pi\`u antico testimone che abbiamo degli \textit{Elementi}. Tale manoscritto, databile circa al IX secolo, \`e uno dei sei (tutti greci) utilizzati da Johann Ludvig Heiberg per l'edizione di riferimento di oggi. Interessante notare che le figure dell'edizione di Heiberg non sono quelle originali: quest'ultime infatti presentavano spesso casi particolari, mentre Heiberg preferisce offrire casi generali.

\subsection{Archimede e la geometria della misura.}\label{ArchimedeELaGeometriaDiMisura}
\paragraph{Biografia di Archimede.} Poco si sa della vita di Archimede: la sua figura ci \`e stata tramandata in una forma pi\`u leggendaria che storica, ma sia i dati storici che quelli leggendari contribuiscono a descrivere il famoso matematico siracusano.
\par Archimede nasce nel 287 a.C. a Siracusa e muore nel 212 a.C nella stessa citt\`a: la data di morte, corrispondente alla presa di Siracusa da parte dei romani, \`e certa. Non \`e noto se Archimede fosse di umili o di nobili natali, ma una teoria interessante lo vede figlio di un astronomo di nome Fidia, il quale sarebbe stato verosimilmente il primo ad indirizzarlo verso gli studi matematici.
\par La giovent\`u di Archimede vede l'ascesa di Gerone come re di Siracusa, eletto alla carica nel 270 a.C. Si pu\`o pensare che il giovane Archimede fu ulteriormente sollecitato verso gli studi meccanici osservando macchine da guerra, per esempio nello scontro di Gerone coi mercenari Mamertini che scatur\`i poi nella prima guerra punica (264 a.C. - 241 a.C.), o i numerosi cantieri aperti nella citt\`a.
\par Alcuni elementi suggeriscono un viaggio di Archimede ad Alessandria d'Egitto. Innanzittutto, \`e evidente l'attrazione che l'importante centro culturale esercita su qualsiasi intellettuale dell'epoca. L'amicizia di Archimede con Conone di Samo, verosimilmente stretta in Sicilia quando Conone vi si reca per compiere alcune misure astronomiche, potrebbe aver spinto Archimede ad un viaggio in Egitto, dove Conone lavora alla corte di Tolomeo III Evergete intorno al 246-245 a.C. allo scopo di rintracciare per il sovrano il ``ricciolo'' o la ``chioma'' di Berenice, un insieme di stelle che la regina Berenice aveva dedicato agli d\`ei; dopo la morte di Conone, Archimede prosegue i contatti con Dositeo, anch'egli attivo ad Alessandria, ma sempre in forma epistolare dalla sua Siracusa. Da segnalare la notizia, riferita da Diodoro Siculo (I secolo a.C.), secondo cui Archimede avrebbe inventato (o perfezionato) proprio in Egitto la coclea, nota anche come pompa a spirale o vite di Archimede. Infine, altro indizio significativo in favore di un eventuale viaggio del giovane Archimede nella terra del Nilo \`e il destinatario del \textit{Metodo meccanico}, il suo testamento scientifico: Eratostene di Cirene (276 a.C. - 194 a.C.), direttore della biblioteca di Alessandria dal 245 a.C. al 204 a.C.
\par Dal 240 a.C, anno in cui Archimede dedica l'\textit{Arenario} a Gelone, figlio di re Gerone, appena associato al regno, vive verosimilmente a Siracusa, dove si svolgono alcuni anedotti pi\`u o meno leggendari. Il primo riguarda la \textit{Syrakosia}: una nave colossale, descritta da Ateneo di Naucrati (II secolo d.C - III secolo d.C.) come una citt\`a in movimento, che viene utilizzata nel 238 a.C, ribattezzata \textit{Alexandris}, per inviare ad Alessandria le risorse necessarie per combattere una grave carestia. La leggenda, riferita da Plutarco (46-48 d.C - 125-127 d.C.), narra che Gerone, dopo aver sentito Archimede affermare di essere capace di muovere qualsiasi peso, lo sfida a muovere una delle sue navi: Archimede vara la \textit{Syrakosia}, con l'equipaggio a bordo, senza sforzo con un suo marchingegno. \`E in quest'occasione che viene pronunciata la famosa frase ``Datemi un punto d'appoggio e vi sollever\`o il mondo!''
\par Altro famoso aneddoto \`e quello della corona, raccontato da Vitruvio (I secolo d.C.): Gerone, sospettando che un artigiano a cui ha commissionato una corona in oro lo abbia truffato tenendo per s\'e dell'oro sostituendolo con argento, chiede ad Archimede di verificare. Archimede non pu\`o rompere o fondere la corona perch\'e ormai \`e stata consacrata: si narra che l'idea risolutiva viene in mente al matematico siracusano mentre si trova ai bagni e che egli sia investito da un tale entusiasmo da correre a casa ancora nudo gridando ``Eureka!'' Quale metodo egli abbia effettivamente usato non \`e chiaro: il racconto di Vitruvio prevede l'immersione della corona in acqua e il calcolo del volume da essa spostata, confrontata con analoghe misure per una certa quantit\`a di oro e una certa quantit\`a d'argento. Il principio usato \`e verosimilmente il famoso principio che porta il nome di Archimede unito al calcolo di pesi specifici.
\par L'ultimo aneddoto riguarda la morte di Archimede avvenuta nel 212 a.C, nel corso della seconda guerra punica (218 a.C. - 202 a.C.). Siracusa, inizialmente schierata con Roma, vede affermarsi al suo interno un forte partito filocartaginese, di cui, secondo Livio, fa parte anche Gelone. Gelone muore nel 216 a.C, cosicch\'e alla morte di Gerone, nel 215 a.C, \`e Geronimo, figlio di Gelone, a prendere il potere: essendo Geronimo solo quindicenne gli vengono affiancati numerosi tutori. Geronimo, filocartaginese, viene presto ucciso (214 a.C.) dal partito filoromano e nel 212 a.C Roma, davanti all'instabilit\`a di Siracusa, decide di attaccarla: dell'attacco \`e incaricato Marco Claudio Marcello (ante 268 a.C. - 208 a.C.), mentre la difesa della citt\`a \`e affidata ad Archimede.
\par La famosa storia degli specchi che bruciano le navi dei romani \`e poco verosimile: gli specchi parabolici infatti non possono concentrare i raggi luminosi su grandi distanze ed \`e assurdo pensare che le navi romane rimangano ferme ad attendere di prendere fuoco; pi\`u probabile \`e l'uso di specchi come ``accendini'' da usare per proiettili incendiari o per scaldare l'acqua di ``cannoni a vapore''. Altre macchine sfruttate da Archimede per l'occasione sono potenti gru che tirano massi sulle navi romane, la \textit{manus ferrea} che afferra le navi per poi ributtarle violentemente in acqua.
\par Archimede muore al termine della difficile conquista della citt\`a da parte dei romani. Secondo Livio (59 a.C. - 17 d.C.), Archimede \`e ucciso da un soldato romano intento al saccheggio. Plutarco d\`a diverse versioni che prevedono tutte l'uccisione da parte di un soldato romano: in un caso il soldato scambierebbe le attrezzature matematiche con cui Archimede si starebbe recando da Marcello per oggeti di valore; in un altro, dopo aver intimato ad Archimede di seguirlo per portarlo da Marcello, lo ucciderebbe in uno scatto d'ira sentendosi rispondere che avrebbe dovuto attendere che il matematico finisse di lavorare al problema che lo stava occupando; nell'ultimo il soldato lo minaccerebbe subito e lo ucciderebbe ignorando la richiesta del grande scienziato di terminare la sua dimostrazione. Le fonti riferiscono che Marcello non vuol che Archimede venga ucciso, tuttavia una teoria moderna vede la morte di Archimede come la conseguenza naturale della sua guida a capo della difesa della citt\`a, verosimilmente dovuta, oltre che alle sue evidenti capacit\`a, anche ad una presa di posizione politica a favore di Cartagine.
\par Sulla tomba di Archimede viene posta una sfera inscritta in un cilindro col rapporto tra il volume dei due solidi inciso -- $\frac{2}{3}$ -- che \`e lo stesso rapporto fra le aree laterali, secondo le volont\`a dello stesso Archimede, che aveva considerato il teorema su questi due rapporti il suo risultato pi\`u importante.
\paragraph{Opere.} L'elenco che segue riporta le opere di Archimede che ci sono pervenute, nello stesso ordine proposto dal filologo danese Heiberg:
\begin{itemize}
	\item \textit{Sulla sfera e il cilindro}: opera in due libri indirizzata a Dositeo. Nel primo libro Archimede stabilisce che il volume della sfera \`e due terzi del volume di un cilindro ad essa circostritto che che la superficie della sfera \`e quattro cerchi massimi, nel secondo libro sfrutta i risultati del primo per risolvere problemi correlati, come la divisione di una sfera per mezzo di un piano in due parti aventi un rapporto dato;
	\item \textit{Misura del cerchio}: comprende tre proposizione. La prima prova l'equivalenza tra il cerchio e un triangolo rettangolo avente per cateti raggio e circonferenza rettificata, la terza fornisce un'approssimazione di $\pi$: $3 + \frac{10}{71} < \pi < 3 + \frac{1}{7}$;
	\item \textit{Sui conoidi e sferoidi}: indirizzata a Dositeo, usa risultati della geometria di misura su solidi di rivoluzione: paraboloide e iperbolide (conoidi), ellissoidi (sferoidi);
	\item \textit{Sulle spirali}: altra opera indirizzata a Dositeo, studia la spirale di Archimede;
	\item \textit{Sull'equilibrio dei piani}: opera in due libri. Nel primo libro si studiano le leve e si ricercano i centri di gravit\`a di alcune figure piane, mentre il secondo si occupa della ricerca di centri di gravit\`a del segmento di parabola;
	\item \textit{Arenario}: dedicata a Gelone di Siracusa, descrive un sistema di numerazione per contare quantit\`a molto grandi, per esempio di granelli di sabbia;
	\item \textit{Quadratura della parabola}: indirizzata a Dositeo, dimostra che il segmento di parabola equivale ai $\frac{4}{3}$ del triangolo avente stessa base e stessa altezza del segmento di parabola. La dimostrazione viene svolta in due modi, uno meccanico e un altro geometrico;
	\item \textit{Sui galleggianti}: opera in due libri. Nel primo libro Archimede enuncia il famoso principio che porta il suo nome e descrive la condizione di galleggiamento di un segmento di sfera, nel secondo studia il galleggiamento di un segmento di paraboloide;
	\item \textit{Stomachion}: affronta il problema di suddivedere un quadrato o un rettangolo in quattordici pezzi commensurabili;
	\item \textit{Sul metodo meccanico}: indirizzata a Eratostene. L'opera descrive, con esempi, i metodi euristici adottati da Archimede e studia i problemi dell'``unghia'' cilindrica e del solido che si ottiene da due cilindri inscritti in un cubo con le basi sulle facce;
	\item \textit{Libro dei lemmi}: studia figure ottenute intersecando cerchi. Ci \`e pervenuto solo attraverso una parafrasi in arabo;
	\item \textit{Il problema dei buoi}: indovinello sul conteggio di buoi suddivisi in quattro colori e sui quali sono date certe condizioni numeriche; comporta numeri decisamente elevati. Non conosciamo la soluzione di Archimede.
\end{itemize}
\paragraph{La geometria di misura.} Archimede, come si \`e visto, lavora in numerosi settori: uno di quelli pi\`u importanti \`e la cosidetta geometria di misura. La geometria di misura si occupa del calcolo di aree e volumi, che all'epoca di Archimede non siginifica esprimere aree e volumi in termini numerici, bens\`i esprimere aree e volumi in termini di rapporti con aree e volumi pi\`u semplici (e per questo considerati noti).
\par Il metodo di Archimede \`e molto sofisticato, ma vago: le idee di fondo sono sempre le stesse, ma l'applicazione \`e diversa a seconda del problema particolare.
\par Il punto di partenza \`e l'individuazione del rapporto che si desidera dimostrare. Per questo scopo Archimede compie un ragionamento informale che descrive nel \textit{Metodo}, di cui riportiamo l'esempio relativo al confronto tra paraboloide e cilindro, appoggiandoci alla figura \ref{Archimede_MetodoMeccanico}.
\begin{figure}
	\includegraphics[width=0.8\textwidth]{Storia_della_matematica/Immagine_Archimede_MetodoMeccanico.png}
	\centering
	\caption{Confronto tra paraboloide e cilindro secondo il \textit{Metodo meccanico}.}
	\label{Archimede_MetodoMeccanico}
\end{figure}
\par L'idea di Archimede \`e la seguente: immaginiamo un paraboloide (in rosso) e il cilindro ad esso circoscritto (in blu). Supponiamo di tagliare il cilindro e il paraboloide secondo un piano parallelo alle basi del cilindro: otteniamo due sezioni circolari. \`E facile dimostrare che il rapporto tra la sezione del paraboloide $P$ e quella del cilindro $C$ \`e lo stesso rapporto che intercorre tra il segmento $s$ di estremi il vertice del paraboloide e il centro delle sezioni concentriche e l'altezza $h$ del cilindro: $P:C = s: h$. Immaginiamo ora di traslare la sezione $P$ lungo l'asse del paraboloide per una distanza oltre il vertice pari all'altezza $h$, ottenendo il cerchio con circonferenza magenta. Se immaginiamo inoltre che l'asse del paraboloide sia una bilancia con fulcro nel vertice del paraboloide (il triangolo nero), sulla quale poniamo solo la sezione magenta e quella blu, avremo, per la legge delle leva (che Archimede aveva gi\`a dimostrato nell'\textit{Equilibrio dei piani}: la leva \`e in equilibrio quando il braccio della forza sta al braccio della resistenza come la resistenza sta alla forza), la bilancia in equilibrio. L'argomentazione si pu\`o ripetere per tutte le sezioni analoghe del paraboloide e del cilindro.
\par Ora, Archimede conosce anche il centro di massa del paraboloide, situato nel punto medio del suo asse finito: egli immagina dunque di traslare adesso l'intero paraboloide, portandone il centro di massa nello stesso punto in cui portava il centro delle sezioni circolari (paraboloide magenta). Per quanto dimostrato in precedenza, la bilancia, caricata col paraboloide traslato e con il cilindro, sar\`a in equilibrio (per le propriet\`a del baricentro, \`e come se avessimo caricato tutte le sezioni del paraboloide nell'estremo sinistro della bilancia, ciascuna delle quali controbilancia una sezione del cilindro), e dunque il rapporto tra i volumi deve essere lo stesso che intercorre tra i bracci della leva: il volume del paraboloide, il cui centro di massa si trova a distanza $h$ dal fulcro, \`e met\`a del volume del cilindro, il cui centro di massa si trova a distanza $\frac{h}{2}$ dal fulcro.
\par Il motivo per cui Archimede si serva di questo metodo solo come di un ``trucco'' per individuare il rapporto da dimostrare poi pi\`u rigorosamente e non consideri gi\`a quest'argomentazione come una dimostrazione risiede probabilmente nel passaggio, da noi posto tra parentesi, in cui egli considera i due solidi come somma delle infinite sezioni che li compongono: questo passaggio \`e inaccettabile per la matematica greca che non riesce a concepire il continuo. Un'altra motivazione pu\`o risiedere nello status inferiore della meccanica rispetto alla geometria.
\par Archimede procede dunque con la dimostrazione geometrica vera e propria (l'unica che compare nei suoi scritti: il metodo euristico \`e esposto solo nel \textit{Metodo meccanico}). L'idea generale, che illustriamo in linguaggio moderno, consiste nel costruire due successioni di figure $(I_n)_{n \in \mathbb{N}}$ e $(C_n)_{n \in \mathbb{N}}$, le prime inscritte alla figura $X$ che si desidera ``calcolare'', le seconde circoscritte, con le ulteriori propriet\`a che, per ogni $n \in \mathbb{N}$, si abbia $I_n < R < C_n$, dove $R$ \`e la figura nota di riferimento, e la succesione $(C_n - I_n)_{n \in \mathbb{N}}$ sia infinitesima. Si procede ora alla dimostrazione che $X = R$ per un doppio assurdo:
\begin{itemize}
	\item supponiamo $X < R$, allora per $n$ sufficientemente grande abbiamo $C_n - I_n < R - X$, dunque $C_n - I_n < C_n - X$ da cui $X < I_n$, che contraddice il fatto he $I_n$ sia inscritto in $X$;
	\item supponiamo $X > R$, allora per $n$ sufficientemente grande abbiamo $C_n - I_n < X - R$, dunque $C_n - I_n < X - I_n$ da cui $X > C_n$, che contraddice il fatto he $C_n$ sia circoscritto ad $X$.
\end{itemize}
\par Dunque non pu\`o che essere $X = R$. Questo metodo, chiamato impropriamente \Define{metodo di esaustione} (impropriamente perch\'e non \`e mai stato formalizzato come metodo), era gi\`a stato utilizzato da Eudosso di Cnido (fine V secolo a.C. - IV secolo a.C.) per ``calcolare'' l'area del cerchio, come riportato anche nel libro XII degli \textit{Elementi} di Euclide. Eudosso non segue esattamente la procedura sopradescritta e spesso non la segue neanche Archimede: quest'ultimo la applica per dimostrare che un paraboloide ha volume dato dalla met\`a del volume del cilindro circoscritto (lo stesso esempio che abbiamo trattato per l'illustrazione del metodo euristico), ma la modifca secondo le necessit\`a del caso, ci\`o che prova che il metodo di esaustione, pi\`u che un metodo, \`e un'``impostazione mentale''. Vediamo l'esempio della quadratura di un segmento di parabola $P$, basandoci sulla figura \ref{Archimede_QuadraturaParabola}.
\begin{figure}
	\includegraphics[width=0.8\textwidth]{Storia_della_matematica/Immagine_Archimede_QuadraturaParabola.png}
	\centering
	\caption{Quadratura della parabola.}
	\label{Archimede_QuadraturaParabola}
\end{figure}
\par Prima di affrontare la dimostrazione vera e propria, Archimede enuncia, senza dimostrazioni, alcuni risultati della teoria delle coniche preapolloniana:
\begin{itemize}
	\item i \Define{diametri} (le rette che bisecano un fascio parallelo di corde; i punti di intersezione tra i diametri e la parabola sono detti \Define{vertici}) sono tutti paralleli all'asse della parabola;
	\item fissato un diametro, i quadrati delle \Define{ordinate} (le semicorde individuate dal diametro) sono proporizionali ai segmenti di diametro aventi per estremi il vertice e il punto medio della corda in questione;
	\item fissato un diametro, le corde sono parallele alla tangente nel vertice.
\end{itemize}
\par Definiamo \Define{base} di un segmento di parabola la corda che lo delimita e \Define{altezza} il segmento della perpendicolare alla base passante per il vertice. La tesi, che Archimede intuisce grazie al suo metodo euristico\footnote{La \Define{Quadratura della parabola} \`e in effetti diviso in due parti: la dimostrazione geometrica che qui riportiamo \`e la seconda parte, mentre la prima parte era un dimostrazione meccanica, piuttosto oscura, probabilmente da collegare a quanto descritto nel \Define{Metodo meccanico}.}, \`e che un segmento di parabola \`e equivalente ai $\frac{4}{3}$ del triangolo con stessa base e stessa altezza.
\par Archimede procede inscrivendo nella parabola una successione di poligoni ottenuta tramite triangoli: il primo poligono $P_0$ (in nero) \`e il triangolo che ha base coincidente con quella della parabola e per terzo vertice il vertice del parabola, il secondo poligono $P_1$ (in rosso, pi\`u la base del triangolo nero) si ottiene dal primo costruendo nei due segmenti di parabola staccati da $P_0$ due nuovi triangoli con la stessa tecnica descritta in precedenza e ``saldandoli'' a $P_0$ e i poligoni successivi si costruiscono analogamente, inscrivendo triangoli nei nuovi segmenti di parabola e ``saldandoli''.
\par Ora, Archimede afferma che il triangolo $P_0$ \`e maggiore della met\`a del segmento parabolico: ci\`o si vede facilmente considerando il parallelogramma circoscritto al segmento di parabola, che \`e doppio di $P_0$ e maggiore del segmento di parabola. Quindi, con un'argomentazione un po' oscura, Archimede ne deduce che i poligoni $P_n$ approssimano il segmento di parabola $P$, cio\`e che la successione $P_n - P$ \`e infinitesima.
\par Ora, sfruttando le propriet\`a sopraelencate della parabola, si dimostra che, per $n > 0$,  la superficie $T_n = P_n - P_{n - 1}$ \`e $\frac{T_{n - 1}}{4}$. Ne risulta, posto $T_0 = P_0$ che $P_n = T_0 + \sum_{i = 0}^n T_i = \frac{4}{3}T_0 - \frac{T_0}{3 \cdot 4^n}$. Notiamo che $(T_n)_{n \in \mathbb{N}}$ \`e una successione infinitesima.
\par Segue la doppia riduzione all'assurdo: si noti che non abbiamo costruito nessuna successione di poligoni circoscritti e che utlizziamo ora due successioni infinitesime. Abbiamo
\begin{itemize}
	\item supponiamo $P > \frac{4}{3}T_0$, allora per $n$ sufficientemente grande abbiamo $P - P_n < P - \frac{4}{3}T_0$, da cui $P_n > \frac{4}{3}T_0$, ma questo \`e assurdo perch\'e abbiamo anche $P_n = \frac{4}{3}T_0 - \frac{T_0}{3 \cdot {1}{4^n}}$;
	\item supponiamo $P < \frac{4}{3}T_0$, allora per $n$ sufficientemente grande abbiamo $T_n < \frac{4}{3}T_0 - P$, dunque $\frac{4}{3}T_0 - P_n = \frac{T_n}{3} < T_n < \frac{4}{3}T_0 - P$, da cui $P < P_n$ che \`e assurdo essendo $P_n$ inscritto in $P$.
\end{itemize}
\par Dunque deve essere $P = \frac{4}{3}T_0$.

\subsection{La geometria di posizione: da Apollonio a Pappo - Il metodo dell'analisi e sintesi.}\label{LaGeometriaDiPosizione}
\paragraph{Biografia di Apollonio di Perga (circa 262 a.C. - 190 a.C.).} Apollonio nasce a Perga nel regno di Pergamo e si trasferisce giovane ad Alessandria di Egitto dove studia geometria euclidea. Eccelle in geometria, tanto da essere chiamato il ``grande geometra'', e in astronomia. Il suo capolavoro sono le \textit{Coniche}.
\paragraph{La geometria di posizione: le \textit{Coniche}.} La geometria di posizione, a differenza della geometria di misura che abbiamo visto con Archimede, anzich\'e confrontare grandezze (che come sappiamo si identificano con le figure a cui si riferiscono) studia in quali relazioni posizionali si trovino i vari elementi geometrici: tra tali relazioni annoveriamo l'incidenza, il parallelismo, la tangenza, la giacenza. Vediamo alcuni esempi di studio di geometria di posizione nello studio delle coniche.
\par Esponiamo brevemente come venivano definite le coniche prima di Apollonio, per esempio nei lavori di Euclide. Euclide definisce il cono nei suoi \textit{Elementi}\footnote{Ricordiamo che le sezioni coniche, invece, non sono studiate dagli \textit{Elementi}.} come la figura che si ottiene dalla rotazione di un triangolo rettangolo attorno ad uno dei suoi cateti, distinguendo dunque tra
\begin{itemize}
	\item \Define{cono acutangolo}, quando la rotazione avviene attorno al cateto maggiore;
	\item \Define{cono ottusangolo}, quando avviene attorno al cateto minore;
	\item \Define{cono rettangolo}, quando il triangolo rettangolo \`e isoscele.
\end{itemize}
Le sezione coniche si ottengono allora sezionando il cono con un piano perpendicolare al'apotema del cono (l'ipotenusa del triangolo in rotazione): otteniamo
\begin{itemize}
	\item un'\Define{ellise}, quando il cono \`e acutangolo;
	\item un'\Define{iperbole}, quando il cono \`e ottusangolo;
	\item una \Define{parabola}, quando il cono \`e rettangolo.
\end{itemize}
\par Quest'impostazione ha alcuni importanti difetti:
\begin{itemize}
	\item tutti i coni sono retti;
	\item tutti i coni hanno una sola falda e dunque tutte le iperboli hanno un solo ramo;
	\item tutti i coni non solo sono finiti, ma non sono neppure prolungabili\footnote{La matematica greca non concepisce l'infinito attuale, tutt'al pi\`u intuisce quello potenziale. Ci\`o che chiamiamo impropriamente \Define{rette} riferendoci per esempio agli scritti di Euclide sono in realt\`a segmenti, con la caratteristica per\`o di essere prolungabili, ci\`o che giustifica l'uso del nostro linguaggio moderno.}: dato il processo di costruzione, non \`e concepibile nessun concetto di prolungamento;
	\item ad ogni cono corrisponde una ed una sola sezione conica\footnote{Per tipologia e per grandezza. Chiaramente sussiste la possibilit\`a di tagliare un apotema diverso del cono e di tagliare lo stesso apotema in un punto diverso; in particolare, osserviamo che se l'apotema \`e tagliato troppo lontano dal vertice l'ellisse pu\`o risultare ``tagliata'' rispetto a ci\`o che intendiamo solitamente.}.
\end{itemize}
\par Apollonio rivoluziona tutto ci\`o. Dati una circonferenza ed un punto esterno al piano della circonferenza, egli definisce \Define{cono} la figura generata da un retta passante per il punto -- detto \Define{vertice} -- e da qualsiasi altro punto appartenente alla circonferenza facendo scorrere questo secondo punto lungo tutta la circonferenza. Una \Define{sezione conica} \`e la figura individuata tagliando il cono con un piano qualsiasi.
\par La classificazione delle sezione coniche comincia da quelle banali: se il cono viene sezionato da un piano su cui giace l'asse si ottiene un (doppio) triangolo, tutti i piani paralleli al piano della base individuano sezioni circolari. Sono circolari anche le sezioni \Define{subcontrarie}, cio\`e le sezioni che si ottengono da una situazione come quella di figura \ref{Apollonio_subcontraria}, dove il triangolo blu giace su un piano passante per l'asse e perpendicolare alla base del cono ed \`e simile al triangolo rosso.
\begin{wrapfigure}{R}{0.5\textwidth}
	\includegraphics[width=0.48\textwidth]{Storia_della_matematica/Immagine_Apollonio_subcontraria.png}
	\caption{Sezione subcontraria.}
	\label{Apollonio_subcontraria}
\end{wrapfigure}
\par Apollonio passa dunque a classificare le coniche restanti. Egli definisce
\begin{itemize}
	\item \Define{ellisse} la sezione ottenuta da un piano secante due lati del triangolo per l'asse sulla stessa falda;
	\item \Define{iperbole} la sezione ottenuta da un piano secante due lati del triangolo per l'asse su falde diverse;
	\item \Define{parabola} la sezione ottenuta da un piano parallelo ad uno dei lati del triangolo per l'asse (e secante dunque l'altro lato).
\end{itemize}
\par Un altro problema molto interessante di geometria di posizione affrontato da Apollonio \`e quello delle tangenti: Euclide definisce la tangente ad una circonferenza come una retta che tocca la circonferenza senza tagliarla. Si noti che \`e una definizione topologica, cio\`e appunto di geometria di posizione. Inoltre, solo in un secondo momento Euclide dimostra che la tangente alla circonferenza in un punto \`e la perpendicolare al diametro per quel punto e che esiste, per ogni punto della circonferenza, una sola tangente. Allo stesso modo, Apollonio definisce la tangente per le coniche come la retta che tocca la conica lasciandola tutta da una parte, dimostra che \`e perpendicolare al diametro nel punto di tangenza e che per ogni punto della conica esiste una e una sola tangente.
\paragraph{Analisi e sintesi nelle \textit{Coniche}.} Apollonio \`e uno dei primi matematici ad usare il metodo dell'analisi e sintesi: prima di lui il metodo tipico era quello esemplificato da Euclide, esclusivamente sintetico -- si parte dalle premesse e si arriva alle conclusioni, in quest'ordine. Il metodo dell'analisi e della sintesi parte invece dalle conclusioni per arrivare a qualcosa di noto, dunque ritorna alle conclusioni. Apollonio, per quanto a noi noto, non codifica esplicitamente il metodo, neanche approssimativamente come fece Archimede col suo \textit{Metodo meccanico}, ma nelle \textit{Coniche} appaiono diversi esempi di applicazioni. Vediamone uno appoggiandoci alla figura \ref{Apollonio_diametro}.
\par Apollonio desidera costruire un diametro di una conica (in blu), cio\`e una retta che biseca un fascio di corde parallele. Egli suppone di aver gi\`a costruito il diametro (in magenta) e osserva allora che, date le corde rosse corripondenti al diametro magenta, esse sono bisecate dal diametro. Dunque, prese due corde parallele qualsiasi e individuato il loro punto medio, la retta passante per i punti medi \`e un diametro.
\begin{wrapfigure}{R}{0.5\textwidth}
	\includegraphics[width=0.48\textwidth]{Storia_della_matematica/Immagine_Apollonio_diametro.png}
	\caption{Costruzione di un diametro.}
	\label{Apollonio_diametro}
\end{wrapfigure}
\par Il metodo di analisi e sintesi sar\`a successivamente formalizzato da Pappo.
\paragraph{Biografia di Pappo.} Pappo di Alessandria (circa 290 d.C. - circa 350 d.C.) lavora alcuni secoli pi\`u tardi di Apollonio, in un'epoca dominata dalla civilt\`a romana, poco interessata alla matematica che essa limitava ad usare per scopi pratici senza per\`o svilupparla: Pappo \`e uno dei pochi matematici dell'epoca di cui abbiamo notizia che scopre teoremi nuovi (e non \`e certo un caso che lavori ad Alessandria piuttosto che a Roma). \`E ricordato soprattutto per alcuni teoremi e per le \textit{Colletiones mathematicae}, un compendio di matematica in otto libri, di cui purtroppo sono andati perduti il primo libro e parte del secondo.
\paragraph{Analisi e sintesi nelle \textit{Colletiones mathematicae}.} Nelle \textit{Colletiones}, Pappo cerca di descrivere in cosa consiste il metodo dell'analisi e della sintesi. La spiegazione \`e piuttosto oscura, ma alcuni fatti sembrano chiari:
\begin{itemize}
	\item l'analisi parte da ci\`o che si desidera dimostrare o costruire e arriva a qualcosa di noto;
	\item la sintesi \`e il processo reciproco dell'analisi: parte dal qualcosa di noto raggiunto dall'analisi e torna a ci\`o che si desidera dimostrare o costruire;
	\item la sintesi consiste nel ribaltamento solamente delle premesse e delle conclusioni, non di ogni singolo passaggio: se cos\`i non fosse non ci sarebbe alcun bisogno di ribaltare esplicitamente un'analisi nel caso di una dimostrazione (persisterebbe comunque la necessit\`a di ribaltamento nel caso di una costruzione);
	\item l'analisi \`e metodologicamente prioritaria rispetto alla sintesi, ma, essendo il punto di partenza dell'analisi lo scopo della sintesi, la sintesi \`e logicamente o ontologicamente prioritaria;
	\item l'analisi, secondo Pappo, porta a un fatto noto vero quando l'ipotesi \`e vera e falso quando l'ipotesi \`e falsa: questo mostra come l'analisi comprenda la tecnica di dimostrazione per assurdo, ma anche che Pappo \`e inconsapevole della possibilit\`a di ottenere conclusioni giuste da premesse sbagliate.
\end{itemize}


\section{Caratteri generali del rinascimento (XIII - XVI secolo).}\label{CaratteriGeneraliDelRinascimento}
\par La matematica del rinasciamento \`e caratterizzata dall'intreccio di tre fili, intreccio che esamineremo nelle pagine seguenti:
\begin{itemize}
	\item il primo filo \`e quello dell'aritmetica mercantile, che nasce dalla matematica di Leonardo Fibonacci e in particolare del suo \textit{Liber Abaci}, e sfocia in numerosi ambiti teorici e applicativi attraverso le scuole d'abaco (sottosezione \ref{LeonardoFibonacciELeScuoleDAbaco}), come gli studi prospettici di Piero della Francesca (circa 1416 - 1492) e la teoria delle equazioni, che preparer\`a il campo all'algebra simbolica di Fran\c{c}ois Vi\`ete (sottosezione \ref{FrancoisVieteELInvenzioneDellAlgebraSimbolica});
	\item il secondo filo \`e quello dell'umanesimo e della ricerca dei classici (sottosezione \ref{LUmanesimoEIlRecuperoDellaMatematicaGreca}), favorito dall'invenzione della stampa (sottosezione \ref{LInvenzioneDellaStampaELaDiffusioneDellaCulturaScientifica});
	\item l'ultimo filo \`e quello della tradizione dei testi filosofici antichi, in particolare della filosofia naturale (sottosezione \ref{LaTraditioDellOperaDiArchimedeDiApollonioEDiPappo}), che genera studi di fisica che favoriranno l'avvento della rivoluzione scientifica galileiana (sottosezione \ref{IlProblemaDeiCentriDiGravitaELOperaDiLucaValerio}).
\end{itemize}
\subsection{La cultura dell'abaco e dell'umanesimo.}\label{Storia_della_matematiLaCulturaDellAbacoEDellUmanesimo}
\input{Storia_della_matematica/LeonardoFibonacciELEScuoleDAbaco.tex}
\input{Storia_della_matematica/LUmanesimoEIlRecuperoDellaMatematicaGreca.tex}
\input{Storia_della_matematica/LInvenzioneDellaStampaELaDiffusioneDellaCulturaScientifica.tex}

\subsection{Dalla riappropriazione dei classici a nuovi orizzonti metodologici.}\label{DallaRiappropriazioneDeiClassiciANuoviOrizzontiMetodologici}
\input{Storia_della_matematica/LaTraditioDellOperaDiArchimedeDiApollonioEDiPappo.tex}
\input{Storia_della_matematica/IlProblemaDeiCentriDiGravitaELOperaDiLucaValerio.tex}
\input{Storia_della_matematica/FrancoisVieteELInvenzioneDellAlgebraSimbolica.tex}
\input{Storia_della_matematica/LaMeccanicaDaGiordanoAGalileo}


\section{La Nascita della Matematica Moderna.}\label{LaNascitaDellaMatematicaModerna}
\subsection{La teoria degli indivisibili di Bonaventura Cavalieri}\label{LaTeoriaDegliIndivisibiliDiBonaventuraCavalieri}
\paragraph{Bonaventura Cavalieri (1598 - 1647).} Bonaventura Cavalieri nasce presumibilmente a Milano nel 1598. Nel 1615 entra a far parte dell'ordine religioso dei Gesuati di San Girolamo. L'anno successivo l'ordine lo trasferisce a Pisa, dove Cavalieri impara la matematica sotto la guida di Benedetto Castelli, discepolo di Galileo, che gli presenta Galileo stesso.
\par Nel 1620 Cavalieri viene ritrasferito a Milano per studiare teologia. Gi\`a dal 1621 egli comincia per\`o a lavorare alla sua teoria degli indivisibili: nel 1627 viene pubblicata una prima versione della \textit{Geometria indivisibilibus}, con la quale si guadagna la stima di Galileo e l'amicizia di Torricelli.
\par Nel 1629 Cavalieri, con l'appoggio di Galileo, riesce ad aggiudicarsi una delle due letture di matematica a Bologna, dove rimane per il resto della sua vita, pubblicando, accanto ad i suoi importanti lavori di matematica, anche testi di astronomia e astrologia per volont\`a del senato bolognese. Nel 1641 lo svizzero Paolo Guldino (1577 - 1643) attacca ferocemente la teoria degli indivisibili nel quarto libro della \textit{Centrobaryca} (1641), con accuse sia di plagio nei confronti di Keplero e Sovero che di errori metodologici, spingendo Cavalieri a rispondere prima con un \textit{Dialogo}, ritirato su consiglio degli amici per la recente morte di Guldino, poi con le \textit{Exercitationes geometriae} (1647).
\paragraph{La teoria degli indivisibili -- Prima versione.} Per la sua teoria degli indivisibili Cavalieri si isipira al \textit{De centro} di Luca Valerio. Egli concepisce in momenti diversi due teorie diverse.
\par In un primo momento il tentativo \`e quello di introdurre una nuova grandezza geometrica: ``tutte le linee'' (\textit{omnes linae}) di una figura piana. L'idea \`e quella di tagliare il piano della figura con un altro piano libero di scorrere sul piano originale ma rimanendo sempre parallelo alla stessa direzione: i segmenti della figura cos\`i individuati sono ``tutte le linee'' della figura; analogamente sono definibili ``tutti i piani'' di un solido. I cosiddetti ``indivisibili'' sono dunque le suddette linee e i suddetti piani. Perch\'e questo oggetto venga effettivamente riconosciuto come una grandezza, \`e necessario che sia inquadrabile dalla teoria delle proporzioni descritta nel libro V degli \textit{Elementi} di Euclide: Cavalieri si cimenta dunque in una serie di dimostrazioni per provare questa inquadrabilit\`a, tuttavia le sue argomentazioni sono vaghe e finiscono generalmente per identificare ``tutte le linee'' di una figura con la figura stessa.
\par Nonostante i problemi fondazionali, i nuovi enti si dimostrano particolamente validi per affrontare il calcolo di aree e volumi, per esempio per calcolare il volume di nuovi solidi di rotazione introdotti da Keplero.
\par Vale la pena di segnalare un teorema importante sui rapporti tra solidi. Cavalieri definisce un solido \Define{simile} quando le sue sezioni verticali da una parte e le sue sezioni orizzontali dall'altra sono tutte simili tra loro: per esempio una piramide a base quadrata \`e un solido simile perch\'e tutte le sue sezioni verticali sono triangoli simili tra loro e tutte le sue sezioni orizzontali sono quadrati. Vi sono inoltre i solidi \Define{mutuamente simili}: si tratta di solidi simili con la stessa base, ma con sezioni verticali diverse. Cavalieri dimostra che il rapporto di solidi mutuamente simili non dipende dalla base, ma solo dai ``profili'' dei solidi, cio\`e dalle rispettive sezioni verticali massime.
\par Diamo un cenno della dimostrazione. Consideriamo due solidi mutuamente simili $A$ e $B$. Si consideri un ulteriore solido simile $A'$ con sezione orizzontale quadrata e ``profilo'' uguale a quello di $A$: se tagliamo $A$ e $A'$ con un piano orizzontale otteniamo figure piane simili alle rispettive basi. Il rapporto tra ciascuna sezione orizzontale di $A$ e la sezione orizzontale di $A'$ corrispondente \`e indipendente dal particolare piano orizzontale considerato: in effetti la sezione orizzontale di $A$ sta alla base di $A$  come la sezione orizzontale di $A'$ sta alla base di $A'$, e dunque le due sezioni stanno tra loro come le rispettive basi.
\par Ora, Cavalieri fa uso della propriet\`a del comporre delle proporzioni per dedurre che la somma di tutte le sezioni di $A$ sta alla somma di tutte le sezioni di $A'$ come le rispettive basi, e cio\`e che i solidi $A$ e $A'$ stanno tra loro nel medesimo rapporto. Ripetendo il ragionamento sul solido $B$ e utilizzando la stessa base usata per $A'$ per costruire $B'$, si deduce che ancora nello stesso rapporto stanno $B$ e $B'$. Quindi $A$ sta a $B$ come $A'$ sta a $B'$, cio\`e i due solidi di partenza stanno tra loro come i quadrati costruiti sulle basi delle rispettive sezioni massime.
\par Il teorema \`e molto importante perch\'e permette di generalizzare il risultato che stabilisce che il volume di una piramide \`e un terzo di quello del prisma con stessa base e stessa altezza, non solo al caso di cono e cilindro, ma a copiramidi con basi qualsiasi (per esempio ellitiche) e dei corrispondenti cilindprismi, risultato tutt'altro che banale. Tuttavia la dimostrazione evidenzia alcune difficolt\`a: la propriet\`a del comporre \`e applicata sommando un numero infinito di termini, senza nessuna legittimizzazione dell'operazione, e inoltre si afferma che la somma di ``tutti i piani'' sia il solido intero, di nuovo senza giustificazione.
\paragraph{La teoria degli indivisibili -- Seconda versione.} Cavalieri \`e consapevole dei difetti della sua teoria, che come abbiamo accennato gli vengono anche sottolineati pi\`u tardi da critici quali Guldino, \`e gi\`a mentre viene stampato il sesto libro della \textit{Geometria} ne prepara un settimo per superarle: ivi, egli evita discorsi sull'infinito, abbandonando anche l'introduzione degli indivisibili come nuovi enti e proponendo invece un metodo, detto \Define{metodo distributivo}, il quale consiste essenzialmente nell'applicazione di un teorema che afferma quanto segue: se due figure piane sono tagliate da un fascio di rette parallele in modo che i segmenti staccati sulle due figure stiano sempre tra loro nello stesso rapporto, allora il rapporto tra le due figure \`e il medesimo rapporto. Naturalmente vale un analogo enunciato per i solidi (quello che ancora oggi chiamiamo ``principio di Cavalieri''). Qui Cavalieri si muove in ampia generalit\`a, una generalit\`a mai vista prima, considerando solidi con buchi, convessit\`a, concavit\`a -- figure che non hanno pi\`u nulla a che vedere con l'ideale perfezione, cio\`e in questo caso la semplicit\`a, del cerchio o del quadrato degli antichi: egli tenta di ricondurre le sue dimostrazioni alla teoria delle figure digradanti di Valerio, si affida cio\`e all'autorit\`a di Valerio come Valerio si \`e affidato all'autorit\`a di Archimede.
\par Sia Valerio che Cavalieri dimostrano un gran coraggio nell'affrontare pionieristicamente problemi generali, studiando metodi generali, ma entrambi, come anche Galileo in campo fisico, si abbattono sullo stesso scoglio: la mancanza di un linguaggio adeguato. Essi sono tutti ancora troppo legati ad Archimede e al suo linguaggio basato sulla teoria delle proporzioni; per compiere il passo successivo, l'invenzione del calcolo infinitesimale, manca ancora il linguaggio giusto: la geometria analitica di Cartesio.

\subsection{La \textit{G\'eom\'etrie} di Ren\'e Descartes.}\label{LaGeometrieDiReneDescartes}
\paragraph{Il \textit{Discours de la m\'ethode} (1637).} La \textit{G\'eom\'etrie}, che descriveremo nei suoi caratteri principali nei paragrafi seguenti, costituisce un saggio facente parte del \textit{Discours de la m\'ethode} di Ren\'e Descartes (1596 - 1650), atto a dimostrare l'efficacia della \textit{m\'ethode}. Il \textit{Discours}  \`e un trattato sulla conoscenza secondo Descartes, la quale proviene per lui da idee chiare e distinte, che vengono utilizzate per costruire idee progressivamente pi\`u complesse. Nella \textit{G\'eom\'etrie}, le idee chiare e distinte sono sostanzialmente gli assiomi e le definizioni di partenza e il processo di costruzione di idee pi\`u complesse corrisponde al processo di costruzione di linee -- cio\`e curve -- pi\`u complesse, complessit\`a misurata dal \Define{genere} (\Cfr\ pi\`u avanti) della linea: il Cartesio geometra \`e dunque anche un po' inventore come Archimede. Abbiamo appena fatto riferimento al concetto di costruzione: in effetti, bench\'e sia proprio a partire dalla \textit{G\'eom\'etrie}, oltre che dai lavori di Fermat\footnote{Entrambi in questo allievi di Vi\`ete.}, che i matematici smetteranno di studiare curve date dal processo di costruzione e cominceranno a studiare curve date da un'equazione, Cartesio resta attaccato al concetto antico che identifica costruibilit\`a con esistenza. Come vedremo nel Libro I, egli calcola la lunghezza di segmenti, ma, alla luce delle sue idee chiare e distinte, la formula che ottiene per il calcolo equivale ad un processo di costruzione.
\par Il metodo che Descartes intende illustare diventa per\`o fonte di molteplici contraddizioni e difficolt\`a. Per prima cosa, come vedremo, le ``idee chiare e distinte'' su cui basa il suo trattato non hanno proprio nulla di chiaro: non \`e affatto chiaro perch\'e il prodotto di due segmenti dovrebbe essere definito in un modo piuttosto che in un altro; l'unica vera legittimizzazione della scelta delle definizioni del Libro I \`e costituita pragmaticamente dal loro successo. Il metodo cartesiano diventa dunque un involontario stratagemma per evitare di fornire scomode giustificazioni.
\par I concetti non vengono presentati n\'e sono presentabili coerentemente con l'ordine di complessit\`a previsto: l'intenzione \`e quello di introdurre per primi i segmenti come oggetto geometrico pi\`u semplice, tuttavia il calcolo della radice quadrata di un segmento richiede l'uso di un compasso, cio\`e della circonferenza, e le operazioni sugli oggetti pi\`u semplici si trovano ad essere definite in termini di oggetti pi\`u complessi.
\par Infine, Descartes si trova ad essere forse pi\`u schiavo del proprio programma che padrone: il suo metodo gli impone delle limitazioni oltre le quali egli non tenta neppure di andare. Per esempio, egli si occupa solo di linee \Define{geometriche} rinunciando a prescindere allo studio delle linee \Define{meccaniche} (\Cfr\ pi\`u avanti).
\paragraph{La \textit{G\'eom\'etrie} -- Libro I.} Il primo libro, come anticipato, si occupa di dare le definizioni di base sui segmenti e fornisce la prima prova dell'efficacia del metodo attraverso il problema di Pappo. Descartes vi fa ampio riferimento all'algebra simbolica di Vi\`ete, rinnovandola: \`e Descartes che stabilisce la convenzione ancora oggi in uso di utilizzare le prime lettere dell'alfabeto per le quantit\`a note e le ultime per le incognite; ma, soprattutto, \`e Descartes che rompe, una volta per tutte, con la \textit{regola di omogeneit\`a}.
\par Dagli antichi greci sino ai matematici del Cinquecento passando per il mondo arabo, \`e sempre stato assunto da tutti che il prodotto di due lunghezze fosse un'area e il prodotto di tre un volume; inoltre, lunghezze si sommano con lunghezze, aree con aree, volumi con volumi: \`e il contenuto della regola di omogeneit\`a, ribadita anche da Fran\c{c}ois Vi\`ete. Descartes invece, per la prima volta, costruisce un'algebra chiusa delle lunghezze: il prodotto di due segmenti, e dunque di un numero arbitrario di segmenti, \`e ancora un segmento e diventa dunque possibile sommare un segmento col prodotto di tre segmenti contro la regola di omogeneit\`a.
\par Le operazioni sui segmenti sono definite, dopo aver fissato un segmento unit\`a, in modo che la lunghezza del segmento risultante coincida col risultato che si otterrebbe applicando la stessa operazione alle lunghezze dei segmenti anzich\'e sui segmenti stessi. Diamo rapidamente tre esempi.
\par Definiamo il prodotto di due segmenti. Si dispongono i due segmenti in maniera che abbiano un estremo comune: per esempio nella figura \ref{Geometrie_Prodotto} abbiamo disposto i due fattori $OA$ e $OB$ -- i segmenti blu -- in modo tale che abbiano in comune l'estremo $O$. Si riporta su uno dei due fattori, per esempio $OA$, il segmento unit\`a con un estremo in $O$: otteniamo nel nostro caso il segmento $OU$. Tracciamo dunque la parallela ad $UB$ passante per $A$: essa incontra $OB$ nel punto $C$. Il segmento $OC$ \`e il prodotto di $OA$ per $OB$.
\begin{wrapfigure}{R}{0.5\textwidth}
	\includegraphics[width=0.48\textwidth]{Storia_della_matematica/Immagine_Geometrie_Prodotto.png}
	\caption{Definizione del prodotto di due segmenti.}
	\label{Geometrie_Prodotto}
\end{wrapfigure}
\par In effetti, i triangoli $OUB$ e $OAC$ sono simili ed abbiamo $OC:OA=OB:OU$, da cui $OC = \frac{OA \cdot OB}{OU} = OA \cdot OB$.
\par Analogamente possiamo definire la divisione di segmenti. Basandoci sulla medesima figura, la divisione di due segmenti si ottiene facendo coincidere due degli estremi -- ottenendo per esempio i segmenti $OA$ e $OC$ --, riportando su uno dei due segmenti il segmento unit\`a -- costruiamo $OU$ --, intersecando la parallela ad $AC$ passante per $U$ con $OC$ -- individuiamo $B$: il segmento $OB$, di lunghezza $OB = \frac{OC \cdot OU}{OA} = \frac{OC}{OA}$, \`e per definizione il rapporto dei segmenti $OC$ e $OA$.
\begin{wrapfigure}{R}{0.5\textwidth}
	\includegraphics[width=0.48\textwidth]{Storia_della_matematica/Immagine_Geometrie_Radice.png}
	\caption{Definizione della radice di un segmento.}
	\label{Geometrie_Radice}
\end{wrapfigure}
\par Diamo infine la definizione della radice di un segmento. Supponiamo di voler costruire la radice del segmento $AB$ (\Cfr\ figura \ref{Geometrie_Radice}): dobbiamo prolungare la retta $AB$ da una parte di un'unit\`a, ottenendo per esempio il segmento $UA$; tracceremo dunque una semicirconferenza di diametro $UB$ e intersecheremo la perpendicolare al diametro passante per $A$ con la semicirconferenza individuando il punto $R$. Il segmento $AR$ \`e per definizione la radice di $AB$. Abbiamo, per quanto riguarda le lunghezze, $AR = \sqrt{ \left ( \frac{UB}{2} \right ) ^2 - \left ( \frac{UB}{2} - UA \right )^2 } = \sqrt{ 2 \frac{UA \cdot UB}{2} - UA^2} = \sqrt{UB - 1} = \sqrt{AB}$.
\par Con l'introduzione dell'algebra chiusa dei segmenti (di cui ovviamente Descartes non scrive in termini cos\`i moderni), Descartes \`e capace di risolvere qualsiasi problema di costruzione con riga e compasso: l'idea \`e quella di utilizzare un metodo di analisi e sintesi (secondo la terminologia pappiana). Si suppone che la costruzione sia stata eseguita, si esprime la grandezza ignota che si desidera costruire in termini delle grandezze note, ottenendo un'equazione di grado al pi\`u $2$, la cui risoluzione \`e possibile, in termini delle grandezze note, con l'uso solo di somme, sottrazioni, prodotti, divisioni ed estrazioni di radici: la formula trovata fornisce dunque anche il metodo di costruzione.
\par Il trattato passa alla risoluzione del problema di Pappo. Sono date nel piano $n$ rette $l_i$ e altrettanti angoli $\phi_i$; dato un generico punto $P$, si denota $d_i$ la distanza di $P$ da $l_i$ calcolata lungo la retta passante per $P$ che compie con $l_i$ l'angolo $\phi_i$: si chiede di trovare tutti i punti $P$ tali che il rapporto del prodotto delle prime $\left \lfloor \frac{n}{2} \right \rfloor$ distanze $d_i$ con le restanti distanze $d_i$ sia uguale a un certo rapporto dato $\frac{\alpha}{\beta}$\footnote{Descartes, nel caso $n$ sia dispari, aggiunge a denominatore un'ulteriore distanza pari ad un'unit\`a, presumibilmente per calcolare un rapporto tra grandezze omogenee, a dispetto della rinuncia alla legge di omogeneit\`a vista prima: pu\`o darsi che egli non voglia estendere il suo nuovo punto di vista ai rapporti o che desideri riportare il problema di Pappo secondo lo stile antico.}. Prima della \textit{M\'ethode}, il problema era stato risolto solo in parte da Apollonio, per casi molto particolari: Apollonio era riuscito a dimostrare che il luogo descritto da $P$ nel caso in cui $n = 4$ \`e una sezione conica, senza saper descrivere quale precisamente; ora Descartes riesce a fornire una soluzione generale al problema, sebbene con alcune limitazioni.
\par La soluzione cartesiana prevede di fissare due rette principali: la prima retta \`e una delle rette date, per esempio $l_1$, e la seconda \`e la retta passante per $P$ che interseca $l_1$ con angolo $\phi_1$ -- chiamiamo $B$ il punto di intersezione e $r$ la retta. Descartes denota con $x$ la distanza $AB$, dove $A$ \`e un'intersezione di $l_1$ con un'altra delle rette date, e denota con $y$ la distanza $d_1$. Il processo \`e analogo ad aver fissato un sistema di assi cartesiani nel piano con origine in $A$ (gli assi non sono per\`o ortogonali).
\par A questo punto Descartes ricava, servendosi di considerazioni geometriche sui triangoli di cui conosce gli angoli per ipotesi o i lati per calcoli precedenti, una per una le distanze $d_i$ in termini di $x$ ed $y$: egli determina che ogni $d_i$ \`e somma di un termine lineare in $x$, di un termine lineare in $y$ e di un termine noto (il termine in $x$ o quello in $y$ pu\`o mancare se $l_i$ \`e parallela a $l_1$ o $r$ rispettivamente). Egli fissa ora un valore arbitrario per la $y$: la proporzione che definisce il problema diventa dunque un'equazione in $x$ di grado $\left \lfloor \frac{n}{2} \right \rfloor$, la cui risoluzione fornisce i punti $P(x,y)$ per il valore fissato di $y$.
\par Il problema di Pappo \`e dunque stato ricondotto al problema della soluzione di equazioni polinomiali, che Descartes non approfondisce ulteriormente, se non attraverso i metodi di costruzione del libro II, che per\`o hanno poco di algebrico o di geometrico e sono solamente metodi meccanici, in verit\`a di poca utilit\`a.
\par Concludiamo sul problema di Pappo che, se si sceglie di studiare il caso in cui vengono date due rette perpendicolari e gli angoli fissati sono retti, ritroviamo il ben noto piano cartesiano (e aggiungendo una terza retta arriviamo al classico spazio cartesiano).
\par Lo studio del problema di Pappo \`e coerente col metodo cartesiano: Descartes ha prima studiato i segmenti, che egli considera gli oggetti pi\`u semplici, ed ora passa al livello successivo di difficolt\`a, al luogo geometrico, definito tramite rette; in altre parole, Descartes ha inizialmente studiato problemi in un'incognita ed ora passa a problemi in due incognite. Il problema di Pappo suggerisce inoltre un metodo di classificazione dei problemi: noi moderni tendiamo a classificare le curve in funzione del loro grado, Descartes invece classifica i problemi (e non le curve) secondo le curve che sono necessarie per costruire le soluzioni delle equazioni che ne descrivono la soluzione. Ed \`e proprio a problemi di costruzioni e classificazione che \`e dedicato il secondo libro.
\paragraph{La \textit{G\'eom\'etrie} -- Libro II.} Il secondo libro \`e dedicato alla classificazione dei luoghi geometrici e alla loro costruzione. Le costruzioni, in linea col metodo cartesiano, prevedono l'utilizzo di macchinari tanto pi\`u complessi quanto pi\`u complessa \`e la linea da disegnare. Le macchine descritte da Descartes sono piuttosto complicate e probabilmente anche piuttosto scomode da usare: si tratta di macchine ideali, atte a dimostrare l'efficacia della \textit{m\'ethode}, piuttosto che di macchine veramente utili a disegnare linee che nessuno sente veramente la necessit\`a di costruire. La loro necessit\`a deriva dal fatto che il problema di Pappo descritto nel primo libro ha permesso di definire nuove linee, ma la loro costruzione punto per punto non si presta ad uno studio matematico efficace: occorre costruirle praticamente. Tuttavia il legame tra le macchine escogitate da Descartes e il problema di Pappo, non viene mai esplicitato, solo accennato, coerentemente con lo stile cartesiano.
\par Vediamo le due macchine descritte da Descartes: nelle descrizioni che seguono supponiamo tutti gli assi allungabili secondo necessit\`a (stando alle figure della \textit{G\'eom\'etrie}, Descartes invece li supponeva di lunghezza fissa arbitaria e che essi potessero ``uscire'' dallo strumento a volont\`a). La prima, esemplificata dalla figura \ref{Geometrie_Macchina1}, \`e una sorta di generalizzazione del comune compasso: essa \`e munita di due bracci, tracciati in colore blu, che si possono aprire; vi sono poi altri assi organizzati in modo tale
\begin{itemize}
	\item gli assi rossi e gli assi neri mantengono sempre le loro giunture, le quali sono libere di scorrere sui bracci;
	\item gli assi rossi si mantengono sempre perpendicolari al braccio superiore;
	\item gli assi neri si mantengono sempre perpendicolari al braccio inferiore.
\end{itemize}
\par Se manteniamo fisso il braccio inferiore e la prima giuntura sul braccio superiore, le giunture degli assi rossi col braccio superiore descrivono, partendo dalla circonferenza, curve sempre pi\`u complesse (cos\`i si esprime Descartes, senza precisare altro sulla crescente complessit\`a, soddisfatto della coerenza del macchinario col suo metodo). Sfruttando i triangoli simili nella macchina, \`e possibile costruire segmenti in particolari relazioni di medioproporzionalit\`a.
\begin{wrapfigure}{R}{0.5\textwidth}
	\includegraphics[width=0.48\textwidth]{Storia_della_matematica/Immagine_Geometrie_Macchina1.png}
	\caption{Una macchina descritta da Descartes.}
	\label{Geometrie_Macchina1}
\end{wrapfigure}
\par Passiamo alla seconda macchina: la descriviamo con l'aiuto della figura \ref{Geometrie_Macchina2}. Essa prevede
\begin{itemize}
	\item due assi perpendicolari neri;
	\item un asse blu legato agli assi neri in modo tale che la giuntura sull'asse orizzontale sia fissa e quella sull'asse verticale mobile;
	\item una curva rossa fissa che si abbassa o si alza in maniera solidale all'asse blu in modo che la loro distanza misurata sull'asse verticale resti costante.
\end{itemize}
\par La linea viene tracciata considerando il luogo dell'intersezione dell'asse blu con la linea rossa al ruotare dell'asse blu attorno al suo perno. Se la linea rossa \`e un'iperbole otteniamo la \Define{parabola di Cartesio}, che non \`e una parabola bens\`i la parte sinistra della curva di equazione $xy + x^3 + x^2 + x = 1$, che riportiamo in figura \ref{Geometrie_ParabolaCartesiana}.
\begin{wrapfigure}{R}{0.5\textwidth}
	\includegraphics[width=0.48\textwidth]{Storia_della_matematica/Immagine_Geometrie_Macchina2.png}
	\caption{Un'altra macchina descritta da Descartes.}
	\label{Geometrie_Macchina2}
\end{wrapfigure}
\par Sono presenti due nuove classificazioni dei problemi di costruzione geometrica (e quindi, indirettamente, delle curve), entrambe riconducibili alla prima illustrata nel Libro I. La prima classificazione riprende quella greca, la quale distingue tra problemi
\begin{itemize}
	\item \Define{piani}, che richiedono l'uso della riga e del compasso;
	\item \Define{solidi}, che richiedono inoltre l'uso di almeno una sezione conica;
	\item \Define{lineari}, che richiedono inoltre l'uso di qualche altra linea, detta linea \Define{meccanica}.
\end{itemize}
\par Descartes spezza per\`o la classe delle line meccaniche in due classi:
\begin{itemize}
	\item le linee \Define{geometriche}, le uniche accettabili in geometria, che si costruiscono per mezzo di meccanismi del tipo da lui descritto, in cui si collegano rette o curve create direttamente o indirettamente da rette, le quali si muovono continuamente e sincronicamente. Descartes ritiene, a ragione, che tutti i luoghi pappiani siano costruibili per mezzo di linee geometriche e, a torto, che tutte le linee geometriche siano luoghi pappiani;
	\item le linee \Define{meccaniche} (sottoinsieme dunque delle meccaniche dei greci), che si ottengono tramite la combinazione di movimenti indipendenti, come la spirale o la quadratrice.
\end{itemize}
\par Come si vede, la classificazione cartesiana appena descritta \`e puramente geometrica. Descartes propone una seconda classificazione, questa volta algebrica, basandosi sulla descrizione di una linea per mezzo di un'equazione in due incognite raggiunta tramite il problema di Pappo; si tratta dunque di una classificazione delle linee geometriche. Egli definisce il concetto di \Define{genere}: in termini moderni, possiamo dire che il genere di una linea \`e la parte intera della met\`a del suo grado aumentato di $1$. Ma al solito, Descartes, pur fornendo effettivamente, nel linguaggio che gli \`e proprio, la suddetta definizione, ragiona comunque in termini di costruibilit\`a e di complessit\`a progressiva e stabilisce dunque che le linee di genere $1$ sono quelle costrubili con riga, compasso e sezioni coniche, quelle di genere $2$ necessitano di qualche linea pi\`u complicata, quella di genere $3$ di altre linee ancora e cos\`i via, senza maggiori precisazioni.
\par Vi sono diversi motivi per cui Descartes considera il genere di una curva anzich\'e il suo grado:
\begin{wrapfigure}{R}{0.5\textwidth}
	\includegraphics[width=0.48\textwidth]{Storia_della_matematica/Immagine_Geometrie_ParabolaCartesiana.png}
	\caption{La parabola cartesiana.}
	\label{Geometrie_ParabolaCartesiana}
\end{wrapfigure}
\begin{itemize}
	\item la seconda macchina descritta gli ha suggerito, erroneamente, che fosse vero in generale che partendo da una curva di genere $a$ essa costruisse una curva di genere $a + 1$;
	\item egli conosceva una regola, probabilmente la regola oggi attribuita a Ferrari, per abbassare il grado di un'equazione di grado $4$ a quella di un'equazione di grado $3$;
	\item egli ha osservato che aggiungendo la parabola all'insieme delle linee costrubili erano risolubili tutti i problemi di genere $1$ e aggiungendovi la parabola cartesiana tutti i problemi di genere $2$.
\end{itemize}
\paragraph{La \textit{G\'eom\'etrie} -- Libro III.} Descartes non ignora per\`o neanche la classificazione della complessit\`a di una curva secondo il grado della sua equazione: espone questa possibilit\`a nel Libro III. Il terzo libro \`e dedicato alla descrizione dei metodi di risoluzione teorica dei problemi. Si insegna che per attaccare un problema bisogna riuscire ad esprimerlo in termini algebrici secondo quanto visto nel libro I, quindi bisogna abbassare il grado dell'equazione che ne risulta il pi\`u possibile e servirsi di strategie canoniche di risoluzione di queste equazioni. Non mancano considezioni geometriche tuttavia e vi si spiega per esempio come calcolare la normale e dunque la tangente ad una curva (\Cfr\ sottosezione \ref{IMetodiDiDescartesEDiFermat}).
\paragraph{La \textit{G\'eom\'etrie} -- Influenza.} Nel corso della descrizione della \textit{G\'eom\'etrie} abbiamo accennato ad alcuni dei suoi limiti che ora esplicitiamo:
\begin{itemize}
	\item Descartes vede sempre l'algebra come ausiliare alla geometria: le curve restano sempre le uniche protagoniste della teoria e lo scopo \`e la loro costruzione. Egli non compie, al contrario di Fermat, il passo di identificare la curva con la sua equazione;
	\item nonostante l'evidente ricerca di metodi generali, efficaci ma anche semplici, egli non pensa mai a fissare la convenzione degli assi ortogonali;
	\item il suo metodo della tangente non si avvicina minimamente ai concetti del calcolo infinitesimale, e questo di nuovo al contrario di quello di Fermat.
\end{itemize}
\par La \textit{G\'eom\'etrie} \`e vista inizialmente come un testo molto complesso, ma si impone comunque rapidamente grazie soprattutto alle numerose edizioni tradotte in lingua latina ampiamente commentate di van Schoten (1615 - 1660). Molti matematici, principalmente francesi, olandesi ed inglesi, iniziano a studiare da un punto di vista geometrico sistematicamente equazioni in una o due incognite di gradi diversi. Vale la pena inoltre citare De Witt (1625 - 1672) che negli \textit{Elementa curvarum linearum} (1646) per la prima volta dichiara esplicitamente che un'equazione di primo grado in $x$ ed $y$ \Define{\`e} una retta.

\subsection{La nascita delle accademie e delle riviste scientifiche.}\label{LaNascitaDelleAccademieEDelleRivisteScientifiche}
\paragraph{La nascita delle accademie.} Le accadamie sorgono in Europa nella seconda met\`a del XVII secolo. Si tratta di associazioni, nel caso di nostro interesse di scienziati, che lavorono insieme per promuovere la ricerca. Sino al XV secolo gli scienziati erano individui, lavoravano essenzialmente da soli, eccezion fatta per qualche scambio epistolare, e ognuno aveva i propri metodi, le proprie credenze, i propri interessi: tutto questo cambia con le accademie. Queste nuove comunit\`a scientifiche vengono a costituire una sorta di ``scienziato collettivo'': ci\`o che conta non sono pi\`u le teorie dei singoli scienziati, bens\`i quello che la comunit\`a scientifica accetta o respinge.
\par Accademie significative nate nel XVII secolo sono
\begin{itemize}
	\item l'\textit{Accademia dei Nazionale dei Lincei} italiana nata (in Umbria) nel 1603;
	\item la \textit{Royal Society} inglese nata nel 1661;
	\item l'\textit{Acad\'emie des Sciences} francese nata nel 1666.
\end{itemize}
\par Alcune accademie sono pubbliche, altre private, altre ancora nascono private e diventano pubbliche: quest'ultimo \`e il caso della \textit{Royal Society} nata da un gruppo di studiosi nel 1645 presso il Gresham College.
\paragraph{La nascita delle riviste scientifiche.} Simultaneamente alle accademie nascono le prime riviste scientifiche. Abbiamo inizialmente
\begin{itemize}
	\item resoconti dell'attivit\`a delle accademie: \`e il caso delle \textit{Philosophical Transactions of the Royal Society} (1665) e dei \textit{M\'emoires de l'Acad\'emie Royale des Sciences};
	\item pubblicazioni private: per esempio il \textit{Journal des S\c{c}avans} (1666);
	\item raccolte di recensioni: prima il \textit{Giornale de' Letterati} (1668 - 1680 a Roma, ricomparso il secolo successivo in varie forme e varie citt\`a), ma poi, soprattutto gli \textit{Acta Eruditorum} (Lipsia, 1682).
\end{itemize}
\par Tuttavia la comunit\`a scientifica si accorge rapidamente delle nuove possibilit\`a offerte dalla pubblicazione delle riviste e inizia a utilizzarle per comunicare risultati nuovi e originali, in maniera pi\`u semplice di quella che avevano consentito fino ad allora i volumi comuni: un esempio molto significativo \`e quello della memoria di Leibniz fondatrice del calcolo infinitesimale.
\par Le riviste scientifiche contribuiscono in maniera decisiva all'affermazione del cosiddetto scienziato collettivo.

\subsection{Il problema delle tangenti: i metodi di Descartes e di Fermat.}\label{IMetodiDiDescartesEDiFermat}
\par Uno dei problemi pi\`u antichi \`e quello delle tangenti\footnote{\CFR\ sottosezione \ref{LaGeometriaDiPosizione}.}: nel corso del Seicento molti matematici trovano nuovi metodi per calcolare tangenti, ma i due principali, dai quali prendono le mosse quelli successivi sono quelli di Descartes e di Fermat.
\paragraph{Il metodo di Descartes.} Il metodo di Descartes consiste nel trovare una circonferenza tangente alla curva nel punto dato, cio\`e che abbia con la curva un'intersezione doppia; di solito la circonferenza viene cercata col centro sull'asse delle ascisse. Trovata la circonferenza, la tangente sar\`a la retta perpendicolare al raggio passante per il punto voluto.
\par Supponiamo di voler cercare la tangente nel punto $(x_0,y_0)$ alla curva $P(x,y) = 0$. La circonferenza ha equazione $(x - v)^2 + y^2 = r^2$, con $v$ e $r$ da determinare. Si mettono a sistema le due equazioni eliminando la $y$; se $P(x,y)$ \`e di grado $n$, allora l'equazione $Q(x) = 0$ che ne risulta \`e di grado $2n$ (e questo gi\`a suggerisce la complessit\`a dei calcoli previsti dal metodo). Si impone allora che $Q(x)$ sia della forma $(x - x_0)^2 R(x)$, con $R(x)$ polinomio di grado $2n - 2$ e eguagliando i coefficienti delle potenze di $x$ si trovano $v$ ed $r$.
\par Diamo un esempio di applicazione nel caso della parabola $y = mx^2$, mostrando quanto complicati possano essere i calcoli anche nei casi pi\`u semplici. Il sistema tra la parabola e la circonferenza ci conduce all'equazione $x^2 - 2vx + v^2 + m^2x^4 - r^2 = 0$. Il primo membro deve essere uguale a $(x - x_0)^2(ax^2 + bx + c)$, che \`e uguale a $ax^4 + bx^3 + cx^2 - 2x_0ax^3 - 2x_0bx^2 -2x_0cx + x_0^2ax^2 + x_0^2bx + x_0^2c = ax^4 + (b - 2x_0a)x^3 + (c - 2x_0b + x_0^2a)x^2 + (x_0^2b - 2x_0c)x + x_0^2c$.
\par Eguagliamo i coefficienti e risolviamo il sistema (per sostituzione ovviamente: Descartes non conosceva certo le matrici; in questo caso comunque il sistema \`e abbastanza facile).
\begin{align}
	&\begin{cases}
		a = m^2,\\
		b - 2 x_0a = 0,\\
		c - 2 x_0b + x_0^2a = 1,\\
		x_0^2b - 2 x_0c = - 2v,\\
		x_0^2c = v^2 - r^2,
	\end{cases}\nonumber\\
	&\begin{cases}
		a = m^2,\\
		b = 2x_0m^2,\\
		c = 1 + 3x_0^2m^2,\\
		v = - \frac{x_0^2b - 2 x_0c}{2} = - \frac{2x_0^3m^2 - 2x_0 - 6x_0^3m^2}{2} = x_0 + 2x_0^3m^2,\\
		r^2 = v^2 - x_0^2c = x_0^2 + 4x_0^4m^2 + 4x_0^6m^2 - x_0^2 - 3 x_0^4m^2 = x_0^4m^2 + 4x_0^6m^2.
	\end{cases}\nonumber
\end{align}
\par Il problema a questo punto pu\`o considerarsi risolto (il calcolo della perpendicolare al raggio \`e banale), ma con un po' di geometria (\Cfr\ figura \ref{Tangenti_Descartes}) possiamo ricavare il risultato classico, secondo cui per costruire la tangente \`e sufficiente riportare l'ordinata del punto di tangenza sull'asse, dalla parte opposta rispetto al vertice, e tracciare la retta che unisce il punto cos\`i individuato col punto di tangenza.
\begin{wrapfigure}{R}{0.5\textwidth}
	\includegraphics[width=0.48\textwidth]{Storia_della_matematica/Immagine_Tangenti_Descartes.png}
	\caption{Dimostrazione della propriet\`a classica della tangente ad una parabola secondo Descartes.}
	\label{Tangenti_Descartes}
\end{wrapfigure}
\par I triangoli $ABT$ e $TBC$ sono entrambi rettangoli e gli angoli $ATB$ e $TCB$ sono entrambi complementari dell'angolo $BTC$ e dunque uguali: i triangoli $ABT$ e $TBC$ sono dunque simili. Per similitudine, $BT:BC=AB:BT$, da cui $AB = \frac{BT^2}{BC} = \frac{y_0^2}{v - x_0} = \frac{m^2x_0^4}{2x_0^3m^2} = \frac{x_0}{2}$; quindi $A$ \`e il punto medio di $OB$. Ne consegue che i triangoli rettangoli $ODA$ e $ABT$, i cui angoli $OAD$ e $TAB$ sono uguali, sono congruenti. Quindi $OD = BT$, che \`e il risultato classico.
\paragraph{Il metodo di Fermat.} Fermat concepisce un proprio metodo per il calcolo delle tangenti, indipendentemente da Descartes, basandosi sul proprio metodo per il calcolo dei massimi e dei minimi di una grandezza variabile\footnote{Il concetto di funzione non \`e ancora stato inventato: siamo ancora nell'ambito dell'algebra simbolica di Fran\c{c}ois Vi\`ete, \Cfr\ sottosezione \ref{FrancoisVieteELInvenzioneDellAlgebraSimbolica} e, come Vi\`ete e Fermat, useremo vocali per denotare quantit\`a ignote e consonanti per quantit\`a note}: cominciamo proprio da quest'ultimo.
\par Sia $F$ una una grandezza variabile e cerchiamone il massimo $M$; Fermat sottointende di cercare un massimo locale, che $F$ sia continua e di lavorare in un intorno in cui il massimo locale \`e unico. Per considerazioni che risalgono a Pappo, scelto $Z < M$ a destra e a sinistra del massimo esistono due punti $A$ ed $E$ per cui $F(A) = F(E) = Z$; inoltre, avvicinando $Z$ a $M$, i punti $A$ ed $E$ si avvicineranno sempre di pi\`u, pertanto, l'equazione $\frac{F(A) - F(E)}{A - E} = 0$ che vale per $Z < M$ fonir\`a una nuova equazione, valida quando $Z = M$, ponendo $E = A$ dopo aver effettuato le dovute semplificazioni (altrimenti avremmo una divisione per zero): risolvendo l'equazione in $A$, si trover\`a il punto in cui $F$ assume il suo massimo $M$.
\par Fermat osserva per\`o che l'operazione di divisione per il binomio $A - E$ risulta nei casi pratici piuttosto complicata, egli modifica dunque il proprio metodo prescrivendo di sostituire alla variabile $E$ il binomio $A + E$. L'equazione $\frac{F(A) - F(E)}{A - E} = 0$ diventa dunque $\frac{F(A) - F(A + E)}{E} = 0$ e la divisione per $E$ \`e molto pi\`u facile: a questo punto si pone $A + E = A$, cio\`e $E = 0$. Tuttavia il passaggio non \`e un semplice cambio di notazioni come pu\`o sembrare a prima vista, vi sono anche importanti differenze concettuali implicite: nel metodo originale $A$ ed $E$ erano entrambe grandezze indipendenti, ora invece $A$ \`e una grandezza indipendente ed $E$ \`e una variazione della grandezza $A$; nel metodo originale $A$ aveva inizialmente un valore a cui non corrispondeva il massimo di $F$ e poi il valore corrispondente al massimo, ora invece $A$ ha da subito il valore corrispondente al massimo ed $A + E$ viene a coincidere con $A$ azzerando $E$, cio\`e spostando il solo punto $A + E$; nel metodo originale, prima del ``passaggio al limite'', l'equazione $\frac{F(A) - F(E)}{A - E}$ era vera, ora invece la relazione $\frac{F(A) - F(A + E)}{E} = 0$ \`e vera solo ``al limite'' e in effetti Fermat si riferisce ad essa chiamandola non \textit{\'egalit\'e} o \textit{\'equation}, bens\'i \textit{ad\'egalit\'e}.
\par Alcuni storici hanno visto nel metodo di Fermat un'anticipazione dell'invenzione della derivata, ma ci\`o \`e vero solo sino ad un certo punto: la forza del concetto di derivata risiede nelle regole generali che consentono di ottenere la derivata di una funzione combinando opportunamente derivate di funzioni pi\`u semplici; Fermat invece intuisce solo il concetto di derivata in un punto come limite del rapporto incrementale e cos\`i, per cercare un massimo, si vede costretto ad eguagliare a $0$ un complicato limite anzich\'e una semplice derivata.
\par Tornando al problema delle tangenti, Fermat sostiene che le adeguaglianze possano essere applicate con successo anche in questo caso. Vediamo come con un esempio: il calcolo della tangente ad una parabola (\Cfr\ figura \ref{Tangenti_Fermat}).
\begin{wrapfigure}{R}{0.5\textwidth}
	\includegraphics[width=0.48\textwidth]{Storia_della_matematica/Immagine_Tangenti_Fermat.png}
	\caption{Dimostrazione della propriet\`a classica della tangente ad una parabola secondo Fermat.}
	\label{Tangenti_Fermat}
\end{wrapfigure}
\par Calcolare la tangente in un punto $T$ in questo caso significa calcolare la lunghezza del segmento $VP$. Per le propriet\`a della parabola, $VB:VC=RB^2:TC^2$. Ora, sostituiamo nella proporzione $SB$ a $RB$, ottenendo $VB:VC < SB^2:TC^2$. Per la similitudine tra triangoli, abbiamo $SB:TC=PB:PC$ e dunque $VB:VC < PB^2:PC^2$. Ora, Fermat vede che spostando il punto $B$ verso $C$ la disuguaglianza appena trovata diventa un'uguaglianza, dunque pone l'\textit{ad\'egalit\'e} $VB:VC = PB^2:PC^2$.
\par Poniamo l'\textit{ad\'egalit\'e} in termini algebrici: poniamo $PV = A$, $CB = E$, $CV = K$. Abbiamo allora
\begin{align}
	&(K - E):K=(A + K - E)^2:(A + K)^2,\nonumber\\
	&(K - E)(A + K)^2=(A + K - E)^2 K,\nonumber\\
	&(K - E)(A^2 + 2AK + K^2)=A^2K + K^3 - E^2K + 2AK^2 - 2 AEK - 2EK^2,\nonumber\\
	&A^2K + 2AK^2 + K^3 - A^2E - 2AEK - EK^2=A^2K + K^3 - E^2K + 2AK^2 - 2 AEK - 2EK^2,\nonumber\\
	&- A^2E = - E^2K - EK^2,\nonumber\\
\end{align}
\par Dividiamo per $E$ e otteniamo $A^2 = EK + K^2$; poniamo $E = 0$ ed arriviamo a $A^2 = K^2$ e dunque $A = K$, che il risultato classico ben noto.
\par Naturalmente, se lo si desidera, \`e possibile risolvere il problema tramite il metodo di massimi e minimi vero e proprio (la grandezza $\frac{VB}{SB^2}$ ha un massimo per $B = C$), ma Fermat non compie questo passaggio: esso appesantirebbe solamente la dimostrazione e probabilmente egli non lo concepisce nemmeno ma pensa direttamente ad applicare in ambito geometrico il concetto di adeguaglianza.

\subsection{Il problema delle tangenti e il calcolo differenziale di Leibniz.}\label{IlCalcoloDifferenzialeDiLeibniz}
\par Come abbiamo visto nella sottosezione \ref{IMetodiDiDescartesEDiFermat}, nel Seicento cominciano a farsi strada diversi metodi per il calcolo delle tangenti, in particolare il metodo di Descartes e quello di Fermat, dai quali derivano varie regole per trovare le tangenti ai polinomi: questi ultimi si ispirano prevalentemente all'impostazione di Fermat, ma la loro preferenza per i polinomi \`e chiaramente di origine cartesiana. Tutti questi metodi hanno per\`o l'inconveniente di diventare rapidamente molto complessi, troppo complessi per essere portati fino in fondo, a meno di munirsi di molto tempo e di molta pazienza: basta introdurre un radicale o considerare una funzione fratta per andare incontro a complicate operazioni, consistenti appunto nella rimozioni dei radicali e delle frazioni. Il metodo di Leibniz invece intoduce i \Define{differenziali} (che lui chiama \Define{differenze}, ma noi usiamo il termine moderno), i quali consentono di spezzare il problema complesso in tanti piccoli problemi pi\`u semplici.
\par Il calcolo differenziale di Gottfried Wilhelm Leibniz (1646 - 1716) nasce nel 1684 con la memoria \textit{Nova methodus pro maximis et minimis, itemque tangentibus, quae nec fractas nec irractionalas quantites moratur, et singulare pro illis calculi genus}, pubblicata sugli \textit{Acta Eruditorum}. Leibniz fissa un riferimento cartesiano con assi ortogonali e vi considera una generica curva $\gamma$: l'asse delle ordinate viene scelto comodamente passante per il punto di $\gamma$ in cui si desidera definire i differenziali. Egli sceglie il differenziale $dx$ come un segmento orizzontale arbitrario, quindi stabilisce che il differenziale $dy$ corrispondente \`e il segmento che verifica la proporzione $dy : dx = OP : OT$ (\Cfr\ figura \ref{Leibniz_Differenziale}).
\begin{wrapfigure}{R}{0.5\textwidth}
	\includegraphics[width=0.48\textwidth]{Storia_della_matematica/Immagine_Leibniz_Differenziale.png}
	\caption{Definizione del differenziale di Leibniz.}
	\label{Leibniz_Differenziale}
\end{wrapfigure}
\par \`E importante notare che i differenziali, sebbene nella teoria di Leibniz abbiano lo stesso carattere infinitesimo che gli attribuiamo oggi, sono grandezze finite. Tuttavia, questo punto importante viene contraddetto indirettamente pi\`u avanti, quando Leibniz definisce la tangente ad una curva, retta che come abbiamo visto interviene nella definizione dei differenziali: la tangente \`e quella retta che congiunge due punti infinitamente vicini della curva, ovvero il lato del poligono ``ad infiniti lati'' che equivale alla curva. Chiaramente, Leibniz ha intuito un concetto di limite, ma lo considera solo da un punto di vista geometrico, mentre mantiene l'algebra in un ambito finito.
\par E proprio con l'algebra prosegue la memoria, dopo aver illustrato brevemente alcuni usi del differenziale (studio della crescenza di una curva, dei suoi massimi e minimi relativi, della concavit\`a): Leibniz compie il passo rivoluzionario di ricavare le regole a noi ben familiari di calcolo per il differenziale della somma di grandezze, del prodotto, del rapporto, della radice, delle potenze. Tutto ci\`o gli consente di esibirsi nella ricerca di quei massimi e minimi che prima erano inaccessibili, esibizione che egli ripete pi\`u volte anche nei suoi scambi epistolari per dimostrare l'efficacia dei suoi differenziali.
\par L'introduzione dei differenziali ha per naturale conseguenza la ricerca di grandezze i cui differenziali sono in una data relazione o, per dire la stessa cosa in termini pi\`u familiari, lo studio di equazioni differenziali. Ci vuole del tempo perch\'e si imponga il teorema fondamentale del calcolo integrale, ma Evangelista Torricelli (1608 - 1647) e Isaac Barrow (1630 - 1677) riescono ad anticiparlo abbastanza presto (tanto presto da precedere entrambi la memoria leibniziana: ambedue morirono prima della sua pubblicazione), motivo per cui il teorema \`e intitolato in loro onore.
\par Segnaliamo infine l'importanza delle notazioni Leibniziane: la $d$ del differenziale si presta bene ad essere interpretata come un'operazione, anche se la definizione del differenziale stesso non lo rende subito evidente. Questa ed altre notazioni, come il simbolo $\int$ di integrale, sono dovuti a Leibniz ed in uso ancora oggi.

\subsection{Serie e flussioni}\label{SerieEFlussioni}
\par Isaac Newton (1642 - 1727) \`e l'altro grande inventore del calcolo infinitesimale oltre a Leibniz. Il metodo di Newton prevede due concetti fondamentali: quello delle \Define{flussioni}, analogo delle nostre derivate, e quello degli sviluppi in serie. Newton considera questi due metodi l'uno parte integrante dell'altro, a tal punto che, nella famosa disputa con Leibniz, egli accusa il filosofo tedesco di voler spezzare il suo lavoro in due per potersi appropriare di una parte di esso. La Storia, tuttavia, dimostra che l'atteggiamento pi\`u fecondo \`e quello di Leibniz: mentre i matematici inglesi riterranno la nuova teoria del calcolo infinitesimale completa grazie agli sviluppi in serie, sul continente invece il nuovo strumento dato dal differenziale diventa il protagonista di nuovi progressi, suggerendo lo studio delle equazioni differenziali e manifestando progressivamente la relazione tra derivazione e integrazione.
\paragraph{Flussioni.} Newton ragiona in termini cartesiani e meccanici. Egli vede un'equazione nelle incognite $x$ ed $y$ non come una curva nel piano, ma come due grandezze variabili (\Define{fluenti}) nel tempo: le loro \Define{flussioni}, cio\`e quelle che noi chiamiamo derivate, si indicano $\dot{x}$ e $\dot{y}$; la notazione \`e in uso ancora oggi, soprattutto nella derivazione rispetto al tempo di funzioni meccaniche. Le flussioni risolvono anche il problema della tangente ad una curva, per il noto risultato secondo cui la velocit\`a \`e appunto tangente alla traiettoria; se si desidera conoscere la direzione della tangente, \`e sufficiente calcolare il rapporto delle flussioni\footnote{In notazioni Leibniziane, $\frac{\frac{dx}{dt}}{\frac{dy}{dt}} = \frac{dx}{dt} \frac{dt}{dy} = \frac{dx}{dy}$.}.
\par A differenza di Leibniz, Newton, facendo appello al concetto di velocit\`a, vago ma intuitivo all'epoca, riesce a non introdurre mai grandezze infinitesime nei suoi enunciati, sebbene esse compaiano nelle sue dimostrazioni. Diamo un esempio con la derivazione della regola per il calcolo della flussione di un prodotto, che naturalmente \`e analoga per Leibniz: dato un tempo infinitesimo $o$, le grandezze $x$ ed $y$ aumentano nel tempo $o$ di $o\dot{x}$ e $o\dot{y}$ (Newton chiama quest'ultime grandezze \Define{momenti} di $x$ ed $y$), abbiamo che il momento di $xy$ \`e $(x + o\dot{x})(y + o\dot{y}) - xy = x \cdot o\dot{y} + y \cdot o\dot{x} + ox \cdot oy = o(x\dot{y} + \dot{x}y) + o^2 xy = o(x\dot{y} + \dot{x}y)$ visto che $o^2 xy$ \`e infinitesimo\footnote{Diremmo noi infinitesimo di ordine superiore rispetto ad $o$.}; la flussione \`e allora il rapporto del momento col tempo, cio\`e $x\dot{y} + \dot{x}y$.
\paragraph{Serie.} Alcuni matematici, come Mercatore (1512 - 1594) e Gregory (1638 - 1675), avevano dato alcuni sviluppi in serie delle funzioni del tempo, ma Newton riesce a darne molti di pi\`u: sfruttando svariate tecniche che vanno dall'integrazione termine a termine all'interpolazione esaurisce l'insieme delle funzioni considerate all'epoca.
\par Prima di Newton, salvo eccezioni, le serie venivano viste come utili mezzi di approssimazione; egli invece le usa come base per la sua nuova teoria:
\begin{itemize}
	\item per la ricerca della tangente ad una curva, basta ricavare la serie corrispondente ed applicarvi uno dei tanto metodi disponibili per i polinomi, come quelli di Hudde (1628 - 1704) o Sluse (1622 - 1685);
	\item ogni quadratura di una funzione viene sistematicamente ricondotta alla quadratura termine a termine della serie corrispondente alla funzione stessa;
	\item si possono anche risolvere equazioni differenziali come $\dot{y} = 1 - 3x + y + x^2 + xy$, dove si suppone $y(0) = 0$ (esempio trattato nel \textit{Methodus fluxionum et serierum infinitarum}): il metodo procede prendendo nel secondo membro tutti i termini di grado $0$; otteniamo $\dot{y} = 1$, da cui $y = x$; sostituiamo quanto trovato e consideriamo solo i termini di grado $1$; otteniamo $\dot{y} = 1 - 2x$, da cui $y = x - x^2$; procedendo grado per grado, sostituendo ogni volta, otteniamo uno sviluppo in serie di $y$.
\end{itemize}



	\part{Appunti di corsi non integrati nelle parti precedenti}
	\chapter{Finanza matematica}
	\section{Prime definizione e primi teoremi.}\label{PrimeDefinizioniEPrimiTeoremi}
\par In questa prima sezione descriviamo gli strumenti finanziari con cui lavoreremo e presentiamo un paio di teoremi, senza per il momento introdurre un modello matematico preciso.
\begin{Definition}\label{AttivoIntuitivo}
	Si chiama \Define{strumento finanziario} (anche \Define{attivit\`a finanziaria} o \Define{attivo finanziario}) un qualsiasi oggetto soggetto ad acquisto e vendita. Gli strumenti finanziari vengono scambiati nei \Define{mercati}: si chiama \Define{mercato primario} quello in cui \`e acquistabile per la prima volta lo strumento, \Define{mercato secondario} quello in cui \`e acquistabile successivamente (se i due mercati coincidono non si adotta nessuna distinzione e si parla solo di \Define{mercato}).
\end{Definition}
	Esempi di strumenti finanziari sono titoli di Stato, azioni, merci (tipicamente materie prime), valute... Invitiamo il lettore a familiarizzarsi con le varie tipologie di strumenti finanziari. Un esempio importante di distinzione tra mercato primario e mercato secondario si ha con i titoli di Stato, i quali sono (per lo meno in Italia) emessi dal Tesoro e messi in vendita in apposite aste (mercato primario), dopo le quali sono scambiabili in Borsa (mercato secondario).
\begin{Definition}
	Si chiama \Define{strumento finanziario derivato} qualsiasi strumento finanziario che dipenda da un altro strumento finanziario, detto \Define{sottostante}, per la sua emissione e il suo funzionamento (e dunque necessariamente anche per il suo valore).
\end{Definition}
	\par Presentiamo i derivati che studieremo nel corso di queste note.
\begin{Definition}
	Si chiama \Define{opzione} uno strumento finanziario definito dai seguenti parametri:
	\begin{itemize}
		\item il \Define{sottostante}: un qualsiasi strumento finanziario (di solito non derivato);
		\item il \Define{tipo}: l'opzione si dice \Define{d'acquisto} (\Call) se fornisce la possibilit\`a (ma non l'obbligo) di acquistare il sottostante, \Define{di vendita} (\Put) se fornisce la possibilit\`a (ma non l'obbligo) di vendere il sottostante;
		\item il \Define{prezzo di esercizio} (\Strike): il prezzo al quale viene eventualmente acquistato o ceduto il sottostante. Esso \`e fissato al momento dell'emissione dell'opzione: proprio in questo fatto risiede l'interesse dello strumento;
		\item la \Define{scadenza}: il periodo nel quale (opzione \Define{europea}) o entro il quale (opzione \Define{americana}) l'opzione \`e esercitabile.
	\end{itemize}
\end{Definition}
	\par Salvo avvisi contrari, considereremo solo opzioni europee.
	\par Le opzioni, cos\`i come tutti gli strumenti derivati, pongono due importanti problemi:
\begin{itemize}
	\item la \Define{valutazione} (\Pricing): consiste nel determinare quale sia il giusto prezzo per la sottoscrizione di un'opzione;
	\item la \Define{copertura} (\Hedging): \'e il problema, per chi ha emesso l'opzione, che consiste nell'essere capace di effettuare effettivamente l'operazione nel caso il sottoscrittore lo richieda.
\end{itemize}
	\par Presenteremo ora due teoremi che riguardano la valutazione delle opzioni, ma per fare ci\`o abbiamo prima bisogno di definire un altro paio di concetti e di dimostrare un Lemma.
\begin{Definition}\label{PortafoglioIntuitivo}
	Si chiama \Define{portafoglio} un qualsiasi insieme di strumenti finanziari (di solito si usa quest'espressione per indicare l'insieme di tutti gli strumenti in possesso alla stessa persona o alla stessa azienda). Il \Define{valore del portafoglio} al tempo $t$ \`e la somma di denaro che si otterrebbe vendendo tutte le attivit\`a finanziarie simultaneamente al tempo $t$.
\end{Definition}
\begin{Definition}\label{ArbitraggioIntuitivo}
	Si chiama \Define{arbitraggio} un'operazione finanziaria che consente di guadagnare denaro (o quantomeno di non perderne) senza l'uso di nessun capitale iniziale.
\end{Definition}
\begin{Definition}
	Si chiama \Define{vendita allo scoperto} la vendita di strumenti finanziari di cui non si dispone e che si prendono dunque in prestito: alla scadenza del prestito, essi devono essere ricomprati da chi ha eseguito la vendita (non necessarimente dallo stesso compratore) e resi al prestatore.
\end{Definition}
\begin{Lemma}\label{lemma_arbitraggio}
	Siano $P_1$ e $P_2$ due portafogli e siano $V_1$ e $V_2$ i loro rispettivi valori al tempo $t_0$ e $W_1$ e $W_2$ i loro rispettivi valori al tempo $t_1 > t_0$. Se $V_1 > V_2$ e al tempo $t_0$ \`e gi\`a noto che $W_1 = W_2$, allora il possessore del portafoglio $P_1$ pu\`o realizzare un arbitraggio vendendo allo scoperto $P_1$ con scadenza $t_1$ e acquistando $P_2$.
\end{Lemma}
\Proof \`E chiaro che al tempo $t_0$ ricaviamo un profitto. La copertura della vendita allo scoperto si fa al tempo $t_1$ vendendo $P_2$ e ricomprando $P_1$, operazione senza senza costi. \EndProof
\begin{Theorem}\label{parita_callput}
	\TheoremName{Parit\`a \Call-\Put} Si consideri un mercato con un'attivit\`a finanziara $S$ di valore $S_t$ al generico tempo $t$, la quale fa da sottostante ad un'opzione\Call\ di \Strike $K$ a scadenza $T$ dal valore $C_t$ e ad un'opzione \Put\ di stesso \Strike e stessa scadenza dal valore di $P_t$. Supponiamo inoltre che sia possibile prestare denaro\footnote{Salvo avvisi contrari, considereremo gli interessi sempre continuamente composti (anche detti matematici): pertanto se la somma presa in prestito al tempo $t$ al tasso $r$ \`e $x$, al tempo $T > t$ la somma da restituire \`e $xe^{r(T - t)}$ e l'interesse maturato $x(e^{r(T - t)} - 1)$.} al tasso $r$. Affinch\'e non vi siano arbitraggi \`e necessario che sia verifacata la seguente identit\`a: $$C_t - P_t = S_t - Ke^{-r(T-t)}.$$
\end{Theorem}
\par Diamo ora la dimostrazione. Seguir\`a una piccola spiegazione per chiarire i alcuni punti.\\\\
\Proof Consideriamo la formazione di due portafogli al tempo $t < T$:
\begin{itemize}
	\item $P_1$ \`e realizzato comprando una \Call\ e vendendo una \Put\ (si suppone di disporre di liquidit\`a a sufficienza): il suo valore $V_1$ \`e $C_t - P_t$;
	\item $P_2$ invece \`e costruito comprando un'unit\`a dell'attivit\`a finanziaria $S$ e prendendo a prestito la somma di denaro $Ke^{-r(T-t)}$ con scadenza $T$: il suo valore $V_2$ \`e $S_t - Ke^{-r(T-t)}$.
\end{itemize}
\par Vediamo ora cosa succede al tempo $T$ al portafoglio $P_1$. Se $S_T \geq K$, allora la \Call\ avr\`a per valore $S_T - K$ mentre la \Put\ perder\`a ogni valore: il valore del portafoglio \`e dunque $W_1 = S_T - K$. Se invece $K \leq S_T$, allora perder\`a valore la \Call\ mentre verr\`a esercitata la \Put\ cosicch\'e il valore del portafoglio diventer\`a $W_1 = S_T - K$.
\par Vediamo ora il portafoglio $P_2$. Il valore del portafoglio sar\`a $W_2 = S_T - Ke^{-r(T-t)} \cdot e^{r(T - t)} = S_T - K$.
\par Il lemma \ref{lemma_arbitraggio} conclude la dimostrazione. \EndProof
\par Perch\'e la precedente dimostrazione sia pienamente convincente a necessario giustifacare alcune affermazioni relative al problema del \Pricing, in particolare la nostra valutazione del prezzo delle opzioni al tempo $T$ e del prestito, sia al tempo $t$ che al tempo $T$.
\par Introduciamo una notazione: $(x)^+ = \begin{cases} x \text{ se } x \geq 0,\\ 0 \text{ altrimenti}.\end{cases}$
\par Vediamo il \Pricing\ della \Call\ al tempo $T$. Essa, se esercitata, permette di incassare la somma $(S_T - K)^+$. Pertanto, \`e chiaro che chi ne \`e in possesso non ha nessuna convenienza a venderla ad un prezzo inferiore di $(S_T - K)^+$, mentre a chi vorrebbe acquistarla non conviene pagare pi\`u di $(S_T - K)^+$: il suo giusto prezzo \`e dunque $(S_T - K)^+$. Un analogo ragionamento porta ad individuare il giusto prezzo della \Put\ in $(K - S_T)^+$.
\par Per quanto riguarda il prestito, bisogna considerare in ogni momento qual \`e l'ammontare della somma da restituire: nessuno accetterebbe di acquistare un debito per una somma minore dell'importo che gli consentirebbe di pagarlo, mentre nessuno lo cederebbe pagando una somma maggiore.
\begin{Theorem}
	\TheoremName{Convessit\`a del \Pricing\ del \Call\ rispetto al prezzo \Strike.} Sia $S$ un attivo di valore $S_t$ al tempo $t$ e consideriamo una \Call\ $C$ con $S$ per sottostante valutata $C_t$ al tempo $t$ e di scadenza $T$ (indipendente da $t$). Perch\'e non vi siano arbitraggi, \`e necessario che $C_t$ sia funzione convessa dello \Strike, cio\`e che valga la relazione $C_t \left (\frac{K_1 + K_2}{2} \right ) \leq \frac{C_t(K_1)}{2} + \frac{C_t(K_2)}{2}$\footnote{La definizione generale di convessit\`a prevede che una funzione $f$ sia convessa quando $f(\lambda x + (1 - \lambda) y) \leq \lambda f(x) + (1 - \lambda)f(y)$: la propriet\`a qui enunciata \`e equivalente se assumiamo che $C_t$ sia continua, oppure limitata superiormente o ancora misurabile.}.
\end{Theorem}
\Proof Supponiamo al contrario $C_t \left (\frac{K_1 + K_2}{2} \right ) > \frac{C_t(K_1)}{2} + \frac{C_t(K_2)}{2}$ e sia, per fissare le idee, $K_1 < K_2$. Al tempo $t$, vendiamo due \Call\ con \Strike\ $\frac{K_1 + K_2}{2}$ e compriamo due \Call, una con \Strike\ $K_1$ e l'altra con \Strike\ $K_2$: per ipotesi, il ricavo \`e positivo. Vediamo cosa succede alla scadenza per assicurarci che la mossa non implichi una perdita successiva (e, soprattutto, maggiore del ricavo appena incassato); indichiamo con $V$ il valore del portafoglio al tempo $T$. Abbiamo
\begin{equation}
	V =
	\begin{cases}
		0\text{, se }S_T \leq K_1,\\
		S_T - K_1\text{, se }K_1 < S_T \leq \frac{K_1 + K_2}{2},\\
		S_T - K_1 - 2S_T + K_1 + K_2 = K_2 - S_T\text{, se }\frac{K_1 + K_2}{2} < S_T \leq K_2,\\
		S_T - K_1 - 2S_T + K_1 + K_2 + S_T - K_2 = 0\text{, se } K_2 < S_T.
	\end{cases}
\end{equation}
\par Si vede dunque che, in ogni caso $V \geq 0$: l'operazione \`e dunque un arbitraggio.\EndProof

\section{Modelli discreti.}\label{ModelliDiscreti}
\par In questa sezione analizzeremo i mercati finanziari, concentrandoci maggiormente sulle opzioni, tramite un modelli discreti.
\subsection{Descrizione del modello generale.}\label{DescrizioneDelModelloGenerale}
\par D'ora in poi supporremo sempre dato uno spazio di probabilit\`a filtrato $(\SamplesSpace,\SigmaAlgebra,\Probability{})$, dove $\SamplesSpace$ \`e lo spazio dei campioni, $\SigmaAlgebra$ la $\sigma$-algebra degli eventi, $\Probability{}$ la misura di probabilit\`a e la filtrazione \`e data dalla famiglia di $\sigma$-algebre $\SigmaAlgebra_t$: supporremo $T = \lbrace 0, 1, ..., N \rbrace$ ($N \in \mathbb{N}$; il nostro \`e un modello discreto a tempi finiti) e interpreteremo $\SigmaAlgebra_t$ come la $\sigma$-algebra degli eventi noti al tempo $t$. La costante $N$ viene a volte chiamata \Define{orizzonte} del modello.
\par \`E inoltre usuale in ambito finanziario supporre $(\forall \omega)(\Probability{\omega} > 0)$, $\SigmaAlgebra_0 = \lbrace \emptyset, \SamplesSpace \rbrace$ e $\SigmaAlgebra_N = \SigmaAlgebra$. Definiamo $\SigmaAlgebra_{-1} = \SigmaAlgebra_0$.
\begin{Definition}
	Chiamiamo \Define{attivo finanziario} un processo stocastico adattato $(S_t)_{t \in T}$.
\end{Definition} 
\par Gli attivi cos\`i definiti sono l'analogo matematico degli strumenti finanziari definiti dalla definizione \ref{AttivoIntuitivo}. L'ipotesi di adattibilit\`a significa che il valore di un attivo $S_t$ in ogni istante $t$ dipende da l'informazione disponibile all'istante $t$.
\par Fissiamo $d \in \mathbb{N}$ e $I = \lbrace 0, ..., d \rbrace$. Supponiamo dato, per ogni $i \in I$, un attivo $(S^i_t)_{t \in T}$.
\par Supporremo che $S^0$ sia un attivo senza rischio -- corrispondente ad esempio all'aver prestato denaro con la ragionevole certezza di vederselo restituire al tasso d'interesse fissato, come nel caso di un titolo di Stato\footnote{Infatti alcuni autori chiamano \Define{bond} l'attivo senza rischio.} -- regolato dal regime dell'interesse composto discontinuo con un tasso d'interesse fisso pari a $r$. Fissiamo inoltre $S^0_0 = 1$ (\`e una semplice normalizzazione): avremo allora $S^0_t = (1 + r)^t$. L'interpretazione pi\`u naturale di $S^0$ \`e quella di liquidit\`a: il tasso $r$ \`e allora il tasso d'interesse esercitato dalla banca centrale che regola la valuta (la BCE per l'euro, la Fed per il dollaro americano). Naturalmente questa interpretazione, per quanto detto prima, presupppone che la valuta in questione sia stabile e in particolare che il tasso d'interesse applicato sia costante nel tempo per tutta la durata del modello; inoltre, per modelli di durata pi\`u corta (per esempio per lo studio di strumenti finanziari di scadenza di tre mesi o sei mesi, come i Buoni Ordinari del Tesoro) si pu\`o assumere $r = 0$.
\par Salvo indicazioni contrari, interpreteremo $S^0$ come liquidit\`a e assumeremo $r > 0$.
\begin{Definition}
	Chiamiamo la grandezza $\beta_t = S^0_t$ \Define{coefficiente di attualizzazione al tempo $t$}.
\end{Definition}
\par Gli altri attivi sono con rischio, sono cio\`e processi stocastici veri e propri: corrispondono ad azioni, opzioni, qualsiasi strumento finanziario quotato in borsa\footnote{In questa categoria rientrano di nuovo anche i titoli di Stato, ma mentre nel caso degli attivi senza rischio si suppone che il titolo sia sicuro, in quest'altro caso si suppone invece che il titolo possa anche essere non rimborsato o che esso sia regolato da un tasso variabile che possa diventare sfavorevole.}. Supponiamo, per ogni $i \in I$ e $t \in T$, che $S^i_t > 0$: quest'ipotesi significa che tutti gli strumenti finanziari hanno sempre un valore strettamente positivo. Definiamo inoltre il vettore $S_t = (S^0_t, ..., S^d_t)$ dei prezzi al tempo $t$.
\begin{Definition}\label{PortafoglioFormale}
	Fissato $t \in T$, definiamo \Define{portafoglio} un processo stocastico prevedibile $\phi_t = (\phi^0,...,\phi^d)$\footnote{$\phi_t$ \`e dunque un processo stocastico prevedibile.}. Chiamiamo poi \Define{valore del portafoglio} al tempo $t$ la grandezza $V(\phi_t) = \phi_t \cdot S_t$.
\end{Definition}
\par Confrontiamo la definizione \ref{PortafoglioFormale} con la definizione \ref{PortafoglioIntuitivo}. Se le componenti di $\phi_t$ sono tutte numeri naturali non abbiamo difficolt\`a a ritrovare la definizione data nella sottosezione precedente: l'$i$-esima componente di $\phi_t$ rappresenta chiaramente la quantit\`a dell'$i$-esimo attivo nel portafoglio e le due definizioni di valore del portafoglio coincidono. Con un po' d'impegno si riesce a dare un senso al possesso di una quantit\`a reale di un attivo\footnote{Possedere $-1$ \Put significa aver venduto un \Put e esserne vincolato, acquistare $\sqrt{2}$ di un titolo di Stato che vale $100$ significa aver acquistato un titolo di Stato che vale $100\sqrt{2}$...} e automaticamente si estende il significato del valore del portafoglio cos\`i definito. Ma perch\'e fare delle componenti di $\phi_t$ (e dunque di $\phi_t$ e di $V(\phi_t)$) addirittura delle variabili aleatorie? E perch\'e con quei precisi requisiti di misurabilit\`a? Ci\`o viene chiarito dalla prossima definizione.
\begin{Definition}
	Chiamiamo \Define{strategia di portafoglio} un qualsiasi processo stocastico $\phi = (\phi_t)_{t \in T}$ dove ogni $\phi_t$ \`e un portafoglio al tempo $t$.
\end{Definition}
\par Una strategia di portafoglio \`e un processo stocastico prevedibile.
\par Capiamo adesso che la necessit\`a di estendere la definizione di portafoglio al caso di componenti aleatorie risiede nel nostro desiderio di studiare l'evoluzione incerta del mercato e dunque le strategie incerte degli attori: al generico tempo $t \in T$ ogni investitore dispone solo dell'informazione data dagli eventi noti all'istante $t$, codificata nella $\sigma$-algebra $\SigmaAlgebra_t$, e dunque la quantit\`a $\phi^i_t$ dell'$i$-esimo attivo $S^i$ inizialmente disponibile nel periodo $[t,t + 1[$, essendo risultato delle scelte effettuate all'istante $t - 1$, deve essere una variabile aleatoria $\SigmaAlgebra_{t - 1}$-misurabile.
\par Possiamo vedere una strategia di portafoglio come un portafoglio dipendente dal parametro tempo, cio\`e come un processo stocastico definito da portafogli, tanto che a volte i concetti di portafoglio e di strategia di portafoglio si confondono e vengono usati l'uno al posto dell'altro: praticheremo (consapevolmente) questo abuso di linguaggio. In questo senso va la definizione successiva.
\begin{Definition}
	Data una strategia di portafoglio $\phi$, costruiamo il processo stocastico $(V_t(\phi))_{t \in T}$ e definiamo \Define{valore del portafoglio $\phi$ al tempo $t$} il valore del portafoglio $\phi_t$.
\end{Definition}
\par Vediamo ora un concetto che ci permetter\`a di confrontare quantit\`a di denaro in momenti diversi. Il problema \`e il seguente: \`e chiaro che \`e preferibile disporre di $100$ euro oggi piuttosto che di $10$ euro il mese prossimo, ma sono preferibili $10$ euro oggi o $100$ euro il mese prossimo? Se davanti a questa scelta \`e ancora, forse, facile decidere, si provi a decidere tra $95$ euro oggi o $100$ euro il mese prossimo: nella vita quotidiana si risponder\`a valutando come \`e possibile spendere $95$ euro a partire oggi e come \`e possibile spenderne $100$ a partire dal prossimo mese (per esempio si valuter\`a la presenza di sconti utili al supermercato riservati a questo mese); in ambito finanziario vale lo stesso concetto, tenendo conto per\`o che in questo caso le spese sono investimenti.
\par La definizione seguente propone un mezzo per compiere tali valutazioni.
\begin{Definition}
	Dato un attivo $S^i_t$ si chiama \Define{attivo attualizzato} rispetto al \Define{numererario} $S^0_t$ la grandezza $\Actualized{S}^i_t = \frac{S^i_t}{S^0_t} = \beta_t S^i_t$. Analogamente, dato un portafoglio $\phi$, definiamo il \Define{valore attualizzato} di $\phi$ come $\Actualized{V}_t(\phi) = \beta_t V_t(\phi)$.
\end{Definition}
\par Dati per esempio due attivi $S^i_{t_0}$ e $S^j_{t_1}$ (possiamo assumere anche $i = j$ o $t_0 = t_1$) possiamo adesso sempre confrontarli confrontando i rispettivi attivi attualizzati.
\par Esaminiamo meglio il significato dell'attualizzazione di un attivo. Se $r = 0$, allora tutti gli attivi coincidono coi rispettivi attivi attualizzati e possiamo dunque confrontare tutti gli attivi in qualsiasi istante come se essi fossero in un certo senso ``contemporanei'': questo \`e coerente con la nostra idea di riservare il tasso nullo per modelli di breve durata. Se invece $r > 0$, allora si vede che l'attivo attualizzato $\Actualized{S}^i_{t_1}$ \`e la somma che \`e necessario tenere liquida (o investire nell'attivo senza rischio) al tempo $t_0 = 0$ per ottenere esattamente la liquidit\`a $S^i_{t_1}$ al tempo $t_1$ (o un investimento di valore equivalente nell'attivo senza rischio).
\begin{Definition}
	Un portafoglio $\phi$ si dice \Define{autofinanziato} quando per ogni istante $t > 0$ vale $V_{t - 1}(\phi) = \phi_t \cdot S_{t - 1}$.
\end{Definition}
\par Un portafoglio autofinanziato \`e dunque un portafoglio tale che, per ogni tempo $t > 0$,
\begin{itemize}
	\item se al tempo $t - 1$ \`e stato realizzato un guadagno (o \`e appena stato aperto il portafoglio nel caso $t = 1$), reinveste (cio\`e sceglie $\phi_t$) tutti i profitti nei $d + 1$ attivi disponibili (di valore $S_{t - 1}$) senza aggiungere nessun capitale\footnote{Se scegliamo di interpretare l'attivo senza rischio come liquidit\`a, questo significa che non aumenta la liquidit\`a \Define{dall'esterno}, ma non esclude che la liquidit\`a non aumenti a seguito della vendita di un attivo con rischio.};
	\item se al tempo $t - 1$ \`e stata realizzata una perdita, non la compensa con un aumento di capitale.
\end{itemize}
\par Questo non significa che il valore del portafoglio si conservi nel tempo: \`e vero che non arriva mai nuovo capitale \Define{dall'esterno}, ma non esclude che gli attivi producano ricchezza\footnote{Cio\`e aumentino di valore. Il nostro modello non tiene conto di cedole o dividendi: eventuali cedole o dividendi possono per\`o essere tenute in conto nella quotazione degli attivi.}.
\par Introduciamo la seguente notazione: se $(X_t)_{t \in T}$ \`e un processo stocastico, poniamo $\Delta X_t = X_t - X_{t - 1}$ per ogni $t > 0$ e $\Delta X_0 = 0$.
\begin{Lemma}\label{LemmaAutofinanziato}
	Sia $\phi$ un portafoglio. Sono equivalente le seguenti affermazioni:
	\begin{itemize}
		\item il portafoglio $\phi$ \`e autofinanziato;
		\item per ogni $t > 0$, $\Delta V_t(\phi) = \phi_t \cdot \Delta S_t$;
		\item per ogni $t > 0$, $\Delta \Actualized{V}_t(\phi) = \phi_t \cdot \Delta \Actualized{S}_t$;
		\item per ogni $t \in T$, $V_t(\phi) = V_0(\phi) + \sum_{j = 1}^t \phi_j \cdot \Delta S_j$;
		\item per ogni $t \in T$, $\Actualized{V}_t(\phi) = \Actualized{V}_0(\phi) + \sum_{j = 1}^t \phi_j \cdot \Delta \Actualized{S}_j$;
	\end{itemize}
\end{Lemma}
\Proof Dimostriamo solo l'equivalenza della prima e della seconda proposizione: l'equivalenza della seconda e della terza e quella della quarta e della quinta sono evidenti moltiplicando entrambi i membri dell'equazione della seconda e della quarta per il coefficiente d'attualizzazione $\beta_t$; l'equivalenza della seconda e della quarta segue da un semplice risultato sulle serie telescopiche finite.
\par Siano $\phi$ autofinanziato e $t > 0$. Allora $\Delta V_t(\phi) = \phi_t \cdot S_t - \phi_t \cdot S_{t -1} = \phi_t \cdot \Delta S_t$.
\par D'altra parte assumendo, per ogni $t > 0$, $\Delta V_t(\phi) = \phi_t \cdot \Delta S_t$, abbiamo $V_{t - 1}(\phi) = V_t - \Delta V_t = \phi_t \cdot S_t - \phi_t \cdot S_t = \phi_t \cdot (S_t - \Delta S_t) = \phi_t \cdot S_{t - 1}$ e il portafoglio \`e dunque autofinanziato. \EndProof
\begin{Theorem}\label{TeoremaEstensioneSenzaRischio}
	Siano $x \in \mathbb{R}$ e $(\phi^1_t,...,\phi^d_t)_{t \in T}$ un processo stocastico prevedibile. Esiste un solo processo stocastico prevedibile $(\phi^0_t)_{t \in T}$ tale che $\phi = (\phi^0_t, \phi^1_t, ...., \phi^d_t)_{t \in T}$ sia un portafoglio autofinanziato per il quale $V_0(\phi) = x$.
\end{Theorem}
\par Intuitivamente, il teorema afferma che se \`e noto il capitale iniziale e se \`e nota la strategia secondo cui vengono comprati e acquistati gli attivi con rischio, allora \`e nota anche l'evoluzione della liquidit\`a, assumendo che non ne venga introdotta di nuova o rimossa \Define{dall'esterno}.
\Proof Assumiamo vera l'affermazione del teorema. Per il lemma \ref{LemmaAutofinanziato}, per ogni $t > 0$, abbiamo
\begin{align}
	&\Actualized{V}_t(\phi) = \Actualized{V}_0(\phi) + \sum_{j = 1}^t \phi_j \cdot \Delta \Actualized{S}_j,\nonumber\\
	&\phi^0_t = \Actualized{V}_0(\phi) + \sum_{j = 1}^t \phi_j \cdot \Delta \Actualized{S}_j - \sum_{i = 1}^d \phi^i_t \Actualized{S}_t,\nonumber\\
	&\phi^0_t = \Actualized{V}_0(\phi) + \sum_{j = 1}^{t - 1} \phi_j \cdot \Delta \Actualized{S}_j - \sum_{i = 1}^d \phi_{t - 1} \cdot \Delta \Actualized{S}_{t - 1}.
\end{align}
\par Questo dimostra l'unicit\`a. Proseguiamo verificando che $(\phi^0)_{t \in T}$ verifichi tutte le condizioni desiderate: esso \`e prevedibile perch\'e tutte le variabili aleatorie coinvolte nella definizione di $\phi^0_t$ ($t \in T$) sono $\SigmaAlgebra_{t - 1}$-misurabili e combinate attraverso somme e prodotti e $\phi$ \`e dunque un portafoglio. L'autofinanziamanto di $\phi$ segue di nuovo dal lemma \ref{LemmaAutofinanziato}. \EndProof
\begin{Definition}
	Un portafoglio $\phi$ si dice \Define{ammissibile} quando \`e autofinanziato e, per ogni $t \in T$, si ha $V_t(\phi) \geq 0$.
\end{Definition}
\par Il significato intuitivo di questa definizione \`e che un portafoglio \`e ammissibile quando, se in qualche istante $t$ \`e composto da attivi dovuti (cio\`e $\phi^i_t < 0$ per qualche $i \in I$), questi sono tutti immediatamente saldabili (vendendo gli attivi detenuti, cio\`e gli attivi $S^i$ per cui $\phi^i_t > 0$).
\par Salvo avvisi contrari, d'ora in poi considereremo solo portafogli ammissibili.
\begin{Definition}
	Sia $X$ una variabile aleatoria $\SigmaAlgebra_N$-misurabile. Diciamo che un portafoglio $\phi$ \`e
	\begin{itemize}
		\item un \Define{portafoglio replicante} quando $V_N(\phi) = X$;
		\item un \Define{portafoglio di copertura} quando $V_N(\phi) \geq X$.
	\end{itemize}
	Denotiamo con $\ReplicantWallet$ l'insieme dei valori attualizzati dei portafogli autofinanziati replicanti a costo $0$ (cio\`e dei portafogli $\phi$ tali che $V_0(\phi) = 0$).
\end{Definition}
\begin{Corollary}
	Abbiamo $\ReplicantWallet = \left \lbrace \sum_{t = 1}^N \sum_{i = 1}^d \phi^i_t \Actualized{S}^i_t | (\phi^1_t,...,\phi^d_t)_{t \in T}\text{ processo stocastico prevedibile} \right \rbrace$.
\end{Corollary}
\Proof Per il teorema \ref{TeoremaEstensioneSenzaRischio}, per ogni processo stocastico prevedibile $(\phi^1_t,....,\phi^d_t)_{t \in T}$ esiste un unico processo stocastico prevedibile $(\phi^0_t)_{t \in T}$ tale che $\phi = (\phi^0_t,\phi^1_t,...,\phi^d_t)_{t \in T}$ sia un portafoglio autofinanziato a costo $0$. Per il lemma \ref{LemmaAutofinanziato}, $\Actualized{V}_N(\phi) = \Actualized{V}_0(\phi) + \sum_{t = 1}^N \phi_t \cdot \Delta \Actualized{S}_t = \sum_{t = 1}^N \sum_{i = 1}^d \phi^i_t \Actualized{S}^i_t$. Questo prova $\left \lbrace \sum_{t = 1}^N \sum_{i = 1}^d \phi^i_t \Actualized{S}^i_t | (\phi^1_t,...,\phi^d_t)_{t \in T}\text{ processo stocastico prevedibile} \right \rbrace \subseteq \ReplicantWallet$. L'inclusione inversa \`e immediata. \EndProof
\par Nel seguito $\RandomVariable^0 = L^0(\SamplesSpace,\SigmaAlgebra,\Probability{})$ \`e lo spazio di tutte le variabili aleatorie munito della convergenza in probabilit\`a e $\RandomVariable^0_+$ il sottospazio di tutte le variabili aleatorie non negative.

\subsection{Arbitraggi.}\label{Arbitraggi}
\begin{Definition}
	Un \Define{arbitraggio} \`e un portafoglio $\phi$ tale che
	\begin{itemize}
		\item $V_0(\phi) = 0$;
		\item $\Probability{V_N(\phi) > 0} > 0$.
	\end{itemize}
\end{Definition}
\par Si vede subito che la precedente definizione \`e in accordo con la definizione \ref{ArbitraggioIntuitivo}. Inoltre, notiamo che se sostituiamo nel modello alla probabilit\`a $\Probability$ un'altra probabilit\`a equivalente, l'insieme dei portafogli che sono arbitraggi non cambia.
\begin{Definition}
	Se in un mercato finanziario non esistono arbitraggi, si dice che vale l'\Define{ipotesi N.A.}.
\end{Definition}
\begin{Theorem}
	L'ipotesi N.A. \`e equivalente alla validit\`a dell'equazione $\ReplicantWallet \cap \RandomVariable{0}{+} = \lbrace 0 \rbrace$.
\end{Theorem}
\Proof \`E chiaro che un portafoglio a costo $0$ che non esegue nessuna operazione finanziaria nel corso del tempo \`e un arbitraggio che porta come valore finale attualizzato $0$.
\par D'altra parte, sia $\phi$ \`e un portafoglio a costo $0$ il cui valore finale attualizzato del portafoglio $\phi$ \`e una variabile aleatoria non negativa. Allora, se il valore finale \`e necessariamente nullo, $\phi$ non pu\`o essere un arbitraggio; se vale l'ipotesi N.A, allora il valore finale \`e necessariamente identicamente nullo. \EndProof
\begin{Definition}
	Una probabilit\`a martingala equivalente $\Probability_2$ \`e una probablit\`a equivalente a $\Probability$ per la quale gli attivi attualizzati sono martingale.
\end{Definition}
\par Esaminiamo cosa significa dire che gli attivi attualizzati sono martingale. Per definizione di martingala, se un attivo attualizzato \`e una martingala allora, fissato $t \in T$, la miglior stima del valore (attuale) dell'attivo al tempo $t + 1$, effettuata con le informazioni note al tempo $t$, \`e il valore (attuale) dell'attivo al tempo $t$; in altri termini, lo scenario pi\`u probabile \`e che il valore dell'attivo resti lo stesso a meno di tassi di interesse.
\begin{Theorem}
	\TheoremName{Primo teorema fondamentale della valutazione degli attivi: Dalang-Morton-Willinger} L'ipotesi N.A. \`e equivalente all'esistenza di una probabilit\`a martingala equivalente. Inoltre, questa probabilit\`a martingale equivalente pu\`o essere scelta tale che $\DensityVersion{\Probability_2}{\Probability} \in \DerivClass{\infty}$.
\end{Theorem}
\Proof Supponiamo prima che esista una probabilit\`a martingala equivalente $\Probability_2$. Supponiamo $\phi = (\phi_t)_{t \in T}$ sia un portafoglio ammissibile di valore iniziale nullo. Abbiamo per ogni $t \in T$, in virt\`u del lemma \ref{LemmaAutofinanziato}, $\Actualized{V}_t(\phi) = \Actualized{V}_0(\phi) + \sum_{j = 1}^t \phi_j \cdot \Delta \Actualized{S}_j$. Poich\'e il processo stocastico $\left ( \Actualized{V}_t(\phi) \right )_{t \in T}$ \`e il trasformato della martingala $(\Actualized{S}_t)_{t \in T}$ per il processo prevedibile $\phi$, allora anch'esso \`e una martingala. Pertanto, per ogni $t \in T$, il valore atteso di $\Actualized{V}_t(\phi)$ rispetto alla probabilit\`a $\Probability_2$ \`e sempre nullo.
\par Ora, per l'ammissibilit\`a di $\phi$, ogni $\Actualized{V}_t(\phi)$ \`e positivo o nullo. Poich\'e, per le ipotesi generali sul modello, la probabilit\`a dei di qualsiasi evento non vuoto \`e strettamente positiva, $\Actualized{V}_t(\phi)$ non pu\`o essere positivo o lo sarebbe anche il suo valore atteso: dunque $\Actualized{V}_t(\phi) = 0$ e $\phi$ non pu\`o essere un arbitraggio.
\par Vediamo l'implicazione inversa. Si verifica immediatamente che i valori attualizzati finali sono funzioni lineari dei portafogli, dunque $\ReplicantWallet$ \`e un sottospazio vettoriale dello spazio vettoriale $\mathbb{R}^\SamplesSpace$. Consideriamo poi l'insieme $K = \lbrace X \in \RandomVariable{0}{+} | \sum_{\omega \in \SamplesSpace} X(\omega) = 1 \rbrace$: esso 

\footnote{Il teorema di Hahn-Banach (o, meglio, di un corollario) stabilisce che dati, in uno spazio vettoriale normato $X$, due chiusi $A$ e $B$, non vuoti, disgiunti, di cui almeno uno \`e compatto, allora esiste un iperpiano $H = c$ che separa $A$ e $B$ in senso stretto.}



	\chapter{Calcolo scientifico}
	\section{Teorema di Perron-Frobenius.}
\label{CalcoloScientifico_TeoremaDiPerronFrobenius}
\begin{Theorem}
  \TheoremName{Teorema di Perron-Frobenius}[di Perron-Frobenius][teorema]
  Siano
  \begin{itemize}
    \item $n \in \NotZero{\mathbb{N}}$;
    \item $\Matrix \in \mathbb{C}^{n \times n}$ non negativa.
  \end{itemize}
  Allora
  \begin{itemize}
    \item $\SpectralRadius{\Matrix} \in \Spectrum{\Matrix}$;
    \item esiste un autovettore sinistro $\VarEigenvector$ tale che
      $\ForAll{k \in n}{\VarEigenvector \geq 0}$;
    \item esiste un autovettore destro $\Eigenvector$ tale che
      $\ForAll{k \in n}{\Eigenvector \geq 0}$.
  \end{itemize}
  Inoltre,
  \begin{itemize}
    \item se $\Matrix$ \`e irriducibile, allora
      \begin{itemize}
        \item $\SpectralRadius{\Matrix}$ \`e un autovalore semplice;
        \item esiste un autovettore sinistro $\VarEigenvector$ tale che
          $\ForAll{k \in n}{\VarEigenvector > 0}$;
        \item esiste un autovettore destro $\Eigenvector$ tale che
          $\ForAll{k \in n}{\Eigenvector > 0}$;
      \end{itemize}
      \item se $\Matrix$ \`e positiva, allora
        $\SpectralRadius{\Matrix}$
        \`e l'unico autovalore di modulo massimo di $\Matrix$.
  \end{itemize}
\end{Theorem}

\section{\English{Page rank}.}
\label{CalcoloScientifico_PageRank}
Il modello denominato
\Define{\English{page rank}}
\`e un modello nato per classificare l'importanza delle pagine
\English{web}, da cui il nome, ma che pu\`o essere riproposto per ogni
rete. Noi lo presenteremo nel suo contesto originale delle reti.
\par Sia $\PagesNumber$ il numero totale di tutte le pagine di una rete data.
Identificheremo ogni pagina con un numero naturale da $0$ a
$\PagesNumber - 1$, dove $\PagesNumber$ \`e il numero delle pagine stesse.
\par Esso viene costruito a partire da un grafo orientato
$\Graph = (\PagesNumber,\Edges)$ dove
$(\Node,\VarNode) \in \PagesNumber \times \PagesNumber$ appartiene a $\Edges$
se e solo se la pagina $\Node$ contiene un collegamento ipertestuale
alla pagina $\VarNode$.
\par Denotiamo $\AdjacencyMatrix$ la matrice di adiacenza di $\Graph$.
\par Lo scopo del modello di \English{page rank} \`e determinare un
\Define{vettore di pesi}[dei pesi (\English{page rank})][vettore]
$\Weights$: la componente $\Weights_k$ indica l'importanza della pagina $k$.
\par Definiamo i vettori
\begin{itemize}
  \item $\AllOnes = \Transposed{(1 \cdots 1)}$;
  \item $\PageDegree = \AdjacencyMatrix \AllOnes$;
  \item $\MatrixPageDegree \in \mathbb{R}^{\PagesNumber \times \PagesNumber}$
    tale che
    \[
      \ForAll{(j,k) \in \PagesNumber \times \PagesNumber}{
        \MatrixPageDegree_j^k =
        \begin{cases}
          0&\text{ se } j \neq k,\\
          \PageDegree_k&\text{ se } j = k;
        \end{cases}}
    \]
  \item $\Matrix = \MatrixPageDegree^{-1}\AdjacencyMatrix$.
\end{itemize}
La componente $\PageDegree_k$ ($k \in \PagesNumber$) indica il grado uscente
della pagina $k$.
\par Vorremmo costruire un modello per cui ogni pagina trasmette equamente
la sua importanza a tutte le pagine da esse puntata e l'importanza di una pagina
\`e esattamente la somma delle importanze cos\`i trasmesse tramite i
collegamenti.
\par Come vedremo, \`e utile ipotizzare che non esistano pagine pendenti
nella rete in esame; questa ipotesi \`e facilmente realizzabile con una delle due strategie seguenti, nessuna delle quali cambia sostanzialmente
il senso del modello della rete:
\begin{itemize}
  \item aggiungiamo ad ogni pagina pendente un collegamento a tutte
    le altre pagine;
  \item intoduciamo una pagina aggiuntiva $\PagesNumber$ che punta a tutte 
  le altre pagine ed \`e puntata da tutte le altre pagine:
  \[
    \ForAll{k > 0}{\AdjacencyMatrix_\PagesNumber^k
    = \AdjacencyMatrix_k^\PagesNumber = 1}.
  \]
\end{itemize}
\par Assumeremo dunque d'ora in poi il seguente assioma.
\begin{Axiom}
  La rete non contiene nodi pendenti.
\end{Axiom}
\begin{Theorem}
  $\Weights$ \`e autovettore sinistro di $\Matrix$.
\end{Theorem}
\Proof Per ogni pagina $k \in \PagesNumber$, abbiamo
\[
  \Weights_k = \sum_{j \in \PagesNumber} \AdjacencyMatrix_j^k \frac{\Weights_j}{\PageDegree_j},
\]
equivalente a
\[
  \Weights = \MatrixPageDegree^{-1}\AdjacencyMatrix\Weights,
\]
vale a dire
\[
  \Transposed{\Weights} = \Transposed{\Weights}\Matrix.\text{ \EndProof}
\]
\begin{Theorem}
  Abbiamo
  \begin{itemize}
    \item $1$ \`e autovalore di $\Matrix$ relvativo a $\AllOnes$;
    \item $\SpectralRadius{\Matrix} = 1$.
  \end{itemize}
\end{Theorem}
\Proof Abbiamo
$\Matrix\AllOnes
= \MatrixPageDegree^{-1}\AdjacencyMatrix\AllOnes
= \MatrixPageDegree^{-1}\PageDegree
= \AllOnes$.
\par Inoltre,
$\Norm{\Matrix}[\infty]
= \max_{k \in \PagesNumber} \sum_{j \in \PagesNumber}
  \AbsoluteValue{\Matrix_k^j}
= \max_{k \in \PagesNumber} \sum_{j \in \PagesNumber}
  \Matrix_k^j
= \max_{k \in \PagesNumber} \sum_{j \in \PagesNumber}
  \frac{\AdjacencyMatrix_k^j}{\PageDegree_k}
= 1$,
e quindi, se $\Eigenvector \in \mathbb{C}^\PagesNumber$ \`e autovettore
di $\Matrix$ relativo a $\Eigenvalue$, allora
$\AbsoluteValue{\Eigenvalue}\Norm{\Eigenvector}[\infty]
= \Norm{\Eigenvalue\Eigenvector}[\infty]
= \Norm{\Matrix\Eigenvector}[\infty]
\leq \Norm{\Matrix}[\infty]\Norm{\Eigenvector}[\infty]$,
da cui
$\AbsoluteValue{\Eigenvalue}
\leq \Norm{\Matrix}[\infty] = 1$. \EndProof
\begin{Theorem}
  Se $\Matrix$ ha un unico autovalore di modulo massimo, qualsiasi successione
  $(\Weights^{(k)})_{k \in \mathbb{N}}$ tale che
  \begin{itemize}
    \item $\Weights^{(0)}$ non \`e ortogonale a $\AllOnes$;
    \item per ogni $k > 0$,
      $\Weights^{(k)} = \Transposed{\Matrix}\Eigenvector^{(k - 1)}$;
    \item se $\Weights^{(0)}$ non ha componenti negative, allora
      $\ForAll{k \in \mathbb{N}}{
      \Norm{\Weights^{(k)}}[1] = \Norm{\Weights^{(0)}}[1]}$;
  \end{itemize}
  converge a un autovettore sinistro di $\Matrix$, cio\`e a $\Weights$
  a meno di prodotto per scalare.
\end{Theorem}
\Proof Riscriviamo il metodo delle potenze applicandolo alla matrice
$\Transposed{\Matrix}$ e al vettore $\Weights^{(0)}$ con una qualsiasi
tolleranza $\Tollerance$:
\begin{enumerate}
  \item\label{PageRank_MetodoDellePotenze_1} si ponga $j = 0$,
    $\Weights_0 = \Weights^{(0)}$,
    $\Eigenvalue_0^{(0)} =
    = \Transposed{\Weights_0}\Matrix\Weights_{0}$;
  \item\label{PageRank_MetodoDellePotenze_2} si ponga
    $\tilde{\Weights}_{j + 1} = \Matrix \Weights_j$;
  \item\label{PageRank_MetodoDellePotenze_3} si ponga
    $\Weights_{j + 1}
    = \frac{\tilde{\Weights}_{j + 1}}{\Norm{\tilde{\Weights_{j + 1}}}}$;
  \item\label{PageRank_MetodoDellePotenze_4} si ponga $\Eigenvalue_0^{(j + 1)}
    = \Transposed{\Weights_{j + 1}}\Matrix\Weights_{j + 1}$;
  \item\label{PageRank_MetodoDellePotenze_5} se
    $\AbsoluteValue{\Eigenvalue_0^{(j + 1)} - \Eigenvalue_0^{(j)}}
    > \Tollerance$
    si incrementi $j$ di $1$ e si torni al punto
    \ref{PageRank_MetodoDellePotenze_2}, altrimenti l'algoritmo termina
    restituendo la coppia $(\Eigenvalue_0^{(j + 1)},\Weights_{j + 1})$.
\end{enumerate}
\par Per ogni $k \in \mathbb{N}$, abbiamo
$\Weights_k = \Weights^{(k)}$ a meno di moltiplicazione per scalare.
\par Supponiamo $\Weights^{(0)}$ non abbia nessuna componente negativa.
Si dimostra immediatamente per induzione, utilizzando anche la non negativit\`a
di $\Matrix$, che per ogni $k \in \mathbb{N}$,
$\Norm{\Weights^{(k)}}[1]
= \Transposed{\Weights^{(k)}}\AllOnes
= \Transposed{\Weights^{(0)}}\AllOnes
\Norm{\Weights^{(0)}}[1]$. \EndProof
\par Con riferimento a quest'ultimo teorema, grazie alla costanza della
norma dei vettori della successione
$(\Weights^{(k)})_{k \in \mathbb{N}}$
\`e possibile calcolare direttamente l'autovettore sinistro $\Weights$
senza il passo di normalizzazione \ref{PageRank_MetodoDellePotenze_3}
evitando qualsiasi rischio di traboccamento dei moduli.
\par Definiamo adesso
\begin{itemize}
  \item $\gamma \in (0,1)$;
  \item $\Vector \in \mathbb{R}^\PagesNumber$ tale che
    $\ForAll{k \in \PagesNumber}{\Vector_k > 0}$ e
    $\Norm{\Vector}[1] = 1$;
  \item $A = \gamma\Matrix + (1 - \gamma)\AllOnes\Transposed{\Vector}$.
\end{itemize}
\begin{Theorem}
  La matrice $A$ \`e positiva.
\end{Theorem}
\Proof Fissiamo $(k,j) \in \PagesNumber \times \PagesNumber$:
$A_k^j$ \`e combinazione convessa di $\Matrix_k^j$ e
$\sum_{k \in \PagesNumber} \Vector_k$ con coefficienti non nulli,
dunque \`e strettamente positivo. \EndProof
\begin{Theorem}
  $\AllOnes$ \`e autovettore di $A$ relativo all'autovalore $1$.
\end{Theorem}
\Proof Abbiamo
$A\AllOnes
= \gamma\Matrix\AllOnes + (1 - \gamma)(\AllOnes\Transposed{\Vector})\AllOnes
= \gamma\AllOnes
  + (1 - \gamma)\AllOnes(\Transposed{\Vector}\AllOnes)
= \gamma\AllOnes + (1 - \gamma)\AllOnes
= \AllOnes$. \EndProof
\begin{Theorem}
  \TheoremName{Teorema di Brauer}[di Brauer][teorema]
  Siano
  \begin{itemize}
    \item $n \in \NotZero{\mathbb{N}}$;
    \item $J \in \mathbb{C}^{n \times n}$;
    \item $\Eigenvector$ autovettore di $J$ relativo all'autovalore
      $\Eigenvalue \in \Spectrum{J}$;
    \item $\VarVector \in \mathbb{C}^n$;
    \item $K = J + \Eigenvector\HermitianTransposed{\VarVector}$.
  \end{itemize}
  Abbiamo
  \[
    \Spectrum{K}
    = \Spectrum{J}
      \SetMin \lbrace \Eigenvalue \rbrace
      \cup \lbrace \Eigenvalue + \HermitianTransposed{\VarVector}\Eigenvector \rbrace.
  \]
\end{Theorem}
\Proof Sia $\UnitaryMatrix \in \mathbb{C}^n$ unitaria tale che
$\SchurNormalForm = \HermitianTransposed{\UnitaryMatrix} J \UnitaryMatrix$
sia una forma normale di Schur di $J$ tale che
$\SchurNormalForm_0^0 = \Eigenvalue$.
\par Denotando $\CanonicalBase^0$ il primo vettore della base canonica, abbiamo
\begin{align}
  \HermitianTransposed{\UnitaryMatrix} K \UnitaryMatrix
  &= \HermitianTransposed{\UnitaryMatrix} J \UnitaryMatrix
    + \HermitianTransposed{\UnitaryMatrix}
      \Eigenvector\HermitianTransposed{\VarVector}
      \UnitaryMatrix,\\
  &= \SchurNormalForm
    + \CanonicalBase^0\HermitianTransposed{\VarVector}
      \UnitaryMatrix.
\end{align}
\par Sia ora $k \in n$. Supponiamo $k = 0$: abbiamo
\[
  (\CanonicalBase^0\HermitianTransposed{\VarVector}\UnitaryMatrix)_0
  = \HermitianTransposed{\VarVector}\UnitaryMatrix.
\]
Se invece $k \neq 0$, allora, per ogni $j \in n$
\[
  (\CanonicalBase^0\HermitianTransposed{\VarVector}\UnitaryMatrix)_k
  = 0.
\]
Dunque tutti gli elementi diagonali della matrice triangolare superiore
$\HermitianTransposed{\UnitaryMatrix} K \UnitaryMatrix$,
forma normale di Schur,
coincidono con quelli di $\SchurNormalForm$ tranne
$(\HermitianTransposed{\UnitaryMatrix} J \UnitaryMatrix)_0^0$
che viene sostituito da
\[
  \Eigenvalue + \HermitianTransposed{\VarVector}\UnitaryMatrix\CanonicalBase^0
  = \Eigenvalue + \HermitianTransposed{\VarVector}\Eigenvector.
  \text{ \EndProof}
\]
\begin{Theorem}
  $1$ \`e autovalore semplice e di modulo massimo di $A$.
\end{Theorem}
\Proof Ricordiamo che
$A = \gamma\Matrix + (1 - \gamma)\AllOnes\Transposed{\Vector}$.
$\AllOnes$ \`e autovettore relativo a $1$ di $\Matrix$ e $1$ \`e autovalore
semplice e di modulo massimo di $\Matrix$ per il teorema di Perron-Frobenius.
Ne deduciamo che
$\AllOnes$ \`e autovettore relativo a $\gamma$ di $\gamma\Matrix$ e
$\gamma$ \`e autovalore semplice e di modulo massimo di $\gamma\Matrix$
per il teorema di Perron-Frobenius.
\par Abbiamo
$\gamma + (1 - \gamma)\Transposed{\AllOnes}\Vector
= \gamma + (1 - \gamma)\Norm{\Vector}[1] = 1$.
Dunque, per il teorema di Bauer, $1$ \`e autovalore semplice e di modulo
massimo di $A$. \EndProof

\section{Vibrazioni.}
\label{CalcoloScientifico_Vibrazioni.}
\subsubsection{Vibrazioni di corde elastiche.}
\label{CalcoloScientifico_VibrazioniDiCordeElastiche}
\par Consideriamo una successione finita di
$N \in \NotZero{\mathbb{N}}$ punti materiali
$(\PointMass_k)_{k \in N}$ di massa $(\Mass_k)_{k \in N}$
e coordinate $(X_k)_{k \in N}$ (nel piano e nello spazio indifferentemente)
tali che
\begin{itemize}
  \item i punti $\PointMass_0$ e $\PointMass_{N - 1}$ siano vincolati
    nella loro posizione al tempo iniziale $t = 0$;
  \item per ogni $k \in N - 1$, $\PointMass_k$ e $\PointMass_{k + 1}$
    sono collegati da una corda elastica di costante elastica
    $\ElasticConstant_k$;
  \item per ogni $k \in \NotZero{(N - 1)}$, $\PointMass_k$ \`e soggetto
    ad una forza d'attrito dinamico con coefficiente $\Friction_k$.
\end{itemize}
\par Sia $k \in \NotZero{(N - 1)}$. Per la seconda legge di Newton, abbiamo
\[
  \Mass_k X_k = - \ElasticConstant X_k - \Friction_k \dot{X_k}.
\]
Inoltre, abbiamo le condizioni al bordo
\[
  \begin{cases}
    \ForAll{t \in \RealNonNegative}{X_0(t) = X_0(0)},\\
    \ForAll{t \in \RealNonNegative}{X_{N - 1}(t) = X_{N - 1}(0)},\\
    \dot{X_0} = \dot{X_{N - 1}} = 0.
  \end{cases}
\]
\par Poniamo ora
\begin{itemize}
  \item $M \in \mathbb{R}^{N \times N}$ uguale alla matrice diagonale i cui
    elementi sono la successione delle masse:
    \[
      M =
      \lmatrix
      \begin{array}{ccc}
        \Mass_0 & &\\
        & \ddots &\\
        & & \Mass_{N - 1}
      \end{array}
      \rmatrix.
    \]
  \item $R \in \mathbb{R}^{N \times N}$ uguale alla matrice diagonale i cui
    elementi sono la successione dei coefficienti di attrito:
    \[
      R =
      \lmatrix
      \begin{array}{ccc}
        \Friction_0 & &\\
        & \ddots &\\
        & & \Friction_{N - 1}
      \end{array}
      \rmatrix.
    \]
  \item $K \in \mathbb{R}^{N \times N}$ uguale alla matrice tridiagonale i
    cui elementi sono
    $K_k^j =
    \begin{cases}
      - \ElasticConstant_k&\text{ se }k = j + 1,\\
      \ElasticConstant_k + \ElasticConstant_{k + 1}&\text{ se }k = j,\\
      - \ElasticConstant_{k + 1}&\text{ se }k = j - 1,
    \end{cases}$
    vale a dire
    \[
      K =
      \lmatrix
      \begin{array}{cccccc}
        \ElasticConstant_0 + \ElasticConstant_1 & - \ElasticConstant_1 &&&&\\
        - \ElasticConstant_0 & \ddots & \ddots &&&\\
        & \ddots & \ddots & \ddots & &\\
        && \ddots & \ddots & - \ElasticConstant\\
        &&& - \ElasticConstant & \ElasticConstant + \ElasticConstant
      \end{array}
      \rmatrix.
    \]
\end{itemize}

\subsubsection{Vibrazioni di membrane elastiche.}
\label{CalcoloScientifico_VibrazioniDiMembraneElastiche}



	\chapter{Metodi numerici per catene di Markov}
	\section{Teorema di Perron-Frobenius.}
\label{CalcoloScientifico_TeoremaDiPerronFrobenius}
\begin{Theorem}
  \TheoremName{Teorema di Perron-Frobenius}[di Perron-Frobenius][teorema]
  Siano
  \begin{itemize}
    \item $n \in \NotZero{\mathbb{N}}$;
    \item $\Matrix \in \mathbb{C}^{n \times n}$ non negativa.
  \end{itemize}
  Allora
  \begin{itemize}
    \item $\SpectralRadius{\Matrix} \in \Spectrum{\Matrix}$;
    \item esiste un autovettore sinistro $\VarEigenvector$ tale che
      $\ForAll{k \in n}{\VarEigenvector \geq 0}$;
    \item esiste un autovettore destro $\Eigenvector$ tale che
      $\ForAll{k \in n}{\Eigenvector \geq 0}$.
  \end{itemize}
  Inoltre,
  \begin{itemize}
    \item se $\Matrix$ \`e irriducibile, allora
      \begin{itemize}
        \item $\SpectralRadius{\Matrix}$ \`e un autovalore semplice;
        \item esiste un autovettore sinistro $\VarEigenvector$ tale che
          $\ForAll{k \in n}{\VarEigenvector > 0}$;
        \item esiste un autovettore destro $\Eigenvector$ tale che
          $\ForAll{k \in n}{\Eigenvector > 0}$;
      \end{itemize}
      \item se $\Matrix$ \`e positiva, allora
        $\SpectralRadius{\Matrix}$
        \`e l'unico autovalore di modulo massimo di $\Matrix$.
  \end{itemize}
\end{Theorem}

\section{\English{Page rank}.}
\label{CalcoloScientifico_PageRank}
Il modello denominato
\Define{\English{page rank}}
\`e un modello nato per classificare l'importanza delle pagine
\English{web}, da cui il nome, ma che pu\`o essere riproposto per ogni
rete. Noi lo presenteremo nel suo contesto originale delle reti.
\par Sia $\PagesNumber$ il numero totale di tutte le pagine di una rete data.
Identificheremo ogni pagina con un numero naturale da $0$ a
$\PagesNumber - 1$, dove $\PagesNumber$ \`e il numero delle pagine stesse.
\par Esso viene costruito a partire da un grafo orientato
$\Graph = (\PagesNumber,\Edges)$ dove
$(\Node,\VarNode) \in \PagesNumber \times \PagesNumber$ appartiene a $\Edges$
se e solo se la pagina $\Node$ contiene un collegamento ipertestuale
alla pagina $\VarNode$.
\par Denotiamo $\AdjacencyMatrix$ la matrice di adiacenza di $\Graph$.
\par Lo scopo del modello di \English{page rank} \`e determinare un
\Define{vettore di pesi}[dei pesi (\English{page rank})][vettore]
$\Weights$: la componente $\Weights_k$ indica l'importanza della pagina $k$.
\par Definiamo i vettori
\begin{itemize}
  \item $\AllOnes = \Transposed{(1 \cdots 1)}$;
  \item $\PageDegree = \AdjacencyMatrix \AllOnes$;
  \item $\MatrixPageDegree \in \mathbb{R}^{\PagesNumber \times \PagesNumber}$
    tale che
    \[
      \ForAll{(j,k) \in \PagesNumber \times \PagesNumber}{
        \MatrixPageDegree_j^k =
        \begin{cases}
          0&\text{ se } j \neq k,\\
          \PageDegree_k&\text{ se } j = k;
        \end{cases}}
    \]
  \item $\Matrix = \MatrixPageDegree^{-1}\AdjacencyMatrix$.
\end{itemize}
La componente $\PageDegree_k$ ($k \in \PagesNumber$) indica il grado uscente
della pagina $k$.
\par Vorremmo costruire un modello per cui ogni pagina trasmette equamente
la sua importanza a tutte le pagine da esse puntata e l'importanza di una pagina
\`e esattamente la somma delle importanze cos\`i trasmesse tramite i
collegamenti.
\par Come vedremo, \`e utile ipotizzare che non esistano pagine pendenti
nella rete in esame; questa ipotesi \`e facilmente realizzabile con una delle due strategie seguenti, nessuna delle quali cambia sostanzialmente
il senso del modello della rete:
\begin{itemize}
  \item aggiungiamo ad ogni pagina pendente un collegamento a tutte
    le altre pagine;
  \item intoduciamo una pagina aggiuntiva $\PagesNumber$ che punta a tutte 
  le altre pagine ed \`e puntata da tutte le altre pagine:
  \[
    \ForAll{k > 0}{\AdjacencyMatrix_\PagesNumber^k
    = \AdjacencyMatrix_k^\PagesNumber = 1}.
  \]
\end{itemize}
\par Assumeremo dunque d'ora in poi il seguente assioma.
\begin{Axiom}
  La rete non contiene nodi pendenti.
\end{Axiom}
\begin{Theorem}
  $\Weights$ \`e autovettore sinistro di $\Matrix$.
\end{Theorem}
\Proof Per ogni pagina $k \in \PagesNumber$, abbiamo
\[
  \Weights_k = \sum_{j \in \PagesNumber} \AdjacencyMatrix_j^k \frac{\Weights_j}{\PageDegree_j},
\]
equivalente a
\[
  \Weights = \MatrixPageDegree^{-1}\AdjacencyMatrix\Weights,
\]
vale a dire
\[
  \Transposed{\Weights} = \Transposed{\Weights}\Matrix.\text{ \EndProof}
\]
\begin{Theorem}
  Abbiamo
  \begin{itemize}
    \item $1$ \`e autovalore di $\Matrix$ relvativo a $\AllOnes$;
    \item $\SpectralRadius{\Matrix} = 1$.
  \end{itemize}
\end{Theorem}
\Proof Abbiamo
$\Matrix\AllOnes
= \MatrixPageDegree^{-1}\AdjacencyMatrix\AllOnes
= \MatrixPageDegree^{-1}\PageDegree
= \AllOnes$.
\par Inoltre,
$\Norm{\Matrix}[\infty]
= \max_{k \in \PagesNumber} \sum_{j \in \PagesNumber}
  \AbsoluteValue{\Matrix_k^j}
= \max_{k \in \PagesNumber} \sum_{j \in \PagesNumber}
  \Matrix_k^j
= \max_{k \in \PagesNumber} \sum_{j \in \PagesNumber}
  \frac{\AdjacencyMatrix_k^j}{\PageDegree_k}
= 1$,
e quindi, se $\Eigenvector \in \mathbb{C}^\PagesNumber$ \`e autovettore
di $\Matrix$ relativo a $\Eigenvalue$, allora
$\AbsoluteValue{\Eigenvalue}\Norm{\Eigenvector}[\infty]
= \Norm{\Eigenvalue\Eigenvector}[\infty]
= \Norm{\Matrix\Eigenvector}[\infty]
\leq \Norm{\Matrix}[\infty]\Norm{\Eigenvector}[\infty]$,
da cui
$\AbsoluteValue{\Eigenvalue}
\leq \Norm{\Matrix}[\infty] = 1$. \EndProof
\begin{Theorem}
  Se $\Matrix$ ha un unico autovalore di modulo massimo, qualsiasi successione
  $(\Weights^{(k)})_{k \in \mathbb{N}}$ tale che
  \begin{itemize}
    \item $\Weights^{(0)}$ non \`e ortogonale a $\AllOnes$;
    \item per ogni $k > 0$,
      $\Weights^{(k)} = \Transposed{\Matrix}\Eigenvector^{(k - 1)}$;
    \item se $\Weights^{(0)}$ non ha componenti negative, allora
      $\ForAll{k \in \mathbb{N}}{
      \Norm{\Weights^{(k)}}[1] = \Norm{\Weights^{(0)}}[1]}$;
  \end{itemize}
  converge a un autovettore sinistro di $\Matrix$, cio\`e a $\Weights$
  a meno di prodotto per scalare.
\end{Theorem}
\Proof Riscriviamo il metodo delle potenze applicandolo alla matrice
$\Transposed{\Matrix}$ e al vettore $\Weights^{(0)}$ con una qualsiasi
tolleranza $\Tollerance$:
\begin{enumerate}
  \item\label{PageRank_MetodoDellePotenze_1} si ponga $j = 0$,
    $\Weights_0 = \Weights^{(0)}$,
    $\Eigenvalue_0^{(0)} =
    = \Transposed{\Weights_0}\Matrix\Weights_{0}$;
  \item\label{PageRank_MetodoDellePotenze_2} si ponga
    $\tilde{\Weights}_{j + 1} = \Matrix \Weights_j$;
  \item\label{PageRank_MetodoDellePotenze_3} si ponga
    $\Weights_{j + 1}
    = \frac{\tilde{\Weights}_{j + 1}}{\Norm{\tilde{\Weights_{j + 1}}}}$;
  \item\label{PageRank_MetodoDellePotenze_4} si ponga $\Eigenvalue_0^{(j + 1)}
    = \Transposed{\Weights_{j + 1}}\Matrix\Weights_{j + 1}$;
  \item\label{PageRank_MetodoDellePotenze_5} se
    $\AbsoluteValue{\Eigenvalue_0^{(j + 1)} - \Eigenvalue_0^{(j)}}
    > \Tollerance$
    si incrementi $j$ di $1$ e si torni al punto
    \ref{PageRank_MetodoDellePotenze_2}, altrimenti l'algoritmo termina
    restituendo la coppia $(\Eigenvalue_0^{(j + 1)},\Weights_{j + 1})$.
\end{enumerate}
\par Per ogni $k \in \mathbb{N}$, abbiamo
$\Weights_k = \Weights^{(k)}$ a meno di moltiplicazione per scalare.
\par Supponiamo $\Weights^{(0)}$ non abbia nessuna componente negativa.
Si dimostra immediatamente per induzione, utilizzando anche la non negativit\`a
di $\Matrix$, che per ogni $k \in \mathbb{N}$,
$\Norm{\Weights^{(k)}}[1]
= \Transposed{\Weights^{(k)}}\AllOnes
= \Transposed{\Weights^{(0)}}\AllOnes
\Norm{\Weights^{(0)}}[1]$. \EndProof
\par Con riferimento a quest'ultimo teorema, grazie alla costanza della
norma dei vettori della successione
$(\Weights^{(k)})_{k \in \mathbb{N}}$
\`e possibile calcolare direttamente l'autovettore sinistro $\Weights$
senza il passo di normalizzazione \ref{PageRank_MetodoDellePotenze_3}
evitando qualsiasi rischio di traboccamento dei moduli.
\par Definiamo adesso
\begin{itemize}
  \item $\gamma \in (0,1)$;
  \item $\Vector \in \mathbb{R}^\PagesNumber$ tale che
    $\ForAll{k \in \PagesNumber}{\Vector_k > 0}$ e
    $\Norm{\Vector}[1] = 1$;
  \item $A = \gamma\Matrix + (1 - \gamma)\AllOnes\Transposed{\Vector}$.
\end{itemize}
\begin{Theorem}
  La matrice $A$ \`e positiva.
\end{Theorem}
\Proof Fissiamo $(k,j) \in \PagesNumber \times \PagesNumber$:
$A_k^j$ \`e combinazione convessa di $\Matrix_k^j$ e
$\sum_{k \in \PagesNumber} \Vector_k$ con coefficienti non nulli,
dunque \`e strettamente positivo. \EndProof
\begin{Theorem}
  $\AllOnes$ \`e autovettore di $A$ relativo all'autovalore $1$.
\end{Theorem}
\Proof Abbiamo
$A\AllOnes
= \gamma\Matrix\AllOnes + (1 - \gamma)(\AllOnes\Transposed{\Vector})\AllOnes
= \gamma\AllOnes
  + (1 - \gamma)\AllOnes(\Transposed{\Vector}\AllOnes)
= \gamma\AllOnes + (1 - \gamma)\AllOnes
= \AllOnes$. \EndProof
\begin{Theorem}
  \TheoremName{Teorema di Brauer}[di Brauer][teorema]
  Siano
  \begin{itemize}
    \item $n \in \NotZero{\mathbb{N}}$;
    \item $J \in \mathbb{C}^{n \times n}$;
    \item $\Eigenvector$ autovettore di $J$ relativo all'autovalore
      $\Eigenvalue \in \Spectrum{J}$;
    \item $\VarVector \in \mathbb{C}^n$;
    \item $K = J + \Eigenvector\HermitianTransposed{\VarVector}$.
  \end{itemize}
  Abbiamo
  \[
    \Spectrum{K}
    = \Spectrum{J}
      \SetMin \lbrace \Eigenvalue \rbrace
      \cup \lbrace \Eigenvalue + \HermitianTransposed{\VarVector}\Eigenvector \rbrace.
  \]
\end{Theorem}
\Proof Sia $\UnitaryMatrix \in \mathbb{C}^n$ unitaria tale che
$\SchurNormalForm = \HermitianTransposed{\UnitaryMatrix} J \UnitaryMatrix$
sia una forma normale di Schur di $J$ tale che
$\SchurNormalForm_0^0 = \Eigenvalue$.
\par Denotando $\CanonicalBase^0$ il primo vettore della base canonica, abbiamo
\begin{align}
  \HermitianTransposed{\UnitaryMatrix} K \UnitaryMatrix
  &= \HermitianTransposed{\UnitaryMatrix} J \UnitaryMatrix
    + \HermitianTransposed{\UnitaryMatrix}
      \Eigenvector\HermitianTransposed{\VarVector}
      \UnitaryMatrix,\\
  &= \SchurNormalForm
    + \CanonicalBase^0\HermitianTransposed{\VarVector}
      \UnitaryMatrix.
\end{align}
\par Sia ora $k \in n$. Supponiamo $k = 0$: abbiamo
\[
  (\CanonicalBase^0\HermitianTransposed{\VarVector}\UnitaryMatrix)_0
  = \HermitianTransposed{\VarVector}\UnitaryMatrix.
\]
Se invece $k \neq 0$, allora, per ogni $j \in n$
\[
  (\CanonicalBase^0\HermitianTransposed{\VarVector}\UnitaryMatrix)_k
  = 0.
\]
Dunque tutti gli elementi diagonali della matrice triangolare superiore
$\HermitianTransposed{\UnitaryMatrix} K \UnitaryMatrix$,
forma normale di Schur,
coincidono con quelli di $\SchurNormalForm$ tranne
$(\HermitianTransposed{\UnitaryMatrix} J \UnitaryMatrix)_0^0$
che viene sostituito da
\[
  \Eigenvalue + \HermitianTransposed{\VarVector}\UnitaryMatrix\CanonicalBase^0
  = \Eigenvalue + \HermitianTransposed{\VarVector}\Eigenvector.
  \text{ \EndProof}
\]
\begin{Theorem}
  $1$ \`e autovalore semplice e di modulo massimo di $A$.
\end{Theorem}
\Proof Ricordiamo che
$A = \gamma\Matrix + (1 - \gamma)\AllOnes\Transposed{\Vector}$.
$\AllOnes$ \`e autovettore relativo a $1$ di $\Matrix$ e $1$ \`e autovalore
semplice e di modulo massimo di $\Matrix$ per il teorema di Perron-Frobenius.
Ne deduciamo che
$\AllOnes$ \`e autovettore relativo a $\gamma$ di $\gamma\Matrix$ e
$\gamma$ \`e autovalore semplice e di modulo massimo di $\gamma\Matrix$
per il teorema di Perron-Frobenius.
\par Abbiamo
$\gamma + (1 - \gamma)\Transposed{\AllOnes}\Vector
= \gamma + (1 - \gamma)\Norm{\Vector}[1] = 1$.
Dunque, per il teorema di Bauer, $1$ \`e autovalore semplice e di modulo
massimo di $A$. \EndProof

\section{Vibrazioni.}
\label{CalcoloScientifico_Vibrazioni.}
\subsubsection{Vibrazioni di corde elastiche.}
\label{CalcoloScientifico_VibrazioniDiCordeElastiche}
\par Consideriamo una successione finita di
$N \in \NotZero{\mathbb{N}}$ punti materiali
$(\PointMass_k)_{k \in N}$ di massa $(\Mass_k)_{k \in N}$
e coordinate $(X_k)_{k \in N}$ (nel piano e nello spazio indifferentemente)
tali che
\begin{itemize}
  \item i punti $\PointMass_0$ e $\PointMass_{N - 1}$ siano vincolati
    nella loro posizione al tempo iniziale $t = 0$;
  \item per ogni $k \in N - 1$, $\PointMass_k$ e $\PointMass_{k + 1}$
    sono collegati da una corda elastica di costante elastica
    $\ElasticConstant_k$;
  \item per ogni $k \in \NotZero{(N - 1)}$, $\PointMass_k$ \`e soggetto
    ad una forza d'attrito dinamico con coefficiente $\Friction_k$.
\end{itemize}
\par Sia $k \in \NotZero{(N - 1)}$. Per la seconda legge di Newton, abbiamo
\[
  \Mass_k X_k = - \ElasticConstant X_k - \Friction_k \dot{X_k}.
\]
Inoltre, abbiamo le condizioni al bordo
\[
  \begin{cases}
    \ForAll{t \in \RealNonNegative}{X_0(t) = X_0(0)},\\
    \ForAll{t \in \RealNonNegative}{X_{N - 1}(t) = X_{N - 1}(0)},\\
    \dot{X_0} = \dot{X_{N - 1}} = 0.
  \end{cases}
\]
\par Poniamo ora
\begin{itemize}
  \item $M \in \mathbb{R}^{N \times N}$ uguale alla matrice diagonale i cui
    elementi sono la successione delle masse:
    \[
      M =
      \lmatrix
      \begin{array}{ccc}
        \Mass_0 & &\\
        & \ddots &\\
        & & \Mass_{N - 1}
      \end{array}
      \rmatrix.
    \]
  \item $R \in \mathbb{R}^{N \times N}$ uguale alla matrice diagonale i cui
    elementi sono la successione dei coefficienti di attrito:
    \[
      R =
      \lmatrix
      \begin{array}{ccc}
        \Friction_0 & &\\
        & \ddots &\\
        & & \Friction_{N - 1}
      \end{array}
      \rmatrix.
    \]
  \item $K \in \mathbb{R}^{N \times N}$ uguale alla matrice tridiagonale i
    cui elementi sono
    $K_k^j =
    \begin{cases}
      - \ElasticConstant_k&\text{ se }k = j + 1,\\
      \ElasticConstant_k + \ElasticConstant_{k + 1}&\text{ se }k = j,\\
      - \ElasticConstant_{k + 1}&\text{ se }k = j - 1,
    \end{cases}$
    vale a dire
    \[
      K =
      \lmatrix
      \begin{array}{cccccc}
        \ElasticConstant_0 + \ElasticConstant_1 & - \ElasticConstant_1 &&&&\\
        - \ElasticConstant_0 & \ddots & \ddots &&&\\
        & \ddots & \ddots & \ddots & &\\
        && \ddots & \ddots & - \ElasticConstant\\
        &&& - \ElasticConstant & \ElasticConstant + \ElasticConstant
      \end{array}
      \rmatrix.
    \]
\end{itemize}

\subsubsection{Vibrazioni di membrane elastiche.}
\label{CalcoloScientifico_VibrazioniDiMembraneElastiche}



	\chapter{Metodi numerici per la grafica}
	\section{Algoritmo di de Casteljau.}
\label{MetodiNumericiPerLaGrafica_AlgoritmoDiDeCasteljau}
\begin{Definition}
  \label{Definition_DeCasteljau}
  Sia $\AffineSpace$ uno spazio $\mathbb{R}$-affine e sia,
  fissato $n \in \mathbb{N}$,
  $(P_k)_{k \in n} \in \AffineSpace^n$.
  Chiamiamo
  \Define{algoritmo di de Casteljau}[di de Casteljau][algoritmo]
  l'algoritmo seguente
  \begin{enumerate}
    \item\label{DeCasteljau_1} si ponga $r = 1$ e, per ogni $k \in n$,
      \[
        \DeCasteljau{k}{0} = P_k;
      \]
    \item\label{DeCasteljau_2} per ogni $k \in n - r$ definiamo
      \[
        \DeCasteljau{k}{r}(t)
          = (1 - t)\DeCasteljau{k}{r - 1} + t\DeCasteljau{k + 1}{r - 1};
      \]
    \item\label{DeCasteljau_3} incrementiamo $r$ di $1$;
    \item\label{DeCasteljau_4} se $r \leq n - 1$ torniamo al punto
      \ref{DeCasteljau_2}, altrimenti l'algoritmo termina.
  \end{enumerate}
  Chiamiamo inoltre
  \begin{itemize}
    \item \Define{punti di controllo}[di controllo][punto]
      i punti $P_k$ al variare di $k \in n$;
    \item \Define{poligono di controllo}[di controllo][poligono]
      il poligono definito dai punti di controllo o, pi\`u frequentemente,
      la spezzata definita da $(P_k)_{k \in n}$;
    \item \Define{curva di B\'ezier}[di B\'ezier][curva]
      la curva $\DeCasteljau{0}{n - 1}$.
  \end{itemize}
\end{Definition}
\begin{listing}
	\insertcode{octave}{"Metodi_numerici_per_la_grafica/DeCasteljau.m"}
	\caption{Implementazione dell'algoritmo di De Casteljau in
  \LanguageName{octave}.}
\end{listing}
\begin{Theorem}
  Con le notazioni della definizione precedente, l'algoritmo di Casteljau costa
  \begin{itemize}
    \item $\frac{n(n + 1)}{2}$ combinazioni lineari;
    \item $\BigO{n}$ spazio.
  \end{itemize}
\end{Theorem}
\Proof Per ogni valore di $r$ vengono calcolate $n - r$ combinazioni lineari:
in totale abbiamo dunque
$\sum_{r = 1}^{n - 1} (n - r)
= n^2 - \sum_{r = 1}^{n - 1} r
= n^2 - \frac{n(n - 1)}{2}
= \frac{n^2 + n}{2}
= \frac{n(n + 1)}{2}$.
\par Per quanto riguarda lo spazio,
\begin{Theorem}
  L'algoritmo di Casteljau \`e numericamente stabile.
\end{Theorem}
\Proof Segue direttamente dalla stabilit\`a numerica delle combinazioni lineari.
\EndProof
\begin{Theorem}
  Sia $\BezierCurve$ una curva di B\'ezier generata dai punti di controllo
  $(P_k)_{k \in n}$ ($n \in \mathbb{N}$). $\BezierCurve$ passa da
  $P_0$ e $P_{n - 1}$.
\end{Theorem}
\Proof Costruiamo $\BezierCurve$ tramite l'algoritmo di Casteljau. Con le
notazioni della definizione \ref{Definition_DeCasteljau}, si dimostra
immediatamente per induzione che
\[
  \ForAll{r \in n}{\DeCasteljau{0}{r}(0) = P_0},
\]
e
\[
  \ForAll{r \in n}{\DeCasteljau{n - r - 1}{r}(1) = P_{n - 1}}.\text{ \EndProof}
\]
\begin{figure}
	\includegraphics[width=0.8\textwidth]{Metodi_numerici_per_la_grafica/Immagine_DeCasteljau.png}
	\centering
	\caption[Esempio di curva di B\'ezier]
  {In blu una curva di B\'ezier, corrispondente al poligono di controllo
  tracciato in rosso.}
\end{figure}

\input{Metodi_numerici_per_la_grafica/PolinomiDiBernstein.tex}

	\chapter{Ricerca operativa}
	\section{Definizioni e teoremi di base.}\label{DefinizioniETeoremiDiBase}
\begin{Definition}
	Chiamiamo
	\Define{problema decisionale}[decisionale][problema] o
	\Define{problema di esistenza}[di esistenza][problema]
	il dato di un insieme $\AdmissibleSet$ detto
  \Define{insieme ammissibile}[ammissibile di un problema di ottimizzazione][insieme],
  che dipende da certi dati detti
  \Define{parametri}[di un problema di ottimizzazione][parametro] o
  \Define{varabili}[di un problema di ottimizzazione][variabile]:
  la specifica di tali parametri produce una particolare
  \Define{istanza}[di un problema di ottimizzazione][istanza].
  Gli elementi di $\AdmissibleSet$ si chiamano
  \Define{soluzioni ammissibili}[ammissible di un problema di ottimizzazione][soluzione];
  se inoltre $\AdmissibleSet$ \`e espresso come sottoinsieme
  di un insieme $\AdmissibleSet'$, allora gli elementi di
  $\AdmissibleSet' - \AdmissibleSet$ vengono talvolta chiamate
  \Define{soluzioni non ammissibili}[non ammissible di un problema di ottimizzazione][soluzione].
  Risolvere un problema decisionale significa determinare se esistono soluzioni ammissibili.
\end{Definition}
\begin{Definition}
	Un problema decisionale $\DecisionalProblem$ si dice \Define{vuoto}[decisionale][problema] quando $\AdmissibleSet = \emptyset$.
\end{Definition}
\begin{Definition}
	Chiamiamo
	\Define{problema di certificato}[di certificato][problema]
	il dato di
	\begin{itemize}
		\item un insieme $\AdmissibleSet$ detto
		\Define{insieme ammissibile}[ammissibile di un problema di ottimizzazione][insieme],
		che dipende da certi dati detti
		\Define{parametri}[di un problema di ottimizzazione][parametro] o
		\Define{varabili}[di un problema di ottimizzazione][variabile]:
		la specifica di tali parametri produce una particolare
		\Define{istanza}[di un problema di ottimizzazione][istanza].
		Gli elementi di $\AdmissibleSet$ si chiamano
		\Define{soluzioni ammissibili}[ammissible di un problema di ottimizzazione][soluzione];
		$\AdmissibleSet$ \`e espresso come sottoinsieme
		di un insieme $\AdmissibleSet'$ e gli elementi di
		$\AdmissibleSet' - \AdmissibleSet$ vengono talvolta chiamate
		\Define{soluzioni non ammissibili}[non ammissible di un problema di ottimizzazione][soluzione];
		\item un elemento $x \in \AdmissibleSet'$
	\end{itemize}
	Risolvere un problema di certificato significa determinare se $x$ \`e o meno una soluzione ammissibile.
\end{Definition}
\begin{Definition}
	Chiamiamo
	\Define{problema di ottimizzazione}[di ottimizzazione][problema]
	il dato di
	\begin{itemize}
		\item un insieme $\AdmissibleSet$ detto
		\Define{insieme ammissibile}[ammissibile di un problema di ottimizzazione][insieme],
		che dipende da certi dati detti
		\Define{parametri}[di un problema di ottimizzazione][parametro] o
		\Define{varabili}[di un problema di ottimizzazione][variabile]:
		la specifica di tali parametri produce una particolare
		\Define{istanza}[di un problema di ottimizzazione][istanza].
		Gli elementi di $\AdmissibleSet$ si chiamano
		\Define{soluzioni ammissibili}[ammissible di un problema di ottimizzazione][soluzione];
		se inoltre $\AdmissibleSet$ \`e espresso come sottoinsieme
		di un insieme $\AdmissibleSet'$, allora gli elementi di
		$\AdmissibleSet' - \AdmissibleSet$ vengono talvolta chiamate
		\Define{soluzioni non ammissibili}[non ammissible di un problema di ottimizzazione][soluzione].
		\item una \Define{funzione obiettivo}[obiettivo][funzione]
		o \Define{funzione di costo}[di costo]{funzione}
		$\ObjectiveFunction: \AdmissibleSet \rightarrow \mathbb{R}$;
		\item l'indicazione di se il problema \`e
		\Define{di minimo}[di minimo][problema] o
		\Define{di massimo}[di massimo][problema]: chiameremo
		\Define{soluzione ottimale}[ottimale di un problema di ottimizzazione][soluzione]
		una soluzione ammissibile $\OptimalSolution$ tale che
		\begin{itemize}
			\item $\ObjectiveFunction(\OptimalSolution) =
			\min_{\AdmissibleSolution \in \AdmissibleSet}
			\ObjectiveFunction(\AdmissibleSolution)$ se il
			problema \`e definito di minimo;
			\item $\ObjectiveFunction(\OptimalSolution) =
			\max_{\AdmissibleSolution \in \AdmissibleSet}
			\ObjectiveFunction(\AdmissibleSolution)$ se il
			problema \`e definito di massimo.
		\end{itemize}
	\end{itemize}
	Denoteremo i dati di un problema di minimo
	$\MinimumProblem$ con la notazione
	$$(\MinimumProblem)\ \min \lbrace \ObjectiveFunction(\AdmissibleSolution): \AdmissibleSolution \in \AdmissibleSet \rbrace$$
	e i dati di un problema di massimo $\MaximumProblem$ con
	$$(\MaximumProblem)\ \max \lbrace \ObjectiveFunction(\AdmissibleSolution): \AdmissibleSolution \in \AdmissibleSet \rbrace.$$
	\par In alternativa, quando l'insieme ammissibile \`e descritto tramite vincoli su variabili, denoteremo il problema scrivendo su una riga $\min$ o $\max$ e la funzione da minimizzare o massimizzare, e nelle righe successive i vincoli posti (forniremo nel seguito molti esempi).
	\par Risolvere un problema di ottimizzazione significa individuarne una soluzione ottimale.
\end{Definition}
\begin{Definition}
	Supponiamo l'insieme ammissibile di un problema (decisionale, di certificato o di ottimizzazione indifferentemente) sia sottoinsieme di uno spazio vettoriale di dimensione $n \in \mathbb{N}$: $n$ si chiama \Define{dimensione del problema}[di un problema][dimensione].
\end{Definition}
\begin{Definition}
	Un problema di ottimizzazione $\OptimizationProblem$ si dice \Define{vuoto}[di ottimizzazione][problema] quando $\AdmissibleSet = \emptyset$.
\end{Definition}
\begin{Example}
	Non tutti i problemi con valore ottimo finito ammettono soluzioni ottime: \`e il caso dei problemi $\min \lbrace x: x > 0 \rbrace$ e $\min \lbrace \frac{1}{x}: x \geq 0 \rbrace$.
\end{Example}
\begin{Example}
	\TheoremName{Problema di equipartizione} Il problema di
	equipartizione \`e determinato da un primo parametro
	$n \in \mathbb{N}$, che determina il numero dei
	restanti parametri, e appunto dai restanti parametri
	$(a_i)_{i \in n} \in \mathbb{N}^n$. Il problema pu\`o
	essere formulato come segue:
	$$(\EquipartitionProblem)\ \min \lbrace \ObjectiveFunction(\Set): \Set \in \PowerSet{n} \rbrace$$
	dove $\ObjectiveFunction(\Set) = \AbsoluteValue{\sum_{i \in S} a_i - \sum_{n \SetMin S} a_i}$.
\end{Example}
\begin{Example}
	\TheoremName{Problema di impaccamento di cerchi (Circle Packing)}
	Il problema di impaccamento di cerchi prevede un unico parametro
	$n \in \mathbb{N}$ e risponde alla seguente domanda: qual \`e il
	massimo raggio che possono avere $n$ cerchi identici posti
	all'interno di un quadrato di lato unitario che non si
	sovrappongono?
	\par Consideriamo l'insieme $\AdmissibleSet'$ di tutte le coppie
	$((x_i,y_i)_{i \in n}, r)$, con, per ogni $i \in n$, $x_i, y_i \in
	[0,1]$ e $r \in \mathbb{R}$: interpretiamo $(x_i, y_i)$ come le
	coordinate del centro del cerchio di indice $i$ e $r$ come il
	raggio comune di tutti gli $n$ cerchi. Restringiamo
	$\AdmissibleSet'$ ad un insieme ammissible $\AdmissibleSet$
	aggiungendo le condizioni seguenti:
	\begin{itemize}
		\item affinch\'e il cerchio di centro $(x_i,y_i)$ sia
		interamente contenuto nel quadrato $[0,1]^2$ richiediamo
		$r \leq x_i \leq 1 - r$ e $r \leq y_i \leq 1 - r$;
		\item affinch\'e i cerchi di centro $(x_i,y_i)$ e
		$(x_j,y_j)$, con $i \neq j$, non si sovrappongano
		richiediamo
		$\Distance[2]((x_i,y_i),(x_j,y_j)) \leq 2r$.
	\end{itemize}
	Il problema pu\`o allora essere descritto come segue:
	$$(\CirclePacking)\ \max \lbrace r: ((x_i,y_i)_{i \in n}, r) \in [0,1]^n \times \mathbb{R} \rbrace.$$
\end{Example}
\begin{Definition}
	Un problema di ottimizzazione si dice
	\begin{itemize}
		\item \Define{combinatorio}[di ottimizzazione combinatorio][problema] quando il suo insieme ammissible \`e finito o numerabile;
		\item \Define{continuo}[di ottimizzazione continuo][problema] quando il suo insieme ammissible \`e pi\`u che numerabile.
	\end{itemize}
\end{Definition}
\begin{Theorem}
	Problemi di minimo e problemi di massimo sono equivalenti, nel senso che
	\begin{itemize}
		\item per ogni problema di minimo esiste un problema di massimo le cui soluzioni ottimali coincidono con le soluzioni ottimali del problema di minimo;
		\item per ogni problema di massimo esiste un problema di minimo le cui soluzioni ottimali coincidono con le soluzioni ottimali del problema di massimo.
	\end{itemize}
\end{Theorem}
\Proof Per trasformare un problema di minimo o di massimo in un problema dell'altro tipo \`e sufficiente moltiplicare la funzione obiettivo per $-1$. \EndProof
\par Nel seguito, per il teorema appena enunciato, enunceremo alcune definizioni o dimostreremo alcuni teoremi validi sia per problemi di minimo che di massimo solo per problemi di minimo: il lettore ricaver\`a facilmente i risultati equivalenti per i problemi di massimo.
\begin{Theorem}
	Problemi decisionali e problemi di ottimizzazione sono equivalenti, nel senso che
	\begin{itemize}
		\item per ogni problema decisionale esiste un problema di ottimizzazione le cui soluzioni ottimali coincidono con le soluzioni ammissibili del problema decisionale;
		\item per ogni problema di ottimizzazione esiste un problema decisionale le cui soluzioni ammissibili coincidono con le soluzioni ottimali del prolbema di ottimizzazione.
	\end{itemize}
\end{Theorem}
\Proof Sia $\DecisionalProblem$ un problema decisionale di insieme ammissibile $\AdmissibleSet$ definito come sottoinsieme di $\AdmissibleSet'$. Il problema di massimo avente $\AdmissibleSet'$ come insieme ammissibile e funzione obiettivo coincidente con la funzione caratteristica di $\AdmissibleSet$ ha la propriet\`a richiesta.
\par Sia $\OptimizationProblem$ un problema di ottimizzazione di insieme ammissible $\AdmissibleSet$. Il problema decisionale il cui insieme ammissibile \`e il sottoinsieme di $\AdmissibleSet$ costituito da tutte le soluzioni ottimali ha la propriet\`a richiesta. \EndProof
\par Nel seguito, per il teorema appena enunciato, enunceremo alcune definizioni o dimostreremo alcuni teoremi validi sia per i problemi di ottimizzazione che per i problemi decisionali solo per problemi di ottimizzazione: il lettore ricaver\`a facilmente i risultati equivalenti per i problemi decisionali.
\begin{Definition}
	Sia $\MinimumProblem$ un problema di minimo con insieme ammissibile $\AdmissibleSet$ e funzione obiettivo $\ObjectiveFunction$. Fissato $k \in \mathbb{R}$, definiamo
	\Define{problema decisionale associato}[decisionale associato][problema] o
	\Define{versione decisionale}[di un problema di ottimizzazione][versione decisionale]
	il problema decisionale consistente nell'individuare l'esistenza di elementi nell'insieme
	$\AdmissibleSet_k = \lbrace \AdmissibleSolution \in \AdmissibleSet | \ObjectiveFunction(\AdmissibleSolution) \leq k$.
\end{Definition}
\begin{Definition}
	Sia $\MinimumProblem$ un problema di minimo. Un
	\Define{algoritmo di risoluzione}[di risoluzione di un problema][algoritmo] \`e
	un algoritmo che associa ad un'istanza di $\MinimumProblem$ il suo valore ottimo (da cui \`e poi generalmente possibile individuare una soluzione ottima). Un algoritmo capace di individuare il valore ottimo di qualsiasi istanza di un determinato problema si dice
	\Define{esatto}[algoritmo esatto][algoritmo].
\end{Definition}
\begin{Definition}
	Siano dati i seguenti problemi di minimo:
	$$(\MinimumProblem_1) \min \lbrace \ObjectiveFunction_1(\AdmissibleSolution): \AdmissibleSolution \in \AdmissibleSet_1 \rbrace,$$
	$$(\MinimumProblem_2) \min \lbrace \ObjectiveFunction_2(\AdmissibleSolution): \AdmissibleSolution \in \AdmissibleSet_2 \rbrace.$$
	Se abbiamo
	\begin{itemize}
		\item $\AdmissibleSet_1 \subseteq \AdmissibleSet_2$;
		\item $\ForAll{\AdmissibleSolution \in \AdmissibleSet_1}{\ObjectiveFunction_2(\AdmissibleSolution) \leq \ObjectiveFunction_1(\AdmissibleSolution)}$;
	\end{itemize}
	Allora diciamo che $\MinimumProblem_2$ \`e un
	\Define{rilassamento}[di un problema di ottimizzazione][rilassamento]
	di $\MinimumProblem_1$.
\end{Definition}
\begin{Definition}
	Dato un problema di ottimizzazione, definiamo
	\begin{itemize}
		\item \Define{variabile quantitativa}[quantitativa][variabile] una variabile che esprime una quantit\`a del problema;
		\item \Define{variabile decisionale}[decisionale][variabile] una variabile che esprime una decisione da prendere nella soluzione del problema;
		\item \Define{variabile booleana}[booleana][variabile] o \Define{variabile binaria}[binaria][variabile] o ancora \Define{variabile logica}[logica][variabile] una variabile decisionale che pu\`o valere $0$ -- equivalente al significato di falso -- o $1$ -- equivalente a vero --;
		\item \Define{variabile strutturale}[strutturale][variabile] una variabile decisionale essenziale del problema, togliendo la quale non \`e pi\`u possibile descriverne una soluzione (a meno di introdurne altre equivalenti);
		\item \Define{variabile ausiliaria}[ausiliaria][variabile] una variabile decisionale non indispensabile alla descrizione di una soluzione del problema, ma ci\`o non di meno utile per esempio ad esprimere vincoli.
	\end{itemize}
\end{Definition}

\section{Programmazione lineare.}\label{ProgrammazioneLineare}
\begin{Definition}
	Si chiama
	\Define{problema di programmazione lineare}[di programmazione lineare][problema]
	un problema di ottimizzazione tale che
	\begin{itemize}
		\item esistono $\ConstraintsNumber \in \mathbb{N}$ condizioni del tipo $a \cdot \AdmissibleSolution \left \lbrace \begin{array}{c} \leq\\=\\\geq \end{array} \right \rbrace b$, con $a \in \mathbb{R}^\ProblemDimension$ e $b \in \mathbb{R}$ e dove $\ProblemDimension$ \`e la dimensione del problema, tale che $\AdmissibleSolution \in \AdmissibleSet$ se e solo se $\AdmissibleSolution$ verifica tutte le condizioni: tali condizioni si chiamano \Define{vincoli lineari}[lineare][vincolo];
		\item la sua funzione obiettivo $\ObjectiveFunction$ sia della forma
		$\ObjectiveFunction(\AdmissibleSolution) = \CostsVector \cdot \AdmissibleSolution$, dove $\CostsVector \in \mathbb{R}^\ProblemDimension$ \`e un vettore caratteristico della particolare istanza del problema in esame detto \Define{vettore dei costi}[dei costi][vettore]: $\ObjectiveFunction$ \`e dunque una funzione lineare, pi\`u precisamente un funzionale.
	\end{itemize}
	Il problema si dice di \Define{programmazione lineare intera}[di programmazione lineare intera][problema] quando $\AdmissibleSet \subseteq \mathbb{Z}^\ProblemDimension$; si dice di \Define{programmazione lineare mista}[di programmazione lineare mista][problema] quando si richiede che solo alcune delle componenti di una soluzione ammissibile siano intere.
\end{Definition}
\begin{Definition}
	Dato un problema di ottimizzazione e una sua variabile $x$, il vincolo $x \in \mathbb{Z}$ si chiama \Define{vincolo di integrit\`a}[di integrit\`a][vincolo].
\end{Definition}
\begin{Theorem}
	Ogni problema di programmazione lineare di dimensione $\ProblemDimension$ e con $\ConstraintsNumber$ vincoli lineari pu\`o essere messo nella forma $\max\ \lbrace \CostsVector \cdot x : A\AdmissibleSolution \leq b \rbrace$, dove $\CostsVector,\AdmissibleSolution \in \mathbb{R}^\ProblemDimension$, $A \in \mathbb{R}^{\ConstraintsNumber \times \ProblemDimension}$, $b \in \mathbb{R}^\ConstraintsNumber$ e su $\ConstraintsNumber$ consideriamo l'ordinamento parziale definito da $\Coimplies{(x_i)_{i \in \ConstraintsNumber} \leq (y_i)_{i \in \ConstraintsNumber}}{\ForAll{i \in \ConstraintsNumber}{x_i \leq y_i}}$.
\end{Theorem}
\Proof Abbiamo gi\`a provato l'equivalenza tra problemi di massimo e di minimo tramite la moltiplifcazione per $-1$ della funzione obiettivo. Basta dunque provare che tutti i vincoli lineari possono essere riscritti nella forma $a \cdot \AdmissibleSolution \leq b$ per opportuni $a \in \mathbb{R}^\ProblemDimension$ e $b \in \mathbb{R}$.
\par Supponiamo dato il vincolo lineare $a \cdot \AdmissibleSolution \geq b$ ($a \in \mathbb{R}^\ProblemDimension$, $b \in \mathbb{R}$): esso \`e equivalente al vincolo $-a \cdot \AdmissibleSolution \leq -b$. Supponiamo dato il vincolo $a \cdot \AdmissibleSolution = b$ ($a \in \mathbb{R}^\ProblemDimension$, $b \in \mathbb{R}$): esso \`e equivalente alla congiunzione dei vincoli $a \cdot \AdmissibleSolution \leq b$ e $a \cdot \AdmissibleSolution \geq b$. \EndProof
\begin{Definition}
	Dati $a \in \mathbb{R}^\ProblemDimension$ e $b \in \mathbb{R}$, sia un vincolo lineare del tipo $a \cdot \AdmissibleSolution \leq b$ o $a \cdot \AdmissibleSolution \geq b$. La variabile $b - a \cdot \AdmissibleSolution$ nel primo caso e $a \cdot \AdmissibleSolution - b$ nel secondo caso si chiamano \Define{variabili di scarto}[di scarto][variabile].
\end{Definition}
\subsection{Esempi di programmazione lineare.}\label{EsempiDiProgrammazioneLineare}
\begin{Example}
	\TheoremName{Il problema dello zaino (\textit{Knapsack Problem})} Siano
	\begin{itemize}
		\item $I$ un insieme finito, detti \NIDefine{oggetti};
		\item $(a_i)_{i \in I} \in \mathbb{N}^I$ una famiglia di elementi, detti \NIDefine{pesi};
		\item $(c_i)_{i \in I} \in \mathbb{N}^I$ una famiglia di elementi, detti \NIDefine{costi};
		\item $b \in \mathbb{N}$ una costante, detta \NIDefine{massimo peso dello zaino}.
	\end{itemize}
	Lo scopo del problema \`e individuare un sottoinsieme $Z \subseteq I$ di oggetti (lo \NIDefine{zaino}) il cui peso non supera il peso $b$ e di massimo costo possibile.
	\par Il problema pu\`o dunque essere formulato come segue:
	$$(\KnapsackProblem)\ \max \lbrace \sum_{i \in Z} c_i: Z \in \AdmissibleSet \rbrace,$$
	dove $\AdmissibleSet = \lbrace X \subseteq I: \sum_{i \in X} a_i \leq b \rbrace$.
	\par Questo problema pu\`o essere riformulato come problema di programmazione lineare intero definendo, per ogni soluzione ammissible $Z$, la famiglia di variabili booleane $(x_i)_{i \in Z}$, coincidente con la funzione caratteristica di $Z$: allora la funzione obiettivo pu\`o essere scritta come $\CostsVector \cdot x$, dove $\CostsVector = (c_i)_{i \in I}$ e $x = (x_i)_{i \in Z}$, e l'insieme ammissibile pu\`o essere descritto come quel sottoinsieme di $\mathbb{Z}^{\Cardinality{I}}$ tale che $a \cdot x \leq b$, dove $a = (a_i)_{i \in I}$.
\end{Example}
\begin{Example}
	\TheoremName{Problema dell'albero di copertura di costo minimo (\textit{Minimal Spanning Tree})} Sia $\Graph = (\Nodes, \Edges)$ un grafo di nodi $\Nodes$ e archi $\Edges \subseteq \Nodes^2$; sia $(c_{i,j})_{(i,j) \in \Edges}$ una famiglia di numeri reali positivi che chiameremo \NIDefine{costo} dell'arco $(i,j)$. Vogliamo determinare un grafo parziale di $\Graph$ connesso e di costo minimo, definendo il costo del grafo come la somma dei costi dei suoi archi: si dimostra facilmente che un grafo con queste caratteristiche \`e un albero di copertura di $\Graph$, i.e. un grafo connesso e privo di cicli.
	\par Possiamo dunque definire il problema con
	$$(\MinimalSpanningTree)\ \min \lbrace \sum_{(i,j) \in \Edges'} c_{i,j} : (\Nodes,\Edges')\text{ \`e un albero di copertura di }\Graph \rbrace.$$
	\par Determinare un sottoinsieme $\Edges' \subseteq \Edges$ equivale a definire una famiglia $(x_{(i,j)})_{(i,j) \in \Edges}$ coincidente con la funzione caratteristica di $\Edges'$. A questo punto l'insieme ammissibile del problema pu`\o essere ridefinito come l'insieme di tutte le possibile famiglie $(x_{(i,j)})_{(i,j) \in \Edges}$ e la condizione di connessione come l'insieme (finito) di condizioni $\sum_{i \in S\\j \in \Nodes \SetMin S} x_{(i,j)} \geq 1$ per ogni sottoinsieme proprio $S$ non vuoto di $\Nodes$. Il problema diventa cos\`i un problema di programmazione lineare intera.
\end{Example}
\begin{Example}
	\TheoremName{Il problema del commesso viaggiatore (\textit{Travelling Salesman Problem})}
	Dato un grafo completo $\Graph = (\Nodes, \Nodes^2)$ e una famiglia di numeri reali positivi $(c_{i,j})_{(i,j) \in \Nodes^2}$, detti \NIDefine{costi}, si ricerca un ciclo hamiltoniano di costo minimo. Ci\`o pu\`o essere modellizzato introducendo le stesse variabili $(x_{(i,j)})_{(i,j) \in \Nodes^2}$ utilizzate nell'esempio del problema dell'albero di copertura di costo minimo, imponendo la stessa restrizione $\sum_{i \in S\\j \in \Nodes \SetMin S} x_{(i,j)} \geq 1$ per garantire la connessione del ciclo hamiltoniano e aggiungendo, per ogni nodo $i \in \Nodes$, la condizione $\sum_{j \in \Nodes} x_{i,j} = 2$ per garantire che ogni nodo sia estremo di due e solo due archi; per funzione obiettivo si pu`\o usare la stessa usata per il problema dell'albero di copertura di costo minimo.
	\par Osserviamo che con le formulazioni fornite il problema dell'albero di copertura di costo minimo costituisce un rilassamento del problema del commesso viaggiatore.
\end{Example}
\par Alcuni problemi, per essere espressi in termini di programmazione lineare, richiedono di combinare valori booleani diversi secondo le classiche regole della logica booleana. A tal fine \`e utile tenere a mente le relazioni seguenti:
\begin{itemize}
	\item se $a$ \`e una variabile booleana, $\Not{a}$ \`e rappresentata da $1 - a$;
	\item se $a$ e $b$ sono variabili booleane, $a \rightarrow b$ equivale alla relazione $a \leq b$;
	\item se $a$ e $b$ sono variaibli booleane, $c = \Or{a}{b}$ equivale alle relazioni
	\begin{itemize}
		\item $a \leq c$;
		\item $b \leq c$;
		\item $c \leq a + b$ (si rammenta che $c \in \lbrace 0, 1 \rbrace$, per cui non \`e possibile $c = 2$ nel caso $a = b = 1$);
	\end{itemize}
	\item se $a$ e $b$ sono variaibli booleane, $c = \Xor{a}{b}$ equivale alle relazioni
	\begin{itemize}
		\item $a - b \leq c$;
		\item $b - a \leq c$;
		\item $c \leq a + b$;
		\item $c \leq 2 - a - b$;
	\end{itemize}
	\item se $a$ e $b$ sono variaibli booleane, $c = \And{a}{b}$ equivale alle relazioni
	\begin{itemize}
		\item $c \leq a$;
		\item $c \leq b$;
		\item $c \geq a + b - 1$.
	\end{itemize}
\end{itemize}
\begin{Example}
	\TheoremName{Il problema della soddisfattibilit\`a proposizione (\textit{Satisfatibility problem})} Sia $(\Litteral_j)_{j \in n}$ ($n \in \mathbb{N}$) una famiglie di letterali. Sia $(C_i)_{i \in m}$ ($m \in \mathbb{N}$) una famiglia di proposizioni che sono la disgiunzione di letterali di $(\Litteral_j)_{j \in n}$ o di loro negazioni. Sia infine $\Formula$ la congiunzione dei $(C_i)_{i \in m}$. Il problema richiede di determinare se esistono valori dei letterali $\Litteral_j$ tali da rendere vera $\Formula$: la sua importanza risiede nella possibilit\`a di mettere qualsiasi formula del calcolo proposizionale in forma normale congiuntiva.
	\par Introduciamo la matrice $\Matrix = (\Matrix_{i,j})_{(i,j) \in m \times n}$: $\Matrix_{i,j} = \begin{cases} 1 \text{ se }\Litteral_j\text{ \`e presente in } C_i; -1 \text{ se }\Litteral_j\text{ \`e presente negato in } C_i;0 \text{altrimenti}.\end{cases}$
	\par Ora, se $X \in \mathbb{R}^n$ \`e un vettore booleano le cui componenti sono i valori di verit\`a dei letterali $\Litteral_j$, $C_i$ equivale a $\Transposed{\Matrix_i} X \geq 1 - n(i)$, dove $n(i)$ \`e il numero di letterali negati in $C_i$.
	\par Il problema di programmazione lineare pu`\o dunque essere definito usando come insieme ammissibile l'insieme di tutti gli $X$ che soddisfano la disugualgianza $\Transposed{\Matrix_i} X \geq 1 - n(i)$, per ogni $i \in n$. Per funzione obiettivo si pu\`o scegliere una qualsiasi funzione costante (in effetti qui siamo di fronte ad un problema di decisione pi\`u che di ottimizzazione).
\end{Example}
\begin{Definition}
	Siano $A$ e $B$ due insiemi finiti e sia $(x_{i,j})_{(i,j) \in A \times B}$ una famiglia di variabili booleane. Chiamiamo
	\begin{itemize}
	\item $\Define{vincoli di semiassegnamento}[di semiassegnamento][vincoli]$ le equazioni $\sum_{j \in B} x_{i,j} = 1$ al variare di $i \in A$;
	\item $\Define{vincoli di assegnamento}[di assegnamento][vincoli]$ le equazioni $\sum_{j \in B} x_{i,j} = 1$ al variare di $i \in A$ e $\sum_{i \in A} x^{i,j} = 1$ al variare di $j \in B$: $(x_{i,j})_{(i,j) \in A \times B}$ risulta cos\`i essere una matrice di permutazione.
	\end{itemize}
\end{Definition}
\begin{Example}
	\TheoremName{Il problema di minimizzazione del numero delle macchine (\textit{Minimal Cardinality Machine Scheduling})} Siano
	\begin{itemize}
		\item $N$ un insieme finito di lavori;
		\item $(d_i)_{i \in N}$ la durata del lavoro $i \in N$;
		\item $(t_i)_{i \in N}$ il momento in cui deve iniziare il lavoro $i \in N$.
	\end{itemize}
	Sapendo che ogni lavoro deve essere svolto da una macchina e che ogni macchina pu\`o svolgere al massimo un lavoro per volta, si chiede quale sia il numero minimo di macchine necessario per compiere i lavori previsti nei termini previsti.
	\par Osserviamo per prima cosa che se mettiamo a disposizione tante macchine quanti sono i lavori possiamo senz'altro eseguire tutti i lavori: introduciamo dunque un insieme di macchine $M$ equicardinale ad $N$ e consideriamo il problema equivalente di determinare quale sia il numero minimo di macchine di $M$ necessario per compiere tutti i lavori. Introduciamo le variabili booleane $(x_{i,j})_{(i,j) \in N \times M}$: $x_{i,j} = \begin{cases} 1\text{ se il lavoro }i\text{ \`e assegnato alla macchina }j;\\0\text{ altrimenti.}\end{cases}$
	\par Il fatto che un lavoro venga eseguito una sola volta, e dunque da una sola macchina, si esprime coi vincoli di semiassegnamento $\sum_{j \in M}x_{i,j} = 1$ per ogni $i \in N$.
	\par Per ogni lavoro $i \in N$ definiamo l'insieme $S_i$ dei lavori che non possono essere eseguiti dalla stessa macchina che esegue $i$: $S_i = \lbrace h \in N: [t_i;t_i + d_i] \cap [t_h;t_h + d_h] \neq \emptyset \rbrace$. Poniamo allora ulteriori vincoli per garantire che ogni macchina esegua solo un lavoro per volta: per ogni $(i,j) \in N \times M$ e per ogni $h \in S_i$ poniamo $x_{i,j} + x_{h,j} \leq 1$. Osserviamo che per ogni coppia di lavori $(i,h) \in N^2$ abbiamo introdotto lo stesso vincolo due volte (la cosa sarebbe facilmente evitabile ordinando $N$ e introducendo il vincolo solo quando $i < h$).
	\par L'insieme ammissible $\AdmissibleSet$ \`e dunque il sottoinsieme di tutte le applicazioni $x_{i,j}: N \times M \rightarrow \lbrace 0, 1\rbrace$ che rispetta tutti i vincoli dati.
	\par Introduciamo ora ulteriori variabili booleane $(y_j)_{j \in M}$, vere se la macchina $j \in M$ risulta usata. Possiamo dunque esprimere la funzione obiettivo come $\ObjectiveFunction = \sum_{j \in M} y_j$.
	\par Per esprimere che la macchina $j \in M$ \`e usata, imponiamo $ny_j \geq \sum_{i \in N} x_{i,j}$: non si tratta di ulteriori vincoli per definire l'insieme ammissibile, ma solo di una disequazione che determina i valori $(y_j)_{j \in M}$ in maniera univoca. Le variabili $(y_j)_{j \in M}$ sono variabili ausiliari, in opposizione alle variabili $(x_{i,j})_{(i,j) \in N \times M}$ invece strutturali.
	Il problema si definisce dunque come
	$$(\MinimalCardinalityMachineScheduling)\ \min \lbrace \sum_{j \in M} y_j: (x_{i,j})_{(i,j) \in N \times M} \in \AdmissibleSet \rbrace.$$
\end{Example}
\begin{Example}
	\TheoremName{Il problema della colorazione di grafo (\textit{Graph Coloring})} Sia $\Graph = (\Nodes,\Edges)$ un grafo con $n = \Cardinality{\Nodes} \in \NotZero{\mathbb{N}}$ nodi: vogliamo associare ad ogni nodo un elemento di un insieme $C$, per esempio un insieme di colori (da qui il nome del problema) in modo tale che a due nodi adiacenti non sia associato lo stesso elemento e usando il minor numero di elementi di $C$ possibile.
	\par Come per il problema $\MinimalCardinalityMachineScheduling$, possiamo assumere $\Cardinality{C} = \Cardinality{\Nodes}$. Inoltre, introduciamo le variabili booleane $(x_{i,c})_{(i,c) \in \Nodes \times C}$ tali che $x_{i,j} = \begin{cases} 1\text{ se }j\text{ \`e associato a }i,\\0\text{ altrimenti.}\end{cases}$.
	\par Ad ogni nodo deve essere associato un solo elemento di $C$, per cui imponiamo i vincoli di semiassegnamento $\ForAll{i \in \Nodes}{\sum_{c \in C} x_{i,c} = 1}$.
	\par Inoltre a due nodi adiacenti non deve essere associato lo stesso elemento di $C$: imponiamo dunque i vincoli $\ForAll{(i,j) \in \Nodes^2}{\ForAll{c \in C}{x_{i,c} + x_{j,c} \leq 1}}$.
	\par Scegliamo dunque come insieme ammissibile $\AdmissibleSet$ l'insieme $\lbrace 0, 1\rbrace^{\Nodes \times C}$ con i vincoli sopraindicati.
	\par Di nuovo in analogia col problema $\MinimalCardinalityMachineScheduling$, introduciamo le variabili booleane $(y_c)_{c \in C}$: vogliamo che esse risultino vere quando $c \in C$ \`e associato a qualche $i \in \Nodes$; otteniamo ci\`o imponendo $\ForAll{c \in C}{\ForAll{i \in \Nodes}{y_c \geq x_{i,c}}}$.
	\par Il problema si definisce dunque come
	$$(\GraphColoring)\ \min \lbrace \sum_{j \in M} y_j: (x_{i,j})_{(i,j) \in \Nodes \times C} \in \AdmissibleSet \rbrace.$$
\end{Example}
\begin{Definition}
	Siano $\Omega$ un insieme finito e $P \subseteq \PowerSet{\Omega}$. Siano le variabili booleane $(x_p)_{p \in P}$. Chiamiamo
	\begin{itemize}
		\item \Define{vincoli di copertura}[di copertura][vincoli]: $\ForAll{\omega \in \Omega}{\sum_{p \in P\\\omega \in p} x_p \geq 1}$;
		\item \Define{vincoli di partizione}[di partizione][vincoli]: $\ForAll{\omega \in \Omega}{\sum_{p \in P\\\omega \in p} x_p = 1}$;
		\item \Define{vincoli di riempimento}[di riempimento][vincoli]: $\ForAll{\omega \in \Omega}{\sum_{p \in P\\\omega \in p} x_p \leq 1}$.
	\end{itemize}
\end{Definition}
\begin{Example}
	\TheoremName{Problemi di copertura, partizione e riempimento} Siano $\Omega$ un insieme finito e $P \subseteq \PowerSet{\Omega}$. Siano inoltre definiti i numeri $(c_p)_{p \in \Omega}$, detti \emph{costi}. Vogliamo trovare, se esiste, $S \subseteq P$ tale che $\bigcup{s \in S}s = \Omega$ e tale che il suo costo, definito come $\sum_{s \in S} c_s$ sia minimo: \`e il problema di copertura.
	\par Introduciamo le variabili booleane $(x_p)_{p \in P}$, da considerarsi vere se e solo se selezioniamo $p \in P$ per entrare in $S$. Possiamo allora definire il problema come
$$(\CoveringProblem)\ \min \lbrace \sum_{p \in P} c_p x_p: (x_p)_{p \in P} \in \AdmissibleSet \rbrace,$$
	dove $\AdmissibleSet$ \`e l'insieme di tutte le famiglie $(x_p)_{p \in P}$ che rispettano i vincoli di copertura.
	\par Aggiungendo alla definizione del problema di copertura l'ulteriore condizione che elementi distinti di $S$ siano disgiunti otteniamo il problema di partizione $\PartitioningProblem$, che risolviamo allo stesso modo sostituendo ai vincoli di copertura quelli di partizionamento.
	\par Infine, se trasformiamo il problema di partizione togliendo la condizione $\bigcup{s \in S}s = \Omega$, otteniamo il problema di riempimento $\FillingProblem$, che risolviamo sostituendo ai vincoli di partizionamento quelli di di riempimento.
\end{Example}
\begin{Definition}
	Data una famiglia di variabili quantitative $(x_i)_{i \in I}$ e un'ulteriore variabile $V$, chiamiamo \Define{vincoli soglia}[di soglia][vincoli] disuguaglianze del tipo $V \leq x_i$ o $x_i \leq V$, con $i \in I$
\end{Definition}
\begin{Example}
	\TheoremName{Il problema del minor tempo di produzione su macchine (\textit{Minimal Makespan Machine Scheduling})} Siano dati un insieme di lavori $N$ e un insieme di macchine $M$: ogni macchina pu\`o eseguire un solo lavoro per volta. Ad ogni lavoro \`e associata una durata $(d_i)_{i \in N}$. A differenza di quanto supporto per il problema $\MinimalCardinalityMachineScheduling$, in questo caso supponiamo che tutti i lavori possano essere svolti in qualsiasi momento. Si chiede di distribuire i lavori in maniera tale da vederli tutti completati il prima possibile.
	\par Introduciamo le variabili booleane $(x_{i,j})_{(i,j) \in N \times M}$: $x_{i,j}$ \`e vera se e solo se il lavoro $i$ \`e associato alla macchina $j$. Come insieme ammissibile $\AdmissibleSet$ scegliamo l'insieme di tutte le famiglia $(x_{i,j})_{(i,j) \in N \times M}$ che rispettano i vincoli di semiassegnamento $\ForAll{i \in N}{\sum_{j \in M} x_{i,j} = 1}$ (ogni lavoro deve essere svolto da una sola macchina).
	\par Introduciamo la variabile ausiliaria $t$ e i vincoli di soglia $\ForAll{j \in M}{\sum_{i \in N} d_ix_{i,j} \leq t}$: imponiamo cio\`e che $t$ sia grande almeno quanto la durata di svolgimento di tutti i lavori. Minimizzando $t$, minimizziamo dunque anche la durata dei lavori:
$$(\MinimalMakespanMachineScheduling)
\begin{array}{ll}
	\min &t,\\
	&\ForAll{i \in N}{\sum_{j \in M} x_{i,j} = 1},\\
	&\ForAll{j \in M}{\sum_{i \in N} d_ix_{i,j} \leq t},\\
	&\ForAll{(i,j) \in N \times M}{x_{i,j} \in \lbrace 0, 1 \rbrace}.
\end{array}$$
\end{Example}
\begin{Example}
	In alcuni problemi pu\`o capitare di trovarsi di fronte ad una funzione lineare a tratti
	$$f(x) = \begin{cases}
	b_0 + c_0x\text{ se }x \in [a_0,a_1],\\
	b_1 + c_1x\text{ se }x \in (a_1,a_2],\\
	...\\
	b_{n - 1} + c_{n - 1}x\text{ se }x \in [a_{n - 1},a_n],
	\end{cases}$$
	con $\ForAll{i \in n}{a_i \leq a_{i + 1}}$.
	\`E possibile trasformare $f(x)$ in una funzione lineare. Introduciamo le variabili booleane $(y_i)_{i \in n}$ e le variabili quantitative $(z_i)_{i \in n}$ e imponiamo i seguenti vincoli:
	\begin{itemize}
		\item $\Implies{x = a_0}{y_0 = 1}$ e $\ForAll{i \in n}{\Implies{x \in (a_i,a_{i + 1}]}{y_i = 1}}$: questi vincoli garantiscono che $y_i$ sia vero se e solo se $x$ appartiene all'$i$-esimo intervallo di definizione;
		\item $\ForAll{i \in n}{0 \leq z_i \leq (a_{i + 1} - a_i)y_i}$: gli $z_i$ sono dunuqe grandezze reali positive grandi al pi\`u quando la lunghezza dell'$i$-esimo intervallo e certamente nulle se $x$ non appartiene all'$i$-esimo intervallo.
	\end{itemize}
	\par Abbiamo dunque $x = \sum_{i \in n}{a_iy_i + z_i} = \sum_{i \in n}{a_iy_i} + \sum_{i \in n} z_i$: $x$ \`e funzione lineare degli $(y_i)_{i \in n}$ e dei $(z_i)_{i \in n}$.
	\par Inoltre $f(x) = g(z_0,...,z_n,y_0,...,y_n) = \sum_{i \in n}b_iy_i + c_i(a_iy_i + z_i) = \sum_{i \in n}(b_i + c_ia_i)y_i + \sum_{i \in n}{c_iz_i}$: come si vede, il secondo membro \`e una funzione lineare dei $(y_i)_{i \in n}$ e dei $(z_i)_{i \in n}$.
	\par Nel calcolare massimi o minimi di $f(x)$ occorre essere particolarmente prudenti nei punti di discontinuit\`a: ciascuno di essi pu\`o essere rappresentato attravero due diversi assegnazioni degli argomenti di $g$, ma solo una di queste assegnazioni corrisponde ad un effettivo valore di $f(x)$.
\end{Example}
\begin{Example}
	Riprendiamo la funzione $f$ dell'esempio precedente e supponiamo stavolta che essa sia continua e che la successione dei $(c_i)_{i \in n}$ sia non decresente.
	\par Allora \`e possibile sostituire $f$ con una funzione lineare $g$ introducendo stavolta solo una famiglia di variabili $(z_i)_{i \in n}$. In tal caso, i vincoli sono dati da $\ForAll{i \in n}{0 \leq z_i \leq a_{i + 1} - a_i}$ e poniamo $g(z_0,...,z_{n - 1}) = b_0 + \sum_{i \in n}c_iz_i$: se $g$ ha minimo in $(z_i)_{i \in n}$, allora abbiamo
	\begin{itemize}
		\item per opportuno $h \in n$, $z_i = \begin{cases}a_{i + 1} - a_i\text{ se }i < h,\\0\text{ se }i > h;\end{cases}$
		\item $f$ ha minimo in $x = a_0 + \sum_{i \in n} z_i$.
	\end{itemize}
	\par La prima affermazione discende dal fatto che, per la crescenza della successione $c_i$, se sostituiamo in $g$ una successione $(z_i)_{i \in n}$ arbitraria con $(z_i')_{i \in n}$ nella forma descritta della prima affermazione e tale che $\sum_{i \in n} z_i = \sum_{i \in n} = z_i'$, allora il valore di $g$ cala.
	\par La seconda affermazione
\end{Example}
\begin{Definition}
	Fissato $n \in \mathbb{N}$, siano dati numeri reali $(A_i)_{i \in n}$ e $(b_i)_{i \in n}$ e supponiamo che un problema sia sottoposto ai vincoli $\ForAll{i \in n}{A_ix \leq b}$
\end{Definition}



	\chapter{Istituzioni di analisi matematica}
	\section{Teorema di Perron-Frobenius.}
\label{CalcoloScientifico_TeoremaDiPerronFrobenius}
\begin{Theorem}
  \TheoremName{Teorema di Perron-Frobenius}[di Perron-Frobenius][teorema]
  Siano
  \begin{itemize}
    \item $n \in \NotZero{\mathbb{N}}$;
    \item $\Matrix \in \mathbb{C}^{n \times n}$ non negativa.
  \end{itemize}
  Allora
  \begin{itemize}
    \item $\SpectralRadius{\Matrix} \in \Spectrum{\Matrix}$;
    \item esiste un autovettore sinistro $\VarEigenvector$ tale che
      $\ForAll{k \in n}{\VarEigenvector \geq 0}$;
    \item esiste un autovettore destro $\Eigenvector$ tale che
      $\ForAll{k \in n}{\Eigenvector \geq 0}$.
  \end{itemize}
  Inoltre,
  \begin{itemize}
    \item se $\Matrix$ \`e irriducibile, allora
      \begin{itemize}
        \item $\SpectralRadius{\Matrix}$ \`e un autovalore semplice;
        \item esiste un autovettore sinistro $\VarEigenvector$ tale che
          $\ForAll{k \in n}{\VarEigenvector > 0}$;
        \item esiste un autovettore destro $\Eigenvector$ tale che
          $\ForAll{k \in n}{\Eigenvector > 0}$;
      \end{itemize}
      \item se $\Matrix$ \`e positiva, allora
        $\SpectralRadius{\Matrix}$
        \`e l'unico autovalore di modulo massimo di $\Matrix$.
  \end{itemize}
\end{Theorem}

\section{\English{Page rank}.}
\label{CalcoloScientifico_PageRank}
Il modello denominato
\Define{\English{page rank}}
\`e un modello nato per classificare l'importanza delle pagine
\English{web}, da cui il nome, ma che pu\`o essere riproposto per ogni
rete. Noi lo presenteremo nel suo contesto originale delle reti.
\par Sia $\PagesNumber$ il numero totale di tutte le pagine di una rete data.
Identificheremo ogni pagina con un numero naturale da $0$ a
$\PagesNumber - 1$, dove $\PagesNumber$ \`e il numero delle pagine stesse.
\par Esso viene costruito a partire da un grafo orientato
$\Graph = (\PagesNumber,\Edges)$ dove
$(\Node,\VarNode) \in \PagesNumber \times \PagesNumber$ appartiene a $\Edges$
se e solo se la pagina $\Node$ contiene un collegamento ipertestuale
alla pagina $\VarNode$.
\par Denotiamo $\AdjacencyMatrix$ la matrice di adiacenza di $\Graph$.
\par Lo scopo del modello di \English{page rank} \`e determinare un
\Define{vettore di pesi}[dei pesi (\English{page rank})][vettore]
$\Weights$: la componente $\Weights_k$ indica l'importanza della pagina $k$.
\par Definiamo i vettori
\begin{itemize}
  \item $\AllOnes = \Transposed{(1 \cdots 1)}$;
  \item $\PageDegree = \AdjacencyMatrix \AllOnes$;
  \item $\MatrixPageDegree \in \mathbb{R}^{\PagesNumber \times \PagesNumber}$
    tale che
    \[
      \ForAll{(j,k) \in \PagesNumber \times \PagesNumber}{
        \MatrixPageDegree_j^k =
        \begin{cases}
          0&\text{ se } j \neq k,\\
          \PageDegree_k&\text{ se } j = k;
        \end{cases}}
    \]
  \item $\Matrix = \MatrixPageDegree^{-1}\AdjacencyMatrix$.
\end{itemize}
La componente $\PageDegree_k$ ($k \in \PagesNumber$) indica il grado uscente
della pagina $k$.
\par Vorremmo costruire un modello per cui ogni pagina trasmette equamente
la sua importanza a tutte le pagine da esse puntata e l'importanza di una pagina
\`e esattamente la somma delle importanze cos\`i trasmesse tramite i
collegamenti.
\par Come vedremo, \`e utile ipotizzare che non esistano pagine pendenti
nella rete in esame; questa ipotesi \`e facilmente realizzabile con una delle due strategie seguenti, nessuna delle quali cambia sostanzialmente
il senso del modello della rete:
\begin{itemize}
  \item aggiungiamo ad ogni pagina pendente un collegamento a tutte
    le altre pagine;
  \item intoduciamo una pagina aggiuntiva $\PagesNumber$ che punta a tutte 
  le altre pagine ed \`e puntata da tutte le altre pagine:
  \[
    \ForAll{k > 0}{\AdjacencyMatrix_\PagesNumber^k
    = \AdjacencyMatrix_k^\PagesNumber = 1}.
  \]
\end{itemize}
\par Assumeremo dunque d'ora in poi il seguente assioma.
\begin{Axiom}
  La rete non contiene nodi pendenti.
\end{Axiom}
\begin{Theorem}
  $\Weights$ \`e autovettore sinistro di $\Matrix$.
\end{Theorem}
\Proof Per ogni pagina $k \in \PagesNumber$, abbiamo
\[
  \Weights_k = \sum_{j \in \PagesNumber} \AdjacencyMatrix_j^k \frac{\Weights_j}{\PageDegree_j},
\]
equivalente a
\[
  \Weights = \MatrixPageDegree^{-1}\AdjacencyMatrix\Weights,
\]
vale a dire
\[
  \Transposed{\Weights} = \Transposed{\Weights}\Matrix.\text{ \EndProof}
\]
\begin{Theorem}
  Abbiamo
  \begin{itemize}
    \item $1$ \`e autovalore di $\Matrix$ relvativo a $\AllOnes$;
    \item $\SpectralRadius{\Matrix} = 1$.
  \end{itemize}
\end{Theorem}
\Proof Abbiamo
$\Matrix\AllOnes
= \MatrixPageDegree^{-1}\AdjacencyMatrix\AllOnes
= \MatrixPageDegree^{-1}\PageDegree
= \AllOnes$.
\par Inoltre,
$\Norm{\Matrix}[\infty]
= \max_{k \in \PagesNumber} \sum_{j \in \PagesNumber}
  \AbsoluteValue{\Matrix_k^j}
= \max_{k \in \PagesNumber} \sum_{j \in \PagesNumber}
  \Matrix_k^j
= \max_{k \in \PagesNumber} \sum_{j \in \PagesNumber}
  \frac{\AdjacencyMatrix_k^j}{\PageDegree_k}
= 1$,
e quindi, se $\Eigenvector \in \mathbb{C}^\PagesNumber$ \`e autovettore
di $\Matrix$ relativo a $\Eigenvalue$, allora
$\AbsoluteValue{\Eigenvalue}\Norm{\Eigenvector}[\infty]
= \Norm{\Eigenvalue\Eigenvector}[\infty]
= \Norm{\Matrix\Eigenvector}[\infty]
\leq \Norm{\Matrix}[\infty]\Norm{\Eigenvector}[\infty]$,
da cui
$\AbsoluteValue{\Eigenvalue}
\leq \Norm{\Matrix}[\infty] = 1$. \EndProof
\begin{Theorem}
  Se $\Matrix$ ha un unico autovalore di modulo massimo, qualsiasi successione
  $(\Weights^{(k)})_{k \in \mathbb{N}}$ tale che
  \begin{itemize}
    \item $\Weights^{(0)}$ non \`e ortogonale a $\AllOnes$;
    \item per ogni $k > 0$,
      $\Weights^{(k)} = \Transposed{\Matrix}\Eigenvector^{(k - 1)}$;
    \item se $\Weights^{(0)}$ non ha componenti negative, allora
      $\ForAll{k \in \mathbb{N}}{
      \Norm{\Weights^{(k)}}[1] = \Norm{\Weights^{(0)}}[1]}$;
  \end{itemize}
  converge a un autovettore sinistro di $\Matrix$, cio\`e a $\Weights$
  a meno di prodotto per scalare.
\end{Theorem}
\Proof Riscriviamo il metodo delle potenze applicandolo alla matrice
$\Transposed{\Matrix}$ e al vettore $\Weights^{(0)}$ con una qualsiasi
tolleranza $\Tollerance$:
\begin{enumerate}
  \item\label{PageRank_MetodoDellePotenze_1} si ponga $j = 0$,
    $\Weights_0 = \Weights^{(0)}$,
    $\Eigenvalue_0^{(0)} =
    = \Transposed{\Weights_0}\Matrix\Weights_{0}$;
  \item\label{PageRank_MetodoDellePotenze_2} si ponga
    $\tilde{\Weights}_{j + 1} = \Matrix \Weights_j$;
  \item\label{PageRank_MetodoDellePotenze_3} si ponga
    $\Weights_{j + 1}
    = \frac{\tilde{\Weights}_{j + 1}}{\Norm{\tilde{\Weights_{j + 1}}}}$;
  \item\label{PageRank_MetodoDellePotenze_4} si ponga $\Eigenvalue_0^{(j + 1)}
    = \Transposed{\Weights_{j + 1}}\Matrix\Weights_{j + 1}$;
  \item\label{PageRank_MetodoDellePotenze_5} se
    $\AbsoluteValue{\Eigenvalue_0^{(j + 1)} - \Eigenvalue_0^{(j)}}
    > \Tollerance$
    si incrementi $j$ di $1$ e si torni al punto
    \ref{PageRank_MetodoDellePotenze_2}, altrimenti l'algoritmo termina
    restituendo la coppia $(\Eigenvalue_0^{(j + 1)},\Weights_{j + 1})$.
\end{enumerate}
\par Per ogni $k \in \mathbb{N}$, abbiamo
$\Weights_k = \Weights^{(k)}$ a meno di moltiplicazione per scalare.
\par Supponiamo $\Weights^{(0)}$ non abbia nessuna componente negativa.
Si dimostra immediatamente per induzione, utilizzando anche la non negativit\`a
di $\Matrix$, che per ogni $k \in \mathbb{N}$,
$\Norm{\Weights^{(k)}}[1]
= \Transposed{\Weights^{(k)}}\AllOnes
= \Transposed{\Weights^{(0)}}\AllOnes
\Norm{\Weights^{(0)}}[1]$. \EndProof
\par Con riferimento a quest'ultimo teorema, grazie alla costanza della
norma dei vettori della successione
$(\Weights^{(k)})_{k \in \mathbb{N}}$
\`e possibile calcolare direttamente l'autovettore sinistro $\Weights$
senza il passo di normalizzazione \ref{PageRank_MetodoDellePotenze_3}
evitando qualsiasi rischio di traboccamento dei moduli.
\par Definiamo adesso
\begin{itemize}
  \item $\gamma \in (0,1)$;
  \item $\Vector \in \mathbb{R}^\PagesNumber$ tale che
    $\ForAll{k \in \PagesNumber}{\Vector_k > 0}$ e
    $\Norm{\Vector}[1] = 1$;
  \item $A = \gamma\Matrix + (1 - \gamma)\AllOnes\Transposed{\Vector}$.
\end{itemize}
\begin{Theorem}
  La matrice $A$ \`e positiva.
\end{Theorem}
\Proof Fissiamo $(k,j) \in \PagesNumber \times \PagesNumber$:
$A_k^j$ \`e combinazione convessa di $\Matrix_k^j$ e
$\sum_{k \in \PagesNumber} \Vector_k$ con coefficienti non nulli,
dunque \`e strettamente positivo. \EndProof
\begin{Theorem}
  $\AllOnes$ \`e autovettore di $A$ relativo all'autovalore $1$.
\end{Theorem}
\Proof Abbiamo
$A\AllOnes
= \gamma\Matrix\AllOnes + (1 - \gamma)(\AllOnes\Transposed{\Vector})\AllOnes
= \gamma\AllOnes
  + (1 - \gamma)\AllOnes(\Transposed{\Vector}\AllOnes)
= \gamma\AllOnes + (1 - \gamma)\AllOnes
= \AllOnes$. \EndProof
\begin{Theorem}
  \TheoremName{Teorema di Brauer}[di Brauer][teorema]
  Siano
  \begin{itemize}
    \item $n \in \NotZero{\mathbb{N}}$;
    \item $J \in \mathbb{C}^{n \times n}$;
    \item $\Eigenvector$ autovettore di $J$ relativo all'autovalore
      $\Eigenvalue \in \Spectrum{J}$;
    \item $\VarVector \in \mathbb{C}^n$;
    \item $K = J + \Eigenvector\HermitianTransposed{\VarVector}$.
  \end{itemize}
  Abbiamo
  \[
    \Spectrum{K}
    = \Spectrum{J}
      \SetMin \lbrace \Eigenvalue \rbrace
      \cup \lbrace \Eigenvalue + \HermitianTransposed{\VarVector}\Eigenvector \rbrace.
  \]
\end{Theorem}
\Proof Sia $\UnitaryMatrix \in \mathbb{C}^n$ unitaria tale che
$\SchurNormalForm = \HermitianTransposed{\UnitaryMatrix} J \UnitaryMatrix$
sia una forma normale di Schur di $J$ tale che
$\SchurNormalForm_0^0 = \Eigenvalue$.
\par Denotando $\CanonicalBase^0$ il primo vettore della base canonica, abbiamo
\begin{align}
  \HermitianTransposed{\UnitaryMatrix} K \UnitaryMatrix
  &= \HermitianTransposed{\UnitaryMatrix} J \UnitaryMatrix
    + \HermitianTransposed{\UnitaryMatrix}
      \Eigenvector\HermitianTransposed{\VarVector}
      \UnitaryMatrix,\\
  &= \SchurNormalForm
    + \CanonicalBase^0\HermitianTransposed{\VarVector}
      \UnitaryMatrix.
\end{align}
\par Sia ora $k \in n$. Supponiamo $k = 0$: abbiamo
\[
  (\CanonicalBase^0\HermitianTransposed{\VarVector}\UnitaryMatrix)_0
  = \HermitianTransposed{\VarVector}\UnitaryMatrix.
\]
Se invece $k \neq 0$, allora, per ogni $j \in n$
\[
  (\CanonicalBase^0\HermitianTransposed{\VarVector}\UnitaryMatrix)_k
  = 0.
\]
Dunque tutti gli elementi diagonali della matrice triangolare superiore
$\HermitianTransposed{\UnitaryMatrix} K \UnitaryMatrix$,
forma normale di Schur,
coincidono con quelli di $\SchurNormalForm$ tranne
$(\HermitianTransposed{\UnitaryMatrix} J \UnitaryMatrix)_0^0$
che viene sostituito da
\[
  \Eigenvalue + \HermitianTransposed{\VarVector}\UnitaryMatrix\CanonicalBase^0
  = \Eigenvalue + \HermitianTransposed{\VarVector}\Eigenvector.
  \text{ \EndProof}
\]
\begin{Theorem}
  $1$ \`e autovalore semplice e di modulo massimo di $A$.
\end{Theorem}
\Proof Ricordiamo che
$A = \gamma\Matrix + (1 - \gamma)\AllOnes\Transposed{\Vector}$.
$\AllOnes$ \`e autovettore relativo a $1$ di $\Matrix$ e $1$ \`e autovalore
semplice e di modulo massimo di $\Matrix$ per il teorema di Perron-Frobenius.
Ne deduciamo che
$\AllOnes$ \`e autovettore relativo a $\gamma$ di $\gamma\Matrix$ e
$\gamma$ \`e autovalore semplice e di modulo massimo di $\gamma\Matrix$
per il teorema di Perron-Frobenius.
\par Abbiamo
$\gamma + (1 - \gamma)\Transposed{\AllOnes}\Vector
= \gamma + (1 - \gamma)\Norm{\Vector}[1] = 1$.
Dunque, per il teorema di Bauer, $1$ \`e autovalore semplice e di modulo
massimo di $A$. \EndProof

\section{Vibrazioni.}
\label{CalcoloScientifico_Vibrazioni.}
\subsubsection{Vibrazioni di corde elastiche.}
\label{CalcoloScientifico_VibrazioniDiCordeElastiche}
\par Consideriamo una successione finita di
$N \in \NotZero{\mathbb{N}}$ punti materiali
$(\PointMass_k)_{k \in N}$ di massa $(\Mass_k)_{k \in N}$
e coordinate $(X_k)_{k \in N}$ (nel piano e nello spazio indifferentemente)
tali che
\begin{itemize}
  \item i punti $\PointMass_0$ e $\PointMass_{N - 1}$ siano vincolati
    nella loro posizione al tempo iniziale $t = 0$;
  \item per ogni $k \in N - 1$, $\PointMass_k$ e $\PointMass_{k + 1}$
    sono collegati da una corda elastica di costante elastica
    $\ElasticConstant_k$;
  \item per ogni $k \in \NotZero{(N - 1)}$, $\PointMass_k$ \`e soggetto
    ad una forza d'attrito dinamico con coefficiente $\Friction_k$.
\end{itemize}
\par Sia $k \in \NotZero{(N - 1)}$. Per la seconda legge di Newton, abbiamo
\[
  \Mass_k X_k = - \ElasticConstant X_k - \Friction_k \dot{X_k}.
\]
Inoltre, abbiamo le condizioni al bordo
\[
  \begin{cases}
    \ForAll{t \in \RealNonNegative}{X_0(t) = X_0(0)},\\
    \ForAll{t \in \RealNonNegative}{X_{N - 1}(t) = X_{N - 1}(0)},\\
    \dot{X_0} = \dot{X_{N - 1}} = 0.
  \end{cases}
\]
\par Poniamo ora
\begin{itemize}
  \item $M \in \mathbb{R}^{N \times N}$ uguale alla matrice diagonale i cui
    elementi sono la successione delle masse:
    \[
      M =
      \lmatrix
      \begin{array}{ccc}
        \Mass_0 & &\\
        & \ddots &\\
        & & \Mass_{N - 1}
      \end{array}
      \rmatrix.
    \]
  \item $R \in \mathbb{R}^{N \times N}$ uguale alla matrice diagonale i cui
    elementi sono la successione dei coefficienti di attrito:
    \[
      R =
      \lmatrix
      \begin{array}{ccc}
        \Friction_0 & &\\
        & \ddots &\\
        & & \Friction_{N - 1}
      \end{array}
      \rmatrix.
    \]
  \item $K \in \mathbb{R}^{N \times N}$ uguale alla matrice tridiagonale i
    cui elementi sono
    $K_k^j =
    \begin{cases}
      - \ElasticConstant_k&\text{ se }k = j + 1,\\
      \ElasticConstant_k + \ElasticConstant_{k + 1}&\text{ se }k = j,\\
      - \ElasticConstant_{k + 1}&\text{ se }k = j - 1,
    \end{cases}$
    vale a dire
    \[
      K =
      \lmatrix
      \begin{array}{cccccc}
        \ElasticConstant_0 + \ElasticConstant_1 & - \ElasticConstant_1 &&&&\\
        - \ElasticConstant_0 & \ddots & \ddots &&&\\
        & \ddots & \ddots & \ddots & &\\
        && \ddots & \ddots & - \ElasticConstant\\
        &&& - \ElasticConstant & \ElasticConstant + \ElasticConstant
      \end{array}
      \rmatrix.
    \]
\end{itemize}

\subsubsection{Vibrazioni di membrane elastiche.}
\label{CalcoloScientifico_VibrazioniDiMembraneElastiche}



	\chapter{Istituzioni di fisica matematica}
	\section{Teorema di Perron-Frobenius.}
\label{CalcoloScientifico_TeoremaDiPerronFrobenius}
\begin{Theorem}
  \TheoremName{Teorema di Perron-Frobenius}[di Perron-Frobenius][teorema]
  Siano
  \begin{itemize}
    \item $n \in \NotZero{\mathbb{N}}$;
    \item $\Matrix \in \mathbb{C}^{n \times n}$ non negativa.
  \end{itemize}
  Allora
  \begin{itemize}
    \item $\SpectralRadius{\Matrix} \in \Spectrum{\Matrix}$;
    \item esiste un autovettore sinistro $\VarEigenvector$ tale che
      $\ForAll{k \in n}{\VarEigenvector \geq 0}$;
    \item esiste un autovettore destro $\Eigenvector$ tale che
      $\ForAll{k \in n}{\Eigenvector \geq 0}$.
  \end{itemize}
  Inoltre,
  \begin{itemize}
    \item se $\Matrix$ \`e irriducibile, allora
      \begin{itemize}
        \item $\SpectralRadius{\Matrix}$ \`e un autovalore semplice;
        \item esiste un autovettore sinistro $\VarEigenvector$ tale che
          $\ForAll{k \in n}{\VarEigenvector > 0}$;
        \item esiste un autovettore destro $\Eigenvector$ tale che
          $\ForAll{k \in n}{\Eigenvector > 0}$;
      \end{itemize}
      \item se $\Matrix$ \`e positiva, allora
        $\SpectralRadius{\Matrix}$
        \`e l'unico autovalore di modulo massimo di $\Matrix$.
  \end{itemize}
\end{Theorem}

\section{\English{Page rank}.}
\label{CalcoloScientifico_PageRank}
Il modello denominato
\Define{\English{page rank}}
\`e un modello nato per classificare l'importanza delle pagine
\English{web}, da cui il nome, ma che pu\`o essere riproposto per ogni
rete. Noi lo presenteremo nel suo contesto originale delle reti.
\par Sia $\PagesNumber$ il numero totale di tutte le pagine di una rete data.
Identificheremo ogni pagina con un numero naturale da $0$ a
$\PagesNumber - 1$, dove $\PagesNumber$ \`e il numero delle pagine stesse.
\par Esso viene costruito a partire da un grafo orientato
$\Graph = (\PagesNumber,\Edges)$ dove
$(\Node,\VarNode) \in \PagesNumber \times \PagesNumber$ appartiene a $\Edges$
se e solo se la pagina $\Node$ contiene un collegamento ipertestuale
alla pagina $\VarNode$.
\par Denotiamo $\AdjacencyMatrix$ la matrice di adiacenza di $\Graph$.
\par Lo scopo del modello di \English{page rank} \`e determinare un
\Define{vettore di pesi}[dei pesi (\English{page rank})][vettore]
$\Weights$: la componente $\Weights_k$ indica l'importanza della pagina $k$.
\par Definiamo i vettori
\begin{itemize}
  \item $\AllOnes = \Transposed{(1 \cdots 1)}$;
  \item $\PageDegree = \AdjacencyMatrix \AllOnes$;
  \item $\MatrixPageDegree \in \mathbb{R}^{\PagesNumber \times \PagesNumber}$
    tale che
    \[
      \ForAll{(j,k) \in \PagesNumber \times \PagesNumber}{
        \MatrixPageDegree_j^k =
        \begin{cases}
          0&\text{ se } j \neq k,\\
          \PageDegree_k&\text{ se } j = k;
        \end{cases}}
    \]
  \item $\Matrix = \MatrixPageDegree^{-1}\AdjacencyMatrix$.
\end{itemize}
La componente $\PageDegree_k$ ($k \in \PagesNumber$) indica il grado uscente
della pagina $k$.
\par Vorremmo costruire un modello per cui ogni pagina trasmette equamente
la sua importanza a tutte le pagine da esse puntata e l'importanza di una pagina
\`e esattamente la somma delle importanze cos\`i trasmesse tramite i
collegamenti.
\par Come vedremo, \`e utile ipotizzare che non esistano pagine pendenti
nella rete in esame; questa ipotesi \`e facilmente realizzabile con una delle due strategie seguenti, nessuna delle quali cambia sostanzialmente
il senso del modello della rete:
\begin{itemize}
  \item aggiungiamo ad ogni pagina pendente un collegamento a tutte
    le altre pagine;
  \item intoduciamo una pagina aggiuntiva $\PagesNumber$ che punta a tutte 
  le altre pagine ed \`e puntata da tutte le altre pagine:
  \[
    \ForAll{k > 0}{\AdjacencyMatrix_\PagesNumber^k
    = \AdjacencyMatrix_k^\PagesNumber = 1}.
  \]
\end{itemize}
\par Assumeremo dunque d'ora in poi il seguente assioma.
\begin{Axiom}
  La rete non contiene nodi pendenti.
\end{Axiom}
\begin{Theorem}
  $\Weights$ \`e autovettore sinistro di $\Matrix$.
\end{Theorem}
\Proof Per ogni pagina $k \in \PagesNumber$, abbiamo
\[
  \Weights_k = \sum_{j \in \PagesNumber} \AdjacencyMatrix_j^k \frac{\Weights_j}{\PageDegree_j},
\]
equivalente a
\[
  \Weights = \MatrixPageDegree^{-1}\AdjacencyMatrix\Weights,
\]
vale a dire
\[
  \Transposed{\Weights} = \Transposed{\Weights}\Matrix.\text{ \EndProof}
\]
\begin{Theorem}
  Abbiamo
  \begin{itemize}
    \item $1$ \`e autovalore di $\Matrix$ relvativo a $\AllOnes$;
    \item $\SpectralRadius{\Matrix} = 1$.
  \end{itemize}
\end{Theorem}
\Proof Abbiamo
$\Matrix\AllOnes
= \MatrixPageDegree^{-1}\AdjacencyMatrix\AllOnes
= \MatrixPageDegree^{-1}\PageDegree
= \AllOnes$.
\par Inoltre,
$\Norm{\Matrix}[\infty]
= \max_{k \in \PagesNumber} \sum_{j \in \PagesNumber}
  \AbsoluteValue{\Matrix_k^j}
= \max_{k \in \PagesNumber} \sum_{j \in \PagesNumber}
  \Matrix_k^j
= \max_{k \in \PagesNumber} \sum_{j \in \PagesNumber}
  \frac{\AdjacencyMatrix_k^j}{\PageDegree_k}
= 1$,
e quindi, se $\Eigenvector \in \mathbb{C}^\PagesNumber$ \`e autovettore
di $\Matrix$ relativo a $\Eigenvalue$, allora
$\AbsoluteValue{\Eigenvalue}\Norm{\Eigenvector}[\infty]
= \Norm{\Eigenvalue\Eigenvector}[\infty]
= \Norm{\Matrix\Eigenvector}[\infty]
\leq \Norm{\Matrix}[\infty]\Norm{\Eigenvector}[\infty]$,
da cui
$\AbsoluteValue{\Eigenvalue}
\leq \Norm{\Matrix}[\infty] = 1$. \EndProof
\begin{Theorem}
  Se $\Matrix$ ha un unico autovalore di modulo massimo, qualsiasi successione
  $(\Weights^{(k)})_{k \in \mathbb{N}}$ tale che
  \begin{itemize}
    \item $\Weights^{(0)}$ non \`e ortogonale a $\AllOnes$;
    \item per ogni $k > 0$,
      $\Weights^{(k)} = \Transposed{\Matrix}\Eigenvector^{(k - 1)}$;
    \item se $\Weights^{(0)}$ non ha componenti negative, allora
      $\ForAll{k \in \mathbb{N}}{
      \Norm{\Weights^{(k)}}[1] = \Norm{\Weights^{(0)}}[1]}$;
  \end{itemize}
  converge a un autovettore sinistro di $\Matrix$, cio\`e a $\Weights$
  a meno di prodotto per scalare.
\end{Theorem}
\Proof Riscriviamo il metodo delle potenze applicandolo alla matrice
$\Transposed{\Matrix}$ e al vettore $\Weights^{(0)}$ con una qualsiasi
tolleranza $\Tollerance$:
\begin{enumerate}
  \item\label{PageRank_MetodoDellePotenze_1} si ponga $j = 0$,
    $\Weights_0 = \Weights^{(0)}$,
    $\Eigenvalue_0^{(0)} =
    = \Transposed{\Weights_0}\Matrix\Weights_{0}$;
  \item\label{PageRank_MetodoDellePotenze_2} si ponga
    $\tilde{\Weights}_{j + 1} = \Matrix \Weights_j$;
  \item\label{PageRank_MetodoDellePotenze_3} si ponga
    $\Weights_{j + 1}
    = \frac{\tilde{\Weights}_{j + 1}}{\Norm{\tilde{\Weights_{j + 1}}}}$;
  \item\label{PageRank_MetodoDellePotenze_4} si ponga $\Eigenvalue_0^{(j + 1)}
    = \Transposed{\Weights_{j + 1}}\Matrix\Weights_{j + 1}$;
  \item\label{PageRank_MetodoDellePotenze_5} se
    $\AbsoluteValue{\Eigenvalue_0^{(j + 1)} - \Eigenvalue_0^{(j)}}
    > \Tollerance$
    si incrementi $j$ di $1$ e si torni al punto
    \ref{PageRank_MetodoDellePotenze_2}, altrimenti l'algoritmo termina
    restituendo la coppia $(\Eigenvalue_0^{(j + 1)},\Weights_{j + 1})$.
\end{enumerate}
\par Per ogni $k \in \mathbb{N}$, abbiamo
$\Weights_k = \Weights^{(k)}$ a meno di moltiplicazione per scalare.
\par Supponiamo $\Weights^{(0)}$ non abbia nessuna componente negativa.
Si dimostra immediatamente per induzione, utilizzando anche la non negativit\`a
di $\Matrix$, che per ogni $k \in \mathbb{N}$,
$\Norm{\Weights^{(k)}}[1]
= \Transposed{\Weights^{(k)}}\AllOnes
= \Transposed{\Weights^{(0)}}\AllOnes
\Norm{\Weights^{(0)}}[1]$. \EndProof
\par Con riferimento a quest'ultimo teorema, grazie alla costanza della
norma dei vettori della successione
$(\Weights^{(k)})_{k \in \mathbb{N}}$
\`e possibile calcolare direttamente l'autovettore sinistro $\Weights$
senza il passo di normalizzazione \ref{PageRank_MetodoDellePotenze_3}
evitando qualsiasi rischio di traboccamento dei moduli.
\par Definiamo adesso
\begin{itemize}
  \item $\gamma \in (0,1)$;
  \item $\Vector \in \mathbb{R}^\PagesNumber$ tale che
    $\ForAll{k \in \PagesNumber}{\Vector_k > 0}$ e
    $\Norm{\Vector}[1] = 1$;
  \item $A = \gamma\Matrix + (1 - \gamma)\AllOnes\Transposed{\Vector}$.
\end{itemize}
\begin{Theorem}
  La matrice $A$ \`e positiva.
\end{Theorem}
\Proof Fissiamo $(k,j) \in \PagesNumber \times \PagesNumber$:
$A_k^j$ \`e combinazione convessa di $\Matrix_k^j$ e
$\sum_{k \in \PagesNumber} \Vector_k$ con coefficienti non nulli,
dunque \`e strettamente positivo. \EndProof
\begin{Theorem}
  $\AllOnes$ \`e autovettore di $A$ relativo all'autovalore $1$.
\end{Theorem}
\Proof Abbiamo
$A\AllOnes
= \gamma\Matrix\AllOnes + (1 - \gamma)(\AllOnes\Transposed{\Vector})\AllOnes
= \gamma\AllOnes
  + (1 - \gamma)\AllOnes(\Transposed{\Vector}\AllOnes)
= \gamma\AllOnes + (1 - \gamma)\AllOnes
= \AllOnes$. \EndProof
\begin{Theorem}
  \TheoremName{Teorema di Brauer}[di Brauer][teorema]
  Siano
  \begin{itemize}
    \item $n \in \NotZero{\mathbb{N}}$;
    \item $J \in \mathbb{C}^{n \times n}$;
    \item $\Eigenvector$ autovettore di $J$ relativo all'autovalore
      $\Eigenvalue \in \Spectrum{J}$;
    \item $\VarVector \in \mathbb{C}^n$;
    \item $K = J + \Eigenvector\HermitianTransposed{\VarVector}$.
  \end{itemize}
  Abbiamo
  \[
    \Spectrum{K}
    = \Spectrum{J}
      \SetMin \lbrace \Eigenvalue \rbrace
      \cup \lbrace \Eigenvalue + \HermitianTransposed{\VarVector}\Eigenvector \rbrace.
  \]
\end{Theorem}
\Proof Sia $\UnitaryMatrix \in \mathbb{C}^n$ unitaria tale che
$\SchurNormalForm = \HermitianTransposed{\UnitaryMatrix} J \UnitaryMatrix$
sia una forma normale di Schur di $J$ tale che
$\SchurNormalForm_0^0 = \Eigenvalue$.
\par Denotando $\CanonicalBase^0$ il primo vettore della base canonica, abbiamo
\begin{align}
  \HermitianTransposed{\UnitaryMatrix} K \UnitaryMatrix
  &= \HermitianTransposed{\UnitaryMatrix} J \UnitaryMatrix
    + \HermitianTransposed{\UnitaryMatrix}
      \Eigenvector\HermitianTransposed{\VarVector}
      \UnitaryMatrix,\\
  &= \SchurNormalForm
    + \CanonicalBase^0\HermitianTransposed{\VarVector}
      \UnitaryMatrix.
\end{align}
\par Sia ora $k \in n$. Supponiamo $k = 0$: abbiamo
\[
  (\CanonicalBase^0\HermitianTransposed{\VarVector}\UnitaryMatrix)_0
  = \HermitianTransposed{\VarVector}\UnitaryMatrix.
\]
Se invece $k \neq 0$, allora, per ogni $j \in n$
\[
  (\CanonicalBase^0\HermitianTransposed{\VarVector}\UnitaryMatrix)_k
  = 0.
\]
Dunque tutti gli elementi diagonali della matrice triangolare superiore
$\HermitianTransposed{\UnitaryMatrix} K \UnitaryMatrix$,
forma normale di Schur,
coincidono con quelli di $\SchurNormalForm$ tranne
$(\HermitianTransposed{\UnitaryMatrix} J \UnitaryMatrix)_0^0$
che viene sostituito da
\[
  \Eigenvalue + \HermitianTransposed{\VarVector}\UnitaryMatrix\CanonicalBase^0
  = \Eigenvalue + \HermitianTransposed{\VarVector}\Eigenvector.
  \text{ \EndProof}
\]
\begin{Theorem}
  $1$ \`e autovalore semplice e di modulo massimo di $A$.
\end{Theorem}
\Proof Ricordiamo che
$A = \gamma\Matrix + (1 - \gamma)\AllOnes\Transposed{\Vector}$.
$\AllOnes$ \`e autovettore relativo a $1$ di $\Matrix$ e $1$ \`e autovalore
semplice e di modulo massimo di $\Matrix$ per il teorema di Perron-Frobenius.
Ne deduciamo che
$\AllOnes$ \`e autovettore relativo a $\gamma$ di $\gamma\Matrix$ e
$\gamma$ \`e autovalore semplice e di modulo massimo di $\gamma\Matrix$
per il teorema di Perron-Frobenius.
\par Abbiamo
$\gamma + (1 - \gamma)\Transposed{\AllOnes}\Vector
= \gamma + (1 - \gamma)\Norm{\Vector}[1] = 1$.
Dunque, per il teorema di Bauer, $1$ \`e autovalore semplice e di modulo
massimo di $A$. \EndProof

\section{Vibrazioni.}
\label{CalcoloScientifico_Vibrazioni.}
\subsubsection{Vibrazioni di corde elastiche.}
\label{CalcoloScientifico_VibrazioniDiCordeElastiche}
\par Consideriamo una successione finita di
$N \in \NotZero{\mathbb{N}}$ punti materiali
$(\PointMass_k)_{k \in N}$ di massa $(\Mass_k)_{k \in N}$
e coordinate $(X_k)_{k \in N}$ (nel piano e nello spazio indifferentemente)
tali che
\begin{itemize}
  \item i punti $\PointMass_0$ e $\PointMass_{N - 1}$ siano vincolati
    nella loro posizione al tempo iniziale $t = 0$;
  \item per ogni $k \in N - 1$, $\PointMass_k$ e $\PointMass_{k + 1}$
    sono collegati da una corda elastica di costante elastica
    $\ElasticConstant_k$;
  \item per ogni $k \in \NotZero{(N - 1)}$, $\PointMass_k$ \`e soggetto
    ad una forza d'attrito dinamico con coefficiente $\Friction_k$.
\end{itemize}
\par Sia $k \in \NotZero{(N - 1)}$. Per la seconda legge di Newton, abbiamo
\[
  \Mass_k X_k = - \ElasticConstant X_k - \Friction_k \dot{X_k}.
\]
Inoltre, abbiamo le condizioni al bordo
\[
  \begin{cases}
    \ForAll{t \in \RealNonNegative}{X_0(t) = X_0(0)},\\
    \ForAll{t \in \RealNonNegative}{X_{N - 1}(t) = X_{N - 1}(0)},\\
    \dot{X_0} = \dot{X_{N - 1}} = 0.
  \end{cases}
\]
\par Poniamo ora
\begin{itemize}
  \item $M \in \mathbb{R}^{N \times N}$ uguale alla matrice diagonale i cui
    elementi sono la successione delle masse:
    \[
      M =
      \lmatrix
      \begin{array}{ccc}
        \Mass_0 & &\\
        & \ddots &\\
        & & \Mass_{N - 1}
      \end{array}
      \rmatrix.
    \]
  \item $R \in \mathbb{R}^{N \times N}$ uguale alla matrice diagonale i cui
    elementi sono la successione dei coefficienti di attrito:
    \[
      R =
      \lmatrix
      \begin{array}{ccc}
        \Friction_0 & &\\
        & \ddots &\\
        & & \Friction_{N - 1}
      \end{array}
      \rmatrix.
    \]
  \item $K \in \mathbb{R}^{N \times N}$ uguale alla matrice tridiagonale i
    cui elementi sono
    $K_k^j =
    \begin{cases}
      - \ElasticConstant_k&\text{ se }k = j + 1,\\
      \ElasticConstant_k + \ElasticConstant_{k + 1}&\text{ se }k = j,\\
      - \ElasticConstant_{k + 1}&\text{ se }k = j - 1,
    \end{cases}$
    vale a dire
    \[
      K =
      \lmatrix
      \begin{array}{cccccc}
        \ElasticConstant_0 + \ElasticConstant_1 & - \ElasticConstant_1 &&&&\\
        - \ElasticConstant_0 & \ddots & \ddots &&&\\
        & \ddots & \ddots & \ddots & &\\
        && \ddots & \ddots & - \ElasticConstant\\
        &&& - \ElasticConstant & \ElasticConstant + \ElasticConstant
      \end{array}
      \rmatrix.
    \]
\end{itemize}

\subsubsection{Vibrazioni di membrane elastiche.}
\label{CalcoloScientifico_VibrazioniDiMembraneElastiche}



	\chapter{Istituzioni di analisi numerica}
	\section{Polinomi ortogonali.}
\label{IstituzioniDiAnalisiNumerica_PolinomiOrtogonali}
\subsection{Definizioni e teoremi di base.}
\label{IstituzioniDiAnalisiNumerica_PolinomiOrtogonali}
\begin{Definition}
	Chiamiamo \Define{sequenza polinomiale}[polinomiale][sequenza] o \Define{successione polinomiale graduale}[polinomiale graduale][successione] una successione di polinomi $(\Polynomial_i)_{i \in N} \in (\RingAdjunction{\mathbb{C}}{x})^N)$ ($N \in \mathbb{N} \cup \lbrace \omega_0 \rbrace$) tale che $\ForAll{i \in I}{\PolynomialDegree{\Polynomial_i} = i}$.
\end{Definition}
\par Nel seguito, quando parleremo di \Define{polinomi ortogonali}[ortogonale][polinomio] o \Define{polinomi ortonormali}[ortonormale][polinomio] assumeremo sempre implicitamente che essi costituiscono una sequenza polinomiale, salvo avvisi contrari.
\begin{Theorem}
	Siano $(\OrthogonalPolynomial_i)_{i \in \mathbb{N}}$ polinomi ortogonali rispetto a un prodotto hermitiano $\HermitianProduct{\cdot}{\cdot}$ fissato. Per ogni $n \in \mathbb{N}$, $(\OrthogonalPolynomial_i)_{i \in n}$ costituisce una base ortogonale del sottospazio di $\RingAdjunction{\mathbb{C}}{x}$ di tutti i polinomi di grado minore di $n$. Inoltre $(\OrthogonalPolynomial_i)_{i \in \mathbb{N}}$ costituisce una base di $\RingAdjunction{\mathbb{C}}{x}$.
\end{Theorem}
\Proof I polinomi $(\OrthogonalPolynomial_i)_{i \in \mathbb{N}}$ sono ortogonali quindi linearmente indipendenti: per ogni $n \in \mathbb{N}$, $(\OrthogonalPolynomial_i)_{i \in n}$ generano dunque uno spazio di dimensione $n$, che \`e appunto la dimensione del sottospazio di $\RingAdjunction{\mathbb{C}}{x}$ di tutti i polinomi di grado minore di $n$, a cui essi evidentemente appartengono.
\par Ogni polinomio $\Polynomial \in \RingAdjunction{\mathbb{C}}{x}$ appartiene al sottospazio di $\RingAdjunction{\mathbb{C}}{x}$ di tutti i polinomi di grado al pi\`u $\PolynomialDegree{\Polynomial}$ ed \`e dunque combinazione lineare dei polinomi $(\OrthogonalPolynomial_i)_{i \in n}$ e a maggior ragione dei polinomi $(\OrthogonalPolynomial_i)_{i \in \mathbb{N}}$. Dunque $(\OrthogonalPolynomial_i)_{i \in \mathbb{N}}$ \`e una base di $\RingAdjunction{\mathbb{C}}{x}$. \EndProof
\begin{Corollary}
	Fissato un prodotto hermitiano $\HermitianProduct{\cdot}{\cdot}$ su $\RingAdjunction{\mathbb{C}}{x}$, esistono e sono unici, a meno di costanti moltiplicative, i polinomi $(\OrthogonalPolynomial_i)_{i \in \mathbb{N}}$ ortogonali rispetto ad esso; pi\`u precisamente, siano $(\OrthogonalPolynomial_i)_{i \in \mathbb{N}}$ e $(\VarOrthogonalPolynomial_i)_{i \in \mathbb{N}}$ due famiglie di polinomi ortogonali: abbiamo $\ForAll{i \in \mathbb{N}}{\Exists{C \in \mathbb{C}}{\VarOrthogonalPolynomial_i = C \OrthogonalPolynomial_i}}$.
\end{Corollary}
\Proof L'esistenza segue direttamente dal teorema precedente e dal teorema di ortogonalizzazione di Gram-Schmidt.
\par Dimostriamo l'unicit\`a e meno di costanti moltiplicative. Per il teorema precedente, per ogni $i \in \mathbb{N}$ esistono unici scalari $(\Scalar_k)_{k \in i + 1} \in \mathbb{C}^{i + 1}$ tali che $\VarOrthogonalPolynomial_i = \sum_{k \in i + 1} \Scalar_k \OrthogonalPolynomial_k$. Abbiamo
$\ScalarProduct{\VarOrthogonalPolynomial_i}{\OrthogonalPolynomial_i} =
\ScalarProduct{\sum_{k \in i + 1} \Scalar_k \OrthogonalPolynomial_k}{\OrthogonalPolynomial_i} =
\sum _{k \in i + 1} \Scalar_k \ScalarProduct{\OrthogonalPolynomial_k}{\OrthogonalPolynomial_i} =
\Scalar_i \ScalarProduct{\OrthogonalPolynomial_i}{\OrthogonalPolynomial_i}$. Ne deduciamo $\ScalarProduct{\VarOrthogonalPolynomial_i - \Scalar_i\OrthogonalPolynomial_i}{\OrthogonalPolynomial_i} = 0$ e dunque $\VarOrthogonalPolynomial_i = \Scalar_i \OrthogonalPolynomial_i$. \EndProof
\par Per il corollario precedente, nel seguito, salvo avvisi contrari, considereremo sempre successioni di polinomi graduali indicizzate in tutto $\mathbb{N}$.
\begin{Theorem}
	Siano $(\OrthogonalPolynomial_i)_{i \in \mathbb{N}} \in (\RingAdjunction{\mathbb{C}}{x})^\mathbb{N}$ polinomi ortogonali rispetto a un prodotto hermitiano $\HermitianProduct{\cdot}{\cdot}$ fissato. Fissiamo $n \in \mathbb{N}$ e sia $\Polynomial \in \RingAdjunction{\mathbb{C}}{x}$ di grado $m = \PolynomialDegree{\Polynomial} < n$. Abbiamo $\HermitianProduct{\Polynomial}{\OrthogonalPolynomial_n} = 0$.
\end{Theorem}
\Proof Esistono scalari $(\Scalar_i)_{i \in n} \in \mathbb{C}^n$ tali che $\Polynomial = \sum_{i \in n} \Scalar_i \OrthogonalPolynomial_i$. Abbiamo $\HermitianProduct{\Polynomial}{\OrthogonalPolynomial_n} = \sum_{i \in n} \Scalar_i\HermitianProduct{\OrthogonalPolynomial_i}{\OrthogonalPolynomial_n} = 0$. \EndProof
\begin{Theorem}
	\TheoremName{Propriet\`a di minima norma} Siano $(\OrthogonalPolynomial_i)_{i \in \mathbb{N}} \in (\RingAdjunction{\mathbb{C}}{x})^\mathbb{N}$ polinomi ortogonali rispetto a un prodotto hermitiano $\HermitianProduct{\cdot}{\cdot}$ fissato. Sia $\Polynomial \in \RingAdjunction{\mathbb{C}}{x}$ di grado $n = \PolynomialDegree{\Polynomial}$ tale che $\LeadingCoefficient{\Polynomial} = \LeadingCoefficient{\OrthogonalPolynomial_n}$. Allora
	\begin{itemize}
		\item $\Norm{\OrthogonalPolynomial_n} \leq \Norm{\Polynomial}$;
		\item $\Implies{\Norm{\OrthogonalPolynomial_n} = \Norm{\Polynomial}}{\OrthogonalPolynomial_n = \Polynomial}$;
	\end{itemize}
\end{Theorem}
\Proof Poniamo $\VarPolynomial = \Polynomial - \OrthogonalPolynomial_n$. Abbiamo $\PolynomialDegree{\VarPolynomial} < n$ in virt\`u di $\LeadingCoefficient{\Polynomial} = \LeadingCoefficient{\OrthogonalPolynomial_n}$.
\par Abbiamo ${\Norm{\Polynomial}}^2 = \HermitianProduct{\OrthogonalPolynomial_n + \VarPolynomial}{\OrthogonalPolynomial_n + \VarPolynomial} = \HermitianProduct{\OrthogonalPolynomial_n}{\OrthogonalPolynomial_n} + 2 \HermitianProduct{\OrthogonalPolynomial_n}{\VarPolynomial} + \HermitianProduct{\VarPolynomial}{\VarPolynomial} = {\Norm{\OrthogonalPolynomial_n}}^2 + {\Norm{\VarPolynomial}}^2$, da cui segue immediatamente la tesi. \EndProof
\begin{Theorem}
	\label{IstituzioniDiAnalisiNumerica_ProdottoHermitianoIntegrale}
	Sia data una \Define{funzione peso}[peso][funzione] $\WeightFunction: [a,b] \rightarrow \RealExtended$ tale che
	\begin{itemize}
		\item se $a < x < b$, allora  $0 < \WeightFunction(x) < + \infty$;
		\item per ogni polinomio $\Polynomial \in \RingAdjunction{\mathbb{R}}{x}$ esiste finito $\int_a^b \WeightFunction(x) \Polynomial(x) dx$.
	\end{itemize}
	Sia $\HermitianProduct{\cdot}{\cdot}: (\RingAdjunction{\mathbb{C}}{x})^2 \rightarrow \mathbb{C}$ l'applicazione che a $(\Polynomial,\VarPolynomial) \in (\RingAdjunction{\mathbb{C}}{x})^2$ associa $\int_a^b \WeightFunction(x) f(x) \Conjugate{g(x)} dx$. $\HermitianProduct{\cdot}{\cdot}$ \`e un prodotto scalare.
\end{Theorem}
\Proof Segue direttamente dalla definizione di prodotto hermitiano e dalle propriet\`a dell'integrale. \EndProof
\begin{Theorem}
	$X: \RingAdjunction{\mathbb{R}}{x} \rightarrow \RingAdjunction{\mathbb{R}}{x}$ che a $\Polynomial \in \RingAdjunction{\mathbb{R}}{x}$ associa $x\Polynomial \in \RingAdjunction{\mathbb{R}}{x}$ \`e un operatore lineare autoaggiunto.
\end{Theorem}
\Proof Segue direttamente dalle definizioni. \EndProof
\begin{Theorem}
	\TheoremName{Propriet\`a degli zeri dei polinomi ortogonali} Siano $(\OrthogonalPolynomial_i)_{i \in \mathbb{N}}$ polinomi ortogonali rispetto ad un prodotto hermitiano della forma specificata dal teorema \ref{PolinomiOrtogonali_ProdottoHermitianoIntegrale}. Per ogni $i \in \mathbb{N}$, $\OrthogonalPolynomial_i$ ha tutti gli zeri semplici e appartenenti a $(a,b)$.
\end{Theorem}
\Proof Fissiamo $n \in \mathbb{N}$ e siano $(x_j)_{j \in m}$ ($m \in \mathbb{N}$) la famiglia degli zeri reali e distinti di $\OrthogonalPolynomial_n$ nell'intervallo $(a,b)$. Per il teorema fondamentale dell'algebra, certamente $m \leq n$.
\par Supponiamo $m < n$ e, se necessario, riordiniamo gli indici di $(x_j)_{j \in m}$ in modo tale che esista $k \in m$ tale che $x_j$ ha moltiplicit\`a dispari se $j \in k$ e molteplicit\`a pari altrimenti. Poniamo $\VarPolynomial(x) = 1$ se $k = 0$ e $\VarPolynomial(x) = \prod_{j \in k} x_j$ altrimenti.
\par Il polinomio $\OrthogonalPolynomial_n \VarPolynomial$ ha allora tutte le radici reali di molteplicit\`a pari e non cambia segno in $(a,b)$. Dunque $\HermitianProduct{\OrthogonalPolynomial_n}{\VarPolynomial} = \int_a^b \WeightFunction(x)\OrthogonalPolynomial_n(x)\Conjugate{\VarPolynomial(x)}dx = \int_a^n \WeightFunction(x)\OrthogonalPolynomial_n(x)\VarPolynomial(x) \neq 0$. Ma abbiamo anche $\PolynomialDegree{\VarPolynomial} < \PolynomialDegree{\OrthogonalPolynomial_n}$, da cui $\HermitianProduct{\OrthogonalPolynomial_n}{\VarPolynomial} = 0$, e questo \`e assurdo.
\par Deve dunque essere $m = n$. \EndProof

\subsection{Ricorrenza a tre termini.}
\label{IstituzioniDiAnalisiNumerica_PolinomiOrtogonali}
\begin{Theorem}
\label{IstituzioniDiAnalisiNumerica_RicorrenzaATreTermini}
	\TheoremName{Ricorrenza a tre termini}[a tre termini][ricorrenza] Siano $(\OrthogonalPolynomial_i)_{i \in \mathbb{N}}$ polinomi ortogonali rispetto a un qualsiasi prodotto scalare $\HermitianProduct{\cdot}{\cdot}$ per cui l'operatore lineare di moltiplicazione per il polinomio $x$ sia autoaggiunto. Siano inoltre
	\begin{itemize}
		\item $a_0, a_1, b_1 \in \mathbb{C}$;
		\item $\OrthogonalPolynomial_0 = a_0$;
		\item $\OrthogonalPolynomial_1(x) = a_1x + b_1$.
	\end{itemize}
	Per ogni $i \in N \SetMin 1$, abbiamo
	\[
	\OrthogonalPolynomial_{i + 1}(x) = (A_{i + 1}x + B_{i + 1}) p_i(x) - C_i p_{i - 1}
	\]
	per opportuni $A_i, B_i, C_i \in \mathbb{C}$. In particolare, abbiamo
	\begin{itemize}
		\item $A_{i + 1} = \frac{\HermitianProduct{\OrthogonalPolynomial_{i + 1}}{\OrthogonalPolynomial_{i + 1}}}{\HermitianProduct{x\OrthogonalPolynomial_i}{\OrthogonalPolynomial_{i + 1}}}$;
		\item $B_{i + 1} = - A_{i + 1} \frac{\HermitianProduct{x\OrthogonalPolynomial_i}{\OrthogonalPolynomial_i}}{\HermitianProduct{\OrthogonalPolynomial_i}{\OrthogonalPolynomial_i}}$;
		\item $C_i = A_{i + 1} \frac{\HermitianProduct{x\OrthogonalPolynomial_i}{\OrthogonalPolynomial_{i - 1}}}{\HermitianProduct{\OrthogonalPolynomial_{i - 1}}{\OrthogonalPolynomial_{i - 1}}} = \frac{A_{i + 1}}{A_i} \frac{\HermitianProduct{\OrthogonalPolynomial_i}{\OrthogonalPolynomial_i}}{\HermitianProduct{\OrthogonalPolynomial_{i - 1}}{\OrthogonalPolynomial_{i - 1}}}$.
	\end{itemize}
	Inoltre, denotati $a_i$ e $b_i$ i coefficienti di $x^i$ e $x^{i - 1}$ rispettivamente in $\OrthogonalPolynomial_i$ e posto $h_i = \ScalarProduct{\OrthogonalPolynomial_i}{\OrthogonalPolynomial_i}$, abbiamo
	\begin{itemize}
		\item $A_{i + 1} = \frac{a_{i + 1}}{a_i}$;
		\item $B_{i + 1} = \frac{a_{i + 1}}{a_i} \left ( \frac{b_{i + 1}}{a_{i + 1}} - \frac{b_i}{a_i} \right )$;
		\item $C_i = \frac{a_{i + 1}a_{i - 1}}{a_i^2} \frac{h_i}{h_{i - 1}}$.
	\end{itemize}
\end{Theorem}
\Proof Fissiamo $i \geq 1$. Il polinomio $x \OrthogonalPolynomial_i$ ha grado $i + 1$ ed \`e dunque linearmente indipendente dai polinomi $(\OrthogonalPolynomial_j)_{j \in i + 1}$: insieme formano una base dello spazio di tutti i polinomi di grado al pi\`u $i + 1$. Esistono quindi scalari $(\Scalar_j)_{j \in i + 2} \in \mathbb{C}^{i + 2}$ tali che $\OrthogonalPolynomial_{i + 1} = \sum_{j \in i + 1} \Scalar_j \OrthogonalPolynomial_j + \Scalar_{i + 1} x \OrthogonalPolynomial_i$.
\par Per ogni $k \in i + 1$ abbiamo $\HermitianProduct{\OrthogonalPolynomial_{i + 1}}{\OrthogonalPolynomial_k} =
\sum_{j \in i + 1} \Scalar_j \HermitianProduct{\OrthogonalPolynomial_j}{\OrthogonalPolynomial_k} + \Scalar_{i + 1} \HermitianProduct{x \OrthogonalPolynomial_i}{\OrthogonalPolynomial_k} =
\sum_{j \in i + 1} \Scalar_j \HermitianProduct{\OrthogonalPolynomial_j}{\OrthogonalPolynomial_k} + \Scalar_{i + 1} \HermitianProduct{\OrthogonalPolynomial_i}{x \OrthogonalPolynomial_k} = \Scalar_k \HermitianProduct{\OrthogonalPolynomial_k}{\OrthogonalPolynomial_k} + \Scalar_{i + 1} \HermitianProduct{\OrthogonalPolynomial_i}{x\OrthogonalPolynomial_k}$
e, naturalmente, $\HermitianProduct{\OrthogonalPolynomial_{i + 1}}{\OrthogonalPolynomial_k} = 0$.
Ora, se $k \leq i - 2$, allora $\PolynomialDegree{x \OrthogonalPolynomial_k} < i$, dunque $\HermitianProduct{\OrthogonalPolynomial_i}{x \OrthogonalPolynomial_k} = 0$. Ne deduciamo $\ForAll{k \in i - 1}{\Scalar_k = 0}$.
\par Abbiamo dunque $\OrthogonalPolynomial_{i + 1} = \Scalar_{i - 1} \OrthogonalPolynomial_{i - 1} + \Scalar_i \OrthogonalPolynomial_i + \Scalar_{i + 1} x \OrthogonalPolynomial_i = 
(\Scalar_{i + 1} x + \Scalar_i) \OrthogonalPolynomial_i + \Scalar_{i - 1} \OrthogonalPolynomial_{i - 1}$. Poniamo
\begin{itemize}
	\item $A_{i + 1} = \Scalar_{i + 1}$;
	\item $B_{i + 1} = \Scalar_i$;
	\item $C_i = - \Scalar_{i - 1}$.
\end{itemize}
\par Abbiamo inoltre
\begin{itemize}
		\item $\HermitianProduct{\OrthogonalPolynomial_{i + 1}}{\OrthogonalPolynomial_{i + 1}} = A_{i + 1} \HermitianProduct{x \OrthogonalPolynomial_i}{\OrthogonalPolynomial_{i +1}}$, da cui $A_{i + 1} = \frac{\HermitianProduct{\OrthogonalPolynomial_{i + 1}}{\OrthogonalPolynomial_{i + 1}}}{\HermitianProduct{x\OrthogonalPolynomial_i}{\OrthogonalPolynomial_{i + 1}}}$;
		\item $\HermitianProduct{\OrthogonalPolynomial_{i + 1}}{\OrthogonalPolynomial_i} = A_{i + 1}\HermitianProduct{x\OrthogonalPolynomial_i}{\OrthogonalPolynomial_i} + B_{i + 1}\HermitianProduct{\OrthogonalPolynomial_i}{\OrthogonalPolynomial_i} = 0$, da cui $B_{i + 1} = - A_{i + 1} \frac{\HermitianProduct{x\OrthogonalPolynomial_i}{\OrthogonalPolynomial_i}}{\HermitianProduct{\OrthogonalPolynomial_i}{\OrthogonalPolynomial_i}}$;
		\item $\HermitianProduct{\OrthogonalPolynomial_{i + 1}}{\OrthogonalPolynomial_{i - 1}} = A_{i + 1}\HermitianProduct{x\OrthogonalPolynomial_i}{\OrthogonalPolynomial_{i - 1}} - C_i\HermitianProduct{\OrthogonalPolynomial_{i - 1}}{\OrthogonalPolynomial_{i - 1}} = 0$, da cui $C_i = A_{i + 1} \frac{\HermitianProduct{x\OrthogonalPolynomial_i}{\OrthogonalPolynomial_{i - 1}}}{\HermitianProduct{\OrthogonalPolynomial_{i - 1}}{\OrthogonalPolynomial_{i - 1}}} = \frac{A_{i + 1}}{A_i} \frac{\HermitianProduct{\OrthogonalPolynomial_i}{\OrthogonalPolynomial_i}}{\HermitianProduct{\OrthogonalPolynomial_{i - 1}}{\OrthogonalPolynomial_{i - 1}}}$.
\end{itemize}
\par Per l'ultima parte del teorema, osserviamo che la formula di ricorrenza a tre termini implica quanto segue:
\begin{itemize}
	\item il coefficiente $a_{i + 1}$ di $x^{i + 1}$ \`e uguale a $A_{i + 1}a_i$;
	\item il coefficente $b_{i + 1}$ di $x^i$ \`e uguale a $A_{i + 1} b_i + B_{i + 1}a_i$.
\end{itemize}
\par Ne seguono immediatamente le tre uguaglianze finali dell'enunciato. \EndProof
\begin{Corollary}
	Con le notazioni del teorema \ref{IstituzioniDiAnalisiNumerica_RicorrenzaATreTermini},
	se $\OrthogonalPolynomial_0(x) = 1$ e $\OrthogonalPolynomial_1(x) = x$, allora, per ogni $i \in N$ con $i > 1$, $\OrthogonalPolynomial_i$ \`e monico.
\end{Corollary}
\Proof Segue direttamente dalle espressioni individuate per $A_{i + 1}$ e $C_i$. \EndProof
\begin{Corollary}
	Con le notazioni del teorema \ref{IstituzioniDiAnalisiNumerica_RicorrenzaATreTermini},
	per ogni $i \in N$ con $i > 1$ abbiamo $A_{i + 1} \neq 0$ e $C_i \neq 0$. Inoltre se tutti i polinomi ortogonali sono monici, allora, per ogni $i \in N$ con $i > 1$, $C_i$ \`e reale e strettamente positivo.
\end{Corollary}
\Proof Segue direttamente dalle espressioni individuate per i coefficenti della ricorrenza a tre termini e dal fatto che i prodotti hermitiani sono forme sesquilineari definite positive. \EndProof
\begin{Corollary}
	Con le notazioni del teorema \ref{IstituzioniDiAnalisiNumerica_RicorrenzaATreTermini},
	per ogni $i > 1$, $\OrthogonalPolynomial_{i - 1}$ \`e il resto della divisione di $\OrthogonalPolynomial_{i + 1}$ per $\OrthogonalPolynomial_i$.
\end{Corollary}
\Proof Segue direttamente dal teorema \ref{IstituzioniDiAnalisiNumerica_RicorrenzaATreTermini} e dalla definizione di resto della divisione tra polinomi. \EndProof
\begin{Theorem}
	\label{IstituzioniDiAnalisiNumerica_Christoffel_Darboux}
	\TheoremName{Formula di Christoffel-Darboux}[di Christoffel-Darboux][formula]
	Nelle ipotesi del teorema \ref{IstituzioniDiAnalisiNumerica_RicorrenzaATreTermini}, assumendo $(\OrthogonalPolynomial_i)_{i \in \mathbb{N}} \in (\RingAdjunction{\mathbb{R}}{x})^\mathbb{N}$, per ogni $n \geq 0$ e per ogni $(x,y) \in \mathbb{R}^2$ abbiamo
	\[
	(x - y) \sum_{i = 0}^n h_i^{-1}\OrthogonalPolynomial_i(x)\OrthogonalPolynomial_i(y) = \gamma_n (\OrthogonalPolynomial_{n + 1}(x)\OrthogonalPolynomial_n(y) - \OrthogonalPolynomial_{n + 1}(y)\OrthogonalPolynomial_n(x)),
	\]
	dove, per ogni $i \in \mathbb{N}$, $h_i = \ScalarProduct{\OrthogonalPolynomial_i}{\OrthogonalPolynomial_i}$ e $\gamma_n = \frac{a_n}{a_{n + 1}h_n}$.
\end{Theorem}
\Proof Procediamo per induzione su $n$.
\par Se $n = 0$, allora abbiamo
\begin{align*}
\gamma_0((a_1x + b_1)a_0 - (a_1y + b_1)a_0)
&= \frac{a_0}{h_0a_1}a_0((a_1x + b_1) - (a_1y + b_1)),\\
&= \frac{a_0^2}{h_0a_1}a_1(x - y),\\
&= (x - y)h_0^{-1}a_0^2.
\end{align*}
\par Assumiamo ora il teorema dimostrato per $n = N - 1$ e proviamolo per $n = N$. Abbiamo
\begin{align*}
\OrthogonalPolynomial_{n + 1}(x)\OrthogonalPolynomial_n(y) - \OrthogonalPolynomial_{n + 1}(y)\OrthogonalPolynomial_n(x)
&= ((A_{n + 1}x + B_{i + 1})\OrthogonalPolynomial_n(x) - C_n\OrthogonalPolynomial_{n - 1}(x))\OrthogonalPolynomial_n(y) -
((A_{n + 1}y + B_{i + 1})\OrthogonalPolynomial_n(y) - C_n\OrthogonalPolynomial_{n - 1}(y))\OrthogonalPolynomial_n(x),\\
&= (x - y)A_{n + 1}\OrthogonalPolynomial_n(x)\OrthogonalPolynomial_n(y) + C_n(\OrthogonalPolynomial_n(x)\OrthogonalPolynomial_{n - 1}(y) - \OrthogonalPolynomial_n(y)\OrthogonalPolynomial_{n - 1}(x)),\\
&= (x - y)\left (A_{n + 1}\OrthogonalPolynomial_n(x)\OrthogonalPolynomial_n(y) + \frac{C_n}{\gamma_{n - 1}} \sum_{i = 0}^{n - 1} h_i^{-1}\OrthogonalPolynomial_i(x)\OrthogonalPolynomial_i(y)\right ),\\
&= (x - y)\left (\gamma_n^{-1}h_n^{-1}\OrthogonalPolynomial_n(x)\OrthogonalPolynomial_n(y) + \gamma_n^{-1} \sum_{i = 0}^{n - 1} h_i^{-1}\OrthogonalPolynomial_i(x)\OrthogonalPolynomial_i(y)\right ),\\
&= (x - y) \gamma_n^{-1} \sum_{i = 0}^n h_i^{-1}\OrthogonalPolynomial_i(x)\OrthogonalPolynomial_i(y).\text{\EndProof}
\end{align*}
\Proof Offriamo una dimostrazione alternativa. Il caso $n = 0$ si dimostra identicamente a quanto appena visto. Supponiamo nel seguito $n > 0$.
\par Dalla ricorrenza a tre termini deduciamo, per ogni intero $i > 0$.
\begin{itemize}
	\item $\OrthogonalPolynomial_{i + 1}(x)\OrthogonalPolynomial_i(y) = (A_{i + 1}x + B_{i + 1})\OrthogonalPolynomial_i(x)\OrthogonalPolynomial_i(y) - C_i\OrthogonalPolynomial_{i - 1}(x)\OrthogonalPolynomial_i(y)$;
	\item $\OrthogonalPolynomial_{i + 1}(y)\OrthogonalPolynomial_i(x) = (A_{i + 1}y + B_{i + 1})\OrthogonalPolynomial_i(y)\OrthogonalPolynomial_i(x) - C_i\OrthogonalPolynomial_{i - 1}(y)\OrthogonalPolynomial_i(x)$.
\end{itemize}
\par Sottraendo la seconda equazione alla prima otteniamo
$\OrthogonalPolynomial_{i + 1}(x)\OrthogonalPolynomial_i(y) - \OrthogonalPolynomial_{i + 1}(y)\OrthogonalPolynomial_i(x) =
(x - y)A_{i + 1}\OrthogonalPolynomial_i(x)\OrthogonalPolynomial_i(y) - C_i(\OrthogonalPolynomial_{i - 1}(x)\OrthogonalPolynomial_i(y) - \OrthogonalPolynomial_{i - 1}(y)\OrthogonalPolynomial_i(x))$,
equivalente a
$(x - y)A_{i + 1}\OrthogonalPolynomial_i(x)\OrthogonalPolynomial_i(y) = \OrthogonalPolynomial_{i + 1}(x)\OrthogonalPolynomial_i(y) - \OrthogonalPolynomial_{i + 1}(y)\OrthogonalPolynomial_i(x) + C_i(\OrthogonalPolynomial_{i - 1}(x)\OrthogonalPolynomial_i(y) - \OrthogonalPolynomial_{i - 1}(y)\OrthogonalPolynomial_i(x))$.
Dividamo ambi i membri per $h_iA_{i + 1}$:
$(x - y)h_i^{-1}\OrthogonalPolynomial_i(x)\OrthogonalPolynomial_i(y) = h_i^{-1}A_{i + 1}^{-1} \OrthogonalPolynomial_{i + 1}(x)\OrthogonalPolynomial_i(y) - \OrthogonalPolynomial_{i + 1}(y)\OrthogonalPolynomial_i(x) + \frac{C_i}{h_iA_{i + 1}}(\OrthogonalPolynomial_{i - 1}(x)\OrthogonalPolynomial_i(y) - \OrthogonalPolynomial_{i - 1}(y)\OrthogonalPolynomial_i(x))$,
da cui
$(x - y)h_i^{-1}\OrthogonalPolynomial_i(x)\OrthogonalPolynomial_i(y) = \gamma_i\OrthogonalPolynomial_{i + 1}(x)\OrthogonalPolynomial_i(y) - \OrthogonalPolynomial_{i + 1}(y)\OrthogonalPolynomial_i(x) + \gamma_{i - 1}(\OrthogonalPolynomial_{i - 1}(x)\OrthogonalPolynomial_i(y) - \OrthogonalPolynomial_{i - 1}(y)\OrthogonalPolynomial_i(x))$.
\par La tesi segue sommando ambi i membri per $i \in n + 1$. \EndProof
\begin{Theorem}
	\label{IstituzioniDiAnalisiNumerica_TeoremaDiOrtogonalitaDiscreta1}
	\TheoremName{Teorema di ortogonalit\`a discreta}[di ortogonalit\`a discreta][teorema] Siano $(\OrthogonalPolynomial_i)_{i \in \mathbb{N}}$ polinomi ortogonali rispetto ad un prodotto scalare $\VarScalarProduct{\cdot}{\cdot}$ su $\RingAdjunction{\mathbb{R}}{x}$ per cui vale la formula di Christoffel-Darboux. Siano inoltre
	\begin{itemize}
		\item $n \in \mathbb{N}$;
		\item $(x_i)_{i \in m}$ ($m \leq n$) gli zeri distinti di $\OrthogonalPolynomial_n$;
		\item $(\Vector^{(j)})_{j \in n}$ i vettori di $\mathbb{R}^n$ definiti da $\ForAll{(i,j) \in n \times n}{\Vector^{(j)}_i = \OrthogonalPolynomial_i(x_j)}$;
		\item $(\sigma_k)_{k \in n}$ gli scalari definiti da $\ForAll{k \in n}{\sigma_k = \Norm{\Vector^{(k)}}^{-1}}$;
		\item $\ScalarProduct{\cdot}{\cdot}$ il prodotto scalare definito su $\mathbb{R}^n$ da $\ScalarProduct{\Vector}{\VarVector} = \sum_{k \in n} \sigma_k^2\Vector_k\VarVector_k$.
	\end{itemize}
	Allora i vettori $(\Vector^{(j)})_{j \in n}$ sono non nulli e ortogonali rispetto a $\ScalarProduct{\cdot}{\cdot}$.
\end{Theorem}
\Proof Denotiamo $\gamma_n = \frac{\LeadingCoefficient{\OrthogonalPolynomial_n}}{\LeadingCoefficient{\OrthogonalPolynomial_{n + 1}}\VarScalarProduct{\OrthogonalPolynomial_n}{\OrthogonalPolynomial_n}}$ e $(\OrthonormalPolynomial_i)_{i \in n}$ i polinomi definiti da $\ForAll{i \in n}{\OrthonormalPolynomial_i = \ScalarProduct{\OrthogonalPolynomial_i}{\OrthogonalPolynomial_i}^{-\frac{1}{2}}\OrthogonalPolynomial_i}$;
\par Abbiamo
\begin{align*}
	\Norm{\Vector^{(i)}}^2
	&= \ScalarProduct{\Vector^{(i)}}{\Vector^{(i)}},\\
	&= \sum_{k \in n} \sigma_k^2 \OrthogonalPolynomial_k(x_i)^2,\\
	&= \sum_{k \in n} \OrthonormalPolynomial_k(x_i)^2,\\
	&\geq \OrthonormalPolynomial_0(x_i)^2 = 1.
\end{align*}
\par Inoltre, per $i, j \in n$ con $i \neq j$,
\begin{align*}
	\ScalarProduct{\Vector^{(i)}}{\Vector^{(j)}}
	&= \sum_{k \in n} \sigma_k^2 \OrthogonalPolynomial_k(x_i)\OrthogonalPolynomial_k(x_j),\\
	&= \sum_{k \in n} \OrthonormalPolynomial_k(x_i)\OrthonormalPolynomial_k(x_j),\\
	&= \gamma_n(\OrthonormalPolynomial_n(x_i)\OrthonormalPolynomial_{n - 1}(x_j) - \OrthonormalPolynomial_n(x_j)\OrthonormalPolynomial_{n - 1}(x_i)) = 0.\text{ \EndProof}
\end{align*}
\begin{Theorem}
	\label{IstituzioniDiAnalisiNumerica_TeoremaDiOrtonormalitaDiscreta}
	\TheoremName{Teorema di ortonormalit\`a discreta}[di ortonormalit\`a discreta][teorema] Assumiamo tutte le ipotesi del teorema \ref{IstituzioniDiAnalisiNumerica_TeoremaDiOrtogonalitaDiscreta1}. Assumiamo inoltre
	\begin{itemize}
		\item $m = n$;
		\item $(\OrthonormalPolynomial_i)_{i \in n}$ i polinomi definiti da $\ForAll{i \in n}{\OrthonormalPolynomial_i = \ScalarProduct{\OrthogonalPolynomial_i}{\OrthogonalPolynomial_i}^{-\frac{1}{2}}\OrthogonalPolynomial_i}$;
		\item $(\VarVector^{(j)})_{j \in n}$ i vettori di $\mathbb{R}^n$ definiti da $\ForAll{(i,j) \in n \times n}{\VarVector^{(j)}_i = \OrthonormalPolynomial_i(x_j)}$;
	\end{itemize}
	Allora i vettori $(\VarVector^{(j)})_{j \in n}$ sono non nulli e ortonormali rispetto a $\ScalarProduct{\cdot}{\cdot}$.
\end{Theorem}
\Proof Sia $V \in \mathbb{R}^{n \times n}$ la matrice tale che, per ogni $j \in n$, $W^j = \VarVector^{(j)}$: per il teorema \ref{IstituzioniDiAnalisiNumerica_TeoremaDiOrtogonalitaDiscreta1} $\Transposed{W}W$ \`e una matrice diagonale. Sia $\DiagonalMatrix \in \mathbb{R}^{n \times n}$ la matrice diagonale tale che $\ForAll{k \in n}{\DiagonalMatrix_k^k = \sigma_k}$: abbiamo $W\DiagonalMatrix = \Identity$. Abbiamo
\begin{itemize}
	\item $W\DiagonalMatrix$ \`e invertibile perch\'e lo sono $W$ e $\DiagonalMatrix$;
	\item $\Transposed{(W\DiagonalMatrix)}W\DiagonalMatrix = \DiagonalMatrix \Transposed{W}W \DiagonalMatrix = \Identity$.
\end{itemize}
\par Dunque $W\DiagonalMatrix$ \`e ortogonale. Abbiamo allora $W\DiagonalMatrix\Transposed{(W\DiagonalMatrix)} = (W\DiagonalMatrix) \DiagonalMatrix\Transposed{W} = \Identity$, da cui, per ogni $(i,j) \in n \times n$, $\sum_{k \in n} (\VarVector_k^{(j)}\sigma_k)(\sigma_k\VarVector_k^{(i)}) = \sum_{k \in n} \sigma_k^2 \OrthonormalPolynomial_k(x_j) \OrthonormalPolynomial_k(x_i) = \KroneckerDelta{i}{j}$. \EndProof
\begin{Definition}
	\label{IstituzioniDiAnalisiNumerica_Bezoutiano}
	Siano $\Polynomial, \VarPolynomial \in \RingAdjunction{\mathbb{C}}{x}$. Fissata un'ulteriore incognita $y$, definiamo \Define{b\'ezoutiano} di $\Polynomial$ e $\VarPolynomial$, denotato $\Bezoutian{\Polynomial}{\VarPolynomial}$, l'espressione  $\Bezoutian{\Polynomial}{\VarPolynomial}(x,y) = \frac{\Polynomial(x)\VarPolynomial(y) - \Polynomial(y)\VarPolynomial(x)}{x - y}$.
\end{Definition}
\begin{Theorem}
	Con le notazioni della definizione \ref{IstituzioniDiAnalisiNumerica_Bezoutiano}, $\Bezoutian{\Polynomial}{\VarPolynomial}(x,y)$ \`e un polinomio di $\RingAdjunction{\mathbb{C}}{x,y}$ di grado al pi\`u $2(n + 1)$, dove $n = \max \lbrace \PolynomialDegree{\Polynomial}, \PolynomialDegree{\VarPolynomial} \rbrace$. 
\end{Theorem}
\begin{Definition}
	\label{IstituzioniDiAnalisiNumerica_MatriceBezoutiana}
	Con le notazioni del teorema precedente, definiamo \Define{matrice b\'ezoutiana}[b\'ezoutiana][matrice] associata a $\Polynomial$ e $\VarPolynomial$ la matrice quadrata $\BezoutianMatrix$ di ordine $n$ tale che $\Bezoutian{\Polynomial}{\VarPolynomial}(x,y) = \sum_{(i,j) \in n \times n} \BezoutianMatrix_i^j x^i y^j$. Definiamo inoltre \Define{risultante}[di due polinomi][risultante] di $\Polynomial$ e $\VarPolynomial$, denotato $\Resultant{\Polynomial}{\VarPolynomial}$, lo scalare $\Determinant{\BezoutianMatrix}$.
\end{Definition}
\begin{Theorem}
	Con le notazioni della definizione \ref{IstituzioniDiAnalisiNumerica_MatriceBezoutiana},
	\begin{itemize}
		\item $\BezoutianMatrix$ \`e simmetrica;
		\item $\BezoutianMatrix$ \`e invertibile se e solo se $\Polynomial$ e $\VarPolynomial$ sono coprimi;
		\item $\Bezoutian{\Polynomial}{\VarPolynomial}(x,y) = \Transposed{(x^i)_{i \in n}} \BezoutianMatrix (y^i)_{i \in n}$.
	\end{itemize}
\end{Theorem}
\begin{Theorem}
	Nelle ipotesi del teorema \ref{IstituzioniDiAnalisiNumerica_Christoffel_Darboux}, fissiamo $n \in \mathbb{N}$ e, per ogni $k \in \mathbb{N}$, poniamo $\vec{\OrthogonalPolynomial_k}$ uguale al vettore dei coefficienti di $\OrthogonalPolynomial_n$ in $\mathbb{C}^n$. Sia inoltre $\BezoutianMatrix$ la matrice b\'ezoutiana associata ai polinomi $\OrthogonalPolynomial_n$ e $\OrthogonalPolynomial_{n + 1}$. Abbiamo
	\begin{itemize}
		\item $\Bezoutian{\OrthogonalPolynomial_n}{\OrthogonalPolynomial_{n + 1}}$ \`e somma di prodotti di polinomi ortogonali;
		\item $\BezoutianMatrix = \sum_{k = 1}^n \gamma_n^{-1} h^{-k} \vec{\OrthogonalPolynomial_k} \Transposed{\vec{\OrthogonalPolynomial_k}}$;
		\item $\gamma_n \BezoutianMatrix = \Transposed{L}L$, dove $L$ \`e la matrice triangolare inferiore quadrata di ordine $n$ tale che, per ogni $i \in n$, $L_i = h_i^{-\frac{1}{2}} \Transposed{\vec{\OrthogonalPolynomial_i}}$.
	\end{itemize}
\end{Theorem}
%c'e' un'altro teorema sulla fattorizzazione UL. Se voglio aggiungerlo posso forse capirci qualcosa usando l'articolo https://www.tu-chemnitz.de/mathematik/preprint/2008/PREPRINT_09.php

\subsection{Matrici tridiagonali.}
\label{IstituzioniDiAnalisiNumerica_MatriciTridiagonali}
\begin{Theorem}
	\label{IstituzioniDiAnalisiNumerica_MatriciTridiagonali_Teorema}
	Con le ipotesi e le notazioni del teorema \ref{IstituzioniDiAnalisiNumerica_RicorrenzaATreTermini}, posto inoltre $A_1 = a_1$ e $B_1 = b_1$, sia $\TridiagonalMatrix_n$ la matrice tridiagonale definita da
	\[
		(\TridiagonalMatrix_n)_i^j =
		\begin{cases}
			A_i x + B_i&\text{ se }i = j,\\
			- a_ 0&\text{ se }(i,j) = (1,2),\\
			- C_j&\text{ se }i = j + 1,\\
			- 1&\text{ se }\And{j = i + 1}{i \neq 1};
		\end{cases}
	\]
	vale a dire
	\[
	\TridiagonalMatrix_n =
	\lmatrix
	\begin{array}{ccccc}
		A_1x + B_1	&	-a_0		&		&		&\\
		- C_1		&	A_2x + B_2	&	-1	&		&\\
				&	\ddots		&	\ddots	&	\ddots	&\\
				&			&	\ddots	&	\ddots	&	- 1\\
				&			&	&	-C_{n - 1}	&	A_nx + B_n
	\end{array}
	\rmatrix.
	\]
	Abbiamo $\ForAll{n \in \NotZero{\mathbb{N}}}{\Determinant \TridiagonalMatrix_n = \OrthogonalPolynomial_n}$.
\end{Theorem}
\Proof Procediamo per induzione su $n$.
\par Abbiamo $\Determinant{\TridiagonalMatrix_1}(x) = A_1x + B_1 = \OrthogonalPolynomial_1(x)$ per costruzione e $\Determinant{\TridiagonalMatrix_2}(x) = (A_2x + B_2)(A_1x + B_1) - C_1a_0 = \OrthogonalPolynomial_2$ per il teorema di ricorrenza a tre termini.
\par Supponiamo la tesi dimostrata per $n = N$ e proviamola per $n = N + 1$. Abbiamo, sviluppando il determinante con la regola di Laplace rispetto all'ultima riga,
$\Determinant{\TridiagonalMatrix_n}(x) = (A_nx + B_n) \OrthogonalPolynomial_{n - 1}(x) - C_{n - 1} \OrthogonalPolynomial_{n - 2}(x) = \OrthogonalPolynomial_n(x)$. \EndProof
\begin{Theorem}
	Con le ipotesi e le notazioni del teorema \ref{IstituzioniDiAnalisiNumerica_MatriciTridiagonali_Teorema}, per ogni $\xi \in \mathbb{C}$, vale $\TridiagonalMatrix_n(\xi) (\OrthogonalPolynomial_i(\xi))_{i \in n} = \Vector$, dove $\ForAll{i \in n - 1}{\Vector_i = 0}$ e $\Vector_n = \OrthogonalPolynomial_n(\xi)$; vale a dire
	\[
		\TridiagonalMatrix_n \lmatrix \begin{array}{c} 0\\ \vdots\\ 0\\ \OrthogonalPolynomial_n(\xi)\end{array} \rmatrix.
	\] 
\end{Theorem}
\Proof Abbiamo
\begin{itemize}
	\item $(\TridiagonalMatrix_n(\xi) (\OrthogonalPolynomial_j(\xi))_{j \in n})_1 = (a_1\xi + b_1)a_0 - a_0(a_1\xi + b_1) = 0$;
	\item $(\TridiagonalMatrix_n(\xi) (\OrthogonalPolynomial_j(\xi))_{j \in n})_i = - C_{i - 1} \OrthogonalPolynomial_{i - 2}(\xi) + (A_i \xi + B_i)\OrthogonalPolynomial_{i - 1}(\xi) - \OrthogonalPolynomial_i(\xi) = 0$;
	\item $(\TridiagonalMatrix_n(\xi) (\OrthogonalPolynomial_j(\xi))_{j \in n})_n = - C_{n - 1} \OrthogonalPolynomial_{n - 2}(\xi) + (A_n \xi + B_n)\OrthogonalPolynomial_{n - 1}(\xi) = \OrthogonalPolynomial_n(\xi)$. \EndProof
\end{itemize}
\begin{Corollary}
	Con le notazioni del teorema precedente, $(\OrthogonalPolynomial_i(\xi))_{i \in n} \in \Kernel{\TridiagonalMatrix_n(\xi)}$ se e solo se $\xi$ \`e radice di $\OrthogonalPolynomial_i$.
\end{Corollary}
\Proof Segue direttamente dal teorema precedente. \EndProof
\begin{Theorem}
	\label{IstituzioniDiAnalisiNumerica_RadiciEAutovalori}
	Con le ipotesi e le notazioni del teorema \ref{IstituzioniDiAnalisiNumerica_MatriciTridiagonali_Teorema}, supposti gli $(\OrthogonalPolynomial_i)_{i \in \mathbb{N}}$ monici, il polinomio caratteristico $\CharacteristicPolynomial_n(x)$ della matrice
	\[
		(\VarTridiagonalMatrix_n)_i^j =
		\begin{cases}
			- B_i&\text{ se }i = j,\\
			C_j&\text{ se }i = j + 1,\\
			1&\text{ se }j = i + 1;
		\end{cases}
	\]
	vale a dire
	\[
	\VarTridiagonalMatrix_n =
	\lmatrix
	\begin{array}{ccccc}
		- B_1	&	1		&		&		&\\
		C_1		&	- B_2	&	1	&		&\\
				&	\ddots		&	\ddots	&	\ddots	&\\
				&			&	\ddots	&	\ddots	&	1\\
				&			&	&	C_{n - 1}	&	- B_n
	\end{array}
	\rmatrix,
	\]
	\`e $- \OrthogonalPolynomial_n(x)$.
	Inoltre,
	\begin{itemize}
		\item $\xi \in \mathbb{C}$ \`e autovalore di $\VarTridiagonalMatrix_n$ se e solo se \`e radice di $\OrthogonalPolynomial_n(x)$;
		\item $\Eigenvector \in \mathbb{C}^n$ \`e autovettore di $\VarTridiagonalMatrix_n$ relativo ad un autovalore $\xi$ se e solo se \`e della forma $\Eigenvector = (\OrthogonalPolynomial_i(\xi))_{i \in n}$;
		\item definita la matrice diagonale $\DiagonalMatrix \in \mathbb{C}^{n \times n}$ come $\DiagonalMatrix_j^j = \begin{cases}1\text{ se }j = 1,\\\DiagonalMatrix_{j - 1}^{j - 1}\sqrt{C_{j - 1}}\text{ altrimenti}.\end{cases}$, la matrice $\DiagonalMatrix^{-1}\VarTridiagonalMatrix\DiagonalMatrix$ \`e simmetrica; pi\`u precisamente abbiamo
	\[
	\DiagonalMatrix^{-1}\VarTridiagonalMatrix\DiagonalMatrix =
	\begin{cases}
		- B_i&\text{ se }i = j,\\
		\sqrt{C_j}&\text{ se }i = j + 1,\\
		\sqrt{C_{j - 1}}&\text{ se }j = i + 1,
	\end{cases}
	\]
	vale a dire
	\[
		\DiagonalMatrix^{-1}\VarTridiagonalMatrix\DiagonalMatrix =
		\lmatrix
		\begin{array}{ccccc}
			- B_1	&	\sqrt{C_2}		&		&		&\\
			\sqrt{C_2}		&	- B_2	&	\sqrt{C_3}	&		&\\
					&	\ddots		&	\ddots	&	\ddots	&\\
					&			&	\ddots	&	\ddots	&	\sqrt{C_{n - 1}}\\
					&			&	&	\sqrt{C_{n - 1}}	&	- B_n
		\end{array}
		\rmatrix.
	\]
	\end{itemize}
\end{Theorem}
\Proof Il polinomio caratteristico $\CharacteristicPolynomial(x)$ di $\VarTridiagonalMatrix_n$ \`e dato dal determinante della matrice $\VarTridiagonalMatrix_n - x \Identity$ che si verifica immediatamente essere la matrice $- \TridiagonalMatrix_n(x)$.
\par Dunque $\CharacteristicPolynomial(x) = - \Determinant{\TridiagonalMatrix_n(x))} = - \OrthogonalPolynomial_n(x)$.
\par Infine, segue immediatamente dal corollario precedente che $(\OrthogonalPolynomial_i(\xi))_{i \in n} \in \Kernel (\VarTridiagonalMatrix - \xi \Identity ) = \Kernel{\TridiagonalMatrix_n(\xi)}$ se e solo se $\xi$ \`e autovalore di $\VarTridiagonalMatrix$, o equivalentemente $\xi$ \`e radice di $\OrthogonalPolynomial_n(x)$.
\par Infine abbiamo
\[
\DiagonalMatrix^{-1}\VarTridiagonalMatrix\DiagonalMatrix =
\begin{cases}
	- B_i&\text{ se }i = j,\\
	\left (\DiagonalMatrix_{j + 1}^{j + 1} \right)^{-1}C_j\DiagonalMatrix_j^j = \left (\DiagonalMatrix_j^j \right )^{-1} C_j^{-\frac{1}{2}} C_j D_j^j = \sqrt{C_j}&\text{ se }i = j + 1,\\
	\left (\DiagonalMatrix_{j - 1}^{j - 1} \right )^{-1}\DiagonalMatrix_j^j = \left (\DiagonalMatrix_{j - 1}^{j - 1} \right )^{-1} \DiagonalMatrix_{j - 1}^{j - 1} \sqrt{C_{j - 1}} = \sqrt{C_{j - 1}}&\text{ se }j = i + 1.\text{ \EndProof}
\end{cases}
\]
\begin{Theorem}
	\label{IstituzioniDiAnalisiNumerica_TeoremaDiOrtogonalitaDiscreta2}
	\TheoremName{Teorema di ortogonalit\`a discreta (solo caso monico\footnote{Abbiamo gi\`a dimostrato il caso generale con il teorema \ref{IstituzioniDiAnalisiNumerica_TeoremaDiOrtogonalitaDiscreta1}.})}[di ortogonalit\`a discreta][teorema] Siano $(\OrthogonalPolynomial_i)_{i \in \mathbb{N}}$ polinomi ortogonali monici rispetto ad un prodotto scalare $\VarScalarProduct{\cdot}{\cdot}$ su $\RingAdjunction{\mathbb{R}}{x}$ per cui valgono le ipotesi del teorema di ricorrenza a tre termini. Siano inoltre
	\begin{itemize}
		\item $n \in \mathbb{N}$;
		\item $(x_i)_{i \in m}$ ($m \leq n$) gli zeri distinti di $\OrthogonalPolynomial_n$;
		\item $(\Vector^{(j)})_{j \in n}$ i vettori di $\mathbb{R}^n$ definiti da $\ForAll{(i,j) \in n \times n}{\Vector^{(j)}_i = \OrthogonalPolynomial_i(x_j)}$;
		\item $(\sigma_k)_{k \in n}$ gli scalari definiti da $\ForAll{k \in n}{\sigma_k = \Norm{\Vector^{(k)}}^{-1}}$;
		\item $\ScalarProduct{\cdot}{\cdot}$ il prodotto scalare definito su $\mathbb{R}^n$ da $\ScalarProduct{\Vector}{\VarVector} = \sum_{k \in n} \sigma_k^2\Vector_k\VarVector_k$.
	\end{itemize}
	Allora i vettori $(\Vector^{(j)})_{j \in n}$ sono non nulli e ortogonali rispetto a $\ScalarProduct{\cdot}{\cdot}$.
\end{Theorem}
\Proof Con le notazioni del teorema precedente,
\begin{itemize}
	\item $\DiagonalMatrix_i^i = \sqrt{h_i}$: infatti $\DiagonalMatrix_1^1 = 1$ e, per $i > 1$, assumendo $\DiagonalMatrix_{i - 1}^{i - 1} = \sqrt{h_{i - 1}}$ abbiamo $\DiagonalMatrix_i^i = \DiagonalMatrix_{i - 1}^{i - 1}\sqrt{C_{i - 1}} = \sqrt{h_{i - 1}} \frac{\sqrt{h_i}}{\sqrt{h_{i - 1}}} = \sqrt{h_i}$;
	\item $\DiagonalMatrix^{-1}(\OrthogonalPolynomial_k(x_i))_{k \in n} = (h_k^{-\frac{1}{2}}\OrthogonalPolynomial_k(x_i))_{k \in n}$ \`e autovettore relativo a $x_i$ di $\DiagonalMatrix^{-1}\VarTridiagonalMatrix\DiagonalMatrix$;
	\item $\DiagonalMatrix^{-1}(\OrthogonalPolynomial_k(x_j))_{k \in n} = (h_k^{-\frac{1}{2}}\OrthogonalPolynomial_k(x_i))_{k \in n}$ \`e autovettore relativo a $x_j$ di $\DiagonalMatrix^{-1}\VarTridiagonalMatrix\DiagonalMatrix$.
\end{itemize}
\par Poich\'e $\DiagonalMatrix^{-1}\VarTridiagonalMatrix\DiagonalMatrix$ \`e una matrice reale simmetrica, e dunque normale, gli autovettori corrispondenti ad autovalori distinti sono ortogonali per il prodotto scalare standard. Ne segue immediatamente la tesi del teorema. \EndProof
\begin{Definition}
	\label{IstituzioniDiAnalisiNumerica_QuozienteDiRayleigh}
	Sia $\HermitianMatrix \in \mathbb{C}^{n \times n}$ ($n \in \mathbb{N}$). Chiamiamo \Define{quoziente di Rayleigh}[di Rayleigh][quoziente] relativo a $\HermitianMatrix$ l'applicazione $\RayleighQuotient: \NotZero{\mathbb{C}^n} \rightarrow \mathbb{R}$ che a $x \in \NotZero{\mathbb{C}^n}$ associa $\frac{\HermitianTransposed{x}\HermitianMatrix x}{\HermitianTransposed{x}x}$.
\end{Definition}
\begin{Theorem}
	Con le notazioni della definizione \ref{IstituzioniDiAnalisiNumerica_QuozienteDiRayleigh}, fissato $j \in n$, abbiamo$\PartialDerivative{\RayleighQuotient}{x_j} = \frac{2}{\HermitianTransposed{x}{x}} (\HermitianMatrix - \RayleighQuotient(x)x)_j$.
\end{Theorem}
\Proof Abbiamo
\begin{align*}
	\PartialDerivative{\RayleighQuotient}{x_j}
	&= \frac{\PartialDerivative{\HermitianTransposed{x}\HermitianMatrix x}{x_j} \HermitianTransposed{x}x - \HermitianTransposed{x}\HermitianMatrix x\PartialDerivative{\HermitianTransposed{x}x}{x_j}}{(\HermitianTransposed{x}x)^2},\\
	&= \frac{\PartialDerivative{\HermitianTransposed{x}\HermitianMatrix x}{x_j}}{\HermitianTransposed{x}x} - \frac{\HermitianTransposed{x}\HermitianMatrix x\PartialDerivative{\HermitianTransposed{x}x}{x_j}}{(\HermitianTransposed{x}x)},\\
	&= \frac{2(\HermitianMatrix x)_j}{\HermitianTransposed{x}x} - \frac{\HermitianTransposed{x}\HermitianMatrix x \cdot 2 x_j}{(\HermitianTransposed{x}x)},\\
	&= \frac{2}{\HermitianTransposed{x}x} (\HermitianMatrix x - \RayleighQuotient(x)x)_j.\text{ \EndProof}
\end{align*}
\begin{Corollary}
	Con le notazioni della definizione \ref{IstituzioniDiAnalisiNumerica_QuozienteDiRayleigh}, $\Gradient{\RayleighQuotient} = \frac{2}{\HermitianTransposed{x}{x}}(\HermitianMatrix - \RayleighQuotient \Identity) x$.
\end{Corollary}
\Proof Segue direttamente dal teorema precedente. \EndProof
\begin{Corollary}
	Con le notazioni della definizione \ref{IstituzioniDiAnalisiNumerica_QuozienteDiRayleigh}, i punti stazionari di $\RayleighQuotient$ sono tutti e soli gli autovettori di $\HermitianMatrix$. Il punto stazinario $\Eigenvector$ corrisponde all'autoavalore $\RayleighQuotient(\Eigenvector)$.
\end{Corollary}
\Proof $\Eigenvector$ \`e punto stazionario di $\RayleighQuotient$ se e solo se $(\HermitianMatrix - \RayleighQuotient \Identity)\Eigenvector = 0$, equivalente a $\HermitianMatrix \Eigenvector =\RayleighQuotient(\Eigenvector) \Eigenvector$. \EndProof
\begin{Corollary}
	Con le notazioni della definizione \ref{IstituzioniDiAnalisiNumerica_QuozienteDiRayleigh}, abbiamo
	\begin{itemize}
		\item $\min \Spectrum{\HermitianMatrix} = \min_{x \in \Dom{\RayleighQuotient}} \RayleighQuotient(x)$;
		\item $\max \Spectrum{\HermitianMatrix} = \max_{x \in \Dom{\RayleighQuotient}} \RayleighQuotient(x)$.
	\end{itemize}
\end{Corollary}
\Proof Poich\'e $\HermitianMatrix$ \`e hermitiana esiste una matrice unitaria $\UnitaryMatrix$ tale che $\UnitaryMatrix^{-1}\HermitianMatrix\UnitaryMatrix$ \`e diagonale, dove gli elementi diagonali sono precisamente gli autovalori di $\HermitianMatrix$. Abbiamo allora
\begin{align*}
	\RayleighQuotient(x)
	&= \frac{\HermitianTransposed{x}\HermitianMatrix x}{\HermitianTransposed{x}x}, \\
	&= \frac{\HermitianTransposed{x}\UnitaryMatrix^{-1}\HermitianMatrix\UnitaryMatrix x}{\HermitianTransposed{x}x}, \\
	&= \frac{\sum_{i \in n} \Eigenvalue_i \Conjugate{x_i}x_i}{\Conjugate{x_i}x_i},
\end{align*}
dove $(\Eigenvalue_i)_{i \in n}$ \`e la successione degli autovalori di $\HermitianMatrix$ tale che $\ForAll{i \in n}{\Eigenvalue_i = (\UnitaryMatrix^{-1}\HermitianMatrix\UnitaryMatrix)_i^i}$.
\par Vediamo dunque che, per ogni $x \in \Dom{\RayleighQuotient}$, $\RayleighQuotient$ \`e una media pesata degli autovalori (con molteplicit\`a algebrica) di $\HermitianMatrix$: dunque
$\Spectrum{\HermitianMatrix} \subseteq [\min_{x \in \Dom{\RayleighQuotient}} \RayleighQuotient(x),\max_{x \in \Dom{\RayleighQuotient}} \RayleighQuotient(x)]$.
\par Infine, valutando $\RayleighQuotient$ sugli autovettori di norma $1$ relativi al minimo e al massimo di $\Spectrum{\HermitianMatrix}$, si vede subito che $\min \Spectrum{\HermitianMatrix}, \max \Spectrum{\HermitianMatrix} \in \Image{\RayleighQuotient}$. \EndProof 
\begin{Theorem}
	\TheoremName{Teorema del minimax di Courant - Fisher} Siano
	\begin{itemize}
		\item $n \in \NotZero{\mathbb{N}}$;
		\item $\HermitianMatrix \in \mathbb{C}^{n \times n}$;
		\item $(\Eigenvalue_i)_{i \in \NaturalShift{n}} \in \Spectrum{\HermitianMatrix}^n$ tale che
		\begin{itemize}
			\item $(\Eigenvalue_i)_{i \in \NaturalShift{n}}$ sia una successione non crescente;
			\item ogni autovalore di $\HermitianMatrix$ compaia in $(\Eigenvalue_i)_{i \in n}$ un numero di volte pari alla sua molteplicit\`a algebrica;
		\end{itemize}
		\item per ogni $k \in n$, sia $S_k$ la classe di tutti i sottospazi di $\mathbb{C}^n$ di dimensione $k$.
	\end{itemize}
	Allora
	\begin{itemize}
		\item $\ForAll{k \in \NaturalShift{n}}{\max_{\LinearSpace \in S_k} \min_{\substack{x \in \LinearSpace\\x \neq 0}} \RayleighQuotient(x) = \Eigenvalue_k}$.
		\item $\ForAll{k \in \NaturalShift{n}}{\min_{\LinearSpace \in S_k} \max_{\substack{x \in \LinearSpace\\x \neq 0}} \RayleighQuotient(x) = \Eigenvalue_{n - k + 1}}$;
	\end{itemize}
\end{Theorem}
\Proof $\HermitianMatrix$ \`e hermitiana, dunque esiste una base $(\OrthonormalBase_i)_{i \in n}$ di $\mathbb{C}^n$ tale che
\begin{itemize}
	\item $\ForAll{i \in n}{\HermitianMatrix\OrthonormalBase_i = \Eigenvalue_i\OrthonormalBase_i}$;
	\item $(\OrthonormalBase_i)_{i \in n}$ \`e ortonormale.
\end{itemize}
\par Fissato $k \in n$, siano
\begin{itemize}
	\item $\LinearSpace \in S_k$;
	\item $\VarLinearSpace = \LinearSpan{(\OrthonormalBase_i)_{i \in n \SetMin k}}$.
\end{itemize}
\par Per la formula di Grassmann, $\LinearDimension{\LinearSpace \cap \VarLinearSpace} = \LinearDimension{\LinearSpace} + \LinearDimension{\VarLinearSpace} - \LinearDimension{\LinearSpace + \VarLinearSpace} = k + n - k + 1 - \LinearDimension{\LinearSpace + \VarLinearSpace} \geq 1$, dunque $\LinearSpace \cap \VarLinearSpace \neq \lbrace 0 \rbrace$. Sia $\Vector \in \LinearSpace \cap \VarLinearSpace \SetMin \lbrace 0 \rbrace$ e sia $(\Scalar_i)_{i \in n} \in \mathbb{C}^n$ tale che $\Vector = \sum_{i \in n} \Scalar_i \OrthonormalBase_i$.
\par Poich\'e $\HermitianMatrix$ \`e hermitiana esiste una matrice unitaria $\UnitaryMatrix$ tale che $\UnitaryMatrix^{-1}\HermitianMatrix\UnitaryMatrix$ \`e diagonale, dove gli elementi diagonali sono precisamente gli autovalori di $\HermitianMatrix$. Abbiamo allora, considerando il quoziente di Rayleigh $\RayleighQuotient(x)$ definito dalla matrice $\HermitianMatrix$,
\begin{align*}
	\RayleighQuotient(\Vector)
	&= \frac{\HermitianTransposed{\Vector}\HermitianMatrix \Vector}{\HermitianTransposed{\Vector}\Vector}, \\
	&= \frac{\HermitianTransposed{\Vector}\UnitaryMatrix^{-1}\HermitianMatrix\UnitaryMatrix \Vector}{\HermitianTransposed{\Vector}\Vector}, \\
	&= \frac{\sum_{i \in n} \Eigenvalue_i \Conjugate{\Vector_i}\Vector_i}{\Conjugate{\Vector_i}\Vector_i},
	&= \frac{\sum_{i = k}^n \Eigenvalue_i \Conjugate{\Vector_i}\Vector_i}{\Conjugate{\Vector_i}\Vector_i},
	&\leq \frac{\sum_{i = k}^n \Eigenvalue_k \Conjugate{\Vector_i}\Vector_i}{\Conjugate{\Vector_i}\Vector_i} = \Eigenvalue_k.
\end{align*}
\par Ne deduciamo $\max_{\LinearSpace \in S_k} \min_{\substack{x \in \LinearSpace\\x \neq 0}} \RayleighQuotient(x) \leq \Eigenvalue_k$.
\par D'altra parte, per $\VarVector \in \LinearSpace{(\OrthonormalBase_i)_{i \in k + 1}} \SetMin \lbrace 0 \rbrace$ abbiamo
\begin{align*}
	\RayleighQuotient(\VarVector)
	&= \frac{\HermitianTransposed{\VarVector}\HermitianMatrix \VarVector}{\HermitianTransposed{\VarVector}\VarVector}, \\
	&= \frac{\HermitianTransposed{\VarVector}\UnitaryMatrix^{-1}\HermitianMatrix\UnitaryMatrix \VarVector}{\HermitianTransposed{\VarVector}\VarVector}, \\
	&= \frac{\sum_{i \in n} \Eigenvalue_i \Conjugate{\VarVector_i}\VarVector_i}{\Conjugate{\VarVector_i}\VarVector_i},
	&= \frac{\sum_{i = 0}^k \Eigenvalue_i \Conjugate{\VarVector_i}\VarVector_i}{\Conjugate{\VarVector_i}\VarVector_i},
	&\geq \frac{\sum_{i = k}^n \Eigenvalue_k \Conjugate{\VarVector_i}\VarVector_i}{\Conjugate{\VarVector_i}\VarVector_i} = \Eigenvalue_k.
\end{align*}
\par Quindi deve essere $\max_{\LinearSpace \in S_k} \min_{\substack{x \in \LinearSpace\\x \neq 0}} \RayleighQuotient(x) = \Eigenvalue_k$: ci\`o prova la prima affermazione. La seconda affermazione segue direttamente dalla prima moltiplicando $\HermitianMatrix$ per $-1$. \EndProof
\begin{Corollary}
	Con le notazioni del teorema precedente, supponiamo inoltre
	\begin{itemize}
		\item $M \in \mathbb{C}^{n \times (n - 1)}$ tale che $\HermitianTransposed{M} M = \Identity$;
		\item $\VarHermitianMatrix = \HermitianTransposed{M}\HermitianMatrix M$;
		\item $(\VarEigenvalue_i)_{i \in \NaturalShift{n}} \in \Spectrum{\VarHermitianMatrix}^n$ tale che
		\begin{itemize}
			\item $(\VarEigenvalue_i)_{i \in \NaturalShift{n}}$ sia una successione non crescente;
			\item ogni autovalore di $\VarHermitianMatrix$ compaia in $(\VarEigenvalue_i)_{i \in n}$ un numero di volte pari alla sua molteplicit\`a algebrica.
		\end{itemize}
	\end{itemize}
	Allora
	\begin{itemize}
		\item $\VarHermitianMatrix$ \`e hermitiana;
		\item gli autovalori di $A$ separano gli autovalori di $\HermitianMatrix$: $\Eigenvalue_1 \geq \VarEigenvalue_1 \geq ... \geq \VarEigenvalue_{n - 1} \geq \Eigenvalue_n$.
	\end{itemize}
\end{Corollary}
\Proof L'hermitianit\`a di $\VarHermitianMatrix$ segue da $\HermitianTransposed{\HermitianTransposed{M} \HermitianMatrix M} = \HermitianTransposed{M} \HermitianTransposed{\HermitianMatrix} \HermitianTransposed{\HermitianTransposed{M}} = \HermitianTransposed{M} \HermitianMatrix M$.
\par Siano ora
\begin{itemize}
	\item $k \in \NaturalShift{n - 1}$;
	\item $\RayleighQuotient_\HermitianMatrix(x)$ il quoziente di Rayleigh rispetto a $\HermitianMatrix$;
	\item $\RayleighQuotient_\VarHermitianMatrix(x)$ il quoziente di Rayleigh rispetto a $\VarHermitianMatrix$;
	\item per ogni $\LinearSpace \in S_k$, $\LinearSpace' = \lbrace x \in \mathbb{C}^n | \Exists{y \in \LinearSpace}{x = My} \rbrace$.
\end{itemize}
\par Per il teorema di Binet, $\Determinant{M} \neq 0$, dunque $M$ \`e invertibile e $\ForAll{\LinearSpace \in S_k}{\LinearSpace' \in S_k}$.
\par Abbiamo
\begin{align*}
	\VarEigenvalue_k
	&= \max_{\LinearSpace \in S_k} \min_{\substack{x \in \LinearSpace\\x \neq 0}} \RayleighQuotient_\VarHermitianMatrix(x),\\
	&= \max_{\LinearSpace \in S_k} \min_{\substack{x \in \LinearSpace\\x \neq 0}} \frac{\HermitianTransposed{x} \VarHermitianMatrix x}{\HermitianTransposed{x} x},\\
	&= \max_{\LinearSpace \in S_k} \min_{\substack{x \in \LinearSpace\\x \neq 0}} \frac{\HermitianTransposed{x} \HermitianTransposed{M} \HermitianMatrix M x}{\HermitianTransposed{x} \HermitianTransposed{M}M x},\\
	&= \max_{\LinearSpace \in S_k} \min_{\substack{x \in \LinearSpace\\x \neq 0}} \frac{\HermitianTransposed{(Mx)} \HermitianMatrix (Mx)}{\HermitianTransposed{(Mx)}(Mx)},\\
	&= \max_{\LinearSpace \in S_k} \min_{\substack{x \in \LinearSpace\\x \neq 0}} \RayleighQuotient_\HermitianMatrix(Mx),\\
	&= \max_{\LinearSpace \in S_k} \min_{\substack{x \in \LinearSpace'\\x \neq 0}} \RayleighQuotient_\HermitianMatrix(x),\\
	&\leq \max_{\LinearSpace \in S_k} \min_{\substack{x \in \LinearSpace\\x \neq 0}} \RayleighQuotient_\HermitianMatrix(x) = \Eigenvalue_k.
\end{align*}
\par Il resto del teorema segue da quanto appena dimostrato applicato agli autovalori delle matrici $- \HermitianMatrix$ e $- \VarHermitianMatrix$ in luogo di $\HermitianMatrix$ e $\VarHermitianMatrix$ rispettivamente. \EndProof
\begin{Corollary}
	Sia $\HermitianMatrix \in \mathbb{C}^{n \times n}$ hermitiana. Gli autovalori di ogni matrice principale $(n - 1) \times (n - 1)$ di $\HermitianMatrix$ separano gli autovalori di $\HermitianMatrix$.
\end{Corollary}
\Proof Consideriamo la matrice $M \in \mathbb{C}^{n \times (n - 1)}$ tale che $M_i^j = \begin{cases}1\text{ se }i = j,\\0\text{ altrimenti.}\end{cases}$. Abbiamo
\[
	M = \lmatrix
	\begin{array}{ccc}
	1 & &\\
	& \ddots &\\
	& & 1\\
	0 & \cdots & 0
	\end{array}
	\rmatrix.
\]
\par Sia $A$ una sottomatrice principale di $\HermitianMatrix$ e sia $\PermutationMatrix \in \mathbb{C}^{n \times n}$ una matrice di permutazione tale che $\PermutationMatrix^{-1} \HermitianMatrix \PermutationMatrix$ ha $A$ come sottomatrice principale di testa. Abbiamo dunque $A = \HermitianTransposed{M} \PermutationMatrix^{-1} \HermitianMatrix \PermutationMatrix M = \HermitianTransposed{(\PermutationMatrix M)} \HermitianMatrix (\PermutationMatrix M)$. Tenendo conto del fatto che $\HermitianTransposed{(\PermutationMatrix M)} (\PermutationMatrix M) = \Identity$, segue immediatamente dal corollario precedente che gli autovalori di $A$ separano gli autovalori di $\HermitianMatrix$. \EndProof
\begin{Theorem}
	\TheoremName{Propriet\`a degli zeri dei polinomi ortogonali.} Siano $(\OrthogonalPolynomial_n)_{n \in \\mathbb{N}} \in \RingAdjunction{\mathbb{C}}{x}^\mathbb{N}$ polinomi ortogonali rispetto ad un prodotto hermitiano $\HermitianProduct{\cdot}{\cdot}$ su $\RingAdjunction{\mathbb{C}}{x}$. Per ogni $n \in \NotZero{\mathbb{N}}$, gli zeri di $\OrthogonalPolynomial_n$ separano gli zeri di $\OrthogonalPolynomial_{n + 1}$.
\end{Theorem}
\Proof Fissiamo $n \in \NotZero{\mathbb{N}}$. Per prima cosa, osserviamo che nessuna radice di $\OrthogonalPolynomial_n$ \`e radice anche di $\OrthogonalPolynomial_{n + 1}$, infatti se $\xi$ fosse radice di entrambi i polinomi, allora, per la relazione a tre termini, sarebbe anche radice di $\OrthogonalPolynomial_{n - 1}$ e dunque, induttivamente, di $\OrthogonalPolynomial_0$, ma questo sarebbe assurdo.
\par Il teorema segue dunque direttamente dal corollario precedente osservando che la matrice tridiagonale del teorema \ref{IstituzioniDiAnalisiNumerica_RadiciEAutovalori} d'ordine $n$ e la sottomatrice principale di testa della matrice analoga d'ordine $n + 1$. \EndProof
\begin{Theorem}
	\TheoremName{Rappresentazione dei polinomi ortogonali mediante matrice dei momenti}[dei polinomi ortogonali mediante matrice dei momementi][rappresentazione]
	Sia fissato un prodotto hermitiano su $\RingAdjunction{\mathbb{C}}{x}$ della forma indicata dal teorema \ref{IstituzioniDiAnalisiNumerica_ProdottoHermitianoIntegrale} per un'opportuna funzione peso $\WeightFunction$. Per ogni $n \in \mathbb{N}$, siano
	\begin{itemize}
		\item $\Moment_n$ il momento di ordine $n$ di $\WeightFunction$;
		\item $M_n \in \RingAdjunction{\mathbb{C}}{x}^{(n + 1) \times (n + 1)}$ la matrice definita da
		\[
			(M_n)_i^j =
			\begin{cases}
				\Moment_{i + j}\text{ se }j \neq n,\\ 
				x^j\text{ altrimenti;}
			\end{cases}
		\]
		vale a dire
		\[
			M_n = 
			\lmatrix
			\begin{array}{ccccc}
				\Moment_0 & \Moment_1 & \cdots & \Moment_{n - 1} & \Moment_n\\
				\Moment_1 & \Moment_2 & \cdots & \Moment_n & \Moment_{n+1}\\
				\vdots & \vdots & \ddots & \vdots & \vdots\\
				\Moment_{n - 2} & \Moment_{n - 1} & \cdots & \Moment_{2n - 3} & \Moment_{2n-2}\\
				\Moment_{n - 1} & \Moment_n & \cdots & \Moment_{2n - 2} & \Moment_{2n-1}\\
				1 & x & \cdots & x^{n - 1} & x^n
			\end{array}
			\rmatrix.
		\]
	\end{itemize}
	I polinomi $(\OrthogonalPolynomial_n)_{n \in \mathbb{N}}$, definiti per ogni $n \in \mathbb{N}$ da $\OrthogonalPolynomial_n(x) = \Determinant{M_n(x)}$ sono ortogonali rispetto al prodotto hermitiano fissato $\HermitianProduct{\cdot}{\cdot}$.
\end{Theorem}
\Proof Per ogni $n \in \mathbb{N}$, si osserva immediatamente dallo sviluppo di Laplace del determinante di $M_n$ lungo l'ultima riga che $\PolynomialDegree{\OrthogonalPolynomial_n} = n$.
\par Siano $n \in \mathbb{N}$ e $k \in n$. Abbiamo $\HermitianProduct{\OrthogonalPolynomial_n}{x^k} = \int_a^b \WeightFunction(x) \OrthogonalPolynomial_n(x) x^k dx$, poich\'e l'integrale \`e calcolato su un intervallo reale. Dunque $\HermitianProduct{\OrthogonalPolynomial_n}{x^k} = \int_a^b \WeightFunction(x) \Determinant{(M_n(x))} x^k dx$. Poich\'e il determinante \`e una forma multilineare alternante sulle righe di matrici, abbiamo $\HermitianProduct{\OrthogonalPolynomial_n}{x^k} = \int_a^b \Determinant{(\bar{M}_n(x))} dx$, dove $\bar{M}_n(x) \in \RingAdjunction{\mathbb{C}}{x}^{(n + 1) \times (n + 1)}$ \`e la matrice ottenuta da $M(x)$ moltiplicandone l'ultima riga per $\WeightFunction(x) x^k$, vale a dire
\[
	\bar{M}_n(x) =
	\lmatrix
	\begin{array}{ccccc}
		\Moment_0 & \Moment_1 & \cdots & \Moment_{n - 1} & \Moment_n\\
		\Moment_1 & \Moment_2 & \cdots & \Moment_n & \Moment_{n+1}\\
		\vdots & \vdots & \ddots & \vdots & \vdots\\
		\Moment_{n - 2} & \Moment_{n - 1} & \cdots & \Moment_{2n - 3} & \Moment_{2n-2}\\
		\Moment_{n - 1} & \Moment_n & \cdots & \Moment_{2n - 2} & \Moment_{2n-1}\\
		\WeightFunction(x) x^k & \WeightFunction(x) x^{k + 1} & \cdots & \WeightFunction(x) x^{k + n - 1} & \WeightFunction{x^{k + n}}
	\end{array}
	\rmatrix.
\]
\par Ora, considerando lo sviluppo del determinante di $\bar{M}_n(x)$ secondo l'ultima riga e sfruttando la linearit\`a dell'integrale si osserva che $\int_a^b \Determinant{(\bar{M}_n(x))} dx = \Determinant{ \widetilde{M}_n(x) }$, dove $\widetilde{M}_n(x) \in \RingAdjunction{\mathbb{C}}{x}^{(n + 1) \times (n + 1)}$ \`e la matrice ottenuta da $\bar{M}_n(x)$ integrando termine a termine l'ultima riga, vale a dire
\[
	\widetilde{M}_n(x) =
	\lmatrix
	\begin{array}{ccccc}
		\Moment_0 & \Moment_1 & \cdots & \Moment_{n - 1} & \Moment_n\\
		\Moment_1 & \Moment_2 & \cdots & \Moment_n & \Moment_{n+1}\\
		\vdots & \vdots & \ddots & \vdots & \vdots\\
		\Moment_{n - 2} & \Moment_{n - 1} & \cdots & \Moment_{2n - 3} & \Moment_{2n-2}\\
		\Moment_{n - 1} & \Moment_n & \cdots & \Moment_{2n - 2} & \Moment_{2n-1}\\
		\int_a^b \WeightFunction(x) x^k dx & \int_a^b \WeightFunction(x) x^{k + 1} dx & \cdots & \int_a^b \WeightFunction(x) x^{k + n - 1} dx & \int_a^b \WeightFunction{x^{k + n}} dx
	\end{array}
	\rmatrix.
\]
\par Ma l'ultima riga di $\widetilde{M}_n(x)$, per definizione di momento, coincide con la $(k + 1)$-esima. Dunque $\HermitianProduct{\OrthogonalPolynomial_n}{x^k} = 0$, da cui, per ogni $m \in \mathbb{N}$, $\Implies{m \neq n}{\HermitianProduct{\OrthogonalPolynomial_n}{\OrthogonalPolynomial_m} = 0}$. \EndProof

\begin{Theorem}
	Sia $u = (u_i)_{i = 1...n}$ e $v = (v_i)_{i = 1...n}$. Abbiamo $\Transposed{u}H_nv = \sum_i\sum_j u_i\mu_{i + j - 2}v_j = \ScalarProduct{\sum_i u_i x^{i - 1}}{\sum_j v_j x^{j - 1}}$.
	$\Transposed{u}H_n u = \ScalarProduct{.}{.} > 0$ se $u \neq 0$ quindi $H_n$ \`e definita positiva.
	In particolare $\Transposed{u}H_nv$ \`e prodotto scalre su $\mathbb{R}^n$.
\end{Theorem}
\begin{Theorem}
	Sia $S(x) \in \CClass{n}[[a,b]]$ tale che $S^{(k)}(a) = S^{(k)}(b) = 0$ per $k = 0, ..., n - 1$. Sia $t(x) = \frac{S^{(n)}(x)}{\WeightFunction(x)}$. Allora $\int_a^b t(x)q(x)\WeightFunction(x) = 0$ per ogni polinomio $q$ di grado al pi\`u $n - 1$.
\end{Theorem}
\Proof vedi dispense. si usa integrazione per parti e si usa il fatto che $q^{(n)} = 0$.
\begin{Theorem}
	\TheoremName{Formula di Rodrigues} se $p_n(x)$ \`e un polinomio di grado $n$ tale $p_n(x) = \frac{\beta_n}{\WeightFunction(x)} \frac{d^n S_n(x)}{dx^n}$, dove $\beta_n \in \mathbb{R}$, $S_n(x) \in \CClass{n}[[a,b]]$ e $S_n^{(k)}(a) = ... (b) = 0$ per $k=0..n-1$, allora $p_0,...,p_n$ sono ortogonali.
\end{Theorem}

\subsection{Polinomi di \Chebyshev.}
\label{IstituzioniDiAnalisiNumerici_AltriPolinomiOrtogonalSpecifici}
\begin{Definition}
	Chiamiamo \Define{polinomi di \Chebyshev\ di prima specie}[di \Chebyshev\ di prima specie][polinomi] i polinomi $(\FirstChebyshevPolynomial_n)_{n \in \mathbb{N}}$ definiti da
	\[
		\FirstChebyshevPolynomial_{n + 1}(x) =
		\begin{cases}
			\FirstChebyshevPolynomial_0(x) = 1\text{ se }n = - 1,\\
			\FirstChebyshevPolynomial_1(x) = x\text{ se }n = 0,\\
			2x\FirstChebyshevPolynomial_n(x) - \FirstChebyshevPolynomial_{n - 1}(x)\text{ altrimenti}.
		\end{cases}
	\]
\end{Definition}
\begin{Theorem}
	I polinomi di \Chebyshev\ di prima specie $(\FirstChebyshevPolynomial_n)_{n \in \mathbb{N}}$ verificano la relazione
\[
	\ForAll{(n,\theta) \in \mathbb{N} \times \mathbb{R}}{\FirstChebyshevPolynomial_n(\cos{\theta}) = \cos{n \theta}}.
\]
\end{Theorem}
\Proof Procediamo per induzione su $i$.
\par Nei casi $n = 0$ e $n = 1$ la tesi \`e immediata: dimostriamola per $n > 1$. Abbiamo. per qualunque $\theta \in \mathbb{R}$,
\begin{align*}
	\FirstChebyshevPolynomial_{n + 1}(\cos{\theta})
	&= 2 \cos{\theta} \FirstChebyshevPolynomial_n(\cos{\theta}) - \FirstChebyshevPolynomial_{n - 1}(\cos{\theta}),\\
	&= 2 \cos{\theta} \cos{n \theta} - \cos{(n - 1) \theta},\\
	&= 2 \cos{\theta} \cos{n \theta} - \cos{n\theta} \cos{\theta} - \sin{n\theta} \sin{\theta},\\
	&= \cos{\theta} \cos{n \theta} - \sin{n\theta} \sin{\theta},\\
	&= \cos{(n + 1) \theta}.\text{ \EndProof}
\end{align*}
\begin{Corollary}
	Definiamo su $\RingAdjunction{\mathbb{C}}{x}$ il prodotto hermitiano $\HermitianProduct{\Polynomial}{\VarPolynomial} = \int_{-1}^1 (1 - x^2)^{-\frac{1}{2}} \Polynomial(x)\Conjugate{\VarPolynomial(x)}dx$. I polimoni di \Chebyshev\ di prima specie $(\FirstChebyshevPolynomial_i)_{n \in \mathbb{N}}$ verificano la relazione
	\[
		\HermitianProduct{\FirstChebyshevPolynomial_n}{\FirstChebyshevPolynomial_m} =
		\begin{cases}
			0\text{ se }n \neq m,\\
			\pi\text{ se }n = m = 0,\\
			\frac{\pi}{2}\text{ se }n = m \neq 0.
		\end{cases}
	\]
	In particolare i polimoni di \Chebyshev\ di prima specie sono i polinomi ortogonali rispetto al prodotto hermitiano $\HermitianProduct{\cdot}{\cdot}$.
\end{Corollary}
\Proof Abbiamo, per ogni $(n,m) \in \mathbb{N}^2$ e sfruttando il cambio di variabile $x = \cos{\theta}$ per il calcolo dell'integrale,
\begin{align*}
	\HermitianProduct{\FirstChebyshevPolynomial_n}{\FirstChebyshevPolynomial_m}
	&= \int_{-1}^1 (1 - x^2)^{-\frac{1}{2}}\FirstChebyshevPolynomial_n(x)\Conjugate{\FirstChebyshevPolynomial_m(x)} dx,\\
	&= \int_\pi^0 (1 - (\cos{\theta})^2)^{-\frac{1}{2}} \cos{n \theta} \cos{m \theta} (- \sin{\theta}) d\theta,\\
	&= \int_0^\pi \cos{(n \theta)} \cos{(m \theta)} d\theta. 
\end{align*}
\par Abbiamo poi
\begin{itemize}
	\item $\HermitianProduct{\FirstChebyshevPolynomial_0}{\FirstChebyshevPolynomial_0} = \int_0^\pi d\theta = \pi$;
	\item supposto $n = m \neq 0$, $\HermitianProduct{\FirstChebyshevPolynomial_n}{\FirstChebyshevPolynomial_m} = \int_0^\pi (\cos{n \theta})^2 d\theta = \int_0^\pi \frac{\cos{2 n \theta} + 1}{2} d\theta = \left [\frac{\sin{2 n\theta}}{4i} + \frac{\theta}{2} \right ]_0^\pi = \frac{\pi}{2}$;
	\item supposto $n \neq m$, $\HermitianProduct{\FirstChebyshevPolynomial_n}{\FirstChebyshevPolynomial_m} = \int_0^\pi \cos{(n\theta)} \cos{(m\theta)} d\theta = \int_0^\pi \frac{\cos{(n + m)\theta} + \cos{(n - m)\theta}}{2} d\theta = \frac{1}{2} \left [ \frac{\sin{(n + m)\theta}}{n + m} + \frac{\sin{(n - m)\theta}}{n - m} \right ]_0^\pi = 0$. \EndProof
\end{itemize}
\begin{Corollary}
	Dati i polinomi di \Chebyshev\ di prima specie $(\FirstChebyshevPolynomial_n)_{n \in \mathbb{N}}$, abbiamo, per ogni $n \in \mathbb{N}$
	\begin{itemize}
		\item gli zeri di $\FirstChebyshevPolynomial_n$ sono $\left ( \cos{\frac{(2 k - 1)\pi}{2n}} \right )_{k \in n}$;
		\item il massimo di $\FirstChebyshevPolynomial_n$ su $[-1;1]$ \`e $1$ e viene assunto nei punti $\left ( \cos \frac{2k\pi}{n} \right )_{k \in \lfloor \frac{n}{2} \rfloor}$;
		\item il minimo di $\FirstChebyshevPolynomial_n$ su $[-1;1]$ \`e $- 1$ e viene assunto nei punti $\left ( \cos \frac{(2k + 1)\pi}{n} \right )_{k \in \lfloor \frac{n}{2} \rfloor}$;
		\item $\FirstChebyshevPolynomial_n$ ha $n + 1$ estremi locali su $[-1;1]$, alternati tra massimi e minimi.
	\end{itemize}
\end{Corollary}
\Proof Fissiamo $n \in \mathbb{N}$.
\par Cerchiamo gli zeri di $\FirstChebyshevPolynomial_n$ nell'intervallo $[-1;1]$: essi possono essere messi nella forma $\cos{\theta}$ per opportuno $\theta \in \mathbb{R}$. Abbiamo che $\FirstChebyshevPolynomial_n(\cos(\theta)) = \cos(n \theta)$ \`e nullo se e solo se $\Exists{k \in \mathbb{Z}}{n \theta = \frac{\pi}{2} + k\pi}$, equivalente a $\Exists{k \in \mathbb{Z}}{\theta = \frac{(2k + 1)\pi}{2n}}$. Quindi gli zeri in $[-1;1]$ di $\FirstChebyshevPolynomial_n$ sono $\cos \frac{(2k + 1)\pi}{2n}$ con $k \in n$: essi sono in numero $n$ e dunque sono tutte le radici del polinomio $\FirstChebyshevPolynomial_n$.
\par Cerchiamo i punti di massimo e minimo di $\FirstChebyshevPolynomial_n$ nell'intervallo $[-1;1]$: essi possono essere messi nella forma $\cos{\theta}$ per opportuno $\theta \in \mathbb{R}$. Abbiamo
\[
	\FirstChebyshevPolynomial_n(\cos(\theta)) = \cos(n \theta) =
	\begin{cases}
		1\text{ se e solo se }\Exists{k \in \mathbb{Z}}{\theta = \frac{2k\pi}{n}},\\
		- 1\text{ se e solo se }\Exists{k \in \mathbb{Z}}{\theta = \frac{(2k + 1)\pi}{n}}.
	\end{cases}
\]
\par Quindi i massimi di $\FirstChebyshevPolynomial_n$ sono $\cos \frac{2k\pi}{n}$ con $k \in \lfloor \frac{n}{2} \rfloor$ e i minimi $\cos \frac{2(k + 1)\pi}{n}$ con $k \in \lfloor \frac{n - 1}{2} \rfloor$.
\par Infine,
\begin{itemize}
	\item se $n$ \`e pari, poniamo $a = \frac{n}{2}$: abbiamo $\left \lfloor \frac{n}{2} \right \rfloor + 1 + \left \lfloor \frac{n - 1}{2} \right \rfloor  + 1 = a + 1 + \left \lfloor a - \frac{1}{2} \right \rfloor + 1 = a + 1 + a - 1 + 1= 2a + 1 = n + 1$.
	\item se $n$ \`e dispari, poniamo $a = \frac{n - 1}{2}$: abbiamo $\left \lfloor \frac{n}{2} \right \rfloor + 1 + \left \lfloor \frac{n - 1}{2} \right \rfloor  + 1 = \left \lfloor \frac{n - 1 + 1}{2} \right \rfloor + 1 + a + 1 = \left \lfloor a + \frac{1}{2} \right \rfloor + 1 + a + 1 = a + 1 + a + 1 = 2a + 2 = n - 1 + 2 = n + 1$. \EndProof 
\end{itemize}
\begin{Theorem}
	Sia $n \in \NotZero{\mathbb{N}}$ e sia $\FirstChebyshevPolynomial_n$ il polinomio di \Chebyshev di prima specie di grado $n$. Allora,
	\begin{itemize}
		\item $\frac{\FirstChebyshevPolynomial_n}{2^{n - 1}}$ \`e monico;
		\item se $\Polynomial \in \RingAdjunction{C}{x}$ \`e di grado $n$ e monico, allora, considerando $\FirstChebyshevPolynomial_n$ e $\Polynomial$ sul dominio $[-1;1]$, abbiamo $\Norm{\frac{\FirstChebyshevPolynomial_n}{2^{n - 1}}}[\infty] \leq \Norm{\Polynomial}[\infty]$.
	\end{itemize}
\end{Theorem}
\Proof Se $n = 1$, allora $\frac{\FirstChebyshevPolynomial_n}{2^{n - 1}}(x) = \frac{x}{2^{1 - 1}} = x$.
\par Supponiamo ora
\[
	\ForAll{m \in n}{\LeadingCoefficient{\frac{\FirstChebyshevPolynomial_m}{2^{m - 1}}} = 1}.
\]
\par Abbiamo $\LeadingCoefficient{\FirstChebyshevPolynomial_n(x)} = \LeadingCoefficient{2x\FirstChebyshevPolynomial_{n - 1} - \FirstChebyshevPolynomial_{n - 2}} = \LeadingCoefficient{2x\FirstChebyshevPolynomial_{n - 1}} = 2\LeadingCoefficient{\FirstChebyshevPolynomial_{n - 1}}$. Da cui $\LeadingCoefficient{\frac{\FirstChebyshevPolynomial_n}{2^{n - 1}}} = \frac{2 \LeadingCoefficient{\FirstChebyshevPolynomial_{n - 1}}}{2^{n - 1}} = 1$.
\par Quindi, per induzione, $\frac{\FirstChebyshevPolynomial_n}{2^{n - 1}}$ \`e monico.
\par Ora, $\Norm{\frac{\FirstChebyshevPolynomial_n}{2^{n - 1}}}[\infty] = \frac{\Norm{\FirstChebyshevPolynomial_n}[\infty]}{2^{n - 1}} = 2^{1 - n}$.
\par Supponiamo $\Polynomial \in \RingAdjunction{C}{x}$ di grado $n$ e monico tale che $\Norm{\Polynomial}[\infty] \leq \Norm{\frac{\FirstChebyshevPolynomial_n}{2^{n - 1}}}[\infty] = 2^{1 - n}$.
\par Poniamo $\VarPolynomial = \Polynomial - \frac{\FirstChebyshevPolynomial_n}{2^{n - 1}}$. Abbiamo $\PolynomialDegree{\VarPolynomial} \leq n - 1$, quindi o $\VarPolynomial = 0$ o $\PolynomialDegree$ ha al pi\`u $n - 1$ radici distinte.
\par D'altra parte,
\begin{itemize}
	\item sia $M \in [-1;1]$ massimo di $\FirstChebyshevPolynomial_n$: abbiamo $\VarPolynomial(M) = \Polynomial(M) - \frac{\FirstChebyshevPolynomial_n(M)}{2^{n - 1}} = \Polynomial(M) - 2^{1 - n} < 0$;
	\item sia $m \in [-1;1]$ massimo di $\FirstChebyshevPolynomial_n$: abbiamo $\VarPolynomial(m) = \Polynomial(m) - \frac{\FirstChebyshevPolynomial_n(m)}{2^{n - 1}} = \Polynomial(m) + 2^{1 - n} > 0$.
\end{itemize}
\par Per la continuit\`a di $\VarPolynomial$ e poich\'e $\VarPolynomial$ cambia segno $n$ volte, $\VarPolynomial$ deve avere almeno $n$ radici. Dunque, necessariamente, $\VarPolynomial = 0$, a cui $\Polynomial = \frac{\FirstChebyshevPolynomial_n}{2^{n - 1}}$. \EndProof
\begin{Definition}
	Chiamiamo \Define{polinomi di \Chebyshev\ di seconda specie}[di \Chebyshev\ di seconda specie][polinomi] gli unici polinomi $(\SecondChebyshevPolynomial_n)_{n \in \mathbb{N}}$ tali che
	\begin{itemize}
		\item essi siano ortogonali rispetto al prodotto scalare $\ScalarProduct{\Polynomial}{\VarPolynomial} = \int_{-1}^1 (1 - x^2)^{\frac{1}{2}} \Polynomial(x)\VarPolynomial(x)dx$;
		\item $\SecondChebyshevPolynomial_0 = 1$;
		\item $\SecondChebyshevPolynomial_1 = 2x$.
	\end{itemize}
\end{Definition}

\subsection{Altri polinomi ortogonali specifici.}
\label{IstituzioniDiAnalisiNumerici_AltriPolinomiOrtogonalSpecifici}
\begin{Definition}
	Chiamiamo \Define{polinomi di Legendre}[di Legendre][polinomi] gli unici polinomi $(\LegendrePolynomial_i)_{i \in \mathbb{N}}$ tali che
	\begin{itemize}
		\item essi siano ortogonali rispetto al prodotto scalare $\ScalarProduct{\Polynomial}{\VarPolynomial} = \int_{-1}^1 \Polynomial(x)\VarPolynomial(x)dx$;
		\item $\LegendrePolynomial_0 = 1$;
		\item $\LegendrePolynomial_1 = x$.
	\end{itemize}
\end{Definition}
\begin{Theorem}
	Per i polinomi di Legendre $(\LegendrePolynomial_i)_{i \in \mathbb{N}}$ vale la relazione a tre termini
	\[
		\LegendrePolynomial_{i + 1}(x) = 
		\begin{cases}
			\LegendrePolynomial_0(x) = 1\text{ se }i = -1,\\
			\LegendrePolynomial_1(x) = x\text{ se }i = 0,\\
			\frac{2i + 1}{i + 1} x \LegendrePolynomial_i(x) - \frac{i}{i + 1}\LegendrePolynomial_{i - 1}(x)\text{ altrimenti}.
		\end{cases}
	\]
\end{Theorem}
\begin{Definition}
	Chiamiamo \Define{polinomi di Laguerre}[di Laguerre][polinomi] gli unici polinomi $(\LaguerrePolynomial_i)_{i \in \mathbb{N}}$ tali che
	\begin{itemize}
		\item essi siano ortogonali rispetto al prodotto scalare $\ScalarProduct{\Polynomial}{\VarPolynomial} = \int_0^{+ \infty} e^{-x} \Polynomial(x)\VarPolynomial(x)dx$;
		\item $\LaguerrePolynomial_0 = 1$;
		\item $\LaguerrePolynomial_1 = - x + 1$.
	\end{itemize}
\end{Definition}
\begin{Definition}
	Chiamiamo \Define{polinomi di Hermite probabilistici}[di Hermite probabilistici][polinomi] gli unici polinomi $(\ProbabilistHermitePolynomial_i)_{i \in \mathbb{N}}$ tali che
	\begin{itemize}
		\item essi siano ortogonali rispetto al prodotto scalare $\ScalarProduct{\Polynomial}{\VarPolynomial} = \int_{-infty}^{+\infty} e^{-x^2} \Polynomial(x)\VarPolynomial(x)dx$;
		\item $\ProbabilistHermitePolynomial_0 = 1$;
		\item $\ProbabilistHermitePolynomial_1 = x$.
	\end{itemize}
\end{Definition}
\begin{Definition}
	Chiamiamo \Define{polinomi di Hermite fisici}[di Hermite fisici][polinomi] gli unici polinomi $(\PhysicistHermitePolynomial_i)_{i \in \mathbb{N}}$ tali che
	\begin{itemize}
		\item essi siano ortogonali rispetto al prodotto scalare $\ScalarProduct{\Polynomial}{\VarPolynomial} = \int_{-infty}^{+\infty} e^{-x^2} \Polynomial(x)\VarPolynomial(x)dx$;
		\item $\PhysicistHermitePolynomial_0 = 1$;
		\item $\PhysicistHermitePolynomial_1 = 2x$.
	\end{itemize}
\end{Definition}



	\chapter{Algoritmi e strutture dati}
	\section{Definizioni e teoremi di base.}
\label{AlgoritmiEStruttureDiDati_DefinizioniETeoremiDiBase}
\begin{Definition}
	Chiamiamo \Define{algoritmo} una successione di operazioni svolte a partire da un insieme di dati inizali, detti \Define{dati in ingresso}[in ingresso][dati] o \Define{\English{input}}, che termina e, terminando, restituisce un insieme di dati finali, detti \Define{dati in uscita}[in uscita][dati] o \Define{\English{output}}.
\end{Definition}
\begin{Definition}
	Un \Define{problema} \`e definito da
	\begin{itemize}
		\item un insieme di dati iniziali, detti \Define{dati in ingresso}[in ingresso][dati] o \Define{\English{input}};
		\item una regola di trasformazione dei dati iniziali in dati finali, detti \Define{dati in uscita}[in uscita][dati] o \Define{\English{output}}.
	\end{itemize}
\end{Definition}
\begin{Definition}
	Dato un problema, un algoritmo si dice \Define{corretto} per quel problema se applicato ai dati di ingresso del problema restituisce i dati in uscita del problema. Diciamo anche che l'algoritmo \Define{risolve}[di un problema][risoluzione] il problema.
\end{Definition}
\begin{Definition}
	Nella ricerca di algoritmo che risolva un problema, chiamiamo \Define{baseline} la prima versione di un algoritmo individuata che risolve il problema, versione che contiamo migliorare successivamente.
\end{Definition}

\section{Complessit\`a computazionle.}
\label{AlgoritmiEStruttureDiDati_ComplessitaComputazionale}
\begin{Definition}
	A seconda dei contesti, definiamo il \Define{costo}[di un algoritmo][costo] o \Define{complessit\`a}[di un algoritmo][complessit\`a] di un algoritmo in uno dei modi seguenti, ordinati dalla definizione pi\`u frequentemente usata alla meno frequente:
	\begin{itemize}
		\item il costo computazionale \`e il tempo di esecuzione dell'algoritmo;
		\item il costo computazionale \`e lo spazio di lavoro dell'algoritmo, cio\`e il numero di celle di memoria utilizzate (talvolta si escludono le celle usate per il passaggio dei parametri di ingresso);
		\item il costo computazionale \`e l'energia consumata dall'algoritmo.
	\end{itemize}
	Salvo avvisi contrari, utilizzeremo sempre la prima definizione.
\end{Definition}
\begin{Definition}
	Diciamo che un algoritmo ha costo $C$
	\begin{itemize}
		\item al \Define{caso pessimo}[pessimo][caso] quando esiste un insieme di
dati in ingresso per cui l'algoritmo ha costo computazionale $C$ e nessun altro
insieme di dati in ingresso genera un costo maggiore;
		\item al \Define{caso pessimo}[pessimo][caso] quando esiste un insieme di
dati in ingresso per cui l'algoritmo ha costo computazionale $C$ e nessun altro
insieme di dati in ingresso genera un costo minore;
		\item al \Define{caso medio}[medio][caso] quando $C$ \`e la media dei costi
generati da tutti i possibili insiemi di dati in ingresso.
	\end{itemize}
\end{Definition}
\par Salvo ove espressamente indicato il contrario, analizzaremo sempre i costi
degli algoritmi al caso pessimo.
\begin{Definition}
	Data una funzione $f$ della dimensione $n$ dei dati di ingresso di un problema, diciamo che $f$ ha \Define{complessit\`a}[di un problema][complessit\`a]
	\begin{itemize}
		\item $\BigO{f(n)}$ se ammette un algoritmo di complessit\`a $\BigO{f(n)}$: diciamo anche che il problema ha \Define{limite superiore}[superiore di un problema][limite];
		\item $\BigOmega{f(n)}$ se ammette un algoritmo di complessit\`a $\BigOmega{f(n)}$: diciamo anche che il problema ha \Define{limite inferiore}[inferiore di un problema][limite];
		\item $\BigTheta{f(n)}$ se ha simultaneamente complessit\`a $\BigO{f(n)}$ e $\BigOmega{f(n)}$.
	\end{itemize}
\end{Definition}
\begin{Definition}
	Data una funzione $f$ della dimensione $n$ dei dati di ingresso di un problema, diciamo che un algoritmo di complessit\`a $\BigO{f(n)}$ per un problema \`e di complessit\`a $\BigTheta{f(n)}$ \`e \Define{asintoticamente ottimo}[asintoticamente ottimo][algoritmo].
\end{Definition}
\begin{Definition}
	Chiamiamo \Define{analisi di complessit\`a}[di complessit\`a][analisi] la ricerca dei migliori algoritmi che risolvono un problema dato tra un insieme di algoritmi proposto, condotta attraverso i tre passi seguenti:
	\begin{itemize}
		\item definizione di un modello computazionale che definisca i costi delle varie operazioni elementari;
		\item valutare il costo degli algoritmi risolutivi individuati;
		\item implementare le soluzioni che hanno un costo minore e sperimentarle su dati reali.
	\end{itemize}
\end{Definition}
\begin{Definition}
	Chiamiamo \Define{analisi asintotica}[asintotica][analisi] di un algoritmo lo studio della sua complessit\`a al tendere ad infinito della dimensione dei dati in ingresso.
\end{Definition}


\section{Algoritmi}
\label{AlgoritmiEStruttureDiDati_Algoritmi}
\subsection{Modello RAM.}
\label{AlgoritmiEStruttureDiDati_ModelloRAM}
\begin{Definition}
	Chiamiamo \Define{modello RAM}[RAM][modello] (\textit{Random access memory}) un modello computazionale costituito da
	\begin{itemize}
		\item registri;
		\item memoria;
		\item operazioni logiche, aritmetiche, booleane, di confronto, di trasferimento tra registri e memoria, di controllo.
	\end{itemize}
	Si suppone che tutte queste operazioni abbiano costo $\BigO{1}$.
\end{Definition}
\begin{Theorem}
	Il costo di un costrutto \mintinline{c}{if} con guardia non banale \`e, secondo il modello RAM, $G + \max\lbrace A,B \rbrace$, dove
	\begin{itemize}
		\item $G$ \`e il costo della guardia;
		\item $A$ \`e il costo del blocco di istruzioni corrispondenti alla guardia vera;
		\item $B$ il costo del blocco di istruzioni corrispondenti alla guardia falsa.
	\end{itemize}
\end{Theorem}
\Proof Il costo del costrutto \`e $G + A$ o $G + B$, a seconda dei dati effettivamente passati in ingresso. Al caso pessimo \`e $\max \lbrace G + A, G + B \rbrace = G + \max \lbrace A, B \rbrace$. \EndProof
\begin{Theorem}
	Il costo di un costrutto \mintinline{c}{for(int i = 0; i < n; i++)} \`e, secondo il modello RAM, $\sum_{i = 0}^{n - 1} t_i$, dove $t_i$ \`e il costo dell'$i$-esima iterazione del ciclo, comprensiva delle istruzioni di inizializzazione delle variabili di controllo (per la prima iterazione), di aggiornamento della variabile di controllo e dell'istruzione di valutazione della guardia.
\end{Theorem}
\Proof Segue direttamente dalle definizioni. \EndProof
\begin{Theorem}
	Il costo di un costrutto \mintinline{c}{while} o \mintinline{c}{do}-\mintinline{c}{while} \`e, secondo il modello RAM, $\sum_{i = 0}^m g_i + \sum_{i = 1}^m t_i$, dove $g_i$ ($i \in m + 1$) \`e il costo di esecuzione della guardia all'$i$-esima itereazione del ciclo e $t_i$ ($i \in \NaturalShift{m}$) \`e il costo dell'esecuzione del corpo del ciclo all'$i$-esima iterazione.
\end{Theorem}
\Proof Segue direttamente dalle definizioni osservando che all'ultima iterazione del ciclo viene eseguita la guardia, ma non il corpo. \EndProof

\subsection{Problemi decomponibili.}
\label{AlgoritmiEStruttureDiDati_ProblemiDecomponibili}
\begin{Definition}
	Chiamiamo \Define{decomponibile}[decomponibile][problema] un problema che pu\`o essere risolto induttivamente.
\end{Definition}
\begin{Theorem}
	Se un problema \`e decomponibile in $n$ passi e $C$ \`e il costo della rimcomposizione del problema, allora il costo dell'algoritmo induttivo \`e $\BigO{n C}$.
\end{Theorem}
\begin{listing}
	\insertcode{cpp}{"Algoritmi_e_strutture_di_dati/Decomponi.cpp"}
	\caption{Implementazione generica di un algoritmo per la risoluzione di un problema decomponibile in \LanguageName{C++}.}
\end{listing}

\subsection{Divide et impera.}
\label{AlgoritmiEStruttureDiDati_DivideEtImpera}
\begin{Theorem}
  \TheoremName{Teorema fondamentale delle ricorrenze}[fondamentale delle ricorrenze][teorema]
  Siano
  \begin{itemize}
    \item $f: \mathbb{N} \rightarrow \RealNonNegative$ una funzione non decresente;
    \item $\alpha \geq 1$;
    \item $\beta \geq 1$;
    \item $n_0, c_0 \geq 0$.
  \end{itemize}
  Sia $T: \mathbb{N} \rightarrow \RealNonNegative$ tale che
  \[
    T(n) \leq
    \begin{cases}
      c_0\text{ se }n \leq n_0,\\
      \alpha T \left ( \rho \left (\frac{n}{\beta} \right ) \right ) + f(n)\text{ altrimenti}.
    \end{cases}
  \]
  con
  $\rho \left ( \frac{n}{\beta} \right )
    = \left \lfloor \frac{n}{\beta} \right \rfloor$
  o
  $\rho \left ( \frac{n}{\beta} \right )
    = \left \lceil \frac{n}{\beta} \right \rceil$.
  \par Supponiamo che esista $\gamma > 0$ tale che per $n$ sufficientemente
  grande abbiamo
  $\alpha f \left ( \rho \left ( \frac{n}{\beta} \right ) \right )
    \leq \gamma f(n)$.
  Abbiamo
  \begin{itemize}
    \item $T(n) = \BigO{f(n)}$ se $\gamma < 1$;
    \item $T(n) = \BigO{f(n) \log_\beta n}$ se $\gamma = 1$;
    \item $T(n) = \BigO{n^{\log_\beta \alpha}}$ se $\gamma > 1$.
  \end{itemize}
\end{Theorem}
\begin{listing}
	\insertcode{cpp}{"Algoritmi_e_strutture_di_dati/DivideEtImpera.cpp"}
	\caption{Implementazione dell'algoritmo di ricerca binaria in \LanguageName{C++}.}
\end{listing}

\subsection{Algoritmi aleatori.}
\label{AlgoritmiEStruttureDiDati_AlgoritmiAleatori}
\begin{Definition}
  Un algoritmo si dice \Define{aleatorio}[aleatorio][algoritmo] se prevede
  l'uso di qualche variabile aleatoria.
\end{Definition}
\begin{Definition}
  Si chiama \Define{algoritmo di Fisher-Yates}[di Fisher-Yates][algoritmo]
  l'algoritmo che agisce su una successione finita
  $(e_i)_{i \in n}$ ($n \in \mathbb{N}$, $n \geq 2$)
  data in ingresso nel modo seguente:
  \begin{enumerate}
    \item\label{FisherYates_1} si ponga $I = 0$;
    \item\label{FisherYates_2} si ponga $J$ uguale a un numero naturale casuale
      in $[I;n - 1]$ con probabilit\`a uniforme;
    \item\label{FisherYates_3} si scambino $e_I$ e $e_J$;
    \item\label{FisherYates_5} si incrementi $I$ di $1$;
    \item\label{FisherYates_4} se $I < n - 1$ si
      torni al punto \ref{FisherYates_2}, altrimenti l'algoritmo termina.
  \end{enumerate}
\end{Definition}
\begin{Theorem}
  Con le notazioni della definizione precedente,
  per ogni
  $\Permutation \in \SymmetricGroup{n}$
  l'algoritmo di Fisher-Yates permuta la successione
  $(e_i)_{i \in n}$
  nella successione
  $(e_{\Permutation(i)})_{i \in n}$
  con probabilit\`a
  $n!$.
\end{Theorem}
\Proof Supponiamo $n = 2$. Allora la probabilit\`a che $e_0$ e $e_1$ vengano
scambiati \`e uguale alla probabilit\`a che $J = 1$, cio\`e $\frac{1}{2}$.
\par Fissiamo $N > 2$ e assumiamo il teorema dimostrato per $n = N - 1$.
\par Al primo passaggio da \ref{FisherYates_2} ogni possibile valore di $J$
viene scelto con probabilit\`a $\frac{1}{n}$; dunque, dopo lo scambio del punto
\ref{FisherYates_3}, ogni elemento della successione pu\`o diventare $e_0$ con
probabilit\`a $\frac{1}{n}$.
\par Ora, osserviamo che nel seguito dell'algoritmo $e_0$ non verr\`a pi\`u
toccato e che il resto delle operazione consiste esattamente nell'eseguire
l'algoritmo di Fisher-Yates sull'\English{input} $(e_i)_{i \in \NotZero{n}}$.
Per ipotesi induttiva, ogni permutazione degli elementi
$(e_i)_{i \in \NotZero{n}}$ avviene con probabilit\`a $\frac{1}{(n - 1)!}$.
Dunque ogni permutazione degli elementi della successione di partenza
$(e_i)_{i \in n}$ avviene con probabilit\`a
$\frac{1}{n} \cdot \frac{1}{(n - 1)!} = \frac{1}{n!}$. \EndProof

\subsection{Problema di ordinamento.}
\label{AlgoritmiEStruttureDiDati_ProblemaDiOrdinamento}
\begin{Definition}
	Chiamiamo \Define{problema di ordinamento}[di ordinamento][problema] il problema che consiste nel riordinare una successione finita rispetto ad un ordine totale dato.
\end{Definition}
\begin{Definition}
	Chiamiamo \Define{\English{comparison model}}[\English{comparison}][\English{model}] un modello computazionale per il problema di ordinamento tale che
	\begin{itemize}
		\item misuri la complessit\`a di un algoritmo in numero di confronti;
		\item ammetta come unica operazione tra due elementi dell'insieme da ordinare il confronto.
	\end{itemize}
\end{Definition}
\par Nel seguito denoteremo $(e_i)_{i \in n}$ ($n \in \mathbb{N}$) la
successione dei dati di ingresso del problema di ordinamento e $\leq$ l'ordine
totale rispetto al quale si desidera riordinare la successione. Inoltre
analizzeremo la complessit\`a degli algoritmi per il problema di ordinamento
secondo il \English{comparison model}.
\begin{Theorem}
	Il problema di ordinamento ha complessit\`a $\BigOmega{n \log n}$.
\end{Theorem}
\Proof Gli elementi dell'insieme da ordinare possono essere disposti in $n!$.
\par Effettuando $1$ confronto, l'algoritmo pu\`o discriminare tra al pi\`u $3$
configurazioni: denotati con $a$ e $b$ i due elementi confrontati, pu\`o
rilevare se $a < b$, se $a = b$ o infine se $a > b$. Assumiamo che con $k$
confronti l'algoritmo possa discriminare tra $3^k$ configurazioni: allora, con
$1$ confronto aggiuntivo, esso pu\`o discrimare tra $3^k \cdot 3 = 3^{k + 1}$
configurazioni. Dunque, segue per induzione che con $k$ confronti si pu\`o
discrimare tra $3^k$ configurazioni.
\par Perch\'e l'algoritmo possa fornire la risposta corretta al problema, \`e
dunque necessario che il numero $k$ di confronti da esso effettuato verifichi la
condizione $3^k \geq n!$, equivalente a $k \geq \log_3 n!$. La tesi segue dal
fatto che $\IsBigOmega{\log_3 n!}{n \log n}$.
\par Osserviamo che la base del logaritmo nell'enunciato non ha importanza
poich\'e il cambio di base si effettua moltiplicando per una costante. Inoltre,
se assumiamo tutti gli elementi dell'insieme da ordinare distinti, nella
dimostrazione possiamo sostituire ogni occorrenza del numero $3$ col numero $2$,
ma il limite inferiore dimostrato \`e lo stesso. \EndProof
\input{Algoritmi_e_strutture_di_dati/SelectionSort.tex}
\input{Algoritmi_e_strutture_di_dati/InsertionSort.tex}
\input{Algoritmi_e_strutture_di_dati/MergeSort.tex}
\input{Algoritmi_e_strutture_di_dati/QuickSort.tex}
\input{Algoritmi_e_strutture_di_dati/QuickSelect.tex}

\subsection{Problema di ricerca.}
\label{AlgoritmiEStruttureDiDati_ProblemaDiRicerca}
\begin{Definition}
	Chiamiamo \Define{problema di ricerca}[di ricerca][problema] il problema che
  consiste nel ricercare un elemento in una successione finita e, nel caso esso
  sia effettivamente presente, restituire la posizione esatta di un elemento
  nella successione uguale all'elemento cercato.
\end{Definition}
\par Nel seguito denoteremo
\begin{itemize}
  \item $e$ l'elemento cercato;
  \item $(e_i)_{i \in n}$ ($n \in \mathbb{N}$) la
    successione nella quale avviene la ricerca.
\end{itemize}
\begin{Theorem}
	Il problema di ricerca in una successione ordinata ha complessit\`a
  $\BigOmega{\log n}$.
\end{Theorem}
\Proof L'elemento cercato pu\`o coincidere con uno degli $n$ elementi della
successione o essere assente.
\par Effettuando $1$ confronto, l'algoritmo pu\`o discriminare tra al pi\`u $2$
configurazioni: denotati con $a$ e $b$ i due elementi confrontati, pu\`o
rilevare se $a = b$ o se $a \neq b$. Assumiamo che con $k$
confronti l'algoritmo possa discriminare tra $2^k$ confronti: allora, con $1$
confronto aggiuntivo, esso pu\`o discrimare tra $2^k \cdot 2 = 2^{k + 1}$
configurazioni. Dunque, segue per induzione che con $k$ confronti si pu\`o
discrimare tra $2^k$ configurazioni.
\par Perch\'e l'algoritmo possa fornire la risposta corretta al problema, \`e
dunque necessario che il numero $k$ di confronti da esso effettuato verifichi la
condizione $2^k \geq n + 1$, equivalente a $k \geq \log_2 (n + 1)$. La tesi
segue dal fatto che $\IsBigOmega{\log_2 (n + 1)}{\log n}$. \EndProof
\begin{Definition}
  Supponiamo la successione $(e_i)_{i \in n}$ ordinata secondo un ordinamento
dato $\leq$. Si chiama \Define{ricerca binaria}[binaria][ricerca] l'algoritmo
seguente:
  \begin{enumerate}
    \item\label{RicercaBinaria_1} se $n = 1$ e $e_0 \neq e$, allora restituire
      $-1$ (o un qualunque altro valore convenzionale per indicare l'assenza
      dell'elemento cercato nella successione);
    \item\label{RicercaBinaria_2} se $n = 1$ e $e_0 = e$, restituire $0$;
    \item\label{RicercaBinaria_3} si ponga
      $M = \left \lfloor \frac{n - 1}{2} \right \rfloor$;
    \item\label{RicercaBinaria_4} se $e \leq e_M$, allora si restituisca quanto
      restituito dall'esecuzione dell'intero algoritmo cercando $e$ nella
      successione $(e_i)_{i = 0}^M$;
    \item\label{RicercaBinaria_5} denotato $I$ il valore
      restituito dall'esecuzione dell'intero algoritmo cercando $e$ nella
      successione
      $(f_i)_{i \in n - M - 1}$,
      dove per ogni $i \in n - M - 1$ poniamo
      $f_i = e_{i + M + 1}$,
      si restituisca $M + 1 + I$.
  \end{enumerate}
\end{Definition}
\begin{listing}
	\insertcode{cpp}{"Algoritmi_e_strutture_di_dati/RicercaBinaria.cpp"}
	\caption{Ricerca binaria implementata in \LanguageName{C++}.}
\end{listing}
\begin{Theorem}
  La ricerca binaria risolve il problema della ricerca per una successione
  ordinata.
\end{Theorem}
\Proof Supponiamo $e$ non faccia parte della successione $(e_i)_{i \in n}$.
Allora i passi \ref{RicercaBinaria_4} e \ref{RicercaBinaria_5} accorceranno
la successione ove avviene la ricerca fino a ridurla ad un solo elemento:
per il passo \ref{RicercaBinaria_1} verr\`a resituito $-1$.
\par Supponiamo ora $e$ appaia in $(e_i)_{i \in n}$.
Se $n = 1$ la tesi \`e immediata, altrimenti procediamo
per induzione su $n$ assumendo la tesi vera per ogni $n < N$
($N \in \mathbb{N} \SetMin 2$) e proviamola per $N$.
\par Se $e \leq e_M$, allora la successione
$(e_i)_{i = 0}^M$
contiene $e$ e per ipotesi induttiva il passo \ref{RicercaBinaria_4}
trova l'elemento $e$; altrimenti \`e la successione
$(e_i)_{i = M + 1}^{n - 1}$ a contenere $e$, e per ipotesi induttiva
il passo \ref{RicercaBinaria_5} trova l'elemento $e$. \EndProof
\begin{Theorem}
  La ricerca binaria ha costo $\BigO{\log n}$, secondo il \English{comparison
  model}.
\end{Theorem}
\Proof Denotiamo $T(n)$ il costo della ricerca binaria in una successione di
dimensione $n$. Abbiamo evidentemente $T(1) = 1$ (i confronti dei passi
\ref{RicercaBinaria_1} e \ref{RicercaBinaria_2} sono lo stesso confronto: non
serve verificare due volte se $e_0 = e$; il confronto viene ripetuto nella
descrizione dell'algoritmo solo per chiarezza di esposizione).
\par Contiamo i confronti, assumendo $n > 1$; dato che siamo interessati ad una
stima asintotica, possiamo assumere $n$ pari senza perdita di generalit\`a:
\begin{itemize}
  \item non contiamo il confronto dei passi \ref{RicercaBinaria_1} e
    \ref{RicercaBinaria_2} perch\'e superfluo dopo aver gi\`a osservato
    che $n = 1$ (potremmo comunque anche contarlo ottenendo lo stesso risultato
    asintotico);
  \item abbiamo il confronto $e \leq e_M$ al passo \ref{RicercaBinaria_4};
  \item se effettivamente $e \leq e_M$, allora abbiamo
    $T(M + 1) = T \left ( \left \lfloor \frac{n + 1}{2} \right \rfloor \right )
      = T \left ( \frac{n}{2} \right )$
    ulteriori confronti;
  \item se invece $e > e_M$, allora abbiamo
    $T(n - 1 - (M + 1) + 1) = T \left ( n - \frac{n}{2} \right )
      = T \left ( \frac{n}{2} \right )$
    ulteriori confronti.
\end{itemize}
\par Dunque, per opportune costanti $C_0, C > 0$, abbiamo
\[
	T(n) \leq
	\begin{cases}
		1\text{ se }n = 1,\\
		T(\frac{n}{2}) + 1\text{ per } n > 1.
	\end{cases}
\]
\par Per il teorema fondamentale delle ricorrenze, abbiamo
$T(n) = \BigO{\log n}$. \EndProof
\begin{Corollary}
	Il problema di ricerca in una successione ordinata ha complessit\`a
  $\BigTheta{n \log n}$ e la ricerca binaria \`e asintoticamente ottima.
\end{Corollary}
\Proof Abbiamo gi\`a dimostrato che il problema di ricerca in una successione
ordinata  ha complessit\`a
$\BigOmega{\log n}$.
Il teorema precedente ha provato che il problema ha complessit\`a
$\BigO{\log n}$.
Ne deduciamo infine che esso ha complessit\`a $\BigTheta{\log n}$. \EndProof


\section{Strutture di dati}
\label{AlgoritmiEStruttureDiDati_StruttureDiDati}
\begin{Definition}
	Una \Define{struttura di dati}[di dati][struttura] \`e il dato di
	\begin{itemize}
		\item una collezione di dati;
		\item operazioni sui campi.
	\end{itemize}
\end{Definition}
\begin{Definition}
	Una \Define{classe} \`e uno strumento fornito da alcuni linguaggi di programmazione (per esempio il \LanguageName{C++} o il \LanguageName{Python}) che consente di implementare una struttura di dati in maniera unitaria e senza specificarne esplicitamente i valori dei dati. In questo contesto chiamiamo
	\begin{itemize}
		\item \Define{campi}[di una classe][campo] la rappresentazione della classe della collezione di dati della struttura implementata;
		\item \Define{metodi}[di una classe ][metodo] l'implementazione delle operazioni della struttura implementata;
		\item \Define{istanza della classe}[di una classe][istanza] o \Define{oggetto} una struttura di dati specifica, cio\`e i cui dati hanno un valore specificato, implementata per mezzo della classe;
		\item \Define{costruttore}[di una classe][costruttore] il metodo che crea un oggetto in memoria;
		\item \Define{distruttore}[di una classe][distruttore] il metodo che libera la memoria occupata da un oggetto.
	\end{itemize}
\end{Definition}
\subsection{Liste e alberi.}
\label{AlgoritmiEStruttureDiDati_ListeEAlberi}
\begin{Definition}
  Un \Define{albero} \`e una struttura di dati
  \begin{itemize}
    \item o vuota;
    \item o costituita da un dato e da puntatori ad altri alberi; a seconda del
    contesto, chiameremo \Define{nodo} o il dato in questione o lo stesso dato
    insieme ai suddetti puntatori.
  \end{itemize}
  Se ogni albero che interviene nella costruzione ricorsiva di un albero ha lo
  stesso numero $n \in \mathbb{N}$ di puntatori, allora l'albero di dice
  \Define{$n$-ario}[$n$-ario][albero].
  Inoltre,
  \begin{itemize}
    \item un albero unario si chiama \Define{lista};
    \item gli alberi puntati da un albero binario non vuoto si chiamano
      \Define{figlio sinistro}[sinistro][figlio]
      e
      \Define{figlio destro}[destro][figlio]; ne conseguono ulteriori metafore
      parentali con ovvio significato.
  \end{itemize}
\end{Definition}
\begin{listing}
	\insertcode{cpp}{"Algoritmi_e_strutture_di_dati/Lista.h"}
	\caption{Implementazione di una struttura di lista in \LanguageName{C++}.}
\end{listing}
\begin{listing}
	\insertcode{cpp}{"Algoritmi_e_strutture_di_dati/AlberoBinario.h"}
	\caption{Implementazione di una struttura di albero binario in \LanguageName{C++}: dichiarazione della classe.}
\end{listing}
\begin{listing}
	\insertcode{cpp}{"Algoritmi_e_strutture_di_dati/AlberoBinario.cpp"}
	\caption{Implementazione di una struttura di albero binario in \LanguageName{C++}: costruttore.}
\end{listing}
\begin{Theorem}
  Una struttura di albero $n$-ario implementa il concetto di albero $n$-ario
come sinonimo di arborescenza di $n$ figli.
\end{Theorem}
\par Grazie al teorema precedente, trasportiamo sugli alberi $n$-ari tutte le
definizioni e tutti i teoremi dimostrati.
\begin{listing}
	\insertcode{cpp}{"Algoritmi_e_strutture_di_dati/Dimensione.cpp"}
	\caption{Implementazione di un metodo \LanguageName{C++} per il calcolo della dimensione di un albero.}
\end{listing}
\begin{listing}
	\insertcode{cpp}{"Algoritmi_e_strutture_di_dati/Altezza.cpp"}
	\caption{Implementazione di un metodo \LanguageName{C++} per il calcolo dell'altezza di un albero.}
\end{listing}
\begin{Definition}
		Consideriamo un albero $n$-ario e una qualche funzione
    \mintinline{c}{Funzione} avente tra gli argomenti il dato di un nodo. Si
    dice che un algoritmo che parte dalla radice dell'albero effettua
\begin{itemize}
	\item \Define{visite anticipate}[anticipata][visita] se prima viene chiamata
    \mintinline{c}{Funzione} e poi viene applicato ricorsivamente l'algoritmo ai
    figli;
	\item \Define{visite posticipate}[posticipata][visita] se prima viene viene
    applicato ricorsivamente l'algoritmo ai figli e poi viene chiamata
    \mintinline{c}{Funzione};
	\item \Define{visite per ampiezza}[per ampiezza][visita] se applica la
    funzione \mintinline{c}{Funzione} a tutti i nodi in ordine di altezza
    crescente.
\end{itemize}
  Se $n = 2$, diciamo anche che un algoritmo effettua
  \begin{itemize}
	  \item \Define{visite simmetriche}[simmetrica][visita] se prima viene applicato
      l'algoritmo ricorsivamente al figlio sinistro, poi viene chiamata
      \mintinline{c}{Funzione}, infine viene applicato ricorsivamente l'algoritmo
      al figlio destro.
  \end{itemize}
\end{Definition}
\begin{listing}
	\insertcode{cpp}{"Algoritmi_e_strutture_di_dati/VisitaAnticipata.cpp"}
	\caption{Implementazione di un algoritmo che effettua visite anticipate in linguaggio \LanguageName{C++}.}
\end{listing}
\begin{listing}
	\insertcode{cpp}{"Algoritmi_e_strutture_di_dati/VisitaPosticipata.cpp"}
	\caption{Implementazione di un algoritmo che effettua visite posticipate in linguaggio \LanguageName{C++}.}
\end{listing}
\begin{listing}
	\insertcode{cpp}{"Algoritmi_e_strutture_di_dati/VisitaSimmetrica.cpp"}
	\caption{Implementazione di un algoritmo che effettua visite simmetriche in linguaggio \LanguageName{C++}.}
\end{listing}

\subsection{Alberi binari di ricerca.}
\label{AlgoritmiEStruttureDiDati_AlberiBinariDiRicerca}
\begin{listing}
	\insertcode{cpp}{"Algoritmi_e_strutture_di_dati/AlberoBinarioDiRicerca.h"}
	\caption{Implementazione di una struttura di albero binario di ricerca in \LanguageName{C++}: dichiarazione della classe.}
\end{listing}
\begin{listing}
	\insertcode{cpp}{"Algoritmi_e_strutture_di_dati/AlberoBinarioDiRicerca.cpp"}
	\caption{Implementazione di una struttura di albero binario di ricerca in \LanguageName{C++}: costruttore.}
\end{listing}
\begin{listing}
	\insertcode{cpp}{"Algoritmi_e_strutture_di_dati/AlberoBinarioDiRicerca_Inserisci.cpp"}
	\caption{Implementazione dell'inserzione di un nodo in un albero binario di ricerca in \LanguageName{C++}.}
\end{listing}
\begin{listing}
	\insertcode{cpp}{"Algoritmi_e_strutture_di_dati/AlberoBinarioDiRicerca_Ricerca.cpp"}
	\caption{Implementazione della ricerca di un nodo in un albero binario di ricerca in \LanguageName{C++}.}
\end{listing}
\begin{listing}
	\insertcode{cpp}{"Algoritmi_e_strutture_di_dati/AlberoBinarioDiRicerca_Cancella.cpp"}
	\caption{Implementazione della cancellazione di nodi in un albero binario di ricerca in \LanguageName{C++}.}
\end{listing}
\begin{listing}
	\insertcode{cpp}{"Algoritmi_e_strutture_di_dati/AlberoBinarioDiRicerca_MinimoSottoAlbero.cpp"}
	\caption{Implementazione dell'individuazione del minimo sottoalbero in un albero binario di ricerca in \LanguageName{C++}.}
\end{listing}

\subsection{Alberi AVL.}
\label{AlgoritmiEStruttureDiDati_AlberiAVL}
\begin{listing}
	\insertcode{cpp}{"Algoritmi_e_strutture_di_dati/AVL_Altezza.cpp"}
	\caption{Implementazione di una funzione \LanguageName{C++} per il calcolo dell'altezza di un alberoi AVL.}
\end{listing}
\begin{listing}
	\insertcode{cpp}{"Algoritmi_e_strutture_di_dati/AVL_Inserisci.cpp"}
	\caption{Implementazione dell'inserzione di un nodo in un albero AVL in \LanguageName{C++}.}
\end{listing}
\begin{listing}
	\insertcode{cpp}{"Algoritmi_e_strutture_di_dati/AVL_NuovaFoglia.cpp"}
	\caption{Implementazione della creazione di una nuova foglia in un albero AVL in \LanguageName{C++}.}
\end{listing}
\begin{listing}
	\insertcode{cpp}{"Algoritmi_e_strutture_di_dati/AVL_RuotaOraria.cpp"}
	\caption{Implementazione della rotazione oraria di alcuni nodi in un albero AVL in \LanguageName{C++}.}
\end{listing}
\begin{listing}
	\insertcode{cpp}{"Algoritmi_e_strutture_di_dati/AVL_RuotaAntiOraria.cpp"}
	\caption{Implementazione della rotazione antioraria di alcuni nodi in un albero AVL in \LanguageName{C++}.}
\end{listing}

\subsection{Pila.}
\label{AlgoritmiEStruttureDiDati_Pila}
\begin{Definition}
	Si dice che una struttura di dati implementa la strategia \Define{LIFO}
  (\English{Last in first out}) quando memorizza dati in successione e, quando
  interrogata, restituisce ed espelle il dato memorizzato pi\`u di recente.
\end{Definition}
\begin{Definition}
  Una struttura di dati che implementa la strategia LIFO si chiama \Define{pila}
  (\Define{\English{stack}}). Inoltre,
  \begin{itemize}
    \item il metodo che aggiunge un dato alla pila \`e solitamente chiamato
      \Define{\English{push}};
    \item il metodo che estrae un dato dalla pila \`e solitamente chiamato
      \Define{\English{pop}}.
  \end{itemize}
\end{Definition}
\par Un'implementazione di una pila tramite \English{array}
richiede normalmente di conoscere a tempo di compilazione qual \`e il massimo
numero di elementi che possa essere impilato; tuttavia, qualora ci\`o non fosse
possibile, esiste comunque la possibilit\`a di ricollocare una pila esistente in
un nuovo \English{array} pi\`u lungo quando
necessario, e successivamente di ricollocare nuovamente la pila in un
\English{array} pi\`u piccolo qualora si ritenesse utile mantenere limitata la
quantit\`a di memoria occupata da un programma. Al fine di ammortizzare i costi
di ricollocazione, conviene evitare di allungare o accorciare l'\English{array}
di una cella per volta, per esempio preferendo invece procedere per
raddoppiamenti e dimezzamenti.
\begin{listing}
	\insertcode{cpp}{"Algoritmi_e_strutture_di_dati/Pila.h"}
	\caption{Implementazione di una pila in \LanguageName{C++}: dichiarazione della classe.}
\end{listing}
\begin{listing}
	\insertcode{cpp}{"Algoritmi_e_strutture_di_dati/Pila.cpp"}
	\caption{Implementazione di una pila in \LanguageName{C++}: costruttore.}
\end{listing}
\begin{listing}
	\insertcode{cpp}{"Algoritmi_e_strutture_di_dati/Pila_Push.cpp"}
	\caption{Implementazione di un metodo \LanguageName{C++} per l'aggiunta di un elemento ad una pila.}
\end{listing}
\begin{listing}
	\insertcode{cpp}{"Algoritmi_e_strutture_di_dati/Pila_Pop.cpp"}
	\caption{Implementazione di un metodo \LanguageName{C++} per l'estrazione di un elemento da una pila.}
\end{listing}
\begin{listing}
	\insertcode{cpp}{"Algoritmi_e_strutture_di_dati/Pila_Top.cpp"}
	\caption{Implementazione di un metodo \LanguageName{C++} per la restituzione di un elemento da una pila, ma senza estrazione.}
\end{listing}
\begin{listing}
	\insertcode{cpp}{"Algoritmi_e_strutture_di_dati/Pila_Vuota.cpp"}
	\caption{Implementazione di un metodo \LanguageName{C++} per determinare se la pila \`e vuota.}
\end{listing}

\subsection{\English{Array} circolare.}
\label{AlgoritmiEStruttureDiDati_ArrayCircolare}
\begin{Definition}
  Un
  \Define{\English{array} circolare}[circolare][\English{array}]
  \`e una struttura di dati caratterizzata da
  \begin{itemize}
    \item una dimensione $n \in \mathbb{N}$;
    \item $n$ dati $(e_i)_{i \in n}$;
    \item un sistema di indicizzazione su $\mathbb{Z}$ (nella pratica, su un suo
      sottoinsieme sufficientemente grande) tale che l'elemento indicizzato da
      $i \in \mathbb{Z}$ \`e l'elemento $e_I$ dove $I \in \mathbb{Z}$ \`e il
      pi\`u piccolo rappresentante non negativo della classe di resto di $i$
      modulo $n$.
  \end{itemize}
\end{Definition}
\begin{listing}
	\insertcode{cpp}{"Algoritmi_e_strutture_di_dati/ArrayCircolare.h"}
	\caption{Implementazione di un \English{array} circolare in
    \LanguageName{C++}: dichiarazione della classe.}
\end{listing}
\begin{listing}
	\insertcode{cpp}{"Algoritmi_e_strutture_di_dati/ArrayCircolare.cpp"}
	\caption{Implementazione di un \English{array} circolare in
    \LanguageName{C++}: costruttore e distruttore.}
\end{listing}
\begin{listing}
	\insertcode{cpp}{"Algoritmi_e_strutture_di_dati/ArrayCircolare_Indexing.cpp"}
	\caption{Implementazione di un \English{array} circolare in
    \LanguageName{C++}: definizione dell'operazione di indicizzazione.}
\end{listing}

\subsection{Coda.}
\label{AlgoritmiEStruttureDiDati_Coda}
\begin{Definition}
	Si dice che una struttura di dati implementa la strategia \Define{FIFO}
  (\English{First in first out}) quando memorizza dati in successione e, quando
  interrogata, restituisce ed espelle il dato memorizzato meno di recente.
\end{Definition}
\begin{Definition}
  Una struttura di dati che implementa la strategia FIFO si chiama \Define{coda}
  (\Define{\English{queue}}). Inoltre,
  \begin{itemize}
    \item il metodo che aggiunge un dato alla coda \`e solitamente chiamato
      \Define{\English{enqueue}};
    \item il metodo che estrae un dato dalla coda \`e solitamente chiamato
      \Define{\English{dequeue}}.
  \end{itemize}
\end{Definition}
\par Strutture di dati particolaremente utili per l'implementazione di una coda
sono l'\English{array} circolare e la lista. Generalmente, \`e preferita la
lista, che come si pu\`o osservare dai codici \`e di pi\`u immediata
implementazione.
\par Per l'esemplificazione di un'implementazione
tramite \English{array} circolare, ci asteniamo da utilizzare i codici gi\`a
usati nella sezione \ref{AlgoritmiEStruttureDiDati_ArrayCircolare} perch\'e non
sono convenienti: in effetti, come si pu\`o notare, gli indici
\mintinline{cpp}{Cima} e
\mintinline{cpp}{Fondo} crescono sempre di un'unit\`a per volta e necessitano,
in generale, di essere rimpiazzati da rappresentati della stessa classe di resto
modulo la lunghezza della coda per evitare errori generati da un eventuale
traboccamento.
\par Segnaliamo, senza dettagli, che \`e possibile implementare una coda anche
con un \English{array} non circolare: ci\`o richiede attenzione con l'uso dei
puntatori che tengono conto del progresso della cosa e il mantenimento di una
cella di memoria vuota per poter distinguere casi limiti di riempimento della
coda.
\par Un'implementazione di una coda tramite \English{array}, circolari o meno,
richiede normalmente di conoscere a tempo di compilazione qual \`e il massimo
numero di elementi che possa essere accodato; tuttavia, qualora ci\`o non fosse
possibile, esiste comunque la possibilit\`a di ricollocare una coda esistente in
un nuovo \English{array} pi\`u lungo quando necessario, e successivamente di
ricollocare nuovamente la coda in un
\English{array} pi\`u piccolo qualora si ritenesse utile mantenere limitata la
quantit\`a di memoria occupata da un programma. Al fine di ammortizzare i costi
di ricollocazione, conviene evitare di allungare o accorciare l'\English{array}
di una cella per volta, per esempio preferendo invece procedere per
raddoppiamenti e dimezzamenti.
\begin{listing}
	\insertcode{cpp}{"Algoritmi_e_strutture_di_dati/Coda_ArrayCircolare.h"}
	\caption{Implementazione di una coda in \LanguageName{C++} tramite
    \English{array} circolare: dichiarazione della classe.}
\end{listing}
\begin{listing}
	\insertcode{cpp}{"Algoritmi_e_strutture_di_dati/Coda_Enqueue_ArrayCircolare.cpp"}
	\caption{Implementazione di un metodo \LanguageName{C++} per l'aggiunta di un
    elemento ad una coda implementata tramite \English{array} circolare.}
\end{listing}
\begin{listing}
	\insertcode{cpp}{"Algoritmi_e_strutture_di_dati/Coda_Dequeue_ArrayCircolare.cpp"}
	\caption{Implementazione di un metodo \LanguageName{C++} per l'estrazione di
    un elemento da una coda implementata tramite \English{array} circolare.}
\end{listing}
\begin{listing}
	\insertcode{cpp}{"Algoritmi_e_strutture_di_dati/Coda_RestituisciPrimo_ArrayCircolare.cpp"}
	\caption{Implementazione di un metodo \LanguageName{C++} per la restituzione
    di un elemento da una coda implementata tramite \English{array} circolare,
    ma senza estrazione.}
\end{listing}
\begin{listing}
	\insertcode{cpp}{"Algoritmi_e_strutture_di_dati/Coda_Vuota_ArrayCircolare.cpp"}
	\caption{Implementazione di un metodo \LanguageName{C++} per determinare se
    una coda implementata tramite \English{array} circolare \`e vuota.}
\end{listing}
\begin{listing}
	\insertcode{cpp}{"Algoritmi_e_strutture_di_dati/Coda_Lista.h"}
	\caption{Implementazione di una coda in \LanguageName{C++} tramite una lista:
    dichiarazione della classe.}
\end{listing}
\begin{listing}
	\insertcode{cpp}{"Algoritmi_e_strutture_di_dati/Coda_Lista.cpp"}
	\caption{Implementazione di una coda in \LanguageName{C++} tramite una lista:
    distruttore.}
\end{listing}
\begin{listing}
	\insertcode{cpp}{"Algoritmi_e_strutture_di_dati/Coda_Enqueue_Lista.cpp"}
	\caption{Implementazione di un metodo \LanguageName{C++} per l'aggiunta di un
    elemento ad una coda implementata tramite una lista.}
\end{listing}
\begin{listing}
	\insertcode{cpp}{"Algoritmi_e_strutture_di_dati/Coda_Dequeue_Lista.cpp"}
	\caption{Implementazione di un metodo \LanguageName{C++} per l'estrazione di
    un elemento da una coda implementata tramite una lista.}
\end{listing}
\begin{listing}
	\insertcode{cpp}{"Algoritmi_e_strutture_di_dati/Coda_RestituisciPrimo_Lista.cpp"}
	\caption{Implementazione di un metodo \LanguageName{C++} per la restituzione
    di un elemento da una coda implementata tramite una lista, ma senza
    estrazione.}
\end{listing}
\begin{listing}
	\insertcode{cpp}{"Algoritmi_e_strutture_di_dati/Coda_Vuota_Lista.cpp"}
	\caption{Implementazione di un metodo \LanguageName{C++} per determinare se
    una coda implementata tramite una lista \`e vuota.}
\end{listing}

\subsection{Coda con priorit\`a e \English{heap}.}
\label{AlgoritmiEStruttureDiDati_CodaConPrioritaEHeap}
\begin{Definition}
  Una \Define{coda con priorit\`a}[con priorit\`a][coda]
  (\Define{\English{priority queue}}) \`e una struttura di dati che memorizza
  dati in successione accoppiati con un numero chiamato \Define{priorit\`a} e li
  restituisce in ordine di priorit\`a, decrescente o crescente a seconda
  dell'implementazione.
  Inoltre, come per le code senza priorit\`a,
  \begin{itemize}
    \item il metodo che aggiunge un dato alla coda \`e solitamente chiamato
      \Define{\English{enqueue}};
    \item il metodo che estrae un dato dalla coda \`e solitamente chiamato
      \Define{\English{dequeue}}.
  \end{itemize}
\end{Definition}
\par Salvo indicazioni contrarie, assumeremo sempre che una coda di priorit\`a
estrae gli elementi con priorit\`a massima.
Ai fini della teoria, possiamo assumere tutte le
priorit\`a distinte senza perdita di generalit\`a: infatti, se una coda di
priorit\`a ha pi\`u elementi con la massima priorit\`a, allora dovr\`a essere
usato un criterio per scegliere quale elemento effettivamente estrarre; ma 
questo criterio pu\`o anche essere scelto per aumentare la priorit\`a di
quell'elemento o abbassare la priorit\`a degli altri, e un ragionamento analogo
vale per qualsiasi altra situazione in cui serva discrimanre tra elementi con
la stessa prioririt\`a.
\par Inoltre, senza perdita di generalit\`a, esamineremo solo code con
priorit\`a che memorizzano dati vuoti, cio\`e che memorizzano solo le priorit\`a
stesse: in effetti, il funzionamento della struttura dipende esclusivamente
dalle priorit\`a, mentre per quanto riguarda i costi in termini di spazio,
baster\`a aumentare il costo della priorit\`a in modo tale che esso includa
anche il costo del dato ad esso associato, operazione che per un'analisi
asintotica non implica nessuna conseguenza.
\par Una possibile implementazione di una coda con priorit\`a prevede l'uso di
liste; essa consente
\begin{itemize}
  \item l'accodamento di nuovi dati in tempo $\BigO{1}$ mettendo in testa alla
    lista il nuovo elemento;
  \item l'estrazione di dati in tempo $\BigO{n}$, leggendo tutti gli elementi
    della lista in successione per individuare l'elemento di priorit\`a pi\`u
    alta o pi\`u bassa.
\end{itemize}
\par Un'ulteriore possibilit\`a \`e l'implementazione tramite un \English{array}
ordinato, che naturalmente deve essere sufficientemente grande o essere
ricollocato in \English{array} pi\`u o meno grandi a seconda delle necessit\`a.
In tal caso, accodamento ed estrazione possono essere effettuati sfruttando
l'algoritmo di ricerca binaria e hanno dunque entrambi costo $\BigO{\log n}$.
\par Generalmente, le code con priorit\`a vengono implementate tramite un'aaltra
struttura ancora denominata \English{heap}, motivo per cui le code di priorit\`a
sono spesso chiamate anche \English{heap}.
\begin{Definition}
  Si chiama \Define{\English{heap}} un albero completo da sinistra che registra
  dati numerici tale che una delle condizioni seguenti sia verificata
  \begin{itemize}
    \item se $\Node$ \`e genitore di $\VarNode$, allora 
      $\Node \geq \VarNode$: in tal caso si dice che l'\English{heap}
      \`e un \Define{\English{heap} $\max$};
    \item se $\Node$ \`e genitore di $\VarNode$, allora 
      $\Node \leq \VarNode$: in tal caso si dice che l'\English{heap}
      \`e un \Define{\English{heap} $\min$}.
  \end{itemize}
  Se l'\English{heap} \`e un albero $n$-ario, allora si dice che
  l'\English{heap} \`e \English{$n$-ario}[$n$-ario][\English{heap}].
\end{Definition}
\begin{Definition}
  Si chiama
  \Define{\English{heap} binario implicito}[binario implicito][\English{heap}]
  la struttura di dati definita da un \English{array} di numeri
  $(e_i)_{i \in n}$ che verifica una delle due delle condizioni seguenti:
  \begin{itemize}
    \item \TheoremName{propriet\`a di \English{heap} $\max$}[di \English{heap} $\max$][propriet\`a]
      se $i \in n$, allora
      \begin{itemize}
        \item se $j = \left \lfloor \frac{i - 1}{2} \right \rfloor \in n$,
          allora $e_j \geq e_i$;
        \item se $j = 2i + 1 \in n$,
          allora $e_i \geq e_j$;
        \item se $j = 2i + 2 \in n$,
          allora $e_i \geq e_j$:
      \end{itemize}
      in tal caso si dice che l'\English{heap} binario implicito \`e un
      \Define{\English{heap} binario implicito $\max$}[binario implicito $\max$][\English{heap}];
    \item \TheoremName{propriet\`a di \English{heap} $\min$}[di \English{heap} $\min$][propriet\`a]
      se $i \in n$, allora
      \begin{itemize}
        \item se $j = \left \lfloor \frac{i - 1}{2} \right \rfloor \in n$,
          allora $e_j \leq e_i$;
        \item se $j = 2i + 1 \in n$,
          allora $e_i \leq e_j$;
        \item se $j = 2i + 2 \in n$,
          allora $e_i \leq e_j$:
      \end{itemize}
      in tal caso si dice che l'\English{heap} binario implicito \`e un
      \Define{\English{heap} binario implicito $\min$}[binario implicito $\min$][\English{heap}].
  \end{itemize}
\end{Definition}
\begin{Theorem}
  Un \English{heap} binario $\max$ di $n \in \NotZero{\mathbb{N}}$ nodi pu\`o
  essere implementato tramite un \English{heap} binario implicito $\max$ di
  dimensione $n$ tramite una corrispondenza che associa al $k$-esimo (contando
  da $0$) nodo da sinistra di livello $L$ l'elemento di indice $2^L + k - 1$
  dell'\English{heap} binario impliciato in modo tale che il dato numerico
  registrato nel nodo sia lo stesso registrato nell'elemento dell'\English{heap}
  binario implicito ad esso associato.
\end{Theorem}
\Proof Occorre verificare che l'\English{array} immagine della corrispondenza
sia effettivamente un \English{heap} binario implicito.
\par Consideriamo il $k$-esimo (contando da $0$) nodo $\Node$ di livello $L$
dell'\English{heap} binario. I suoi figli, se esistono, si ottengono scorrendo i
figli di tutti i nodi pi\`u a sinistra di $\Node$: essi sono in numero $2k$.
Dunque, i figli di $\Node$ sono, se esistono, il $2k$-esimo e il
$(2k + 1)$-esimo nodo al livello $L + 1$: le loro immagini, rispetto alla
corrispondenza dell'enunciato, sono
l'$(2^{L + 1} + 2k - 1)$-esimo
e
l'$(2^{L + 1} + 2k)$-esimo elemento dell'\English{array}.
\par Fissiamo ora $I \in n$. Siano
\begin{itemize}
  \item $L \in \mathbb{N}$;
  \item $k \in \mathbb{N}$ con $0 \leq k < 2^L$;
\end{itemize}
tali che $I = 2^L + k - 1$.
Denotiamo $(e_i)_{i \in n}$ gli elementi dell'\English{array}.
\par Per costruzione, $e_I$ \`e immagine del $k$-esimo (contando da $0$) nodo
$\Node$ di livello $L$ dell'\English{heap} binario. Per la propritet\`a di
\English{heap} $\max$, il nodo \`e maggiore o uguale dei suoi figli, e dunque
\begin{itemize}
  \item se $J = 2I + 1 \in n$, abbiamo $J = 2^{L + 1} + 2k - 1$ e $e_J$ \`e
    l'immagine del $(2k + 1)$-esimo nodo di livello $L + 1$, dunque
    $e_I \geq e_J$;
  \item se $J = 2I + 2 \in n$, abbiamo $J = 2^{L + 1} + 2k$ e $e_J$ \`e
    l'immagine del $(2k)$-esimo nodo di livello $L + 1$, dunque
    $e_I \geq e_J$.
\end{itemize}
\par Inoltre,
\begin{itemize}
  \item se $I = 0$, allora
    $J = \left \lfloor \frac{I - 1}{2} \right \rfloor = - 1 \notin n$;
  \item se $I > 0$ \`e dispari, allora $L > 0$, $k > 0$ e $k$ \`e pari, da cui
    $J = \left \lfloor \frac{I - 1}{2} \right \rfloor
    = \frac{2^L + k - 2}{2} = 2^{L - 1} + \frac{k}{2} - 1$: $e_J$ \`e l'immagine
    del $\frac{k}{2}$-esimo nodo di livello $L - 1$ e ha per figlio sinistro
    l'$I$-esimo nodo di livello $L$, dunque $e_J \geq e_I$;
  \item se $I > 0$ \`e pari, allora $L > 0$, $k > 0$ e $k$ \`e dispari, da cui
    $J = \left \lfloor \frac{I - 1}{2} \right \rfloor
    = \left \lfloor \frac{2^L + k - 2}{2} \right \rfloor
    = 2^{L - 1} + \frac{k - 1}{2} - 1$: $e_J$ \`e
    l'immagine del $\frac{k - 1}{2}$-esimo nodo di
    livello $L - 1$ e ha per figlio destro l'$I$-esimo nodo di livello $L$,
    dunque $e_J \geq e_I$. \EndProof
\end{itemize}
\begin{Theorem}
  Un \English{heap} binario di $n \in \mathbb{N}$ elementi costa
  \begin{itemize}
    \item $\BigO{n}$ spazio se implementato tramite liste;
    \item $n$ spazio se implementato tramite \English{heap} binario implicito.
  \end{itemize}
\end{Theorem}
\Proof L'implementazione tramite liste richiede la memorizzazione di puntatori,
mentre l'implementazione tramite \English{heap} binario implicito memorizza solo
le priorit\`a. \EndProof



	\chapter{Metodi numerici per equazioni differenziali ordinarie}
	\section{Definizioni e teoremi di base}
\label{MetodiNumericiPerEquazioniDifferenzialiOrdinarie_DefinizioniETeoremiDiBase}
\begin{Definition}
	Sia un problema di Cauchy
	\[
	\begin{cases}
		y' = f(t,y),\\
		y(t_0) = y_0.
	\end{cases}
	\]
	Chiamiamo il problema di Cauchy
	\[
	\begin{cases}
		z' = f(t,z) + \Perturbation(t),\\
		z(t_0) = y_0 + \Perturbation_0,
	\end{cases}
	\]
	con $\Perturbation(t): \Dom{y} \rightarrow \Cod{y}$ e
  $\Perturbation_0 \in \Cod{y}$,
  \Define{problema perturbato}[di Cauchy perturbato][problema] del problema
  originale.
  Chiamiamo inoltre
  le quantit\`a $\Perturbation_0$ e
  $\Perturbation(t)$ al variare di $t \in \Cod{y}$
  \Define{perturbazione}[di un problema di Cauchy][perturbazione] o
  \Define{errore}[di un problema di Cauchy][errore]
\end{Definition}
\begin{Definition}
	Siano il problema di Cauchy sull'intervallo $I$
	\[
	\begin{cases}
		y' = f(t,y),\\
		y(t_0) = y_0,
	\end{cases}
	\]
	e il problema di Cauchy perturbato
	\[
	\begin{cases}
		z' = f(t,z) + \Perturbation(t),\\
		z(t_0) = y_0 + \Perturbation_0.
	\end{cases}
	\]
	Il problema di Cauchy originale si definisce
  \Define{stabile}[di Cauchy stabile][problema]
  quando esistono $\epsilon > 0$ e $C \geq 0$ tali che,
  per ogni $t \in \Dom{y}$ abbiamo
	\begin{itemize}
    \item $\Norm{\Perturbation_0} \leq \epsilon$,
		\item $\Norm{z'(t) - y'(t)} = \Norm{\Perturbation(t)} \leq \epsilon$,
		\item $\Norm{z(t) - y(t)} \leq C \epsilon$.
	\end{itemize}
	\par Nel caso in cui $I$ sia illimitato, diciamo inoltre che il problema di Cauchy originale \`e \Define{asintoticamente stabile}[stabile][asintoticamente] quando esiste $\epsilon > 0$ tale che $\lim_{t \rightarrow + \infty} \Norm{z(t) - y(t)} = 0$.
\end{Definition}
\begin{Lemma}
	\TheoremName{Lemma di Gronwall}[di Gronwall][lemma] Siano
	\begin{itemize}
		\item $I = [t_0,t_{\text{max}}) \subseteq \mathbb{R}$;
		\item $f,g \in \CClass{0}$;
		\item $h: I \rightarrow \RealPositive$ integrabile;
		\item $g$ non decrescente;
		\item $t \in I$.
	\end{itemize}
	Se $f(t) \leq g(t) + \int_{t_0}^t h(\tau)f(\tau)d\tau$, allora abbiamo anche $f(t) \leq g(t) \Exponential{\int_{t_0}^th(\tau)d\tau}$.
\end{Lemma}
\begin{Theorem}
	Sia il problema di Cauchy sull'intervallo $I$
	\[
	\begin{cases}
		y' = f(t,y),\\
		y(t_0) = y_0.
	\end{cases}
	\]
	Se $I$ \`e limitato e $f(t,y)$ \`e lipschitziana nella seconda variable
  rispetto a distanze indotte da norme,
  allora il problema di Cauchy \`e stabile.
\end{Theorem}
\Proof Sia il problema di Cauchy perturbato,
\[
\begin{cases}
	z' = f(t,z) + \Perturbation(t),\\
	z(t_0) = y_0 + \Perturbation_0.
\end{cases}
\]
\par Supponiamo le perturbazioni $\delta_0$ e $\delta(t)$ limitate da
$\epsilon > 0$: in simboli, $\delta_0 \leq \epsilon$ e
$\ForAll{t \in I}{\Perturbation(t) \leq \epsilon}$. Supponiamo inoltre $f$ sia
$L$-lipschitziana rispetto alle distanze indotte da norme fissate
$\Norm{\cdot}$, dove la norma sul codominio di $f$ \`e il valore assoluto.
\par Definiamo la funzione $w(t) = z(t) - y(t)$. Abbiamo un ulteriore problema di Cauchy:
\[
\begin{cases}
  w'(t) = z'(t) - y'(t) = f(t,z) + \Perturbation(t) - f(t,y),\\
  w(t_0) = z(t_0) - y(t_0) = \Perturbation_0.
\end{cases}
\]
\par Abbiamo
$w(t) =
  w(t_0)
  + \int_{t_0}^t \Perturbation(\tau)d\tau
  + \int_{t_0}^t (f(\tau,z) - f(\tau,y))d\tau$,
da cui
$\Norm{w} \leq
  \epsilon
  + \AbsoluteValue{t - t_0}\epsilon
  + \int_{t_0}^t L \Norm{z - y}
  = (1 + \AbsoluteValue{t - t_0})\epsilon + \int_{t_0}^t L \Norm{w}d\tau$.
\par Per il lemma di Gronwall, abbiamo inoltre
$\Norm{w} \leq
(1 + \AbsoluteValue{t - t_0})\epsilon \Exponential{(L\AbsoluteValue{t - t_0})}$,
da cui deduciamo la stabilit\`a del problema di Cauchy. \EndProof
\begin{Example}
	Consideriamo il problema di Cauchy su $I = [0,10]$
	\[
	\begin{cases}
		y' = -10y,\\
		y_0 = 1.
	\end{cases}
	\]
	L'equazione differenziale \`e lineare e dunque si calcola immediatamente che
  l'unica soluzione \`e $y(t) = e^{-10t}$.
	\par Ora, $-10y$ \`e $10$-lipschitziana. Se adottiamo la maggiorazione
  dell'errore dovuto alla perturbazione del sistema ricavata al termine della
  dimostrazione precedente, otteniamo che esso \`e minore o uguale a
  $11 \epsilon e^{100}$, dove $\epsilon$ \`e una costante maggiorante
  $\Perturbation_0$ e $\Perturbation(t)$ per ogni $t \in I$. Per ottenere una
  maggiorazione dell'errore minore di $1$ occorre scegliere un $\epsilon$
  nell'intervallo $(10^{-45},10^{-44})$.
  \par Si consideri che la precisione di macchina delle macchine moderne comune
  \`e circa $10^{-16}$.
\end{Example}
\begin{Definition}
	Sia $I = [a,b]$ un intervallo della retta reale estesa $\RealExtended$.
  Chiamiamo \Define{discretizzazione} una famiglia crescente di punti
  $(t_j)_{j \in N}$ ($N \in \NaturalExtended$) tale che $t_0 = a$ e, se $N$ \`e
  finito, $t_{N - 1} = b$. Chiamiamo i punti di $(t_j)_{j \in N}$
  \Define{nodi della discretizzazione}[di una discretizzazione][nodo] della
  discretizzazione. Se
  $\Exists{h \in \RealPositive}{\ForAll{j \in \NotZero{N}}{t_j - t_{j - 1}=h}}$,
  allora la discretizzazione si dice
  \Define{uniforme}[uniforme][discretizzazione] e la costante
  $\Step = t_1 - t_0$ si chiama
  \Define{passo della discretizzazione}[della discretizzazione][passo].
\end{Definition}
\begin{Definition}
	Dato un problema di Cauchy su un intervallo $I$
	\[
	\begin{cases}
		y' = f(t,y),\\
		y(t_0) = y_0,
	\end{cases}
	\]
	chiamiamo \Define{metodo numerico per un problema di Cauchy}[numerico per un problema di Cauchy][metodo] un algoritmo che associa al problema di Cauchy e a una discretizzazione dell'intervallo di definizione del problema di $N \in \NaturalExtended$ nodi un vettore $(y_j)_{i \in N}$ che approssima il vettore $(y(t_j))_{j \in N}$.
\end{Definition}

\section{Metodi numerici a un passo.}
\label{MetodiNumericiPerEquazioniDifferenzialiOrdinarie_MetodiNumericiAUnPasso}
\begin{Definition}
\label{MetodiNumericiPerEquazioniDifferenzialiOrdinarie_DefinizioneMetodoNumerico}
	Dati un problema di Cauchy su un intervallo $I$
	\[
	\begin{cases}
		y' = f(t,y),\\
		y(t_0) = y_0,
	\end{cases}
	\]
	e una discretizzazione uniforme di $I$ di passo $\Step \in \RealPositive$ e
  $N$ nodi, chiamiamo
  \Define{metodo a un passo}[a un passo][metodo] un metodo numerico che associa
  al problema di Cauchy e alla una discretizzazione uniforme un vettore
  $(y_j)_{j \in N}$ definito da
	\[
	\begin{cases}
		y_0 = y_0,\\
		y_{j + 1} = y_j + \Step\phi_\Step(t_j,y_j)\text{ per }j \geq 0,
	\end{cases}
	\]
	dove $\phi_\Step(t,y): \Dom{f} \rightarrow \Cod{f}$. Diciamo che il metodo \`e
	\begin{itemize}
		\item \Define{esplicito}[a un passo esplicito][metodo]
          quando $\phi_\Step(t,y)$ \`e data in forma esplicita;
	 	\item \Define{implicito}[a un passo implicito][metodo]
          quando $\phi_\Step(t,y)$ \`e data in forma implicita.
	\end{itemize}
\end{Definition}
\begin{Definition}
	Con le notazioni della definizione
\ref{MetodiNumericiPerEquazioniDifferenzialiOrdinarie_DefinizioneMetodoNumerico},
	definiamo il metodo numerico
	\begin{itemize}
		\item \Define{convergente}[di un metodo numerico per un problema di Cauchy][convergenza]
          quando $\lim_{\Step \rightarrow 0} \max_{j \in N} \Norm{y(t_j) - y_j} = 0$;
		\item \Define{convergente di ordine $p$}[di ordine $p$ di un metodo numerico per un problema di Cauchy][convergenza]
          quando \`e convergente e inoltre
          $\IsBigO{\max_{j \in N} \Norm{y_j - y(t_j)}}{h^p}$.
	\end{itemize}
\end{Definition}
\begin{Definition}
	\label{MetodiNumericiPerEquazioniDifferenzialiOrdinarie_DefinizioneConsistenza}
	Con le notazioni della definizione
\ref{MetodiNumericiPerEquazioniDifferenzialiOrdinarie_DefinizioneMetodoNumerico},
	definiamo gli \Define{errori locali di troncamento}[locale di troncamento][errore] $(\LocalTruncationError_j)_{j \in \NotZero{N}}$ come le quantit\`a che, al variare di $j \in N - 1$, verificano $y(t_{j + 1}) = y(t_j) + \Step \phi_\Step(t_j,y(t_j)) + \Step \LocalTruncationError_{j + 1}$.
	Definiamo il metodo numerico
	\begin{itemize}
		\item \Define{consistente}[di un metodo numerico per un problema di Cauchy][consistenza] quando $\ForAll{j \in N}{\lim_{\Step \rightarrow 0} \LocalTruncationError_j = 0}$;
		\item \Define{consistente d'ordine $p$}[di ordine $p$ di un metodo numerico per un problema di Cauchy][consistenza] quando \`e consistente e inoltre $\ForAll{j \in N}{\IsBigO{\Norm{\LocalTruncationError_j}}{\Step^p}}$.
	\end{itemize}
\end{Definition}
\begin{Theorem}
	Con le notazioni del teorema \ref{MetodiNumericiPerEquazioniOrdinarie_DefinizioneConsistenza}, il metodo numerico \`e consistente di ordine $p$ se e solo se
\[
	\ForAll{j \in \NotZero{N}}{\IsBigO{y(t_{j + 1}) - y(t_j) - \Step \phi_\Step(t_j,y(t_j)))}{\Step^{p + 1}}}.
\]
\end{Theorem}
\Proof Il metodo numerico \`e consistente di ordine $p$ se e solo se, per ogni $j \in \NotZero{N}$, abbiamo 
$\lim_{\Step \rightarrow 0} \frac{\Norm{\LocalTruncationError_j}}{\Step^p} = 0$, che equivale a
$\lim_{\Step \rightarrow 0} \frac{\Norm{y(t_{j + 1}) - y(t_j) - \Step \phi_\Step(t_j,y(t_j))}}{\Step^{p + 1}} = 0$. \EndProof
\begin{Theorem}
	Con le notazioni del teorema \ref{MetodiNumericiPerEquazioniOrdinarie_DefinizioneConsistenza}, se
\[
	\ForAll{z_0 \in \Image{f} \cup \lbrace y_0 \rbrace}{\IsBigO{y(t_1) - z_0 - \Step\phi_\Step(t_0,z_0)}{h^{\ConsistencyOrder + 1}}}.
\]
	allora il metodo numerico \`e consistente di ordine $p$.
\end{Theorem}
\Proof Fissiamo $j \in N$. Definiamo un nuovo problema di Cauchy a partire da quello vecchio modificando solamente la condizione iniziale:
\[
\begin{cases}
	z' = f(t,z),\\
	z(t_0) = y(t_j).
\end{cases}
\]
Applichiamo al problema nuovo lo stesso metodo numerico ottenendo la successione
\[
\begin{cases}
	z_0 = y(t_j),\\
	z_{j + 1} = z_j + \Step\phi_\Step(t_j,z_j)\text{ per }j \geq 0,
\end{cases}
\]
Abbiamo allora $y(t_{j + 1}) - y(t_j) - \Step \phi_\Step(t_j,y(t_j)) = z(t_1) - z_0 - \Step \phi_\Step(t_0,z_0) = \BigO{\Step^{\ConsistencyOrder + 1}}$. \EndProof
\begin{Definition}
	Siano un problema di Cauchy sull'intervallo $I$ definito da
	\[
	\begin{cases}
		y' = f(t,y),\\
		y(t_0) = y_0,
	\end{cases}
	\]
	una discretizzazione uniforme di $I$ di passo $\Step \in \RealPositive$ e $N$ nodi e il metodo numerico a un passo
	\[
	\begin{cases}
		y_0 = y_0,\\
		y_{j + 1} = y_j + \Step\phi_\Step(t_j,y_j)\text{ per }j \geq 0.
	\end{cases}
	\]
	Chiamiamo il metodo numerico
	\[
	\begin{cases}
		z_0 = z_0 + \delta_0,\\
		z_{j + 1} = z_j + \Step(\phi_\Step(t_j,z_j) + \delta_j)\text{ per }j \geq 0.
	\end{cases}
	\]
	con $(\Perturbation_j)_{j \in N}$. Chiamiamo $(\Perturbation_j)_{j \in N}$ \Define{perturbazione}[di un metodo numerico per un problema di Cauchy][perturbazione] o \Define{errore}[di un metodo numerico per un problema di Cauchy][errore] e il secondo metodo numerico \Define{metodo numerico perturbato}[numerico perturbato per un problema di Cauchy][metodo] del primo.
\end{Definition}
\begin{Definition}
	Dati un problema di Cauchy su un intervallo $I$ definito da
	\[
	\begin{cases}
		y' = f(t,y),\\
		y(t_0) = y_0,
	\end{cases}
	\]
	una discretizzazione uniforme di $I$ di passo $\Step \in \RealPositive$ e $N$ nodi, il metodo numerico a un passo
	\[
	\begin{cases}
		y_0 = y_0,\\
		y_{j + 1} = y_j + \Step\phi_\Step(t_j,y_j)\text{ per }j \geq 0,
	\end{cases}
	\]
	e il metodo numerico perturbato
	\[
	\begin{cases}
		z_0 = z_0 + \delta_0,\\
		z_{j + 1} = z_j + \Step(\phi_\Step(t_j,z_j) + \delta_j)\text{ per }j \geq 0.
	\end{cases}
	\]
	Il metodo numerico originale si definisce \Define{zero-stabile}[numerico zero-stabile][metodo] quando, per $\Step$ sufficicentemente piccolo, esistono $\epsilon > 0$ e $C \geq 0$ tali che, per ogni $j \in N$ abbiamo
	\begin{itemize}
		\item $\Norm{\Perturbation_j} \leq \epsilon$,
		\item $\Norm{z_j - y_j} \leq C \epsilon$.
	\end{itemize}
\end{Definition}
\begin{Lemma}
	\TheoremName{Lemma di Gronwall discreto} Siano
	\begin{itemize}
		\item $N \in \NaturalExtended$;
		\item $(f_j)_{j \in N}, (g_j)_{j \in N}, (h_j)_{j \in N} \in \mathbb{R}^n$;
		\item $\ForAll{j \in N}{\And{g_j \geq 0}{h_j \geq 0}}$;
		\item $(g_j)_{j \in N}$ non decrescente;
	\end{itemize}
	Se $\begin{cases} f_0 \leq g_0,\\f_{j + 1} \leq f_j \leq g_j + \sum_{j = 0}^i h_j f_j\text{ per }j \in N,\end{cases}$ allora abbiamo anche $f_j \leq g_j \Exponential{\int_{j = 0}^ih_j}$.
\end{Lemma}
\begin{Theorem}
	Siano dati un problema di Cauchy su un intervallo $I$ definito da
	\[
	\begin{cases}
		y' = f(t,y),\\
		y(t_0) = y_0,
	\end{cases}
	\]
	una discretizzazione uniforme di $I$ di passo $\Step \in \RealPositive$ e $N$ nodi e il metodo numerico a un passo
	\[
	\begin{cases}
		y_0 = y_0,\\
		y_{j + 1} = y_j + \Step\phi_\Step(t_j,y_j)\text{ per }j \geq 0.
	\end{cases}
	\]
	Se $I$ \`e limitato e $\phi_\Step(t,y)$ \`e $L$-lipschitizana ($L \in \RealPositive$) in $y$, allora il metodo numerico \`e zero-stabile.
\end{Theorem}
\Proof Sia il metodo numerico perturbato
\[
\begin{cases}
	z_0 = z_0 + \delta_0,\\
	z_{j + 1} = z_j + \Step(\phi_\Step(t_j,z_j) + \delta_j)\text{ per }i > 0.
\end{cases}
\]
\par Supponiamo le perturbazioni limitate da $\epsilon > 0$: $\ForAll{t \in N}{\Perturbation_j \leq \epsilon}$
\par Definiamo la funzione $w(t) = z(t) - y(t)$. Abbiamo
\[
w_0  = z_0 - y_0 = \Perturbation_0.
w_{j + 1} = z_{j + 1} - y_{j + 1}\text{ per }j \geq 0.\\
\]
\par Abbiamo, per ogni $j \in N$, $w_{j + 1} = w_j + \Step (\Perturbation_j \phi_\Step(t_j,z_j) - \phi_\Step(t_j,y_j)) = w_0 \Step \sum_{i = 0}^j(\phi(t_j,z_j) - \phi(t_j,y_j)) + \Step \sum_{i = 0}^j \Perturbation_j$, da cui
$\Norm{w_{j + 1}} \leq \Norm{w_0} + \Step L \sum_{i = 0}^j \Norm{w_j} + \Step \sum_{i = 0}^j \Perturbation_j \leq \epsilon(j + 1)\Step + \Step L \sum_{i = 0}^j \Norm{w_j}$.
\par Per il lemma di Gronwall discreto, abbiamo inoltre $\Norm{w_{j + 1}} \leq \epsilon(j + 1)\Step \Exponential{(j + 1)\Step L} \leq \epsilon(\max{I} + 1) \Step \Exponential{(\max{I} + 1)\Step L}$, da cui deduciamo la zero-stabilit\`a del metodo numerico. \EndProof
\begin{Theorem}
	\TheoremName{Teorema di Lax-Richtmyer, o di equivalenza (semplificato)}[di Lax-Richtmyer][teorema]
	Siano dati un problema di Cauchy su un intervallo $I$ definito da
	\[
	\begin{cases}
		y' = f(t,y),\\
		y(t_0) = y_0,
	\end{cases}
	\]
	una discretizzazione uniforme di $I$ limitato di passo $\Step \in \RealPositive$ e $N$ nodi e il metodo numerico a un passo
	\[
	\begin{cases}
		y_0 = y_0,\\
		y_{j + 1} = y_j + \Step\phi_\Step(t_j,y_j)\text{ per }j \geq 0.
	\end{cases}
	\]
	Se $\phi$ \`e lipschitziana e il metodo numerico \`e consistente di ordine $p$ e zero-stabile, allora esso \`e anche convergente di ordine\footnote{Vale anche il viceversa, ma qui non lo dimostriamo. Inoltre anche l'implicazione enunciata \`e valida sotto ipotesi pi\`u generali.} $p$.
\end{Theorem}
\Proof Per la consistenza di ordine $p$, abbiamo
\[
\begin{cases}
	y(t_0) = y_0,\\
	y(t_{j + 1}) = y(t_j) + \Step\phi_\Step(t_j,y(t_j)) + \Step\LocalTruncationError_{j + 1}\text{ per }j \geq 0,
\end{cases}
\]
dove $\IsBigO{(\Norm{\LocalTruncationError_j})_{j \in N}}{\Norm{h}^p}$. Esiste dunque $C \in \RealPositive$ tale che, per $\Norm{h}$ sufficientemente piccolo, $\Norm{\LocalTruncationError_j} \leq C \Norm{h}^p$.
\par Interpretiamo la successione $(\tau_{j + 1})_{j \in N}$ come una perturbazione del metodo numerico in esame. Allora, per la zero-stabilit\`a, per $h$ sufficientemente piccolo, abbiamo, per opportuna costante $D \in \RealPositive$, $\Norm{y(t_j) - y_j} \leq D C \Norm{h}^p$: il metodo numerico converge con ordine $p$. \EndProof
\begin{Definition}
	Chiamiamo \Define{problema test di Dahlquist}[test di Dalquist][problema] di parametro $\lambda$ il problema di Cauchy definito su $\RealPositive$ dalle equazioni
	\[
	\begin{cases}
		y' = \DahlquistParameter y,\\
		y(0) = 1,
	\end{cases}
	\]
	per qualche $\DahlquistParameter \in \mathbb{C}$.
\end{Definition}
\begin{Theorem}
	Sia dato un metodo numerico a un passo $\Step$:
	\[
	\begin{cases}
		y_0 = y_0,\\
		y_{j + 1} = y_j + \Step\phi_\Step(t_j,y_j)\text{ per }j \geq 0.
	\end{cases}
	\]
	Siano $(\Step_1,\DahlquistParameter_1), (\Step_2,\DahlquistParameter_2) \in \RealPositive \times \mathbb{C}$ tali che $\Step_1\DahlquistParameter_1 = \Step_2\DahlquistParameter_2$. Il metodo numerico genera una successione $(y_j^{(1)})_{j \in \mathbb{N}}$ infinitesima quando applicato con passo $\Step = \Step_1$ al problema test di Dahlquist con parametro $\DahlquistParameter = \DahlquistParameter_1$ se e solo se lo stesso metodo numerico genera una successione $(y_j^{(2)})_{j \in \mathbb{N}}$ infinitesima quando applicato con passo $\Step = \Step_2$ al problema test di Dahlquist con parametro $\DahlquistParameter = \DahlquistParameter_2$.
\end{Theorem}
\Proof Supponiamo il metodo generi $(y_j^{(1)})_{j \in \mathbb{N}}$ infinitesima per $\Step = \Step_1$ e $\DahlquistParameter = \DahlquistParameter_1$. Abbiamo $\DahlquistParameter_2 = \frac{\Step_1}{\Step_2} \DahlquistParameter_1$. OMESSA
\begin{Definition}
	Dato un metodo numerico per un problema di Cauchy, se esiste $\StabilityFunction: \mathbb{C} \rightarrow \Cod{y}$ la successione generata per il problema test di Dalhquist di parametro $\DahlquistParameter$ \`e della forma $y_j = \StabilityFunction(z)^j$, per ogni $j \in \mathbb{N}$, allora $\StabilityFunction$ si chiama \Define{funzione di stabilit\`a}[di stabilit\`a][funzione].
\end{Definition}
\begin{Definition}
	Chiamiamo \Define{regione di stabilit\`a di un metodo numerico per un problema di Cauchy}[di stabilit\`a di un metodo numerico per un problema di Cauchy][regione] l'insieme $\StabilityRegion$ di tutti i $z \in \mathbb{C}$ della forma $z = \Step \DahlquistParameter$ ($(\Step,\DahlquistParameter) \in \RealPositive \times \mathbb{C}$) tale che il metodo numerico con passo $\Step$ generi una successione infinitesima\footnote{\label{MetodiNumericiPerEquazioniDifferenzialiOrdinarie_Nota_RegioneDiStabilita}Alcuni autori si accontentano di richiedere una successione limitata.} per il problema test di Dahlquist di parametro $\DahlquistParameter$.
\end{Definition}
\begin{Theorem}
	Se un metodo numerico per un problema di Cauchy ammette funzione di stabilit\`a $\StabilityFunction(z)$, allora $\StabilityRegion = \lbrace z \in \mathbb{C} | \AbsoluteValue{\StabilityFunction(z)} < 1 \rbrace$ \`e la sua regione di stabilit\`a\footnote{Nel caso si modifichi la definizione di regione di stabilit\`a secondo quanto riportato dalla nota \ref{MetodiNumericiPerEquazioniDifferenzialiOrdinarie_Nota_RegioneDiStabilita} avremo invece $\StabilityRegion = \lbrace z \in \mathbb{C} | \AbsoluteValue{\StabilityFunction(z)} \leq 1 \rbrace$. Numericamente, la definizione che abbiamo scelto per la regione di stabilit\`a \`e generalmente preferibile, dato che i punti di frontiera della regione di stabilit\`a possono dar luogo a successioni che escono dalla regione (e dunque non sono pi\`u limitate) a causa di errori numerici.}.
\end{Theorem}
\Proof Segue dal fatto che una serie geometrica converge se e solo se il modulo della sua ragione \`e minore di $1$. \EndProof
\begin{Definition}
	Un metodo numerico che ammette una regione di stabilit\`a $\StabilityRegion$ tale che
\[
	\ForAll{z \in \mathbb{C}}{\Implies{\RealPart{z} < 0}{z \in \StabilityRegion}}
\]
si dice \Define{$A$-stabile}[numerico $A$-stabile per un problema di Cauchy][metodo].
\end{Definition}
\begin{Definition}
	Un metodo numerico che ammette una funzione di stabilit\`a $\StabilityFunction(z)$ tale che $\lim_{z \rightarrow - \infty} \StabilityFunction(z) = 0$ si dice \Define{$L$-stabile}[numerico $L$-stabile per un problema di Cauchy][metodo].
\end{Definition}
\begin{Definition}
	Dato $\alpha \in \left [ 0,\frac{\pi}{4} \right ]$, un metodo numerico con regione di stabilit\`a $\StabilityRegion$ si dice \Define{$A(\alpha)$-stabile}[numerico $A(\alpha)$-stabile per un problema di Cauchy][metodo] ($\alpha \in (-\pi;+\pi]$) quando $\lbrace z \in \mathbb{C} | \Argument{-z} \in (-\alpha,\alpha) \rbrace \subseteq \StabilityRegion$.
\end{Definition}
\begin{Lemma}
	Sia un'equazione differenziale $y' = f(t,y)$ definita su un intervallo $I$.
	Definiamo $\bar{y}: I \times \Cod{y}$ che associa a $(t,y_0) \in I \times \Cod{y}$ il valore $y(t)$ dove $y$ \`e la soluzione del problema di Cauchy
	\[
	\begin{cases}
		y' = f(t,y),\\
		y(t_0) = y_0.
	\end{cases}
	\]
	Supponiamo che l'applicazione parziale $\bar{y}_t(y_0)$ sia di classe $\CClass{2}$ e fissiamo una perturbazione $\Perturbation \in \mathbb{R}$.
	Abbiamo, per $\AbsoluteValue{t - t_0}$ sufficientemente piccolo, $\bar{y}(t,y_0 + \Perturbation) = y(t,y_0) + \left ( 1 + (t - t_0) \PartialDerivative{f}{y}(t_0,y_0) \right ) \Perturbation + \BigO{\AbsoluteValue{t - t_0}^2 + \Perturbation^2}$.
\end{Lemma}
\Proof Abbiamo
\[
	\bar{y}(t,y_0) = y_0 + \int_{t_0}^t f(\tau,y(\tau,y_0)) d\tau,	
\]
da cui, derivando rispetto a $y_0$,
\begin{align*}
	\PartialDerivative{\bar{y}}{y_0}(t,y_0)
	&= 1 + \int_{t_0}^t \PartialDerivative{f}{y_0} (\tau,\bar{y}(\tau,y_0)) d\tau,\\
	&= 1 + \int_{t_0}^t \PartialDerivative{f}{\bar{y}} (\tau,\bar{y}(\tau,y_0)) \PartialDerivative{\bar{y}}{y}(\tau,y_0) d\tau,\\
	&= 1 + \int_{t_0}^t \left (\PartialDerivative{f}{\bar{y}} (t_0,\bar{y}(t_0,y_0)) \PartialDerivative{\bar{y}}{y_0}(t_0,y_0) + \Littleo{\tau - t_0} \right ) d\tau,\\
	&= 1 + (t - t_0)\PartialDerivative{f}{\bar{y}} (t_0,\bar{y}(t_0,y_0)) \PartialDerivative{\bar{y}}{y_0}(t_0,y_0) + \Littleo{\AbsoluteValue{t - t_0}^2}.
\end{align*}
\par Ne deduciamo, supponendo $\AbsoluteValue{t - t_0}$ sufficientemente piccolo affinch\'e la serie $\sum_{k \in \mathbb{N}}(t - t_0)\PartialDerivative{f}{y}(t_0,y_0)$ converga,
\begin{align*}
	\PartialDerivative{\bar{y}}{y_0}(t,y_0)
	&= \left (1 - (t - t_0)\PartialDerivative{f}{\bar{y}}(t_0,y_0) \right )^{-1} \left ( 1 + \Littleo{\AbsoluteValue{t - t_0}^2} \right ),\\
	&= \left ( \sum_{k \in \mathbb{N}}\left ( (t - t_0)\PartialDerivative{f}{y}(t_0,y_0) \right )^k \right ) \left (1 + \Littleo{\AbsoluteValue{t - t_0}^2} \right ),\\
	&= \left ( 1 + (t - t_0)\PartialDerivative{f}{\bar{y}}(t_0,y_0) + \Littleo{\left ( (t - t_0)\PartialDerivative{f}{\bar{y}}(t_0,y_0) \right )^2} \right ) \left ( 1 + \Littleo{\AbsoluteValue{t - t_0}^2} \right ),\\
	&= (1 + (t - t_0)\PartialDerivative{f}{\bar{y}}(t_0,y_0) + \Littleo{(t - t_0)^2})(1 + \Littleo{\AbsoluteValue{t - t_0}^2}),\\
	&= 1 + (t - t_0)\PartialDerivative{f}{\bar{y}}(t_0,y_0) + \Littleo{(t - t_0)^2}).\\
\end{align*}
\par Sostituendo l'espressione appena trovata per $\PartialDerivative{\bar{y}}{y}(t,y_0)$ nello sviluppo di Taylor al primo ordine
\[
	\bar{y}(t,y_0 + \Perturbation) = \bar{y}(t,y_0) + \PartialDerivative{\bar{y}}{y_0}(t,y_0) \Perturbation + \BigO{\Perturbation^2},
\]
e osservando che $\IsLittleo{\delta\Littleo{\AbsoluteValue{t - t_0}^2}}{\AbsoluteValue{t - t_0}^2 + \Perturbation^2}$ concludiamo la dimostrazione. \EndProof
\begin{Lemma}
	Sia il metodo numerico a un passo $\Step$
	\[
	\begin{cases}
		y_0 = y_0,\\
		y_{j + 1} = y_j + \Step\phi(t_j,y_j)\text{ per }j \geq 0,
	\end{cases}
	\]
	per il problema di Cauchy su un intervallo $I$
	\[
	\begin{cases}
		y' = f(t,y),\\
		y(t_0) = y_0,
	\end{cases}
	\]
	con $\phi_\Step(t,y): \Dom{f} \rightarrow \Cod{f}$ e ordine di consistenza $\ConsistencyOrder$. Esiste $\xi \in [t_0,t_0 + \Step]$ tale che $y(t_0 + \Step) = y_1 + C y^{(\ConsistencyOrder + 1)}(\xi) \Step^{\ConsistencyOrder + 1} + \BigO{\Step^{\ConsistencyOrder + 2}}$, dove $C$ \`e una costante dipendente esclusivamente dal metodo numerico. %Probabilmente l'intervallo di xi puo essere ridotto alla sua parte interna
\end{Lemma}
\begin{Corollary}
	Sia il metodo numerico a un passo $\Step$
	\[
	\begin{cases}
		y_0 = y_0,\\
		y_{j + 1} = y_j + \Step\phi(t_j,y_j)\text{ per }j \geq 0,
	\end{cases}
	\]
	per il problema di Cauchy su un intervallo $I$
	\[
	\begin{cases}
		y' = f(t,y),\\
		y(t_0) = y_0,
	\end{cases}
	\]
	con $\phi_\Step(t,y): \Dom{f} \rightarrow \Cod{f}$ e ordine di consistenza $\ConsistencyOrder$. Abbiamo $y(t_0 + \Step) = y_1 + C y^{(\ConsistencyOrder + 1)}(t_0) \Step^{\ConsistencyOrder + 1} + \BigO{\Step^{\ConsistencyOrder + 2}}$, dove $C$ \`e una costante dipendente esclusivamente dal metodo numerico.
\end{Corollary}
\Proof Segue direttamente dal lemma precedente osservando che $y^{(\ConsistencyOrder + 1)}(\xi) - y^{(\ConsistencyOrder + 1)}(t_0) = \BigO{\Step}$. \EndProof
\begin{Theorem}
	\TheoremName{Estrapolazione di Richardson}
	Sia il metodo numerico a un passo $\Step$
	\[
	\begin{cases}
		y_0 = y_0,\\
		y_{j + 1} = y_j + \Step\phi(t_j,y_j)\text{ per }j \geq 0,
	\end{cases}
	\]
	per il problema di Cauchy su un intervallo $I$
	\[
	\begin{cases}
		y' = f(t,y),\\
		y(t_0) = y_0,
	\end{cases}
	\]
	con $\phi_\Step(t,y): \Dom{f} \rightarrow \Cod{f}$ e ordine di consistenza $\ConsistencyOrder$. Sia inoltre $w_2 = y_0 + 2 \Step\phi(t_0,y_0)$. Abbiamo
	\[
		y(t_0 + 2 \Step) - y_2 = \frac{y_2 - w_2}{2^\ConsistencyOrder - 1} + \BigO{\Step^{p + 2}}.
	\]
\end{Theorem}
\Proof Abbiamo, per i risultati precedenti e per opportuna costante $C$,
\[
	y(t_0 + \Step) - y_1 = Cy^{(\ConsistencyOrder + 1)}(t_0) \Step^{\ConsistencyOrder + 1} + \BigO{\Step^{\ConsistencyOrder + 2}}.
\]
\par Definiamo il problema di Cauchy
\[
\begin{cases}
	\tilde{y}' = f(t,\tilde{y}),\\
	\tilde{y}(t_0 + \Step) = y_1.
\end{cases}
\]
Considerando il metodo numerico
\[
\begin{cases}
	\tilde{y}_1 = y_1,\\
	\tilde{y}_{j + 1} = \tilde{y}_j + \Step\phi(t_j,\tilde{y}_j)\text{ per }j > 0
\end{cases}
\]
e posto $t_1 = t_0 + \Step$, deduciamo, per $\Step$ sufficientemente piccolo e trascurando tutto ci\`o che \`e $\BigO{\Step^{\ConsistencyOrder + 2}}$,
\begin{align*}
	y(t_0 + 2\Step) - y_2
	&= y(t_0 + 2\Step) - \tilde{y}(t_0 + 2\Step) + \tilde{y}(t_0 + 2\Step) - y_2,\\
	&= \left ( 1 + \Step \PartialDerivative{f}{y}(t_1,y(t_1)) \right ) (Cy^{(\ConsistencyOrder + 1)}(t_0) \Step^{p + 1}) + C y^{(\ConsistencyOrder + 1)}(t_1) \Step^{\ConsistencyOrder + 1},\\
	&= C (y^{(\ConsistencyOrder + 1)}(t_0) + y^{(\ConsistencyOrder + 1)}(t_1))\Step^{\ConsistencyOrder + 1},\\
	&= C (2 y^{(\ConsistencyOrder + 1)}(t_0) + \BigO{\Step}) \Step^{\ConsistencyOrder + 1},\\
	&= 2 C y^{(\ConsistencyOrder + 1)}(t_0) \Step^{\ConsistencyOrder + 1}.
\end{align*}
E dunque
\[
	y(t_0 + 2\Step) - y_2 = 2 C y^{(\ConsistencyOrder + 1)}(t_0) \Step^{\ConsistencyOrder + 1} + \BigO{\Step^{\ConsistencyOrder + 2}}.
\]
Abbiamo inoltre
\[
	y(t_0 + 2\Step) - w_2 = Cy^{(\ConsistencyOrder + 1)}(t_0) (2\Step)^{\ConsistencyOrder + 1} + \BigO{\Step^{\ConsistencyOrder + 2}}.
\]
Ora
\begin{align*}
	y_2 - w_2
	&= y(t_0 + 2\Step) - w_2 - (y(t_0 + 2\Step) - y_2),\\
	&= Cy^{(\ConsistencyOrder + 1)}(t_0) (2\Step)^{\ConsistencyOrder + 1} + \BigO{\Step^{\ConsistencyOrder + 2}} - 2 C y^{(\ConsistencyOrder + 1)}(t_0) \Step^{\ConsistencyOrder + 1} - \BigO{\Step^{\ConsistencyOrder + 2}},\\
	&= 2Cy^{(\ConsistencyOrder + 1)}(t_0)\Step^{\ConsistencyOrder + 1}(2^\ConsistencyOrder - 1) + \BigO{\Step^{\ConsistencyOrder + 2}}.
\end{align*}
\par Ne segue la tesi. \EndProof
\begin{Corollary}
	Con le notazioni del teorema precedente, consideriamo i metodi numerici di passo $2 \Step$
	\[
	\begin{cases}
		w_0 = y_0,\\
		w_{2j} = w_{2j - 2} + 2 \Step\phi(t_{2j - 2},w_{2j - 2})\text{ per }j \geq 0.
	\end{cases}
	\]
	e
	\[
	\begin{cases}
		z_0 = y_0,\\
		z_{2j} = y_{2j} + \frac{y_{2j} - w_{2j}}{2^\ConsistencyOrder - 1}\text{ per }j \geq 0.
	\end{cases}
	\]
	$(z_j)_{j \in \left \lfloor \frac{N}{2} \right \rfloor}$ \`e un metodo numerico di ordine di consistenza $p + 1$.
\end{Corollary}
\Proof Segue direttamente dall'estrapolazione di Richardson osservando che il risultato non dipende dallo specifico punto iniziale $y_0$. \EndProof
\begin{Definition}
\label{MetodiNumericiPerEquazioniDifferenzialiOrdinarie_DefinizioneMetodoNumericoAUnPassoAdattivo}
	Dati un problema di Cauchy su un intervallo $I$
	\[
	\begin{cases}
		y' = f(t,y),\\
		y(t_0) = y_0,
	\end{cases}
	\]
	e una discretizzazione uniforme di $I$ di passo $\Step \in \RealPositive$, chiamiamo \Define{metodo a un passo adattivo}[a un passo adattivo][metodo] un metodo numerico che segue lo schema seguente:
	\begin{itemize}
		\item si fissa una tolleranza $\epsilon > 0$;
		\item si fissa un passo iniziale $\Step$;
		\item si fissa un metodo numerico a un passo $\Step$;
		\item si effettuano due passi del metodo fissato e si stima l'errore usando l'estrapolazione di Richardson;
		\item se l'errore \`e minore della tolleranza $\epsilon$, raddoppio il passo $\Step$ finch\'e la stima dell'errore rimane sotto la tollerenza; se invece \`e maggiore di $\epsilon$ dimezzo $\Step$ sino a quando la stima dell'errore scende sotto $\epsilon$;
		\item si applica il metodo numerico col nuovo passo $\Step$, ripetendo ogni tanto la stima dell'errore secondo l'estrapolazione di Richardson e aggiustando il passo $\Step$ secondo le istruzioni del punto precedente.
	\end{itemize}
\end{Definition}
\subsection{Metodi di Runge-Kutta.}
\label{MetodiNumericiPerEquazioniDifferenzialiOrdinarie_MetodiDiRungeKutta}
\begin{Definition}
\label{MetodiNumericiPerEquazioniDifferenzialiOrdinarie_Definizione_MetodoDiRungeKutta}
	Siano dati un problema di Cauchy su un intervallo $I$
	\[
	\begin{cases}
		y' = f(t,y),\\
		y(t_0) = y_0,
	\end{cases}
	\]
	e una discretizzazione uniforme di $I$ di passo $\Step \in \RealPositive$ ed $N \in \NaturalExtended$ nodi. Siano inoltre
	\begin{itemize}
		\item $s \in \mathbb{N}$;
		\item $A = (a_{i,j})_{(i,j) \in \NaturalShift{s} \times \NaturalShift{s}} \in \mathbb{R}^{\NaturalShift{s} \times \NaturalShift{s}}$;
		\item $b, c \in \mathbb{R}^s$.
	\end{itemize}
	Chiamiamo \Define{metodo di Runge-Kutta ad $s$ stadi}[di Runge-Kutta ad $s$-stadi][metodo] il metodo numerico definito da
	\[
	\begin{cases}
		y_0 = y_0,\\
		Y_i^{(j + 1)} = y_j + \Step\sum_{k \in \NaturalShift{s}} a_{i,k}f(t_j + c_k\Step,Y_k^{(j + 1)})\text{ per }i \in \NaturalShift{s}\text{ e }j \in N,\\
		y_{j + 1} = y_j + \Step\sum_{k \in \NaturalShift{s}} b_k f(t_j + c_k\Step,Y_k^{(j + 1)})\text{ per }j \in N.
	\end{cases}
	\]
	Rappresenteremo spesso un metodo di Runge-Kutta tramite una tabella del tipo
	\[
	\begin{array}{c|c}
		c	&	A\\
		\hline
			&	b
	\end{array},
	\]
	detta \Define{\tableau\ di Butcher}[di Butcher][tableau].
\end{Definition}
\begin{Definition}
	Diciamo che un metodo di Runge-Kutta \`e \Define{banale}[di Runge-Kutta banale][metodo] quando ha $0$ stadi.
\end{Definition}
\begin{Theorem}
	Con le notazioni della definizione precendente, se $A$ \`e strettamente triangolare inferiore, allora il metodo di Runge-Kutta \`e esplicito.
\end{Theorem}
\Proof Se $A$ \`e strettamente triangolare inferiore, abbiamo allora, per ogni $(i,j) \in \NaturalShift{s} \times N$,
$Y_i^{(j + 1)} = y_j + \Step\sum_{k \in \NaturalShift{s}} a_{i,k}f(t_j + c_k\Step,Y_k^{(j + 1)}) = y_j + \Step\sum_{k \in \NaturalShift{i - 1}} a_{i,k}f(t_j + c_k\Step,Y_k^{(j + 1)})$. Quindi tutti gli elementi della famiglia $(Y_i^{(j + 1)})_{i \in \NaturalShift{s}}$ sono calcolabili in ordine direttamente, conoscendo solo gli elementi precedenti. \EndProof
\begin{Corollary}
	Con le notazioni del teorema precedente, se $A$ \`e simile a una matrice triangolare inferiore per mezzo di una matrice di permutazione, allora il metodo di Runge-Kutta \`e esplicito.
\end{Corollary}
\Proof Segue immediatamente dal teorema precedente, dopo riordino delle componenti di $(Y_i^{(j + 1)})_{i \in \NaturalShift{s}}$ ($(i,j) \in \NaturalShift{s} \times N$). \EndProof
\par Nel seguito, assumeremo sempre, con le notazioni della definizione \ref{MetodiNumericiPerEquazioniDifferenzialiOrdinarie_Definizione_MetodoDiRungeKutta}:
\begin{itemize}
	\item $\ForAll{i \in \NaturalShift{s}}{\sum_{k \in \NaturalShift{s}} a_{i,k} = c_k}$;
	\item $\sum_{k \in \NaturalShift{s}} b_k = 1$.
\end{itemize}
\par Quest'assunto \`e giustificato dal teorema seguente, che ci consente di applicare un metodo numerico ad un'equazione differenziale autonoma ottenuta aggiungendo la variabile indipendente dalle funzioni incognite integrando esattamente il tempo.
\begin{Theorem}
	Con le notazioni della definizione \ref{MetodiNumericiPerEquazioniDifferenzialiOrdinarie_Definizione_MetodoDiRungeKutta}, se abbiamo
	\begin{itemize}
		\item $\ForAll{i \in \NaturalShift{s}}{\sum_{k \in \NaturalShift{s}} a_{i,k} = c_k}$;
		\item $\sum_{k \in \NaturalShift{s}} b_k = 1$;
	\end{itemize}
	allora il problema di Cauchy
	\[
	\begin{cases}
		y' = 1,\\
		y(0) = 0,
	\end{cases}
	\]
	\`e risolto esattamente nei nodi $(t_j)_{j \in N}$ negli stadi intermedi.
\end{Theorem}
\Proof Abbiamo
\[
\begin{cases}
	Y_i^{(j + 1)} = y_j + \Step\sum_{k \in \NaturalShift{s}} a_{i,k}\text{ per }i \in \NaturalShift{s}\text{ e }j \in N,\\
	y_{j + 1} = y_j + \Step\sum_{k \in \NaturalShift{s}} b_k\text{ per }j \in N,
\end{cases}
\]
da cui
\[
\begin{cases}
	t_j + c_i \Step = t_j + \Step\sum_{k \in \NaturalShift{s}} a_{i,k}\text{ per }i \in \NaturalShift{s}\text{ e }j \in N,\\
	t_j + \Step = t_j + \Step\sum_{k \in \NaturalShift{s}} b_k\text{ per }j \in N,
\end{cases}
\]
da cui la tesi. \EndProof
\begin{Theorem}
	Un metodo di Runge-Kutta
	\[
	\begin{cases}
		y_0 = y_0,\\
		y_{j + 1} = y_j + \Step\phi_\Step(t_j,y_j)\text{ per }i > 0,
	\end{cases}
	\]
	per un problema di Cauchy
	\[
	\begin{cases}
		y' = f(t,y),\\
		y(t_0) = y_0,
	\end{cases}
	\]
	con $f$ lipschitziana \`e convergente se e solo se \`e consistente.
\end{Theorem}
\Proof Segue dal teorema di Lax-Richmyer, osservando che la lipschitzianit\`a di $f$ implica quella di $\phi$. \EndProof
\begin{Definition}
	Chiamiamo \Define{serie di Neumann}[di Neumann][serie] una serie $\sum_{n \in \mathbb{N}} A^n$, dove $A$ \`e un operatore qualsiasi per cui abbiano un senso le operazioni di somma e composizione.
\end{Definition}
\begin{Lemma}
	Sia $\Matrix$ una matrice quadrata tale che $\SpectralRadius{\Matrix} < 1$. Allora
	\begin{itemize}
		\item $\Identity - \Matrix$ \`e invertibile;
		\item la serie di Neumann $\sum_{n \in \mathbb{N}} \Matrix^n$ converge;
		\item $(\Identity - \Matrix)^{-1} = \sum_{n \in \mathbb{N}} \Matrix^n$.
	\end{itemize}
\end{Lemma}
\Proof Poich\'e $\SpectralRadius{\Matrix} < 1$, $1$ non \`e autovalore di $\Matrix$ e dunque $\Identity - \Matrix$ \`e invertibile.
\par Esiste una norma matriciale $\Norm{\cdot}$ tale che, per opportuna costante $a \in (\SpectralRadius{\Matrix},1)$, $\SpectralRadius{\Matrix} \leq \Norm{\Matrix} \leq a$. Dunque la serie $\sum_{n \in \mathbb{N}} \Matrix^n$ \`e assolutamente convergente per confronto con la serie geometrica di ragione $a$.
\par Abbiamo infine $\lim_{N \rightarrow \infty} \Norm{\Identity - (\Identity - \Matrix)\sum_{n \in N} \Matrix^n} = \lim_{N \rightarrow \infty} \Norm{\Matrix^N} \leq \lim_{N \rightarrow \infty} \Norm{\Matrix}^N \leq \lim_{N \rightarrow \infty} a^N = 0$. \EndProof
\begin{Theorem}
	Sia un metodo di Runge-Kutta a $\StagesNumber$ stadi e consistente di ordine $\ConsistencyOrder$ definito dal \tableau\ di Butcher
	\[
	\begin{array}{c|c}
		c	&	A\\
		\hline
			&	b
	\end{array}.
	\]
	e siano inoltre
	\begin{itemize}
		\item $e \in \mathbb{R}^\StagesNumber$ tale che $\ForAll{i \in \NaturalShift{\StagesNumber}}{e_i = 1}$;
		\item $C = (c_i\KroneckerDelta{i}{j})_{(i,j) \in \NaturalShift{\StagesNumber}\times\NaturalShift{\StagesNumber}}$;
		\item $(l,k) \in \mathbb{N}^2$ tali che $l > 0$, $k \geq 0$ e $1 \leq l + k \leq \ConsistencyOrder$.
	\end{itemize}
	Allora
	\begin{itemize}
		\item vale l'identit\`a $\Transposed{b}A^kC^{l - 1} e = \frac{(l - 1)!}{(l + k)!}$;
		\item se il metodo di Runge-Kutta \`e esplicito, deve essere $\ConsistencyOrder \leq \StagesNumber$.
	\end{itemize}
\end{Theorem}
\Proof Fissiamo $l \in \mathbb{N}$ tale che $1 \leq l \leq \ConsistencyOrder$ e consideriamo il problema di Cauchy definito su $\RealPositive$
\[
\begin{cases}
	y' = y + t^{l - 1},\\
	y(0) = 0.
\end{cases}
\]
\par Esso ha soluzione $\int_0^t e^{t -\tau}\tau^{l - 1}d\tau \in \CClass{\infty}$.
\par Applichiamo un passo del metodo di Runge-Kutta:
\[
\begin{cases}
	Y_i = \Step\sum_{k \in \NaturalShift{\StagesNumber}} A_{i,k}(Y_k + (\Step c_k)^{l - 1})\text{ per }i \in \NaturalShift{s},\\
	y_1 = \Step\sum_{k \in \NaturalShift{\StagesNumber}} b_k(Y_k + (\Step c_k)^{l - 1}).
\end{cases}
\]
\par Posto $Y = (Y_i)_{i \in \NaturalShift{s}}$, la prima equazione \`e equivalente a $Y = \Step A Y + \Step^l A C^{l - 1} e$ cio\`e a $(\Identity - \Step A) Y = \Step^l A C^{l - 1} e$ e dunque, per il lemma precedente assumendo $h$ sufficientemente piccolo da garantire $\SpectralRadius{A} < 1$, a $Y = \Step^{l} \left( \sum_{k \in \mathbb{N}} \Step^k A^k \right ) A C^{l - 1} e$.
\par Per l'errore locale di troncamento $\LocalTruncationError$ abbiamo, utilizzando la seconda equazione,
\begin{align*}
	\Step\LocalTruncationError = y(h) - y_1 &=y(h) - \Step \Transposed{b} Y - \Transposed{b} \Step^l C^{l - 1} e,\\
	&= y(h) - \Step \Transposed{b}\Step^l \left ( \sum_{k \in \mathbb{N}} \Step^k A^k \right ) A C^{l - 1} e - \Transposed{b} \Step^l C^{l - 1}e,\\
	&= y(h) - \Transposed{b}\Step^l \left ( \left ( \sum_{k \in \mathbb{N}} \Step^k A^k \right ) \Step A + \Identity \right ) C^{l - 1}e,\\
	&= y(h) - \Transposed{b}\Step^l \left ( \left ( \sum_{k \in \mathbb{N}} \Step^{n + 1} A^{n + 1} \right ) + \Identity \right ) C^{l - 1}e,\\
	&= y(h) - \Transposed{b}\Step^l \left ( \sum_{k \in \mathbb{N}} \Step^k A^k \right ) C^{l - 1}e.
\end{align*}
\par Consideriamo lo sviluppo di Taylor di $y(h) = \sum_{k = 0}^\infty \frac{h^k y^{(k)}(0)}{n!}$. Al variare di $k$ in $\mathbb{N}$ abbiamo
\[
y^{(k)} = \begin{cases}
y^{(k - 1)} + \frac{(l - 1)!}{(l - k)!}t^{l - k}&\text{ se }k \in l + 1,\\
y^{(k - 1)}&\text{ altrimenti},
\end{cases}
\]
da cui
\[
y^{(k)}(0) = \begin{cases}
0&\text{ se }k \in l,\\
(l - 1)!&\text{ altrimenti}.
\end{cases}
\]
\par Abbiamo dunque $y(h) = \sum_{k \in \mathbb{N}} \frac{h^{l + k} (l - 1)!}{(l + k)!}$. Ne deduciamo, per l'errore locale di troncamento,
\begin{align*}
	\Step\LocalTruncationError &= y(h) - \Transposed{b}\Step^l \left ( \sum_{k \in \mathbb{N}} \Step^k A^k \right ) C^{l - 1}e,\\
	&= \sum_{k \in \mathbb{N}} \frac{h^{l + k} (l - 1)!}{(l + k)!} - \Transposed{b}\Step^l \left ( \sum_{k \in \mathbb{N}} \Step^k A^k \right ) C^{l - 1}e,\\
	&= \sum_{k \in \mathbb{N}} h^{l + k} \left ( \frac{(l - 1)!}{(l + k)!} - \Transposed{b} A^k C^{l - 1}e \right ).
\end{align*}
\par L'identit\`a $\Transposed{b}A^kC^{l - 1} e = \frac{(l - 1)!}{(l + k)!}$ per ogni $(l,k) \in \mathbb{N}^2$ tale che $l > 0$ e $1 \leq l + k \leq \ConsistencyOrder$ segue dalla consistenza di ordine $\ConsistencyOrder$ del metodo.
\par Infine, poniamo $l = 1$ e consideriamo il coefficiente $\gamma$ di $\Step^\StagesNumber$ nello sviluppo di Taylor di $\LocalTruncationError$: esso \`e $\gamma = \frac{1}{(\StagesNumber + 1)!} - \Transposed{b}A^{\StagesNumber}e$. Se il metodo \`e esplicito, allora $A$ \`e simile a una matrice strettamente triangolare inferiore per mezzo di una matrice di permutazione e dunque $A^{\StagesNumber} = 0$, da cui $\gamma = \frac{1}{(\StagesNumber + 1)!} \neq 0$ e quindi $\StagesNumber \geq \ConsistencyOrder$. \EndProof
\par Per un metodo di Runge-Kutta di $\StagesNumber \in \NaturalShift{10}$ stadi esplicito, si dimostra che il massimo ordine di convergenza possibile \`e dato dalla seguente tabella \ref{RungeKutta_OrdiniDiConvergenza}.
\begin{table}[h]
	\centering
	\begin{tabular}{|c|cccccccccc|}
		\hline
		Numero di stadi&1&2&3&4&5&6&7&8&9&10\\
		\hline
		Ordine massimo di convergenza&1&2&3&4&4&5&6&6&7&7\\
		\hline
	\end{tabular}
	\caption{Massimi ordini di convergenza di un metodo di Runge-Kutta esplicito al variare del numero di stadi in $\NaturalShift{10}$.}
	\label{RungeKutta_OrdiniDiConvergenza}
\end{table}
\par Nella pratice si tende a preferire metodi di Runge-Kutta a $4$ stadi, dato che essi possono arrivare ad ordine di convergenza $4$ mentre i metodi con un numero superiore di stadi aumentano s\'i l'ordine di convergenza, ma troppo lentamente rispetto alla complessit\`a del calcolo di stadi ulteriori. Si usano anche metodi di Runge-Kutta con un numero minore di $4$ stadi in quei casi in cui la soluzione del problema di Cauchy non \`e sufficientemente regolare per il raggiungimento dell'ordine di convergenza $4$.
\begin{Definition}
	Un metodo di Runge-Kutta esplicito che ha numero di stadi $\StagesNumber$ uguale al suo ordine di consistenza $\ConsistencyOrder$ si dice \Define{ottimale}[di Runge-Kutta ottimale][metodo].
\end{Definition}
\begin{Theorem}
	Sia un metodo di Runge-Kutta a $\StagesNumber$ stadi definito dal \tableau\ di Butcher
	\[
	\begin{array}{c|c}
		c	&	A\\
		\hline
			&	b
	\end{array}.
	\]
	Il metodo ha funzione di stabilit\`a $\StabilityFunction(z) = 1 + z \Transposed{b} (\Identity - z A)^{-1} e$, con $e \in \mathbb{R}^\StagesNumber$ tale che $\ForAll{i \in \NaturalShift{\StagesNumber}}{e_i = 1}$.
\end{Theorem}
\Proof Applichiamo il metodo di Runge-Kutta al problema test di Dahlquist di parametro $\DahlquistParameter$: fissato $j \in \mathbb{N}$ abbiamo
\[
\begin{cases}
	Y_i = y_j + \Step\sum_{k \in \NaturalShift{s}} a_{i,k}\DahlquistParameter Y_k)\text{ per }i \in \NaturalShift{s},\\
	y_{j + 1} = y_j + \Step\sum_{k \in \NaturalShift{s}} b_k \DahlquistParameter Y_k.
\end{cases}
\]
Definendo $Y = (Y_i)_{i \in \NaturalShift{s}}$ abbiamo
\[
\begin{cases}
	Y = y_j e + \Step \DahlquistParameter A Y\text{ per }i \in \NaturalShift{s},\\
	y_{j + 1} = y_j + \Step \DahlquistParameter \Transposed{b_k} Y.
\end{cases}
\]
\par Ponendo $z = \Step \DahlquistParameter$, dalla prima equazione deduciamo, assumendo che $z^{-1}$ non sia autovalore di $A$ (\`e sempre possibile scegliere $\Step$ in modo da evitarlo\footnote{Se $z^{-1}$ fosse autovalore di $A$ il sistema lineare che definisce $Y$ sarebbe singolare e il metodo di Runge-Kutta non sarebbe applicabile.}) $Y = y_j (\Identity - zA)^{-1}e$. Dunque $y_{j + 1} = y_j + z \Transposed{b} y_j (\Identity - zA)^{-1}e = (1 + z \Transposed{b}(\Identity - zA)^{-1}) y_j$. \EndProof
\begin{Corollary}
	Con le notazioni del teorema precedente, abbiamo $\StabilityRegion = \lbrace z \in \mathbb{C} | \AbsoluteValue{1 + z\Transposed{b}(\Identity - z A)^{-1}e} < 1 \rbrace$.
\end{Corollary}
\Proof Segue direttmante dal teorema precedente. \EndProof
\begin{Theorem}
	Sia un metodo di Runge-Kutta esplicito a $\StagesNumber$ stadi e consistente di ordine $\ConsistencyOrder$ definito dal \tableau\ di Butcher
	\[
	\begin{array}{c|c}
		c	&	A\\
		\hline
			&	b
	\end{array}.
	\]
	e sia inoltre $e \in \mathbb{R}^\StagesNumber$ tale che $\ForAll{i \in \NaturalShift{\StagesNumber}}{e_i = 1}$.
	Il metodo ha funzione di stabilit\`a $\StabilityFunction(z) = \sum_{j = 0}^\ConsistencyOrder \frac{z^\ConsistencyOrder}{\ConsistencyOrder !} + \sum_{j = \ConsistencyOrder + 1}^\StagesNumber z^j \Transposed{b} A^{j - 1} e$.
\end{Theorem}
\Proof Per l'esplicit\`a del metodo, $A$ \`e nilpotente di ordine $\StagesNumber$, dunque $\ForAll{z \in \mathbb{C}}{\SpectralRadius{zA} = 0}$. Abbiamo allora $(\Identity - z A)^{-1} = \sum_{j = 0}^{\StagesNumber - 1} z^jA^j$.
\par Il metodo ha funzione di stabilit\`a
\begin{align*}
	\StabilityFunction(z) &= 1 + z \Transposed{b} \left ( \sum_{j = 0}^{\StagesNumber - 1} z^jA^j \right ) e,\\
	&= 1 + \left ( \sum_{j = 0}^{\StagesNumber - 1} z^{j + 1} \Transposed{b} A^j e \right ),\\
	&= \sum_{j = 0}^\ConsistencyOrder \frac{z^\ConsistencyOrder}{\ConsistencyOrder !} + \sum_{j = \ConsistencyOrder + 1}^\StagesNumber z^j \Transposed{b} A^{j - 1} e.\text{ \EndProof}
\end{align*}
\begin{Corollary}
	Tutti i metodi di Runge-Kutta espliciti ottimali di ordine di consistenza $\ConsistencyOrder$ hanno la stessa funzione di stabilit\`a.
\end{Corollary}
\Proof Qualsiasi metodo di Runge-Kutta esplicito ottimale di ordine di consistenza $\ConsistencyOrder$ ha funzione di stabilit\`a $\StabilityFunction(z) = \sum_{j = 0}^\ConsistencyOrder \frac{z^\ConsistencyOrder}{\ConsistencyOrder !}$. \EndProof
\begin{figure}
	\includegraphics[width=0.8\textwidth]{Metodi_numerici_per_equazioni_differenziali_ordinarie/Immagine_RegioneDiStabilitaMetodiEsplicitiOttimali.png}
	\centering
	\caption{Regione di stabilit\`a dei metodi di Runge-Kutta espliciti ottimali.}
\end{figure}
\begin{Corollary}
	Sia un metodo di Runge-Kutta esplicito a $\StagesNumber > 0$ stadi e consistente di ordine $\ConsistencyOrder$ definito dal \tableau\ di Butcher
	\[
	\begin{array}{c|c}
		c	&	A\\
		\hline
			&	b
	\end{array}.
	\]
	Il metodo non \`e
	\begin{itemize}
		\item $A$-stabile;
		\item $A(\alpha)$-stabile ($\alpha \in (-\pi,\pi]$);
		\item $L$-stabile.
	\end{itemize}
\end{Corollary}
\Proof Per la non banalit\`a del metodo di Runge-Kutta, deve essere $\ConsistencyOrder > 0$, quindi $\StabilityFunction$ \`e un polinomio di grado almento $1$. Pertanto $\lim_{\AbsoluteValue{z} \rightarrow + \infty} \AbsoluteValue{\StabilityFunction(z)} = + \infty$.
\par Ora,
\begin{itemize}
	\item il metodo non \`e $A$-stabile o $A(\alpha)$-stabile ($\alpha \in (-\pi,\pi]$)  perch\'e la regione di stabilit\`a \`e limitata;
	\item il metodo non \`e $L$-stabile perch\'e $\lim_{z \rightarrow - \infty} \AbsoluteValue{\StabilityFunction(z)} \neq 0$. \EndProof
\end{itemize}
\begin{Definition}
\label{MetodiNumericiPerEquazioniDifferenzialiOrdinarie_Definizione_MetodoDiRungeKutta_Embedded}
	Siano dati un problema di Cauchy su un intervallo $I$
	\[
	\begin{cases}
		y' = f(t,y),\\
		y(t_0) = y_0,
	\end{cases}
	\]
	e una discretizzazione uniforme di $I$ di passo $\Step \in \RealPositive$ ed $N \in \NaturalExtended$ nodi. Siano inoltre due metodi Runge-Kutta di ordine $p$ e $q < p$ (tipicamente $q = p - 1$) definiti rispettivamente dai \tableau
	\[
		\begin{array}{c|c}
		c	&	A\\
		\hline
			&	b
		\end{array},
		\begin{array}{c|c}
		c	&	A,\\
		\hline
			&	b'
		\end{array},
	\]
	che riscriviamo pi\`u compattamente
	\[
		\begin{array}{c|c}
		c	&	A\\
		\hline
			&	b\\
		\hline
			&	b'
		\end{array},
	\]
	dove
	\begin{itemize}
		\item $s \in \mathbb{N}$;
		\item $A = (a_{i,j})_{(i,j) \in \NaturalShift{s} \times \NaturalShift{s}} \in \mathbb{R}^{\NaturalShift{s} \times \NaturalShift{s}}$;
		\item $b, b', c \in \mathbb{R}^s$.
	\end{itemize}
	Chiamiamo \Define{metodo di Runge-Kutta \embedded}[di Runge-Kutta \embedded][metodo] il metodo numerico a passo adattivo definito dallo schema seguente
	\begin{itemize}
		\item si fissa una tolleranza $\epsilon > 0$;
		\item si fissa un passo iniziale $\Step$;
		\item si effettuano due passi del metodo di ordine $p$ e si stima l'errore utilizzando il metodo di ordine $q$ ripercorrendo i passaggi della dimostrazione del teorema di estrapolazione di Richardson;
		\item se l'errore \`e minore della tolleranza $\epsilon$, si raddoppia il passo $\Step$ finch\'e la stima dell'errore rimane sotto la tolleranza; se invece \`e maggiore di $\epsilon$ si dimezza $\Step$ sino a quando la stima dell'errore scende sotto $\epsilon$;
		\item si applica il metodo numerico di ordine $p$ col nuovo passo $\Step$, ripetendo ogni tanto la stima dell'errore e aggiustando il passo $\Step$ secondo le istruzioni del punto precedente.
	\end{itemize}
\end{Definition}
\input{Metodi_numerici_per_equazioni_differenziali_ordinarie/MetodoDiEuleroEsplicito.tex}
\input{Metodi_numerici_per_equazioni_differenziali_ordinarie/MetodoDiEuleroImplicito.tex}
\input{Metodi_numerici_per_equazioni_differenziali_ordinarie/MetodoDeiTrapezi.tex}
\input{Metodi_numerici_per_equazioni_differenziali_ordinarie/MetodoDiHeun.tex}



	\chapter{Elementi di meccanica celeste}
	\section{Introduzione.}
\label{ElementiDiMeccanicaCeleste_CenniDiAstronomia}
\begin{Definition}
	La \Define{meccanica celeste}[celeste][meccanica] \`e lo studio del moto dei corpi celesti, i quali comprendono, nel sistema solare,
	\begin{itemize}
		\item i \Define{pianeti}[pianeta] (in ordine crescente di distanza dal Sole: Mercurio, Venere, Terra, Marte, Giove, Saturno, Urano, Nettuno), che
		\begin{itemize}
			\item orbitano intorno al Sole;
			\item sono in equilibrio idrostatico (e quindi sono approssimativamente sferici);
			\item hanno attratto tutti i corpuscoli che li circondano entro un certo raggio;
		\end{itemize}
		\item i \Define{pianeti nani}[nano][pianeta] (tra cui Plutone), che
		\begin{itemize}
			\item orbitano intorno al Sole;
			\item sono in equilibrio idrostatico (e quindi sono approssimativamente sferici);
			\item non hanno attratto tutti i corpuscoli che li circondano entro un certo raggio;
		\end{itemize}
		\item i \Define{piccoli corpi}[piccolo][corpo], altri corpi intorno al Sole come asteroidi o comete.
	\end{itemize}
	La meccanica celeste studia anche il moto dei corpi in sistemi diversi dal sistema solare, ad esempio nei \Define{sistemi planetari}[planetario][sistema], dati da un pianeta e i suoi satelliti.
\end{Definition}
\begin{Definition}
	La meccanica celeste applicata all'invio di razzi e navicelle nello spazio prende il nome di \Define{astrodinamica}.
\end{Definition}
\par La meccanica celeste nasce ufficialmente nel 1687 con la pubblicazione dell'opera \textit{Philosophiae naturalis principia mathematica} di Sir Isaac Newton (1642 - 1726). Precursori di Newton sono stati
\begin{itemize}
	\item Nicola Copernico (1473 - 1543), per aver smontato la concezione secondo cui la Terra fosse al centro dell'universo;
	\item Tycho Brahe (1546-1601), per aver compiuto osservazioni di Marte che permisero a Johannes Kepler (1571 - 1630) di formulare le sue famose leggi;
	\item Galileo Galilei (1584 - 1642), per i suoi studi sulla caduta di gravi, i quali diedero a Newton l'intuizione che gli permise di formulare la sua teoria della gravitazione universale, intuizione consistente nell'osservare che la gittata di una palla di cannone dipende solo dalla velocit\`a iniziale con cui essa viene sperata e dunque nell'immaginare che la stessa palla, sparata con una velocit\`a iniziale elevatissima, avrebbe cominciato a ruotare attorno alla Terra senza atterrare mai;
	\item Robert Hooke (1635 - 1703), per aver contribuito in maniera determinante alla nascita della teoria della gravitazione universale, in forte rivalit\`a con Newton;
	\item Edmond Halley (1656 - 1742), il cui intervento permise la pubblicazione dei \text{Principia} di Newton.
\end{itemize}
\par Newton risove il problema dei $2$ corpi, ma non riusce a risolvere il problema dei $3$ corpi, che Henri Poincar\'e (1854 - 1912) e Heinrich Bruns (1848 - 1919) dimostrano essere non integrabile nella sua forma generale. Poincar\'e scopre anche il primo sistema dinamico caotico deterministico, cio\`e il primo sistema tale che piccole perturbazioni delle condizioni iniziali determinano grandi differenze della soluzione corrispondente; tali sistemi sono numericamente imprevedibili: superato un certo tempo, detto tempo di Lyapounov, gli errori diventono troppo elevati perch\'e le approssimazioni mantengano un senso. Dopo il primo incontro di Poincar\'e col caos, questo fenomeno viene riscoperto e studiato da Edward Norton Lorenz (1917 - 2008).
\par I sistemi lineari non sono caotici, ma i sistemi non lineari di solito lo sono. Esempi di sistemi caotici sono il doppio pendolo e lo stesso sistema solare.
\par Il caos pu\`o essere amplificato da alcuni fenomeni quali
\begin{itemize}
	\item le collisioni, per le quali non conosciamo generalmente la costituzione dei corpi coinvolti;
	\item gli incontri ravvicinati, cio\`e corpi che passano molto vicini ad altri corpi, turbandone il moto pur non urtandoli;
	\item le risonanze, cio\`e il ripetersi di certe configurazioni;
	\item gli effetti non gravitazionali, per esempio termini o elettromagnetici.
\end{itemize}
\par La ricerca in meccanica celeste si divide tra
\begin{itemize}
	\item ricerca pura, che studia i sistemi dinamici, il caos, le perturbazioni, metodi analitici e geometrici;
	\item ricerca applicata, che studia le determinazioni orbitali, i disegni delle orbite e progetta le missioni spaziali.
\end{itemize}
\par Collegati ad entrambi i tipi di ricerca sono i metodi numerici.
\begin{Definition}
	Chiamiamo \Define{problema degli $n$ corpi}[degli $n$ corpi][problema] ($n \in \mathbb{N}$) il problema che studia il moto di $n$ punti materiali $(\PointMass_i)_{i \in \NaturalShift{n}}$ di massa $(\Mass_i)_{i \in \NaturalShift{n}}$ rispettivamente soggetti solamente alle forze date dall'attrazione gravitazionale.
\end{Definition}

\section{Moti centrali.}
\label{ElementiDiMeccanicaCeleste_MotiCentrali}
\begin{Definition}
\label{ElementiDiMeccanicaCeleste_CampiCentrali}
	Un campo vettoriale $\VectorField: \mathbb{R}^n \rightarrow \mathbb{R}^n$ ($n \in \mathbb{N}$) si dice \Define{centrale rispetto all'origine}[vettoriale centrale rispetto all'origine][campo] quando \`e della forma $\VectorField(\vec{X}) = f(X) \frac{\vec{X}}{X}$, per qualche funzione $f: \RealPositive \rightarrow \mathbb{R}$ regolare a sufficienza. Un campo vettoriale si dice \Define{centrale}[vettoriale centrale][campo] se \`e centrale rispetto all'origine a seguito di un'oppotuna traslazione.
\end{Definition}
\par Nel seguito studieremo solo campi centrali rispetto all'origine: ogni altra situazione potr\`a essere ricondotta a questa dopo opportuna traslazione.
\begin{Definition}
	Chiamiamo \Define{moto centrale} il moto di un punto materiale in un campo di forze centrali.
\end{Definition}
\begin{Theorem}
	Un campo vettoriale centrale \`e conservativo.
\end{Theorem}
\Proof Con le notazioni della definizione \ref{ElementiDiMeccanicaCeleste_CampiCentrali}, basta scegliere come potenziale l'opposto di una primitiva di $f$. \EndProof
\begin{Theorem}
	Sia un punto materiale di massa $\Mass$ e posizione $X$ in un campo di forze centrali $\VectorField$ e sia $\vec{\AngularMoment}$ il suo momento angolare: $\vec{\AngularMoment}$ \`e un integrale primo del moto.
\end{Theorem}
\Proof Per definizione di campo di forze centrali, $\vec{X}$ e $\VectorField(\vec{X})$ sono sempre vettori paralleli, dunque il momento dell forze sul punto materiale \`e nullo e quindi $\dot{\vec{\AngularMoment}} = 0$. \EndProof
\begin{Theorem}
	Sia un punto materiale di massa $\Mass$ e posizione $X$ in un campo di forze centrali $\VectorField$ e sia $\vec{\AngularMoment}$ il suo momento angolare:
	\begin{itemize}
		\item se $\vec{\AngularMoment} = 0$, allora il punto si muove di moto rettilineo uniforme lungo una retta passante per l'origine;
		\item se $\vec{\AngularMoment} \neq 0$, allora il punto si muove interamente nel piano passante per l'origine e ortogonale ad $\vec{\AngularMoment}$.
	\end{itemize}
\end{Theorem}
\Proof Se $\vec{\AngularMoment} = 0$, allora $X$ e $\dot{\vec{X}}$ sono sempre paralleli e dunque il punto materiale si muove di moto rettilineo uniforme lungo una retta passante per l'origine.
\par Se invece $\vec{\AngularMoment} \neq 0$, $X$ e $\dot{\vec{X}}$ sono sempre ortogonali al vettore costante $\vec{\AngularMoment}$ ed apparatengono dunque sempre al piano passante per l'origine e ortogonale ad $\vec{\AngularMoment}$. \EndProof
\begin{Definition}
	Con le notazioni del teorema precedente, nell'ipotesi in cui $\vec{\AngularMoment} \neq 0$, il piano in cui giace l'orbita del punto materiale si chiama \Define{piano orbitale}[orbitale][piano].
\end{Definition}
\par  Proseguiamo la trattazione considerando un punto materiale di massa $\Mass$, posizione $X$ e con momento angolare $\vec{\AngularMoment} \neq 0$.
\begin{Theorem}
	Sia un sistema di riferimento di coordinate polari $\lbrace \ver{e_1}, \ver{e_2}, \ver{e_3} \rbrace$ tale che
\begin{itemize}
	\item $\ver{e_1}$ e $\ver{e_2}$ nel piano orbitale;
	\item $\ver{e_3}$ parrallelo e concorde con $\vec{\AngularMoment}$;
	\item $\ver{e_3} = \VectorProduct{\ver{e_1}}{\ver{e_2}}$.
\end{itemize}
	Posto $\rho = \Norm{\vec{X}}$ e $\theta$ uguale all'anomalia del vettore $\vec{X}$, definiamo i versori $\ver{e_\rho} = \frac{\vec{X}}{\rho}$ e $\ver{e_\theta} = \VectorProduct{\ver{e_3}}{\ver{e_\rho}}$. Abbiamo
	\begin{itemize}
		\item $\ver{e_\rho} = \cos(\theta) \ver{e_1} + \sin(\theta)\ver{e_2}$;
		\item $\ver{e_\theta} = - \sin(\theta) \ver{e_1} + \cos(\theta)\ver{e_2}$;
		\item $\dot{\ver{e_\rho}} = \dot{\theta} \ver{e_\theta}$;
		\item $\dot{\ver{e_\theta}} = - \dot{\theta} \ver{e_\rho}$.
	\end{itemize}
	Inoltre
	\begin{itemize}
		\item $\vec{X} = \rho \ver{e_\rho}$;
		\item $\dot{\vec{X}} = \dot{\rho} \ver{e_\rho} + \rho \dot{\theta} \ver{e_\theta}$;
		\item $\ddot{\vec{X}} = (\ddot{\rho} - \rho \dot{\theta}^2) \ver{e_\rho} + (2 \dot{\rho} \dot{\theta} + \rho \ddot{\theta}) \ver{e_\theta}$.
	\end{itemize}
\end{Theorem}
\begin{figure}
	\includegraphics[width=0.8\textwidth]{Elementi_di_meccanica_celeste/Immagine_SistemiDiRiferimentoPerUnMotoCentrale.png}
	\centering
	\caption{Sistemi di riferimento nel piano orbitale di un moto centrale con momento angolare non nullo.}
\end{figure}
\Proof Abbiamo, dalle definizioni e da calcoli diretti,
\begin{itemize}
	\item $\ver{e_\rho} = \cos(\theta) \ver{e_1} + \sin(\theta)\ver{e_2}$;
	\item $\ver{e_\theta} = \VectorProduct{\ver{e_3}}{(\cos(\theta) \ver{e_1} + \sin(\theta)\ver{e_2})} = \VectorProduct{\cos(\theta) \ver{e_3}}{\ver{e_1}} + \VectorProduct{\sin(\theta) \ver{e_3}}{\ver{e_2}} = - \sin(\theta) \ver{e_1} + \cos(\theta)\ver{e_2}$;
	\item $\dot{\ver{e_\rho}} = - \sin(\theta) \dot{\theta} \ver{e_1} + \cos{\theta} \dot{\theta} \ver{e_2} = \dot{\theta}\ver{e_\theta}$;
	\item $\dot{\ver{e_\theta}} = - \cos(\theta) \dot{\theta} \ver{e_1} - \sin(\theta) \dot{\theta} \ver{e_2} = - \dot{\theta} \ver{e_\rho}$;
	\item $\vec{X} = \rho \ver{e_\rho}$;
	\item $\dot{\vec{X}} = \dot{\rho} \ver{e_\rho} + \rho \dot{\ver{e_\rho}} = \dot{\rho} \ver{e_\rho} + \rho \dot{\theta} \ver{e_\theta}$;
	\item $\ddot{\vec{X}} = \ddot{\rho} \ver{e_\rho} + \dot{\rho} \dot{\ver{e_\rho}} + (\dot{\rho} \dot{\theta} + \rho \ddot{\theta}) \ver{e_\theta} + \rho \dot{\theta} \dot{\ver{e_\theta}} = \ddot{\rho} \ver{e_\rho} + \dot{\rho} \dot{\theta} \ver{e_\theta} + \dot{\rho} \dot{\theta} \ver{e_\theta} + \rho \ddot{\theta} \ver{e_\theta} - \rho \dot{\theta}^2 \ver{e_\rho} = (\ddot{\rho} - \rho \dot{\theta}^2) \ver{e_\rho} + (2 \dot{\rho} \dot{\theta} + \rho \ddot{\theta}) \ver{e_\theta}$. \EndProof
\end{itemize}
\begin{Theorem}
	Con le notazioni del teorema precedente, definiamo $\EffectivePotential = \Potential(\rho) + \frac{l^2}{2 \Mass \rho^2}$, dove
	\begin{itemize}
		\item $\Potential(\rho)$ \`e l'opposto di una primitiva dell'applicazione $f$ che a $\rho$ associa $\StandardScalarProduct{\VectorField(\vec{X})}{\ver{e_\rho}}$, con $\vec{X}$ vettore qualsiasi avente $\rho$ come modulo\footnote{Ricordiamo che stiamo supponendo $\VectorField$ centrale.};
		\item $l = \StandardScalarProduct{\vec{\AngularMoment}}{\ver{e_3}}$.
	\end{itemize}
	Abbiamo:
	\begin{itemize}
		\item $l$ \`e costante;
		\item $\Mass \ddot{\vec{X}} = \VectorField(\vec{X})$ equivale a $\begin{cases}\Mass \ddot{\rho} = - \TotalDerivative{}{\rho} \EffectivePotential(\rho),\\\dot{\theta} = \frac{l}{\Mass \rho^2}\end{cases}.$
	\end{itemize}
\end{Theorem}
\Proof Poich\'e $\vec{\AngularMoment}$ e ${\ver{e_3}}$ sono costanti deve essere costante anche $l$. Abbiamo
\begin{align*}
l &= \StandardScalarProduct{\vec{\AngularMoment}}{\ver{e_3}}\\
&= \StandardScalarProduct{\VectorProduct{\rho\ver{e_\rho}}{\Mass \dot{\vec{X}}}}{\ver{e_3}},\\
&= \rho \Mass \StandardScalarProduct{\VectorProduct{\ver{e_\rho}}{\dot{\rho}\ver{e_\rho} + \rho \dot{\theta}\ver{e_\theta}}}{\ver{e_3}},\\
&= \rho \Mass \rho \dot{\theta},\\
&= \Mass \rho^2 \dot{\theta},
\end{align*}
da cui $\dot{\theta} = \frac{l}{\Mass \rho^2}$.
\par L'equazione $\Mass \ddot{\vec{X}} = \VectorField(\Vec{X})$ si trasforma in coordinate polari, in $\Mass ((\ddot{\rho} - \rho \dot{\theta}^2)\ver{e_\rho} + (2\dot{\rho}\dot{\theta} + \rho\ddot{\theta})\ver{e_\theta}) = \StandardScalarProduct{\VectorField(\vec{X})}{\ver{e_\rho}} \ver{e_\rho} + \StandardScalarProduct{\VectorField(\vec{X})}{\ver{e_\theta}} \ver{e_\theta}$, equivalente al sistema
\[
\begin{cases}
\Mass(\ddot{\rho} - \rho \dot{\theta}^2) = f(\rho),\\
\Mass(2 \dot{\rho}\dot{\theta} + \rho\ddot{\theta}) = 0.
\end{cases}
\]
\par La prima equazione equivale a $\Mass \ddot{\rho} = f(p) + \frac{l^2}{\Mass \rho^3}$. Poich\'e $\TotalDerivative{}{\rho} \EffectivePotential(\rho) = f(\rho) + \frac{l^2}{2\Mass\rho^3}$, abbiamo $\Mass \ddot{\rho} = - \TotalDerivative{}{\rho} \EffectivePotential(\rho)$. \EndProof
\begin{Corollary}
	L'applicazione $\theta$ \`e monotona.
\end{Corollary}
\Proof $\dot{\theta}$ mantiene sempre lo stesso segno di $l$, che \`e costante. \EndProof
\begin{Definition}
	Dato un moto descritto in coordinate polari $(\rho,\theta)$, definiamo
	\begin{itemize}
		\item \Define{spostamento areolare}[areolare][spostamento] la quantit\`a
	$\ArealDisplacement(t) =
	\int_{\theta(0)}^{\theta(t)} \int_0^{\rho(t)} \rho d\rho d\theta =
	\frac{1}{2} \int_{\theta(0)}^{\theta(t)} (\rho(t))^2 d\theta$;
		\item \Define{velocit\`a areolare}[areolare][velocit\`a] la quantit\`a $\ArealVelocity(t) = \dot{\ArealDisplacement}$.
	\end{itemize}
\end{Definition}
\begin{Theorem}
	La velocit\`a areolare $\ArealVelocity$ di un moto centrale \`e
	\begin{itemize}
		\item $\ArealVelocity(t) = \frac{1}{2} (\rho(t))^2\dot{\theta}$;
		\item costante \TheoremName{seconda legge di Keplero}[seconda di Keplero][legge]. 
	\end{itemize}
\end{Theorem}
\begin{figure}
	\includegraphics[width=0.8\textwidth]{Elementi_di_meccanica_celeste/Immagine_MotiCentrali_SecondaLeggeDiKeplero.png}
	\centering
	\caption{Illustrazione della seconda legge di Keplero.}
\end{figure}
\Proof Abbiamo
$\ArealVelocity(t) =
\TotalDerivative{}{t} \frac{1}{2} \int_{\theta(0)}^{\theta(t)} (\rho(t))^2 d\theta =
\frac{1}{2} \TotalDerivative{}{t} \int_0^t (\rho(t))^2 \dot{\theta} d\tau =
\frac{1}{2} (\rho(t))^2 \dot{\theta} =
\frac{l^2}{2\Mass}$. \EndProof

\section{Problema di Keplero.}
\label{ElementiDiMeccanicaCeleste_ProblemaDiKeplero}
Assumiamo nel seguito di aver fissato un sistema di riferimento inerziale.
\begin{Definition}
	Chiamiamo \Define{problema di Keplero}[di Keplero][problema] il problema dello studio del moto di un punto materiale di massa $\ReducedMass$ in un campo gravitazionale generato da una massa $\TotalMass$ vincolata nell'origine del sistema.
\end{Definition}
\begin{figure}
	\includegraphics[width=0.8\textwidth]{Elementi_di_meccanica_celeste/Immagine_Keplero.png}
	\centering
	\caption{Problema di Keplero.}
\end{figure}
\begin{Theorem}
	Il moto del punto materiale del problema di Keplero giace in un piano ortogonale al momento angolare del punto materiale stesso.
\end{Theorem}
\Proof Segue direttamente dal fatto che la forza che definisce il problema di Keplero \`e centrale. \EndProof
\par Nel seguito suporremo fissato un sistema di riferimento inerziale $(\vec{e_x},\vec{e_y},\vec{e_z})$ tale che $\ver{e_x}$ e $\ver{e_y}$ giaciano nel piano orbitale del problema di Keplero Utilizzermo inoltre un secondo sistema di riferimento $(\ver{e_r},\ver{e_\theta},\ver{e_z})$, definito da
\begin{itemize}
	\item $\ver{e_r} = \frac{\vec{r}}{r}$;
	\item $\ver{e_\theta} = \ver{e_z} \times \ver{e_r}$.
\end{itemize}
\begin{Definition}
	Dato un punto materiale $\PointMass$ di massa $\Mass$, chiamiamo \Define{momento angolare per unit\`a di massa}[angolare per unit\`a di massa][momento] il vettore $\vec{\AngularMomentPerUnitMass} = \frac{\AngularMoment}{\Mass}$, dove $\AngularMoment$ \`e il momento angolare del punto $\PointMass$.
\end{Definition}
\begin{Theorem}
	Il momento angolare per unit\`a di massa di un punto materiale $\PointMass$ in posizione $\vec{r}$ in moto nel piano dei versori $\vec{e_x}$ e $\vec{e_y}$\ \`e $\vec{\AngularMomentPerUnitMass} = r^2\dot{\theta}\ver{e_z}$.
\end{Theorem}
\Proof Denotata $\Mass$ la massa del punto materiale, abbiamo $\vec{\AngularMomentPerUnitMass} = \frac{\vec{\AngularMoment}}{\Mass} = \frac{\VectorProduct{\vec{r}}{\Mass \dot{\vec{r}}}}{\Mass} = \VectorProduct{\vec{r}}{\dot{\vec{r}}} = \VectorProduct{r \ver{e_r}}{\dot{r}\ver{e_r} + r\dot{\theta}\ver{e_\theta}} = r^2\dot{\theta}) \ver{e_z}$. \EndProof
\begin{Theorem}
	Il problema di Keplero
	\[
		\ReducedMass \ddot{\vec{r}} = - \NewtonConstant \frac{\TotalMass\ReducedMass}{r^3}\vec{r}
	\]
	\`e equivalente a
	\[
		\begin{cases}
			\ddot{r} = \frac{\AngularMomentPerUnitMass}{r^3} - \NewtonConstant \frac{\TotalMass}{r^2},\\
			2 \dot{r}\dot{\theta} + r \ddot{\theta} = 0,
		\end{cases}
	\]
	dove $\AngularMomentPerUnitMass$ \`e il momento angolare del punto materiale di massa $\ReducedMass$ e $\theta$ l'anomalia del vettore $\vec{r}$, tramite la trasformazione che trasforma la base cartesiana $(\ver{e_x},\ver{e_y},\ver{e_z})$ nella base polare $(\ver{e_r},\ver{e_\theta},\ver{e_z})$, definita da $\ver{e_r} = \frac{\vec{r}}{r}$ e $\ver{e_\theta} = \ver{e_z} \times \ver{e_r}$.
\end{Theorem}
\Proof Essendo il campo di forze di un problema di Keplero centrale,
\[
	\ReducedMass \ddot{r} = - \NewtonConstant \frac{\TotalMass\ReducedMass}{r^3}\vec{r}
\]
equivale, tramite trasformazione di cambiamento delle coordinate cartesiani in coordinate polari,
\[
	\begin{cases}
		\ReducedMass (\ddot{r} - r \dot{\theta}^2) = - \NewtonConstant \frac{\TotalMass\ReducedMass}{r^2},\\
		\ReducedMass (2\dot{r}\dot{\theta} + r \ddot{\theta}) = 0;
	\end{cases}
\]
vale a dire, dividendo per $\ReducedMass$,
\[
	\begin{cases}
		\ddot{r} - r \dot{\theta}^2 = - \NewtonConstant \frac{\TotalMass}{r^2},\\
		2\dot{r}\dot{\theta} + r \ddot{\theta} = 0.
	\end{cases}
\]
\par Utilizzando l'equazione $\AngularMomentPerUnitMass = r^2\dot{\theta}$, possiamo ricondurre la prima equazione a un'equazione nella sola incognita $r$, ottenendo infine
\[
	\begin{cases}
		\ddot{r} = \frac{\AngularMomentPerUnitMass}{r^3} - \NewtonConstant \frac{\TotalMass}{r^2},\\
		2 \dot{r}\dot{\theta} + r \ddot{\theta} = 0.\text{ \EndProof}
	\end{cases}
\]
\begin{Definition}
	\label{ElementiDiMeccanicaCeleste_VettoreDiLenz}
	Dato un problema di Keplero
	\[
		\ReducedMass \ddot{\vec{r}} = - \NewtonConstant \frac{\TotalMass\ReducedMass}{r^3}\vec{r},
	\]
	definiamo \Define{vettore di Lenz}[di Lenz][vettore] o \Define{vettore di Laplace-Runge-Lenz}[di Laplace-Runge-Lenz][vettore] il vettore
	\[
		\vec{\LenzVector} = \frac{1}{\NewtonConstant\TotalMass} \VectorProduct{\dot{\vec{r}}}{\vec{\AngularMomentPerUnitMass}} - \ver{e_r}.
	\]
\end{Definition}
\begin{Theorem}
	Il vettore di Lenz $\vec{\LenzVector}$ \`e costante.
\end{Theorem}
\Proof Con le notazioni della definizione \ref{ElementiDiMeccanicaCeleste_VettoreDiLenz}, abbiamo
\begin{align*}
	\dot{\vec{\LenzVector}}
	&= \frac{1}{\NewtonConstant \TotalMass} (\VectorProduct{\ddot{\vec{r}}}{\vec{\AngularMomentPerUnitMass}} + \VectorProduct{\dot{\vec{r}}}{\dot{\AngularMomentPerUnitMass}}) - \dot{\ver{e_r}},\\
	&= \frac{1}{\NewtonConstant \TotalMass} (\VectorProduct{\ddot{\vec{r}}}{\vec{\AngularMomentPerUnitMass}}) - \dot{\ver{e_r}},\\
	&= - \frac{1}{\NewtonConstant \TotalMass} (\frac{\NewtonConstant\TotalMass}{r^3}\VectorProduct{\vec{r}}{\vec{\AngularMomentPerUnitMass}}) - \dot{\ver{e_r}},\\
	&= - r^{-2} \VectorProduct{\ver{e_r}}{\vec{\AngularMomentPerUnitMass}}) - \dot{\ver{e_r}},\\
	&= - r^{-2} r^2 \dot{\theta} \VectorProduct{\ver{e_r}}{\ver{e_z}}) - \dot{\ver{e_r}},\\
	&= - \dot{\theta} \VectorProduct{\ver{e_r}}{\ver{e_z}}) - \dot{\ver{e_r}},\\
	&= \dot{\theta} \ver{e_\theta} - \dot{\theta}\ver{e_\theta} = 0.\text{ \EndProof}
\end{align*}
\begin{Definition}
	\label{ElementiDiMeccanicaCeleste_AnomaliaVera}
	Dato un problema di Keplero
	\[
		\ReducedMass \ddot{\vec{r}} = - \NewtonConstant \frac{\TotalMass\ReducedMass}{r^3}\vec{r},
	\]
	definiamo \Define{anomalia vera}[vera][anomalia] la quantit\`a $\TrueAnomaly = \theta - \delta$, dove $\delta$ \`e l'anomalia del vettore di Lenz.
\end{Definition}
\begin{figure}
	\includegraphics[width=0.8\textwidth]{Elementi_di_meccanica_celeste/Immagine_Keplero_AnomaliaVera.png}
	\centering
	\caption{Definizione di anomalia vera (notazione della definizione \ref{ElementiDiMeccanicaCeleste_AnomaliaVera}).}
\end{figure}
\begin{Lemma}
	Con le notazioni della definizione precedente, abbiamo $\dot{\TrueAnomaly} = \dot{\theta}$.
\end{Lemma}
\Proof Abbiamo $\dot{\TrueAnomaly} = \dot{\theta} - \dot{\delta} = \dot{\theta}$. \EndProof
\begin{Theorem}
	L'orbita del problema di Keplero
	\[
		\ReducedMass \ddot{\vec{r}} = - \NewtonConstant \frac{\TotalMass\ReducedMass}{r^3}\vec{r},
	\]
	in dipendenza dell'anomalia vera $\TrueAnomaly$ \`e
	\[
		r(\TrueAnomaly) = \frac{\SemilatusRectum}{1 + \LenzVector \cos{\TrueAnomaly}},
	\]
	dove $\SemilatusRectum = \frac{\AngularMomentPerUnitMass^2}{\NewtonConstant \TotalMass}$.
\end{Theorem}
\Proof Abbiamo $\StandardScalarProduct{\ver{e_r}}{\LenzVector} = \LenzVector \cos{\TrueAnomaly}$.
\par Inoltre,
\begin{align*}
	\StandardScalarProduct{\ver{e_r}}{\LenzVector}
	&= \StandardScalarProduct{\ver{e_r}}{\frac{1}{\NewtonConstant\TotalMass}\VectorProduct{\dot{\vec{r}}}{\vec{\AngularMomentPerUnitMass}} - \ver{e_r}},\\
	&= \frac{1}{\NewtonConstant\TotalMass} \StandardScalarProduct{\ver{e_r}}{\VectorProduct{\dot{\vec{r}}}{\vec{\AngularMomentPerUnitMass}}} - 1,\\
	&= \frac{1}{\NewtonConstant\TotalMass} \StandardScalarProduct{\vec{\AngularMomentPerUnitMass}}{\VectorProduct{\ver{e_r}}{\dot{\vec{r}}}} - 1,\\
	&= \frac{1}{\NewtonConstant\TotalMass} \StandardScalarProduct{\vec{\AngularMomentPerUnitMass}}{\VectorProduct{\ver{e_r}}{\dot{r}\ver{e_r} + r\dot{\TrueAnomaly}\ver{e_\theta}}} - 1,\\
	&= \frac{r\dot{\TrueAnomaly}}{\NewtonConstant\TotalMass} \StandardScalarProduct{\vec{\AngularMomentPerUnitMass}}{\ver{e_z}} - 1,\\
	&= \frac{r\dot{\TrueAnomaly}}{\NewtonConstant\TotalMass}\AngularMomentPerUnitMass - 1,\\
	&= \frac{\AngularMomentPerUnitMass^2}{r\NewtonConstant\TotalMass} - 1.
\end{align*}
\par Dunque $\LenzVector \cos{\TrueAnomaly} = \frac{\AngularMomentPerUnitMass^2}{r\NewtonConstant\TotalMass} - 1$, equivalente a $\frac{\SemilatusRectum}{1 + \LenzVector \cos{\TrueAnomaly}}$. \EndProof
\begin{Theorem}
	L'orbita del problema di Keplero
	\[
		\ReducedMass \ddot{\vec{r}} = - \NewtonConstant \frac{\TotalMass\ReducedMass}{r^3}\vec{r}
	\]
	\`e una conica di eccentricit\`a $\LenzVector$.
\end{Theorem}
\Proof $\frac{\SemilatusRectum}{1 + \LenzVector \cos{\TrueAnomaly}}$ equivale a $r = \SemilatusRectum - r \LenzVector\cos{\TrueAnomaly}$, vale a dire, in coordinate cartesiane, $\sqrt{x^2 + y^2} = \SemilatusRectum - \LenzVector\cos{\TrueAnomaly}$, a sua volta equivalente a $x^2 + y^2 = (\SemilatusRectum - \LenzVector x)^2$. \EndProof
\begin{Corollary}
	\TheoremName{Classificazione delle orbite del problema di Keplero in base al modulo del vettore di Lenz}[delle orbite del problema di Keplero in base al modulo del vettore di Lenz][classificazione]
	L'orbita del problema di Keplero
	\[
		\ReducedMass \ddot{\vec{r}} = - \NewtonConstant \frac{\TotalMass\ReducedMass}{r^3}\vec{r}
	\]
	\`e
	\begin{itemize}
		\item un'ellisse se $0 \leq \LenzVector < 1$, e in particolare una circonferenza per $\LenzVector = 0$;
		\item una parabola se $\LenzVector = 1$;
		\item un ramo di iperbole se $\LenzVector > 1$.
	\end{itemize}
\end{Corollary}
\Proof Segue direttamente dal teorema precedente e dalla classificazione delle coniche in base all'eccentricit\`a. \EndProof
\begin{Definition}
	Data un'orbita del problema di Keplero
	\[
		r(\TrueAnomaly) = \frac{\SemilatusRectum}{1 + \LenzVector \cos{\TrueAnomaly}},
	\]
	dove
	\begin{itemize}
		\item $\SemilatusRectum$ \`e il semilato retto dell'orbita;
		\item $\LenzVector$ \`e il modulo del vettore di Lenz del problema, cio\`e l'eccentricit\`a dell'orbita,
	\end{itemize}
	chiamiamo
	\begin{itemize}
		\item \Define{pericentro} o \Define{periapside} o ancora, nel
      caso del sistema solare, \Define{perielio} il punto dell'orbita a
      distanza $\min r$;
		\item \Define{apocentro} o \Define{apoapside} o ancora, nel caso del
      sistema solare, \Define{afelio} il punto dell'orbita a distanza
      $\max r$.
	\end{itemize}
\end{Definition}
\begin{Theorem}
	Data un'orbita del problema di Keplero
	\[
		r(\TrueAnomaly) = \frac{\SemilatusRectum}{1 + \LenzVector \cos{\TrueAnomaly}},
	\]
	dove
	\begin{itemize}
		\item $\SemilatusRectum$ \`e il semilato retto dell'orbita;
		\item $\LenzVector$ \`e il modulo del vettore di Lenz del problema, cio\`e l'eccentricit\`a dell'orbita.
	\end{itemize}
	Se l'orbita \`e ellittica, allora
	\begin{itemize}
		\item il suo semiasse maggiore \`e $\SemiMajorAxis = \frac{\SemilatusRectum}{1 - \LenzVector^2}$;
		\item il suo semiasse minore \`e $\SemiMinorAxis = \SemiMajorAxis \sqrt{1 - \Eccentricity^2}$;
		\item la sua semidistanza focale \`e $\SemiFocalLength = \SemiMajorAxis \LenzVector$;
		\item $\min r = \SemiMajorAxis(1 - \LenzVector)$;
		\item $\max r = \SemiMajorAxis(1 + \LenzVector)$.
	\end{itemize}
	Se l'orbita \`e parabolica, allora
	\begin{itemize}
		\item $\min r = \frac{\SemilatusRectum}{2}$;
		\item l'equazione della parabola nel sistema di riferimento $Oxy$, dove $O$ coincide con l'origine del sistema di partenza e $x$ \`e nella direzione del vettore di Lenz $\LenzVector$, \`e $x = - \frac{y^2}{2 \SemilatusRectum} + \frac{\SemilatusRectum}{2}$.
	\end{itemize}
	Se l'orbita \`e iperbolica, allora
	\begin{itemize}
		\item $\min r = \frac{\SemilatusRectum}{1 + \LenzVector} < \frac{\SemilatusRectum}{2}$;
		\item il semiasse maggiore \`e $- \SemiMajorAxis = - \frac{\SemilatusRectum}{1 - \LenzVector^2}$;
		\item la sua semidistanza focale \`e $\SemiFocalLength = \AbsoluteValue{\SemiMajorAxis} \LenzVector$;
		\item $\cos \nu_{as} - \frac{1}{\LenzVector}$, dove $\nu_as$ \`e l'angolo formato da un asintoto con la retta la cui direzione \`e data dal vettore di Lenz (per esempio quello segnato in figura \ref{ElementiDiMeccanicaCeleste_Keplero_Iperbole}).
	\end{itemize}
\end{Theorem}
\begin{figure}
	\includegraphics[width=0.8\textwidth]{Elementi_di_meccanica_celeste/Immagine_Keplero_Ellisse.png}
	\centering
	\caption{Orbita ellittica per un problema di Keplero.}
\end{figure}
\Proof Supponiamo l'orbita ellittica. Abbiamo
\begin{itemize}
	\item $2 \SemiMajorAxis = r(0) + r(\pi) = \frac{\SemilatusRectum}{1 + \LenzVector \cos{0}} + \frac{\SemilatusRectum}{1 + \LenzVector \cos(\pi)} = \frac{\SemilatusRectum}{1 + \LenzVector} + \frac{\SemilatusRectum}{1 - \LenzVector} = \frac{2 \SemilatusRectum}{1 - \LenzVector^2}$, da cui $\SemiMajorAxis = \frac{\SemilatusRectum}{1 - \LenzVector^2}$;
	\item $\SemiFocalLength = \SemiMajorAxis \LenzVector$ e $\SemiMinorAxis = \SemiMajorAxis \sqrt{1 - \LenzVector}$ per le propriet\`a geometriche delle ellissi;
	\item $\min r = r(0) = \frac{\SemilatusRectum}{1 + \LenzVector} = \SemiMajorAxis (1 - \LenzVector)$;
	\item $\max r = r(0) = \frac{\SemilatusRectum}{1 - \LenzVector} = \SemiMajorAxis (1 + \LenzVector)$.
\end{itemize}
\begin{figure}
	\includegraphics[width=0.8\textwidth]{Elementi_di_meccanica_celeste/Immagine_Keplero_Parabola.png}
	\centering
	\caption{Orbita parabolica per un problema di Keplero.}
\end{figure}
\par Supponiamo l'orbita parabolica. Abbiamo
\begin{itemize}
	\item $\min r = \frac{\SemilatusRectum}{1 + \LenzVector} = \frac{\SemilatusRectum}{2}$;
	\item per ragioni di simmetria, l'equazione della parabola \`e della forma $x = c_1y^2 + c_2y + c_3$ per opportuni $c_1, c_2, c_3 \in \mathbb{R}$; abbiamo
\[
\begin{cases}
	- \frac{c_2}{2 c_1} = 0,\\
	c_3 = \SemilatusRectum,\\
	c_1 \SemilatusRectum^2 + c_2 \SemilatusRectum + c_3 = 0,\\
\end{cases}
\]
	e quindi $x = -\frac{y^2}{2 \SemilatusRectum} + \frac{\SemilatusRectum}{2}$.
\end{itemize}
\begin{figure}
	\label{ElementiDiMeccanicaCeleste_Keplero_Iperbole}
	\includegraphics[width=0.8\textwidth]{Elementi_di_meccanica_celeste/Immagine_Keplero_Iperbole.png}
	\centering
	\caption{Orbita iperbolica per un problema di Keplero.}
\end{figure}
\par Supponiamo l'orbita iperbolica. Abbiamo
\begin{itemize}
	\item $\min r = \frac{\SemilatusRectum}{1 + \LenzVector} < \frac{\SemilatusRectum}{2}$ poich\'e $\LenzVector > 1$;
	\item $\begin{cases}\SemiFocalLength = - \LenzVector \SemiMajorAxis,\\\SemiFocalLength =  - \SemiMajorAxis - \frac{\SemilatusRectum}{1 + \LenzVector}\end{cases}$, da cui $\SemiMajorAxis = \frac{\SemilatusRectum}{(1 - \LenzVector)^2}$;
	\item $\frac{\SemilatusRectum}{1 + \LenzVector \cos{\TrueAnomaly}} \rightarrow + \infty$ se e solo se $\cos{\TrueAnomaly} \rightarrow - \frac{1}{\LenzVector}$, dunque $\nu_{as} = \frac{1}{\LenzVector}$. \EndProof
\end{itemize}
\begin{Theorem}
	L'energia per unit\`a di massa $\EnergyPerUnitMass$ del problema di Keplero
	\[
		\ReducedMass \ddot{\vec{r}} = - \NewtonConstant \frac{\TotalMass\ReducedMass}{r^3}\vec{r}
	\]
	\`e
	\[
		\EnergyPerUnitMass =
		\begin{cases}
			0\text{, se l'orbita \`e parabolica;}\\
			- \NewtonConstant \frac{\TotalMass}{2 \SemiMajorAxis}\text{, altrimenti.}
		\end{cases}
	\]
	Nel secondo caso, $\SemiMajorAxis$ indica
	\begin{itemize}
		\item il semiasse maggiore dell'ellisse nel caso di un'orbita ellittica;
		\item l'opposto del semiasse maggiore dell'iperbole nel caso di un'orbita iperbolica.
	\end{itemize}
\end{Theorem}
\par Usiamo le notazioni del teorema \label{ElementiDiMeccanicaCeleste_VettoreDiLenz} e denotiamo con $\TrueAnomaly$ l'anomalia vera. Abbiamo
\begin{align*}
	\vec{\LenzVector}
	&= \frac{1}{\NewtonConstant\TotalMass} \VectorProduct{\dot{\vec{r}}}{\vec{\AngularMomentPerUnitMass}} - \ver{e_r},\\
	&= \frac{1}{\NewtonConstant\TotalMass} \VectorProduct{\dot{r}\ver{e_r} + r\dot{\TrueAnomaly}\ver{e_{\TrueAnomaly}}}{r^2\dot{\TrueAnomaly} \ver{e_z}} - \ver{e_r},\\
	&= \frac{1}{\NewtonConstant\TotalMass} (- r^2 \dot{r} \dot{\TrueAnomaly} \ver{e_\TrueAnomaly} + r^3 \dot{\TrueAnomaly}^2) - \ver{e_r},\\
	&= \left ( \frac{r^3\dot{\TrueAnomaly}^2}{\NewtonConstant\TotalMass} - 1 \right ) \ver{e_r} - \frac{r^2\dot{r}\dot{\TrueAnomaly}}{\NewtonConstant\TotalMass} \ver{e_\TrueAnomaly}.
\end{align*}
\par Abbiamo dimostrato che nessuna orbita del problema di Keplero passa dall'origine, cio\`e che per ogni punto dell'orbita $r \neq 0$. Abbiamo adesso
\begin{align*}
	\LenzVector^2
	&= \left ( \frac{r^3\dot{\TrueAnomaly}^2}{\NewtonConstant\TotalMass} - 1 \right )^2 + \left ( \frac{r^2\dot{r}\dot{\TrueAnomaly}}{\NewtonConstant\TotalMass} \right )^2,\\
	&= \frac{r^6\dot{\TrueAnomaly}^4}{\NewtonConstant^2\TotalMass^2} - 2 \frac{r^3\dot{\TrueAnomaly}^2}{\NewtonConstant\TotalMass} + 1 + \frac{r^4\dot{r}^2\dot{\TrueAnomaly}^2}{\NewtonConstant^2\TotalMass^2},\\
	&= 1 + 2 \frac{r^3\dot{\TrueAnomaly}^2}{\NewtonConstant^2\TotalMass^2} \left (\frac{r^3\dot{\TrueAnomaly}^2}{2} - \NewtonConstant\TotalMass + \frac{r\dot{r}^2}{2} \right ),\\
	&= 1 + 2 \frac{r^4\dot{\TrueAnomaly}^2}{\NewtonConstant^2\TotalMass^2} \left (\frac{r^2\dot{\TrueAnomaly}^2}{2} - \frac{\NewtonConstant\TotalMass}{r} + \frac{\dot{r}^2}{2} \right ),\\
	&= 1 + 2 \frac{r^4\dot{\TrueAnomaly}^2}{\NewtonConstant^2\TotalMass^2} \left (\frac{r^2\dot{\TrueAnomaly}^2 + \dot{r}}{2} - \frac{\NewtonConstant\TotalMass}{r} \right ),\\
	&= 1 + 2 \frac{r^4\dot{\TrueAnomaly}^2}{\NewtonConstant^2\TotalMass^2} \left (\frac{\Norm{\dot{\vec{r}}^2}}{2} - \frac{\NewtonConstant\TotalMass}{r} \right ),\\
	&= 1 + 2 \frac{r^4\dot{\TrueAnomaly}^2}{\NewtonConstant^2\TotalMass^2} \EnergyPerUnitMass,\\
	&= 1 + 2 \frac{\AngularMomentPerUnitMass^2}{\NewtonConstant^2\TotalMass^2} \EnergyPerUnitMass,\\
	&= 1 + 2 \frac{\SemilatusRectum}{\NewtonConstant\TotalMass} \EnergyPerUnitMass.
\end{align*}
\par Se l'orbita \`e parabolica, allora $\LenzVector = 1$ da cui $\EnergyPerUnitMass = 0$.
\par Supponiamo ora l'orbita ellittica o iperbolica. In tal caso \`e definito il semiasse maggiore $\SemiMajorAxis$ e abbiamo $\EnergyPerUnitMass = 1 + 2 \frac{\SemiMajorAxis (1 - \LenzVector^2)}{\NewtonConstant\TotalMass} \EnergyPerUnitMass$. Ne deduciamo $\EnergyPerUnitMass = \frac{(\LenzVector^2 - 1)\NewtonConstant\TotalMass}{2\SemiMajorAxis(1 - \LenzVector^2)} = - \frac{\NewtonConstant\TotalMass}{2\SemiMajorAxis}$. \EndProof
\begin{Corollary}
	\TheoremName{Classificazione delle orbite del problema di Keplero in base all'energia}[delle orbite del problema di Keplero in base all'energia][classificazione]
	L'orbita del problema di Keplero
	\[
		\ReducedMass \ddot{\vec{r}} = - \NewtonConstant \frac{\TotalMass\ReducedMass}{r^3}\vec{r}
	\]
	\`e, denotata $\EnergyPerUnitMass$ l'energia per unit\`a di massa,
	\begin{itemize}
		\item un'ellisse se $\EnergyPerUnitMass < 0$;
		\item una parabola se $\EnergyPerUnitMass = 0$;
		\item un ramo di iperbole se $\EnergyPerUnitMass > 0$.
	\end{itemize}
\end{Corollary}
\Proof Segue direttamente dal teorema precedente. \EndProof
\begin{Corollary}
	Nelle ipotesi del teorema precedente, se l'orbita \`e aperta, abbiamo
	\begin{itemize}
		\item $\lim_{r \rightarrow + \infty} \Norm{\dot{\vec{r}}} = 0$ nel caso parabolico;
		\item $\lim_{r \rightarrow + \infty} \Norm{\dot{\vec{r}}} = \sqrt{\NewtonConstant \frac{\TotalMass}{\SemiMajorAxis}}$.
	\end{itemize}
\end{Corollary}
\Proof Se l'orbita \`e parabolica, allora $\frac{\Norm{\dot{\vec{r}}}^2}{2} - \NewtonConstant \frac{\TotalMass}{r} = 0$, da cui $\lim_{r \rightarrow \infty} \Norm{\dot{\vec{r}}} = 0$.
\par Se l'orbita \`e iperbolica, allora
\[
\begin{cases}
	\EnergyPerUnitMass = \frac{\Norm{\dot{\vec{r}}}^2}{2} - \NewtonConstant \frac{\TotalMass}{r},\\
	\EnergyPerUnitMass = - \NewtonConstant \frac{\TotalMass}{2\SemiMajorAxis},
\end{cases}
\]
da cui $\lim_{r \rightarrow \infty} \Norm{\dot{\vec{r}}} = \sqrt{2\EnergyPerUnitMass} = \sqrt{\NewtonConstant\frac{\TotalMass}{\SemiMajorAxis}}$. \EndProof
\begin{Definition}
	Dato il problema di Keplero
	\[
		\ReducedMass \ddot{\vec{r}} = - \NewtonConstant \frac{\TotalMass\ReducedMass}{r^3}\vec{r},
	\]
	denotato $\AngularMomentPerUnitMass$ il momento angolare per unit\`a di massa, chiamiamo \Define{potenziale efficace}[efficace][potenziale] la quantit\`a $\EffectivePotential = \frac{\AngularMomentPerUnitMass}{2r^2} - \NewtonConstant\frac{\TotalMass}{r}$.
\end{Definition}
\begin{figure}
	\includegraphics[width=0.8\textwidth]{Elementi_di_meccanica_celeste/Immagine_Keplero_PotenzialeEfficace.png}
	\centering
	\caption{Esempio di potenziale efficace di un problema di Keplero.}
\end{figure}
\begin{Theorem}
	Nelle ipotesi della definizione precedente, l'energia per unit\`a di massa $\EnergyPerUnitMass$ \`e data da $\EnergyPerUnitMass = \frac{\dot{r}^2}{2} + \EffectivePotential$.
\end{Theorem}
\Proof Abbiamo, denotando con $\TrueAnomaly$ l'anomalia vera,
\begin{align*}
	\EnergyPerUnitMass
	&= \frac{\Norm{\dot{\vec{r}}}^2}{2} - \NewtonConstant\frac{\TotalMass}{r},\\
	&= \frac{\dot{r}^2 + \dot{r}^2\dot{\TrueAnomaly^2}}{2} - \NewtonConstant\frac{\TotalMass}{r},\\
	&= \frac{\dot{r}^2}{2} + \frac{\AngularMomentPerUnitMass^2}{2r^2} - \NewtonConstant\frac{\TotalMass}{r},\\
	&= \frac{\dot{r}^2}{2} + \frac{\AngularMomentPerUnitMass^2}{2r^2} - \NewtonConstant\frac{\TotalMass}{r}.\text{ \EndProof}
\end{align*}
\begin{Theorem}
	Se l'orbita del problema di Keplero
	\[
		\ReducedMass \ddot{\vec{r}} = - \NewtonConstant \frac{\TotalMass\ReducedMass}{r^3}\vec{r}
	\]
	\`e chiusa, allora \`e periodica.
\end{Theorem}
\Proof Poich\'e l'orbita \`e chiusa, abbiamo $\vec{r}(t + \Period) = \vec{r}(t)$ per opportuno $T \in \mathbb{R}$. Inoltre, per la conversazione del momento angolare per unit\`a di massa, abbiamo
\begin{align*}
	\VectorProduct{\vec{r}(t + \Period)}{\dot{\vec{r}}(t + \Period)}
	&= \vec{\AngularMomentPerUnitMass}(t + \Period),\\
	&= \vec{\AngularMomentPerUnitMass}(t),\\
	&= \VectorProduct{\vec{r}(t)}{\dot{\vec{r}}(t)},\\
	&= \VectorProduct{\vec{r}(t + \Period)}{\dot{\vec{r}}(t)}.
\end{align*}
\par Poich\'e inoltre, per la conservazione dell'energia, il modulo della velocit\`a in $\vec{r}(t + \Period) = \vec{r}(t)$ \`e uguale sia al tempo $t$ che al tempo $t + \Period$, deve essere $\dot{\vec{r}}(t + \Period) = \dot{\vec{r}}(t)$. \EndProof
\begin{Definition}
	Data un'orbita periodica di periodo $\Period$ di un problema di Keplero
	\[
		\ReducedMass \ddot{\vec{r}} = - \NewtonConstant \frac{\TotalMass\ReducedMass}{r^3}\vec{r},
	\]
	definiamo \Define{moto medio}[medio][moto] la quantit\`a $\MeanMotion = \frac{2\pi}{\Period}$.
\end{Definition}
\begin{Theorem}
	\TheoremName{Terza legge di Keplero}[terza di Keplero][legge] 
	Sia il problema di Keplero
	\[
		\ReducedMass \ddot{\vec{r}} = - \NewtonConstant \frac{\TotalMass\ReducedMass}{r^3}\vec{r}.
	\]
	Se l'orbita \`e ellittica con semiasse maggiore $\SemiMajorAxis$ e periodo $\Period$, allora abbiamo
	\[
		\frac{\Period^2}{\SemiMajorAxis^3} = \frac{4\pi^2}{\NewtonConstant\TotalMass},
	\]
	o, equivalentemente, denotando $\MeanMotion$ il moto medio,
	\[
		\SemiMajorAxis^3\MeanMotion^2 = \NewtonConstant\TotalMass.
	\]
\end{Theorem}
\Proof L'area dell'orbita ellittica \`e $\pi\SemiMajorAxis\SemiMinorAxis$, dove $\SemiMinorAxis$ \`e il semiasse minore dell'ellisse. La velocit\`a areolare invece \`e $\frac{r^2\dot{\TrueAnomaly}}{2} = \frac{\AngularMomentPerUnitMass}{2}$.
Abbiamo allora, denotato $\LenzVector$ il modulo del vettore di Lenz,
\begin{align*}
	\frac{\Period^2}{\SemiMajorAxis^3}
	&= \frac{4 \pi^2\SemiMajorAxis^2\SemiMinorAxis^2}{\AngularMomentPerUnitMass^2\SemiMajorAxis^3},\\
	&= \frac{4 \pi^2\SemiMinorAxis^2}{\AngularMomentPerUnitMass^2\SemiMajorAxis},\\
	&= \frac{4 \pi^2 \SemiMajorAxis^2(1 - \LenzVector^2)}{\NewtonConstant\TotalMass\SemiMajorAxis(1 - \LenzVector^2)\SemiMajorAxis},\\
	&= \frac{4 \pi^2}{\NewtonConstant\TotalMass}.\text{ \EndProof}
\end{align*}
\begin{Definition}
	Data un'orbita periodica di periodo $\Period$ di un problema di Keplero
	\[
		\ReducedMass \ddot{\vec{r}} = - \NewtonConstant \frac{\TotalMass\ReducedMass}{r^3}\vec{r},
	\]
	denotati
	\begin{itemize}
		\item $\SemiMajorAxis$ il semiasse maggiore dell'orbita ellittica;
		\item $\EnergyPerUnitMass$ l'energia per unit\`a di massa;
		\item $\MeanMotion$ il \Define{moto medio}[medio][moto],
	\end{itemize}
	abbiamo
	\[
		\MeanMotion = - \frac{2 \EnergyPerUnitMass}{\NewtonConstant^2\TotalMass^2}.
	\]
\end{Definition}
\Proof Abbiamo $\SemiMajorAxis^3 = - \left ( \frac{2 \EnergyPerUnitMass}{\NewtonConstant\TotalMass} \right )^3$ e, per la terza legge di Keplero,
\begin{align*}
	\MeanMotion
	&= \sqrt{\frac{\NewtonConstant\TotalMass}{\SemiMajorAxis^3}},\\
	&= \sqrt{\frac{ - (2 \EnergyPerUnitMass)^3 \NewtonConstant\TotalMass}{\NewtonConstant^3\TotalMass^3}},\\
	&= - \frac{2 \EnergyPerUnitMass}{\NewtonConstant^2\TotalMass^2}.\text{ \EndProof}
\end{align*}
\begin{Definition}
	Dato un problema di Keplero
	\[
		\ReducedMass \ddot{\vec{r}}
    = - \NewtonConstant \frac{\TotalMass\ReducedMass}{r^3}\vec{r},
	\]
	chiamiamo
	\begin{itemize}
		\item
    \Define{tempo di passaggio al pericentro}[di passaggio al pericentro][tempo]
      un qualsiasi tempo $\PericenterPassageTime$ tale che
      $\TrueAnomaly(\PericenterPassageTime) = 0$, dove $\TrueAnomaly$ \`e
      l'anomalia vera;
		\item se l'orbita \`e ellittica,
      fissato un tempo di passaggio al pericentro,
      \Define{anomalia media}[media][anomalia] la quantit\`a
      $\MeanAnomaly(t) = \MeanMotion(t - \PericenterPassageTime)$;
		\item se l'orbita \`e ellittica,
      con riferimento alla figura
      \ref{ElementiDiMeccanicaCeleste_Figura_AnomaliaEccentricaEllittica},
      chiamiamo l'angolo $\EccentricAnomaly$
      \Define{anomalia eccentrica ellittica}[eccentrica ellittica][anomalia]:
      \`e l'angolo individuato dalla direzione $CP$ coincidente con
      quella del vettore di Lenz $\vec{\LenzVector}$ e dalla congiungente
      il centro dell'ellisse $C$ con la proiezione $Q$ del punto $P$
      individuato dal vettore $\vec{r}$ sulla circonferenza di stesso
      centro dell'ellisse e raggio il semiasse maggiore $\SemiMajorAxis$
      dell'ellisse;
    \item se l'orbita \`e iperbolica,
      con riferimento alla figura
      \ref{ElementiDiMeccanicaCeleste_Figura_AnomaliaEccentricaIperbolica},
      denominati $D$ il pericentro dell'orbita,
      $C$ il punto d'intersezione degli asintoti dell'iperbole
      equilatera con vertice sinistro coincidente con $D$,
      $B$ la proiezione della posizione del punto materiale sul semiramo
      di iperbole equilatera dello stesso semipiano (rispetto all'asse $x$),
      chiamiamo la quantit\`a $\EccentricAnomaly$, definita come il doppio
      dell'area compresa tra i segmenti $CB$, $CD$ e
      l'orbita iperbolica divisa per il quadarato del semiasse maggiore
      $\SemiMajorAxis$, con segno positivo se l'area \`e contenuta nel
      semipiano positivo e con segno negativo se l'area \`e contenuta nel
      semipiano negativo,
      \Define{anomalia eccentrica iperbolica}[eccentrica iperbolica][anomalia]
	\end{itemize}
\end{Definition}
\par Ometteremo l'aggettivo ``ellittica'' o ``iperbolica'' quando
il tipo di anomalia eccentrica sar\`a evidente dal contesto. Inoltre,
poich\'e, nel considerare un'unica orbita, \`e definibile al pi\`u  solo
uno dei due tipi di anomalia eccentrica, possiamo utilizzare la stessa
notazione per entrambi i tipi senza rischi di confusione come abbiamo
fatto nella definizione precedente.
\begin{figure}
	\includegraphics[width=0.8\textwidth]{Elementi_di_meccanica_celeste/Immagine_Keplero_AnomaliaEccentricaEllittica.png}
	\centering
	\caption{Definizione di anomalia eccentrica ellittica.}
	\label{ElementiDiMeccanicaCeleste_Figura_AnomaliaEccentricaEllittica}
\end{figure}
\begin{figure}
	\includegraphics[width=0.8\textwidth]{Elementi_di_meccanica_celeste/Immagine_Keplero_AnomaliaEccentricaIperbolica.png}
	\centering
	\caption{Definizione di anomalia eccentrica iperbolica.}
	\label{ElementiDiMeccanicaCeleste_Figura_AnomaliaEccentricaIperbolica}
\end{figure}
\begin{Theorem}
  \label{ElementiDiMeccanicaCeleste_ThTrueEccentric}
	Con le notazioni della definizione precedente, abbiamo
	\begin{itemize}
		\item $r = \SemiMajorAxis(1 - \LenzVector \cos{\EccentricAnomaly})$;
		\item
		$
			\begin{cases}
				\cos{\TrueAnomaly}
          = \frac{\cos{\EccentricAnomaly} - \LenzVector}
              {1 - \LenzVector \cos{\EccentricAnomaly}},\\
				\sin{\TrueAnomaly}
          = \frac{(\sin{\EccentricAnomaly})\sqrt{1 - \LenzVector^2}}
            {1 - \LenzVector \cos{\EccentricAnomaly}}; 
			\end{cases}
		$
    \item $\tan \frac{\TrueAnomaly}{2}
            = \sqrt{\frac{1 + \LenzVector}{1 - \LenzVector}}
              \tan \frac{\EccentricAnomaly}{2}$;
    \item
    $
      \begin{cases}
        \cos{\EccentricAnomaly}
          = \frac{\cos{\TrueAnomaly} + \LenzVector}
              {1 + \LenzVector \cos{\TrueAnomaly}},\\
        \sin{\EccentricAnomaly}
          = \frac{\sqrt{1 - \EccentricAnomaly^2} \sin{\TrueAnomaly}}
              {1 + \LenzVector \cos{\TrueAnomaly}}.
      \end{cases}
    $
	\end{itemize}
\end{Theorem}
\Proof Abbiamo
\begin{itemize}
	\item $\Length{HB} = \SemiMajorAxis \sin{\EccentricAnomaly}$;
	\item $\Length{HA} = r \sin{\TrueAnomaly}$;
	\item $\Length{HA} = \Length{HB} \sqrt{1 - \LenzVector^2}$;
	\item $\Length{OH} = r \cos{\TrueAnomaly}$;
	\item $\Length{OH}
          = \Length{CH} - \Length{CS}
          = \SemiMajorAxis \cos{\EccentricAnomaly}
            - \SemiMajorAxis \LenzVector$.
\end{itemize}
Da cui
\[
\begin{cases}
	r \sin{\TrueAnomaly}
    = \SemiMajorAxis(\sin{\EccentricAnomaly})\sqrt{1 - \LenzVector^2},\\
	r \cos{\TrueAnomaly}
    = \SemiMajorAxis (\cos{\EccentricAnomaly} - \LenzVector).
\end{cases}
\]
\par Abbiamo allora
\begin{align*}
	r^2
	&= r^2(\sin^2{\TrueAnomaly} + \cos^2{\TrueAnomaly}),\\
	&= \SemiMajorAxis^2 (\sin^2{\EccentricAnomaly}) (1 - \LenzVector^2)
    + \SemiMajorAxis^2 (\cos^2{\EccentricAnomaly}
    - 2 \LenzVector \cos{\EccentricAnomaly} + \LenzVector^2),\\
	&= \SemiMajorAxis^2 (\sin^2{\EccentricAnomaly}
    - (\sin^2{\EccentricAnomaly})\LenzVector^2
    + \cos^2{\EccentricAnomaly} - 2 \LenzVector \cos{\EccentricAnomaly}
    + \LenzVector^2),\\
	&= \SemiMajorAxis^2 (1 + (1 - \sin^2{\EccentricAnomaly})\LenzVector^2
    - 2 \LenzVector \cos{\EccentricAnomaly}),\\
	&= \SemiMajorAxis^2 (1 + \cos^2{\EccentricAnomaly}\LenzVector^2
    - 2 \LenzVector \cos{\EccentricAnomaly}),\\
	&= \SemiMajorAxis^2 (1 - \LenzVector \cos{\EccentricAnomaly})^2.
\end{align*}
\par E quindi
\begin{itemize}
	\item $r = \SemiMajorAxis (1 - \LenzVector \cos{\EccentricAnomaly})$;
	\item
	$
		\begin{cases}
			\cos{\TrueAnomaly}
        = \frac{\SemiMajorAxis(\cos{\EccentricAnomaly} - \LenzVector)}
          {\SemiMajorAxis (1 - \LenzVector \cos{\EccentricAnomaly})}
        =  \frac{\cos{\EccentricAnomaly} - \LenzVector}
          {1 - \LenzVector \cos{\EccentricAnomaly}};\\
			\sin{\TrueAnomaly}
        = \frac{\SemiMajorAxis (\sin{\EccentricAnomaly})
            \sqrt{1 - \LenzVector^2})}
          {\SemiMajorAxis(1 - \LenzVector \cos{\EccentricAnomaly})}
        = \frac{\sin{\EccentricAnomaly}\sqrt{1 - \LenzVector^2}}
          {1 - \LenzVector \cos{\EccentricAnomaly}}. 
		\end{cases}
	$
\end{itemize}
\par Abbiamo inoltre
\begin{align*}
  \tan \frac{\TrueAnomaly}{2}
    &= \frac{\sin \TrueAnomaly}{1 + \cos \TrueAnomaly},\\
    &=  \frac{\frac{\sin{\EccentricAnomaly}\sqrt{1 - \LenzVector^2}}
          {1 - \LenzVector \cos{\EccentricAnomaly}}}
         {1 + \frac{\cos{\EccentricAnomaly} - \LenzVector}
          {1 - \LenzVector \cos{\EccentricAnomaly}}},\\
    &= \frac{\sqrt{1 - \LenzVector^2} \sin \EccentricAnomaly}
        {1 - \LenzVector \cos \EccentricAnomaly + \cos \EccentricAnomaly
          -  \LenzVector},\\
    &= \frac{\sqrt{1 - \LenzVector^2}}{1 - \LenzVector}
        \frac{\sin \EccentricAnomaly}{1 + \cos \EccentricAnomaly},\\
    &= \sqrt{\frac{1 + \LenzVector}{1 - \LenzVector}}
        \tan \frac{\EccentricAnomaly}{2}.
\end{align*}
\par Inoltre, da
\[
  \begin{cases}
	  r = \SemiMajorAxis(1 - \LenzVector \cos{\EccentricAnomaly}),\\
    r = \frac{\SemiMajorAxis(1 - \LenzVector^2)}
        {1 + \LenzVector\cos{\TrueAnomaly}},
  \end{cases}
\]
deduciamo
\begin{align*}
  \cos \EccentricAnomaly
  &= - \frac{1}{\LenzVector} \left (
    \frac{1 - \LenzVector^2}{1 + \LenzVector \cos \TrueAnomaly}
    - 1 \right ),\\
  &= \frac{1 + \LenzVector \cos \TrueAnomaly - 1 + \LenzVector^2}
      {\LenzVector(1 + \LenzVector \cos \TrueAnomaly)},\\
  &= \frac{\LenzVector + \cos \TrueAnomaly}
      {1 + \LenzVector \cos \TrueAnomaly};
\end{align*}
e inoltre
\[
	  \frac{\sqrt{1 - \LenzVector^2} \sin \EccentricAnomaly}
      {\sin \TrueAnomaly}
    = \frac{1 - \LenzVector^2}
      {1 + \LenzVector\cos{\TrueAnomaly}},
\]
da cui
\begin{align*}
  \sin \EccentricAnomaly
  &= \frac{\sin \TrueAnomaly}
      {\sqrt{1 - \LenzVector^2}}
      \frac{1 - \LenzVector^2}
      {1 + \LenzVector\cos{\TrueAnomaly}},\\
  &= \frac{\sqrt{1 - \LenzVector^2} \sin \TrueAnomaly}
      {1 + \LenzVector\cos{\TrueAnomaly}}.\text{ \EndProof}
\end{align*}
\begin{Theorem}
  \label{ElementiDiMeccanicaCeleste_ThTrueEccentricHyperbolic}
	Con le notazioni della definizione precedente, abbiamo
	\begin{itemize}
		\item $r = \SemiMajorAxis(1 - \LenzVector \cosh{\EccentricAnomaly})$;
		\item
		$
			\begin{cases}
				\cos{\TrueAnomaly}
          = \frac{\cosh{\EccentricAnomaly} - \LenzVector}
              {1 - \LenzVector \cosh{\EccentricAnomaly}},\\
				\sin{\TrueAnomaly}
          = \frac{- (\sinh{\EccentricAnomaly})\sqrt{\LenzVector^2 - 1}}
            {1 - \LenzVector \cosh{\EccentricAnomaly}}; 
			\end{cases}
		$
    \item $\tanh \frac{\TrueAnomaly}{2}
            = \sqrt{\frac{\LenzVector + 1}{\LenzVector - 1}}
              \tanh \frac{\EccentricAnomaly}{2}$;
    \item
    $
      \begin{cases}
        \cosh{\EccentricAnomaly}
          = \frac{\cos{\TrueAnomaly} + \LenzVector}
              {1 + \LenzVector \cos{\TrueAnomaly}},\\
        \sinh{\EccentricAnomaly}
          = \frac{\sqrt{\EccentricAnomaly^2 - 1} \sin{\TrueAnomaly}}
              {1 + \LenzVector \cos{\TrueAnomaly}}.
      \end{cases}
    $
	\end{itemize}
\end{Theorem}
%\Proof Abbiamo
%\begin{itemize}
%	\item $\Length{HB} = \SemiMajorAxis \sin{\EccentricAnomaly}$;
%	\item $\Length{HA} = r \sin{\TrueAnomaly}$;
%	\item $\Length{HA} = \Length{HB} \sqrt{1 - \LenzVector^2}$;
%	\item $\Length{OH} = r \cos{\TrueAnomaly}$;
%	\item $\Length{OH}
%          = \Length{CH} - \Length{CS}
%          = \SemiMajorAxis \cos{\EccentricAnomaly}
%            - \SemiMajorAxis \LenzVector$.
%\end{itemize}
%Da cui
%\[
%\begin{cases}
%	r \sin{\TrueAnomaly}
%    = \SemiMajorAxis(\sin{\EccentricAnomaly})\sqrt{1 - \LenzVector^2},\\
%	r \cos{\TrueAnomaly}
%    = \SemiMajorAxis (\cos{\EccentricAnomaly} - \LenzVector).
%\end{cases}
%\]
%\par Abbiamo allora
%\begin{align*}
%	r^2
%	&= r^2(\sin^2{\TrueAnomaly} + \cos^2{\TrueAnomaly}),\\
%	&= \SemiMajorAxis^2 (\sin^2{\EccentricAnomaly}) (1 - \LenzVector^2)
%    + \SemiMajorAxis^2 (\cos^2{\EccentricAnomaly}
%    - 2 \LenzVector \cos{\EccentricAnomaly} + \LenzVector^2),\\
%	&= \SemiMajorAxis^2 (\sin^2{\EccentricAnomaly}
%    - (\sin^2{\EccentricAnomaly})\LenzVector^2
%    + \cos^2{\EccentricAnomaly} - 2 \LenzVector \cos{\EccentricAnomaly}
%    + \LenzVector^2),\\
%	&= \SemiMajorAxis^2 (1 + (1 - \sin^2{\EccentricAnomaly})\LenzVector^2
%    - 2 \LenzVector \cos{\EccentricAnomaly}),\\
%	&= \SemiMajorAxis^2 (1 + \cos^2{\EccentricAnomaly}\LenzVector^2
%    - 2 \LenzVector \cos{\EccentricAnomaly}),\\
%	&= \SemiMajorAxis^2 (1 - \LenzVector \cos{\EccentricAnomaly})^2.
%\end{align*}
%\par E quindi
%\begin{itemize}
%	\item $r = \SemiMajorAxis (1 - \LenzVector \cos{\EccentricAnomaly})$;
%	\item
%	$
%		\begin{cases}
%			\cos{\TrueAnomaly}
%        = \frac{\SemiMajorAxis(\cos{\EccentricAnomaly} - \LenzVector)}
%          {\SemiMajorAxis (1 - \LenzVector \cos{\EccentricAnomaly})}
%        =  \frac{\cos{\EccentricAnomaly} - \LenzVector}
%          {1 - \LenzVector \cos{\EccentricAnomaly}};\\
%			\sin{\TrueAnomaly}
%        = \frac{\SemiMajorAxis (\sin{\EccentricAnomaly})
%            \sqrt{1 - \LenzVector^2})}
%          {\SemiMajorAxis(1 - \LenzVector \cos{\EccentricAnomaly})}
%        = \frac{\sin{\EccentricAnomaly}\sqrt{1 - \LenzVector^2}}
%          {1 - \LenzVector \cos{\EccentricAnomaly}}. 
%		\end{cases}
%	$
%\end{itemize}


\par Abbiamo inoltre
\begin{align*}
  \tan \frac{\TrueAnomaly}{2}
    &= \frac{\sin \TrueAnomaly}{1 + \cos \TrueAnomaly},\\
    &=  \frac{\frac{- \sinh{\EccentricAnomaly}\sqrt{\LenzVector^2 - 1}}
          {1 - \LenzVector \cosh{\EccentricAnomaly}}}
         {1 + \frac{\cosh{\EccentricAnomaly} - \LenzVector}
          {1 - \LenzVector \cosh{\EccentricAnomaly}}},\\
    &= \frac{- \sqrt{\LenzVector^2 - 1} \sinh \EccentricAnomaly}
        {1 - \LenzVector \cosh \EccentricAnomaly + \cosh \EccentricAnomaly
          -  \LenzVector},\\
    &= \frac{\sqrt{\LenzVector^2 - 1}}{\LenzVector - 1}
        \frac{\sinh \EccentricAnomaly}{1 + \cosh \EccentricAnomaly},\\
    &= \sqrt{\frac{\LenzVector + 1}{\LenzVector - 1}}
        \tanh \frac{\EccentricAnomaly}{2}.
\end{align*}
\par Inoltre, da
\[
  \begin{cases}
	  r = \SemiMajorAxis(1 - \LenzVector \cosh{\EccentricAnomaly}),\\
    r = \frac{\SemiMajorAxis(1 - \LenzVector^2)}
        {1 + \LenzVector\cos{\TrueAnomaly}},
  \end{cases}
\]
deduciamo
\begin{align*}
  \cosh \EccentricAnomaly
  &= - \frac{1}{\LenzVector} \left (
    \frac{1 - \LenzVector^2}{1 + \LenzVector \cos \TrueAnomaly}
    - 1 \right ),\\
  &= \frac{1 + \LenzVector \cos \TrueAnomaly - 1 + \LenzVector^2}
      {\LenzVector(1 + \LenzVector \cos \TrueAnomaly)},\\
  &= \frac{\LenzVector + \cos \TrueAnomaly}
      {1 + \LenzVector \cos \TrueAnomaly};
\end{align*}
e inoltre
\[
	  \frac{- \sqrt{\LenzVector^2 - 1} \sinh \EccentricAnomaly}
      {\sin \TrueAnomaly}
    = \frac{1 - \LenzVector^2}
      {1 + \LenzVector\cos{\TrueAnomaly}},
\]
da cui
\begin{align*}
  \sinh \EccentricAnomaly
  &= \frac{\sin \TrueAnomaly}
      {\sqrt{\LenzVector^2 - 1}}
      \frac{\LenzVector^2 - 1}
      {1 + \LenzVector\cos{\TrueAnomaly}},\\
  &= \frac{\sqrt{\LenzVector^2 - 1} \sin \TrueAnomaly}
      {1 + \LenzVector\cos{\TrueAnomaly}}.\text{ \EndProof}
\end{align*}
\begin{Theorem}
	\TheoremName{Formula di Keplero}[di Keplero][formula]
	Data un'orbita periodica di periodo $\Period$ di un problema di Keplero
	\[
		\ReducedMass \ddot{\vec{r}}
    = - \NewtonConstant \frac{\TotalMass\ReducedMass}{r^3}\vec{r},
	\]
	abbiamo
	\[
		\MeanAnomaly = \EccentricAnomaly
      - \LenzVector \sin{\EccentricAnomaly},
	\]
	dove
	\begin{itemize}
		\item $\MeanAnomaly$ \`e l'anomalia media;
		\item $\EccentricAnomaly$ \`e l'anomalia eccentrica;
		\item $\vec{\LenzVector}$ \`e il vettore di Lenz.
	\end{itemize}
\end{Theorem}
\Proof Con riferimento alla figura
\ref{ElementiDiMeccanicaCeleste_Figura_AnomaliaEccentricaEllittica},
denotiamo
\begin{itemize}
	\item $A_{HOA}$
    l'area del triangolo rettangolo individuato dai punti $H$, $O$, $A$;
	\item $A_{DCB}$
    l'area della porzione di circonferenza individuata dai raggi $HD$ e
    $HB$;
	\item $A_{HCQ}$
    l'area del triangolo rettangolo individuato dai punti $H$, $C$, $Q$;
	\item $A_{DHB}$
    l'area della porzione di circonferenza individuata dai segmenti $HD$
     e $HB$;
	\item $A_{DHA}$
    l'area della porzione di ellisse individuata dai segmenti $HD$ e
    $HA$;
	\item $A_{DOA}$
    l'area della porzione di ellisse individuata dai segmenti $OD$ e
    $OA$;
	\item $A$ l'area dell'ellisse.
\end{itemize}
\par Abbiamo
\begin{itemize}
	\item $A
    = \pi \SemiMajorAxis^2 \sqrt{1 - \LenzVector^2}$;
	\item $A_{HOA}
    = \frac{(\SemiMajorAxis \cos{\EccentricAnomaly} - \SemiMajorAxis \LenzVector) \cdot \sqrt{1 - \LenzVector^2} \SemiMajorAxis \sin{\EccentricAnomaly}}{2} = \SemiMajorAxis^2 \frac{(\cos{\EccentricAnomaly} - \LenzVector) \sqrt{1 - \LenzVector^2} \sin{\EccentricAnomaly}}{2}$;
	\item $A_{DCB}
    = \frac{\EccentricAnomaly \SemiMajorAxis^2}{2}$;
	\item $A_{HCQ}
    = \frac{\SemiMajorAxis^2 \cos{\EccentricAnomaly} \sin{\EccentricAnomaly}}{2}$;
	\item $A_{DHB}
    = A_{DCB} - A_{HCQ}$;
	\item $A_{DHA}
    = \sqrt{1 - \LenzVector^2} A_{DHB}$;
	\item $A_{DOA}
    = A_{DHA} + A_{HOA}$.
\end{itemize}
\par Ne deduciamo
\begin{align*}
	A_{DOA}
	&= A_{DHA} + A_{HOA},\\
	&= \sqrt{1 - \LenzVector^2} \left ( \frac{\EccentricAnomaly \SemiMajorAxis^2}{2} - \frac{\SemiMajorAxis^2 \cos{\EccentricAnomaly} \sin{\EccentricAnomaly}}{2} \right ) + \SemiMajorAxis^2 \frac{(\cos{\EccentricAnomaly} - \LenzVector) \sqrt{1 - \LenzVector^2} \sin{\EccentricAnomaly}}{2},\\
	&= \frac{\SemiMajorAxis^2\sqrt{1 - \LenzVector^2}}{2} ( \EccentricAnomaly - \cos{\EccentricAnomaly} \sin{\EccentricAnomaly} + \cos{\EccentricAnomaly} \sin{\EccentricAnomaly} - \LenzVector \sin{\EccentricAnomaly} ),\\
	&= \frac{\SemiMajorAxis^2\sqrt{1 - \LenzVector^2}}{2} ( \EccentricAnomaly - \LenzVector \sin{\EccentricAnomaly} ).
\end{align*}
Infine abbiamo, denotando $\MeanMotion$ il moto medio e tenendo conto della seconda legge di Keplero,
\begin{align*}
	\MeanAnomaly
	&= \MeanMotion (t - \PericenterPassageTime),\\
	&= \frac{2 \pi}{\Period} (t - \PericenterPassageTime),\\
	&= 2 \pi \frac{A_{DOA}}{A},\\
	&= 2 \pi \frac{\SemiMajorAxis^2\sqrt{1 - \LenzVector^2}}{2} ( \EccentricAnomaly - \LenzVector \sin{\EccentricAnomaly} ) \frac{1}{\pi \SemiMajorAxis^2 \sqrt{1 - \LenzVector^2}},\\
	&= \EccentricAnomaly - \LenzVector \sin{\EccentricAnomaly}.\text{ \EndProof}
\end{align*}
\begin{Corollary}
	Con le ipotesi e le notazioni del teorema precedente, abbiamo
	\[
		\PericenterPassageTime = \frac{\LenzVector \sin{\EccentricAnomaly(0)} - \EccentricAnomaly(0)}{\MeanMotion}
	\]
	dove
	\begin{itemize}
		\item $\PericenterPassageTime$ \`e il tempo di passaggio al pericentro;
		\item $\MeanMotion$ \`e il moto medio.
	\end{itemize}
\end{Corollary}
\Proof Da $\MeanAnomaly(0) = \MeanMotion(0 - \PericenterPassageTime)$,
deduciamo, tramite la formula di Keplero,
$\PericenterPassageTime
= - \frac{\MeanAnomaly(0)}{\MeanMotion}
= \frac{\LenzVector \sin{\EccentricAnomaly(0)} - \EccentricAnomaly(0)}{\MeanMotion}$.
\EndProof
\begin{Theorem}
	Dato un problema di Keplero
	\[
		\ReducedMass \ddot{\vec{r}}
    = - \NewtonConstant \frac{\TotalMass\ReducedMass}{r^3}\vec{r},
	\]
  se l'orbita \`e ellittica, allora fornire
  \begin{itemize}
    \item tempo di passaggio al pericentro $\PericenterPassageTime$;
    \item moto medio $\MeanMotion$;
    \item eccentricit\`a $\LenzVector$;
  \end{itemize}
  determina univocamente l'orbita.
\end{Theorem}
\Proof Da $\PericenterPassageTime$ e $\MeanMotion$ deduciamo
$\MeanAnomaly
= \MeanMotion(t - \PericenterPassageTime)$. Possiamo allora risolvere
l'equazione
\[
	\MeanAnomaly = \EccentricAnomaly
    - \LenzVector \sin{\EccentricAnomaly},
\]
data dalla formula di Keplero
\footnote{Concretamente, si usa un metodo numerico come per esempio il
metodo di Newton).}.
\par La terza legge di Keplero fornisce il semiasse maggiore
$\SemiMajorAxis
= \sqrt[3]{\frac{\NewtonConstant\TotalMass}{\MeanMotion^2}}$.
\par Si conclude tramite il teorema
\ref{ElementiDiMeccanicaCeleste_ThTrueEccentric}. \EndProof
\begin{Theorem}
	Data un'orbita periodica un problema di Keplero
	\[
		\ReducedMass \ddot{\vec{r}}
    = - \NewtonConstant \frac{\TotalMass\ReducedMass}{r^3}\vec{r},
	\]
  abbiamo
  \[
    \dot{\EccentricAnomaly} = \frac{\MeanMotion \SemiMajorAxis}{r},
  \]
  dove
	\begin{itemize}
		\item $\EccentricAnomaly$ \`e l'anomalia eccentrica;
		\item $\MeanAnomaly$ \`e l'anomalia media;
		\item $\SemiMajorAxis$ \`e il semiasse maggiore dell'orbita
      ellittica.
	\end{itemize}
\end{Theorem}
\Proof Denotiamo
\begin{itemize}
  \item $\SemilatusRectum$ il semilato retto;
  \item $\TrueAnomaly$ l'anomalia vera.
\end{itemize}
Da
\[
  \begin{cases}
    r = \frac{\SemilatusRectum}{1 + \LenzVector \cos{\TrueAnomaly}},\\
    r = \SemiMajorAxis(1 - \LenzVector \cos{\EccentricAnomaly}),
  \end{cases}
\]
deduciamo
\begin{align*}
  \frac{\SemilatusRectum}{1 + \LenzVector \cos{\TrueAnomaly}}
    &= \SemiMajorAxis(1 - \LenzVector \cos{\EccentricAnomaly}),\\
  \frac{\SemiMajorAxis(1 - \LenzVector^2)}
    {1 + \LenzVector \cos{\TrueAnomaly}}
    &= \SemiMajorAxis(1 - \LenzVector \cos{\EccentricAnomaly}),\\
  1 - \LenzVector^2
    &= (1 + \LenzVector \cos{\TrueAnomaly})
    (1 - \LenzVector\cos{\EccentricAnomaly}),\\
  1 - \LenzVector^2
    &= 1 + \LenzVector \cos{\TrueAnomaly}
    - \LenzVector \cos{\EccentricAnomaly}
    - \LenzVector^2 \cos{\EccentricAnomaly} \cos{\TrueAnomaly},\\
  0 &= \LenzVector \cos{\TrueAnomaly}
  - \LenzVector \cos{\EccentricAnomaly}
  - \LenzVector^2 \cos{\EccentricAnomaly} \cos{\TrueAnomaly}
  + \LenzVector^2.
\end{align*}
Deriviamo rispetto al tempo:
\begin{align*}
  - \LenzVector \sin{\TrueAnomaly} \dot{\TrueAnomaly}
    + \LenzVector \sin{\EccentricAnomaly} \dot{\EccentricAnomaly}
    + \LenzVector^2 \sin{\EccentricAnomaly} \cos{\TrueAnomaly}
      \dot{\EccentricAnomaly}
    + \LenzVector^2 \sin{\TrueAnomaly} \cos{\EccentricAnomaly}
      \dot{\TrueAnomaly}
    &= 0,\\
  (\LenzVector^2 \sin{\TrueAnomaly} \cos{\EccentricAnomaly}
    - \LenzVector \sin{\TrueAnomaly}) \dot{\TrueAnomaly}
    + (\LenzVector^2 \sin{\EccentricAnomaly} \cos{\TrueAnomaly}
    + \LenzVector \sin{\EccentricAnomaly}) \dot{\EccentricAnomaly}
    &= 0,
\end{align*}
da cui, denotando $\AngularMomentPerUnitMass$ il momento angolare per
unit\`a di massa,
\begin{align*}
  \dot{\EccentricAnomaly} &=
    \frac{\LenzVector \sin{\TrueAnomaly}
      - \LenzVector^2 \sin{\TrueAnomaly} \cos{\EccentricAnomaly}}
    {\LenzVector^2 \sin{\EccentricAnomaly} \cos{\TrueAnomaly}
      + \LenzVector \sin{\EccentricAnomaly}} \dot{\TrueAnomaly},\\
  &=
    \frac{\sin{\TrueAnomaly} (1 - \LenzVector \cos{\EccentricAnomaly})}
      {\sin{\EccentricAnomaly} (\LenzVector \cos{\TrueAnomaly} + 1)}
    \frac{\AngularMomentPerUnitMass}{r^2},\\
  &=
    \frac{\sin{\EccentricAnomaly} \sqrt{1 - \LenzVector^2}}
      {1 - \LenzVector \cos{\EccentricAnomaly}}
    \frac{(1 - \LenzVector \cos{\EccentricAnomaly})}
      {\sin{\EccentricAnomaly} (\LenzVector \cos{\TrueAnomaly} + 1)}
    \frac{\AngularMomentPerUnitMass}{r^2},\\
  &=
    \sqrt{1 - \LenzVector^2}
    \frac{\AngularMomentPerUnitMass}
    {(\LenzVector \cos{\TrueAnomaly} + 1) r^2},\\
  &=
    \sqrt{1 - \LenzVector^2}
    \frac{\sqrt{\NewtonConstant \TotalMass}}{r^2}
    \frac{\sqrt{\SemilatusRectum}}
      {1 + \LenzVector \cos{\TrueAnomaly}},\\
  &=
    \sqrt{1 - \LenzVector^2}
    \frac{\sqrt{\NewtonConstant \TotalMass}}{r}
    \frac{1}{\sqrt{\SemilatusRectum}},\\
  &=
    \sqrt{1 - \LenzVector^2}
    \MeanMotion \SemiMajorAxis^{\frac{3}{2}}
    \frac{1}{r}
    \frac{1}{\sqrt{\SemiMajorAxis}\sqrt{1 - \LenzVector^2}},\\
  &=
    \frac{\MeanMotion \SemiMajorAxis}{r}.\text{ \EndProof}
\end{align*}
\begin{Theorem}
	Dato un problema di Keplero
	\[
		\ReducedMass \ddot{\vec{r}}
    = - \NewtonConstant \frac{\TotalMass\ReducedMass}{r^3}\vec{r},
	\]
  se l'orbita \`e ellittica, allora, denotati
  \begin{itemize}
    \item $\MeanMotion$ l'anomalia media;
    \item $\EnergyPerUnitMass$ l'integrale dell'energia per unit\`a di
      massa;
    \item $\vec{\LenzVector}$ il vettore di Lenz;
  \end{itemize}
  abbiamo
  \begin{itemize}
    \item $\MeanMotion = \frac{\NewtonConstant^2\TotalMass^2}
    {2 \AbsoluteValue{\EnergyPerUnitMass}\sqrt{- 2 \EnergyPerUnitMass}}$;
    \item $\PericenterPassageTime$ pu\`o essere calcolato direttamente
      noto $\EccentricAnomaly(0)$ soluzione del sistema
      \[
        \begin{cases}
          \cos{\EccentricAnomaly(0)} = \frac{1}{\LenzVector}
            \left ( 1 - \frac{r(0)}{\SemiMajorAxis} \right ),\\
          \sin{\EccentricAnomaly(0)} = \frac{r(0)\dot{r}(0)}
            {\SemiMajorAxis^2 \LenzVector \MeanMotion}.
        \end{cases}
      \]
  \end{itemize}
\end{Theorem}
\Proof Per il moto medio abbiamo
\begin{align*}
  \MeanMotion
  &= \sqrt{\frac{\NewtonConstant\TotalMass}{\SemiMajorAxis^3}},\\
  &= \sqrt
    {- \frac{\NewtonConstant^4\TotalMass^4}{8 \EnergyPerUnitMass^3}},\\
  &= \frac{\NewtonConstant^2\TotalMass^2}
    {2 \AbsoluteValue{\EnergyPerUnitMass}\sqrt{- 2 \EnergyPerUnitMass}}.
\end{align*}
\par Da $r = \SemiMajorAxis(1 - \LenzVector \cos{\EccentricAnomaly(0)})$
deduciamo immediatamente
$\cos{\EccentricAnomaly(0)} = \frac{1}{\LenzVector}
\left ( 1 - \frac{r(0)}{\SemiMajorAxis} \right )$.
\par Abbiamo inoltre
 $\dot{r}
= \SemiMajorAxis \LenzVector \sin{(\EccentricAnomaly)}
\dot{\EccentricAnomaly}$, da cui
\begin{align*}
  \sin{\EccentricAnomaly(0)}
  &=\frac{\dot{r}(0)}{\SemiMajorAxis\LenzVector\dot{\EccentricAnomaly}},
  &=\frac{r(0) \dot{r}(0)}{\SemiMajorAxis^2\LenzVector\MeanMotion}.
\end{align*}
\par Possiamo allora calcolare $\EccentricAnomaly(0)$ e ottenere
$\PericenterPassageTime
= \frac{\LenzVector \sin{\EccentricAnomaly(0)} - \EccentricAnomaly(0)}
{\MeanMotion}$. \EndProof
\par Consideriamo adesso il problema di Keplero in un sistema di
riferimento in il campo gravitazionale \`e sempre centrale, ma il piano
orbitale non coincide pi\`u col piano $Oxy$.
\begin{Definition}
  Dato un problema di Keplero
	\[
		\ReducedMass \ddot{\vec{r}}
    = - \NewtonConstant \frac{\TotalMass\ReducedMass}{r^3}\vec{r},
	\]
  chiamiamo
  \begin{itemize}
    \item \Define{linea dei nodi}[dei nodi][linea] la traccia del piano
      orbitale sul piano $Oxy$;
    \item \Define{nodo ascendente}[ascendente][nodo] il punto
      d'intersezione tra l'orbita e la linea dei nodi dove il punto
      materiale ha velocit\`a positiva lungo l'asse $z$, denotato
      tipicamente $\AscendingNode$;
    \item \Define{nodo discendente}[discendente][nodo] il punto
      d'intersezione tra l'orbita e la linea dei nodi dove il punto
      materiale ha velocit\`a negativa lungo l'asse $z$, denotato
      tipicamente $\DescendingNode$;
    \item \Define{longitudine del nodo ascendente}[del nodo ascendente][longitudine]
      l'angolo $\LongitudeOfAscendingNode$ individuato dalla
      direzione positiva dell'asse $x$ al nodo ascendente misurato in
      senso positivo;
    \item \Define{argomento del pericentro}[del pericentro][argomento]
      l'angolo $\ArgumentOfPericenter$ individuato dal nodo ascendente
      al vettore di Lenz misurato in senso positivo;
    \item \Define{inclinazione} l'angolo $\Inclination$ individuato
      dall'asse $z$ al momento angolare misurato in senso positivo.
  \end{itemize}
  Chiamiamo inoltre
  \begin{itemize}
    \item \Define{elementi cartesiani}[cartesiani][elementi] le sei
    componenti dei vettori $\vec{r}$ e $\dot{\vec{r}}$;
    \item \Define{elementi kepleriani}[kepleriani][elementi]
      le sei quantit\`a scalari
      \begin{itemize}
        \item semiasse maggiore $\SemiMajorAxis$
          (si considera $\SemiMajorAxis = + \infty$ nel
          caso di orbita parabolica);
        \item eccentricit\`a $\LenzVector$;
        \item inclinazione $\Inclination$;
        \item la longitudine del nodo ascendente
          $\LongitudeOfAscendingNode$;
        \item l'argomento del pericentro $\ArgumentOfPericenter$;
        \item l'anomalia media $\MeanAnomaly$.
      \end{itemize}
  \end{itemize}
\end{Definition}
\begin{figure}
	\includegraphics[width=0.8\textwidth]{Elementi_di_meccanica_celeste/Immagine_Keplero_ElementiKepleriani.png}
	\centering
	\caption{Elementi kepleriani.}
\end{figure}
\begin{Theorem}
  Dato un problema di Keplero
	\[
		\ReducedMass \ddot{\vec{r}}
    = - \NewtonConstant \frac{\TotalMass\ReducedMass}{r^3}\vec{r},
	\]
  la corrispondenza che associa, per un fissato istante $t$, agli
  elementi cartesiani gli elementi kepleriani \`e biunivoca.
\end{Theorem}
\Proof Manteniamo le notazioni della definizione precedente.
Supponiamo noti gli elementi cartesiani.
Abbiamo
\begin{itemize}
  \item l'inclinazione \`e l'angolo compreso tra $0$ e $\pi$ tra il
    vettore
    $\vec{\AngularMomentPerUnitMass}
      = \VectorProduct{\vec{r}}{\vec{\dot{r}}}$ e il versore
    $\ver{e_z}$: \`e soluzione dell'equazione
    $\cos{\Inclination}
      =\frac{\ScalarProduct{\vec{\AngularMomentPerUnitMass}}{\ver{e_z}}}
        {\AngularMomentPerUnitMass}$;
  \item l'eccentrit\`a \`e data dal modulo del vettore di Lenz
    $\vec{\LenzVector}
      = \frac{1}
        {\NewtonConstant\TotalMass}
        \VectorProduct{\vec{r}}{\AngularMomentPerUnitMass}
        - \frac{\vec{r}}{r}$;
  \item la linea dei nodi \`e l'intersezione del piano $Oxy$ e del piano
    orbitale: il versore
    $\ver{\LongitudeOfAscendingNode}
      = \frac{\VectorProduct{\ver{e_z}}{\AngularMomentPerUnitMass}}
        {\Norm{\VectorProduct{\ver{e_z}}{\AngularMomentPerUnitMass}}}$
    giace sulla linea dei nodi: da esso, tramite prodotto scalare con
    $\ver{e_x}$, ricavo $\LongitudeOfAscendingNode$;
  \item l'argomento del pericentro \`e l'angolo compreso tra il versore
    $\ver{\LongitudeOfAscendingNode}$ e il vettore di Lenz: \`e dunque
    soluzione di
    \[
      \begin{cases}
        \cos{\LongitudeOfAscendingNode}
          = \ScalarProduct{\vec{\LenzVector}}
            {\ver{\LongitudeOfAscendingNode}},\\
        \sin{\LongitudeOfAscendingNode}
          = \Norm{\VectorProduct{\vec{\LenzVector}}
            {\ver{\LongitudeOfAscendingNode}}}.
      \end{cases}
    \]
  \item l'energia per unit\`a di massa \`e
    $\EnergyPerUnitMass
      = \frac{\ScalarProduct{\dot{r}}{\dot{r}}}{2}
        - \frac{\NewtonConstant\TotalMass}{r}$. Ora, se
    $\EnergyPerUnitMass = 0$, abbiamo visto che l'orbita \`e parabolica
    e abbiamo allora $\SemiMajorAxis = + \infty$; altrimenti abbiamo
    $\SemiMajorAxis
      = - \frac{\NewtonConstant\TotalMass}{2\EnergyPerUnitMass}$;
  \item la terza legge di Keplero fornisce il moto medio
    $\MeanMotion
      = \sqrt{\frac{\NewtonConstant\TotalMass}
        {\SemiMajorAxis^3}}$,
    da cui calcoliamo l'anomalia eccentrica come soluzione di
    \[
      \begin{cases}
        \cos{\EccentricAnomaly} = \frac{1}{\LenzVector}
          \left ( 1 - \frac{r}{\SemiMajorAxis} \right ),\\
        \sin{\EccentricAnomaly} = \frac{r\dot{r}}
          {\SemiMajorAxis^2 \LenzVector \MeanMotion}.
      \end{cases}
    \];
    ne deduciamo l'anomalia media tramite formula di Keplero.
    SOLO CASO ELLITTICO PER ORA
\end{itemize}
\par Supponiamo ora invece noti gli elementi kepleriani. Abbiamo gi\`a
visto come fornire il moto medio $\MeanMotion$, il tempo di passaggio al
pericentro $\PericenterPassageTime$ e l'eccentricit\`a determini
univocamente l'orbita nel caso in cui il piano $Oxy$ coincida col piano
orbitale.
\par Avendo gli elementi kepleriani abbiamo effettivamente anche
\begin{itemize}
  \item $\MeanMotion
          = \sqrt{\frac{\NewtonConstant\TotalMass}
            {\SemiMajorAxis^3}}$;
  \item l'anomalia eccentrica soluzione di
    \[
      \begin{cases}
        \cos{\EccentricAnomaly} = \frac{1}{\LenzVector}
          \left ( 1 - \frac{r}{\SemiMajorAxis} \right ),\\
        \sin{\EccentricAnomaly} = \frac{r\dot{r}}
          {\SemiMajorAxis^2 \LenzVector \MeanMotion}.
      \end{cases}
    \];
  \item il tempo di passaggio al pericentro
    $\PericenterPassageTime
      = \frac{\LenzVector \sin{\EccentricAnomaly(0)}
          - \EccentricAnomaly(0)}{\MeanMotion}$.
\end{itemize}
\par Possiamo dunque calcolare i vettori $\vec{r}$ e $\dot{\vec{r}}$ in
un tale sistema di riferimento. Per ottenere le loro coordinate nel
sistema di riferimento originale occorrer\`a dunque eseguire la
trasformazione
$R
  = R_{\LongitudeOfAscendingNode}
    R_{\Inclination}
    R_{\ArgumentOfPericenter}$,
dove $R_{\LongitudeOfAscendingNode}$, $R_{\Inclination}$,
$R_{\ArgumentOfPericenter}$, sono le rotazioni
\begin{itemize}
  \item $R_{\LongitudeOfAscendingNode} =
        \lmatrix
        \begin{array}{ccc}
          \cos{\LongitudeOfAscendingNode} &
          - \sin{\LongitudeOfAscendingNode} &
          0\\
          \sin{\LongitudeOfAscendingNode} &
          \cos{\LongitudeOfAscendingNode} &
          0\\
          0 &
          0 &
          1
        \end{array}
        \rmatrix$;
  \item $R_{\Inclination}
        \lmatrix
        \begin{array}{ccc}
          1 &
          0 &
          0\\
          \cos{\Inclination} &
          - \sin{\Inclination} &
          0\\
          \sin{\Inclination} &
          \cos{\Inclination} &
          0\\
        \end{array}
        \rmatrix$;
  \item $R_{\ArgumentOfPericenter},
        \lmatrix
        \begin{array}{ccc}
          \cos{\ArgumentOfPericenter} &
          - \sin{\ArgumentOfPericenter} &
          0\\
          \sin{\ArgumentOfPericenter} &
          \cos{\ArgumentOfPericenter} &
          0\\
          0 &
          0 &
          1
        \end{array}
        \rmatrix$. \EndProof
\end{itemize}
\begin{Theorem}
  \TheoremName{Equazione di Baker}[di Baker][equazione]
  Dato un problema di Keplero
	\[
		\ReducedMass \ddot{\vec{r}}
    = - \NewtonConstant \frac{\TotalMass\ReducedMass}{r^3}\vec{r},
	\]
  se l'orbita \`e iperbolica allora
  \[ 
    t - t_0
    = \frac{1}{2}
    \sqrt{\frac{\SemilatusRectum^3}{\NewtonConstant\TotalMass}}
    \left [
    \tan \frac{\TrueAnomaly}{2}
    + \frac{1}{3} \tan^3 \frac{\TrueAnomaly}{3}
    \right ]_{\TrueAnomaly_0}^{\TrueAnomaly}.
  \]
  dove
  \begin{itemize}
    \item $t_0$ \`e un tempo iniziale fissato;
    \item $\TrueAnomaly_0$ \`e l'anomalia vera al tempo $t_0$;
    \item $\TrueAnomaly$ l'anomalia vera al tempo $t$;
    \item $\SemilatusRectum$ il semilato retto.
  \end{itemize}
\end{Theorem}
\Proof Dall'equazione
$\AngularMomentPerUnitMass = r^2 \dot{\TrueAnomaly}$
deduciamo
$\int_{t_0}^t \AngularMomentPerUnitMass d\tau
= \int_{\TrueAnomaly_0}^\TrueAnomaly r^2 d\TrueAnomaly$,
dove $\TrueAnomaly_0$ \`e l'anomalia vera al tempo $t_0$:
$\TrueAnomaly_0 = \TrueAnomaly(t_0)$.
\par Ora, abbiamo
\begin{align*}
  r
  &= \frac{\SemilatusRectum}{1 + \cos \TrueAnomaly},\\
  &= \frac{\SemilatusRectum}{2}
      \frac{1}{\cos^2 \frac{\TrueAnomaly}{2}},\\
  &= \frac{\SemilatusRectum}{2}
      \frac{\cos^2\frac{\TrueAnomaly}{2} + \sin^2\frac{\TrueAnomaly}{2}}
        {\cos^2 \frac{\TrueAnomaly}{2}},\\
  &= \frac{\SemilatusRectum}{2}
      \left ( 1 + \tan^2 \frac{\TrueAnomaly}{2} \right );
\end{align*}
inoltre, posto $x = \tan \frac{\TrueAnomaly}{2}$, abbiamo
\begin{align*}
  dx
    &= \frac{1}{2}
      \left (1 + \tan^2\frac{\TrueAnomaly}{2} \right ) d\TrueAnomaly,\\
  d\TrueAnomaly = \frac{2}{1 + x^2} dx.
\end{align*}
Dunque
\begin{align*}
t - t_0
&= \frac{1}{\AngularMomentPerUnitMass}
    \int_{t_0}^t \AngularMomentPerUnitMass d\tau,\\
&= \frac{1}{\sqrt{\NewtonConstant\TotalMass\SemilatusRectum}}
    \int_{\TrueAnomaly_0}^\TrueAnomaly r^2 d\TrueAnomaly,\\
&= \frac{\SemilatusRectum^2}
    {4\sqrt{\NewtonConstant\TotalMass\SemilatusRectum}}
    \int_{\TrueAnomaly_0}^\TrueAnomaly
    \left (
    1 + \tan^2 \frac{\TrueAnomaly}{2}
    \right )^2
    d\TrueAnomaly,\\
&= \frac{1}{4}
    \sqrt{\frac{\SemilatusRectum^3}{\NewtonConstant\TotalMass}}
    \int_{x(\TrueAnomaly_0)}^{x(\TrueAnomaly)}
    \frac{(1 + x^2)^2}
    {x^2 + 1}
    dx,\\
&= \frac{1}{4}
    \sqrt{\frac{\SemilatusRectum^3}{\NewtonConstant\TotalMass}}
    \int_{x(\TrueAnomaly_0)}^{x(\TrueAnomaly)}
    2 (1 + x^2)
    dx,\\
&= \frac{1}{2}
    \sqrt{\frac{\SemilatusRectum^3}{\NewtonConstant\TotalMass}}
    \left [
    x + \frac{x^3}{3}
    \right ]_{x(\TrueAnomaly_0)}^{x(\TrueAnomaly)},\\
&= \frac{1}{2}
    \sqrt{\frac{\SemilatusRectum^3}{\NewtonConstant\TotalMass}}
    \left [
    \tan \frac{\TrueAnomaly}{2}
    + \frac{1}{3} \tan^3 \frac{\TrueAnomaly}{3}
    \right ]_{\TrueAnomaly_0}^{\TrueAnomaly}.
\end{align*}
\begin{Definition}
  Dato un problema di Keplero
	\[
		\ReducedMass \ddot{\vec{r}}
    = - \NewtonConstant \frac{\TotalMass\ReducedMass}{r^3}\vec{r},
	\]
  chiamiamo
  \Define{tempo di volo}[di volo][tempo]
  la quantit\`a $t - \PericenterPassageTime$, dove
  $\PericenterPassageTime$ \`e il tempo di passaggio al pericentro.
\end{Definition}
\begin{Theorem}
  Dato un problema di Keplero
	\[
		\ReducedMass \ddot{\vec{r}}
    = - \NewtonConstant \frac{\TotalMass\ReducedMass}{r^3}\vec{r},
	\]
  il tempo di volo \`e
  \begin{itemize}
    \item $t - \PericenterPassageTime
            = \sqrt{\frac{\SemiMajorAxis^3}{\NewtonConstant\TotalMass}}
              (\EccentricAnomaly - \LenzVector \sin \EccentricAnomaly)$
          se l'orbita \`e ellittica;
    \item $t - \PericenterPassageTime
            = \frac{1}{2}
              \sqrt{\frac{\SemilatusRectum^3}{\NewtonConstant\TotalMass}}
              \left (
              \tan \frac{\TrueAnomaly}{2}
              + \frac{1}{3} \tan^3 \frac{\TrueAnomaly}{2}
              \right )$
          se l'orbita \`e parabolica;
    \item $t - \PericenterPassageTime
            = \sqrt{-\frac{\SemiMajorAxis^3}{\NewtonConstant\TotalMass}}
              (\LenzVector \sinh \EccentricAnomaly - \EccentricAnomaly)$
          se l'orbita \`e iperbolica.
  \end{itemize}
\end{Theorem}
\Proof Supponiamo l'orbita ellittica. Abbiamo
\[
  \MeanAnomaly
    = \EccentricAnomaly - \LenzVector \sin \EccentricAnomaly,\\
  \MeanAnomaly
    = \MeanMotion (t - \PericenterPassageTime).
\]
Da cui, tramite la terza legge di Keplero,
\begin{align*}
  t - \PericenterPassageTime
    &= \frac{1}{\MeanMotion} 
      \EccentricAnomaly - \LenzVector \sin \EccentricAnomaly,\\
    &= \sqrt{\frac{\SemiMajorAxis^3}{\NewtonConstant\TotalMass}}
      \EccentricAnomaly - \LenzVector \sin \EccentricAnomaly.
\end{align*}
\par Supponiamo l'orbita parabolica. Segue direttamente dall'equazione
di Baker che il tempo di volo \`e
$t - \PericenterPassageTime
= \frac{1}{2}
\sqrt{\frac{\SemilatusRectum^3}{\NewtonConstant\TotalMass}}
\left (
\tan \frac{\TrueAnomaly}{2}
+ \frac{1}{3} \tan^3 \frac{\TrueAnomaly}{2}
\right )$.

\section{Problema di due corpi.}
\label{ElementiDiMeccanicaCeleste_ProblemaDiDueCorpi}
\begin{figure}
	\includegraphics[width=0.8\textwidth]{Elementi_di_meccanica_celeste/Immagine_2Corpi.png}
	\centering
	\caption{Problema di due corpi.}
\end{figure}
\par Assumiamo nel seguito di aver fissato un sistema di riferimento inerziale.
\begin{Definition}
	Dato un sistema isolato di due punti materiali
	\begin{itemize}
		\item $\PointMass_1$ di $\Mass_1$ in posizione $\vec{\rho_1}$;
		\item $\PointMass_2$ di $\Mass_2$ in posizione $\vec{\rho_2}$;
	\end{itemize}
	chiamiamo
	\begin{itemize}
		\item \Define{massa ridotta}[ridotta][massa], denotata $\ReducedMass$, la quantit\`a $\ReducedMass = \Mass_1^{-1} + \Mass_2^{-1} = \frac{\Mass_1\Mass_2}{\Mass_1 + \Mass_2}$;
		\item \Define{problema ridotto}[ridotto][problema] o \Define{problema di Keplero}[di Keplero][problema] il problema del moto di un punto materiale di massa $\ReducedMass$ che si muove in un campo gravitazionale generato da una massa $\TotalMass = \Mass_1 + \Mass_2$ vincolata nell'origine.
	\end{itemize}
\end{Definition}
\begin{Theorem}
	Dato un sistema isolato di due punti materiali
	\begin{itemize}
		\item $\PointMass_1$ di $\Mass_1$ in posizione $\vec{\rho_1}$;
		\item $\PointMass_2$ di $\Mass_2$ in posizione $\vec{\rho_2}$;
	\end{itemize}
	sono costanti 
	\begin{itemize}
		\item l'energia $\Energy$ del sistema;
		\item le componenti del momento angolare $\vec{\AngularMoment}$;
		\item le componenti della quantit\`a di moto $\vec{\Momentum}$.
	\end{itemize}
\end{Theorem}
\Proof Abbiamo $\Energy = \frac{1}{2}\Mass_1\dot{\rho_1}^2 + \frac{1}{2}\Mass_2\dot{\rho_2}^2 - \NewtonConstant \frac{\Mass_1\Mass_2}{r}$, da cui
\begin{align*}
	\dot{\Energy}
	&= \Mass_1\ddot{\rho_1}\dot{\rho_1} + \Mass_2\ddot{\rho_2}\dot{\rho_2} + \NewtonConstant \frac{\Mass_1\Mass_2}{r^2} \dot{r},\\
	&= - \NewtonConstant \frac{\Mass_1\Mass_2}{r^2}\dot{r} + \NewtonConstant \frac{\Mass_1\Mass_2}{r^2} \dot{r} = 0.
\end{align*}
(La stessa cosa poteva essere dedotta direttamente dal primo principio della termodinamica.)
\par Per il momento angolare, osserviamo che la forza gravitazionale \`e centrale, dunque il momento angolare dei singoli punti materiali \`e costante e quindi costante \`e anche il momento angolare $\vec{\AngularMoment}$ del sistema.
\par Per la quantit\`a di moto, abbiamo, poich\'e il sistema \`e isolato, $(\Mass_1 + \Mass_2) \ddot{\vec{\CenterOfMass}} = 0$, dove $\CenterOfMass = \frac{\Mass_1\vec{\rho_1} + \Mass_2\vec{\rho_2}}{\Mass_1 + \Mass_2}$; quindi $\vec{\Momentum} = (\Mass_1 + \Mass_2) \dot{\vec{\CenterOfMass}}$ \`e costante. \EndProof
\begin{Corollary}
	Il problema di due corpi \`e integrabile.
\end{Corollary}
\Proof Il problema di due corpi \`e un problema con $6$ gradi di libert\`a e $6$ integrali primi indipendenti\footnote{Il momento angolare costituisce in realt\`a solo due integrali primi perch\'e non si tratta propriamente di un vettore, bens\'i di uno pseudovettore.}. \EndProof
\begin{Theorem}
	Dato un sistema isolato di due punti materiali
	\begin{itemize}
		\item $\PointMass_1$ di $\Mass_1$ in posizione $\vec{\rho_1}$;
		\item $\PointMass_2$ di $\Mass_2$ in posizione $\vec{\rho_2}$;
	\end{itemize}
	il loro centro di massa $\vec{\CenterOfMass}$ si muove di moto rettilinero uniforme.
\end{Theorem}
\Proof La quantit\`a di moto del sistema $\vec{\Momentum}$ \`e costante. \`E dunque costante anche $\dot{\vec{\CenterOfMass}} = \frac{\vec{\Momentum}}{\Mass_1 + \Mass_2}$. \EndProof
\begin{Corollary}
	Con le notazioni del teorema precedente, il sistema di riferimento ottenuto per traslazione dell'origine nel centro di massa $\vec{\CenterOfMass}$ \`e anch'esso inerziale.
\end{Corollary}
\Proof Segue direttamente dal teorema precedente.
\begin{Theorem}
	\label{ElementiDiMeccanicaCeleste_RiduzioneAlProblemaDiKeplero}
	Consideriamo un sistema isolato di due punti materiali riferito ad un sistema inerziale centrato nel loro centro di massa
	\begin{itemize}
		\item $\PointMass_1$ di $\Mass_1$ in posizione $\vec{r_1}$;
		\item $\PointMass_2$ di $\Mass_2$ in posizione $\vec{r_2}$;
	\end{itemize}
	il problema di due corpi
	\[
		\begin{cases}
			\Mass_1 \ddot{\vec{r_1}} = \NewtonConstant \frac{\Mass_1\Mass_2}{r^3}\vec{r},\\
			\Mass_2 \ddot{\vec{r_2}} = - \NewtonConstant \frac{\Mass_1\Mass_2}{r^3}\vec{r},
		\end{cases}
	\]
	posto,
	\begin{itemize}
		\item $\ReducedMass = \Mass_1^{-1} + \Mass_2^{-1}$;
		\item $\TotalMass = \Mass_1 + \Mass_2$;
		\item $\vec{r} = \vec{r_2} - \vec{r_1}$.
	\end{itemize}
	equivale al problema ridotto
	\[
		\ReducedMass \ddot{\vec{r}} = - \NewtonConstant \frac{\TotalMass\ReducedMass}{r^3}\vec{r},
	\]
	tramite la trasformazione lineare invertibile $\vec{x} \mapsto - \frac{\TotalMass}{\Mass_2} \vec{x}$ dallo spazio delle soluzioni di $\vec{r_1}$ nello spazio delle soluzioni di $\vec{r}$.
\end{Theorem}
\Proof Supponiamo $\vec{r_1}$ sia soluzione del problema a due corpi per il punto materiale $\PointMass_1$.
\par Abbiamo, a seguito della trasformazione,
\begin{align*}
	\ReducedMass \ddot{\vec{r}}
	&= - \frac{\TotalMass}{\Mass_1\Mass_2} \frac{\TotalMass}{\Mass_2} \NewtonConstant \frac{\Mass_2}{r^3}\vec{r},\\
	&= - \NewtonConstant \frac{\TotalMass\ReducedMass}{r^3}\vec{r}.
\end{align*}
\par D'altra parte, supponiamo $\vec{r}$ sia soluzione del problema di Keplero.
\par Abbiamo, a seguito della trasformazione,
\begin{align*}
	\Mass_1 \ddot{\vec{r_1}}
	&= \Mass_1 \frac{\Mass_2}{\TotalMass} \NewtonConstant \frac{\TotalMass}{r^3}\vec{r},\\
	&= \NewtonConstant \frac{\Mass_1\Mass_2}{r^3}\vec{r};
\end{align*}
e inoltre
\begin{align*}
	\Mass_2 \ddot{\vec{r_2}}
	&= \Mass_2 (\ddot{\vec{r}} + \ddot{\vec{r_1}}),\\
	&= - \Mass_2 \NewtonConstant \frac{\TotalMass}{r^3}\vec{r} + \Mass_2 \NewtonConstant \frac{\Mass_2}{r^3}{r},\\
	&= \NewtonConstant \frac{\Mass_2(\Mass_2 - \TotalMass)}{r^3}\vec{r},\\
	&= - \NewtonConstant \frac{\Mass_1\Mass_2}{r^3}\vec{r}.\text{ \EndProof}
\end{align*}
\begin{Corollary}
	\TheoremName{Prima legge di Keplero}[prima di Keplero][legge] La Terra si muove attorno al Sole secondo un'orbita ellittica di cui il Sole occupa uno dei due fuochi.
\end{Corollary}
\Proof Per la grande disparit\`a di massa tra il Sole e la Terra, il centro di massa dei due corpi pu\`o essere posto con ottima approssimazione nel centro del Sole: ne risulta che il problema di due corpi composto da Sole e Terra \`e con ottima approssimazione identico al suo problema di Keplero equivalente. L'esperienza diretta dimostra che l'orbita \`e periodica e dunque ellittica (non circolare). \EndProof
\begin{Theorem}
	Consideriamo un sistema isolato di due punti materiali riferito ad un
  sistema inerziale centrato nel loro centro di massa
	\begin{itemize}
		\item $\PointMass_1$ di $\Mass_1$ in posizione $\vec{r_1}$;
		\item $\PointMass_2$ di $\Mass_2$ in posizione $\vec{r_2}$.
	\end{itemize}
	Consideriamo il problema di due corpi
	\[
		\begin{cases}
			\Mass_1 \ddot{\vec{r_1}}
        = \NewtonConstant \frac{\Mass_1\Mass_2}{r^3}\vec{r},\\
			\Mass_2 \ddot{\vec{r_2}}
        = - \NewtonConstant \frac{\Mass_1\Mass_2}{r^3}\vec{r},
		\end{cases}
	\]
	equivalente al problema di Keplero
	\[
		\ReducedMass \ddot{r}
      = - \NewtonConstant \frac{\TotalMass\ReducedMass}{r^3}\vec{r},
	\]
	con
	\begin{itemize}
		\item $\ReducedMass = \Mass_1^{-1} + \Mass_2^{-1}$;
		\item $\TotalMass = \Mass_1 + \Mass_2$;
		\item $\vec{r} = \vec{r_2} - \vec{r_1}$;
	\end{itemize}
	tramite la trasformazione lineare invertibile
$\vec{x} \mapsto - \frac{\TotalMass}{\Mass_2} \vec{x}$ dallo spazio
delle soluzioni di $\vec{r_1}$ nello spazio delle soluzioni di
$\vec{r}$.
	Le orbite del problema di due corpi sono coniche
	\begin{itemize}
		\item di eccentricit\`a $\LenzVector$, dove $\vec{\LenzVector}$ \`e
      il vettore di Lenz del problema ridotto ;
		\item semilati retti $\SemilatusRectum_1 =
      \frac{\Mass_2}{\TotalMass} \SemilatusRectum$ e
      $\SemilatusRectum_2 = \frac{\Mass_1}{\TotalMass} \SemilatusRectum$
      per le orbite di $\PointMass_1$ e $\PointMass_2$ rispettivamente,
      dove $\SemilatusRectum$ \`e il semilato dell'orbita del problema
      ridotto;
		\item i semilati retti $\SemilatusRectum_1$ e $\SemilatusRectum_2$
      sono tali che $\SemilatusRectum_1 + \SemilatusRectum_2
      = \SemilatusRectum$;
		\item semiassi maggiori $\SemiMajorAxis_1$ e $\SemiMajorAxis_2$ tali
      che $\SemiMajorAxis_1 + \SemiMajorAxis_2 = \SemiMajorAxis$, dove
      $\SemiMajorAxis$ il semiassemaggiore dell'orbita del problema
      ridotto;
		\item $\PointMass_1$ e $\PointMass_2$ si trovano allineati col
      centro di massa e situati da parti opposte rispetto ad esso;
		\item le orbite di $\PointMass_1$ e $\PointMass_2$ vengono percorse
      con velocit\`a allineate e opposte e tali che
      $\dot{r_1} + \dot{r_2} = \dot{r}$.
	\end{itemize}
\end{Theorem}
\Proof Abbiamo
\begin{align*}
	\vec{r_1} = - \frac{\Mass_2}{\TotalMass} \vec{r},\\
	\vec{r_2} = \frac{\Mass_1}{\TotalMass} \vec{r},
\end{align*}
da cui, passando ai moduli e utilizzando la soluzione del problema di Keplero,
\begin{align*}
	r_1(\TrueAnomaly) = \frac{\Mass_2}{\TotalMass} \frac{\SemilatusRectum}{1 + \LenzVector \cos{\TrueAnomaly}},\\
	r_2(\TrueAnomaly) = \frac{\Mass_1}{\TotalMass} \frac{\SemilatusRectum}{1 + \LenzVector \cos{\TrueAnomaly}}.
\end{align*}
\par Abbiamo dunque
\begin{align*}
	\SemilatusRectum_1 = \frac{\Mass_2}{\TotalMass} \SemilatusRectum,\\
	\SemilatusRectum_2 = \frac{\Mass_1}{\TotalMass} \SemilatusRectum,
\end{align*}
da cui,
\begin{align*}
	\SemiMajorAxis_1 = \frac{\SemilatusRectum_1}{1 - \LenzVector^2},\\
	\SemiMajorAxis_2 = \frac{\SemilatusRectum_2}{1 - \LenzVector^2}.
\end{align*}
\par Abbiamo inoltre
\begin{itemize}
	\item $\SemilatusRectum_1 + \SemilatusRectum_2 = \frac{\Mass_2}{\TotalMass} \SemilatusRectum + \frac{\Mass_1}{\TotalMass} \SemilatusRectum = \SemilatusRectum$;
	\item $\SemiMajorAxis_1 + \SemiMajorAxis_2 = \frac{\SemilatusRectum_1 + \SemilatusRectum_2}{1 - \LenzVector^2} = \frac{\SemilatusRectum}{1 -\LenzVector^2} = \SemiMajorAxis$.
\end{itemize}
\par $\PointMass_1$ e $\PointMass_2$ sono allineati con il loro centro di massa e situati da parti opposte rispetto ad essi per la definizione stessa di centro di massa.
\par Infine, per valutare i versi di percorrenza, deriviamo $\vec{r_1}$ e $\vec{r_2}$:
\begin{align*}
	\dot{\vec{r_1}} = - \frac{\Mass_2}{\TotalMass} \dot{\vec{r}},\\
	\dot{\vec{r_2}} = \frac{\Mass_1}{\TotalMass} \dot{\vec{r}}.
\end{align*}
Abbiamo in particolare $\dot{r_1} + \dot{r_2} = \frac{\Mass_2}{\TotalMass} \dot{r} + \frac{\Mass_1}{\TotalMass} \dot{r} = \dot{r}$. \EndProof
\begin{figure}
	\includegraphics[width=0.8\textwidth]{Elementi_di_meccanica_celeste/Immagine_2Corpi_Orbite.png}
	\centering
	\caption{Orbite del problema dei due corpi.}
\end{figure}

\section{Problema di due corpi perturbato.}
\label{ElementiDiMeccanicaCeleste_ProblemaDiDueCorpiPerturbato}

\section{Problema di tre corpi ristretto circolare.}
\label{ElementiDiMeccanicaCeleste_ProblemaDiTreCorpiRistrettoCircolare}


	\cleardoublepage
	\phantomsection
	\addcontentsline{toc}{part}{\listfigurename}
	\listoffigures
	\cleardoublepage
	\phantomsection
	\addcontentsline{toc}{part}{\listtablename}
	\listoftables
	\cleardoublepage
	\phantomsection
	\addcontentsline{toc}{part}{\listoflistingscaption}
	\listoflistings
	\cleardoublepage
	\phantomsection
	\addcontentsline{toc}{part}{\indexname}
	\printindex
\end{document}
