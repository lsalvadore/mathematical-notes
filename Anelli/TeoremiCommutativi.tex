\section{Teoremi generali sugli anelli commutativi.}\label{TeoremiCommutativi}
\begin{Theorem}
\TheoremName{Teorema binomiale di Newton}[binomiale di Newton][teorema]
	Sia $\Ring$ un anello commutativo e siano $a,b \in \Ring$.
	Per ogni $n \in \mathbb{N}$ \`e verificata l'equazione
	$(a + b)^n = \sum_{i = 0}^n \binom{n}{i} a^ib^{n - i}$.
\end{Theorem}
\Proof
Per la distributivit\`a del prodotto rispetto alla somma,
$(a + b)^n$ \`e uguale a una somma di prodotti di $n$ fattori,
ciascuno dei quali pu\`o assumere il valore $a$ o $b$.
Poich\'e esistono $\binom{n}{i}$ prodotti
in cui esattamente $i$ fattori sono uguali ad $a$ e
i restanti sono uguali a $b$, si ha la tesi.
\EndProof
\begin{Theorem}
	Dato un anello $\Ring$ di caratteristica
	$\Characteristic{\Ring}$ finita,
	per ogni $a, b \in \Ring$,
	abbiamo $(a + b)^\Prime = a^\Prime + b^\Prime$.
\end{Theorem}
\Proof
Applichiamo la formula del binomio di Newton:
$(a + b)^{\Characteristic{\Ring}} =
\sum_{k = 0}^{\Characteristic{\Ring}}
\binom{\Characteristic{\Ring}}{k} a^kb^{n - k}$.
Per $k = 0$ e $k = \Characteristic{\Ring}$ otteniamo i termini
$b^{\Characteristic{\Ring}}$ e $a^{\Characteristic{\Ring}}$ rispettivamente.
Tutti gli altri termini hanno coefficiente divisible per
$\Characteristic{\Ring}$, e dunque nullo.
\EndProof
\begin{Theorem}
	Sia $\Ring$ un anello di caratteristica $\Prime \in \mathbb{Z}$ finita e
	sia $\Frobenius: \Ring \rightarrow \Ring$ l'applicazione che ad
	$x \in \Ring$ associa $x^{\Prime}$.
	$\Frobenius$ \`e un endomorfismo.
\end{Theorem}
\Proof
Per ogni $x,y \in \Ring$, abbiamo
$\Frobenius(x + y) =
(x + y)^{\Prime} =
x^\Prime + y^\Prime =
\Frobenius(x) + \Frobenius(y)$ e
$\Frobenius(xy) =
(xy)^\Prime =
x^\Prime y^\Prime =
\Frobenius(x)\Frobenius(y)$.
\EndProof
\begin{Definition}
	L'endomorfismo $\Frobenius$ del teorema precedente si chiama
	\Define{endomorfismo di Frobenius}[di Frobenius][endomorfismo].
\end{Definition}
