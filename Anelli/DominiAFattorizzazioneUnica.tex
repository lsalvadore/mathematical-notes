\subsection{Domini a fattorizzazione unica.}\label{DominiAFattorizzazioneUnica}
\begin{Definition}
	Un \Define{dominio a fattorizzazione unica}[a fattorizzazione unica][dominio] $\UFD$ \`e un dominio di integrit\`a tale che ogni suo elemento $a \in \NotZero{\UFD}$ non invertibile si scrive in modo unico come prodotto di fattori irriducibili, a meno di unit\`a.
\end{Definition}
\begin{Theorem}
	In un dominio a fattorizzazione unica $\UFD$ ogni elemento irriducibile \`e primo.
\end{Theorem}
\Proof Sia $p \in \UFD$ irriducibile. Siano $a,b \in \UFD$ tali che $p|ab$. Il prodotto delle fattorizzazioni in irriducibili di $a$ e $b$ d\`a la fattorizzazione in irriducibili di $ab$: in quest'ultima appare necessariamente $p$, il quale appare dunque necessariamente nella fattorizzazione di $a$ o in quella di $b$, da cui $p|a \vee p|b$. \EndProof
\begin{Theorem}
	In un dominio a fattorizzazione unica $\UFD$ ogni famiglia finita di elementi ammette un massimo comun divisore.
\end{Theorem}
\Proof Si verifica immediatamente che il massimo comun divisore cercato \`e dato dal prodotto di tutti i fattori primi che compaiono in tutti gli elementi della famiglia finita (tenendo conto delle ripetezioni). \EndProof
