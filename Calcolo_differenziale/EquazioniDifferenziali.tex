\section{Equazioni differenziali.}
\label{CalcoloDifferenziale_EquazioniAlleDerivateParzialiEEquazioniDifferenziali}
\begin{Definition}
	Un'\Define{equazione funzionale}[funzionale][equazion] \`e un'equazione dove le incognite sono funzioni.
\end{Definition}
\begin{Definition}
	Un'\Define{equazione differenziale}[differenziale] \`e un'equazione funzionale che coinvolge almeno una derivata di una funzione incognita. L'equazione si chiama
	\begin{itemize}
		\item \Define{equazione differenziale ordinaria}[differenziale ordinaria][equazione] quando le funzioni incognite sono funzioni in una sola variabile;
		\item \Define{equazione alle derivate parziali}[alle derivate parziali][equazione] quando le funzioni incognite sono funzioni in pi\`u variabili.
	\end{itemize}
	Chiamiamo \Define{ordine}[di un'equazione differenziale] dell'equazione il massimo degli ordini delle derivate presenti nell'equazione stessa.
\end{Definition}
\begin{Definition}
	Un'\Define{equazione differenziale autonoma} \`e un'equazione differenziale ordinaria che non dipende esplicitamente dalla variabile indipendente. Vale a dire che \`e della forma $f((x^{(i)})_{i \in n})) = 0$, dove $n \in \mathbb{N}$, $f$ \`e una funzione qualsiasi e $x(t)$ \`e la funzione incognita.
\end{Definition}
