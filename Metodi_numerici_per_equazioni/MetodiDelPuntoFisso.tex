\section{Metodi del punto fisso.}
\label{MetodiNumericiPerLaRisoluzioneDiEquazioni_MetodiDelPuntoFisso}
\begin{Definition}
	Sia $f: I \rightarrow \mathbb{C}$ ($I \subseteq \mathbb{C}$) una funzione. Chiamiamo \Define{metodo del punto fisso}[del punto fisso][metodo] il dato di
	\begin{itemize}
		\item una funzione $g: J \rightarrow J$ ($J\subseteq \mathbb{C}$) tale che $f(x) = 0$ se e solo se $x = g(x)$;
		\item un punto $x_0 \in J$, detto \Define{punto iniziale}[iniziale di un metodo del punto fisso][punto].
	\end{itemize}
	Diciamo che il metodo \Define{converge}[di un metodo del punto fisso][convergenza] a un punto $\xi \in \mathbb{C}$ quando converge a $\xi$ la successione $(x_n)_{n \in \mathbb{N}}$, definita dalla relazione $\ForAll{k \in \mathbb{N}}{x_{n + 1} = g(x_n)}$.
\end{Definition}
\begin{Theorem}
	Con le notazioni della definizione precedente, se $g$ \`e continua e il metodo converge a $\xi$, allora $f(\xi) = 0$.
\end{Theorem}
\Proof Abbiamo $\xi = \lim_{n \rightarrow \infty} x_{n + 1} = \lim_{n \rightarrow \infty} g(x_n) = g \left ( \lim_{n \rightarrow \infty} x_n \right ) = g(\xi)$, che \`e equivalente a $f(\xi) = 0$. \EndProof
\subsection{Metodo delle secanti.}
\label{MetodiNumericiPerLaRisoluzioneDiEquazioni_MetodoDelleSecanti}

\subsection{Metodo di Newton o delle tangenti.}
\label{MetodiNumericiPerEquazioni_MetodoDiNewton}
\begin{Definition}
	Sia $f: I \rightarrow \mathbb{C}$ ($I \subseteq \mathbb{R}^n$) una funzione differenziabile. Chiamiamo \Define{metodo di Newton}[di Newton][metodo] o \Define{metodo delle tangenti}[delle tangenti][metodo] un metodo del punto fisso definito da un punto iniziale qualsiasi e della funzione $g(x) = x - (\Jacobian(x))^{-1} f(x)$, dove $\Jacobian$ \`e lo Jacobiano di $f$.
\end{Definition}
\begin{listing}
	\insertcode{octave}{"Metodi_numerici_per_equazioni/MetodoDiNewton.m"}
	\caption{Implementazione del metodo di Newton in \LanguageName{octave}.}
\end{listing}

