\section{Definizioni e teoremi di base.}
\label{DefinizioniETeoremiDiBase}
\begin{Definition}
  Sia $\Class$ una classe. Definiamo
  \Define{nozione di convergenza}[di convergenza][nozione]
  una parte
  $\NotionOfConvergence \subseteq \Class^\mathbb{N} \times \Class$.
  Dato
  $((x_n)_{n \in \mathbb{N}},x) \in \NotionOfConvergence$,
  diciamo che
  $(x_n)_{n \in \mathbb{N}}$
  \Define{converge}[convergente][successione]
  a $x$.
\end{Definition}
\begin{Theorem}
  La definizione di convergenza in uno spazio topologico
  $(\TopologicalSpace,\Topology)$
  definisce una nozione di convergenza
  $\NotionOfConvergence$ i cui elementi sono tutte e sole le coppie
  $((x_n)_{n \in \mathbb{N}},x) \in \Class^\mathbb{N} \times \Class$,
  tali che
  $(x_n)_{n \in \mathbb{N}}$
  converge a $x$.
\end{Theorem}
\Proof Immediata. \EndProof
\begin{Definition}
  Una classe $\Class$ in cui \`e definita una nozione di convergenza
  $\NotionOfConvergence$ si dice
  \Define{compatta per successioni}[per successioni][compattezza]
  rispetto a $\NotionOfConvergence$
  se ogni successione
  $(x_n)_{n \in \mathbb{N}} \in \Class^\mathbb{N}$
  ammette una sottosuccessione
  $(x_f(n))_{n \in \mathbb{N}} \in \Class^\mathbb{N}$,
  con $f: \mathbb{N} \rightarrow \mathbb{N}$ applicazione crescente,
  convergente a un punto $x \in \Class$.
\end{Definition}
