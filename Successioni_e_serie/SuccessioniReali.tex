\section{Successioni reali.}\label{SuccessioniReali}
\begin{Definition}
	Sia $(a_n)_{n \in \mathbb{N}} \in \mathbb{R}^\mathbb{N}$.
  $(a_n)_{n \in \mathbb{N}}$ si dice
	\begin{itemize}
		\item
      \Define{a termini positivi}[a termini positivi][successione]
      quando $\ForAll{n \in \mathbb{N}}{a_n \geq 0}$;
		\item
      \Define{a termini strettamente positivi}[a termini positivi][successione]
      quando $\ForAll{n \in \mathbb{N}}{a_n > 0}$;
		\item
      \Define{a termini negativi}[a termini negativi][successione]
      quando $\ForAll{n \in \mathbb{N}}{a_n \leq 0}$;
		\item
      \Define{a termini strettamente negativi}[a termini negativi][successione]
      quando $\ForAll{n \in \mathbb{N}}{a_n < 0}$;
		\item
      \Define{divergente positivamente}[divergente positivamente][successione]
      o anche
      \Define{divergente a $+\infty$}[divergente a $+\infty$][successione] 
      quando
      \[
        \ForAll{M \in \mathbb{R}}{
          \Exists{N \in \mathbb{N}}{
            \Implies{n > N}{a_n > M}}},
      \]
      e in tal caso scriviamo anche
      $\lim_{n \rightarrow \infty} a_n = + \infty$;
		\item
      \Define{divergente negativamente}[divergente negativamente][successione]
      o anche
      \Define{divergente a $-\infty$}[divergente a $-\infty$][successione]
      quando
      \[
        \ForAll{M \in \mathbb{R}}{
          \Exists{N \in \mathbb{N}}{
            \Implies{n > N}{
              a_n < M}}},
      \]
      e in tal caso scriviamo anche
      $\lim_{n \rightarrow \infty} a_n = - \infty$.
	\end{itemize}
\end{Definition}
\begin{Theorem}
	Sia $(a_n)_{n \in \mathbb{N}} \in \mathbb{R}^\mathbb{N}$.
  $(a_n)_{n \in \mathbb{N}}$ diverge positivamente se e solo se
  $(- a_n)_{n \in \mathbb{N}}$ diverge negativamente.
\end{Theorem}
\Proof $(a_n)_{n \in \mathbb{N}}$ diverge positivamente se e solo se
$\ForAll{M > 0}{\Exists{N > 0}{\Implies{n > N}{a_n > M}}}$,
equivalente a
$\ForAll{M > 0}{\Exists{N > 0}{\Implies{n > N}{- a_n < - M}}}$,
cio\`e alla divergenza negativa di $(- a_n)_{n \in \mathbb{N}}$. \EndProof
\begin{Theorem}
	Sia $(a_n)_{n \in \mathbb{N}} \in \mathbb{R}^\mathbb{N}$ divergente
  positivamente. Ogni sottosuccessione $(a_{f(n)})_{n \in \mathbb{N}}$,
  con $f: \mathbb{N} \rightarrow \mathbb{N}$ applicazione crescente,
  diverge anch'essa positivamente.
\end{Theorem}
\Proof Sia $M > 0$. Esiste $N \in \mathbb{N}$ tale che
$\Implies{n > N}{a_n > M}$. A maggior ragione,
$\Implies{n > N}{a_{f(n)} > M}$ e dunque $(a_{f(n)})_{n \in \mathbb{N}}$
diverge positivamente. \EndProof
\begin{Corollary}
	Sia $(a_n)_{n \in \mathbb{N}} \in \mathbb{R}^\mathbb{N}$ divergente
  negativamente. Ogni sottosuccessione $(a_{f(n)})_{n \in \mathbb{N}}$,
  con $f: \mathbb{N} \rightarrow \mathbb{N}$ applicazione crescente,
  diverge anch'essa negativamente.
\end{Corollary}
\Proof $(a_n)_{n \in \mathbb{N}}$ diverge negativamente se e solo se
$(- a_n)_{n \in \mathbb{N}}$ diverge positivamente, che implica la divergenza
positiva di $(- a_{f(n)})_{n \in \mathbb{N}}$, equivalente alla divergenza
negativa di $(a_{f(n)})_{n \in \mathbb{N}}$. \EndProof
\begin{Theorem}
	Sia $(a_n)_{n \in \mathbb{N}} \in \mathbb{R}^\mathbb{N}$ una successione
  crescente.
  Se $(a_n)_{n \in \\mathbb{N}}$ \`e limitata, allora \`e convergente;
  altrimenti \`e divergente.
\end{Theorem}
\Proof Consideriamo $L = \sup_{n \in \mathbb{N}} a_n$: segue direttamente dalle
definizioni che
\begin{itemize}
	\item se $L$ \`e finito, $(a_n)_{n \in \mathbb{N}}$ converge a $L$;
	\item se $L$ \`e infinito, $(a_n)_{n \in \mathbb{N}}$ diverge. \EndProof
\end{itemize}
\begin{Theorem}
	Sia $(a_n)_{n \in \mathbb{N}} \in \mathbb{R}^\mathbb{N}$ una successione
  decrescente.
  $(a_n)_{n \in \mathbb{N}} \in \mathbb{R}^\mathbb{N}$ converge.
\end{Theorem} 
\Proof Consideriamo $L = \inf_{n \in \mathbb{N}} a_n$: segue direttamente dalle
definizioni che
\begin{itemize}
	\item se $L$ \`e finito, $(a_n)_{n \in \mathbb{N}}$ converge a $L$;
	\item se $L$ \`e infinito, $(a_n)_{n \in \mathbb{N}}$ diverge. \EndProof
\end{itemize}
\begin{Theorem}
	Le progressioni geometriche $(a^n)_{n \in \mathbb{N}}$ tali che
  $\AbsoluteValue{a} < 1$ convergono a $0$, mentre le progressioni geometriche
  tali che $\AbsoluteValue{a} > 1$ divergono.
\end{Theorem}
\Proof Quando $\AbsoluteValue{a} < 1$,
$(\AbsoluteValue{a}^n)_{n \in \mathbb{N}}$ \`e una successione decrescente;
quando invece $\AbsoluteValue{a} > 1$,
$(\AbsoluteValue{a}^n)_{n \in \mathbb{N}}$ \`e una successione crescente.
\begin{Theorem}
	\TheoremName{Teorema del confronto o dei carabinieri}[del confronto o dei carabinieri][teorema]
	Siano
  $(a_n)_{n \in \mathbb{N}},
  (b_n)_{n \in \mathbb{N}},
  (c_n)_{n \in \mathbb{N}} \in \mathbb{R}^\mathbb{N}$.
  Se abbiamo
	\begin{itemize}
		\item $\ForAll{n \in \mathbb{N}}{a_n \leq b_n \leq c_n}$;
		\item $\lim_{n \rightarrow \infty} a_n = \lim_{n \rightarrow \infty} c_n$;
	\end{itemize}
	allora
  $\lim_{n \rightarrow \infty} a_n =
  \lim_{n \rightarrow \infty} b_n =
  \lim_{n \rightarrow \infty} c_n$.
\end{Theorem}
\Proof Fissiamo $\epsilon > 0$ e poniamo
$L = \lim_{n \rightarrow \infty} a_n = \lim_{n \rightarrow \infty} c_n$.
Per $N \in \mathbb{N}$ abbastanza grande, abbiamo che $n > N$ implica
\begin{itemize}
	\item $L - \epsilon < a_n < L + \epsilon$;
	\item $L - \epsilon < c_n < L + \epsilon$.
\end{itemize}
Abbiamo altres\`i
$L - \epsilon < a_n \leq b_n \leq c_n < L + \epsilon$.
Dunque $\lim_{n \rightarrow \infty} b_n = L$.
\begin{Definition}
  Sia $(a_n)_{n \in \mathbb{N}} \in \mathbb{R}^\mathbb{N}$.
  Definiamo
  \begin{itemize}
    \item \Define{limite superiore}[superiore][limite]
      di $(a_n)_{n \in \mathbb{N}}$ il limite della successione
      $(\sup_{k \geq n} a_k)_{n \in \mathbb{N}}$,
      denotato $\limsup_{n \rightarrow \infty} a_n$;
    \item \Define{limite inferiore}[inferiore][limite]
      di $(a_n)_{n \in \mathbb{N}}$ il limite della successione
      $(\inf_{k \geq n} a_k)_{n \in \mathbb{N}}$,
      denotato $\liminf_{n \rightarrow \infty} a_n$.
  \end{itemize}
\end{Definition}
\begin{Theorem}
  Sia $(a_n)_{n \in \mathbb{N}} \in \mathbb{R}^\mathbb{N}$.
  Abbiamo
  \begin{align*}
    \limsup_{n \rightarrow \infty} a_n
    &= \inf_{n \in \mathbb{N}} \sup_{k \geq n} a_k,\\
    \liminf_{n \rightarrow \infty} a_n
    &= \sup_{n \in \mathbb{N}} \inf_{k \geq n} a_k,\\
    \liminf_{n \rightarrow \infty} a_n
    &\leq \limsup_{n \in \rightarrow \infty} a_n.
  \end{align*}
\end{Theorem}
\Proof La prima relazione Segue direttamente dal fatto che la successione
$(\sup_{k \in n} a_k)_{n \in \mathbb{N}}$
\`e decrescente.
\par La seconda relazione segue dal fatto che la successione
$(\inf_{k \in n} a_k)_{n \in \mathbb{N}}$ \`e
crescente.
\par Dimostriamo la terza relazione. Per ogni $n \in \mathbb{N}$, abbiamo
$\inf_{k \geq n} a_k \leq \sup_{k \in n} a_k$,
da cui
$\sup_{n \in \mathbb{N}} \inf_{k \geq n} a_k \leq \sup_{k \in n} a_k$
e quindi
$\sup_{n \in \mathbb{N}} \inf_{k \geq n} a_k
\leq \inf_{n \in \mathbb{N}} \sup_{k \in n} a_k$,
equivalente proprio a
$\liminf_{n \rightarrow \infty} a_n
\leq \limsup_{n \in \rightarrow \infty} a_n$. \EndProof
\begin{Theorem}
  Sia $(a_n)_{n \in \mathbb{N}} \in \mathbb{R}^\mathbb{N}$.
  La successione $(a_n)_{n \in \mathbb{N}}$
  \begin{itemize}
    \item converge a limite finito $L \in \mathbb{R}$ se e solo se
    $\liminf_{n \rightarrow \infty} a_n
      = \limsup_{n \rightarrow \infty} a_n = L$;
    \item diverge positivamente se e solo se
    $\liminf_{n \rightarrow \infty} a_n
      = \limsup_{n \rightarrow \infty} a_n = + \infty$;
    \item diverge negativamente se e solo se
    $\liminf_{n \rightarrow \infty} a_n
      = \limsup_{n \rightarrow \infty} a_n = - \infty$.
  \end{itemize}
\end{Theorem}
\Proof Supponiamo $(a_n)_{n \in \mathbb{N}}$ converga a $L \in \mathbb{R}$.
Fissato $\epsilon > 0$, esiste $n \in \mathbb{N}$ tale che
$\Implies{k \geq n}{L \leq a_k < L + \epsilon}$.
Ne deduciamo
$L \leq \sup_{k \geq n} a_k \leq L + \epsilon$.
Per l'arbitrariet\`a di $\epsilon$, deve essere
$\limsup_{n \rightarrow \infty} a_k = L$.
Analogamente
$\liminf_{n \rightarrow \infty} a_k = L$.
\par Supponiamo
$\liminf_{n \rightarrow \infty} a_n
  = \limsup_{n \rightarrow \infty} a_n = L$.
Poich\'e, per ogni $n \in \mathbb{N}$,
$\inf_{k \geq n} a_k \leq a_n \leq \sup_{k \in n} a_k$,
dal teorema del confronto segue
$\lim_{n \rightarrow \infty} a_n = L$.
\par Supponiamo $(a_n)_{n \in \mathbb{N}}$ diverga positivamente.
Fissato $M > 0$, esiste $n \in \mathbb{N}$ tale che
$\Implies{k \geq n}{a_k > M}$.
Ne deduciamo
$\inf_{k \geq n}  a_k > M$
e
$\sup_{k \geq n}  a_k > M$.
Per l'arbitrariet\`a di $M$, deve essere
$\liminf_{n \rightarrow \infty} a_n
  = \limsup_{n \rightarrow \infty} a_n = + \infty$.
\par Supponiamo
$\liminf_{n \rightarrow \infty} a_n
  = \limsup_{n \rightarrow \infty} a_n = + \infty$.
Per ogni $n \in \mathbb{N}$, abbiamo
$\inf_{k \geq n} a_k \leq a_n$, dunque
$\liminf_{n \rightarrow \infty} a_n \leq \lim_{n \rightarrow \infty} a_n$
e quindi
$\lim_{n \rightarrow \infty} a_n = + \infty$.
\par Supponiamo $(a_n)_{n \in \mathbb{N}}$ diverga negativamente.
Fissato $M > 0$, esiste $n \in \mathbb{N}$ tale che
$\Implies{k \geq n}{a_k < - M}$.
Ne deduciamo
$\inf_{k \geq n}  a_k < - M$
e
$\sup_{k \geq n}  a_k < - M$.
Per l'arbitrariet\`a di $M$, deve essere
$\liminf_{n \rightarrow \infty} a_n
  = \limsup_{n \rightarrow \infty} a_n = - \infty$.
\par Supponiamo
$\liminf_{n \rightarrow \infty} a_n
  = \limsup_{n \rightarrow \infty} a_n = - \infty$.
Per ogni $n \in \mathbb{N}$, abbiamo
$\sup_{k \geq n} a_k \geq a_n$, dunque
$\limsup_{n \rightarrow \infty} a_n \geq \lim_{n \rightarrow \infty} a_n$
e quindi
$\lim_{n \rightarrow \infty} a_n = - \infty$. \EndProof
\begin{Definition}
	Siano
	\begin{itemize}
		\item $(\VarTopologicalSpace,\VarTopology)$ spazio topologico;
		\item $\Part \subseteq \mathbb{R}$ tale che $\sup{\Part} = + \infty$;
		\item $f: \Part \rightarrow \VarTopologicalSpace$;
		\item $\Limit \in \VarTopologicalSpace$.
	\end{itemize}
	Se $\ForAll{\Neighborhood_\Limit \in \Neighborhoods{\Limit} \cap \VarTopology}{\Exists{N > 0}{\Implies{x > N}{f(x) \in \Neighborhood_\Limit}}}$, allora diciamo che $f(x)$ \Define{converge}[di una funzione a $+\infty$][convergenza] a $\Limit$, in simboli $f(x) \rightarrow \Limit$, per $x$ che tende a $+\infty$, in simboli $x \rightarrow +\infty$. Chiamiamo $\Limit$ \Define{limite}[di una funzione a $+\infty$][limite] di $f(x)$ per $x \rightarrow \infty$, denotato $\lim_{x \rightarrow +\infty} f(x) = \Limit$. In tal caso diciamo inoltre che la retta $y = \Limit$ \`e \Define{asintoto orizzontale positivo}[orizzontale positivo][asintoto] del grafico della funzione $y = f(x)$ nel piano $\OxyPlane$.
\end{Definition}
\begin{Definition}
	Siano
	\begin{itemize}
		\item $(\VarTopologicalSpace,\VarTopology)$ spazio topologico;
		\item $\Part \subseteq \mathbb{R}$ tale che $\inf{\Part} = - \infty$;
		\item $f: \Part \rightarrow \VarTopologicalSpace$;
		\item $\Limit \in \VarTopologicalSpace$.
	\end{itemize}
	Se $\ForAll{\Neighborhood_\Limit \in \Neighborhoods{\Limit} \cap \VarTopology}{\Exists{N > 0}{\Implies{x < -N}{f(x) \in \Neighborhood_\Limit}}}$, allora diciamo che $f(x)$ \Define{converge}[di una funzione a $-\infty$][convergenza] a $\Limit$, in simboli $f(x) \rightarrow \Limit$, per $x$ che tende a $-\infty$, in simboli $x \rightarrow -\infty$. Chiamiamo $\Limit$ \Define{limite}[di una funzione a $-\infty$][limite] di $f(x)$ per $x \rightarrow \infty$, denotato $\lim_{x \rightarrow -\infty} f(x) = \Limit$. In tal caso diciamo inoltre che la retta $y = \Limit$ \`e \Define{asintoto orizzontale negativo}[orizzontale negativo][asintoto] del grafico della funzione $y = f(x)$ nel piano $\OxyPlane$.
\end{Definition}
\begin{Theorem}
	Siano
	\begin{itemize}
		\item $(\VarTopologicalSpace,\VarTopology)$ spazio topologico;
		\item $\Part \subseteq \mathbb{R}$ tale che $\sup{\Part} = - \infty$;
		\item $f: \Part \rightarrow \VarTopologicalSpace$;
		\item $\Limit \in \VarTopologicalSpace$.
	\end{itemize}
	Abbiamo $\lim_{x \rightarrow +\infty} f(x) = \Limit$ per qualche $\Limit \in \VarTopologicalSpace$ se e solo se $\lim_{x \rightarrow -\infty} f(-x) = \Limit$,
\end{Theorem}
\Proof $\lim_{x \rightarrow +infty} f(x) = \Limit$ equivale a $\ForAll{\Neighborhood_\Limit \in \Neighborhoods{\Limit} \cap \VarTopology}{\Exists{N > 0}{\Implies{x > N}{f(x) \in \Neighborhood_\Limit}}}$, a sua volta equivalente a $\ForAll{\Neighborhood_\Limit \in \Neighborhoods{\Limit} \cap \VarTopology}{\Exists{N > 0}{\Implies{x < - N}{f(-x) \in \Neighborhood_\Limit}}}$ e quindi a $\lim_{x \rightarrow -\infty} f(-x) = \Limit$. \EndProof
\begin{Theorem}
	Siano
	\begin{itemize}
		\item $(\VarTopologicalSpace,\VarTopology)$ spazio topologico;
		\item una successione $(a_n)_{n \in \mathbb{N}} \in \mathbb{R}^\mathbb{N}$ divergente positivamente;
		\item $f: \bigcup_{n \in \mathbb{N}} \lbrace a_n \rbrace \rightarrow \VarTopologicalSpace$.
	\end{itemize}
	Se \`e vero l'assioma della scelta, $(f(a_n))_{n \in \mathbb{N}}$ converge a $\Limit$ se e solo se $f(x)$ converge a $\Limit$ per $x \rightarrow +\infty$.
\end{Theorem}
\Proof Supponiamo $f(x) \rightarrow \Limit$ per $x \rightarrow +\infty$. Sia $\Neighborhood_\Limit \in \Neighborhoods{\Limit} \cap \VarTopology$: per opportuno $N > 0$, abbiamo $\Implies{x > N}{f(x) \in \Neighborhood_\Limit}$. Ma $a_n \rightarrow +\infty$ per $n \rightarrow \infty$ e quindi esiste $N_0 \in \mathbb{N}$ tale che $\Implies{n > N_0}{a_n > N}$, da cui $\Implies{n > N_0}{f(a_n) \in \Neighborhood_\Limit}$. Ne deduciamo $f(a_n) \rightarrow \Limit$ per $n \rightarrow \infty$.
\par Supponiamo ora $\lim_{n \rightarrow \infty} f(a_n) = \Limit$. Supponiamo inoltre che $f(x)$ non converga a $\Limit$ per $x \rightarrow \infty$. Allora esiste $\Neighborhood_\Limit \in \Neighborhoods{\Limit} \cap \VarTopology$ tale che $\ForAll{N > 0}{\Exists{m \in \mathbb{N}}{\Implies{a_m > N}{f(a_m) \notin \Neighborhood_\Limit}}}$.
\par Possiamo quindi definire la sottosuccessione $(a_{g(n)})_{n \in \mathbb{N}} \in \mathbb{R}^{\mathbb{N}}$, con $g: \mathbb{N} \rightarrow \mathbb{N}$ applicazione strettamente crescente, tale che $\ForAll{n \in \mathbb{N}}{f(a_{g(n)}) \notin \Neighborhood_\Limit}$. Allora
\begin{itemize}
	\item $(f(a_{g(n)}))_{n \in \mathbb{N}}$ converge a $\Limit$ in quanto sottosuccessione di $(f(a_n))_{n \in \mathbb{N}}$;
	\item $(f(a_{g(n)}))_{n \in \mathbb{N}}$ non converge a $\Limit$ perch\'e $\ForAll{n \in \mathbb{N}}{f(a_{g(n)}) \notin \Neighborhood_\Limit}$.
\end{itemize}
Ma questo \`e assurdo, dunque $f(x)$ deve converge a $\Limit$ per $x \rightarrow +\infty$. \EndProof
\begin{Corollary}
	Siano
	\begin{itemize}
		\item $(\VarTopologicalSpace,\VarTopology)$ spazio topologico;
		\item una successione $(a_n)_{n \in \mathbb{N}} \in \mathbb{R}^\mathbb{N}$ divergente negativamente;
		\item $f: \bigcup_{n \in \mathbb{N}} \lbrace a_n \rbrace \rightarrow \VarTopologicalSpace$.
	\end{itemize}
	Se \`e vero l'assioma della scelta, $(f(a_n))_{n \in \mathbb{N}}$ converge a $\Limit$ se e solo se $f(x)$ converge a $\Limit$ per $x \rightarrow -\infty$.
\end{Corollary}
\Proof Segue dal teorema precedente, moltiplicando per $(-1)$ la successione $a_n$ e sostituendo a $f(x)$ la funzione $f(-x)$. \EndProof
\begin{Definition}
	Siano
	\begin{itemize}
		\item $\Part \subseteq \mathbb{R}$ tale che $\sup{\Part} = + \infty$;
		\item $f: \Part \rightarrow \mathbb{R}$.
	\end{itemize}
	Se $\ForAll{M > 0}{\Exists{N > 0}{\Implies{x > N}{f(x) > M}}}$, allora diciamo che $f(x)$ \Define{diverge}[positia di una funzione a $+\infty$][divergenza] a $+\infty$, in simboli $f(x) \rightarrow +\infty$, per $x$ che tende a $+\infty$, in simboli $x \rightarrow +\infty$; diciamo anche che $+\infty$ \`e il \Define{limite}[infinito positivo di una funzione a $+\infty$][limite], denotato $\lim_{x \rightarrow +\infty} f(x) = +\infty$.
	\par Se invece $\ForAll{M > 0}{\Exists{N > 0}{\Implies{x > N}{f(x) < - M}}}$, allora diciamo che $f(x)$ \Define{diverge}[negativa di una funzione a $+\infty$][divergenza] a $-\infty$, in simboli $f(x) \rightarrow -\infty$, per $x$ che tende a $+\infty$, in simboli $x \rightarrow +\infty$; diciamo anche che $-\infty$ \`e il \Define{limite}[infinito negativo di una funzione a $+\infty$][limite], denotato $\lim_{x \rightarrow +\infty} f(x) = -\infty$.
\end{Definition}
\begin{Definition}
	Siano
	\begin{itemize}
		\item $\Part \subseteq \mathbb{R}$ tale che $\sup{\Part} = - \infty$;
		\item $f: \Part \rightarrow \mathbb{R}$.
	\end{itemize}
	Se $\ForAll{M > 0}{\Exists{N > 0}{\Implies{x < - N}{f(x) > M}}}$, allora diciamo che $f(x)$ \Define{diverge}[positiva di una funzione a $-\infty$][divergenza] a $+\infty$, in simboli $f(x) \rightarrow +\infty$, per $x$ che tende a $-\infty$, in simboli $x \rightarrow -\infty$; diciamo anche che $+\infty$ \`e il \Define{limite}[infinito positivo di una funzione a $-\infty$][limite], denotato $\lim_{x \rightarrow -\infty} f(x) = +\infty$.
	\par Se invece $\ForAll{M > 0}{\Exists{N > 0}{\Implies{x < - N}{f(x) < - M}}}$, allora diciamo che $f(x)$ \Define{diverge}[negativa di una funzione a $-\infty$][divergenza] a $-\infty$, in simboli $f(x) \rightarrow -\infty$, per $x$ che tende a $-\infty$, in simboli $x \rightarrow -\infty$; diciamo anche che $-\infty$ \`e il \Define{limite}[infinito negativo di una funzione a $-\infty$][limite], denotato $\lim_{x \rightarrow -\infty} f(x) = -\infty$.
\end{Definition}
\begin{Theorem}
	Siano
	\begin{itemize}
		\item $\Part \subseteq \mathbb{R}$ tale che $\sup{\Part} = - \infty$;
		\item $f: \Part \rightarrow \mathbb{R}$.
	\end{itemize}
	Sono equivalenti
	\begin{itemize}
		\item $\lim_{x \rightarrow +\infty} f(x) = +\infty$;
		\item $\lim_{x \rightarrow -\infty} f(-x) = +\infty$;
		\item $\lim_{x \rightarrow +\infty} -f(x) = -\infty$;
		\item $\lim_{x \rightarrow -\infty} f(-x) = -\infty$.
	\end{itemize}
\end{Theorem}
\Proof $\lim_{x \rightarrow +infty} f(x) = +\infty$ equivale a $\ForAll{M > 0}{\Exists{N > 0}{\Implies{x > N}{f(x) < M}}}$. Quest'ultima affermazione \`e equivalente a
\begin{itemize}
	\item $\ForAll{M > 0}{\Exists{N > 0}{\Implies{x < - N}{f(-x) > M}}}$ e quindi a $\lim_{x \rightarrow -\infty} f(-x) = +\infty$;
	\item $\ForAll{M > 0}{\Exists{N > 0}{\Implies{x > N}{- f(x) < - M}}}$ e quindi a $\lim_{x \rightarrow +\infty} - f(x) = -\infty$;
	\item $\ForAll{M > 0}{\Exists{N > 0}{\Implies{x > N}{- f(-x) < - M}}}$ e quindi a $\lim_{x \rightarrow -\infty} - f(x) = -\infty$. \EndProof
\end{itemize}
\begin{Theorem}
	Siano
	\begin{itemize}
		\item una successione $(a_n)_{n \in \mathbb{N}} \in \mathbb{R}^\mathbb{N}$ divergente positivamente;
		\item $f: \bigcup_{n \in \mathbb{N}} \lbrace a_n \rbrace \rightarrow \mathbb{R}$.
	\end{itemize}
	Se \`e vero l'assioma della scelta, $(f(a_n))_{n \in \mathbb{N}}$ diverge positivamente se e solo se $f(x)$ diverge positivamente per $x \rightarrow +\infty$.
\end{Theorem}
\Proof Supponiamo $f(x) \rightarrow +\infty$ per $x \rightarrow +\infty$. Sia $M > 0$: per opportuno $N > 0$, abbiamo $\Implies{x > N}{f(x) > M}$. Ma $a_n \rightarrow +\infty$ per $n \rightarrow \infty$ e quindi esiste $N_0 \in \mathbb{N}$ tale che $\Implies{n > N_0}{a_n > N}$, da cui $\Implies{n > N_0}{f(a_n) < M}$. Ne deduciamo $f(a_n) \rightarrow +\infty$ per $n \rightarrow +\infty$.
\par Supponiamo ora $\lim_{n \rightarrow \infty} f(a_n) = +\infty$. Supponiamo inoltre che $f(x)$ non diverga positivamente per $x \rightarrow \infty$. Allora esiste $M > 0$ tale che $\ForAll{N > 0}{\Exists{m \in \mathbb{N}}{\Implies{a_m > N}{f(a_m) \leq M}}}$.
\par Possiamo quindi definire la sottosuccessione $(a_{g(n)})_{n \in \mathbb{N}} \in \mathbb{R}^{\mathbb{N}}$, con $g: \mathbb{N} \rightarrow \mathbb{N}$ applicazione strettamente crescente, tale che $\ForAll{n \in \mathbb{N}}{f(a_{g(n)}) \leq M}$. Allora
\begin{itemize}
	\item $(f(a_{g(n)}))_{n \in \mathbb{N}}$ diverge positivamente in quanto sottosuccessione di $(f(a_n))_{n \in \mathbb{N}}$;
	\item $(f(a_{g(n)}))_{n \in \mathbb{N}}$ non diverge positivamente perch\'e $\ForAll{n \in \mathbb{N}}{f(a_{g(n)}) \leq M}$.
\end{itemize}
Ma questo \`e assurdo, dunque $f(x)$ diverge positivamente per $x \rightarrow +\infty$. \EndProof
\begin{Corollary}
	Siano
	\begin{itemize}
		\item una successione $(a_n)_{n \in \mathbb{N}} \in \mathbb{R}^\mathbb{N}$ divergente negativamente;
		\item $f: \bigcup_{n \in \mathbb{N}} \lbrace a_n \rbrace \rightarrow \mathbb{R}$.
	\end{itemize}
	Se \`e vero l'assioma della scelta, $(f(a_n))_{n \in \mathbb{N}}$ diverge positivamente se e solo se $f(x)$ diverge positivamente per $x \rightarrow -\infty$.
\end{Corollary}
\Proof Segue dal teorema precedente, moltiplicando per $(-1)$ la successione $a_n$ e sostituendo a $f(x)$ la funzione $f(-x)$. \EndProof
\begin{Corollary}
	Siano
	\begin{itemize}
		\item una successione $(a_n)_{n \in \mathbb{N}} \in \mathbb{R}^\mathbb{N}$ divergente positivamente;
		\item $f: \bigcup_{n \in \mathbb{N}} \lbrace a_n \rbrace \rightarrow \mathbb{R}$.
	\end{itemize}
	Se \`e vero l'assioma della scelta, $(f(a_n))_{n \in \mathbb{N}}$ diverge negativamente se e solo se $f(x)$ diverge negativamente per $x \rightarrow +\infty$.
\end{Corollary}
\Proof Segue dal teorema precedente, sostituendo a $f(x)$ la funzione $-f(x)$. \EndProof
\begin{Corollary}
	Siano
	\begin{itemize}
		\item una successione $(a_n)_{n \in \mathbb{N}} \in \mathbb{R}^\mathbb{N}$ divergente negativamente;
		\item $f: \bigcup_{n \in \mathbb{N}} \lbrace a_n \rbrace \rightarrow \mathbb{R}$.
	\end{itemize}
	Se \`e vero l'assioma della scelta, $(f(a_n))_{n \in \mathbb{N}}$ diverge negativamente se e solo se $f(x)$ diverge negativamente per $x \rightarrow -\infty$.
\end{Corollary}
\Proof Segue dal teorema precedente, sostituendo a $f(x)$ la funzione $-f(-x)$. \EndProof
\begin{Definition}
  Dati $x, n \in \mathbb{N}$
  \begin{itemize}
    \item \Define{fattoriale crescente}[crescente][fattoriale]
      di $x$ con $n$ fattori, denotato
      $\RisingFactorialBar{x}{n}$ o anche
      $\RisingFactorialPochammer{x}{n}$ o ancora
      $\RisingFactorialSupscript{x}{n}$,
      la quantit\`a
      $\RisingFactorialBar{x}{n} =
      \begin{cases}
        0,&\text{ se }n = 0,\\
        \prod_{k \in n} (x + k),&\text{ se }n = 0;
      \end{cases}$
    \item \Define{fattoriale decrescente}[decrescente][fattoriale]
      di $x$ con $n$ fattori, denotato
      $\FallingFactorialBar{x}{n}$ o anche
      $\FallingFactorialPochammer{x}{n}$,
      la quantit\`a
      $\FallingFactorialBar{x}{n} =
      \begin{cases}
        0,&\text{ se }n = 0,\\
        \prod_{k \in n} (x - k),&\text{ se }n = 0;
      \end{cases}$.
  \end{itemize}
  Si osservi che la notazione $\FactorialPochammer{x}{n}$, detta
  \Define{simbolo di Pochammer}[di Pochammer][simbolo],
  \`e utilizzata sia per il fattoriale crescente che per quello
  decrescente.
\end{Definition}
\begin{Theorem}
  \TheoremName{Formula di Faulhaber}[di Faulhaber][formula]
  Siano $n \in \mathbb{N}$ e $p \in \NotZero{\mathbb{N}}$. Abbiamo
  \[
    \sum_{k = 1}^n =
    \frac{n^{p + 1}}{p + 1} + \frac{n^p}{2}
    + \sum_{k = 2}^p \frac{\BernoulliNumber_k}{k!} \FallingFactorial{p}{k - 1}
      n^{p - k + 1}.
  \]
\end{Theorem}
\begin{Theorem}
  Sia $n \in \mathbb{N}$. Abbiamo
  \begin{itemize}
    \item $\sum_{k \in n} k^2 = \frac{n(n - 1)(2n - 1)}{6}$;
    \item $\sum_{k \in n} k^3 = \frac{n^2(n - 1)^2}{4}$.
  \end{itemize}
\end{Theorem}
