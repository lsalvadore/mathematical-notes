\section{Definizioni e teoremi di base.}
\label{AnalisiComplessa_DefinizioniETeoremiDiBase}
\begin{Definition}
  Siano
  \begin{itemize}
    \item $\Open \subseteq \mathbb{C}$ aperto;
    \item $z_0 \in \Open$;
    \item $\Holomorphic: \Open \rightarrow \mathbb{C}$.
  \end{itemize}
  $f$ si dice
  \begin{itemize}
    \item \Define{olomorfa}[olomorfa][funzione]
      quando \`e derivabile in ogni punto di $\Open$;
    \item \Define{olomorfa in $z_0$}[olomorfa in un punto][funzione]
      quando \`e olomorfa in un intorno di $z_0$.
  \end{itemize}
\end{Definition}
\begin{Definition}
  Siano
  \begin{itemize}
    \item $\Open \subseteq \mathbb{C}$ aperto;
    \item $\Singularity \in \Open$;
    \item $\Holomorphic: \Open \SetMin \lbrace \Singularity \rbrace
            \rightarrow \mathbb{C}$ olomorfa.
  \end{itemize}
  $\Singularity$ \`e una
  \Define{singolarit\`a isolata}[isolata][singolarit\`a]
  di $\Holomorphic$.
  $\Singularity$ \`e
  \begin{itemize}
    \item una \Define{singolarit\`a eliminabile}[elimibabile][singolarit\`a] se
      $\Holomorphic(z)$ converge per $z \rightarrow \Singularity$;
    \item un \Define{polo}[di una funzione olomorfa][polo]
      se
      $\lim_{z \rightarrow \Singularity}
        \AbsoluteValue{\Holomorphic(z)} = + \infty$;
    \item una \Define{singolarit\`a essenziale}[essenziale][singolarit\`a]
      se non esiste il limite di $\AbsoluteValue{\Holomorphic(z)}$ per
      $z \rightarrow \Singularity$.
  \end{itemize}
\end{Definition}
\begin{Definition}
  Siano
  \begin{itemize}
    \item $\Open \subseteq \mathbb{C}$ aperto;
    \item $\Part \subseteq \Open$ costituito da punti isolati;
    \item $\Meromorphic: \Open \SetMin \Part \rightarrow \mathbb{C}$;
    \item $z_0 \in \Open$.
  \end{itemize}
  $f$ si dice
  \begin{itemize}
    \item \Define{meromorfa}[meromorfa][funzione]
      quando \`e olomorfa in ogni punto di $\Open$ tranne che nei
      punti di $\Part$ che sono tutti poli;
    \item \Define{meromorfa in $z_0$}[meromorfa in un punto][funzione]
      quando \`e meromorfa in un intorno di $z_0$.
  \end{itemize}
\end{Definition}
\begin{Theorem}
  Siano
  \begin{itemize}
    \item $\Open \subseteq \mathbb{C}$ aperto;
    \item $\Holomorphic: \Open \rightarrow \mathbb{C}$.
  \end{itemize}
  $\Holomorphic$ \`e olomorfa se e solo se \`e analitica.
\end{Theorem}
\begin{Theorem}
  Siano
  \begin{itemize}
    \item $n \in \NotZero{\mathbb{N}}$;
    \item $\Matrix \in \mathbb{C}^{n \times n}$ diagonalizzabile;
    \item $\Analytical: \mathbb{C} \rightarrow \mathbb{C}$ analitica.
  \end{itemize}
  $\Eigenvalue \in \Spectrum{\Matrix}$ se e solo se
  $\Analytical(\Eigenvalue) \in \Spectrum{\Analytical(\Matrix)}$.
\end{Theorem}
\Proof Sia $\SpectralBaseChange$ una matrice che diagonalizza $\Matrix$.
$\SpectralBaseChange$ diagonalizza anche $\Analytical{\Matrix}$:
\begin{align*}
  \SpectralBaseChange\Analytical(\Matrix)\SpectralBaseChange^{-1}
  &=\SpectralBaseChange
  \left (
    \sum_{k \in \mathbb{N}} \frac{\Analytical^{(k)}(\Matrix)\Matrix^k}{k!}
  \right )
  \SpectralBaseChange^{-1},\\
  &=
  \left (
    \sum_{k \in \mathbb{N}} \frac{\Analytical^{(k)}(
    (\SpectralBaseChange\Matrix\SpectralBaseChange^{-1})^k)}
    {k!}
  \right ),\\
  &= \Analytical(\SpectralBaseChange\Matrix\SpectralBaseChange^{-1}).
\end{align*}
\par Ne consegue che gli elementi diagonali di
$\SpectralBaseChange\Analytical(\Matrix)\SpectralBaseChange^{-1}$,
vale a dire gli autovalori di $\Analytical(\Matrix)$, sono esattamente le
immagini degli elementi diagonali di
$\SpectralBaseChange\Matrix\SpectralBaseChange^{-1}$.
vale a dire degli autovalori di $\Matrix$. \EndProof
