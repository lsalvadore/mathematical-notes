\section{Operatori lineari limitati.}
\label{AnalisiFunzionale_OperatoriLineariLimitati}
\begin{Definition}
	Sia $\BoundedLinearOperator: \LinearTopologicalSpace \rightarrow \VarLinearTopologicalSpace$ un operatore lineare. $\BoundedLinearOperator$ si dice \Define{limitato}[lineare limitato][operatore] quando trasforma parti limitate in parti limitate. Denoteremo lo spazio di tutte gli operatori lineari da $\LinearTopologicalSpace$ in $\VarLinearTopologicalSpace$ con $\BoundedLinearOperators{\LinearTopologicalSpace}{\VarLinearTopologicalSpace}$.
\end{Definition}
\begin{Theorem}
	Sia $\BoundedLinearOperator: \LinearTopologicalSpace \rightarrow \VarLinearTopologicalSpace$ un operatore lineare. Se $\LinearTopologicalSpace$ \`e finito, allora $\BoundedLinearOperator$ \`e limitato.
\end{Theorem}
\begin{Theorem}
	Sia $\LinearOperator: \NormedSpace \rightarrow \VarNormedSpace$ un operatore lineare tra gli spazi vettoriali normati $\NormedSpace$ e $\VarNormedSpace$. $\LinearOperator$ \`e limitato se e solo se \`e continuo.
\end{Theorem}
\begin{Definition}
	Fissato $n \in \mathbb{N}$, una \Define{norma di matrice}[di matrice][norma] \`e una norma $\Norm{\cdot}$ sullo spazio vettoriale $\mathbb{C}^n$ che \`e inoltre submoltiplicativa: $\ForAll{(\Matrix,\VarMatrix) \in \mathbb{C}^n \times \mathbb{C}^n}{\Norm{\Matrix\VarMatrix} \leq \Norm{\Matrix}\Norm{\VarMatrix}}$.
\end{Definition}
\begin{Theorem}
	Siano $\NormedSpace$ e $\VarNormedSpace$ spazi vettoriali normali. L'applicazione $\Norm{\cdot}: \BoundedLinearOperators{\NormedSpace}{\VarNormedSpace} \rightarrow \RealNonNegative$ che a $\BoundedLinearOperator \in \BoundedLinearOperators{\NormedSpace}{\VarNormedSpace}$ associa $\min \lbrace M \in \RealNonNegative | \ForAll{\Vector \in \NormedSpace}{\Norm{\LinearOperator} \leq M \Norm{\Vector}} \rbrace \in \RealNonNegative$ \`e una norma sullo spazio vettoriale $\BoundedLinearOperators{\NormedSpace}{\VarNormedSpace}$.
\end{Theorem}
\begin{Definition}
	Con le notazioni del teorema precedente, definiamo $\Norm{\BoundedLinearOperator}$ \Define{norma operatore}[di un operatore lineare limitato][norma] di $\BoundedLinearOperator$.
\end{Definition}
\par Ogni volta che tratteremo uno spazio di operatori lineare limitati come uno spazio vettoriale normato, si tratter\`a sempre della norma appena definita, salvo avvisi contrari.
\begin{Theorem}
	Le norme operatore sono norme di matrici.
\end{Theorem}
\begin{Definition}
	Una norma di matrice si dice \Define{indotta}[di matrice indotta][norma] quando \`e una norma operatore.
\end{Definition}
\begin{Theorem}
	Siano $\NormedSpace$ e $\VarNormedSpace$ spazi vettoriali normati.
  Sia
  $\BoundedLinearOperator \in
    \BoundedLinearOperators{\NormedSpace}{\VarNormedSpace}$.
  Abbiamo
	\begin{itemize}
		\item $\Norm{\BoundedLinearOperator} = \sup_{\substack{\Vector \in \NormedSpace\\\Norm{\Vector} \leq 1}} \Norm{\BoundedLinearOperator \Vector}$;
		\item $\Norm{\BoundedLinearOperator} = \sup_{\substack{\Vector \in \NormedSpace\\\Norm{\Vector} < 1}} \Norm{\BoundedLinearOperator \Vector}$;
		\item $\Norm{\BoundedLinearOperator} = \sup_{\substack{\Vector \in \NormedSpace\\\Norm{\Vector} = 1}} \Norm{\BoundedLinearOperator \Vector}$;
		\item $\Norm{\BoundedLinearOperator} = \sup_{\substack{\Vector \in \NormedSpace\\\Vector \neq 0}} \frac{\Norm{\BoundedLinearOperator \Vector}}{\Norm{\Vector}}$.
	\end{itemize}
\end{Theorem}
\begin{Corollary}
	Siano $\NormedSpace$ e $\VarNormedSpace$ spazi vettoriali normati.
  Sia
  $\BoundedLinearOperator \in
    \BoundedLinearOperators{\NormedSpace}{\VarNormedSpace}$.
  Abbiamo
  \[
    \ForAll{\Vector \in \NormedSpace}{
      \Norm{\BoundedLinearOperator \Vector} \leq
      \Norm{\BoundedLinearOperator} \Norm{\Vector}}.
  \]
\end{Corollary}
\Proof Se $\Vector = 0$ la tesi \`e immediata. Assumiamo $\Vector \neq 0$.
\par Abbiamo
\begin{align*}
  \Norm{\BoundedLinearOperator \Vector}
  &= \Norm{\BoundedLinearOperator
    \Norm{\Vector} \frac{\Vector}{\Norm{\Vector}}},\\
  &= \Norm{\Vector}
    \Norm{\BoundedLinearOperator \frac{\Vector}{\Norm{\Vector}}},\\
  &\leq \Norm{\Vector} \Norm{\BoundedLinearOperator}.\text{ \EndProof}
\end{align*}
\begin{Theorem}
  Fissato $n \in \NotZero{\mathbb{N}}$, abbiamo, per ogni
  $\Matrix \in \mathbb{C}^{n \times n}$
  \begin{itemize}
    \item $\Norm{\Matrix}[1]
      = \max_{j \in n} \sum_{k \in n} \AbsoluteValue{\Matrix_k^j}$;
    \item $\Norm{\Matrix}[2]
      = \sqrt{\SpectralRadius{\HermitianTransposed{\Matrix}\Matrix}}$;
    \item $\Norm{\Matrix}[\infty]
      = \max_{k \in n} \sum_{j \in n} \AbsoluteValue{\Matrix_k^j}$.
  \end{itemize}
\end{Theorem}
\begin{Theorem}
	Siano $\NormedSpace$ e $\VarNormedSpace$ spazi vettoriali normati finiti. Sia $\BoundedLinearOperator \in \BoundedLinearOperators{\NormedSpace}{\VarNormedSpace}$. Per ogni $\epsilon > 0$, esiste una norma indotta di $\BoundedLinearOperators{\NormedSpace}{\VarNormedSpace}$ tale che $\SpectralRadius{\BoundedLinearOperator} \leq \Norm{\BoundedLinearOperator} \leq \SpectralRadius{\BoundedLinearOperator} + \epsilon$.
\end{Theorem}
