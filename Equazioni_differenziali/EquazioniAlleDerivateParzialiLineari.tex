\subsection{Equazioni alle derivate parziali lineari.}
\label{EquazioniDifferenziali_EquazioniAlleDerivateParzialiLineari}
\begin{Definition}
  Siano
  \begin{itemize}
    \item $\OpenConnected \subseteq \mathbb{R}^2$ un aperto connesso;
    \item $u: \Closure{\OpenConnected} \rightarrow \mathbb{R}$ incognita;
    \item $f: \OpenConnected \rightarrow \mathbb{R}$ nota e
      sufficientemente regolare;
    \item $A, B, C, D, E, F: \OpenConnected \rightarrow \mathbb{R}$;
    \item $L$ un'operatore differenziale lineare sulle funzioni di classe
      $\CClass{2}[\Closure{\OpenConnected}][\mathbb{R}]$ definito da
      \[
        L[u]
        = A \frac{\partial^2 u}{\partial x^2}
        + B \frac{\partial^2 u}{\partial x \partial y}
        + C \frac{\partial^2 u}{\partial y^2}
        + D \frac{\partial u}{\partial x}
        + E \frac{\partial u}{\partial y}
        + Fu.
      \]
  \end{itemize}
  L'equazione $L[u] = f$ si dice
  \begin{itemize}
    \item \Define{ellittica}[differenziale ellittica][equazione]
      quando $\ForAll{(x,y) \in \OpenConnected}{B^2 - 4AC < 0}$;
    \item \Define{parabolica}[differenziale parabolica][equazione]
      quando $\ForAll{(x,y) \in \OpenConnected}{B^2 - 4AC = 0}$;
    \item \Define{iperbolica}[differenziale iperbolica][equazione]
      quando $\ForAll{(x,y) \in \OpenConnected}{B^2 - 4AC > 0}$.
  \end{itemize}
\end{Definition}
\begin{Theorem}
  Con le notazioni della definizione precedente, $L[u] = f$ \`e
  ellittica, parabolica o iperbolica se e solo se, per ogni
  $(x,y) \in \mathbb{R}^2$, la conica $A\xi^2 + B\xi\eta + C\eta^2 = 0$
  nel piano $\xi\eta$ \`e un'ellisse, una parabola o un'iperbole
  rispettivamente.
\end{Theorem}
\Proof Segue direttamente dalle definizioni e dal teorema di classificazione
delle coniche. \EndProof
