\subsection{Equazioni lineari del secondo ordine.}
\label{EquazioniDifferenziali_EquazioniLineariDelSecondoOrdine}
\begin{Theorem}
  Siano
  \begin{itemize}
    \item $a, b, c \in \mathbb{C}$;
    \item $\lambda_1 \in \mathbb{R}$ una soluzione di
        $a\lambda^2 + b\lambda + c$, se esiste;
    \item $\lambda_2 \in \mathbb{R} \SetMin \lbrace \lambda_1 \rbrace$
        un'ulteriore soluzione di $a\lambda^2 + b\lambda + c$, se esiste;
    \item $\alpha = \RealPart{\tilde{\lambda}}$ e
        $\beta = \ImaginaryPart{\tilde{\lambda}}$, dove
        $\tilde{\lambda} \in \mathbb{C} \SetMin \mathbb{R}$
        \`e una soluzione di
        $a\lambda^2 + b\lambda + c$, se esiste.
  \end{itemize}
  L'equazione in $x(t)$
  \[
    a\ddot{x} + b\dot{x} + cx = 0
  \]
  ammette le seguenti soluzioni particolari:
  \begin{itemize}
    \item $e^{\lambda_1 t}$ e $e^{\lambda_2 t}$
      se $a\lambda^2 + b\lambda + c = 0$
      ammette due soluzioni reali distinte;
    \item $e^{\lambda_1 t}$ e $e^{\lambda_1 t}$
      se $a\lambda^2 + b\lambda + c = 0$
      ammette un'unica soluzione reale;
    \item $e^{\alpha t}\cos \beta t$ e
      $e^{\alpha t}\sin \beta t$
      se $a\lambda^2 + b\lambda + c = 0$
      ammette soluzioni coniugate complesse,
  \end{itemize}
\end{Theorem}
\Proof Supponiamo $\lambda_1$ sia una soluzione reale di
$a\lambda^2 + b\lambda + c = 0$. Abbiamo
\begin{align*}
  a \frac{d^2 e^{\lambda_1 t}}{dt^2}
  + b \frac{d e^{\lambda_1 t}}{dt}
  + c e^{\lambda_1 t}
  &= a \lambda_1^2 e^{\lambda_1 t}
  + b \lambda_1 e^{\lambda_1 t}
  + c e^{\lambda_1 t},\\
  &= e^{\lambda_1 t} (a \lambda_1^2 + b \lambda_1 + c),\\
  &= 0.
\end{align*}
\par Supponiamo ora $\lambda_1$ sia l'unica soluzione reale di
$a\lambda^2 + b\lambda + c = 0$.
Abbiamo dunque
$\lambda_1 = - \frac{b}{2a}$
e
$b^2 - 4 ac = 0$. Inoltre 
\begin{align*}
  a \frac{d^2 t e^{\lambda_1 t}}{dt^2}
  + b \frac{d t e^{\lambda_1 t}}{dt}
  + c t e^{\lambda_1 t}
  &= a \lambda_1 e^{\lambda_1 t}
  + a \lambda_1 e^{\lambda_1 t}
  + a \lambda_1^2 t e^{\lambda_1 t}
  + b e^{\lambda_1 t}
  + b \lambda_1 t e^{\lambda_1 t}
  + c t e^{\lambda_1 t},\\
  &= e^{\lambda_1 t}
    (at \lambda_1^2 + 2a \lambda_1 + bt \lambda_1 + b + c t),\\
  &= e^{\lambda_1 t}
    (a t \frac{b^2}{4 a^2} - 2a \frac{b}{2a} - bt \frac{b}{2 a} + b + ct),\\
  &= t e^{\lambda_1 t}
    (\frac{b^2}{4a} -  \frac{b^2}{2a} + c),\\
  &= t e^{\lambda_1 t}
    \frac{4ac - b^2}{4a}
  &= 0.
\end{align*}
\par Supponiamo infine
$a\lambda^2 + b\lambda + c = 0$
ammetta soluzioni coniugate complesse distinte.
Poniamo $z = e^{\tilde{\lambda}}$. Abbiamo
\begin{align*}
  a \frac{d^2 e^{\tilde{\lambda}_1 t}}{dt^2}
  + b \frac{d e^{\tilde{\lambda}_1 t}}{dt}
  + c e^{\tilde{\lambda}_1 t}
  &= a \tilde{\lambda}_1^2 e^{\tilde{\lambda}_1 t}
  + b \tilde{\lambda}_1 e^{\tilde{\lambda}_1 t}
  + c e^{\tilde{\lambda}_1 t},\\
  &= e^{\tilde{\lambda}_1 t} (a \tilde{\lambda}_1^2 + b \tilde{\lambda}_1 + c),\\
  &= 0.
\end{align*}
Ne consegue la tesi separando parte reale e immaginaria. \EndProof
