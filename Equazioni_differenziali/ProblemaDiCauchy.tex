\section{Problema di Cauchy.}
\label{EquazioniDifferenziali_ProblemaDiCauchy}
\begin{Definition}
	Dato un intervallo $I = [t_0,t_{\text{max}}]$, con
  $t_{\text{max}} \in \mathbb{R} \cup \lbrace + \infty \rbrace$ e
  $t_0 < t_{\text{max}}$, chiamiamo
  \Define{problema di Cauchy}[di Cauchy][problema] o
  \Define{problema ai valori iniziali}[ai valori iniziali][problema] su $I$ il
  sistema
	\[
	\begin{cases}
		y' = f(t,y),\\
		y(t_0) = y_0,
	\end{cases}
	\]
	dove
	\begin{itemize}
		\item $y: I \rightarrow \mathbb{R}^n$;
		\item $f: I \times \mathbb{R}^n \rightarrow \mathbb{R}^n$.
	\end{itemize}
\end{Definition}
\par Assumeremo sempre nel seguito $f \in \CClass{1}[I]$ in modo che siano verificate le ipotesi del teorema seguente.
\begin{Theorem}
	\TheoremName{Teorema di esistenza e unicit\`a di Cauchy-Lipschitz}[di esistenza e unicit\`a di Cauchy-Lipschitz][teorema]
	Sia un problema di Cauchy
	\[
		\begin{cases}
			y' = f(t,y),\\
			y(t_0) = y_0.
		\end{cases}
	\]
	Assumiamo inoltre $f$ lipschitziana nella seconda variabile. Allora esiste una e una sola soluzione del problema di Cauchy.
\end{Theorem}
\begin{Theorem}
	Il problema di Cauchy
	\[
		\begin{cases}
			y' = f(t,y),\\
			y(t_0) = y_0.
		\end{cases}
	\]
	equivale all'equazione integrale $y(t) = y_0 + \int_{t_0}^t f(\tau,y(\tau)) d\tau$.
\end{Theorem}
\Proof Da $y' = f(t,y)$ deduciamo $dy = f(t,y)dt$. Integrando sull'intervallo $[t_0,t]$ e rinominando $\tau$ la variabile $t$ in modo da evitare ambiguit\`a, abbiamo $\int_{y_0}^y dy = \int_{t_0}^t f(\tau,y)d\tau$, equivalente a $y = y_0 + \int_{t_0}^t f(\tau,y)d\tau$.
\par Viceversa, da $y = y_0 + \int_{t_0}^t f(\tau,y)d\tau$ deduciamo
\begin{itemize}
	\item $y' = f(t,y)$ per il teorema fondamentale del calcolo integrale;
	\item $y(t_0) = y_0 + \int_{t_0}^{t_0} f(\tau,y)d\tau = y_0$. \EndProof
\end{itemize}
\begin{Theorem}
  Siano
  \begin{itemize}
    \item $n \in \NotZero{\mathbb{N}}$;
    \item $\Matrix \in \mathbb{C}^{n \times n}$;
    \item $y_0 \in \mathbb{C}^n$.
  \end{itemize}
  Il problema di Cauchy
	\[
		\begin{cases}
			y' = f(t,y),\\
			y(t_0) = y_0.
		\end{cases}
	\]
  ammette l'unica soluzione
  $y = \Exponential(t \Matrix)y_0$.
\end{Theorem}
\Proof Segue per verifica diretta. \EndProof
