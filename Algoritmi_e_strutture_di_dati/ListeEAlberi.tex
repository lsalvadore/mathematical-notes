\subsection{Liste e alberi.}
\label{AlgoritmiEStruttureDiDati_ListeEAlberi}
\begin{Definition}
  Un \Define{albero} \`e una struttura di dati
  \begin{itemize}
    \item o vuota;
    \item o costituita da un dato e da puntatori ad altri alberi; a seconda del
    contesto, chiameremo \Define{nodo} o il dato in questione o lo stesso dato
    insieme ai suddetti puntatori.
  \end{itemize}
  Se ogni albero che interviene nella costruzione ricorsiva di un albero ha lo
  stesso numero $n \in \mathbb{N}$ di puntatori, allora l'albero di dice
  \Define{$n$-ario}[$n$-ario][albero].
  Inoltre,
  \begin{itemize}
    \item un albero unario si chiama \Define{lista};
    \item gli alberi puntati da un albero binario non vuoto si chiamano
      \Define{figlio sinistro}[sinistro][figlio]
      e
      \Define{figlio destro}[destro][figlio]; ne conseguono ulteriori metafore
      parentali con ovvio significato.
  \end{itemize}
\end{Definition}
\begin{listing}
	\insertcode{cpp}{"Algoritmi_e_strutture_di_dati/Lista.h"}
	\caption{Implementazione di una struttura di lista in \LanguageName{C++}.}
\end{listing}
\begin{listing}
	\insertcode{cpp}{"Algoritmi_e_strutture_di_dati/AlberoBinario.h"}
	\caption{Implementazione di una struttura di albero binario in \LanguageName{C++}: dichiarazione della classe.}
\end{listing}
\begin{listing}
	\insertcode{cpp}{"Algoritmi_e_strutture_di_dati/AlberoBinario.cpp"}
	\caption{Implementazione di una struttura di albero binario in \LanguageName{C++}: costruttore.}
\end{listing}
\begin{Theorem}
  Una struttura di albero $n$-ario implementa il concetto di albero $n$-ario
come sinonimo di arborescenza di $n$ figli.
\end{Theorem}
\par Grazie al teorema precedente, trasportiamo sugli alberi $n$-ari tutte le
definizioni e tutti i teoremi dimostrati.
\begin{listing}
	\insertcode{cpp}{"Algoritmi_e_strutture_di_dati/Dimensione.cpp"}
	\caption{Implementazione di un metodo \LanguageName{C++} per il calcolo della dimensione di un albero.}
\end{listing}
\begin{listing}
	\insertcode{cpp}{"Algoritmi_e_strutture_di_dati/Altezza.cpp"}
	\caption{Implementazione di un metodo \LanguageName{C++} per il calcolo dell'altezza di un albero.}
\end{listing}
\begin{Definition}
		Consideriamo un albero $n$-ario e una qualche funzione
    \mintinline{c}{Funzione} avente tra gli argomenti il dato di un nodo. Si
    dice che un algoritmo che parte dalla radice dell'albero effettua
\begin{itemize}
	\item \Define{visite anticipate}[anticipata][visita] se prima viene chiamata
    \mintinline{c}{Funzione} e poi viene applicato ricorsivamente l'algoritmo ai
    figli;
	\item \Define{visite posticipate}[posticipata][visita] se prima viene viene
    applicato ricorsivamente l'algoritmo ai figli e poi viene chiamata
    \mintinline{c}{Funzione};
	\item \Define{visite per ampiezza}[per ampiezza][visita] se applica la
    funzione \mintinline{c}{Funzione} a tutti i nodi in ordine di altezza
    crescente.
\end{itemize}
  Se $n = 2$, diciamo anche che un algoritmo effettua
  \begin{itemize}
	  \item \Define{visite simmetriche}[simmetrica][visita] se prima viene applicato
      l'algoritmo ricorsivamente al figlio sinistro, poi viene chiamata
      \mintinline{c}{Funzione}, infine viene applicato ricorsivamente l'algoritmo
      al figlio destro.
  \end{itemize}
\end{Definition}
\begin{listing}
	\insertcode{cpp}{"Algoritmi_e_strutture_di_dati/VisitaAnticipata.cpp"}
	\caption{Implementazione di un algoritmo che effettua visite anticipate in linguaggio \LanguageName{C++}.}
\end{listing}
\begin{listing}
	\insertcode{cpp}{"Algoritmi_e_strutture_di_dati/VisitaPosticipata.cpp"}
	\caption{Implementazione di un algoritmo che effettua visite posticipate in linguaggio \LanguageName{C++}.}
\end{listing}
\begin{listing}
	\insertcode{cpp}{"Algoritmi_e_strutture_di_dati/VisitaSimmetrica.cpp"}
	\caption{Implementazione di un algoritmo che effettua visite simmetriche in linguaggio \LanguageName{C++}.}
\end{listing}
