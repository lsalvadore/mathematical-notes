\section{Strutture di dati}
\label{AlgoritmiEStruttureDiDati_StruttureDiDati}
\begin{Definition}
	Una \Define{struttura di dati}[di dati][struttura] \`e il dato di
	\begin{itemize}
		\item una collezione di dati;
		\item operazioni sui campi.
	\end{itemize}
\end{Definition}
\begin{Definition}
	Una \Define{classe} \`e uno strumento fornito da alcuni linguaggi di programmazione (per esempio il \LanguageName{C++} o il \LanguageName{Python}) che consente di implementare una struttura di dati in maniera unitaria e senza specificarne esplicitamente i valori dei dati. In questo contesto chiamiamo
	\begin{itemize}
		\item \Define{campi}[di una classe][campo] la rappresentazione della classe della collezione di dati della struttura implementata;
		\item \Define{metodi}[di una classe ][metodo] l'implementazione delle operazioni della struttura implementata;
		\item \Define{istanza della classe}[di una classe][istanza] o \Define{oggetto} una struttura di dati specifica, cio\`e i cui dati hanno un valore specificato, implementata per mezzo della classe;
		\item \Define{costruttore}[di una classe][costruttore] il metodo che crea un oggetto in memoria;
		\item \Define{distruttore}[di una classe][distruttore] il metodo che libera la memoria occupata da un oggetto.
	\end{itemize}
\end{Definition}
\subsection{Liste e alberi.}
\label{AlgoritmiEStruttureDiDati_ListeEAlberi}
\begin{Definition}
  Un \Define{albero} \`e una struttura di dati
  \begin{itemize}
    \item o vuota;
    \item o costituita da un dato e da puntatori ad altri alberi; a seconda del
    contesto, chiameremo \Define{nodo} o il dato in questione o lo stesso dato
    insieme ai suddetti puntatori.
  \end{itemize}
  Se ogni albero che interviene nella costruzione ricorsiva di un albero ha lo
  stesso numero $n \in \mathbb{N}$ di puntatori, allora l'albero di dice
  \Define{$n$-ario}[$n$-ario][albero].
  Inoltre,
  \begin{itemize}
    \item un albero unario si chiama \Define{lista};
    \item gli alberi puntati da un albero binario non vuoto si chiamano
      \Define{figlio sinistro}[sinistro][figlio]
      e
      \Define{figlio destro}[destro][figlio]; ne conseguono ulteriori metafore
      parentali con ovvio significato.
  \end{itemize}
\end{Definition}
\begin{listing}
	\insertcode{cpp}{"Algoritmi_e_strutture_di_dati/Lista.h"}
	\caption{Implementazione di una struttura di lista in \LanguageName{C++}.}
\end{listing}
\begin{listing}
	\insertcode{cpp}{"Algoritmi_e_strutture_di_dati/AlberoBinario.h"}
	\caption{Implementazione di una struttura di albero binario in \LanguageName{C++}: dichiarazione della classe.}
\end{listing}
\begin{listing}
	\insertcode{cpp}{"Algoritmi_e_strutture_di_dati/AlberoBinario.cpp"}
	\caption{Implementazione di una struttura di albero binario in \LanguageName{C++}: costruttore.}
\end{listing}
\begin{Theorem}
  Una struttura di albero $n$-ario implementa il concetto di albero $n$-ario
come sinonimo di arborescenza di $n$ figli.
\end{Theorem}
\par Grazie al teorema precedente, trasportiamo sugli alberi $n$-ari tutte le
definizioni e tutti i teoremi dimostrati.
\begin{listing}
	\insertcode{cpp}{"Algoritmi_e_strutture_di_dati/Dimensione.cpp"}
	\caption{Implementazione di un metodo \LanguageName{C++} per il calcolo della dimensione di un albero.}
\end{listing}
\begin{listing}
	\insertcode{cpp}{"Algoritmi_e_strutture_di_dati/Altezza.cpp"}
	\caption{Implementazione di un metodo \LanguageName{C++} per il calcolo dell'altezza di un albero.}
\end{listing}
\begin{Definition}
		Consideriamo un albero $n$-ario e una qualche funzione
    \mintinline{c}{Funzione} avente tra gli argomenti il dato di un nodo. Si
    dice che un algoritmo che parte dalla radice dell'albero effettua
\begin{itemize}
	\item \Define{visite anticipate}[anticipata][visita] se prima viene chiamata
    \mintinline{c}{Funzione} e poi viene applicato ricorsivamente l'algoritmo ai
    figli;
	\item \Define{visite posticipate}[posticipata][visita] se prima viene viene
    applicato ricorsivamente l'algoritmo ai figli e poi viene chiamata
    \mintinline{c}{Funzione};
	\item \Define{visite per ampiezza}[per ampiezza][visita] se applica la
    funzione \mintinline{c}{Funzione} a tutti i nodi in ordine di altezza
    crescente.
\end{itemize}
  Se $n = 2$, diciamo anche che un algoritmo effettua
  \begin{itemize}
	  \item \Define{visite simmetriche}[simmetrica][visita] se prima viene applicato
      l'algoritmo ricorsivamente al figlio sinistro, poi viene chiamata
      \mintinline{c}{Funzione}, infine viene applicato ricorsivamente l'algoritmo
      al figlio destro.
  \end{itemize}
\end{Definition}
\begin{listing}
	\insertcode{cpp}{"Algoritmi_e_strutture_di_dati/VisitaAnticipata.cpp"}
	\caption{Implementazione di un algoritmo che effettua visite anticipate in linguaggio \LanguageName{C++}.}
\end{listing}
\begin{listing}
	\insertcode{cpp}{"Algoritmi_e_strutture_di_dati/VisitaPosticipata.cpp"}
	\caption{Implementazione di un algoritmo che effettua visite posticipate in linguaggio \LanguageName{C++}.}
\end{listing}
\begin{listing}
	\insertcode{cpp}{"Algoritmi_e_strutture_di_dati/VisitaSimmetrica.cpp"}
	\caption{Implementazione di un algoritmo che effettua visite simmetriche in linguaggio \LanguageName{C++}.}
\end{listing}

\subsection{Alberi binari di ricerca.}
\label{AlgoritmiEStruttureDiDati_AlberiBinariDiRicerca}
\begin{listing}
	\insertcode{cpp}{"Algoritmi_e_strutture_di_dati/AlberoBinarioDiRicerca.h"}
	\caption{Implementazione di una struttura di albero binario di ricerca in \LanguageName{C++}: dichiarazione della classe.}
\end{listing}
\begin{listing}
	\insertcode{cpp}{"Algoritmi_e_strutture_di_dati/AlberoBinarioDiRicerca.cpp"}
	\caption{Implementazione di una struttura di albero binario di ricerca in \LanguageName{C++}: costruttore.}
\end{listing}
\begin{listing}
	\insertcode{cpp}{"Algoritmi_e_strutture_di_dati/AlberoBinarioDiRicerca_Inserisci.cpp"}
	\caption{Implementazione dell'inserzione di un nodo in un albero binario di ricerca in \LanguageName{C++}.}
\end{listing}
\begin{listing}
	\insertcode{cpp}{"Algoritmi_e_strutture_di_dati/AlberoBinarioDiRicerca_Ricerca.cpp"}
	\caption{Implementazione della ricerca di un nodo in un albero binario di ricerca in \LanguageName{C++}.}
\end{listing}
\begin{listing}
	\insertcode{cpp}{"Algoritmi_e_strutture_di_dati/AlberoBinarioDiRicerca_Cancella.cpp"}
	\caption{Implementazione della cancellazione di nodi in un albero binario di ricerca in \LanguageName{C++}.}
\end{listing}
\begin{listing}
	\insertcode{cpp}{"Algoritmi_e_strutture_di_dati/AlberoBinarioDiRicerca_MinimoSottoAlbero.cpp"}
	\caption{Implementazione dell'individuazione del minimo sottoalbero in un albero binario di ricerca in \LanguageName{C++}.}
\end{listing}

\subsection{Alberi AVL.}
\label{AlgoritmiEStruttureDiDati_AlberiAVL}
\begin{listing}
	\insertcode{cpp}{"Algoritmi_e_strutture_di_dati/AVL_Altezza.cpp"}
	\caption{Implementazione di una funzione \LanguageName{C++} per il calcolo dell'altezza di un alberoi AVL.}
\end{listing}
\begin{listing}
	\insertcode{cpp}{"Algoritmi_e_strutture_di_dati/AVL_Inserisci.cpp"}
	\caption{Implementazione dell'inserzione di un nodo in un albero AVL in \LanguageName{C++}.}
\end{listing}
\begin{listing}
	\insertcode{cpp}{"Algoritmi_e_strutture_di_dati/AVL_NuovaFoglia.cpp"}
	\caption{Implementazione della creazione di una nuova foglia in un albero AVL in \LanguageName{C++}.}
\end{listing}
\begin{listing}
	\insertcode{cpp}{"Algoritmi_e_strutture_di_dati/AVL_RuotaOraria.cpp"}
	\caption{Implementazione della rotazione oraria di alcuni nodi in un albero AVL in \LanguageName{C++}.}
\end{listing}
\begin{listing}
	\insertcode{cpp}{"Algoritmi_e_strutture_di_dati/AVL_RuotaAntiOraria.cpp"}
	\caption{Implementazione della rotazione antioraria di alcuni nodi in un albero AVL in \LanguageName{C++}.}
\end{listing}

\subsection{Pila.}
\label{AlgoritmiEStruttureDiDati_Pila}
\begin{Definition}
	Si dice che una struttura di dati implementa la strategia \Define{LIFO}
  (\English{Last in first out}) quando memorizza dati in successione e, quando
  interrogata, restituisce ed espelle il dato memorizzato pi\`u di recente.
\end{Definition}
\begin{Definition}
  Una struttura di dati che implementa la strategia LIFO si chiama \Define{pila}
  (\Define{\English{stack}}). Inoltre,
  \begin{itemize}
    \item il metodo che aggiunge un dato alla pila \`e solitamente chiamato
      \Define{\English{push}};
    \item il metodo che estrae un dato dalla pila \`e solitamente chiamato
      \Define{\English{pop}}.
  \end{itemize}
\end{Definition}
\par Un'implementazione di una pila tramite \English{array}
richiede normalmente di conoscere a tempo di compilazione qual \`e il massimo
numero di elementi che possa essere impilato; tuttavia, qualora ci\`o non fosse
possibile, esiste comunque la possibilit\`a di ricollocare una pila esistente in
un nuovo \English{array} pi\`u lungo quando
necessario, e successivamente di ricollocare nuovamente la pila in un
\English{array} pi\`u piccolo qualora si ritenesse utile mantenere limitata la
quantit\`a di memoria occupata da un programma. Al fine di ammortizzare i costi
di ricollocazione, conviene evitare di allungare o accorciare l'\English{array}
di una cella per volta, per esempio preferendo invece procedere per
raddoppiamenti e dimezzamenti.
\begin{listing}
	\insertcode{cpp}{"Algoritmi_e_strutture_di_dati/Pila.h"}
	\caption{Implementazione di una pila in \LanguageName{C++}: dichiarazione della classe.}
\end{listing}
\begin{listing}
	\insertcode{cpp}{"Algoritmi_e_strutture_di_dati/Pila.cpp"}
	\caption{Implementazione di una pila in \LanguageName{C++}: costruttore.}
\end{listing}
\begin{listing}
	\insertcode{cpp}{"Algoritmi_e_strutture_di_dati/Pila_Push.cpp"}
	\caption{Implementazione di un metodo \LanguageName{C++} per l'aggiunta di un elemento ad una pila.}
\end{listing}
\begin{listing}
	\insertcode{cpp}{"Algoritmi_e_strutture_di_dati/Pila_Pop.cpp"}
	\caption{Implementazione di un metodo \LanguageName{C++} per l'estrazione di un elemento da una pila.}
\end{listing}
\begin{listing}
	\insertcode{cpp}{"Algoritmi_e_strutture_di_dati/Pila_Top.cpp"}
	\caption{Implementazione di un metodo \LanguageName{C++} per la restituzione di un elemento da una pila, ma senza estrazione.}
\end{listing}
\begin{listing}
	\insertcode{cpp}{"Algoritmi_e_strutture_di_dati/Pila_Vuota.cpp"}
	\caption{Implementazione di un metodo \LanguageName{C++} per determinare se la pila \`e vuota.}
\end{listing}

\subsection{\English{Array} circolare.}
\label{AlgoritmiEStruttureDiDati_ArrayCircolare}
\begin{Definition}
  Un
  \Define{\English{array} circolare}[circolare][\English{array}]
  \`e una struttura di dati caratterizzata da
  \begin{itemize}
    \item una dimensione $n \in \mathbb{N}$;
    \item $n$ dati $(e_i)_{i \in n}$;
    \item un sistema di indicizzazione su $\mathbb{Z}$ (nella pratica, su un suo
      sottoinsieme sufficientemente grande) tale che l'elemento indicizzato da
      $i \in \mathbb{Z}$ \`e l'elemento $e_I$ dove $I \in \mathbb{Z}$ \`e il
      pi\`u piccolo rappresentante non negativo della classe di resto di $i$
      modulo $n$.
  \end{itemize}
\end{Definition}
\begin{listing}
	\insertcode{cpp}{"Algoritmi_e_strutture_di_dati/ArrayCircolare.h"}
	\caption{Implementazione di un \English{array} circolare in
    \LanguageName{C++}: dichiarazione della classe.}
\end{listing}
\begin{listing}
	\insertcode{cpp}{"Algoritmi_e_strutture_di_dati/ArrayCircolare.cpp"}
	\caption{Implementazione di un \English{array} circolare in
    \LanguageName{C++}: costruttore e distruttore.}
\end{listing}
\begin{listing}
	\insertcode{cpp}{"Algoritmi_e_strutture_di_dati/ArrayCircolare_Indexing.cpp"}
	\caption{Implementazione di un \English{array} circolare in
    \LanguageName{C++}: definizione dell'operazione di indicizzazione.}
\end{listing}

\subsection{Coda.}
\label{AlgoritmiEStruttureDiDati_Coda}
\begin{Definition}
	Si dice che una struttura di dati implementa la strategia \Define{FIFO}
  (\English{First in first out}) quando memorizza dati in successione e, quando
  interrogata, restituisce ed espelle il dato memorizzato meno di recente.
\end{Definition}
\begin{Definition}
  Una struttura di dati che implementa la strategia FIFO si chiama \Define{coda}
  (\Define{\English{queue}}). Inoltre,
  \begin{itemize}
    \item il metodo che aggiunge un dato alla coda \`e solitamente chiamato
      \Define{\English{enqueue}};
    \item il metodo che estrae un dato dalla coda \`e solitamente chiamato
      \Define{\English{dequeue}}.
  \end{itemize}
\end{Definition}
\par Strutture di dati particolaremente utili per l'implementazione di una coda
sono l'\English{array} circolare e la lista. Generalmente, \`e preferita la
lista, che come si pu\`o osservare dai codici \`e di pi\`u immediata
implementazione.
\par Per l'esemplificazione di un'implementazione
tramite \English{array} circolare, ci asteniamo da utilizzare i codici gi\`a
usati nella sezione \ref{AlgoritmiEStruttureDiDati_ArrayCircolare} perch\'e non
sono convenienti: in effetti, come si pu\`o notare, gli indici
\mintinline{cpp}{Cima} e
\mintinline{cpp}{Fondo} crescono sempre di un'unit\`a per volta e necessitano,
in generale, di essere rimpiazzati da rappresentati della stessa classe di resto
modulo la lunghezza della coda per evitare errori generati da un eventuale
traboccamento.
\par Segnaliamo, senza dettagli, che \`e possibile implementare una coda anche
con un \English{array} non circolare: ci\`o richiede attenzione con l'uso dei
puntatori che tengono conto del progresso della cosa e il mantenimento di una
cella di memoria vuota per poter distinguere casi limiti di riempimento della
coda.
\par Un'implementazione di una coda tramite \English{array}, circolari o meno,
richiede normalmente di conoscere a tempo di compilazione qual \`e il massimo
numero di elementi che possa essere accodato; tuttavia, qualora ci\`o non fosse
possibile, esiste comunque la possibilit\`a di ricollocare una coda esistente in
un nuovo \English{array} pi\`u lungo quando necessario, e successivamente di
ricollocare nuovamente la coda in un
\English{array} pi\`u piccolo qualora si ritenesse utile mantenere limitata la
quantit\`a di memoria occupata da un programma. Al fine di ammortizzare i costi
di ricollocazione, conviene evitare di allungare o accorciare l'\English{array}
di una cella per volta, per esempio preferendo invece procedere per
raddoppiamenti e dimezzamenti.
\begin{listing}
	\insertcode{cpp}{"Algoritmi_e_strutture_di_dati/Coda_ArrayCircolare.h"}
	\caption{Implementazione di una coda in \LanguageName{C++} tramite
    \English{array} circolare: dichiarazione della classe.}
\end{listing}
\begin{listing}
	\insertcode{cpp}{"Algoritmi_e_strutture_di_dati/Coda_Enqueue_ArrayCircolare.cpp"}
	\caption{Implementazione di un metodo \LanguageName{C++} per l'aggiunta di un
    elemento ad una coda implementata tramite \English{array} circolare.}
\end{listing}
\begin{listing}
	\insertcode{cpp}{"Algoritmi_e_strutture_di_dati/Coda_Dequeue_ArrayCircolare.cpp"}
	\caption{Implementazione di un metodo \LanguageName{C++} per l'estrazione di
    un elemento da una coda implementata tramite \English{array} circolare.}
\end{listing}
\begin{listing}
	\insertcode{cpp}{"Algoritmi_e_strutture_di_dati/Coda_RestituisciPrimo_ArrayCircolare.cpp"}
	\caption{Implementazione di un metodo \LanguageName{C++} per la restituzione
    di un elemento da una coda implementata tramite \English{array} circolare,
    ma senza estrazione.}
\end{listing}
\begin{listing}
	\insertcode{cpp}{"Algoritmi_e_strutture_di_dati/Coda_Vuota_ArrayCircolare.cpp"}
	\caption{Implementazione di un metodo \LanguageName{C++} per determinare se
    una coda implementata tramite \English{array} circolare \`e vuota.}
\end{listing}
\begin{listing}
	\insertcode{cpp}{"Algoritmi_e_strutture_di_dati/Coda_Lista.h"}
	\caption{Implementazione di una coda in \LanguageName{C++} tramite una lista:
    dichiarazione della classe.}
\end{listing}
\begin{listing}
	\insertcode{cpp}{"Algoritmi_e_strutture_di_dati/Coda_Lista.cpp"}
	\caption{Implementazione di una coda in \LanguageName{C++} tramite una lista:
    distruttore.}
\end{listing}
\begin{listing}
	\insertcode{cpp}{"Algoritmi_e_strutture_di_dati/Coda_Enqueue_Lista.cpp"}
	\caption{Implementazione di un metodo \LanguageName{C++} per l'aggiunta di un
    elemento ad una coda implementata tramite una lista.}
\end{listing}
\begin{listing}
	\insertcode{cpp}{"Algoritmi_e_strutture_di_dati/Coda_Dequeue_Lista.cpp"}
	\caption{Implementazione di un metodo \LanguageName{C++} per l'estrazione di
    un elemento da una coda implementata tramite una lista.}
\end{listing}
\begin{listing}
	\insertcode{cpp}{"Algoritmi_e_strutture_di_dati/Coda_RestituisciPrimo_Lista.cpp"}
	\caption{Implementazione di un metodo \LanguageName{C++} per la restituzione
    di un elemento da una coda implementata tramite una lista, ma senza
    estrazione.}
\end{listing}
\begin{listing}
	\insertcode{cpp}{"Algoritmi_e_strutture_di_dati/Coda_Vuota_Lista.cpp"}
	\caption{Implementazione di un metodo \LanguageName{C++} per determinare se
    una coda implementata tramite una lista \`e vuota.}
\end{listing}

\subsection{Coda con priorit\`a e \English{heap}.}
\label{AlgoritmiEStruttureDiDati_CodaConPrioritaEHeap}
\begin{Definition}
  Una \Define{coda con priorit\`a}[con priorit\`a][coda]
  (\Define{\English{priority queue}}) \`e una struttura di dati che memorizza
  dati in successione accoppiati con un numero chiamato \Define{priorit\`a} e li
  restituisce in ordine di priorit\`a, decrescente o crescente a seconda
  dell'implementazione.
  Inoltre, come per le code senza priorit\`a,
  \begin{itemize}
    \item il metodo che aggiunge un dato alla coda \`e solitamente chiamato
      \Define{\English{enqueue}};
    \item il metodo che estrae un dato dalla coda \`e solitamente chiamato
      \Define{\English{dequeue}}.
  \end{itemize}
\end{Definition}
\par Salvo indicazioni contrarie, assumeremo sempre che una coda di priorit\`a
estrae gli elementi con priorit\`a massima.
Ai fini della teoria, possiamo assumere tutte le
priorit\`a distinte senza perdita di generalit\`a: infatti, se una coda di
priorit\`a ha pi\`u elementi con la massima priorit\`a, allora dovr\`a essere
usato un criterio per scegliere quale elemento effettivamente estrarre; ma 
questo criterio pu\`o anche essere scelto per aumentare la priorit\`a di
quell'elemento o abbassare la priorit\`a degli altri, e un ragionamento analogo
vale per qualsiasi altra situazione in cui serva discrimanre tra elementi con
la stessa prioririt\`a.
\par Inoltre, senza perdita di generalit\`a, esamineremo solo code con
priorit\`a che memorizzano dati vuoti, cio\`e che memorizzano solo le priorit\`a
stesse: in effetti, il funzionamento della struttura dipende esclusivamente
dalle priorit\`a, mentre per quanto riguarda i costi in termini di spazio,
baster\`a aumentare il costo della priorit\`a in modo tale che esso includa
anche il costo del dato ad esso associato, operazione che per un'analisi
asintotica non implica nessuna conseguenza.
\par Una possibile implementazione di una coda con priorit\`a prevede l'uso di
liste; essa consente
\begin{itemize}
  \item l'accodamento di nuovi dati in tempo $\BigO{1}$ mettendo in testa alla
    lista il nuovo elemento;
  \item l'estrazione di dati in tempo $\BigO{n}$, leggendo tutti gli elementi
    della lista in successione per individuare l'elemento di priorit\`a pi\`u
    alta o pi\`u bassa.
\end{itemize}
\par Un'ulteriore possibilit\`a \`e l'implementazione tramite un \English{array}
ordinato, che naturalmente deve essere sufficientemente grande o essere
ricollocato in \English{array} pi\`u o meno grandi a seconda delle necessit\`a.
In tal caso, accodamento ed estrazione possono essere effettuati sfruttando
l'algoritmo di ricerca binaria e hanno dunque entrambi costo $\BigO{\log n}$.
\par Generalmente, le code con priorit\`a vengono implementate tramite un'aaltra
struttura ancora denominata \English{heap}, motivo per cui le code di priorit\`a
sono spesso chiamate anche \English{heap}.
\begin{Definition}
  Si chiama \Define{\English{heap}} un albero completo da sinistra che registra
  dati numerici tale che una delle condizioni seguenti sia verificata
  \begin{itemize}
    \item se $\Node$ \`e genitore di $\VarNode$, allora 
      $\Node \geq \VarNode$: in tal caso si dice che l'\English{heap}
      \`e un \Define{\English{heap} $\max$};
    \item se $\Node$ \`e genitore di $\VarNode$, allora 
      $\Node \leq \VarNode$: in tal caso si dice che l'\English{heap}
      \`e un \Define{\English{heap} $\min$}.
  \end{itemize}
  Se l'\English{heap} \`e un albero $n$-ario, allora si dice che
  l'\English{heap} \`e \English{$n$-ario}[$n$-ario][\English{heap}].
\end{Definition}
\begin{Definition}
  Si chiama
  \Define{\English{heap} binario implicito}[binario implicito][\English{heap}]
  la struttura di dati definita da un \English{array} di numeri
  $(e_i)_{i \in n}$ che verifica una delle due delle condizioni seguenti:
  \begin{itemize}
    \item \TheoremName{propriet\`a di \English{heap} $\max$}[di \English{heap} $\max$][propriet\`a]
      se $i \in n$, allora
      \begin{itemize}
        \item se $j = \left \lfloor \frac{i - 1}{2} \right \rfloor \in n$,
          allora $e_j \geq e_i$;
        \item se $j = 2i + 1 \in n$,
          allora $e_i \geq e_j$;
        \item se $j = 2i + 2 \in n$,
          allora $e_i \geq e_j$:
      \end{itemize}
      in tal caso si dice che l'\English{heap} binario implicito \`e un
      \Define{\English{heap} binario implicito $\max$}[binario implicito $\max$][\English{heap}];
    \item \TheoremName{propriet\`a di \English{heap} $\min$}[di \English{heap} $\min$][propriet\`a]
      se $i \in n$, allora
      \begin{itemize}
        \item se $j = \left \lfloor \frac{i - 1}{2} \right \rfloor \in n$,
          allora $e_j \leq e_i$;
        \item se $j = 2i + 1 \in n$,
          allora $e_i \leq e_j$;
        \item se $j = 2i + 2 \in n$,
          allora $e_i \leq e_j$:
      \end{itemize}
      in tal caso si dice che l'\English{heap} binario implicito \`e un
      \Define{\English{heap} binario implicito $\min$}[binario implicito $\min$][\English{heap}].
  \end{itemize}
\end{Definition}
\begin{Theorem}
  Un \English{heap} binario $\max$ di $n \in \NotZero{\mathbb{N}}$ nodi pu\`o
  essere implementato tramite un \English{heap} binario implicito $\max$ di
  dimensione $n$ tramite una corrispondenza che associa al $k$-esimo (contando
  da $0$) nodo da sinistra di livello $L$ l'elemento di indice $2^L + k - 1$
  dell'\English{heap} binario impliciato in modo tale che il dato numerico
  registrato nel nodo sia lo stesso registrato nell'elemento dell'\English{heap}
  binario implicito ad esso associato.
\end{Theorem}
\Proof Occorre verificare che l'\English{array} immagine della corrispondenza
sia effettivamente un \English{heap} binario implicito.
\par Consideriamo il $k$-esimo (contando da $0$) nodo $\Node$ di livello $L$
dell'\English{heap} binario. I suoi figli, se esistono, si ottengono scorrendo i
figli di tutti i nodi pi\`u a sinistra di $\Node$: essi sono in numero $2k$.
Dunque, i figli di $\Node$ sono, se esistono, il $2k$-esimo e il
$(2k + 1)$-esimo nodo al livello $L + 1$: le loro immagini, rispetto alla
corrispondenza dell'enunciato, sono
l'$(2^{L + 1} + 2k - 1)$-esimo
e
l'$(2^{L + 1} + 2k)$-esimo elemento dell'\English{array}.
\par Fissiamo ora $I \in n$. Siano
\begin{itemize}
  \item $L \in \mathbb{N}$;
  \item $k \in \mathbb{N}$ con $0 \leq k < 2^L$;
\end{itemize}
tali che $I = 2^L + k - 1$.
Denotiamo $(e_i)_{i \in n}$ gli elementi dell'\English{array}.
\par Per costruzione, $e_I$ \`e immagine del $k$-esimo (contando da $0$) nodo
$\Node$ di livello $L$ dell'\English{heap} binario. Per la propritet\`a di
\English{heap} $\max$, il nodo \`e maggiore o uguale dei suoi figli, e dunque
\begin{itemize}
  \item se $J = 2I + 1 \in n$, abbiamo $J = 2^{L + 1} + 2k - 1$ e $e_J$ \`e
    l'immagine del $(2k + 1)$-esimo nodo di livello $L + 1$, dunque
    $e_I \geq e_J$;
  \item se $J = 2I + 2 \in n$, abbiamo $J = 2^{L + 1} + 2k$ e $e_J$ \`e
    l'immagine del $(2k)$-esimo nodo di livello $L + 1$, dunque
    $e_I \geq e_J$.
\end{itemize}
\par Inoltre,
\begin{itemize}
  \item se $I = 0$, allora
    $J = \left \lfloor \frac{I - 1}{2} \right \rfloor = - 1 \notin n$;
  \item se $I > 0$ \`e dispari, allora $L > 0$, $k > 0$ e $k$ \`e pari, da cui
    $J = \left \lfloor \frac{I - 1}{2} \right \rfloor
    = \frac{2^L + k - 2}{2} = 2^{L - 1} + \frac{k}{2} - 1$: $e_J$ \`e l'immagine
    del $\frac{k}{2}$-esimo nodo di livello $L - 1$ e ha per figlio sinistro
    l'$I$-esimo nodo di livello $L$, dunque $e_J \geq e_I$;
  \item se $I > 0$ \`e pari, allora $L > 0$, $k > 0$ e $k$ \`e dispari, da cui
    $J = \left \lfloor \frac{I - 1}{2} \right \rfloor
    = \left \lfloor \frac{2^L + k - 2}{2} \right \rfloor
    = 2^{L - 1} + \frac{k - 1}{2} - 1$: $e_J$ \`e
    l'immagine del $\frac{k - 1}{2}$-esimo nodo di
    livello $L - 1$ e ha per figlio destro l'$I$-esimo nodo di livello $L$,
    dunque $e_J \geq e_I$. \EndProof
\end{itemize}
\begin{Theorem}
  Un \English{heap} binario di $n \in \mathbb{N}$ elementi costa
  \begin{itemize}
    \item $\BigO{n}$ spazio se implementato tramite liste;
    \item $n$ spazio se implementato tramite \English{heap} binario implicito.
  \end{itemize}
\end{Theorem}
\Proof L'implementazione tramite liste richiede la memorizzazione di puntatori,
mentre l'implementazione tramite \English{heap} binario implicito memorizza solo
le priorit\`a. \EndProof

