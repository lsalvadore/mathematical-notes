\subsection{Problema di ordinamento.}
\label{AlgoritmiEStruttureDiDati_ProblemaDiOrdinamento}
\begin{Definition}
	Chiamiamo \Define{problema di ordinamento}[di ordinamento][problema] il problema che consiste nel riordinare una successione finita rispetto ad un ordine totale dato.
\end{Definition}
\begin{Definition}
	Chiamiamo \Define{\English{comparison model}}[\English{comparison}][\English{model}] un modello computazionale per il problema di ordinamento tale che
	\begin{itemize}
		\item misuri la complessit\`a di un algoritmo in numero di confronti;
		\item ammetta come unica operazione tra due elementi dell'insieme da ordinare il confronto.
	\end{itemize}
\end{Definition}
\par Nel seguito denoteremo $(e_i)_{i \in n}$ ($n \in \mathbb{N}$) la
successione dei dati di ingresso del problema di ordinamento e $\leq$ l'ordine
totale rispetto al quale si desidera riordinare la successione. Inoltre
analizzeremo la complessit\`a degli algoritmi per il problema di ordinamento
secondo il \English{comparison model}.
\begin{Theorem}
	Il problema di ordinamento ha complessit\`a $\BigOmega{n \log n}$.
\end{Theorem}
\Proof Gli elementi dell'insieme da ordinare possono essere disposti in $n!$.
\par Effettuando $1$ confronto, l'algoritmo pu\`o discriminare tra al pi\`u $3$
configurazioni: denotati con $a$ e $b$ i due elementi confrontati, pu\`o
rilevare se $a < b$, se $a = b$ o infine se $a > b$. Assumiamo che con $k$
confronti l'algoritmo possa discriminare tra $3^k$ configurazioni: allora, con
$1$ confronto aggiuntivo, esso pu\`o discrimare tra $3^k \cdot 3 = 3^{k + 1}$
configurazioni. Dunque, segue per induzione che con $k$ confronti si pu\`o
discrimare tra $3^k$ configurazioni.
\par Perch\'e l'algoritmo possa fornire la risposta corretta al problema, \`e
dunque necessario che il numero $k$ di confronti da esso effettuato verifichi la
condizione $3^k \geq n!$, equivalente a $k \geq \log_3 n!$. La tesi segue dal
fatto che $\IsBigOmega{\log_3 n!}{n \log n}$.
\par Osserviamo che la base del logaritmo nell'enunciato non ha importanza
poich\'e il cambio di base si effettua moltiplicando per una costante. Inoltre,
se assumiamo tutti gli elementi dell'insieme da ordinare distinti, nella
dimostrazione possiamo sostituire ogni occorrenza del numero $3$ col numero $2$,
ma il limite inferiore dimostrato \`e lo stesso. \EndProof
\subsubsection{\English{Selection sort}.}
\label{AlgoritmiEStruttureDiDati_SelectionSort}
\begin{Definition}
	Si chiama \Define{\English{selection sort}}[\English{selection}][\English{sort}] l'algoritmo seguente:
	\begin{enumerate}
		\item\label{SelectionSort_1} si ponga $I = 0$;
		\item\label{SelectionSort_2} si scambi $e_I$ con $\min_{j \in n \SetMin I} e_j$;
		\item\label{SelectionSort_3} si incrementi $I$ di $1$;
		\item\label{SelectionSort_4} se $I < n - 1$ si torni al punto \ref{SelectionSort_2}, altrimenti l'algoritmo termina.
	\end{enumerate}
\end{Definition}
\begin{listing}
	\insertcode{cpp}{"Algoritmi_e_strutture_di_dati/SelectionSort.cpp"}
	\caption{\textit{Selection sort} implementato in \LanguageName{C++}.}
\end{listing}
\begin{Theorem}
	Il \English{selection sort} risolve il problema dell'ordinamento.
\end{Theorem}
\Proof Osserviamo che ad ogni passaggio dal punto \ref{SelectionSort_2}
dell'algoritmo di \English{selection sort} $I$ ha un valore diverso,
incrementato di $1$ ad ogni passaggio: procediamo per induzione su $I$,
dimostrando che per ogni $I \in n - 1$, $(e_i)_{i \in I + 1}$ sono ordinati al
termine del punto \ref{SelectionSort_2}. Ne conseguir\`a il teorema nel caso
$I = n - 1$.
\par Fissato $I \in n - 1$, supponiamo vera l'affermazione per $I - 1$ e
dimostriamola per $I$. Per ipotesi induttiva, gli $(e_i)_{i \in I}$ sono
ordinati: occorre provare che
$\ForAll{i \in I}{\min_{j \in n \SetMin I} e_j \geq e_i}$. Ma se per assurdo
cos\`i non fosse, allora in uno dei passaggi precedenti
-- diciamo l'$i$-esimo --,
l'algoritmo non avrebbe scambiato $e_i$ con $\min_{j \in n \SetMin I} e_j$ come
invece prescritto dalla definizione. \EndProof
\begin{Theorem}
	Il \English{selection sort} ha costo $\BigTheta{n^2}$.
\end{Theorem}
\Proof Il punto \ref{SelectionSort_2} calcola un minimo su $n - I$ elementi,
problema che non \`e risolubile in meno di $n - I$ confronti (occorre
confrontare almeno una volta ogni elemento con almeno un altro elemento). Il
costo dell'algoritmo \`e dunque
$\sum_{I = 0}^{n - 2} (n - I)
= (n - 2)n - \frac{(n - 1)(n - 2)}{2} = \BigTheta{n^2}$. \EndProof

\subsubsection{\English{Insertion sort}.}
\label{AlgoritmiEStruttureDiDati_InsertionSort}
\begin{Definition}
	Si chiama
  \Define{\English{insertion sort}}[\English{insertion}][\English{sort}]
  l'algoritmo seguente:
	\begin{enumerate}
		\item\label{InsertionSort_1} si ponga $I = 1$;
		\item\label{InsertionSort_2} se $I = n$, l'algoritmo termina;
		\item\label{InsertionSort_3} si ponga $J = I$;
		\item\label{InsertionSort_4} se $e_I \geq e_J$, allora si incrementi $I$ di
      $1$ e si torni al punto \ref{InsertionSort_2};
		\item\label{InsertionSort_5} si scambino $e_I$ e $e_J$;
		\item\label{InsertionSort_6} se $J = 0$, si incrementi $I$ di $1$ e si torni
      al punto \ref{InsertionSort_2};
		\item\label{InsertionSort_7} si decrementi $J$ di $1$;
		\item\label{InsertionSort_8} si torni al punto \ref{InsertionSort_4}.
	\end{enumerate}
\end{Definition}
\begin{listing}
	\insertcode{cpp}{"Algoritmi_e_strutture_di_dati/InsertionSort.cpp"}
	\caption{\textit{Insertion sort} implementato in \LanguageName{C++}.}
\end{listing}
\begin{Theorem}
	L'\English{insertion sort} risolve il problema dell'ordinamento.
\end{Theorem}
\Proof Osserviamo che la variabile $I$ assume al punto \ref{InsertionSort_2}
valori sempre diverse, incrementati di $1$ ad ogni passaggio: procediamo per
induzione su $I$, dimostrando che per ogni $I \in n$, $(e_i)_{i \in I + 1}$ sono
ordinati al passaggio successivo dell'algoritmo dal punto \ref{InsertionSort_2}.
Ne conseguir\`a il teorema nel caso $I = n$.
\par Fissato $I \in n$, supponiamo vera l'affermazione per tutti gli $I - 1$ e
dimostriamola per $I$. Per ipotesi induttiva, gli $(e_i)_{i \in I}$ sono
ordinati. Ora
\begin{itemize}
	\item se il seguente passaggio dal punto \ref{InsertionSort_2} proviene dal
    punto \ref{InsertionSort_4}, allora
	\begin{itemize}
		\item tutti gli elementi $(e_j)_{j \in J}$ sono rimasti inalterati rispetto
      all'ultimo passaggio dal punto \ref{InsertionSort_2}, e dunque sono
      ordinati;
		\item $e_J$ \`e maggiore di tutti gli $(e_j)_{j \in J}$;
		\item gli $(e_j)_{j \in I \SetMin J}$ sono uguali agli elementi
      $(e_{j - 1})_{j \in I \SetMin J}$ secondo l'indicizzazione dell'ultimo
      passaggio dal punto \ref{InsertionSort_2} e dunque sono ordinati; inoltre
      sono tutti maggiori di $e_J$.
	\end{itemize}
	Dunque gli $(e_i)_{i \in I}$ sono ordinati;
	\item se il seguente passaggio dal punto \ref{InsertionSort_2} proviene dal
    punto \ref{InsertionSort_6}, allora gli $(e_j)_{j \in I \SetMin J}$ sono
    uguali agli elementi $(e_{j - 1})_{j \in I \SetMin J}$ secondo
    l'indicizzazione dell'ultimo passaggio dal punto \ref{InsertionSort_2} e
    dunque sono ordinati; inoltre sono tutti maggiori di $e_0$. Dunque gli
    $(e_i)_{i \in I}$ sono ordinati. \EndProof
\end{itemize}
\begin{Theorem}
	L'\English{insertion sort} ha costo
	\begin{itemize}
		\item $\BigTheta{n}$ al caso ottimo;
		\item $\BigTheta{n^2}$ al caso pessimo;
		\item $\BigTheta{n^2}$ al caso medio.
	\end{itemize}
\end{Theorem}
\Proof I confronti hanno luogo solo al punto \ref{InsertionSort_4}.
\par Al caso ottimo, che si verifica quando la successione \`e gi\`a ordinata,
si passa dal punto \ref{InsertionSort_4} solo una volta per ogni valore
$I \in \NaturalShift{n - 1}$: abbiamo dunque un costo di $n - 1 = \BigTheta{n}$.
\par Al caso pessimo, che si verifica quando la successione \`e ordinata secondo
l'ordinamento opposto a quello voluto, si passa dal punto \ref{InsertionSort_4}
$I$ volte per ogni valore $I \in \NaturalShift{n - 1}$: abbiamo dunque un costo
di $\sum_{I = 0}^{n - 1} I = \frac{n(n - 1)}{2} = \BigTheta{n^2}$.
\par Al caso medio, ad ogni passaggio di $I$, la probabilit\`a di eseguire
$i \in I$ confronti \`e $\frac{1}{I}$: il numero medio di confronti \`e dunque
$\sum_{i \in I} \frac{1}{I}
= \frac{I\left (\frac{1}{I} + \frac{I - 1}{I} \right )}{2}
= \frac{I}{2}$.
Il costo medio \`e dunque
$\sum_{I \in n} \frac{I}{2}
= \frac{n(n - 1)}{4} = \BigTheta{n^2}$. \EndProof

\subsubsection{\English{Merge sort}.}
\label{AlgoritmiEStruttureDiDati_MergeSort}
\begin{Definition}
	Si chiama
  \Define{\English{merge sort}}
  l'algoritmo seguente:
  \begin{enumerate}
    \item\label{MergeSort_1} se $n = 0$ o $n = 1$, allora l'algoritmo termina;
    \item\label{MergeSort_2} si suddivida la successione
      $(e_i)_{i \in n}$
      in due sottosuccessioni
      $(e_i)_{i = 0}^{\left \lfloor \frac{n}{2} \right \rfloor - 1}$
      e 
      $(e_i)_{i = \left \lfloor \frac{n}{2} \right \rfloor}^{n - 1}$
      e si applichi l'intero algoritmo su ciascuna di esse separatamente;
    \item\label{MergeSort_3} si ponga
      $I = 0$,
      $I_1 = 0$,
      $I_2 = \left \lfloor \frac{n}{2} \right \rfloor$;
    \item\label{MergeSort_4} se $e_{I_1} < e_{I_2}$,
      allora si ponga
      $f_I = e_{I_1}$
      e si incrementino di $1$ sia $I$ che $I_1$;
    \item\label{MergeSort_5} se $e_{I_1} \geq e_{I_2}$,
      allora si ponga
      $f_I = e_{I_2}$
      e si incrementino di $1$ sia $I$ che $I_2$;
    \item\label{MergeSort_6} se
      $I_1 > \left \lfloor \frac{n}{2} \right \rfloor - 1$, allora si
      si ponga, per ogni $x \in n - I$,
      $f_{I + x} = e_{I_2 + x}$
      e si vada al punto \ref{MergeSort_9};
    \item\label{MergeSort_7} se $I_2 > n - 1$, allora si
      si ponga, per ogni $x \in n - I$,
      $f_{I + x} = e_{I_1 + x}$
      e si vada al punto \ref{MergeSort_9};
    \item\label{MergeSort_8} si torni al punto \ref{MergeSort_4};
    \item\label{MergeSort_9} per ogni $i \in n$, si ponga $e_i = f_i$:
      l'algoritmo termina.
  \end{enumerate}
\end{Definition}
\begin{listing}
	\insertcode{cpp}{"Algoritmi_e_strutture_di_dati/MergeSort.cpp"}
	\caption{\textit{Merge sort} implementato in \LanguageName{C++}.}
\end{listing}
\begin{Theorem}
	Il \English{merge sort} risolve il problema dell'ordinamento.
\end{Theorem}
\Proof Se $n = 0$ o $n = 1$ la tesi \`e immediata, altrimenti procediamo
per induzione su $n$ assumendo la tesi vera per ogni $n < N$
($N \in \mathbb{N} \SetMin 2$) e proviamola per $N$.
\par Le due successioni
$(e_i)_{i = 0}^{\left \lfloor \frac{N}{2} \right \rfloor - 1}$
e 
$(e_i)_{i = \left \lfloor \frac{N}{2} \right \rfloor}^{N - 1}$
sono ordinate dopo l'esecuzione del punto \ref{MergeSort_2}.
Proviamo adesso per induzione su $I \in N$ che la successione
$(f_i)_{i \in N}$
\`e ordinata. Fissiamo $I \in N$ e supponiamo ordinata la sottosuccesione
$(f_i)_{i = 0}^{I - 1}$.
\par Sia $I \in N$. Se $I = 0$ la successione \`e vuota e non c'\`e nulla
da dimostrare; se $I = 1$, la successione \`e composta da un solo elemento ed
\`e dunque ordianta. Supponiamo $f_I$, con $I > 1$, definito con l'esecuzione
del punto \ref{MergeSort_4}:
\begin{itemize}
  \item se $f_{I - 1}$ \`e stato definito con l'esecuzione del punto
    \ref{MergeSort_4}, poich\'e la successione
    $(e_i)_{i = 0}^{\left \lfloor \frac{n}{2} \right \rfloor - 1}$
    \`e ordinata, deve essere $f_{I - 1} < f_I$;
  \item se $f_{I - 1}$ \`e stato definito con l'esecuzione del punto
    \ref{MergeSort_5}, allora $I_1$ deve essere uguale nell'esecuzione
    di entrambi i punti e dunque $f_{I - 1} < f_I$;
  \item se $f_{I - 1}$ \`e stato definito con l'esecuzione del punto
    \ref{MergeSort_6}, poich\'e la successione
    $(e_i)_{i = \left \lfloor \frac{n}{2} \right \rfloor}^{n - 1}$
    \`e ordinata e $e_{\left \lfloor \frac{n}{2} \right \rfloor} < e_{I_2}$,
    deve essere $f_{I - 1} < f_I$;
  \item se $f_{I - 1}$ \`e stato definito con l'esecuzione del punto
    \ref{MergeSort_7}, poich\'e la successione
    $(e_i)_{i = 0}^{\left \lfloor \frac{n}{2} \right \rfloor - 1}$
    \`e ordinata e $e_{n - 1} < e_{I_1}$, deve essere $f_{I - 1} < f_I$.
\end{itemize}
\par Dunque \`e ordinata anche la successione $(f_i)_{i = 0}^I$.
\par Analogamente, \`e ordinata la successione $(f_i)_{i = 0}^I$, se
$f_I$ \`e definito con l'esecuzione di \ref{MergeSort_5}, \ref{MergeSort_6}
o \ref{MergeSort_7}.
\par Dunque $(f_i)_{i \in N}$ \`e ordinata e ne consegue la correttezza
dell'algoritmo.
\EndProof
\begin{Theorem}
	Il \English{merge sort} ha costo
	\begin{itemize}
		\item $\BigO{n}$ spazio;
		\item $\BigO{n \log n}$ tempo al caso pessimo.
	\end{itemize}
\end{Theorem}
\Proof Il costo in spazio segue dalla necessit\`a di appoggiarsi su una
successione ausiliaria $(f_i)_{i \in n}$ di $n$ elementi.
\par Denotiamo $T(n)$ il costo dell'esecuzione del \English{merge sort} su
una successione di $n$ elementi. Abbiamo chiaramente $T(0) = T(1) = 0$.
\par Contiamo i confronti; dato che siamo interessati a un'analisi
asintotica, possiamo assumere $n$ pari senza perdita di generalit\`a:
\begin{itemize}
  \item al passo \ref{MergeSort_2} si effettuano
    $T \left ( \frac{n}{2} \right )
      + T \left ( n - \frac{n}{2} \right )
      = 2 T \left ( \frac{n}{2} \right )$;
  \item gli altri confronti vengono effettuati ai passi
    \ref{MergeSort_4} e \ref{MergeSort_5}, e ciascuno di questi
    passi viene effettuato al pi\`u $n - 1$ volte: abbiamo dunque
    ulteriori $2(n - 1) = \BigO{n}$ confronti.
\end{itemize}
\par Dunque, per opportuna costante $C > 0$, abbiamo
\[
	T(n) \leq
	\begin{cases}
		0\text{ se }n = 1,\\
		2 T \left (\frac{n}{2} \right ) + C n\text{ per } n > 1.
	\end{cases}
\]
\par Poich\'e abbiamo
$2 \Identity \left ( \frac{n}{2} \right ) = n \leq 1 \Identity(n)$,
per il teorema fondamentale delle ricorrenze, abbiamo
$T(n) = \BigO{n \log n}$. \EndProof
\begin{Corollary}
  \label{AlgoritmiEStruttureDiDati_ComplessitaOrdinamento}
	Il problema di ordinamento ha complessit\`a $\BigTheta{n \log n}$ e il
  \English{merge sort} \`e asintoticamente ottimo.
\end{Corollary}
\Proof Abbiamo gi\`a dimostrato che il problema di ordinamento ha complessit\`a
$\BigOmega{n \log n}$.
Il teorema precedente ha provato che il problema ha complessit\`a
$\BigO{n \log n}$.
Ne deduciamo infine che esso ha complessit\`a $\BigTheta{n \log n}$. \EndProof
\par Segnaliamo che la complessit\`a esatta del problema di ordinamento, cio\`e
il numero minimo di confronti necessario per risolvere il problema di
ordinamento in funzione di $n$, \`e un problema aperto, risolto solo per poche
decine di casi particolari.
\par Se modelliamo il problema con un albero di decisione, il problema equivale
a calcolare l'altezza dell'albero di decisione.

\subsubsection{\English{Quick sort}.}
\label{AlgoritmiEStruttureDiDati_QuickSort}
\begin{Definition}
	Si chiama
  \Define{\English{quick sort}}
  l'algoritmo seguente:
  \begin{enumerate}
    \item\label{QuickSort_1} se $n = 0$ o $n = 1$, allora l'algoritmo termina;
    \item\label{QuickSort_2} si fissi un $P \in n$: chiamiamo
      \Define{\Francais{pivot}}[nel \English{quick sort}][\Francais{pivot}]
      l'elemento $e_P$;
    \item\label{QuickSort_3} si scambino $e_P$ con $e_{n - 1}$;
    \item\label{QuickSort_4} si ponga $I = 0$ e $J = n - 2$;
    \item\label{QuickSort_5} se
      $I \leq J$ e $e_I \leq e_{n - 1}$
      si incrementi $I$ di $1$ e si torni al punto
      \ref{QuickSort_5}
    \item\label{QuickSort_6} se
      $I \leq J$ e $e_J \geq e_{n - 1}$
      si decrementi $J$ di $1$ e si torni al punto
      \ref{QuickSort_6}
    \item\label{QuickSort_7} se $I < J$ si scambino
      $e_I$ e $e_J$;
    \item\label{QuickSort_8} se $I \leq J$, allora vai al punto
      \ref{QuickSort_5};
    \item\label{QuickSort_9} si scambino
      $e_I$ e $e_{n - 1}$;
    \item\label{QuickSort_10} si ripeta l'intero algoritmo sulle successioni
      $(e_i)_{i = 0}^{I - 1}$ e $(e_i)_{I + 1}^{n - 1}$: l'algoritmo termina.
  \end{enumerate}
\end{Definition}
\begin{listing}
	\insertcode{cpp}{"Algoritmi_e_strutture_di_dati/QuickSort.cpp"}
	\caption{\textit{Quick sort} implementato in \LanguageName{C++}.}
\end{listing}
\begin{Theorem}
	Il \English{quick sort} risolve il problema dell'ordinamento.
\end{Theorem}
\Proof Se $n = 0$ e $n = 1$ la tesi \`e immediata, altrimenti procediamo
per induzione su $n$ assumendo la tesi vera per ogni $n < N$
($N \in \mathbb{N} \SetMin 2$) e proviamola per $N$.
\par Per induzione, \`e sufficiente provare che, dopo il passo
\ref{QuickSort_9} tutti gli elementi della successione
$(e_i)_{i = 0}^{I - 1}$
sono minori o uguali di $e_I$ e che tutti gli elementi della successione
$(e_i)_{I + 1}^{N - 1}$
sono maggiori o uguali di $e_I$.
\par Osserviamo che la variabile $I$ prende nel corso dell'algoritmo tutti i
valori da $0$ al valore finale di $I$ in ordine crescente una e una sola volta.
Inoltre,
\begin{itemize}
  \item se al termine del punto \ref{QuickSort_5} l'esecuzione ripete lo stesso
    punto \ref{QuickSort_5}, allora $e_I \leq e_{N - 1}$;
  \item altrimenti, se al punto \ref{QuickSort_7} $I < J$, allora abbiamo
    $e_I \leq e_{N - 1}$ dopo lo scambio;
  \item altrimenti, se al punto \ref{QuickSort_7} $I > J$, $I$ ha raggiunto il
    suo valore finale;
  \item altrimenti, se al punto \ref{QuickSort_7} $I = J$, poich\'e
    simultaneamente
    $e_I \leq e_{N - 1}$ (per il punto \ref{QuickSort_5}) e
    $e_I \geq e_{N - 1}$ (per il punto \ref{QuickSort_6}), abbiamo
    $e_I = e_{N - 1}$.
\end{itemize}
\par Dunque effettivamente  tutti gli elementi della successione
$(e_i)_{i = 0}^{I - 1}$
sono minori o uguali di $e_I$.
\par Analogamente osserviamo che la variabile $J$ prende nel corso
dell'algoritmo tutti i valori da $n - 2$ al valore finale di $I + 1$ in
ordine decrescente una e una sola volta. Inoltre,
\begin{itemize}
  \item se al termine del punto \ref{QuickSort_6} l'esecuzione ripete lo stesso
    punto \ref{QuickSort_6}, allora $e_J > e_{N - 1}$;
  \item altrimenti, se al punto \ref{QuickSort_7} $I < J$, allora abbiamo
    $e_J \geq e_{N - 1}$ dopo lo scambio;
  \item altrimenti, se al punto \ref{QuickSort_7} $I > J$, $J$ ha raggiunto il
    suo valore finale;
  \item altrimenti, se al punto \ref{QuickSort_7} $I = J$, poich\'e
    simultaneamente
    $e_J \leq e_{N - 1}$ (per il punto \ref{QuickSort_5}) e
    $e_J \geq e_{N - 1}$ (per il punto \ref{QuickSort_6}), abbiamo
    $e_J = e_{N - 1}$.
\end{itemize}
\par Dunque effettivamente  tutti gli elementi della successione
$(e_i)_{I + 1}^{N - 1}$
sono maggiori o uguali di $e_I$. \EndProof
\begin{Theorem}
  \label{AlgoritmiEStruttureDiDati_CostoQuickSortDeterministico}
	Il \English{quick sort} ha costo
  \begin{itemize}
    \item $\BigTheta{n^2}$ tempo al caso
      pessimo;
    \item $\BigO{n \log n}$ tempo al caso
      ottimo.
  \end{itemize}
\end{Theorem}
\Proof Se $n = 0$ o $n = 1$ il costo \`e esattamente $0$.
\par Supponiamo $n \geq 2$ senza perdita di generalit\`a dato che
siamo interessati ad un'analisi asintotica.
\par Contiamo i confronti propri di ciascuna singola esecuzione dell'algoritmo:
\begin{itemize}
  \item nella chiamata iniziale abbiamo $n - 1$ confronti;
  \item nelle due chiamate successive abbiamo
      \begin{itemize}
        \item da una parte $I - 1$ confronti se $I \geq 2$, altrimenti $0$;
        \item dall'altra $n - 1 - I - 1 = n - I - 2$ confronti se
          $n - 1 - I \geq 2$, cio\`e $I \leq n - 3$, altrimenti $0$;
      \end{itemize}
      dunque in ogni caso abbiamo $\BigO{n}$ confronti.
\end{itemize}
\par Ad ogni chiamata corrisponde la scelta di un \Francais{pivot},
tutti i \Francais{pivot} sono necessariamente distinti e
non possono essere scelti pi\`u di $n - 1$ \Francais{pivot}, dunque
l'altezza dell'albero di ricorsione \`e al pi\`u $n - 1$, e quindi il costo
\`e al pi\`u $(n - 1) \BigO{n} = \BigO{n^2}$.
\par Ora, ricontiamo i confronti per l'intero algoritmo stavolta, denotanto
$T(n)$ il costo dell'esecuzione dell'algoritmo al caso pessimo:
\begin{itemize}
  \item i passi \ref{QuickSort_5} e \ref{QuickSort_6} eseguono, in totale
    $n - 1$ confronti;
  \item al passo \ref{QuickSort_10} vengono effettuati al pi\`u
    $T(I) + T(n - 1 - I)$ confronti.
\end{itemize}
\par Abbiamo dunque
\[
	T(n) \leq n - 1 + T(I) + T(n - 1 - I).
\]
\par Ora, supponiamo che come \Francais{pivot} sia scelto sempre il minimo:
abbiamo
\[
	T(n) = n - 1 + T(n - 1).
\]
\par Ne deduciamo
\begin{align*}
  T(n)
  &= \sum_{k = 0}^{n - 1} k,\\
  &= \frac{n(n - 1)}{2},\\
  &= \BigTheta{n^2}.
\end{align*}
\par Quindi al caso pessimo $T(n) = \BigTheta{n^2}$.
\par Per il caso ottimo, osserviamo che l'albero di ricorsione contiene meno di
$n$ nodi e che l'albero di $n$ nodi e di altezza minima possibile $h$
verifica la condizione $2^h \leq n$ e dunque necessariamente $h \leq \log n$.
Il costo dell'algoritmo sar\`a allora $\BigO{n \log n}$. \EndProof
\begin{Definition}
  Chiamiamo
  \Define{\English{quick sort} aleatorio}[aleatorio][\English{quick sort}]
  l'algoritmo di \English{quick sort} ottenuto scegliendo $P \in n$
  al passo \ref{QuickSort_2} con probabilit\`a uniforme.
\end{Definition}
\begin{Theorem}
  Il valore atteso del costo del \English{quick sort} aleatorio \`e
  $\BigO{n \log n}$.
\end{Theorem}
\Proof Assumiamo $n \geq 2$, che non \`e restrittivo visto che siamo interessati
ad un'analisi asintotica.
\par Definiamo la successione $(f_r)_{r \in n}$ definita da
$f_r = e_{\Permutation{r}}$
dove
$\Permutation \in \SymmetricGroup{n}$
\`e la permutazione che riordina la successione $(e_i)_{i \in n}$ di partenza.
\par Introduciamo per ogni $(i,j) \in n \times n$ la variabile aleatoria
$\RandomVariable_{i,j}$ il cui valore \`e uguale al numero di volte in cui
l'elemento $f_i$ \`e stato confrontato con l'elemento $f_j$.
\par Osserviamo che i confronti espliciti avvengono solo ai passi
\ref{QuickSort_5}
e
\ref{QuickSort_6}
e in entrambi i casi uno dei due elementi \`e \Francais{pivot}.
Inoltre, poich\'e il \Francais{pivot} \`e escluso dalle sottosuccessioni
su cui viene riapplicato l'algoritmo al passo
\ref{QuickSort_10}, ogni confronto tra due elementi avviene al massimo
una volta.
\par Dunque la variabile aleatoria
$\RandomVariable
= \sum_{\substack{(i,j) \in n \times n\\i < j}} \RandomVariable_{i,j}$
\`e proprio il costo dell'esecuzione del \English{quick sort} aleatorio.
\par Dati $(i,j) \in n \times n$ con $i < j$, la probabilit\`a che il confronto
tra $f_i$ e $f_j$ avvenga equivale alla probabilit\`a che esso avvenga
all'ultima chiamata dell'algoritmo che comprende entrambi gli elementi nella
successione di ingresso. Inoltre, quest'ultima chiamata contiene
necessariamente anche ogni $f_k$ con $k \in j \SetMin i$ poich\'e
\begin{itemize}
  \item o l'istanza \`e l'istanza iniziale e allora contiene tutti gli $f_k$ con
    $k \in n$;
  \item o l'istanza precedente ha definito un \Francais{pivot}, che \`e
    necessariamente o pi\`u grande o pi\`u piccolo sia di $f_i$ che di $f_j$ e
    dunque, per transitivit\`a, anche pi\`u grande o pi\`u piccolo di ogni $f_k$
    con $k \in j \SetMin i$.
\end{itemize}
\par Abbiamo allora che la probabilit\`a in questione \`e al pi\`u di
$\frac{2}{j - i + 1}$.
\par Abbiamo infine
\begin{align*}
  \ExpectedValue{\RandomVariable}
  &= \ExpectedValue{
    \sum_{\substack{(i,j) \in n \times n\\i < j}} \RandomVariable_{i,j}},\\
  &= \sum_{\substack{(i,j) \in n \times n\\i < j}}
    \ExpectedValue{\RandomVariable_{i,j}},\\
  &\leq \sum_{\substack{(i,j) \in n \times n\\i < j}}
    \frac{2}{j - i + 1},\\
  &< \sum_{\substack{(i,j) \in n \times n\\i < j}}
    \frac{1}{j - i + 1},\\
  &= \sum_{i = 0}^{n - 1} \sum_{k = 0}^{n - i}
    \frac{1}{k},\\
  &\leq n \sum_{k = 0}^{i - n + 2}
    \frac{1}{k},\\
  &= \BigO{n \log n}.\text{ \EndProof}
\end{align*}
\begin{Corollary}
  Il costo medio del \English{quick sort} deterministico \`e
  $\BigO{n \log n}$.
\end{Corollary}
%Dimostrazione vaga: da migliorare.
\Proof Si consideri lo spazio $\SamplesSpace$ di tutte le successioni di
lunghezza $n$, nel quale si ritengono uguali due successioni
$(e_i)_{i \in n}$
e
$(f_i)_{i \in n}$
tali che esista una permutazione
$\Permutation \in \SymmetricGroup{n}$
per cui
$(e_{\Permutation(i)})_{i \in n}$
e
$(f_{\Permutation(i)})_{i \in n}$
siano entrambe ordinate. Il costo medio si ottiene calcolando il valore
atteso del costo dell'algoritmo considerando una distribuzione di probabilit\`a
uniforme si $\SamplesSpace$. Tale distribuzione induce una scelta del
\Francais{pivot} in ogni istanza dell'algoritmo con probabilit\`a uniforme
su tutti gli elementi della sottosuccessione passata in ingresso all'istanza
stessa, che \`e esattamente quanto previsto dal \English{quick sort} aleatorio.
\par Dunque il costo medio del \English{quick sort} deterministico coincide col
valore atteso del \English{quick sort} aleatorio e questo vale anche per
un'analisi asintotica. \EndProof

\subsubsection{\English{Quick select}.}
\label{AlgoritmiEStruttureDiDati_QuickSelect}
\par Simile al \English{quick sort} \`e l'algoritmo seguente, che non risolve
per\`o il problema di ordinamento, bens\`i individua quale sarebbe il $k$-esimo
elemento della successione $(e_i)_{i \in n}$ dopo ordinamento.
\begin{Definition}
	Si chiama
  \Define{\English{quick select}}
  l'algoritmo seguente:
  \begin{enumerate}
    \item\label{QuickSelect_1} se $n = 1$, allora si restituisca $e_0$;
    \item\label{QuickSelect_2} si fissi un $P \in n$: chiamiamo
      \Define{\Francais{pivot}}[nel \English{quick select}][\Francais{pivot}]
      l'elemento $e_P$;
    \item\label{QuickSelect_3} scambiamo $e_P$ con $e_{n - 1}$;
    \item\label{QuickSelect_4} si ponga $I = 0$ e $J = n - 2$;
    \item\label{QuickSelect_5} se
      $I \leq J$ e $e_I < e_{n - 1}$
      si incrementi $I$ di $1$ e si torni al punto
      \ref{QuickSelect_5}
    \item\label{QuickSelect_6} se
      $I \leq J$ e $e_J > e_{n - 1}$
      si decrementi $J$ di $1$ e si torni al punto
      \ref{QuickSelect_6}
    \item\label{QuickSelect_7} se $I < J$ si scambino
      $e_I$ e $e_J$;
    \item\label{QuickSelect_8} se $I \leq J$, allora vai al punto
      \ref{QuickSelect_5};
    \item\label{QuickSelect_9} si scambino
      $e_I$ e $e_{n - 1}$;
    \item\label{QuickSelect_10} se $k < I + 1$, si resituisca quanto restituito
      dall'esecuzione dell'intero algoritmo sulla $(e_i)_{i = 0}^{I - 1}$;
    \item\label{QuickSelect_11} se $k > I + 1$, si resituisca quanto restituito
      dall'esecuzione dell'intero algoritmo sulla $(e_i)_{i = I + 1}^{n - 1}$;
    \item\label{QuickSelect_12} si resituisca $e_I$.
  \end{enumerate}
\end{Definition}
\begin{listing}
	\insertcode{cpp}{"Algoritmi_e_strutture_di_dati/QuickSelect.cpp"}
	\caption{\textit{Quick select} implementato in \LanguageName{C++}.}
\end{listing}
\begin{Theorem}
  Il \English{quick select} risolve il problema dell'individuazione di quello
  che sarebbe il $k$-esimo elemento della successione $(e_i)_{i \in n}$ dopo
  ordinamento.
\end{Theorem}
\Proof Se $n = 1$ la tesi \`e immediata, altrimenti procediamo
per induzione su $n$ assumendo la tesi vera per ogni $n < N$
($N \in \mathbb{N} \SetMin 2$) e proviamola per $N$.
\par Per induzione, \`e sufficiente provare che, dopo il passo
\ref{QuickSelect_9} tutti gli elementi della successione
$(e_i)_{i = 0}^{I - 1}$
sono minori o uguali di $e_I$ e che tutti gli elementi della successione
$(e_i)_{I + 1}^{N - 1}$
sono maggiori o uguali di $e_I$.
\par Ma questo \`e esattamente quanto abbiamo gi\`a dimostrato per provare
la correttezza del \English{quick sort}. \EndProof
\begin{Theorem}
  Il \English{quick select} ha costo
  \begin{itemize}
    \item $\BigO{n^2}$ tempo al caso pessimo;
    \item $\BigO{n}$ tempo al caso ottimo.
  \end{itemize}
\end{Theorem}
\Proof Se $n = 0$ o $n = 1$ il costo \`e esattamente $0$.
\par Supponiamo $n \geq 2$ senza perdita di generalit\`a dato che
siamo interessati ad un'analisi asintotica.
\par Contiamo i confronti propri di ciascuna singola esecuzione dell'algoritmo:
\begin{itemize}
  \item nella chiamata iniziale abbiamo $n - 1$ confronti;
  \item nella chiamata successiva abbiamo
      \begin{itemize}
        \item se $k < I$, $I - 1$ confronti se $I \geq 2$, altrimenti $0$;
        \item se $k > I$, $n - 1 - I - 1 = n - I - 2$ confronti se
          $n - 1 - I \geq 2$, cio\`e $I \leq n - 3$, altrimenti $0$;
      \end{itemize}
      dunque in ogni caso abbiamo $\BigO{n}$ confronti.
\end{itemize}
\par Ad ogni chiamata corrisponde la scelta di un \Francais{pivot},
tutti i \Francais{pivot} sono necessariamente distinti e
non possono essere scelti pi\`u di $n - 1$ \Francais{pivot}, dunque
l'altezza dell'albero di ricorsione \`e al pi\`u $n - 1$, e quindi il costo
\`e al pi\`u $(n - 1) \BigO{n} = \BigO{n^2}$.
\par Supponiamo ora invece che il primo \Francais{pivot} corrisponda proprio
alla posizione $k$: in tal caso il costo \`e esattamente $n - 1$, e poich\'e non
pu\`o essere minore, \`e necessariamente il caso ottimo.
\par Dunque il costo al caso ottimo\`e $\BigO{n}$. \EndProof
\begin{Definition}
  Chiamiamo
  \Define{\English{quick select} aleatorio}[aleatorio][\English{quick select}]
  l'algoritmo di \English{quick select} ottenuto scegliendo $P \in n$
  al passo \ref{QuickSelect_2} con probabilit\`a uniforme.
\end{Definition}
\begin{Theorem}
  Il valore atteso del costo del \English{quick select} aleatorio \`e
  $\BigO{n}$.
\end{Theorem}

