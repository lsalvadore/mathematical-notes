\subsection{Pila.}
\label{AlgoritmiEStruttureDiDati_Pila}
\begin{Definition}
	Si dice che una struttura di dati implementa la strategia \Define{LIFO}
  (\English{Last in first out}) quando memorizza dati in successione e, quando
  interrogata, restituisce ed espelle il dato memorizzato pi\`u di recente.
\end{Definition}
\begin{Definition}
  Una struttura di dati che implementa la strategia LIFO si chiama \Define{pila}
  (\Define{\English{stack}}). Inoltre,
  \begin{itemize}
    \item il metodo che aggiunge un dato alla pila \`e solitamente chiamato
      \Define{\English{push}};
    \item il metodo che estrae un dato dalla pila \`e solitamente chiamato
      \Define{\English{pop}}.
  \end{itemize}
\end{Definition}
\par Un'implementazione di una pila tramite \English{array}
richiede normalmente di conoscere a tempo di compilazione qual \`e il massimo
numero di elementi che possa essere impilato; tuttavia, qualora ci\`o non fosse
possibile, esiste comunque la possibilit\`a di ricollocare una pila esistente in
un nuovo \English{array} pi\`u lungo quando
necessario, e successivamente di ricollocare nuovamente la pila in un
\English{array} pi\`u piccolo qualora si ritenesse utile mantenere limitata la
quantit\`a di memoria occupata da un programma. Al fine di ammortizzare i costi
di ricollocazione, conviene evitare di allungare o accorciare l'\English{array}
di una cella per volta, per esempio preferendo invece procedere per
raddoppiamenti e dimezzamenti.
\begin{listing}
	\insertcode{cpp}{"Algoritmi_e_strutture_di_dati/Pila.h"}
	\caption{Implementazione di una pila in \LanguageName{C++}: dichiarazione della classe.}
\end{listing}
\begin{listing}
	\insertcode{cpp}{"Algoritmi_e_strutture_di_dati/Pila.cpp"}
	\caption{Implementazione di una pila in \LanguageName{C++}: costruttore.}
\end{listing}
\begin{listing}
	\insertcode{cpp}{"Algoritmi_e_strutture_di_dati/Pila_Push.cpp"}
	\caption{Implementazione di un metodo \LanguageName{C++} per l'aggiunta di un elemento ad una pila.}
\end{listing}
\begin{listing}
	\insertcode{cpp}{"Algoritmi_e_strutture_di_dati/Pila_Pop.cpp"}
	\caption{Implementazione di un metodo \LanguageName{C++} per l'estrazione di un elemento da una pila.}
\end{listing}
\begin{listing}
	\insertcode{cpp}{"Algoritmi_e_strutture_di_dati/Pila_Top.cpp"}
	\caption{Implementazione di un metodo \LanguageName{C++} per la restituzione di un elemento da una pila, ma senza estrazione.}
\end{listing}
\begin{listing}
	\insertcode{cpp}{"Algoritmi_e_strutture_di_dati/Pila_Vuota.cpp"}
	\caption{Implementazione di un metodo \LanguageName{C++} per determinare se la pila \`e vuota.}
\end{listing}
