\subsection{\English{Array} circolare.}
\label{AlgoritmiEStruttureDiDati_ArrayCircolare}
\begin{Definition}
  Un
  \Define{\English{array} circolare}[circolare][\English{array}]
  \`e una struttura di dati caratterizzata da
  \begin{itemize}
    \item una dimensione $n \in \mathbb{N}$;
    \item $n$ dati $(e_i)_{i \in n}$;
    \item un sistema di indicizzazione su $\mathbb{Z}$ (nella pratica, su un suo
      sottoinsieme sufficientemente grande) tale che l'elemento indicizzato da
      $i \in \mathbb{Z}$ \`e l'elemento $e_I$ dove $I \in \mathbb{Z}$ \`e il
      pi\`u piccolo rappresentante non negativo della classe di resto di $i$
      modulo $n$.
  \end{itemize}
\end{Definition}
\begin{listing}
	\insertcode{cpp}{"Algoritmi_e_strutture_di_dati/ArrayCircolare.h"}
	\caption{Implementazione di un \English{array} circolare in
    \LanguageName{C++}: dichiarazione della classe.}
\end{listing}
\begin{listing}
	\insertcode{cpp}{"Algoritmi_e_strutture_di_dati/ArrayCircolare.cpp"}
	\caption{Implementazione di un \English{array} circolare in
    \LanguageName{C++}: costruttore e distruttore.}
\end{listing}
\begin{listing}
	\insertcode{cpp}{"Algoritmi_e_strutture_di_dati/ArrayCircolare_Indexing.cpp"}
	\caption{Implementazione di un \English{array} circolare in
    \LanguageName{C++}: definizione dell'operazione di indicizzazione.}
\end{listing}
