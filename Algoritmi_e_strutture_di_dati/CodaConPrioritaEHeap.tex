\subsection{Coda con priorit\`a e \English{heap}.}
\label{AlgoritmiEStruttureDiDati_CodaConPrioritaEHeap}
\begin{Definition}
  Una \Define{coda con priorit\`a}[con priorit\`a][coda]
  (\Define{\English{priority queue}}) \`e una struttura di dati che memorizza
  dati in successione accoppiati con un numero chiamato \Define{priorit\`a} e li
  restituisce in ordine di priorit\`a, decrescente o crescente a seconda
  dell'implementazione.
  Inoltre, come per le code senza priorit\`a,
  \begin{itemize}
    \item il metodo che aggiunge un dato alla coda \`e solitamente chiamato
      \Define{\English{enqueue}};
    \item il metodo che estrae un dato dalla coda \`e solitamente chiamato
      \Define{\English{dequeue}}.
  \end{itemize}
\end{Definition}
\par Salvo indicazioni contrarie, assumeremo sempre che una coda di priorit\`a
estrae gli elementi con priorit\`a massima.
Ai fini della teoria, possiamo assumere tutte le
priorit\`a distinte senza perdita di generalit\`a: infatti, se una coda di
priorit\`a ha pi\`u elementi con la massima priorit\`a, allora dovr\`a essere
usato un criterio per scegliere quale elemento effettivamente estrarre; ma 
questo criterio pu\`o anche essere scelto per aumentare la priorit\`a di
quell'elemento o abbassare la priorit\`a degli altri, e un ragionamento analogo
vale per qualsiasi altra situazione in cui serva discrimanre tra elementi con
la stessa prioririt\`a.
\par Inoltre, senza perdita di generalit\`a, esamineremo solo code con
priorit\`a che memorizzano dati vuoti, cio\`e che memorizzano solo le priorit\`a
stesse: in effetti, il funzionamento della struttura dipende esclusivamente
dalle priorit\`a, mentre per quanto riguarda i costi in termini di spazio,
baster\`a aumentare il costo della priorit\`a in modo tale che esso includa
anche il costo del dato ad esso associato, operazione che per un'analisi
asintotica non implica nessuna conseguenza.
\par Una possibile implementazione di una coda con priorit\`a prevede l'uso di
liste; essa consente
\begin{itemize}
  \item l'accodamento di nuovi dati in tempo $\BigO{1}$ mettendo in testa alla
    lista il nuovo elemento;
  \item l'estrazione di dati in tempo $\BigO{n}$, leggendo tutti gli elementi
    della lista in successione per individuare l'elemento di priorit\`a pi\`u
    alta o pi\`u bassa.
\end{itemize}
\par Un'ulteriore possibilit\`a \`e l'implementazione tramite un \English{array}
ordinato, che naturalmente deve essere sufficientemente grande o essere
ricollocato in \English{array} pi\`u o meno grandi a seconda delle necessit\`a.
In tal caso, accodamento ed estrazione possono essere effettuati sfruttando
l'algoritmo di ricerca binaria e hanno dunque entrambi costo $\BigO{\log n}$.
\par Generalmente, le code con priorit\`a vengono implementate tramite un'aaltra
struttura ancora denominata \English{heap}, motivo per cui le code di priorit\`a
sono spesso chiamate anche \English{heap}.
\begin{Definition}
  Si chiama \Define{\English{heap}} un albero completo da sinistra che registra
  dati numerici tale che una delle condizioni seguenti sia verificata
  \begin{itemize}
    \item se $\Node$ \`e genitore di $\VarNode$, allora 
      $\Node \geq \VarNode$: in tal caso si dice che l'\English{heap}
      \`e un \Define{\English{heap} $\max$};
    \item se $\Node$ \`e genitore di $\VarNode$, allora 
      $\Node \leq \VarNode$: in tal caso si dice che l'\English{heap}
      \`e un \Define{\English{heap} $\min$}.
  \end{itemize}
  Se l'\English{heap} \`e un albero $n$-ario, allora si dice che
  l'\English{heap} \`e \English{$n$-ario}[$n$-ario][\English{heap}].
\end{Definition}
\begin{Definition}
  Si chiama
  \Define{\English{heap} binario implicito}[binario implicito][\English{heap}]
  la struttura di dati definita da un \English{array} di numeri
  $(e_i)_{i \in n}$ che verifica una delle due delle condizioni seguenti:
  \begin{itemize}
    \item \TheoremName{propriet\`a di \English{heap} $\max$}[di \English{heap} $\max$][propriet\`a]
      se $i \in n$, allora
      \begin{itemize}
        \item se $j = \left \lfloor \frac{i - 1}{2} \right \rfloor \in n$,
          allora $e_j \geq e_i$;
        \item se $j = 2i + 1 \in n$,
          allora $e_i \geq e_j$;
        \item se $j = 2i + 2 \in n$,
          allora $e_i \geq e_j$:
      \end{itemize}
      in tal caso si dice che l'\English{heap} binario implicito \`e un
      \Define{\English{heap} binario implicito $\max$}[binario implicito $\max$][\English{heap}];
    \item \TheoremName{propriet\`a di \English{heap} $\min$}[di \English{heap} $\min$][propriet\`a]
      se $i \in n$, allora
      \begin{itemize}
        \item se $j = \left \lfloor \frac{i - 1}{2} \right \rfloor \in n$,
          allora $e_j \leq e_i$;
        \item se $j = 2i + 1 \in n$,
          allora $e_i \leq e_j$;
        \item se $j = 2i + 2 \in n$,
          allora $e_i \leq e_j$:
      \end{itemize}
      in tal caso si dice che l'\English{heap} binario implicito \`e un
      \Define{\English{heap} binario implicito $\min$}[binario implicito $\min$][\English{heap}].
  \end{itemize}
\end{Definition}
\begin{Theorem}
  Un \English{heap} binario $\max$ di $n \in \NotZero{\mathbb{N}}$ nodi pu\`o
  essere implementato tramite un \English{heap} binario implicito $\max$ di
  dimensione $n$ tramite una corrispondenza che associa al $k$-esimo (contando
  da $0$) nodo da sinistra di livello $L$ l'elemento di indice $2^L + k - 1$
  dell'\English{heap} binario impliciato in modo tale che il dato numerico
  registrato nel nodo sia lo stesso registrato nell'elemento dell'\English{heap}
  binario implicito ad esso associato.
\end{Theorem}
\Proof Occorre verificare che l'\English{array} immagine della corrispondenza
sia effettivamente un \English{heap} binario implicito.
\par Consideriamo il $k$-esimo (contando da $0$) nodo $\Node$ di livello $L$
dell'\English{heap} binario. I suoi figli, se esistono, si ottengono scorrendo i
figli di tutti i nodi pi\`u a sinistra di $\Node$: essi sono in numero $2k$.
Dunque, i figli di $\Node$ sono, se esistono, il $2k$-esimo e il
$(2k + 1)$-esimo nodo al livello $L + 1$: le loro immagini, rispetto alla
corrispondenza dell'enunciato, sono
l'$(2^{L + 1} + 2k - 1)$-esimo
e
l'$(2^{L + 1} + 2k)$-esimo elemento dell'\English{array}.
\par Fissiamo ora $I \in n$. Siano
\begin{itemize}
  \item $L \in \mathbb{N}$;
  \item $k \in \mathbb{N}$ con $0 \leq k < 2^L$;
\end{itemize}
tali che $I = 2^L + k - 1$.
Denotiamo $(e_i)_{i \in n}$ gli elementi dell'\English{array}.
\par Per costruzione, $e_I$ \`e immagine del $k$-esimo (contando da $0$) nodo
$\Node$ di livello $L$ dell'\English{heap} binario. Per la propritet\`a di
\English{heap} $\max$, il nodo \`e maggiore o uguale dei suoi figli, e dunque
\begin{itemize}
  \item se $J = 2I + 1 \in n$, abbiamo $J = 2^{L + 1} + 2k - 1$ e $e_J$ \`e
    l'immagine del $(2k + 1)$-esimo nodo di livello $L + 1$, dunque
    $e_I \geq e_J$;
  \item se $J = 2I + 2 \in n$, abbiamo $J = 2^{L + 1} + 2k$ e $e_J$ \`e
    l'immagine del $(2k)$-esimo nodo di livello $L + 1$, dunque
    $e_I \geq e_J$.
\end{itemize}
\par Inoltre,
\begin{itemize}
  \item se $I = 0$, allora
    $J = \left \lfloor \frac{I - 1}{2} \right \rfloor = - 1 \notin n$;
  \item se $I > 0$ \`e dispari, allora $L > 0$, $k > 0$ e $k$ \`e pari, da cui
    $J = \left \lfloor \frac{I - 1}{2} \right \rfloor
    = \frac{2^L + k - 2}{2} = 2^{L - 1} + \frac{k}{2} - 1$: $e_J$ \`e l'immagine
    del $\frac{k}{2}$-esimo nodo di livello $L - 1$ e ha per figlio sinistro
    l'$I$-esimo nodo di livello $L$, dunque $e_J \geq e_I$;
  \item se $I > 0$ \`e pari, allora $L > 0$, $k > 0$ e $k$ \`e dispari, da cui
    $J = \left \lfloor \frac{I - 1}{2} \right \rfloor
    = \left \lfloor \frac{2^L + k - 2}{2} \right \rfloor
    = 2^{L - 1} + \frac{k - 1}{2} - 1$: $e_J$ \`e
    l'immagine del $\frac{k - 1}{2}$-esimo nodo di
    livello $L - 1$ e ha per figlio destro l'$I$-esimo nodo di livello $L$,
    dunque $e_J \geq e_I$. \EndProof
\end{itemize}
\begin{Theorem}
  Un \English{heap} binario di $n \in \mathbb{N}$ elementi costa
  \begin{itemize}
    \item $\BigO{n}$ spazio se implementato tramite liste;
    \item $n$ spazio se implementato tramite \English{heap} binario implicito.
  \end{itemize}
\end{Theorem}
\Proof L'implementazione tramite liste richiede la memorizzazione di puntatori,
mentre l'implementazione tramite \English{heap} binario implicito memorizza solo
le priorit\`a. \EndProof
