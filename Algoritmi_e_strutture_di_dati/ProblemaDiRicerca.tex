\subsection{Problema di ricerca.}
\label{AlgoritmiEStruttureDiDati_ProblemaDiRicerca}
\begin{Definition}
	Chiamiamo \Define{problema di ricerca}[di ricerca][problema] il problema che
  consiste nel ricercare un elemento in una successione finita e, nel caso esso
  sia effettivamente presente, restituire la posizione esatta di un elemento
  nella successione uguale all'elemento cercato.
\end{Definition}
\par Nel seguito denoteremo
\begin{itemize}
  \item $e$ l'elemento cercato;
  \item $(e_i)_{i \in n}$ ($n \in \mathbb{N}$) la
    successione nella quale avviene la ricerca.
\end{itemize}
\begin{Theorem}
	Il problema di ricerca in una successione ordinata ha complessit\`a
  $\BigOmega{\log n}$.
\end{Theorem}
\Proof L'elemento cercato pu\`o coincidere con uno degli $n$ elementi della
successione o essere assente.
\par Effettuando $1$ confronto, l'algoritmo pu\`o discriminare tra al pi\`u $2$
configurazioni: denotati con $a$ e $b$ i due elementi confrontati, pu\`o
rilevare se $a = b$ o se $a \neq b$. Assumiamo che con $k$
confronti l'algoritmo possa discriminare tra $2^k$ confronti: allora, con $1$
confronto aggiuntivo, esso pu\`o discrimare tra $2^k \cdot 2 = 2^{k + 1}$
configurazioni. Dunque, segue per induzione che con $k$ confronti si pu\`o
discrimare tra $2^k$ configurazioni.
\par Perch\'e l'algoritmo possa fornire la risposta corretta al problema, \`e
dunque necessario che il numero $k$ di confronti da esso effettuato verifichi la
condizione $2^k \geq n + 1$, equivalente a $k \geq \log_2 (n + 1)$. La tesi
segue dal fatto che $\IsBigOmega{\log_2 (n + 1)}{\log n}$. \EndProof
\begin{Definition}
  Supponiamo la successione $(e_i)_{i \in n}$ ordinata secondo un ordinamento
dato $\leq$. Si chiama \Define{ricerca binaria}[binaria][ricerca] l'algoritmo
seguente:
  \begin{enumerate}
    \item\label{RicercaBinaria_1} se $n = 1$ e $e_0 \neq e$, allora restituire
      $-1$ (o un qualunque altro valore convenzionale per indicare l'assenza
      dell'elemento cercato nella successione);
    \item\label{RicercaBinaria_2} se $n = 1$ e $e_0 = e$, restituire $0$;
    \item\label{RicercaBinaria_3} si ponga
      $M = \left \lfloor \frac{n - 1}{2} \right \rfloor$;
    \item\label{RicercaBinaria_4} se $e \leq e_M$, allora si restituisca quanto
      restituito dall'esecuzione dell'intero algoritmo cercando $e$ nella
      successione $(e_i)_{i = 0}^M$;
    \item\label{RicercaBinaria_5} denotato $I$ il valore
      restituito dall'esecuzione dell'intero algoritmo cercando $e$ nella
      successione
      $(f_i)_{i \in n - M - 1}$,
      dove per ogni $i \in n - M - 1$ poniamo
      $f_i = e_{i + M + 1}$,
      si restituisca $M + 1 + I$.
  \end{enumerate}
\end{Definition}
\begin{listing}
	\insertcode{cpp}{"Algoritmi_e_strutture_di_dati/RicercaBinaria.cpp"}
	\caption{Ricerca binaria implementata in \LanguageName{C++}.}
\end{listing}
\begin{Theorem}
  La ricerca binaria risolve il problema della ricerca per una successione
  ordinata.
\end{Theorem}
\Proof Supponiamo $e$ non faccia parte della successione $(e_i)_{i \in n}$.
Allora i passi \ref{RicercaBinaria_4} e \ref{RicercaBinaria_5} accorceranno
la successione ove avviene la ricerca fino a ridurla ad un solo elemento:
per il passo \ref{RicercaBinaria_1} verr\`a resituito $-1$.
\par Supponiamo ora $e$ appaia in $(e_i)_{i \in n}$.
Se $n = 1$ la tesi \`e immediata, altrimenti procediamo
per induzione su $n$ assumendo la tesi vera per ogni $n < N$
($N \in \mathbb{N} \SetMin 2$) e proviamola per $N$.
\par Se $e \leq e_M$, allora la successione
$(e_i)_{i = 0}^M$
contiene $e$ e per ipotesi induttiva il passo \ref{RicercaBinaria_4}
trova l'elemento $e$; altrimenti \`e la successione
$(e_i)_{i = M + 1}^{n - 1}$ a contenere $e$, e per ipotesi induttiva
il passo \ref{RicercaBinaria_5} trova l'elemento $e$. \EndProof
\begin{Theorem}
  La ricerca binaria ha costo $\BigO{\log n}$, secondo il \English{comparison
  model}.
\end{Theorem}
\Proof Denotiamo $T(n)$ il costo della ricerca binaria in una successione di
dimensione $n$. Abbiamo evidentemente $T(1) = 1$ (i confronti dei passi
\ref{RicercaBinaria_1} e \ref{RicercaBinaria_2} sono lo stesso confronto: non
serve verificare due volte se $e_0 = e$; il confronto viene ripetuto nella
descrizione dell'algoritmo solo per chiarezza di esposizione).
\par Contiamo i confronti, assumendo $n > 1$; dato che siamo interessati ad una
stima asintotica, possiamo assumere $n$ pari senza perdita di generalit\`a:
\begin{itemize}
  \item non contiamo il confronto dei passi \ref{RicercaBinaria_1} e
    \ref{RicercaBinaria_2} perch\'e superfluo dopo aver gi\`a osservato
    che $n = 1$ (potremmo comunque anche contarlo ottenendo lo stesso risultato
    asintotico);
  \item abbiamo il confronto $e \leq e_M$ al passo \ref{RicercaBinaria_4};
  \item se effettivamente $e \leq e_M$, allora abbiamo
    $T(M + 1) = T \left ( \left \lfloor \frac{n + 1}{2} \right \rfloor \right )
      = T \left ( \frac{n}{2} \right )$
    ulteriori confronti;
  \item se invece $e > e_M$, allora abbiamo
    $T(n - 1 - (M + 1) + 1) = T \left ( n - \frac{n}{2} \right )
      = T \left ( \frac{n}{2} \right )$
    ulteriori confronti.
\end{itemize}
\par Dunque, per opportune costanti $C_0, C > 0$, abbiamo
\[
	T(n) \leq
	\begin{cases}
		1\text{ se }n = 1,\\
		T(\frac{n}{2}) + 1\text{ per } n > 1.
	\end{cases}
\]
\par Per il teorema fondamentale delle ricorrenze, abbiamo
$T(n) = \BigO{\log n}$. \EndProof
\begin{Corollary}
	Il problema di ricerca in una successione ordinata ha complessit\`a
  $\BigTheta{n \log n}$ e la ricerca binaria \`e asintoticamente ottima.
\end{Corollary}
\Proof Abbiamo gi\`a dimostrato che il problema di ricerca in una successione
ordinata  ha complessit\`a
$\BigOmega{\log n}$.
Il teorema precedente ha provato che il problema ha complessit\`a
$\BigO{\log n}$.
Ne deduciamo infine che esso ha complessit\`a $\BigTheta{\log n}$. \EndProof
