\subsection{Divide et impera.}
\label{AlgoritmiEStruttureDiDati_DivideEtImpera}
\begin{Theorem}
  \TheoremName{Teorema fondamentale delle ricorrenze}[fondamentale delle ricorrenze][teorema]
  Siano
  \begin{itemize}
    \item $f: \mathbb{N} \rightarrow \RealNonNegative$ una funzione non decresente;
    \item $\alpha \geq 1$;
    \item $\beta \geq 1$;
    \item $n_0, c_0 \geq 0$.
  \end{itemize}
  Sia $T: \mathbb{N} \rightarrow \RealNonNegative$ tale che
  \[
    T(n) \leq
    \begin{cases}
      c_0\text{ se }n \leq n_0,\\
      \alpha T \left ( \rho \left (\frac{n}{\beta} \right ) \right ) + f(n)\text{ altrimenti}.
    \end{cases}
  \]
  con
  $\rho \left ( \frac{n}{\beta} \right )
    = \left \lfloor \frac{n}{\beta} \right \rfloor$
  o
  $\rho \left ( \frac{n}{\beta} \right )
    = \left \lceil \frac{n}{\beta} \right \rceil$.
  \par Supponiamo che esista $\gamma > 0$ tale che per $n$ sufficientemente
  grande abbiamo
  $\alpha f \left ( \rho \left ( \frac{n}{\beta} \right ) \right )
    \leq \gamma f(n)$.
  Abbiamo
  \begin{itemize}
    \item $T(n) = \BigO{f(n)}$ se $\gamma < 1$;
    \item $T(n) = \BigO{f(n) \log_\beta n}$ se $\gamma = 1$;
    \item $T(n) = \BigO{n^{\log_\beta \alpha}}$ se $\gamma > 1$.
  \end{itemize}
\end{Theorem}
\begin{listing}
	\insertcode{cpp}{"Algoritmi_e_strutture_di_dati/DivideEtImpera.cpp"}
	\caption{Implementazione dell'algoritmo di ricerca binaria in \LanguageName{C++}.}
\end{listing}
