\section{Definizioni e teoremi di base.}
\label{AlgoritmiEStruttureDiDati_DefinizioniETeoremiDiBase}
\begin{Definition}
	Chiamiamo \Define{algoritmo} una successione di operazioni svolte a partire da un insieme di dati inizali, detti \Define{dati in ingresso}[in ingresso][dati] o \Define{\English{input}}, che termina e, terminando, restituisce un insieme di dati finali, detti \Define{dati in uscita}[in uscita][dati] o \Define{\English{output}}.
\end{Definition}
\begin{Definition}
	Un \Define{problema} \`e definito da
	\begin{itemize}
		\item un insieme di dati iniziali, detti \Define{dati in ingresso}[in ingresso][dati] o \Define{\English{input}};
		\item una regola di trasformazione dei dati iniziali in dati finali, detti \Define{dati in uscita}[in uscita][dati] o \Define{\English{output}}.
	\end{itemize}
\end{Definition}
\begin{Definition}
	Dato un problema, un algoritmo si dice \Define{corretto} per quel problema se applicato ai dati di ingresso del problema restituisce i dati in uscita del problema. Diciamo anche che l'algoritmo \Define{risolve}[di un problema][risoluzione] il problema.
\end{Definition}
\begin{Definition}
	Nella ricerca di algoritmo che risolva un problema, chiamiamo \Define{baseline} la prima versione di un algoritmo individuata che risolve il problema, versione che contiamo migliorare successivamente.
\end{Definition}
