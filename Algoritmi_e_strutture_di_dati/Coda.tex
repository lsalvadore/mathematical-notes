\subsection{Coda.}
\label{AlgoritmiEStruttureDiDati_Coda}
\begin{Definition}
	Si dice che una struttura di dati implementa la strategia \Define{FIFO}
  (\English{First in first out}) quando memorizza dati in successione e, quando
  interrogata, restituisce ed espelle il dato memorizzato meno di recente.
\end{Definition}
\begin{Definition}
  Una struttura di dati che implementa la strategia FIFO si chiama \Define{coda}
  (\Define{\English{queue}}). Inoltre,
  \begin{itemize}
    \item il metodo che aggiunge un dato alla coda \`e solitamente chiamato
      \Define{\English{enqueue}};
    \item il metodo che estrae un dato dalla coda \`e solitamente chiamato
      \Define{\English{dequeue}}.
  \end{itemize}
\end{Definition}
\par Strutture di dati particolaremente utili per l'implementazione di una coda
sono l'\English{array} circolare e la lista. Generalmente, \`e preferita la
lista, che come si pu\`o osservare dai codici \`e di pi\`u immediata
implementazione.
\par Per l'esemplificazione di un'implementazione
tramite \English{array} circolare, ci asteniamo da utilizzare i codici gi\`a
usati nella sezione \ref{AlgoritmiEStruttureDiDati_ArrayCircolare} perch\'e non
sono convenienti: in effetti, come si pu\`o notare, gli indici
\mintinline{cpp}{Cima} e
\mintinline{cpp}{Fondo} crescono sempre di un'unit\`a per volta e necessitano,
in generale, di essere rimpiazzati da rappresentati della stessa classe di resto
modulo la lunghezza della coda per evitare errori generati da un eventuale
traboccamento.
\par Segnaliamo, senza dettagli, che \`e possibile implementare una coda anche
con un \English{array} non circolare: ci\`o richiede attenzione con l'uso dei
puntatori che tengono conto del progresso della cosa e il mantenimento di una
cella di memoria vuota per poter distinguere casi limiti di riempimento della
coda.
\par Un'implementazione di una coda tramite \English{array}, circolari o meno,
richiede normalmente di conoscere a tempo di compilazione qual \`e il massimo
numero di elementi che possa essere accodato; tuttavia, qualora ci\`o non fosse
possibile, esiste comunque la possibilit\`a di ricollocare una coda esistente in
un nuovo \English{array} pi\`u lungo quando necessario, e successivamente di
ricollocare nuovamente la coda in un
\English{array} pi\`u piccolo qualora si ritenesse utile mantenere limitata la
quantit\`a di memoria occupata da un programma. Al fine di ammortizzare i costi
di ricollocazione, conviene evitare di allungare o accorciare l'\English{array}
di una cella per volta, per esempio preferendo invece procedere per
raddoppiamenti e dimezzamenti.
\begin{listing}
	\insertcode{cpp}{"Algoritmi_e_strutture_di_dati/Coda_ArrayCircolare.h"}
	\caption{Implementazione di una coda in \LanguageName{C++} tramite
    \English{array} circolare: dichiarazione della classe.}
\end{listing}
\begin{listing}
	\insertcode{cpp}{"Algoritmi_e_strutture_di_dati/Coda_Enqueue_ArrayCircolare.cpp"}
	\caption{Implementazione di un metodo \LanguageName{C++} per l'aggiunta di un
    elemento ad una coda implementata tramite \English{array} circolare.}
\end{listing}
\begin{listing}
	\insertcode{cpp}{"Algoritmi_e_strutture_di_dati/Coda_Dequeue_ArrayCircolare.cpp"}
	\caption{Implementazione di un metodo \LanguageName{C++} per l'estrazione di
    un elemento da una coda implementata tramite \English{array} circolare.}
\end{listing}
\begin{listing}
	\insertcode{cpp}{"Algoritmi_e_strutture_di_dati/Coda_RestituisciPrimo_ArrayCircolare.cpp"}
	\caption{Implementazione di un metodo \LanguageName{C++} per la restituzione
    di un elemento da una coda implementata tramite \English{array} circolare,
    ma senza estrazione.}
\end{listing}
\begin{listing}
	\insertcode{cpp}{"Algoritmi_e_strutture_di_dati/Coda_Vuota_ArrayCircolare.cpp"}
	\caption{Implementazione di un metodo \LanguageName{C++} per determinare se
    una coda implementata tramite \English{array} circolare \`e vuota.}
\end{listing}
\begin{listing}
	\insertcode{cpp}{"Algoritmi_e_strutture_di_dati/Coda_Lista.h"}
	\caption{Implementazione di una coda in \LanguageName{C++} tramite una lista:
    dichiarazione della classe.}
\end{listing}
\begin{listing}
	\insertcode{cpp}{"Algoritmi_e_strutture_di_dati/Coda_Lista.cpp"}
	\caption{Implementazione di una coda in \LanguageName{C++} tramite una lista:
    distruttore.}
\end{listing}
\begin{listing}
	\insertcode{cpp}{"Algoritmi_e_strutture_di_dati/Coda_Enqueue_Lista.cpp"}
	\caption{Implementazione di un metodo \LanguageName{C++} per l'aggiunta di un
    elemento ad una coda implementata tramite una lista.}
\end{listing}
\begin{listing}
	\insertcode{cpp}{"Algoritmi_e_strutture_di_dati/Coda_Dequeue_Lista.cpp"}
	\caption{Implementazione di un metodo \LanguageName{C++} per l'estrazione di
    un elemento da una coda implementata tramite una lista.}
\end{listing}
\begin{listing}
	\insertcode{cpp}{"Algoritmi_e_strutture_di_dati/Coda_RestituisciPrimo_Lista.cpp"}
	\caption{Implementazione di un metodo \LanguageName{C++} per la restituzione
    di un elemento da una coda implementata tramite una lista, ma senza
    estrazione.}
\end{listing}
\begin{listing}
	\insertcode{cpp}{"Algoritmi_e_strutture_di_dati/Coda_Vuota_Lista.cpp"}
	\caption{Implementazione di un metodo \LanguageName{C++} per determinare se
    una coda implementata tramite una lista \`e vuota.}
\end{listing}
