\section{Problema di Keplero.}
\label{ElementiDiMeccanicaCeleste_ProblemaDiKeplero}
Assumiamo nel seguito di aver fissato un sistema di riferimento inerziale.
\begin{Definition}
	Chiamiamo \Define{problema di Keplero}[di Keplero][problema] il problema dello studio del moto di un punto materiale di massa $\ReducedMass$ in un campo gravitazionale generato da una massa $\TotalMass$ vincolata nell'origine del sistema.
\end{Definition}
\begin{figure}
	\includegraphics[width=0.8\textwidth]{Elementi_di_meccanica_celeste/Immagine_Keplero.png}
	\centering
	\caption{Problema di Keplero.}
\end{figure}
\begin{Theorem}
	Il moto del punto materiale del problema di Keplero giace in un piano ortogonale al momento angolare del punto materiale stesso.
\end{Theorem}
\Proof Segue direttamente dal fatto che la forza che definisce il problema di Keplero \`e centrale. \EndProof
\par Nel seguito suporremo fissato un sistema di riferimento inerziale $(\vec{e_x},\vec{e_y},\vec{e_z})$ tale che $\ver{e_x}$ e $\ver{e_y}$ giaciano nel piano orbitale del problema di Keplero Utilizzermo inoltre un secondo sistema di riferimento $(\ver{e_r},\ver{e_\theta},\ver{e_z})$, definito da
\begin{itemize}
	\item $\ver{e_r} = \frac{\vec{r}}{r}$;
	\item $\ver{e_\theta} = \ver{e_z} \times \ver{e_r}$.
\end{itemize}
\begin{Definition}
	Dato un punto materiale $\PointMass$ di massa $\Mass$, chiamiamo \Define{momento angolare per unit\`a di massa}[angolare per unit\`a di massa][momento] il vettore $\vec{\AngularMomentPerUnitMass} = \frac{\AngularMoment}{\Mass}$, dove $\AngularMoment$ \`e il momento angolare del punto $\PointMass$.
\end{Definition}
\begin{Theorem}
	Il momento angolare per unit\`a di massa di un punto materiale $\PointMass$ in posizione $\vec{r}$ in moto nel piano dei versori $\vec{e_x}$ e $\vec{e_y}$\ \`e $\vec{\AngularMomentPerUnitMass} = r^2\dot{\theta}\ver{e_z}$.
\end{Theorem}
\Proof Denotata $\Mass$ la massa del punto materiale, abbiamo $\vec{\AngularMomentPerUnitMass} = \frac{\vec{\AngularMoment}}{\Mass} = \frac{\VectorProduct{\vec{r}}{\Mass \dot{\vec{r}}}}{\Mass} = \VectorProduct{\vec{r}}{\dot{\vec{r}}} = \VectorProduct{r \ver{e_r}}{\dot{r}\ver{e_r} + r\dot{\theta}\ver{e_\theta}} = r^2\dot{\theta}) \ver{e_z}$. \EndProof
\begin{Theorem}
	Il problema di Keplero
	\[
		\ReducedMass \ddot{\vec{r}} = - \NewtonConstant \frac{\TotalMass\ReducedMass}{r^3}\vec{r}
	\]
	\`e equivalente a
	\[
		\begin{cases}
			\ddot{r} = \frac{\AngularMomentPerUnitMass}{r^3} - \NewtonConstant \frac{\TotalMass}{r^2},\\
			2 \dot{r}\dot{\theta} + r \ddot{\theta} = 0,
		\end{cases}
	\]
	dove $\AngularMomentPerUnitMass$ \`e il momento angolare del punto materiale di massa $\ReducedMass$ e $\theta$ l'anomalia del vettore $\vec{r}$, tramite la trasformazione che trasforma la base cartesiana $(\ver{e_x},\ver{e_y},\ver{e_z})$ nella base polare $(\ver{e_r},\ver{e_\theta},\ver{e_z})$, definita da $\ver{e_r} = \frac{\vec{r}}{r}$ e $\ver{e_\theta} = \ver{e_z} \times \ver{e_r}$.
\end{Theorem}
\Proof Essendo il campo di forze di un problema di Keplero centrale,
\[
	\ReducedMass \ddot{r} = - \NewtonConstant \frac{\TotalMass\ReducedMass}{r^3}\vec{r}
\]
equivale, tramite trasformazione di cambiamento delle coordinate cartesiani in coordinate polari,
\[
	\begin{cases}
		\ReducedMass (\ddot{r} - r \dot{\theta}^2) = - \NewtonConstant \frac{\TotalMass\ReducedMass}{r^2},\\
		\ReducedMass (2\dot{r}\dot{\theta} + r \ddot{\theta}) = 0;
	\end{cases}
\]
vale a dire, dividendo per $\ReducedMass$,
\[
	\begin{cases}
		\ddot{r} - r \dot{\theta}^2 = - \NewtonConstant \frac{\TotalMass}{r^2},\\
		2\dot{r}\dot{\theta} + r \ddot{\theta} = 0.
	\end{cases}
\]
\par Utilizzando l'equazione $\AngularMomentPerUnitMass = r^2\dot{\theta}$, possiamo ricondurre la prima equazione a un'equazione nella sola incognita $r$, ottenendo infine
\[
	\begin{cases}
		\ddot{r} = \frac{\AngularMomentPerUnitMass}{r^3} - \NewtonConstant \frac{\TotalMass}{r^2},\\
		2 \dot{r}\dot{\theta} + r \ddot{\theta} = 0.\text{ \EndProof}
	\end{cases}
\]
\begin{Definition}
	\label{ElementiDiMeccanicaCeleste_VettoreDiLenz}
	Dato un problema di Keplero
	\[
		\ReducedMass \ddot{\vec{r}} = - \NewtonConstant \frac{\TotalMass\ReducedMass}{r^3}\vec{r},
	\]
	definiamo \Define{vettore di Lenz}[di Lenz][vettore] o \Define{vettore di Laplace-Runge-Lenz}[di Laplace-Runge-Lenz][vettore] il vettore
	\[
		\vec{\LenzVector} = \frac{1}{\NewtonConstant\TotalMass} \VectorProduct{\dot{\vec{r}}}{\vec{\AngularMomentPerUnitMass}} - \ver{e_r}.
	\]
\end{Definition}
\begin{Theorem}
	Il vettore di Lenz $\vec{\LenzVector}$ \`e costante.
\end{Theorem}
\Proof Con le notazioni della definizione \ref{ElementiDiMeccanicaCeleste_VettoreDiLenz}, abbiamo
\begin{align*}
	\dot{\vec{\LenzVector}}
	&= \frac{1}{\NewtonConstant \TotalMass} (\VectorProduct{\ddot{\vec{r}}}{\vec{\AngularMomentPerUnitMass}} + \VectorProduct{\dot{\vec{r}}}{\dot{\AngularMomentPerUnitMass}}) - \dot{\ver{e_r}},\\
	&= \frac{1}{\NewtonConstant \TotalMass} (\VectorProduct{\ddot{\vec{r}}}{\vec{\AngularMomentPerUnitMass}}) - \dot{\ver{e_r}},\\
	&= - \frac{1}{\NewtonConstant \TotalMass} (\frac{\NewtonConstant\TotalMass}{r^3}\VectorProduct{\vec{r}}{\vec{\AngularMomentPerUnitMass}}) - \dot{\ver{e_r}},\\
	&= - r^{-2} \VectorProduct{\ver{e_r}}{\vec{\AngularMomentPerUnitMass}}) - \dot{\ver{e_r}},\\
	&= - r^{-2} r^2 \dot{\theta} \VectorProduct{\ver{e_r}}{\ver{e_z}}) - \dot{\ver{e_r}},\\
	&= - \dot{\theta} \VectorProduct{\ver{e_r}}{\ver{e_z}}) - \dot{\ver{e_r}},\\
	&= \dot{\theta} \ver{e_\theta} - \dot{\theta}\ver{e_\theta} = 0.\text{ \EndProof}
\end{align*}
\begin{Definition}
	\label{ElementiDiMeccanicaCeleste_AnomaliaVera}
	Dato un problema di Keplero
	\[
		\ReducedMass \ddot{\vec{r}} = - \NewtonConstant \frac{\TotalMass\ReducedMass}{r^3}\vec{r},
	\]
	definiamo \Define{anomalia vera}[vera][anomalia] la quantit\`a $\TrueAnomaly = \theta - \delta$, dove $\delta$ \`e l'anomalia del vettore di Lenz.
\end{Definition}
\begin{figure}
	\includegraphics[width=0.8\textwidth]{Elementi_di_meccanica_celeste/Immagine_Keplero_AnomaliaVera.png}
	\centering
	\caption{Definizione di anomalia vera (notazione della definizione \ref{ElementiDiMeccanicaCeleste_AnomaliaVera}).}
\end{figure}
\begin{Lemma}
	Con le notazioni della definizione precedente, abbiamo $\dot{\TrueAnomaly} = \dot{\theta}$.
\end{Lemma}
\Proof Abbiamo $\dot{\TrueAnomaly} = \dot{\theta} - \dot{\delta} = \dot{\theta}$. \EndProof
\begin{Theorem}
	L'orbita del problema di Keplero
	\[
		\ReducedMass \ddot{\vec{r}} = - \NewtonConstant \frac{\TotalMass\ReducedMass}{r^3}\vec{r},
	\]
	in dipendenza dell'anomalia vera $\TrueAnomaly$ \`e
	\[
		r(\TrueAnomaly) = \frac{\SemilatusRectum}{1 + \LenzVector \cos{\TrueAnomaly}},
	\]
	dove $\SemilatusRectum = \frac{\AngularMomentPerUnitMass^2}{\NewtonConstant \TotalMass}$.
\end{Theorem}
\Proof Abbiamo $\StandardScalarProduct{\ver{e_r}}{\LenzVector} = \LenzVector \cos{\TrueAnomaly}$.
\par Inoltre,
\begin{align*}
	\StandardScalarProduct{\ver{e_r}}{\LenzVector}
	&= \StandardScalarProduct{\ver{e_r}}{\frac{1}{\NewtonConstant\TotalMass}\VectorProduct{\dot{\vec{r}}}{\vec{\AngularMomentPerUnitMass}} - \ver{e_r}},\\
	&= \frac{1}{\NewtonConstant\TotalMass} \StandardScalarProduct{\ver{e_r}}{\VectorProduct{\dot{\vec{r}}}{\vec{\AngularMomentPerUnitMass}}} - 1,\\
	&= \frac{1}{\NewtonConstant\TotalMass} \StandardScalarProduct{\vec{\AngularMomentPerUnitMass}}{\VectorProduct{\ver{e_r}}{\dot{\vec{r}}}} - 1,\\
	&= \frac{1}{\NewtonConstant\TotalMass} \StandardScalarProduct{\vec{\AngularMomentPerUnitMass}}{\VectorProduct{\ver{e_r}}{\dot{r}\ver{e_r} + r\dot{\TrueAnomaly}\ver{e_\theta}}} - 1,\\
	&= \frac{r\dot{\TrueAnomaly}}{\NewtonConstant\TotalMass} \StandardScalarProduct{\vec{\AngularMomentPerUnitMass}}{\ver{e_z}} - 1,\\
	&= \frac{r\dot{\TrueAnomaly}}{\NewtonConstant\TotalMass}\AngularMomentPerUnitMass - 1,\\
	&= \frac{\AngularMomentPerUnitMass^2}{r\NewtonConstant\TotalMass} - 1.
\end{align*}
\par Dunque $\LenzVector \cos{\TrueAnomaly} = \frac{\AngularMomentPerUnitMass^2}{r\NewtonConstant\TotalMass} - 1$, equivalente a $\frac{\SemilatusRectum}{1 + \LenzVector \cos{\TrueAnomaly}}$. \EndProof
\begin{Theorem}
	L'orbita del problema di Keplero
	\[
		\ReducedMass \ddot{\vec{r}} = - \NewtonConstant \frac{\TotalMass\ReducedMass}{r^3}\vec{r}
	\]
	\`e una conica di eccentricit\`a $\LenzVector$.
\end{Theorem}
\Proof $\frac{\SemilatusRectum}{1 + \LenzVector \cos{\TrueAnomaly}}$ equivale a $r = \SemilatusRectum - r \LenzVector\cos{\TrueAnomaly}$, vale a dire, in coordinate cartesiane, $\sqrt{x^2 + y^2} = \SemilatusRectum - \LenzVector\cos{\TrueAnomaly}$, a sua volta equivalente a $x^2 + y^2 = (\SemilatusRectum - \LenzVector x)^2$. \EndProof
\begin{Corollary}
	\TheoremName{Classificazione delle orbite del problema di Keplero in base al modulo del vettore di Lenz}[delle orbite del problema di Keplero in base al modulo del vettore di Lenz][classificazione]
	L'orbita del problema di Keplero
	\[
		\ReducedMass \ddot{\vec{r}} = - \NewtonConstant \frac{\TotalMass\ReducedMass}{r^3}\vec{r}
	\]
	\`e
	\begin{itemize}
		\item un'ellisse se $0 \leq \LenzVector < 1$, e in particolare una circonferenza per $\LenzVector = 0$;
		\item una parabola se $\LenzVector = 1$;
		\item un ramo di iperbole se $\LenzVector > 1$.
	\end{itemize}
\end{Corollary}
\Proof Segue direttamente dal teorema precedente e dalla classificazione delle coniche in base all'eccentricit\`a. \EndProof
\begin{Definition}
	Data un'orbita del problema di Keplero
	\[
		r(\TrueAnomaly) = \frac{\SemilatusRectum}{1 + \LenzVector \cos{\TrueAnomaly}},
	\]
	dove
	\begin{itemize}
		\item $\SemilatusRectum$ \`e il semilato retto dell'orbita;
		\item $\LenzVector$ \`e il modulo del vettore di Lenz del problema, cio\`e l'eccentricit\`a dell'orbita,
	\end{itemize}
	chiamiamo
	\begin{itemize}
		\item \Define{pericentro} o \Define{periapside} o ancora, nel
      caso del sistema solare, \Define{perielio} il punto dell'orbita a
      distanza $\min r$;
		\item \Define{apocentro} o \Define{apoapside} o ancora, nel caso del
      sistema solare, \Define{afelio} il punto dell'orbita a distanza
      $\max r$.
	\end{itemize}
\end{Definition}
\begin{Theorem}
	Data un'orbita del problema di Keplero
	\[
		r(\TrueAnomaly) = \frac{\SemilatusRectum}{1 + \LenzVector \cos{\TrueAnomaly}},
	\]
	dove
	\begin{itemize}
		\item $\SemilatusRectum$ \`e il semilato retto dell'orbita;
		\item $\LenzVector$ \`e il modulo del vettore di Lenz del problema, cio\`e l'eccentricit\`a dell'orbita.
	\end{itemize}
	Se l'orbita \`e ellittica, allora
	\begin{itemize}
		\item il suo semiasse maggiore \`e $\SemiMajorAxis = \frac{\SemilatusRectum}{1 - \LenzVector^2}$;
		\item il suo semiasse minore \`e $\SemiMinorAxis = \SemiMajorAxis \sqrt{1 - \Eccentricity^2}$;
		\item la sua semidistanza focale \`e $\SemiFocalLength = \SemiMajorAxis \LenzVector$;
		\item $\min r = \SemiMajorAxis(1 - \LenzVector)$;
		\item $\max r = \SemiMajorAxis(1 + \LenzVector)$.
	\end{itemize}
	Se l'orbita \`e parabolica, allora
	\begin{itemize}
		\item $\min r = \frac{\SemilatusRectum}{2}$;
		\item l'equazione della parabola nel sistema di riferimento $Oxy$, dove $O$ coincide con l'origine del sistema di partenza e $x$ \`e nella direzione del vettore di Lenz $\LenzVector$, \`e $x = - \frac{y^2}{2 \SemilatusRectum} + \frac{\SemilatusRectum}{2}$.
	\end{itemize}
	Se l'orbita \`e iperbolica, allora
	\begin{itemize}
		\item $\min r = \frac{\SemilatusRectum}{1 + \LenzVector} < \frac{\SemilatusRectum}{2}$;
		\item il semiasse maggiore \`e $- \SemiMajorAxis = - \frac{\SemilatusRectum}{1 - \LenzVector^2}$;
		\item la sua semidistanza focale \`e $\SemiFocalLength = \AbsoluteValue{\SemiMajorAxis} \LenzVector$;
		\item $\cos \nu_{as} - \frac{1}{\LenzVector}$, dove $\nu_as$ \`e l'angolo formato da un asintoto con la retta la cui direzione \`e data dal vettore di Lenz (per esempio quello segnato in figura \ref{ElementiDiMeccanicaCeleste_Keplero_Iperbole}).
	\end{itemize}
\end{Theorem}
\begin{figure}
	\includegraphics[width=0.8\textwidth]{Elementi_di_meccanica_celeste/Immagine_Keplero_Ellisse.png}
	\centering
	\caption{Orbita ellittica per un problema di Keplero.}
\end{figure}
\Proof Supponiamo l'orbita ellittica. Abbiamo
\begin{itemize}
	\item $2 \SemiMajorAxis = r(0) + r(\pi) = \frac{\SemilatusRectum}{1 + \LenzVector \cos{0}} + \frac{\SemilatusRectum}{1 + \LenzVector \cos(\pi)} = \frac{\SemilatusRectum}{1 + \LenzVector} + \frac{\SemilatusRectum}{1 - \LenzVector} = \frac{2 \SemilatusRectum}{1 - \LenzVector^2}$, da cui $\SemiMajorAxis = \frac{\SemilatusRectum}{1 - \LenzVector^2}$;
	\item $\SemiFocalLength = \SemiMajorAxis \LenzVector$ e $\SemiMinorAxis = \SemiMajorAxis \sqrt{1 - \LenzVector}$ per le propriet\`a geometriche delle ellissi;
	\item $\min r = r(0) = \frac{\SemilatusRectum}{1 + \LenzVector} = \SemiMajorAxis (1 - \LenzVector)$;
	\item $\max r = r(0) = \frac{\SemilatusRectum}{1 - \LenzVector} = \SemiMajorAxis (1 + \LenzVector)$.
\end{itemize}
\begin{figure}
	\includegraphics[width=0.8\textwidth]{Elementi_di_meccanica_celeste/Immagine_Keplero_Parabola.png}
	\centering
	\caption{Orbita parabolica per un problema di Keplero.}
\end{figure}
\par Supponiamo l'orbita parabolica. Abbiamo
\begin{itemize}
	\item $\min r = \frac{\SemilatusRectum}{1 + \LenzVector} = \frac{\SemilatusRectum}{2}$;
	\item per ragioni di simmetria, l'equazione della parabola \`e della forma $x = c_1y^2 + c_2y + c_3$ per opportuni $c_1, c_2, c_3 \in \mathbb{R}$; abbiamo
\[
\begin{cases}
	- \frac{c_2}{2 c_1} = 0,\\
	c_3 = \SemilatusRectum,\\
	c_1 \SemilatusRectum^2 + c_2 \SemilatusRectum + c_3 = 0,\\
\end{cases}
\]
	e quindi $x = -\frac{y^2}{2 \SemilatusRectum} + \frac{\SemilatusRectum}{2}$.
\end{itemize}
\begin{figure}
	\label{ElementiDiMeccanicaCeleste_Keplero_Iperbole}
	\includegraphics[width=0.8\textwidth]{Elementi_di_meccanica_celeste/Immagine_Keplero_Iperbole.png}
	\centering
	\caption{Orbita iperbolica per un problema di Keplero.}
\end{figure}
\par Supponiamo l'orbita iperbolica. Abbiamo
\begin{itemize}
	\item $\min r = \frac{\SemilatusRectum}{1 + \LenzVector} < \frac{\SemilatusRectum}{2}$ poich\'e $\LenzVector > 1$;
	\item $\begin{cases}\SemiFocalLength = - \LenzVector \SemiMajorAxis,\\\SemiFocalLength =  - \SemiMajorAxis - \frac{\SemilatusRectum}{1 + \LenzVector}\end{cases}$, da cui $\SemiMajorAxis = \frac{\SemilatusRectum}{(1 - \LenzVector)^2}$;
	\item $\frac{\SemilatusRectum}{1 + \LenzVector \cos{\TrueAnomaly}} \rightarrow + \infty$ se e solo se $\cos{\TrueAnomaly} \rightarrow - \frac{1}{\LenzVector}$, dunque $\nu_{as} = \frac{1}{\LenzVector}$. \EndProof
\end{itemize}
\begin{Theorem}
	L'energia per unit\`a di massa $\EnergyPerUnitMass$ del problema di Keplero
	\[
		\ReducedMass \ddot{\vec{r}} = - \NewtonConstant \frac{\TotalMass\ReducedMass}{r^3}\vec{r}
	\]
	\`e
	\[
		\EnergyPerUnitMass =
		\begin{cases}
			0\text{, se l'orbita \`e parabolica;}\\
			- \NewtonConstant \frac{\TotalMass}{2 \SemiMajorAxis}\text{, altrimenti.}
		\end{cases}
	\]
	Nel secondo caso, $\SemiMajorAxis$ indica
	\begin{itemize}
		\item il semiasse maggiore dell'ellisse nel caso di un'orbita ellittica;
		\item l'opposto del semiasse maggiore dell'iperbole nel caso di un'orbita iperbolica.
	\end{itemize}
\end{Theorem}
\par Usiamo le notazioni del teorema \label{ElementiDiMeccanicaCeleste_VettoreDiLenz} e denotiamo con $\TrueAnomaly$ l'anomalia vera. Abbiamo
\begin{align*}
	\vec{\LenzVector}
	&= \frac{1}{\NewtonConstant\TotalMass} \VectorProduct{\dot{\vec{r}}}{\vec{\AngularMomentPerUnitMass}} - \ver{e_r},\\
	&= \frac{1}{\NewtonConstant\TotalMass} \VectorProduct{\dot{r}\ver{e_r} + r\dot{\TrueAnomaly}\ver{e_{\TrueAnomaly}}}{r^2\dot{\TrueAnomaly} \ver{e_z}} - \ver{e_r},\\
	&= \frac{1}{\NewtonConstant\TotalMass} (- r^2 \dot{r} \dot{\TrueAnomaly} \ver{e_\TrueAnomaly} + r^3 \dot{\TrueAnomaly}^2) - \ver{e_r},\\
	&= \left ( \frac{r^3\dot{\TrueAnomaly}^2}{\NewtonConstant\TotalMass} - 1 \right ) \ver{e_r} - \frac{r^2\dot{r}\dot{\TrueAnomaly}}{\NewtonConstant\TotalMass} \ver{e_\TrueAnomaly}.
\end{align*}
\par Abbiamo dimostrato che nessuna orbita del problema di Keplero passa dall'origine, cio\`e che per ogni punto dell'orbita $r \neq 0$. Abbiamo adesso
\begin{align*}
	\LenzVector^2
	&= \left ( \frac{r^3\dot{\TrueAnomaly}^2}{\NewtonConstant\TotalMass} - 1 \right )^2 + \left ( \frac{r^2\dot{r}\dot{\TrueAnomaly}}{\NewtonConstant\TotalMass} \right )^2,\\
	&= \frac{r^6\dot{\TrueAnomaly}^4}{\NewtonConstant^2\TotalMass^2} - 2 \frac{r^3\dot{\TrueAnomaly}^2}{\NewtonConstant\TotalMass} + 1 + \frac{r^4\dot{r}^2\dot{\TrueAnomaly}^2}{\NewtonConstant^2\TotalMass^2},\\
	&= 1 + 2 \frac{r^3\dot{\TrueAnomaly}^2}{\NewtonConstant^2\TotalMass^2} \left (\frac{r^3\dot{\TrueAnomaly}^2}{2} - \NewtonConstant\TotalMass + \frac{r\dot{r}^2}{2} \right ),\\
	&= 1 + 2 \frac{r^4\dot{\TrueAnomaly}^2}{\NewtonConstant^2\TotalMass^2} \left (\frac{r^2\dot{\TrueAnomaly}^2}{2} - \frac{\NewtonConstant\TotalMass}{r} + \frac{\dot{r}^2}{2} \right ),\\
	&= 1 + 2 \frac{r^4\dot{\TrueAnomaly}^2}{\NewtonConstant^2\TotalMass^2} \left (\frac{r^2\dot{\TrueAnomaly}^2 + \dot{r}}{2} - \frac{\NewtonConstant\TotalMass}{r} \right ),\\
	&= 1 + 2 \frac{r^4\dot{\TrueAnomaly}^2}{\NewtonConstant^2\TotalMass^2} \left (\frac{\Norm{\dot{\vec{r}}^2}}{2} - \frac{\NewtonConstant\TotalMass}{r} \right ),\\
	&= 1 + 2 \frac{r^4\dot{\TrueAnomaly}^2}{\NewtonConstant^2\TotalMass^2} \EnergyPerUnitMass,\\
	&= 1 + 2 \frac{\AngularMomentPerUnitMass^2}{\NewtonConstant^2\TotalMass^2} \EnergyPerUnitMass,\\
	&= 1 + 2 \frac{\SemilatusRectum}{\NewtonConstant\TotalMass} \EnergyPerUnitMass.
\end{align*}
\par Se l'orbita \`e parabolica, allora $\LenzVector = 1$ da cui $\EnergyPerUnitMass = 0$.
\par Supponiamo ora l'orbita ellittica o iperbolica. In tal caso \`e definito il semiasse maggiore $\SemiMajorAxis$ e abbiamo $\EnergyPerUnitMass = 1 + 2 \frac{\SemiMajorAxis (1 - \LenzVector^2)}{\NewtonConstant\TotalMass} \EnergyPerUnitMass$. Ne deduciamo $\EnergyPerUnitMass = \frac{(\LenzVector^2 - 1)\NewtonConstant\TotalMass}{2\SemiMajorAxis(1 - \LenzVector^2)} = - \frac{\NewtonConstant\TotalMass}{2\SemiMajorAxis}$. \EndProof
\begin{Corollary}
	\TheoremName{Classificazione delle orbite del problema di Keplero in base all'energia}[delle orbite del problema di Keplero in base all'energia][classificazione]
	L'orbita del problema di Keplero
	\[
		\ReducedMass \ddot{\vec{r}} = - \NewtonConstant \frac{\TotalMass\ReducedMass}{r^3}\vec{r}
	\]
	\`e, denotata $\EnergyPerUnitMass$ l'energia per unit\`a di massa,
	\begin{itemize}
		\item un'ellisse se $\EnergyPerUnitMass < 0$;
		\item una parabola se $\EnergyPerUnitMass = 0$;
		\item un ramo di iperbole se $\EnergyPerUnitMass > 0$.
	\end{itemize}
\end{Corollary}
\Proof Segue direttamente dal teorema precedente. \EndProof
\begin{Corollary}
	Nelle ipotesi del teorema precedente, se l'orbita \`e aperta, abbiamo
	\begin{itemize}
		\item $\lim_{r \rightarrow + \infty} \Norm{\dot{\vec{r}}} = 0$ nel caso parabolico;
		\item $\lim_{r \rightarrow + \infty} \Norm{\dot{\vec{r}}} = \sqrt{\NewtonConstant \frac{\TotalMass}{\SemiMajorAxis}}$.
	\end{itemize}
\end{Corollary}
\Proof Se l'orbita \`e parabolica, allora $\frac{\Norm{\dot{\vec{r}}}^2}{2} - \NewtonConstant \frac{\TotalMass}{r} = 0$, da cui $\lim_{r \rightarrow \infty} \Norm{\dot{\vec{r}}} = 0$.
\par Se l'orbita \`e iperbolica, allora
\[
\begin{cases}
	\EnergyPerUnitMass = \frac{\Norm{\dot{\vec{r}}}^2}{2} - \NewtonConstant \frac{\TotalMass}{r},\\
	\EnergyPerUnitMass = - \NewtonConstant \frac{\TotalMass}{2\SemiMajorAxis},
\end{cases}
\]
da cui $\lim_{r \rightarrow \infty} \Norm{\dot{\vec{r}}} = \sqrt{2\EnergyPerUnitMass} = \sqrt{\NewtonConstant\frac{\TotalMass}{\SemiMajorAxis}}$. \EndProof
\begin{Definition}
	Dato il problema di Keplero
	\[
		\ReducedMass \ddot{\vec{r}} = - \NewtonConstant \frac{\TotalMass\ReducedMass}{r^3}\vec{r},
	\]
	denotato $\AngularMomentPerUnitMass$ il momento angolare per unit\`a di massa, chiamiamo \Define{potenziale efficace}[efficace][potenziale] la quantit\`a $\EffectivePotential = \frac{\AngularMomentPerUnitMass}{2r^2} - \NewtonConstant\frac{\TotalMass}{r}$.
\end{Definition}
\begin{figure}
	\includegraphics[width=0.8\textwidth]{Elementi_di_meccanica_celeste/Immagine_Keplero_PotenzialeEfficace.png}
	\centering
	\caption{Esempio di potenziale efficace di un problema di Keplero.}
\end{figure}
\begin{Theorem}
	Nelle ipotesi della definizione precedente, l'energia per unit\`a di massa $\EnergyPerUnitMass$ \`e data da $\EnergyPerUnitMass = \frac{\dot{r}^2}{2} + \EffectivePotential$.
\end{Theorem}
\Proof Abbiamo, denotando con $\TrueAnomaly$ l'anomalia vera,
\begin{align*}
	\EnergyPerUnitMass
	&= \frac{\Norm{\dot{\vec{r}}}^2}{2} - \NewtonConstant\frac{\TotalMass}{r},\\
	&= \frac{\dot{r}^2 + \dot{r}^2\dot{\TrueAnomaly^2}}{2} - \NewtonConstant\frac{\TotalMass}{r},\\
	&= \frac{\dot{r}^2}{2} + \frac{\AngularMomentPerUnitMass^2}{2r^2} - \NewtonConstant\frac{\TotalMass}{r},\\
	&= \frac{\dot{r}^2}{2} + \frac{\AngularMomentPerUnitMass^2}{2r^2} - \NewtonConstant\frac{\TotalMass}{r}.\text{ \EndProof}
\end{align*}
\begin{Theorem}
	Se l'orbita del problema di Keplero
	\[
		\ReducedMass \ddot{\vec{r}} = - \NewtonConstant \frac{\TotalMass\ReducedMass}{r^3}\vec{r}
	\]
	\`e chiusa, allora \`e periodica.
\end{Theorem}
\Proof Poich\'e l'orbita \`e chiusa, abbiamo $\vec{r}(t + \Period) = \vec{r}(t)$ per opportuno $T \in \mathbb{R}$. Inoltre, per la conversazione del momento angolare per unit\`a di massa, abbiamo
\begin{align*}
	\VectorProduct{\vec{r}(t + \Period)}{\dot{\vec{r}}(t + \Period)}
	&= \vec{\AngularMomentPerUnitMass}(t + \Period),\\
	&= \vec{\AngularMomentPerUnitMass}(t),\\
	&= \VectorProduct{\vec{r}(t)}{\dot{\vec{r}}(t)},\\
	&= \VectorProduct{\vec{r}(t + \Period)}{\dot{\vec{r}}(t)}.
\end{align*}
\par Poich\'e inoltre, per la conservazione dell'energia, il modulo della velocit\`a in $\vec{r}(t + \Period) = \vec{r}(t)$ \`e uguale sia al tempo $t$ che al tempo $t + \Period$, deve essere $\dot{\vec{r}}(t + \Period) = \dot{\vec{r}}(t)$. \EndProof
\begin{Definition}
	Data un'orbita periodica di periodo $\Period$ di un problema di Keplero
	\[
		\ReducedMass \ddot{\vec{r}} = - \NewtonConstant \frac{\TotalMass\ReducedMass}{r^3}\vec{r},
	\]
	definiamo \Define{moto medio}[medio][moto] la quantit\`a $\MeanMotion = \frac{2\pi}{\Period}$.
\end{Definition}
\begin{Theorem}
	\TheoremName{Terza legge di Keplero}[terza di Keplero][legge] 
	Sia il problema di Keplero
	\[
		\ReducedMass \ddot{\vec{r}} = - \NewtonConstant \frac{\TotalMass\ReducedMass}{r^3}\vec{r}.
	\]
	Se l'orbita \`e ellittica con semiasse maggiore $\SemiMajorAxis$ e periodo $\Period$, allora abbiamo
	\[
		\frac{\Period^2}{\SemiMajorAxis^3} = \frac{4\pi^2}{\NewtonConstant\TotalMass},
	\]
	o, equivalentemente, denotando $\MeanMotion$ il moto medio,
	\[
		\SemiMajorAxis^3\MeanMotion^2 = \NewtonConstant\TotalMass.
	\]
\end{Theorem}
\Proof L'area dell'orbita ellittica \`e $\pi\SemiMajorAxis\SemiMinorAxis$, dove $\SemiMinorAxis$ \`e il semiasse minore dell'ellisse. La velocit\`a areolare invece \`e $\frac{r^2\dot{\TrueAnomaly}}{2} = \frac{\AngularMomentPerUnitMass}{2}$.
Abbiamo allora, denotato $\LenzVector$ il modulo del vettore di Lenz,
\begin{align*}
	\frac{\Period^2}{\SemiMajorAxis^3}
	&= \frac{4 \pi^2\SemiMajorAxis^2\SemiMinorAxis^2}{\AngularMomentPerUnitMass^2\SemiMajorAxis^3},\\
	&= \frac{4 \pi^2\SemiMinorAxis^2}{\AngularMomentPerUnitMass^2\SemiMajorAxis},\\
	&= \frac{4 \pi^2 \SemiMajorAxis^2(1 - \LenzVector^2)}{\NewtonConstant\TotalMass\SemiMajorAxis(1 - \LenzVector^2)\SemiMajorAxis},\\
	&= \frac{4 \pi^2}{\NewtonConstant\TotalMass}.\text{ \EndProof}
\end{align*}
\begin{Definition}
	Data un'orbita periodica di periodo $\Period$ di un problema di Keplero
	\[
		\ReducedMass \ddot{\vec{r}} = - \NewtonConstant \frac{\TotalMass\ReducedMass}{r^3}\vec{r},
	\]
	denotati
	\begin{itemize}
		\item $\SemiMajorAxis$ il semiasse maggiore dell'orbita ellittica;
		\item $\EnergyPerUnitMass$ l'energia per unit\`a di massa;
		\item $\MeanMotion$ il \Define{moto medio}[medio][moto],
	\end{itemize}
	abbiamo
	\[
		\MeanMotion = - \frac{2 \EnergyPerUnitMass}{\NewtonConstant^2\TotalMass^2}.
	\]
\end{Definition}
\Proof Abbiamo $\SemiMajorAxis^3 = - \left ( \frac{2 \EnergyPerUnitMass}{\NewtonConstant\TotalMass} \right )^3$ e, per la terza legge di Keplero,
\begin{align*}
	\MeanMotion
	&= \sqrt{\frac{\NewtonConstant\TotalMass}{\SemiMajorAxis^3}},\\
	&= \sqrt{\frac{ - (2 \EnergyPerUnitMass)^3 \NewtonConstant\TotalMass}{\NewtonConstant^3\TotalMass^3}},\\
	&= - \frac{2 \EnergyPerUnitMass}{\NewtonConstant^2\TotalMass^2}.\text{ \EndProof}
\end{align*}
\begin{Definition}
	Dato un problema di Keplero
	\[
		\ReducedMass \ddot{\vec{r}}
    = - \NewtonConstant \frac{\TotalMass\ReducedMass}{r^3}\vec{r},
	\]
	chiamiamo
	\begin{itemize}
		\item
    \Define{tempo di passaggio al pericentro}[di passaggio al pericentro][tempo]
      un qualsiasi tempo $\PericenterPassageTime$ tale che
      $\TrueAnomaly(\PericenterPassageTime) = 0$, dove $\TrueAnomaly$ \`e
      l'anomalia vera;
		\item se l'orbita \`e ellittica,
      fissato un tempo di passaggio al pericentro,
      \Define{anomalia media}[media][anomalia] la quantit\`a
      $\MeanAnomaly(t) = \MeanMotion(t - \PericenterPassageTime)$;
		\item se l'orbita \`e ellittica,
      con riferimento alla figura
      \ref{ElementiDiMeccanicaCeleste_Figura_AnomaliaEccentricaEllittica},
      chiamiamo l'angolo $\EccentricAnomaly$
      \Define{anomalia eccentrica ellittica}[eccentrica ellittica][anomalia]:
      \`e l'angolo individuato dalla direzione $CP$ coincidente con
      quella del vettore di Lenz $\vec{\LenzVector}$ e dalla congiungente
      il centro dell'ellisse $C$ con la proiezione $Q$ del punto $P$
      individuato dal vettore $\vec{r}$ sulla circonferenza di stesso
      centro dell'ellisse e raggio il semiasse maggiore $\SemiMajorAxis$
      dell'ellisse;
    \item se l'orbita \`e iperbolica,
      con riferimento alla figura
      \ref{ElementiDiMeccanicaCeleste_Figura_AnomaliaEccentricaIperbolica},
      denominati $D$ il pericentro dell'orbita,
      $C$ il punto d'intersezione degli asintoti dell'iperbole
      equilatera con vertice sinistro coincidente con $D$,
      $B$ la proiezione della posizione del punto materiale sul semiramo
      di iperbole equilatera dello stesso semipiano (rispetto all'asse $x$),
      chiamiamo la quantit\`a $\EccentricAnomaly$, definita come il doppio
      dell'area compresa tra i segmenti $CB$, $CD$ e
      l'orbita iperbolica divisa per il quadarato del semiasse maggiore
      $\SemiMajorAxis$, con segno positivo se l'area \`e contenuta nel
      semipiano positivo e con segno negativo se l'area \`e contenuta nel
      semipiano negativo,
      \Define{anomalia eccentrica iperbolica}[eccentrica iperbolica][anomalia]
	\end{itemize}
\end{Definition}
\par Ometteremo l'aggettivo ``ellittica'' o ``iperbolica'' quando
il tipo di anomalia eccentrica sar\`a evidente dal contesto. Inoltre,
poich\'e, nel considerare un'unica orbita, \`e definibile al pi\`u  solo
uno dei due tipi di anomalia eccentrica, possiamo utilizzare la stessa
notazione per entrambi i tipi senza rischi di confusione come abbiamo
fatto nella definizione precedente.
\begin{figure}
	\includegraphics[width=0.8\textwidth]{Elementi_di_meccanica_celeste/Immagine_Keplero_AnomaliaEccentricaEllittica.png}
	\centering
	\caption{Definizione di anomalia eccentrica ellittica.}
	\label{ElementiDiMeccanicaCeleste_Figura_AnomaliaEccentricaEllittica}
\end{figure}
\begin{figure}
	\includegraphics[width=0.8\textwidth]{Elementi_di_meccanica_celeste/Immagine_Keplero_AnomaliaEccentricaIperbolica.png}
	\centering
	\caption{Definizione di anomalia eccentrica iperbolica.}
	\label{ElementiDiMeccanicaCeleste_Figura_AnomaliaEccentricaIperbolica}
\end{figure}
\begin{Theorem}
  \label{ElementiDiMeccanicaCeleste_ThTrueEccentric}
	Con le notazioni della definizione precedente, abbiamo
	\begin{itemize}
		\item $r = \SemiMajorAxis(1 - \LenzVector \cos{\EccentricAnomaly})$;
		\item
		$
			\begin{cases}
				\cos{\TrueAnomaly}
          = \frac{\cos{\EccentricAnomaly} - \LenzVector}
              {1 - \LenzVector \cos{\EccentricAnomaly}},\\
				\sin{\TrueAnomaly}
          = \frac{(\sin{\EccentricAnomaly})\sqrt{1 - \LenzVector^2}}
            {1 - \LenzVector \cos{\EccentricAnomaly}}; 
			\end{cases}
		$
    \item $\tan \frac{\TrueAnomaly}{2}
            = \sqrt{\frac{1 + \LenzVector}{1 - \LenzVector}}
              \tan \frac{\EccentricAnomaly}{2}$;
    \item
    $
      \begin{cases}
        \cos{\EccentricAnomaly}
          = \frac{\cos{\TrueAnomaly} + \LenzVector}
              {1 + \LenzVector \cos{\TrueAnomaly}},\\
        \sin{\EccentricAnomaly}
          = \frac{\sqrt{1 - \EccentricAnomaly^2} \sin{\TrueAnomaly}}
              {1 + \LenzVector \cos{\TrueAnomaly}}.
      \end{cases}
    $
	\end{itemize}
\end{Theorem}
\Proof Abbiamo
\begin{itemize}
	\item $\Length{HB} = \SemiMajorAxis \sin{\EccentricAnomaly}$;
	\item $\Length{HA} = r \sin{\TrueAnomaly}$;
	\item $\Length{HA} = \Length{HB} \sqrt{1 - \LenzVector^2}$;
	\item $\Length{OH} = r \cos{\TrueAnomaly}$;
	\item $\Length{OH}
          = \Length{CH} - \Length{CS}
          = \SemiMajorAxis \cos{\EccentricAnomaly}
            - \SemiMajorAxis \LenzVector$.
\end{itemize}
Da cui
\[
\begin{cases}
	r \sin{\TrueAnomaly}
    = \SemiMajorAxis(\sin{\EccentricAnomaly})\sqrt{1 - \LenzVector^2},\\
	r \cos{\TrueAnomaly}
    = \SemiMajorAxis (\cos{\EccentricAnomaly} - \LenzVector).
\end{cases}
\]
\par Abbiamo allora
\begin{align*}
	r^2
	&= r^2(\sin^2{\TrueAnomaly} + \cos^2{\TrueAnomaly}),\\
	&= \SemiMajorAxis^2 (\sin^2{\EccentricAnomaly}) (1 - \LenzVector^2)
    + \SemiMajorAxis^2 (\cos^2{\EccentricAnomaly}
    - 2 \LenzVector \cos{\EccentricAnomaly} + \LenzVector^2),\\
	&= \SemiMajorAxis^2 (\sin^2{\EccentricAnomaly}
    - (\sin^2{\EccentricAnomaly})\LenzVector^2
    + \cos^2{\EccentricAnomaly} - 2 \LenzVector \cos{\EccentricAnomaly}
    + \LenzVector^2),\\
	&= \SemiMajorAxis^2 (1 + (1 - \sin^2{\EccentricAnomaly})\LenzVector^2
    - 2 \LenzVector \cos{\EccentricAnomaly}),\\
	&= \SemiMajorAxis^2 (1 + \cos^2{\EccentricAnomaly}\LenzVector^2
    - 2 \LenzVector \cos{\EccentricAnomaly}),\\
	&= \SemiMajorAxis^2 (1 - \LenzVector \cos{\EccentricAnomaly})^2.
\end{align*}
\par E quindi
\begin{itemize}
	\item $r = \SemiMajorAxis (1 - \LenzVector \cos{\EccentricAnomaly})$;
	\item
	$
		\begin{cases}
			\cos{\TrueAnomaly}
        = \frac{\SemiMajorAxis(\cos{\EccentricAnomaly} - \LenzVector)}
          {\SemiMajorAxis (1 - \LenzVector \cos{\EccentricAnomaly})}
        =  \frac{\cos{\EccentricAnomaly} - \LenzVector}
          {1 - \LenzVector \cos{\EccentricAnomaly}};\\
			\sin{\TrueAnomaly}
        = \frac{\SemiMajorAxis (\sin{\EccentricAnomaly})
            \sqrt{1 - \LenzVector^2})}
          {\SemiMajorAxis(1 - \LenzVector \cos{\EccentricAnomaly})}
        = \frac{\sin{\EccentricAnomaly}\sqrt{1 - \LenzVector^2}}
          {1 - \LenzVector \cos{\EccentricAnomaly}}. 
		\end{cases}
	$
\end{itemize}
\par Abbiamo inoltre
\begin{align*}
  \tan \frac{\TrueAnomaly}{2}
    &= \frac{\sin \TrueAnomaly}{1 + \cos \TrueAnomaly},\\
    &=  \frac{\frac{\sin{\EccentricAnomaly}\sqrt{1 - \LenzVector^2}}
          {1 - \LenzVector \cos{\EccentricAnomaly}}}
         {1 + \frac{\cos{\EccentricAnomaly} - \LenzVector}
          {1 - \LenzVector \cos{\EccentricAnomaly}}},\\
    &= \frac{\sqrt{1 - \LenzVector^2} \sin \EccentricAnomaly}
        {1 - \LenzVector \cos \EccentricAnomaly + \cos \EccentricAnomaly
          -  \LenzVector},\\
    &= \frac{\sqrt{1 - \LenzVector^2}}{1 - \LenzVector}
        \frac{\sin \EccentricAnomaly}{1 + \cos \EccentricAnomaly},\\
    &= \sqrt{\frac{1 + \LenzVector}{1 - \LenzVector}}
        \tan \frac{\EccentricAnomaly}{2}.
\end{align*}
\par Inoltre, da
\[
  \begin{cases}
	  r = \SemiMajorAxis(1 - \LenzVector \cos{\EccentricAnomaly}),\\
    r = \frac{\SemiMajorAxis(1 - \LenzVector^2)}
        {1 + \LenzVector\cos{\TrueAnomaly}},
  \end{cases}
\]
deduciamo
\begin{align*}
  \cos \EccentricAnomaly
  &= - \frac{1}{\LenzVector} \left (
    \frac{1 - \LenzVector^2}{1 + \LenzVector \cos \TrueAnomaly}
    - 1 \right ),\\
  &= \frac{1 + \LenzVector \cos \TrueAnomaly - 1 + \LenzVector^2}
      {\LenzVector(1 + \LenzVector \cos \TrueAnomaly)},\\
  &= \frac{\LenzVector + \cos \TrueAnomaly}
      {1 + \LenzVector \cos \TrueAnomaly};
\end{align*}
e inoltre
\[
	  \frac{\sqrt{1 - \LenzVector^2} \sin \EccentricAnomaly}
      {\sin \TrueAnomaly}
    = \frac{1 - \LenzVector^2}
      {1 + \LenzVector\cos{\TrueAnomaly}},
\]
da cui
\begin{align*}
  \sin \EccentricAnomaly
  &= \frac{\sin \TrueAnomaly}
      {\sqrt{1 - \LenzVector^2}}
      \frac{1 - \LenzVector^2}
      {1 + \LenzVector\cos{\TrueAnomaly}},\\
  &= \frac{\sqrt{1 - \LenzVector^2} \sin \TrueAnomaly}
      {1 + \LenzVector\cos{\TrueAnomaly}}.\text{ \EndProof}
\end{align*}
\begin{Theorem}
  \label{ElementiDiMeccanicaCeleste_ThTrueEccentricHyperbolic}
	Con le notazioni della definizione precedente, abbiamo
	\begin{itemize}
		\item $r = \SemiMajorAxis(1 - \LenzVector \cosh{\EccentricAnomaly})$;
		\item
		$
			\begin{cases}
				\cos{\TrueAnomaly}
          = \frac{\cosh{\EccentricAnomaly} - \LenzVector}
              {1 - \LenzVector \cosh{\EccentricAnomaly}},\\
				\sin{\TrueAnomaly}
          = \frac{- (\sinh{\EccentricAnomaly})\sqrt{\LenzVector^2 - 1}}
            {1 - \LenzVector \cosh{\EccentricAnomaly}}; 
			\end{cases}
		$
    \item $\tanh \frac{\TrueAnomaly}{2}
            = \sqrt{\frac{\LenzVector + 1}{\LenzVector - 1}}
              \tanh \frac{\EccentricAnomaly}{2}$;
    \item
    $
      \begin{cases}
        \cosh{\EccentricAnomaly}
          = \frac{\cos{\TrueAnomaly} + \LenzVector}
              {1 + \LenzVector \cos{\TrueAnomaly}},\\
        \sinh{\EccentricAnomaly}
          = \frac{\sqrt{\EccentricAnomaly^2 - 1} \sin{\TrueAnomaly}}
              {1 + \LenzVector \cos{\TrueAnomaly}}.
      \end{cases}
    $
	\end{itemize}
\end{Theorem}
%\Proof Abbiamo
%\begin{itemize}
%	\item $\Length{HB} = \SemiMajorAxis \sin{\EccentricAnomaly}$;
%	\item $\Length{HA} = r \sin{\TrueAnomaly}$;
%	\item $\Length{HA} = \Length{HB} \sqrt{1 - \LenzVector^2}$;
%	\item $\Length{OH} = r \cos{\TrueAnomaly}$;
%	\item $\Length{OH}
%          = \Length{CH} - \Length{CS}
%          = \SemiMajorAxis \cos{\EccentricAnomaly}
%            - \SemiMajorAxis \LenzVector$.
%\end{itemize}
%Da cui
%\[
%\begin{cases}
%	r \sin{\TrueAnomaly}
%    = \SemiMajorAxis(\sin{\EccentricAnomaly})\sqrt{1 - \LenzVector^2},\\
%	r \cos{\TrueAnomaly}
%    = \SemiMajorAxis (\cos{\EccentricAnomaly} - \LenzVector).
%\end{cases}
%\]
%\par Abbiamo allora
%\begin{align*}
%	r^2
%	&= r^2(\sin^2{\TrueAnomaly} + \cos^2{\TrueAnomaly}),\\
%	&= \SemiMajorAxis^2 (\sin^2{\EccentricAnomaly}) (1 - \LenzVector^2)
%    + \SemiMajorAxis^2 (\cos^2{\EccentricAnomaly}
%    - 2 \LenzVector \cos{\EccentricAnomaly} + \LenzVector^2),\\
%	&= \SemiMajorAxis^2 (\sin^2{\EccentricAnomaly}
%    - (\sin^2{\EccentricAnomaly})\LenzVector^2
%    + \cos^2{\EccentricAnomaly} - 2 \LenzVector \cos{\EccentricAnomaly}
%    + \LenzVector^2),\\
%	&= \SemiMajorAxis^2 (1 + (1 - \sin^2{\EccentricAnomaly})\LenzVector^2
%    - 2 \LenzVector \cos{\EccentricAnomaly}),\\
%	&= \SemiMajorAxis^2 (1 + \cos^2{\EccentricAnomaly}\LenzVector^2
%    - 2 \LenzVector \cos{\EccentricAnomaly}),\\
%	&= \SemiMajorAxis^2 (1 - \LenzVector \cos{\EccentricAnomaly})^2.
%\end{align*}
%\par E quindi
%\begin{itemize}
%	\item $r = \SemiMajorAxis (1 - \LenzVector \cos{\EccentricAnomaly})$;
%	\item
%	$
%		\begin{cases}
%			\cos{\TrueAnomaly}
%        = \frac{\SemiMajorAxis(\cos{\EccentricAnomaly} - \LenzVector)}
%          {\SemiMajorAxis (1 - \LenzVector \cos{\EccentricAnomaly})}
%        =  \frac{\cos{\EccentricAnomaly} - \LenzVector}
%          {1 - \LenzVector \cos{\EccentricAnomaly}};\\
%			\sin{\TrueAnomaly}
%        = \frac{\SemiMajorAxis (\sin{\EccentricAnomaly})
%            \sqrt{1 - \LenzVector^2})}
%          {\SemiMajorAxis(1 - \LenzVector \cos{\EccentricAnomaly})}
%        = \frac{\sin{\EccentricAnomaly}\sqrt{1 - \LenzVector^2}}
%          {1 - \LenzVector \cos{\EccentricAnomaly}}. 
%		\end{cases}
%	$
%\end{itemize}


\par Abbiamo inoltre
\begin{align*}
  \tan \frac{\TrueAnomaly}{2}
    &= \frac{\sin \TrueAnomaly}{1 + \cos \TrueAnomaly},\\
    &=  \frac{\frac{- \sinh{\EccentricAnomaly}\sqrt{\LenzVector^2 - 1}}
          {1 - \LenzVector \cosh{\EccentricAnomaly}}}
         {1 + \frac{\cosh{\EccentricAnomaly} - \LenzVector}
          {1 - \LenzVector \cosh{\EccentricAnomaly}}},\\
    &= \frac{- \sqrt{\LenzVector^2 - 1} \sinh \EccentricAnomaly}
        {1 - \LenzVector \cosh \EccentricAnomaly + \cosh \EccentricAnomaly
          -  \LenzVector},\\
    &= \frac{\sqrt{\LenzVector^2 - 1}}{\LenzVector - 1}
        \frac{\sinh \EccentricAnomaly}{1 + \cosh \EccentricAnomaly},\\
    &= \sqrt{\frac{\LenzVector + 1}{\LenzVector - 1}}
        \tanh \frac{\EccentricAnomaly}{2}.
\end{align*}
\par Inoltre, da
\[
  \begin{cases}
	  r = \SemiMajorAxis(1 - \LenzVector \cosh{\EccentricAnomaly}),\\
    r = \frac{\SemiMajorAxis(1 - \LenzVector^2)}
        {1 + \LenzVector\cos{\TrueAnomaly}},
  \end{cases}
\]
deduciamo
\begin{align*}
  \cosh \EccentricAnomaly
  &= - \frac{1}{\LenzVector} \left (
    \frac{1 - \LenzVector^2}{1 + \LenzVector \cos \TrueAnomaly}
    - 1 \right ),\\
  &= \frac{1 + \LenzVector \cos \TrueAnomaly - 1 + \LenzVector^2}
      {\LenzVector(1 + \LenzVector \cos \TrueAnomaly)},\\
  &= \frac{\LenzVector + \cos \TrueAnomaly}
      {1 + \LenzVector \cos \TrueAnomaly};
\end{align*}
e inoltre
\[
	  \frac{- \sqrt{\LenzVector^2 - 1} \sinh \EccentricAnomaly}
      {\sin \TrueAnomaly}
    = \frac{1 - \LenzVector^2}
      {1 + \LenzVector\cos{\TrueAnomaly}},
\]
da cui
\begin{align*}
  \sinh \EccentricAnomaly
  &= \frac{\sin \TrueAnomaly}
      {\sqrt{\LenzVector^2 - 1}}
      \frac{\LenzVector^2 - 1}
      {1 + \LenzVector\cos{\TrueAnomaly}},\\
  &= \frac{\sqrt{\LenzVector^2 - 1} \sin \TrueAnomaly}
      {1 + \LenzVector\cos{\TrueAnomaly}}.\text{ \EndProof}
\end{align*}
\begin{Theorem}
	\TheoremName{Formula di Keplero}[di Keplero][formula]
	Data un'orbita periodica di periodo $\Period$ di un problema di Keplero
	\[
		\ReducedMass \ddot{\vec{r}}
    = - \NewtonConstant \frac{\TotalMass\ReducedMass}{r^3}\vec{r},
	\]
	abbiamo
	\[
		\MeanAnomaly = \EccentricAnomaly
      - \LenzVector \sin{\EccentricAnomaly},
	\]
	dove
	\begin{itemize}
		\item $\MeanAnomaly$ \`e l'anomalia media;
		\item $\EccentricAnomaly$ \`e l'anomalia eccentrica;
		\item $\vec{\LenzVector}$ \`e il vettore di Lenz.
	\end{itemize}
\end{Theorem}
\Proof Con riferimento alla figura
\ref{ElementiDiMeccanicaCeleste_Figura_AnomaliaEccentricaEllittica},
denotiamo
\begin{itemize}
	\item $A_{HOA}$
    l'area del triangolo rettangolo individuato dai punti $H$, $O$, $A$;
	\item $A_{DCB}$
    l'area della porzione di circonferenza individuata dai raggi $HD$ e
    $HB$;
	\item $A_{HCQ}$
    l'area del triangolo rettangolo individuato dai punti $H$, $C$, $Q$;
	\item $A_{DHB}$
    l'area della porzione di circonferenza individuata dai segmenti $HD$
     e $HB$;
	\item $A_{DHA}$
    l'area della porzione di ellisse individuata dai segmenti $HD$ e
    $HA$;
	\item $A_{DOA}$
    l'area della porzione di ellisse individuata dai segmenti $OD$ e
    $OA$;
	\item $A$ l'area dell'ellisse.
\end{itemize}
\par Abbiamo
\begin{itemize}
	\item $A
    = \pi \SemiMajorAxis^2 \sqrt{1 - \LenzVector^2}$;
	\item $A_{HOA}
    = \frac{(\SemiMajorAxis \cos{\EccentricAnomaly} - \SemiMajorAxis \LenzVector) \cdot \sqrt{1 - \LenzVector^2} \SemiMajorAxis \sin{\EccentricAnomaly}}{2} = \SemiMajorAxis^2 \frac{(\cos{\EccentricAnomaly} - \LenzVector) \sqrt{1 - \LenzVector^2} \sin{\EccentricAnomaly}}{2}$;
	\item $A_{DCB}
    = \frac{\EccentricAnomaly \SemiMajorAxis^2}{2}$;
	\item $A_{HCQ}
    = \frac{\SemiMajorAxis^2 \cos{\EccentricAnomaly} \sin{\EccentricAnomaly}}{2}$;
	\item $A_{DHB}
    = A_{DCB} - A_{HCQ}$;
	\item $A_{DHA}
    = \sqrt{1 - \LenzVector^2} A_{DHB}$;
	\item $A_{DOA}
    = A_{DHA} + A_{HOA}$.
\end{itemize}
\par Ne deduciamo
\begin{align*}
	A_{DOA}
	&= A_{DHA} + A_{HOA},\\
	&= \sqrt{1 - \LenzVector^2} \left ( \frac{\EccentricAnomaly \SemiMajorAxis^2}{2} - \frac{\SemiMajorAxis^2 \cos{\EccentricAnomaly} \sin{\EccentricAnomaly}}{2} \right ) + \SemiMajorAxis^2 \frac{(\cos{\EccentricAnomaly} - \LenzVector) \sqrt{1 - \LenzVector^2} \sin{\EccentricAnomaly}}{2},\\
	&= \frac{\SemiMajorAxis^2\sqrt{1 - \LenzVector^2}}{2} ( \EccentricAnomaly - \cos{\EccentricAnomaly} \sin{\EccentricAnomaly} + \cos{\EccentricAnomaly} \sin{\EccentricAnomaly} - \LenzVector \sin{\EccentricAnomaly} ),\\
	&= \frac{\SemiMajorAxis^2\sqrt{1 - \LenzVector^2}}{2} ( \EccentricAnomaly - \LenzVector \sin{\EccentricAnomaly} ).
\end{align*}
Infine abbiamo, denotando $\MeanMotion$ il moto medio e tenendo conto della seconda legge di Keplero,
\begin{align*}
	\MeanAnomaly
	&= \MeanMotion (t - \PericenterPassageTime),\\
	&= \frac{2 \pi}{\Period} (t - \PericenterPassageTime),\\
	&= 2 \pi \frac{A_{DOA}}{A},\\
	&= 2 \pi \frac{\SemiMajorAxis^2\sqrt{1 - \LenzVector^2}}{2} ( \EccentricAnomaly - \LenzVector \sin{\EccentricAnomaly} ) \frac{1}{\pi \SemiMajorAxis^2 \sqrt{1 - \LenzVector^2}},\\
	&= \EccentricAnomaly - \LenzVector \sin{\EccentricAnomaly}.\text{ \EndProof}
\end{align*}
\begin{Corollary}
	Con le ipotesi e le notazioni del teorema precedente, abbiamo
	\[
		\PericenterPassageTime = \frac{\LenzVector \sin{\EccentricAnomaly(0)} - \EccentricAnomaly(0)}{\MeanMotion}
	\]
	dove
	\begin{itemize}
		\item $\PericenterPassageTime$ \`e il tempo di passaggio al pericentro;
		\item $\MeanMotion$ \`e il moto medio.
	\end{itemize}
\end{Corollary}
\Proof Da $\MeanAnomaly(0) = \MeanMotion(0 - \PericenterPassageTime)$,
deduciamo, tramite la formula di Keplero,
$\PericenterPassageTime
= - \frac{\MeanAnomaly(0)}{\MeanMotion}
= \frac{\LenzVector \sin{\EccentricAnomaly(0)} - \EccentricAnomaly(0)}{\MeanMotion}$.
\EndProof
\begin{Theorem}
	Dato un problema di Keplero
	\[
		\ReducedMass \ddot{\vec{r}}
    = - \NewtonConstant \frac{\TotalMass\ReducedMass}{r^3}\vec{r},
	\]
  se l'orbita \`e ellittica, allora fornire
  \begin{itemize}
    \item tempo di passaggio al pericentro $\PericenterPassageTime$;
    \item moto medio $\MeanMotion$;
    \item eccentricit\`a $\LenzVector$;
  \end{itemize}
  determina univocamente l'orbita.
\end{Theorem}
\Proof Da $\PericenterPassageTime$ e $\MeanMotion$ deduciamo
$\MeanAnomaly
= \MeanMotion(t - \PericenterPassageTime)$. Possiamo allora risolvere
l'equazione
\[
	\MeanAnomaly = \EccentricAnomaly
    - \LenzVector \sin{\EccentricAnomaly},
\]
data dalla formula di Keplero
\footnote{Concretamente, si usa un metodo numerico come per esempio il
metodo di Newton).}.
\par La terza legge di Keplero fornisce il semiasse maggiore
$\SemiMajorAxis
= \sqrt[3]{\frac{\NewtonConstant\TotalMass}{\MeanMotion^2}}$.
\par Si conclude tramite il teorema
\ref{ElementiDiMeccanicaCeleste_ThTrueEccentric}. \EndProof
\begin{Theorem}
	Data un'orbita periodica un problema di Keplero
	\[
		\ReducedMass \ddot{\vec{r}}
    = - \NewtonConstant \frac{\TotalMass\ReducedMass}{r^3}\vec{r},
	\]
  abbiamo
  \[
    \dot{\EccentricAnomaly} = \frac{\MeanMotion \SemiMajorAxis}{r},
  \]
  dove
	\begin{itemize}
		\item $\EccentricAnomaly$ \`e l'anomalia eccentrica;
		\item $\MeanAnomaly$ \`e l'anomalia media;
		\item $\SemiMajorAxis$ \`e il semiasse maggiore dell'orbita
      ellittica.
	\end{itemize}
\end{Theorem}
\Proof Denotiamo
\begin{itemize}
  \item $\SemilatusRectum$ il semilato retto;
  \item $\TrueAnomaly$ l'anomalia vera.
\end{itemize}
Da
\[
  \begin{cases}
    r = \frac{\SemilatusRectum}{1 + \LenzVector \cos{\TrueAnomaly}},\\
    r = \SemiMajorAxis(1 - \LenzVector \cos{\EccentricAnomaly}),
  \end{cases}
\]
deduciamo
\begin{align*}
  \frac{\SemilatusRectum}{1 + \LenzVector \cos{\TrueAnomaly}}
    &= \SemiMajorAxis(1 - \LenzVector \cos{\EccentricAnomaly}),\\
  \frac{\SemiMajorAxis(1 - \LenzVector^2)}
    {1 + \LenzVector \cos{\TrueAnomaly}}
    &= \SemiMajorAxis(1 - \LenzVector \cos{\EccentricAnomaly}),\\
  1 - \LenzVector^2
    &= (1 + \LenzVector \cos{\TrueAnomaly})
    (1 - \LenzVector\cos{\EccentricAnomaly}),\\
  1 - \LenzVector^2
    &= 1 + \LenzVector \cos{\TrueAnomaly}
    - \LenzVector \cos{\EccentricAnomaly}
    - \LenzVector^2 \cos{\EccentricAnomaly} \cos{\TrueAnomaly},\\
  0 &= \LenzVector \cos{\TrueAnomaly}
  - \LenzVector \cos{\EccentricAnomaly}
  - \LenzVector^2 \cos{\EccentricAnomaly} \cos{\TrueAnomaly}
  + \LenzVector^2.
\end{align*}
Deriviamo rispetto al tempo:
\begin{align*}
  - \LenzVector \sin{\TrueAnomaly} \dot{\TrueAnomaly}
    + \LenzVector \sin{\EccentricAnomaly} \dot{\EccentricAnomaly}
    + \LenzVector^2 \sin{\EccentricAnomaly} \cos{\TrueAnomaly}
      \dot{\EccentricAnomaly}
    + \LenzVector^2 \sin{\TrueAnomaly} \cos{\EccentricAnomaly}
      \dot{\TrueAnomaly}
    &= 0,\\
  (\LenzVector^2 \sin{\TrueAnomaly} \cos{\EccentricAnomaly}
    - \LenzVector \sin{\TrueAnomaly}) \dot{\TrueAnomaly}
    + (\LenzVector^2 \sin{\EccentricAnomaly} \cos{\TrueAnomaly}
    + \LenzVector \sin{\EccentricAnomaly}) \dot{\EccentricAnomaly}
    &= 0,
\end{align*}
da cui, denotando $\AngularMomentPerUnitMass$ il momento angolare per
unit\`a di massa,
\begin{align*}
  \dot{\EccentricAnomaly} &=
    \frac{\LenzVector \sin{\TrueAnomaly}
      - \LenzVector^2 \sin{\TrueAnomaly} \cos{\EccentricAnomaly}}
    {\LenzVector^2 \sin{\EccentricAnomaly} \cos{\TrueAnomaly}
      + \LenzVector \sin{\EccentricAnomaly}} \dot{\TrueAnomaly},\\
  &=
    \frac{\sin{\TrueAnomaly} (1 - \LenzVector \cos{\EccentricAnomaly})}
      {\sin{\EccentricAnomaly} (\LenzVector \cos{\TrueAnomaly} + 1)}
    \frac{\AngularMomentPerUnitMass}{r^2},\\
  &=
    \frac{\sin{\EccentricAnomaly} \sqrt{1 - \LenzVector^2}}
      {1 - \LenzVector \cos{\EccentricAnomaly}}
    \frac{(1 - \LenzVector \cos{\EccentricAnomaly})}
      {\sin{\EccentricAnomaly} (\LenzVector \cos{\TrueAnomaly} + 1)}
    \frac{\AngularMomentPerUnitMass}{r^2},\\
  &=
    \sqrt{1 - \LenzVector^2}
    \frac{\AngularMomentPerUnitMass}
    {(\LenzVector \cos{\TrueAnomaly} + 1) r^2},\\
  &=
    \sqrt{1 - \LenzVector^2}
    \frac{\sqrt{\NewtonConstant \TotalMass}}{r^2}
    \frac{\sqrt{\SemilatusRectum}}
      {1 + \LenzVector \cos{\TrueAnomaly}},\\
  &=
    \sqrt{1 - \LenzVector^2}
    \frac{\sqrt{\NewtonConstant \TotalMass}}{r}
    \frac{1}{\sqrt{\SemilatusRectum}},\\
  &=
    \sqrt{1 - \LenzVector^2}
    \MeanMotion \SemiMajorAxis^{\frac{3}{2}}
    \frac{1}{r}
    \frac{1}{\sqrt{\SemiMajorAxis}\sqrt{1 - \LenzVector^2}},\\
  &=
    \frac{\MeanMotion \SemiMajorAxis}{r}.\text{ \EndProof}
\end{align*}
\begin{Theorem}
	Dato un problema di Keplero
	\[
		\ReducedMass \ddot{\vec{r}}
    = - \NewtonConstant \frac{\TotalMass\ReducedMass}{r^3}\vec{r},
	\]
  se l'orbita \`e ellittica, allora, denotati
  \begin{itemize}
    \item $\MeanMotion$ l'anomalia media;
    \item $\EnergyPerUnitMass$ l'integrale dell'energia per unit\`a di
      massa;
    \item $\vec{\LenzVector}$ il vettore di Lenz;
  \end{itemize}
  abbiamo
  \begin{itemize}
    \item $\MeanMotion = \frac{\NewtonConstant^2\TotalMass^2}
    {2 \AbsoluteValue{\EnergyPerUnitMass}\sqrt{- 2 \EnergyPerUnitMass}}$;
    \item $\PericenterPassageTime$ pu\`o essere calcolato direttamente
      noto $\EccentricAnomaly(0)$ soluzione del sistema
      \[
        \begin{cases}
          \cos{\EccentricAnomaly(0)} = \frac{1}{\LenzVector}
            \left ( 1 - \frac{r(0)}{\SemiMajorAxis} \right ),\\
          \sin{\EccentricAnomaly(0)} = \frac{r(0)\dot{r}(0)}
            {\SemiMajorAxis^2 \LenzVector \MeanMotion}.
        \end{cases}
      \]
  \end{itemize}
\end{Theorem}
\Proof Per il moto medio abbiamo
\begin{align*}
  \MeanMotion
  &= \sqrt{\frac{\NewtonConstant\TotalMass}{\SemiMajorAxis^3}},\\
  &= \sqrt
    {- \frac{\NewtonConstant^4\TotalMass^4}{8 \EnergyPerUnitMass^3}},\\
  &= \frac{\NewtonConstant^2\TotalMass^2}
    {2 \AbsoluteValue{\EnergyPerUnitMass}\sqrt{- 2 \EnergyPerUnitMass}}.
\end{align*}
\par Da $r = \SemiMajorAxis(1 - \LenzVector \cos{\EccentricAnomaly(0)})$
deduciamo immediatamente
$\cos{\EccentricAnomaly(0)} = \frac{1}{\LenzVector}
\left ( 1 - \frac{r(0)}{\SemiMajorAxis} \right )$.
\par Abbiamo inoltre
 $\dot{r}
= \SemiMajorAxis \LenzVector \sin{(\EccentricAnomaly)}
\dot{\EccentricAnomaly}$, da cui
\begin{align*}
  \sin{\EccentricAnomaly(0)}
  &=\frac{\dot{r}(0)}{\SemiMajorAxis\LenzVector\dot{\EccentricAnomaly}},
  &=\frac{r(0) \dot{r}(0)}{\SemiMajorAxis^2\LenzVector\MeanMotion}.
\end{align*}
\par Possiamo allora calcolare $\EccentricAnomaly(0)$ e ottenere
$\PericenterPassageTime
= \frac{\LenzVector \sin{\EccentricAnomaly(0)} - \EccentricAnomaly(0)}
{\MeanMotion}$. \EndProof
\par Consideriamo adesso il problema di Keplero in un sistema di
riferimento in il campo gravitazionale \`e sempre centrale, ma il piano
orbitale non coincide pi\`u col piano $Oxy$.
\begin{Definition}
  Dato un problema di Keplero
	\[
		\ReducedMass \ddot{\vec{r}}
    = - \NewtonConstant \frac{\TotalMass\ReducedMass}{r^3}\vec{r},
	\]
  chiamiamo
  \begin{itemize}
    \item \Define{linea dei nodi}[dei nodi][linea] la traccia del piano
      orbitale sul piano $Oxy$;
    \item \Define{nodo ascendente}[ascendente][nodo] il punto
      d'intersezione tra l'orbita e la linea dei nodi dove il punto
      materiale ha velocit\`a positiva lungo l'asse $z$, denotato
      tipicamente $\AscendingNode$;
    \item \Define{nodo discendente}[discendente][nodo] il punto
      d'intersezione tra l'orbita e la linea dei nodi dove il punto
      materiale ha velocit\`a negativa lungo l'asse $z$, denotato
      tipicamente $\DescendingNode$;
    \item \Define{longitudine del nodo ascendente}[del nodo ascendente][longitudine]
      l'angolo $\LongitudeOfAscendingNode$ individuato dalla
      direzione positiva dell'asse $x$ al nodo ascendente misurato in
      senso positivo;
    \item \Define{argomento del pericentro}[del pericentro][argomento]
      l'angolo $\ArgumentOfPericenter$ individuato dal nodo ascendente
      al vettore di Lenz misurato in senso positivo;
    \item \Define{inclinazione} l'angolo $\Inclination$ individuato
      dall'asse $z$ al momento angolare misurato in senso positivo.
  \end{itemize}
  Chiamiamo inoltre
  \begin{itemize}
    \item \Define{elementi cartesiani}[cartesiani][elementi] le sei
    componenti dei vettori $\vec{r}$ e $\dot{\vec{r}}$;
    \item \Define{elementi kepleriani}[kepleriani][elementi]
      le sei quantit\`a scalari
      \begin{itemize}
        \item semiasse maggiore $\SemiMajorAxis$
          (si considera $\SemiMajorAxis = + \infty$ nel
          caso di orbita parabolica);
        \item eccentricit\`a $\LenzVector$;
        \item inclinazione $\Inclination$;
        \item la longitudine del nodo ascendente
          $\LongitudeOfAscendingNode$;
        \item l'argomento del pericentro $\ArgumentOfPericenter$;
        \item l'anomalia media $\MeanAnomaly$.
      \end{itemize}
  \end{itemize}
\end{Definition}
\begin{figure}
	\includegraphics[width=0.8\textwidth]{Elementi_di_meccanica_celeste/Immagine_Keplero_ElementiKepleriani.png}
	\centering
	\caption{Elementi kepleriani.}
\end{figure}
\begin{Theorem}
  Dato un problema di Keplero
	\[
		\ReducedMass \ddot{\vec{r}}
    = - \NewtonConstant \frac{\TotalMass\ReducedMass}{r^3}\vec{r},
	\]
  la corrispondenza che associa, per un fissato istante $t$, agli
  elementi cartesiani gli elementi kepleriani \`e biunivoca.
\end{Theorem}
\Proof Manteniamo le notazioni della definizione precedente.
Supponiamo noti gli elementi cartesiani.
Abbiamo
\begin{itemize}
  \item l'inclinazione \`e l'angolo compreso tra $0$ e $\pi$ tra il
    vettore
    $\vec{\AngularMomentPerUnitMass}
      = \VectorProduct{\vec{r}}{\vec{\dot{r}}}$ e il versore
    $\ver{e_z}$: \`e soluzione dell'equazione
    $\cos{\Inclination}
      =\frac{\ScalarProduct{\vec{\AngularMomentPerUnitMass}}{\ver{e_z}}}
        {\AngularMomentPerUnitMass}$;
  \item l'eccentrit\`a \`e data dal modulo del vettore di Lenz
    $\vec{\LenzVector}
      = \frac{1}
        {\NewtonConstant\TotalMass}
        \VectorProduct{\vec{r}}{\AngularMomentPerUnitMass}
        - \frac{\vec{r}}{r}$;
  \item la linea dei nodi \`e l'intersezione del piano $Oxy$ e del piano
    orbitale: il versore
    $\ver{\LongitudeOfAscendingNode}
      = \frac{\VectorProduct{\ver{e_z}}{\AngularMomentPerUnitMass}}
        {\Norm{\VectorProduct{\ver{e_z}}{\AngularMomentPerUnitMass}}}$
    giace sulla linea dei nodi: da esso, tramite prodotto scalare con
    $\ver{e_x}$, ricavo $\LongitudeOfAscendingNode$;
  \item l'argomento del pericentro \`e l'angolo compreso tra il versore
    $\ver{\LongitudeOfAscendingNode}$ e il vettore di Lenz: \`e dunque
    soluzione di
    \[
      \begin{cases}
        \cos{\LongitudeOfAscendingNode}
          = \ScalarProduct{\vec{\LenzVector}}
            {\ver{\LongitudeOfAscendingNode}},\\
        \sin{\LongitudeOfAscendingNode}
          = \Norm{\VectorProduct{\vec{\LenzVector}}
            {\ver{\LongitudeOfAscendingNode}}}.
      \end{cases}
    \]
  \item l'energia per unit\`a di massa \`e
    $\EnergyPerUnitMass
      = \frac{\ScalarProduct{\dot{r}}{\dot{r}}}{2}
        - \frac{\NewtonConstant\TotalMass}{r}$. Ora, se
    $\EnergyPerUnitMass = 0$, abbiamo visto che l'orbita \`e parabolica
    e abbiamo allora $\SemiMajorAxis = + \infty$; altrimenti abbiamo
    $\SemiMajorAxis
      = - \frac{\NewtonConstant\TotalMass}{2\EnergyPerUnitMass}$;
  \item la terza legge di Keplero fornisce il moto medio
    $\MeanMotion
      = \sqrt{\frac{\NewtonConstant\TotalMass}
        {\SemiMajorAxis^3}}$,
    da cui calcoliamo l'anomalia eccentrica come soluzione di
    \[
      \begin{cases}
        \cos{\EccentricAnomaly} = \frac{1}{\LenzVector}
          \left ( 1 - \frac{r}{\SemiMajorAxis} \right ),\\
        \sin{\EccentricAnomaly} = \frac{r\dot{r}}
          {\SemiMajorAxis^2 \LenzVector \MeanMotion}.
      \end{cases}
    \];
    ne deduciamo l'anomalia media tramite formula di Keplero.
    SOLO CASO ELLITTICO PER ORA
\end{itemize}
\par Supponiamo ora invece noti gli elementi kepleriani. Abbiamo gi\`a
visto come fornire il moto medio $\MeanMotion$, il tempo di passaggio al
pericentro $\PericenterPassageTime$ e l'eccentricit\`a determini
univocamente l'orbita nel caso in cui il piano $Oxy$ coincida col piano
orbitale.
\par Avendo gli elementi kepleriani abbiamo effettivamente anche
\begin{itemize}
  \item $\MeanMotion
          = \sqrt{\frac{\NewtonConstant\TotalMass}
            {\SemiMajorAxis^3}}$;
  \item l'anomalia eccentrica soluzione di
    \[
      \begin{cases}
        \cos{\EccentricAnomaly} = \frac{1}{\LenzVector}
          \left ( 1 - \frac{r}{\SemiMajorAxis} \right ),\\
        \sin{\EccentricAnomaly} = \frac{r\dot{r}}
          {\SemiMajorAxis^2 \LenzVector \MeanMotion}.
      \end{cases}
    \];
  \item il tempo di passaggio al pericentro
    $\PericenterPassageTime
      = \frac{\LenzVector \sin{\EccentricAnomaly(0)}
          - \EccentricAnomaly(0)}{\MeanMotion}$.
\end{itemize}
\par Possiamo dunque calcolare i vettori $\vec{r}$ e $\dot{\vec{r}}$ in
un tale sistema di riferimento. Per ottenere le loro coordinate nel
sistema di riferimento originale occorrer\`a dunque eseguire la
trasformazione
$R
  = R_{\LongitudeOfAscendingNode}
    R_{\Inclination}
    R_{\ArgumentOfPericenter}$,
dove $R_{\LongitudeOfAscendingNode}$, $R_{\Inclination}$,
$R_{\ArgumentOfPericenter}$, sono le rotazioni
\begin{itemize}
  \item $R_{\LongitudeOfAscendingNode} =
        \lmatrix
        \begin{array}{ccc}
          \cos{\LongitudeOfAscendingNode} &
          - \sin{\LongitudeOfAscendingNode} &
          0\\
          \sin{\LongitudeOfAscendingNode} &
          \cos{\LongitudeOfAscendingNode} &
          0\\
          0 &
          0 &
          1
        \end{array}
        \rmatrix$;
  \item $R_{\Inclination}
        \lmatrix
        \begin{array}{ccc}
          1 &
          0 &
          0\\
          \cos{\Inclination} &
          - \sin{\Inclination} &
          0\\
          \sin{\Inclination} &
          \cos{\Inclination} &
          0\\
        \end{array}
        \rmatrix$;
  \item $R_{\ArgumentOfPericenter},
        \lmatrix
        \begin{array}{ccc}
          \cos{\ArgumentOfPericenter} &
          - \sin{\ArgumentOfPericenter} &
          0\\
          \sin{\ArgumentOfPericenter} &
          \cos{\ArgumentOfPericenter} &
          0\\
          0 &
          0 &
          1
        \end{array}
        \rmatrix$. \EndProof
\end{itemize}
\begin{Theorem}
  \TheoremName{Equazione di Baker}[di Baker][equazione]
  Dato un problema di Keplero
	\[
		\ReducedMass \ddot{\vec{r}}
    = - \NewtonConstant \frac{\TotalMass\ReducedMass}{r^3}\vec{r},
	\]
  se l'orbita \`e iperbolica allora
  \[ 
    t - t_0
    = \frac{1}{2}
    \sqrt{\frac{\SemilatusRectum^3}{\NewtonConstant\TotalMass}}
    \left [
    \tan \frac{\TrueAnomaly}{2}
    + \frac{1}{3} \tan^3 \frac{\TrueAnomaly}{3}
    \right ]_{\TrueAnomaly_0}^{\TrueAnomaly}.
  \]
  dove
  \begin{itemize}
    \item $t_0$ \`e un tempo iniziale fissato;
    \item $\TrueAnomaly_0$ \`e l'anomalia vera al tempo $t_0$;
    \item $\TrueAnomaly$ l'anomalia vera al tempo $t$;
    \item $\SemilatusRectum$ il semilato retto.
  \end{itemize}
\end{Theorem}
\Proof Dall'equazione
$\AngularMomentPerUnitMass = r^2 \dot{\TrueAnomaly}$
deduciamo
$\int_{t_0}^t \AngularMomentPerUnitMass d\tau
= \int_{\TrueAnomaly_0}^\TrueAnomaly r^2 d\TrueAnomaly$,
dove $\TrueAnomaly_0$ \`e l'anomalia vera al tempo $t_0$:
$\TrueAnomaly_0 = \TrueAnomaly(t_0)$.
\par Ora, abbiamo
\begin{align*}
  r
  &= \frac{\SemilatusRectum}{1 + \cos \TrueAnomaly},\\
  &= \frac{\SemilatusRectum}{2}
      \frac{1}{\cos^2 \frac{\TrueAnomaly}{2}},\\
  &= \frac{\SemilatusRectum}{2}
      \frac{\cos^2\frac{\TrueAnomaly}{2} + \sin^2\frac{\TrueAnomaly}{2}}
        {\cos^2 \frac{\TrueAnomaly}{2}},\\
  &= \frac{\SemilatusRectum}{2}
      \left ( 1 + \tan^2 \frac{\TrueAnomaly}{2} \right );
\end{align*}
inoltre, posto $x = \tan \frac{\TrueAnomaly}{2}$, abbiamo
\begin{align*}
  dx
    &= \frac{1}{2}
      \left (1 + \tan^2\frac{\TrueAnomaly}{2} \right ) d\TrueAnomaly,\\
  d\TrueAnomaly = \frac{2}{1 + x^2} dx.
\end{align*}
Dunque
\begin{align*}
t - t_0
&= \frac{1}{\AngularMomentPerUnitMass}
    \int_{t_0}^t \AngularMomentPerUnitMass d\tau,\\
&= \frac{1}{\sqrt{\NewtonConstant\TotalMass\SemilatusRectum}}
    \int_{\TrueAnomaly_0}^\TrueAnomaly r^2 d\TrueAnomaly,\\
&= \frac{\SemilatusRectum^2}
    {4\sqrt{\NewtonConstant\TotalMass\SemilatusRectum}}
    \int_{\TrueAnomaly_0}^\TrueAnomaly
    \left (
    1 + \tan^2 \frac{\TrueAnomaly}{2}
    \right )^2
    d\TrueAnomaly,\\
&= \frac{1}{4}
    \sqrt{\frac{\SemilatusRectum^3}{\NewtonConstant\TotalMass}}
    \int_{x(\TrueAnomaly_0)}^{x(\TrueAnomaly)}
    \frac{(1 + x^2)^2}
    {x^2 + 1}
    dx,\\
&= \frac{1}{4}
    \sqrt{\frac{\SemilatusRectum^3}{\NewtonConstant\TotalMass}}
    \int_{x(\TrueAnomaly_0)}^{x(\TrueAnomaly)}
    2 (1 + x^2)
    dx,\\
&= \frac{1}{2}
    \sqrt{\frac{\SemilatusRectum^3}{\NewtonConstant\TotalMass}}
    \left [
    x + \frac{x^3}{3}
    \right ]_{x(\TrueAnomaly_0)}^{x(\TrueAnomaly)},\\
&= \frac{1}{2}
    \sqrt{\frac{\SemilatusRectum^3}{\NewtonConstant\TotalMass}}
    \left [
    \tan \frac{\TrueAnomaly}{2}
    + \frac{1}{3} \tan^3 \frac{\TrueAnomaly}{3}
    \right ]_{\TrueAnomaly_0}^{\TrueAnomaly}.
\end{align*}
\begin{Definition}
  Dato un problema di Keplero
	\[
		\ReducedMass \ddot{\vec{r}}
    = - \NewtonConstant \frac{\TotalMass\ReducedMass}{r^3}\vec{r},
	\]
  chiamiamo
  \Define{tempo di volo}[di volo][tempo]
  la quantit\`a $t - \PericenterPassageTime$, dove
  $\PericenterPassageTime$ \`e il tempo di passaggio al pericentro.
\end{Definition}
\begin{Theorem}
  Dato un problema di Keplero
	\[
		\ReducedMass \ddot{\vec{r}}
    = - \NewtonConstant \frac{\TotalMass\ReducedMass}{r^3}\vec{r},
	\]
  il tempo di volo \`e
  \begin{itemize}
    \item $t - \PericenterPassageTime
            = \sqrt{\frac{\SemiMajorAxis^3}{\NewtonConstant\TotalMass}}
              (\EccentricAnomaly - \LenzVector \sin \EccentricAnomaly)$
          se l'orbita \`e ellittica;
    \item $t - \PericenterPassageTime
            = \frac{1}{2}
              \sqrt{\frac{\SemilatusRectum^3}{\NewtonConstant\TotalMass}}
              \left (
              \tan \frac{\TrueAnomaly}{2}
              + \frac{1}{3} \tan^3 \frac{\TrueAnomaly}{2}
              \right )$
          se l'orbita \`e parabolica;
    \item $t - \PericenterPassageTime
            = \sqrt{-\frac{\SemiMajorAxis^3}{\NewtonConstant\TotalMass}}
              (\LenzVector \sinh \EccentricAnomaly - \EccentricAnomaly)$
          se l'orbita \`e iperbolica.
  \end{itemize}
\end{Theorem}
\Proof Supponiamo l'orbita ellittica. Abbiamo
\[
  \MeanAnomaly
    = \EccentricAnomaly - \LenzVector \sin \EccentricAnomaly,\\
  \MeanAnomaly
    = \MeanMotion (t - \PericenterPassageTime).
\]
Da cui, tramite la terza legge di Keplero,
\begin{align*}
  t - \PericenterPassageTime
    &= \frac{1}{\MeanMotion} 
      \EccentricAnomaly - \LenzVector \sin \EccentricAnomaly,\\
    &= \sqrt{\frac{\SemiMajorAxis^3}{\NewtonConstant\TotalMass}}
      \EccentricAnomaly - \LenzVector \sin \EccentricAnomaly.
\end{align*}
\par Supponiamo l'orbita parabolica. Segue direttamente dall'equazione
di Baker che il tempo di volo \`e
$t - \PericenterPassageTime
= \frac{1}{2}
\sqrt{\frac{\SemilatusRectum^3}{\NewtonConstant\TotalMass}}
\left (
\tan \frac{\TrueAnomaly}{2}
+ \frac{1}{3} \tan^3 \frac{\TrueAnomaly}{2}
\right )$.
