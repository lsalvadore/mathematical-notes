\section{Problema di due corpi.}
\label{ElementiDiMeccanicaCeleste_ProblemaDiDueCorpi}
\begin{figure}
	\includegraphics[width=0.8\textwidth]{Elementi_di_meccanica_celeste/Immagine_2Corpi.png}
	\centering
	\caption{Problema di due corpi.}
\end{figure}
\par Assumiamo nel seguito di aver fissato un sistema di riferimento inerziale.
\begin{Definition}
	Dato un sistema isolato di due punti materiali
	\begin{itemize}
		\item $\PointMass_1$ di $\Mass_1$ in posizione $\vec{\rho_1}$;
		\item $\PointMass_2$ di $\Mass_2$ in posizione $\vec{\rho_2}$;
	\end{itemize}
	chiamiamo
	\begin{itemize}
		\item \Define{massa ridotta}[ridotta][massa], denotata $\ReducedMass$, la quantit\`a $\ReducedMass = \Mass_1^{-1} + \Mass_2^{-1} = \frac{\Mass_1\Mass_2}{\Mass_1 + \Mass_2}$;
		\item \Define{problema ridotto}[ridotto][problema] o \Define{problema di Keplero}[di Keplero][problema] il problema del moto di un punto materiale di massa $\ReducedMass$ che si muove in un campo gravitazionale generato da una massa $\TotalMass = \Mass_1 + \Mass_2$ vincolata nell'origine.
	\end{itemize}
\end{Definition}
\begin{Theorem}
	Dato un sistema isolato di due punti materiali
	\begin{itemize}
		\item $\PointMass_1$ di $\Mass_1$ in posizione $\vec{\rho_1}$;
		\item $\PointMass_2$ di $\Mass_2$ in posizione $\vec{\rho_2}$;
	\end{itemize}
	sono costanti 
	\begin{itemize}
		\item l'energia $\Energy$ del sistema;
		\item le componenti del momento angolare $\vec{\AngularMoment}$;
		\item le componenti della quantit\`a di moto $\vec{\Momentum}$.
	\end{itemize}
\end{Theorem}
\Proof Abbiamo $\Energy = \frac{1}{2}\Mass_1\dot{\rho_1}^2 + \frac{1}{2}\Mass_2\dot{\rho_2}^2 - \NewtonConstant \frac{\Mass_1\Mass_2}{r}$, da cui
\begin{align*}
	\dot{\Energy}
	&= \Mass_1\ddot{\rho_1}\dot{\rho_1} + \Mass_2\ddot{\rho_2}\dot{\rho_2} + \NewtonConstant \frac{\Mass_1\Mass_2}{r^2} \dot{r},\\
	&= - \NewtonConstant \frac{\Mass_1\Mass_2}{r^2}\dot{r} + \NewtonConstant \frac{\Mass_1\Mass_2}{r^2} \dot{r} = 0.
\end{align*}
(La stessa cosa poteva essere dedotta direttamente dal primo principio della termodinamica.)
\par Per il momento angolare, osserviamo che la forza gravitazionale \`e centrale, dunque il momento angolare dei singoli punti materiali \`e costante e quindi costante \`e anche il momento angolare $\vec{\AngularMoment}$ del sistema.
\par Per la quantit\`a di moto, abbiamo, poich\'e il sistema \`e isolato, $(\Mass_1 + \Mass_2) \ddot{\vec{\CenterOfMass}} = 0$, dove $\CenterOfMass = \frac{\Mass_1\vec{\rho_1} + \Mass_2\vec{\rho_2}}{\Mass_1 + \Mass_2}$; quindi $\vec{\Momentum} = (\Mass_1 + \Mass_2) \dot{\vec{\CenterOfMass}}$ \`e costante. \EndProof
\begin{Corollary}
	Il problema di due corpi \`e integrabile.
\end{Corollary}
\Proof Il problema di due corpi \`e un problema con $6$ gradi di libert\`a e $6$ integrali primi indipendenti\footnote{Il momento angolare costituisce in realt\`a solo due integrali primi perch\'e non si tratta propriamente di un vettore, bens\'i di uno pseudovettore.}. \EndProof
\begin{Theorem}
	Dato un sistema isolato di due punti materiali
	\begin{itemize}
		\item $\PointMass_1$ di $\Mass_1$ in posizione $\vec{\rho_1}$;
		\item $\PointMass_2$ di $\Mass_2$ in posizione $\vec{\rho_2}$;
	\end{itemize}
	il loro centro di massa $\vec{\CenterOfMass}$ si muove di moto rettilinero uniforme.
\end{Theorem}
\Proof La quantit\`a di moto del sistema $\vec{\Momentum}$ \`e costante. \`E dunque costante anche $\dot{\vec{\CenterOfMass}} = \frac{\vec{\Momentum}}{\Mass_1 + \Mass_2}$. \EndProof
\begin{Corollary}
	Con le notazioni del teorema precedente, il sistema di riferimento ottenuto per traslazione dell'origine nel centro di massa $\vec{\CenterOfMass}$ \`e anch'esso inerziale.
\end{Corollary}
\Proof Segue direttamente dal teorema precedente.
\begin{Theorem}
	\label{ElementiDiMeccanicaCeleste_RiduzioneAlProblemaDiKeplero}
	Consideriamo un sistema isolato di due punti materiali riferito ad un sistema inerziale centrato nel loro centro di massa
	\begin{itemize}
		\item $\PointMass_1$ di $\Mass_1$ in posizione $\vec{r_1}$;
		\item $\PointMass_2$ di $\Mass_2$ in posizione $\vec{r_2}$;
	\end{itemize}
	il problema di due corpi
	\[
		\begin{cases}
			\Mass_1 \ddot{\vec{r_1}} = \NewtonConstant \frac{\Mass_1\Mass_2}{r^3}\vec{r},\\
			\Mass_2 \ddot{\vec{r_2}} = - \NewtonConstant \frac{\Mass_1\Mass_2}{r^3}\vec{r},
		\end{cases}
	\]
	posto,
	\begin{itemize}
		\item $\ReducedMass = \Mass_1^{-1} + \Mass_2^{-1}$;
		\item $\TotalMass = \Mass_1 + \Mass_2$;
		\item $\vec{r} = \vec{r_2} - \vec{r_1}$.
	\end{itemize}
	equivale al problema ridotto
	\[
		\ReducedMass \ddot{\vec{r}} = - \NewtonConstant \frac{\TotalMass\ReducedMass}{r^3}\vec{r},
	\]
	tramite la trasformazione lineare invertibile $\vec{x} \mapsto - \frac{\TotalMass}{\Mass_2} \vec{x}$ dallo spazio delle soluzioni di $\vec{r_1}$ nello spazio delle soluzioni di $\vec{r}$.
\end{Theorem}
\Proof Supponiamo $\vec{r_1}$ sia soluzione del problema a due corpi per il punto materiale $\PointMass_1$.
\par Abbiamo, a seguito della trasformazione,
\begin{align*}
	\ReducedMass \ddot{\vec{r}}
	&= - \frac{\TotalMass}{\Mass_1\Mass_2} \frac{\TotalMass}{\Mass_2} \NewtonConstant \frac{\Mass_2}{r^3}\vec{r},\\
	&= - \NewtonConstant \frac{\TotalMass\ReducedMass}{r^3}\vec{r}.
\end{align*}
\par D'altra parte, supponiamo $\vec{r}$ sia soluzione del problema di Keplero.
\par Abbiamo, a seguito della trasformazione,
\begin{align*}
	\Mass_1 \ddot{\vec{r_1}}
	&= \Mass_1 \frac{\Mass_2}{\TotalMass} \NewtonConstant \frac{\TotalMass}{r^3}\vec{r},\\
	&= \NewtonConstant \frac{\Mass_1\Mass_2}{r^3}\vec{r};
\end{align*}
e inoltre
\begin{align*}
	\Mass_2 \ddot{\vec{r_2}}
	&= \Mass_2 (\ddot{\vec{r}} + \ddot{\vec{r_1}}),\\
	&= - \Mass_2 \NewtonConstant \frac{\TotalMass}{r^3}\vec{r} + \Mass_2 \NewtonConstant \frac{\Mass_2}{r^3}{r},\\
	&= \NewtonConstant \frac{\Mass_2(\Mass_2 - \TotalMass)}{r^3}\vec{r},\\
	&= - \NewtonConstant \frac{\Mass_1\Mass_2}{r^3}\vec{r}.\text{ \EndProof}
\end{align*}
\begin{Corollary}
	\TheoremName{Prima legge di Keplero}[prima di Keplero][legge] La Terra si muove attorno al Sole secondo un'orbita ellittica di cui il Sole occupa uno dei due fuochi.
\end{Corollary}
\Proof Per la grande disparit\`a di massa tra il Sole e la Terra, il centro di massa dei due corpi pu\`o essere posto con ottima approssimazione nel centro del Sole: ne risulta che il problema di due corpi composto da Sole e Terra \`e con ottima approssimazione identico al suo problema di Keplero equivalente. L'esperienza diretta dimostra che l'orbita \`e periodica e dunque ellittica (non circolare). \EndProof
\begin{Theorem}
	Consideriamo un sistema isolato di due punti materiali riferito ad un
  sistema inerziale centrato nel loro centro di massa
	\begin{itemize}
		\item $\PointMass_1$ di $\Mass_1$ in posizione $\vec{r_1}$;
		\item $\PointMass_2$ di $\Mass_2$ in posizione $\vec{r_2}$.
	\end{itemize}
	Consideriamo il problema di due corpi
	\[
		\begin{cases}
			\Mass_1 \ddot{\vec{r_1}}
        = \NewtonConstant \frac{\Mass_1\Mass_2}{r^3}\vec{r},\\
			\Mass_2 \ddot{\vec{r_2}}
        = - \NewtonConstant \frac{\Mass_1\Mass_2}{r^3}\vec{r},
		\end{cases}
	\]
	equivalente al problema di Keplero
	\[
		\ReducedMass \ddot{r}
      = - \NewtonConstant \frac{\TotalMass\ReducedMass}{r^3}\vec{r},
	\]
	con
	\begin{itemize}
		\item $\ReducedMass = \Mass_1^{-1} + \Mass_2^{-1}$;
		\item $\TotalMass = \Mass_1 + \Mass_2$;
		\item $\vec{r} = \vec{r_2} - \vec{r_1}$;
	\end{itemize}
	tramite la trasformazione lineare invertibile
$\vec{x} \mapsto - \frac{\TotalMass}{\Mass_2} \vec{x}$ dallo spazio
delle soluzioni di $\vec{r_1}$ nello spazio delle soluzioni di
$\vec{r}$.
	Le orbite del problema di due corpi sono coniche
	\begin{itemize}
		\item di eccentricit\`a $\LenzVector$, dove $\vec{\LenzVector}$ \`e
      il vettore di Lenz del problema ridotto ;
		\item semilati retti $\SemilatusRectum_1 =
      \frac{\Mass_2}{\TotalMass} \SemilatusRectum$ e
      $\SemilatusRectum_2 = \frac{\Mass_1}{\TotalMass} \SemilatusRectum$
      per le orbite di $\PointMass_1$ e $\PointMass_2$ rispettivamente,
      dove $\SemilatusRectum$ \`e il semilato dell'orbita del problema
      ridotto;
		\item i semilati retti $\SemilatusRectum_1$ e $\SemilatusRectum_2$
      sono tali che $\SemilatusRectum_1 + \SemilatusRectum_2
      = \SemilatusRectum$;
		\item semiassi maggiori $\SemiMajorAxis_1$ e $\SemiMajorAxis_2$ tali
      che $\SemiMajorAxis_1 + \SemiMajorAxis_2 = \SemiMajorAxis$, dove
      $\SemiMajorAxis$ il semiassemaggiore dell'orbita del problema
      ridotto;
		\item $\PointMass_1$ e $\PointMass_2$ si trovano allineati col
      centro di massa e situati da parti opposte rispetto ad esso;
		\item le orbite di $\PointMass_1$ e $\PointMass_2$ vengono percorse
      con velocit\`a allineate e opposte e tali che
      $\dot{r_1} + \dot{r_2} = \dot{r}$.
	\end{itemize}
\end{Theorem}
\Proof Abbiamo
\begin{align*}
	\vec{r_1} = - \frac{\Mass_2}{\TotalMass} \vec{r},\\
	\vec{r_2} = \frac{\Mass_1}{\TotalMass} \vec{r},
\end{align*}
da cui, passando ai moduli e utilizzando la soluzione del problema di Keplero,
\begin{align*}
	r_1(\TrueAnomaly) = \frac{\Mass_2}{\TotalMass} \frac{\SemilatusRectum}{1 + \LenzVector \cos{\TrueAnomaly}},\\
	r_2(\TrueAnomaly) = \frac{\Mass_1}{\TotalMass} \frac{\SemilatusRectum}{1 + \LenzVector \cos{\TrueAnomaly}}.
\end{align*}
\par Abbiamo dunque
\begin{align*}
	\SemilatusRectum_1 = \frac{\Mass_2}{\TotalMass} \SemilatusRectum,\\
	\SemilatusRectum_2 = \frac{\Mass_1}{\TotalMass} \SemilatusRectum,
\end{align*}
da cui,
\begin{align*}
	\SemiMajorAxis_1 = \frac{\SemilatusRectum_1}{1 - \LenzVector^2},\\
	\SemiMajorAxis_2 = \frac{\SemilatusRectum_2}{1 - \LenzVector^2}.
\end{align*}
\par Abbiamo inoltre
\begin{itemize}
	\item $\SemilatusRectum_1 + \SemilatusRectum_2 = \frac{\Mass_2}{\TotalMass} \SemilatusRectum + \frac{\Mass_1}{\TotalMass} \SemilatusRectum = \SemilatusRectum$;
	\item $\SemiMajorAxis_1 + \SemiMajorAxis_2 = \frac{\SemilatusRectum_1 + \SemilatusRectum_2}{1 - \LenzVector^2} = \frac{\SemilatusRectum}{1 -\LenzVector^2} = \SemiMajorAxis$.
\end{itemize}
\par $\PointMass_1$ e $\PointMass_2$ sono allineati con il loro centro di massa e situati da parti opposte rispetto ad essi per la definizione stessa di centro di massa.
\par Infine, per valutare i versi di percorrenza, deriviamo $\vec{r_1}$ e $\vec{r_2}$:
\begin{align*}
	\dot{\vec{r_1}} = - \frac{\Mass_2}{\TotalMass} \dot{\vec{r}},\\
	\dot{\vec{r_2}} = \frac{\Mass_1}{\TotalMass} \dot{\vec{r}}.
\end{align*}
Abbiamo in particolare $\dot{r_1} + \dot{r_2} = \frac{\Mass_2}{\TotalMass} \dot{r} + \frac{\Mass_1}{\TotalMass} \dot{r} = \dot{r}$. \EndProof
\begin{figure}
	\includegraphics[width=0.8\textwidth]{Elementi_di_meccanica_celeste/Immagine_2Corpi_Orbite.png}
	\centering
	\caption{Orbite del problema dei due corpi.}
\end{figure}
