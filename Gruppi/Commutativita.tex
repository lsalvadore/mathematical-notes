\section{Commutativit\`a.}\label{Commutativita}
\begin{Definition}
	Siano $\Group$ un gruppo e
	$\Class \subseteq \Group$. Chiamiamo
	\Define{centralizzatore}
	di $\Class$, denotato
	$\Centralizer{\Class}$ \`e l'insieme
	$\lbrace x \in \Group |
	\ForAll{a \in \Class}{xa = ax} \rbrace$.
\end{Definition}
\begin{Theorem}
	Siano $\Group$ un gruppo e $\Class \subseteq \Group$.
	Abbiamo $\Centralizer{\Class} \IsNormalSubgroup
	\Normalizer{Class}$.
\end{Theorem}
\Proof
Siano $x, y \in Centralizer{\Class}$ e
sia $a \in \Class$.
Abbiamo
$xy^{-1}a =
xy^{-1}ayy^{-1} =
xy^{-1}yay^{-1} =
xay^{-1} =
axy^{-1}$.
Dunque $xy^{-1} \in \Centralizer{\Class}$ e
$\Centralizer{\Class} \IsSubgroup \Group$.
Inoltre segue direttamente dalle definizioni
che $\Centralizer{\Class} \subseteq \Normalizer{\Class}$ e
dunque $\Centralizer{\Class} \IsSubgroup \Normalizer{\Class}$.
\par
Sia $g \in \Normalizer{\Class}$. Abbiamo,
per opportuno $b \in \Class$,
$gxg^{-1}a =
gxbg^{-1} =
gbxg^{-1} =
gbg^{-1}gxg^{-1} =
gg^{-1}agxg^{-1} =
agxg^{-1}$.
Quindi $\Centralizer{\Class} \IsNormalSubgroup
\Normalizer{\Class}$.
\EndProof
\begin{Theorem}
	Sia $\Group$ un gruppo e consideriamo
	la sua azione interna.
	Sia $x \in \Group$.
	Abbiamo $\Centralizer{x} =
	\Stabilizer{x}$.
\end{Theorem}
\Proof
Segue direttamente dalle definizioni.
\EndProof
\begin{Definition}
	Sia $\Group$ un gruppo.
	Definiamo
	\Define{centro}[di un gruppo][centro]
	il centralizzatore $\Centralizer{\Group}$
	dell'intero gruppo.
\end{Definition}
\begin{Theorem}
	Sia $\Group$ un gruppo.
	Abbiamo $\GroupCenter{\Group} \IsNormalSubgroup \Group$.
\end{Theorem}
\Proof
Siano $x, y \in \GroupCenter{\Group}$ e sia
$a \in \Group$.
\`E immediato verificare che
$(xy^{-1})a = a(xy^{-1})$.
\par
Altrettanto immediato \`e verificare che
$axa^{-1} = x \in \GroupCenter{\Group}$.
\EndProof
\begin{Theorem}
\TheoremName{Equazione delle classi}[delle classi][equazione]
	Sia $\Group$ un gruppo finito e
	sia $R \subseteq \Group$ tale da
	contenere uno e un solo rappresentate di
	ogni classe di coniugio di $\Group$.
	Abbiamo
	$\GroupOrder{\Group} =
	\GroupOrder{\GroupCenter{\Group}} +
	\sum_{x \in R \SetMin \GroupCenter{\Group}}
	\frac{\GroupOrder{\Group}}
	{\GroupOrder{\Centralizer{x}}}$.
\end{Theorem}
\Proof
Consideriamo l'azione interna di $\Group$.
Osserviamo che un elemento costituisce da solo
la propria classe di coniugio se e solo se appartiene
al centro.
Abbiamo
$\Cardinality{\Group} =
\sum_{x \in R} \Cardinality{\Orbit{x}} =
\sum_{x \in R} \frac{\GroupOrder{\Group}}
{\GroupOrder{\Stabilizer{x}}} =
\sum_{x \in R} \frac{\GroupOrder{\Group}}
{\GroupOrder{\Centralizer{x}}} =
\GroupOrder{\GroupCenter{\Group}} +
\sum_{x \in R \SetMin \GroupCenter{\Group}} \frac{\GroupOrder{\Group}}
{\GroupOrder{\Centralizer{x}}}$.
\EndProof
\begin{Definition}
	Sia $\Group$ un gruppo e siano
	$x, y \in \Group$. Definiamo
	\Define{commutatore} di $x$ e $y$,
	denotato $\Commutator{x}{y}$,
	l'elemento $x^{-1}y^{-1}xy$.
\end{Definition}
\begin{Theorem}
	Sia $\Group$ un gruppo e siano
	$x, y \in \Group$. $x$ e $y$ commutano
	se e solo se $\Commutator{x}{y} = 1$.
\end{Theorem}
\Proof
Segue direttamente dalle definizioni.
\EndProof
\begin{Definition}
	Sia $\Group$ un gruppo. Definiamo
	\Define{sottogruppo derivato}{derivato}[sottogruppo]
	di $G$, denotato $\DerivedSubgroup{\Group}$,
	il gruppo generato da tutti i commutatori di
	$\Group$.
\end{Definition}
\begin{Theorem}
	Sia $\Group$ un gruppo.
	Abbiamo $\DerivedSubgroup{\Group} \IsNormalSubgroup
	\Group$.
\end{Theorem}
\Proof
Consideriamo un generico elemento di $\Group$,
vale a dire un prodotto finito di commutatori
di $\Group$: $\prod_{i = 1}^n \Commutator{x_i}{y_i}$,
con $n \in \mathbb{N}$, $(x_i,y_i)_{i = 1}^n \in
\DirectProduct_{i = 1}^n (\Group \times \Group)$.
Sia inoltre $g \in \Group$.
\par
Abbiamo $g\left ( \prod_{i = 1}^n \Commutator{x_i}{y_i} \right )g^{-1} =
\prod_{i = 1}^n g\Commutator{x_i}{y_i}g^{-1} =
\prod_{i = 1}^n \Commutator{gx_ig^{-1}}{gy_ig^{-1}} \in
\DerivedSubgroup{\Group}$.
\EndProof
\begin{Definition}
	Sia $\Group$ un gruppo. Definiamo
	\Define{abelianizzato}[abelianizzato][gruppo]
	di $\Group$, denotato $\Abelianization{\Group}$,
	il quoziente $\Quotient{\Group}{\DerivedSubgroup{\Group}}$.
\end{Definition}
\begin{Theorem}
	Sia $\Group$ un gruppo.
	$\Abelianization{\Group}$ \`e un gruppo abeliano.
\end{Theorem}
\Proof
Immediata osservando che $\DerivedSubgroup{\Abelianization{\Group}} =
\lbrace 1 \rbrace$.
\EndProof
