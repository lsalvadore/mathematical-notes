\section{Prodotti semidiretti e diretti.}\label{ProdottiSemidirettiEDiretti}
\begin{Theorem}
	Siano $\Group_1$ e $\Group_2$ gruppi.
	Sia $\Homomorphism: \Group_2 \rightarrow \Automorphisms{\Group_1}$
	un omomorfismo.
	Il magma definito dalla classe $\Group_1 \times \Group_2$
	e dall'operazione $(x_1,y_1)(x_2,y_2) =
	(x_1\Homomorphism(y_1)(x_2),y_1y_2)$ \`e un gruppo.
\end{Theorem}
\Proof
Siano $x_1, x_2, x_3 \in \Group_1$ e
$y_1, y_2, y_3 \in \Group_2$.
Abbiamo
\begin{itemize}
	\item
	$((x_1,y_1)(x_2,y_2))(x_3,y_3) =
	(x_1\Homomorphism(y_1)(x_2),y_1y_2)(x_3,y_3) =
	(x_1\Homomorphism(y_1)(x_2)\Homomorphism(y_1y_2)(x_3),y_1y_2y_3) =
	(x_1\Homomorphism(y_1)(x_2)\Homomorphism(y_1)(\Homomorphism(y_2)(x_3)),y_1y_2y_3) =
	(x_1\Homomorphism(y_1)(x_2\Homomorphism(y_2)(x_3)),y_1y_2y_3) =
	(x_1,y_1)(x_2\Homomorphism(y_2)(x_3),y_2y_3) =
	(x_1,y_1)((x_2,y_2)(x_3,y_3))$;
	\item
	$(x_1,y_1)(1,1) =
	(x_1\Homomorphism(y_1)(1),y_1) =
	(x_1,y_1)$;
	\item
	$(1,1)(x_1,y_1) =
	(\Homomorphism(1)(x_1),y_1) =
	(x_1,y_1)$;
	\item
	$(x_1,y_1)((\Homomorphism{y_1})^{-1}(x_1^{-1}),y_1^{-1}) =
	(x_1\Homomorphism(y_1)((\Homomorphism(y_1))^{-1}(x_1^{-1}),y_1y_1^{-1}) =
	(1,1)$.
\end{itemize}
\EndProof
\begin{Definition}
	Con le notazioni del teorema precedente, chiamiamo il magma che abbiamo
	dimostrato essere un gruppo
	\Define{prodotto semidiretto}[semidiretto][prodotto]
	di $\Group_1$ e $\Group_2$ secondo $\Homomorphism$
	e lo denotiamo $\Group_1 \SemidirectProduct[\Homomorphism] \Group_2$, omettendo
	$\Homomorphism$ quando esso \`e chiaro dal contesto.
	Chiamiamo il prodotto $\Group_1 \times \Group_2$
	\Define{prodotto diretto}[diretto][prodotto]
	di $\Group_1$ e $\Group_2$ per distinguerlo pi\`u chiaramente
	dal prodotto semidiretto.
\end{Definition}
\begin{Theorem}
	Siano $\Group_1$ e $\Group_2$ gruppi e sia
	$\Homomorphism: \Group_2 \rightarrow \Automorphisms{\Group_1}$
	un omomorfismo.
	$\Group_1 \SemidirectProduct_{\Homomorphism} \Group_2 =
	\Group_1 \times \Group_2$ se e solo se
	$\Homomorphism$ \`e l'omomorfismo
	che a ogni $y \in \Group_2$ associa $\Identity[\Group_1] \in \Automorphisms{\Group_1}$.
\end{Theorem}
\Proof
Siano $x_1, x_2 \in \Group_1$ e $y_1, y_2 \in \Group_2$.
Abbiamo
$(x_1\Homomorphism(y_1)(x_2),y_1y_2)$
se e solo se
$x_1\Homomorphism(y_1)x_2 = x_1x_2$,
che equivale a
$\Homomorphism(y_1)x_2 = x_2$.
Quindi
$\Isomorphic{\Group_1 \SemidirectProduct{\Homomorphism} \Group_2}
{\Group_1 \times \Group_2}$ equivale a
$\ForAll{y \in \Group_2}
{\Homomorphism{y} = \Identity[\Group_1]}$.
\EndProof
\begin{Theorem}
	Sia $\Group$ un gruppo e siano
	$\NormalSubgroup$ e $\Subgroup$ gruppi tali che
	\begin{itemize}
		\item
		$\NormalSubgroup \IsNormalSubgroup \Group$;
		\item
		$\Subgroup \IsSubgroup \Group$;
		\item
		$\Group = \NormalSubgroup\Subgroup$;
		\item
		$\NormalSubgroup \cap \Subgroup = \lbrace 1 \rbrace$.
	\end{itemize}
	Sia inoltre $\InnerAutomorphism: \Subgroup \rightarrow \Automorphisms{\NormalSubgroup}$ l'azione per
	coniugio di $\Subgroup$ su $\NormalSubgroup$.
	L'applicazione
	$\Isomorphism: \NormalSubgroup \SemidirectProduct_{\InnerAutomorphism} \Subgroup \rightarrow
	\Group$
	che a $(x,y) \in \NormalSubgroup \SemidirectProduct_{\InnerAutomorphism} \Subgroup$ associa
	$(x,y) \in \Group$ \`e un isomorfismo.
\end{Theorem}
\Proof
Siano $x_1,x_2 \in \NormalSubgroup$ e $y_1,y_2 \in \Subgroup$.
Abbiamo $\Isomorphism((x_1,y_1)(x_2,y_2)) =
\Isomorphism((x_1y_1x_2y_1^{-1},y_1y_2)) =
x_1y_1x_2y_1^{-1}y_1y_2 =
x_1y_1x_2y_2 =
\Isomorphism((x_1,y_1))\Isomorphism((x_2,y_2))$.
Quindi $\Isomorphism$ \`e un omomorfismo.
\par
Poich\'e $\Group = \NormalSubgroup\Subgroup$,
$\Isomorphism$ \`e suriettivo.
\par
Sia $(x,y) \in \Kernel{\Isomorphism}$. Allora
$xy = 1$, da cui $y = x^{-1}$, ma poich\'e
$\NormalSubgroup \cap \Subgroup = \lbrace 1 \rbrace$,
non pu\`o che essere $x = y = 1$. Quindi $\Isomorphism$
\`e anche iniettivo e pertanto un isomorfismo. \EndProof
\begin{Corollary}
	Con le ipotesi del teorema precedente,
	$\Subgroup \IsNormalSubgroup \Group$
	equivale a $\Isomorphic{\Group}
	{\NormalSubgroup \times \Subgroup}$.
\end{Corollary}
\Proof
$\Group$ \`e isomorfo a $\NormalSubgroup \times \Subgroup$
se e solo se, per ogni $x \in \NormalSubgroup$ e ogni
$y \in \Subgroup$ abbiamo $yxy^{-1} = x$, che equivale a
$yxy^{-1}x = 1$.
\par
Se $\Subgroup \IsNormalSubgroup \Group$, allora
$yxy^{-1}x^{-1} \in \NormalSubgroup \cap \Subgroup$, quindi
certamente $yxy^{-1}x^{-1} = 1$.
\par
Assumiamo ora invece 
$\ForAll{x \in \NormalSubgroup}
{\ForAll{y \in \Subgroup}
{yxy^{-1}x^{-1} = 1}}$.
Sia $g \in \Group$. Per opportuni $a \in \NormalSubgroup$ e
$b \in Subgroup$ abbiamo $g = ab$.
Sia $y \in \Subgroup$.
Abbiamo $gyg^{-1} =
abyb^{-1}a^{-1} =
byb^{-1} \in \Subgroup$.
Dunque $\Subgroup \IsNormalSubgroup \Group$.
\EndProof
