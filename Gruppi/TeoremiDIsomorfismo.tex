\section{Teoremi d'isomorfismo.}\label{TeoremiDIsomorfismo}
\begin{Theorem}
\TheoremName{Primo teorema d'isomorfismo di gruppi}[primo d'isomorfismo di gruppi][teorema]
	Sia $\Homomorphism: \Group_1 \rightarrow \Group_2$ un omomorfismo
	tra i gruppi $\Group_1$ e $\Group_2$. Allora
	$\Isomorphic{\Quotient{\Group_1}{\Kernel{\Homomorphism}}}
	{\Image{\Homomorphism}}$.
\end{Theorem}
\Proof
Sia $\Homomorphism_0$ l'omomorfismo ottenuto da $\Homomorphism$
per passaggio al quoziente: esso \`e chiaramente suriettivo.
Inoltre $\Homomorphism_0$ \`e anche iniettivo perch\'e
$\Kernel(\Homomorphism_0) =
\lbrace \Kernel(\Homomorphism) \rbrace = 1$.
\EndProof
\begin{Theorem}
\TheoremName{Secondo teorema d'isomorfismo di gruppi}[secondo d'isomorfismo di gruppi][teorema]
	Siano
	\begin{itemize}
		\item
		$\Group$ un gruppo;
		\item
		$\Subgroup \IsSubgroup \Group$;
		\item
		$\NormalSubgroup \IsNormalSubgroup \Group$.
	\end{itemize}
	Abbiamo
	\begin{itemize}
		\item
		$\Subgroup\NormalSubgroup \IsSubgroup \Group$;
		\item
		$\NormalSubgroup \IsNormalSubgroup \Subgroup\NormalSubgroup$;
		\item
		$\Subgroup \cap \NormalSubgroup \IsNormalSubgroup \Subgroup$;
		\item
		l'applicazione
		$\Homomorphism: \Subgroup \rightarrow
		\Quotient{\Subgroup\NormalSubgroup}{\NormalSubgroup}$
		che ad $x \in \Subgroup$ associa $\Coset{x}{\NormalSubgroup} \in
		\Quotient{\Subgroup\NormalSubgroup}{\NormalSubgroup}$
		\`e un omomorfismo;
		\item
		l'omomorfismo $\Homomorphism_0:
		\Quotient{\Subgroup}{\Subgroup \cap \NormalSubgroup} \rightarrow
		\Quotient{\Subgroup\NormalSubgroup}{\NormalSubgroup}$,
		ottenuto da $\Homomorphism$ per passaggio al quoziente,
		\`e un isomorfismo,
		e dunque in particolare
		$\Isomorphic
		{\Quotient{\Subgroup}{\Subgroup \cap \NormalSubgroup}}
		{\Quotient{\Subgroup\NormalSubgroup}{\NormalSubgroup}}$.
		\end{itemize}
\end{Theorem}
\Proof
Siano $x_1, x_2 \in \Subgroup$ e $y_1, y_2 \in \NormalSubgroup$.
Abbiamo
$x_1y_1(x_2y_2)^{-1} =
x_1y_1y_2^{-1}x_2^{-1} =
x_1x_2^{-1}x_2y_1y_2^{-1}x_2^{-1}$.
Poich\'e $x_1x_2^{-1} \in \Subgroup$ e
$x_2y_1y_2^{-1}x_2^{-1} \in \NormalSubgroup$ per la normalit\`a
di $\NormalSubgroup \in \Group$, ne deduciamo che $\Subgroup\NormalSubgroup
\IsSubgroup \Group$.
\par
$\NormalSubgroup$ \`e normale in $\Group$ e dunque a maggior ragione lo \`e
in $\Subgroup\NormalSubgroup$.
\par
Siano $x \in \Subgroup \cap \NormalSubgroup$ e $y \in \Subgroup$.
Abbiamo $yxy^{-1} \in \Subgroup$ perch\'e $x,y \in \Subgroup$ e
$yxy^{-1} \in \NormalSubgroup$ per normalit\`a di $\NormalSubgroup$ in
$\Group$. Dunque $\Subgroup \cap \NormalSubgroup \IsNormalSubgroup \Subgroup$.
\par
$\Homomorphism$ \`e un omomorfismo perch\'e composizione dell'inclusione
canonica di $\Subgroup$ in $\Subgroup\NormalSubgroup$ con la proiezione canonica
di $\Subgroup\NormalSubgroup$ in
$\Quotient{\Subgroup\NormalSubgroup}{\NormalSubgroup}$.
\par
Il nucleo di $\Homomorphism$ \`e $\Subgroup \cap \NormalSubgroup$ e
$\Homomorphism$ \`e un omomomorfismo suriettivo.
Dunque $\Homomorphism_0$ \`e un isomorfismo.
\EndProof
\begin{Theorem}
\TheoremName{Terzo teorema d'isomorfismo di gruppi}[terzo d'isomorfismo di gruppi][teorema]
	Siano
	\begin{itemize}
		\item
		$\Group$ un gruppo;
		\item
		$\NormalSubgroup_1,
		\NormalSubgroup_2 \IsNormalSubgroup \Group$, con
		$\NormalSubgroup_1 \subseteq \NormalSubgroup_2$.
	\end{itemize}
	Allora
	\begin{itemize}
		\item
		esiste un'applicazione
		$\Homomorphism: \Quotient{\Group}{\NormalSubgroup_1}
		\rightarrow \Quotient{\Group}{\NormalSubgroup_2}$
		che associa a
		$\Coset{x}{\NormalSubgroup_1} \in
		\Quotient{\Group}{\NormalSubgroup_1}$
		la classe laterale
		$\Coset{x}{\NormalSubgroup_2} \in
		\Quotient{\Group}{\NormalSubgroup_2}$;
		\item
		$\Homomorphism$ \`e un omomorfismo;
		\item l'applicazione
		$\Homomorphism_0:
		\Quotient{\Quotient{\Group}{\NormalSubgroup_1}}
		{\Quotient{\NormalSubgroup_2}{\NormalSubgroup_1}}
		\rightarrow \Quotient{\Group}{\NormalSubgroup_2}$,
		ottenuta da $\Homomorphism$ per passaggio al quoziente,
		\`e un isomorfismo, e dunque in particolare
		$\Isomorphic{\Quotient
		{\Quotient{\Group}{\NormalSubgroup_1}}
		{\Quotient{\NormalSubgroup_2}{\NormalSubgroup_1}}}
		{\Quotient{\Group}{\NormalSubgroup_2}}$.
	\end{itemize}
\end{Theorem}
\Proof
L'esistenza di $\Homomorphism$ \`e garantita dal fatto che,
per ogni $x, y \in \Group$,
$\Implies{xy^{-1} \in \NormalSubgroup_1}{xy^{-1} \in \NormalSubgroup_2}$.
\par
Il fatto che $\Homomorphism$ sia un omomorfismo segue dal fatto che \`e un
omomorfismo la proiezione di $\Group$ al quoziente
$\Quotient{\Group}{\NormalSubgroup_2}$, nonch\'e
dalla struttura di gruppo definita su $\Quotient{\Group}{\NormalSubgroup_1}$.
\par
$\Homomorphism$ \`e un omomorfismo suriettivo per costruzione.
Inoltre il nucleo di $\Homomorphism$ \`e costituito da tutte le classi laterali
$\Coset{x}{\NormalSubgroup_1}$ con $x \in \NormalSubgroup_2$, le quali sono
esattamente gli elementi di $\Quotient{\NormalSubgroup_2}{\NormalSubgroup_1}$.
\EndProof
\begin{Theorem}
\TheoremName{Teorema di corrispondenza di sottogruppi}[di corrispondenza di sottogruppi ][teorema]
	Sia $\Group$ un gruppo e sia
	$\NormalSubgroup \IsNormalSubgroup \Group$.
	Indichiamo con $\Class_1$ la classe di tutti i sottogruppi
	di $\Group$ contenenti $\NormalSubgroup$ e con $\Class_2$ la classe
	di tutti i sottogruppi di $\Quotient{\Group}{\NormalSubgroup}$:
	\begin{itemize}
		\item
		se $X \in \Class_1$,
		allora $\Quotient{X}{\NormalSubgroup} \in \Class_2$;
		\item
		l'applicazione $\Map: \Class_1 \rightarrow \Class_2$ che a
		$X \in \Class_1$ associa
		$\Quotient{X}{\NormalSubgroup} \in \Class_2$
		\`e un isomorfismo d'ordine.
	\end{itemize}
\end{Theorem}
\Proof
Sia $X \in \Class_1$ e sia $a \in X$ tale che
$\Coset{a}{\NormalSubgroup} \in \Quotient{\Group}{\NormalSubgroup}$.
Gli elementi di $\Coset{a}{\NormalSubgroup}$ sono tutti e soli i prodotti di
$a$ con elementi $\NormalSubgroup$. Dunque
$\Coset{a}{\NormalSubgroup} \subseteq X$ e quindi
$\Coset{a}{\NormalSubgroup} \in \Quotient{X}{\NormalSubgroup}$.
\par
Se $X, Y \in \Class_1$ sono tali che $X \IsSubgroup Y$, allora
$\Quotient{X}{\NormalSubgroup} \IsSubgroup \Quotient{Y}{\NormalSubgroup}$.
Dunque $\Map$ \`e un'applicazione crescente.
\par
Siano ora $X, Y \in \Class_1$ qualsiasi e supponiamo
$\Quotient{X}{\NormalSubgroup} = \Quotient{Y}{\NormalSubgroup}$.
Abbiamo
$X =
\bigcup_{A \in \Quotient{X}{\NormalSubgroup}} A =
\bigcup_{A \in \Quotient{Y}{\NormalSubgroup}} A =
Y$. Quindi $\Map$ \`e un'applicazione iniettiva.
\par
Sia ora $Z \in \Class_2$ e poniamo $X = \bigcup_{A \in Z}$.
Per costruzione di $\Class_2$, $\NormalSubgroup \subseteq X$.
Siano poi $x, y \in X$. Allora
$\Coset{x}{\NormalSubgroup}, \Coset{y}{\NormalSubgroup} \in Z$ e,
poich\'e $Z$ \`e un gruppo,
$\Coset{xy^{-1}}{\NormalSubgroup} \in Z$, da cui $xy^{-1} \in X$.
Quindi $X \in \Class_1$.
Se verifica immediatamente che $\Map(X) = Z$. Quindi $\Map$ \`e
un'applicazione suriettiva.
\par
Dalla biettivit\`a e dalla cresenza di $\Map$ deduciamo che
$\Map$ \`e un isomorfismo d'ordine.
\EndProof
