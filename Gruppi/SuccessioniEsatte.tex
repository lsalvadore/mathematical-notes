\section{Successioni esatte.}\label{SuccessioniEsatte}
\begin{Definition}
	Sia $(\Homomorphism_i)_{i \in n}$, con $n \in \omega + 1$, una
	successione di omomorfismi tali che, per ogni $i \in n$ tale che
	$i + 1 \in n$, si abbia
	$\Image{\Homomorphism_i} = \Kernel{\Homomorphism_{i + 1}}$. Si
	dice che la successione $(\Homomorphism_i)_{i \in n}$ \`e
	\Define{esatta}[esatta][successione], denotate
	$\Dom{\Homomorphism_0} \xrightarrow{\Homomorphism_0}
	\Dom{\Homomorphism_1} \xrightarrow{\Homomorphism_1}
	... \xrightarrow{\Homomorphism_{n - 2}}
	\Dom{\Homomorphism_{n - 1}} \xrightarrow{\Homomorphism_{n - 1}}
	\Dom{\Homomorphism_n} \xrightarrow{\Homomorphism_n}$.
\end{Definition}
\begin{Definition}
	Sia una successione esatta della forma
	$0 \xrightarrow{\Homomorphism_0}
	\Group_1 \xrightarrow{\Homomorphism_0}
	\Group_2 \xrightarrow{\Homomorphism_1}
	\Group_3 \xrightarrow{\Homomorphism_2}
	0$:
	si dice che tale successione \`e
	\Define{corta}[esatta corta][succesione].
\end{Definition}
\begin{Theorem}
	Sia la successione esatta corta
	$0 \rightarrow
	\Group_1 \xrightarrow{\Homomorphism_1}
	\Group_2 \xrightarrow{\Homomorphism_2}
	\Group_3 \rightarrow
	0$.
	Sono equivalenti i seguenti fatti:
	\begin{itemize}
		\item $\Homomorphism_1$ \`e una retrazione;
		\item $\Homomorphism_2$ \`e una sezione;
		\item $\Group_2$ \`e isomorfo a
		$\Group_1 \DirectSum \Group_3$ tramite
		un isomorfismo $\Isomorphism$ tale che
		$\Isomorphism \circ \Homomorphism_1 = \Identity_{\Group_1}$
		e $\Section \circ \Isomorphism = \Identity_{\Group_2}$.
	\end{itemize}
\end{Theorem}
\Proof Supponiamo $\Homomorphism$ possieda una retrazione $\Retraction$.
