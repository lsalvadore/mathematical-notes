\section{Definizioni e teoremi di base.}
\label{MetodiNumericiPerSistemiLineari_DefinizioniETeoremiDiBase}
\begin{Theorem}
  Siano
  \begin{itemize}
    \item $n \in \NotZero{\mathbb{N}}$;
    \item $\Vector, \VarVector \in \mathbb{C}^n$.
  \end{itemize}
  Il costo computazionale del prodotto
  $\Transposed{\Vector}\VarVector$ \`e
  $2n - 1 = \BigO{n}$ operazioni aritmetiche.
\end{Theorem}
\Proof Abbiamo
$\Transposed{\Vector}\VarVector
= \Vector_k\VarVector_k$: il calcolo prevede dunque
\begin{itemize}
  \item $n$ operazioni di coniugazione, che non consideriamo operazioni
    aritmetiche perch\'e possiamo effettuarle con un semplice cambio di
    segno se rappresentiamo le componenti di $\Vector$ in forma cartesiana
    o polare;
  \item $n$ moltiplicazioni;
  \item $n - 1$ somme. \EndProof
\end{itemize}
\begin{Corollary}
  Siano
  \begin{itemize}
    \item $n \in \NotZero{\mathbb{N}}$;
    \item $\Matrix, \VarMatrix \in \mathbb{C}^{n \times n}$.
  \end{itemize}
  Il costo computazionale del prodotto
  $\Matrix\VarMatrix$ \`e
  $2n^3 - n^2 = \BigO{n^3}$ operazioni aritmetiche.
\end{Corollary}
\Proof Il calcolo del prodotto
$\Matrix\VarMatrix$
comporta il calcolo di $n^2$ elementi calcolati come prodotto scalare
di vettori di $n$ elementi. \EndProof
