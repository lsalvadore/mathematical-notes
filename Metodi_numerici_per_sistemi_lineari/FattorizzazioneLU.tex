\section{Fattorizzazione $LU$.}
\label{MetodiNumericiPerLaRisoluzioneDiSistemiLineari_FattorizzazioneLU}
\begin{Definition}
  Chiamiamo
  \Define{matrice elementare di Gauss}[elementare di Gauss][matrice]
  una matrice elementare
  $\GaussElementaryMatrix \in \mathbb{C}^{n \times n}$
  ($n \in \NotZero{\mathbb{N}}$)
  tale che
  $\GaussElementaryMatrix
  = \Identity - \Vector\HermitianTransposed{\CanonicalBase_0}$, con
  \begin{itemize}
    \item $\Scalar \in \mathbb{C}^{n \times n}$;
    \item $\Vector \in \mathbb{C}^{n \times n}$, dove $\Vector_0 = 0$;
    \item $\CanonicalBase_0$ primo vettore della base canonica.
  \end{itemize}
  Una tale matrice ha la forma
  \[
    \GaussElementaryMatrix =
    \lmatrix
    \begin{array}{cccc}
      1                 &   &         &   \\
      - \Vector_1       & 1 &         &   \\
      \vdots            &   & \ddots  &   \\
      - \Vector_{n - 1} &   &         & 1 \\
    \end{array}
    \rmatrix.
  \]
\end{Definition}
\begin{Theorem}
  Con le notazioni della definizione precedente,
  $\GaussElementaryMatrix$ \`e invertibile e la sua inversa \`e
  \[
    \GaussElementaryMatrix =
    \lmatrix
    \begin{array}{cccc}
      1                 &   &         &   \\
      \Vector_1         & 1 &         &   \\
      \vdots            &   & \ddots  &   \\
      \Vector_{n - 1}   &   &         & 1 \\
    \end{array}
    \rmatrix.
  \]
  L'inversa \`e calcolabile con $0$ operazioni aritmetiche.
\end{Theorem}
\Proof Segue per calcolo diretto e per il fatto che l'inversa \`e calcolabile
con $n - 1$ cambi di segno, che non sono operazioni aritmetiche. \EndProof
\begin{Theorem}
  Siano
  \begin{itemize}
    \item $n \in \NotZero{\mathbb{N}}$;
    \item $\VarVarVector \in \NotZero{\mathbb{C}^n}$ tale che
      $\VarVarVector_0 \neq 0$.
  \end{itemize}
  Sia
  $\GaussElementaryMatrix
  = \Identity - \Vector\HermitianTransposed{\CanonicalBase^0}$,
  con
  \begin{itemize}
    \item $e^0$ primo vettore della base canonica;
    \item $\Vector_k =
            \begin{cases}
              1&\text{ se }k = 0,\\
              \frac{\VarVarVector_k}{\VarVarVector_0}&\text{ se }k > 0,
          \end{cases}$
  \end{itemize}
  vale a dire
  \[
    \GaussElementaryMatrix =
    \lmatrix
      \begin{array}{cccc}
        1 &&&\\
        - \frac{\VarVarVector_{2}}{\VarVarVector_0}         & \ddots & &\\
        \vdots            & & \ddots &  \\
        - \frac{\VarVarVector_{n - 1}}{\VarVarVector_0}   &   & & 1 \\
      \end{array}
    \rmatrix.
  \]
  $\GaussElementaryMatrix$ \`e l'unica matrice elementare di Gauss tale che
  \begin{itemize}
    \item $\GaussElementaryMatrix\VarVarVector_0 = \VarVarVector_0$;
    \item $\ForAll{k \in \NotZero{n}}{(\GaussElementaryMatrix\VarVarVector)_k = 0}$.
  \end{itemize}
\end{Theorem}
\Proof Si verifica direttamente che $\GaussElementaryMatrix$ verifica le relazioni richieste.
\par Supponiamo ora
$\GaussElementaryMatrix' = \Identity - \VarVector \HermitianTransposed{\CanonicalBase^0}$
con $\VarVector \in \mathbb{C}^n$.
\par Se $\GaussElementaryMatrix'$ verifica le relazioni richieste, abbiamo allora,
\[
  (\Identity - \VarVector \HermitianTransposed{\CanonicalBase^0})\VarVarVector
  = \VarVarVector_0 \CanonicalBase^0,
\]
a sua volta equivalente a
\begin{align*}
  \VarVector
  &= \frac{\VarVarVector - \VarVarVector_0 \CanonicalBase^0}
    {\HermitianTransposed{\CanonicalBase^0}\VarVarVector},\\
  &= \frac{\VarVarVector}{\VarVarVector_0} - \CanonicalBase_0,\\
  &= \VarVector.\text{ \EndProof}
\end{align*}
\begin{Theorem}
  Siano
  \begin{itemize}
    \item $n \in \NotZero{\mathbb{N}}$;
    \item $\Vector \in \mathbb{C}^n$, con $\Vector_0 = 0$;
    \item $\CanonicalBase_0 \in \mathbb{C}^n$ primo vettore della base canonica;
    \item $\GaussElementaryMatrix = \Identity - \Vector\HermitianTransposed{\CanonicalBase_0}$
      matrice elementare di Gauss;
    \item $\PermutationMatrix \in \mathbb{C}^{n \times n}$ matrice di permutazione tale che
      la permutazione $\Permutation \in \SymmetricGroup{n}$ associata alle sue righe lascia
      fisso $0$.
  \end{itemize}
  $\PermutationMatrix\GaussElementaryMatrix\PermutationMatrix^{-1}$
  \`e una matrice elementare di Guass.
\end{Theorem}
\Proof Abbiamo, per $(k,j) \in n \times n$,
$(\PermutationMatrix\Vector\HermitianTransposed{\CanonicalBase_0})_k^j
= \begin{cases}
  \Vector_{\Permutation(k)}&\text{ se }j = 0,\\
  0&\text{ altrimenti}.
\end{cases}$
\par Poich\'e il prodotto a destra per $\PermutationMatrix$ lascia
invariata la prima riga del secondo fattore, il prodotto a sinistra
per $\PermutationMatrix$ lascia invariata la prima colonna del primo fattore.
Ne consegue
$(\PermutationMatrix\Vector\HermitianTransposed{\CanonicalBase_0}\PermutationMatrix^{-1})_k^j
= \begin{cases}
  \Vector_{\Permutation(k)}&\text{ se }j = 0,\\
  0&\text{ altrimenti}.
\end{cases}$
\par Dunque
$\PermutationMatrix\GaussElementaryMatrix\PermutationMatrix^{-1}$
\`e una matrice elementare di Guass. \EndProof
\begin{Definition}
  Sia $\Matrix \in \mathbb{C}^{n \times n}$ ($n \in \mathbb{N}$). Chiamiamo
  \Define{fattorizzazione $LU$}[$LU$][fattorizzazione]
  o
  \Define{decomposizione $LU$}[$LU$][decompisionzione]
  di $\Matrix$ una coppia di matrici
  $(L,U) \in \mathbb{C}^{n \times n} \times \mathbb{C}^{n \times n}$
  tali che
  \begin{itemize}
    \item $\Matrix = LU$;
    \item $L$ \`e triangolare inferiore con tutti gli elementi diagonali uguali
      a $1$;
    \item $U$ \`e triangolare superiore.
  \end{itemize}
  Chiamiamo inoltre
  \Define{fattorizzazione $LUP$}[$LUP$][fattorizzazione]
  o
  \Define{decomposizione $LUP$}[$LUP$][decomposizione]
  di $\Matrix$ una terna di matrici
  $(L,U,P) \in
  \mathbb{C}^{n \times n} \times \mathbb{C}^{n \times n} \times \mathbb{C}^{n \times n}$
  tali che
  \begin{itemize}
    \item $P \Matrix = LU$;
    \item $L$ \`e triangolare inferiore con tutti gli elementi diagonali uguali
      a $1$;
    \item $U$ \`e triangolare superiore;
    \item $P$ \`e una matrice di permutazione.
  \end{itemize}
\end{Definition}
\begin{Definition}
  Con le notazioni del teorema \ref{th_FattorizzazioneMatriciElementari},
  l'algoritmo ottenuto dall'algoritmo del teorema \ref{th_FattorizzazioneMatriciElementari}
  scegliendo come matrici $(\ElementaryMatrix_{(l,0)})_{l \in m}$ matrici elementari di Gauss
  si chiama
  \Define{metodo di Gauss}[di Gauss][metodo].
\end{Definition}
\begin{listing}
	\insertcode{octave}{"Metodi_numerici_per_sistemi_lineari/FattorizzazioneLU.m"}
	\caption{Implementazione del metodo di Gauss in \LanguageName{octave}.}
\end{listing}
\begin{Theorem}
  Con le notazioni del teorema \ref{th_FattorizzazioneMatriciElementari}, il metodo di Gauss
  restituisce una fattorizzazione $LU$ di $\Matrix$.
\end{Theorem}
\Proof $L$ \`e prodotto di matrici triangolari inferiori con tutti gli elementi diagonali
uguali a $1$, dunque \`e trinagolare inferiore con tutti gli elementi diagonali uguali a $1$
essa stessa. \EndProof
\begin{Theorem}
  Se $\Matrix \in \mathbb{C}^{n \times n}$ ($n \in \NotZero{\mathbb{N}}$) ha tutte le
  sottomatrici principali invertibili, allora ammette una fattorizzazione $LU$.
\end{Theorem}
\begin{Theorem}
  \label{th_FattorizzazioneLUP}
  Siano
  \begin{itemize}
    \item $n, m \in \NotZero{\mathbb{N}}$;
    \item $\Matrix \in \mathbb{C}^{n \times m}$.
  \end{itemize}
  Il seguente algoritmo costruisce una fattorizzazione $LUP$ di
  $\Matrix$
  \begin{enumerate}
    \item\label{FattorizzazioneLUP_1} si ponga $l = 0$, $L = \Identity[n]$;
    \item\label{FattorizzazioneLUP_2} si definisca
      $\VarMatrix = (\Matrix_k^j)_{\substack{k = l\\j= l}}^{\substack{k = n - 1}{j = m - 1}}$;
    \item\label{FattorizzazioneLUP_2_1} se possibile, si scelga una matrice di permutazione
      $\PermutationMatrix_{(l)} \in \mathbb{C}^{n \times n}$ che permuti la riga $l$ di
      $\Matrix$ con una riga $r \geq l$ tale che $\Matrix_r^l \neq 0$; se non \`e possibile
      si vada al punto \ref{FattorizzazioneLUP_9};
    \item\label{FattorizzazioneLUP_2_2} si ridefinisca
      $\Matrix = \PermutationMatrix_{(l)} \Matrix$;
    \item\label{FattorizzazioneLUP_2_2} si ridefinisca
      $L = \Transposed{\PermutationMatrix_{(l)}} \Matrix \PermutationMatrix$;;
    \item\label{FattorizzazioneLUP_3} sia $\GaussElementaryMatrix_{(l,0)}$ la matrice elementare
      di Gauss tale che
      $\GaussElementaryMatrix_{(l,0)}\VarMatrix = \CanonicalBase_1^{(l)}$, dove
      $\CanonicalBase_1^{(l)}$ \`e il $l$-esimo vettore della base
      canonica di $\mathbb{C}^{m - l}$;
    \item\label{FattorizzazioneLUP_4} si ponga
      $(\ElementaryMatrix_{(l)})_k^j =
      \begin{cases}
        1&\text{ se }k = j < l,\\
        (\GaussElementaryMatrix_{(l,0)})_{k - l}^{j - l}&\text{ se }\And{k \geq l}{j \geq l},\\
        0&\text{ altrimenti},
      \end{cases}$
      vale a dire
      \[
        \ElementaryMatrix_{(l)} =
        \lmatrix
        \begin{array}{c|c}
          \Identity &\\
          \hline
          & \GaussElementaryMatrix_{(l,0)}
        \end{array}
        \rmatrix;
      \]
    \item\label{FattorizzazioneLUP_5} si ridefinisca
      $\Matrix = \ElementaryMatrix_{(l)}\Matrix$;
    \item\label{FattorizzazioneLUP_6} si incrementi
      $l$ di $1$;
    \item\label{FattorizzazioneLUP_6_1} si ridefinisca
      $L = L \ElementaryMatrix^{-1}$;
    \item\label{FattorizzazioneLUP_7} se $l < m$, si torni al
      punto \ref{FattorizzazioneLUP_2};
    \item\label{FattorizzazioneLUP_9} si ponga
      $U = \Matrix$;
    \item\label{FattorizzazioneLUP_9_1} si ponga
      $\PermutationMatrix = \PermutationMatrix_{(0)} \cdot ... \cdot \PermutationMatrix_{(m)}$;
    \item\label{FattorizzazioneLUP_10} l'algoritmo termina restituendo
      $L$, $U$ e $P$.
  \end{enumerate}
\end{Theorem}
\begin{listing}
	\insertcode{octave}{"Metodi_numerici_per_sistemi_lineari/FattorizzazioneLUP.m"}
	\caption{Implementazione di un algoritmo per il calcolo della decomposizione $LUP$ in
    \LanguageName{octave}.}
\end{listing}
\par In molte applicazioni, siamo interessati ad utilizzare esclusivamente la matrice $U$ di
una decomposizione $LUP$: per esempio essa pu\`o essere usata per calcolare rapidamente il
rango della matrice di partenza $\Matrix$, oppure si pu\`o ridurre a scalini la matrice completa
di un sistema per perlo poi risolvere tramite sostituzione all'indietro. In tal caso, l'algoritmo
seguente consente di ottenere la matrice $U$ senza impiegare tempo di calcolo ulteriore per le
matrici $L$ e $P$.
\begin{Definition}
  \label{def_EliminazioneGaussiana}
  Siano
  \begin{itemize}
    \item $n, m \in \NotZero{\mathbb{N}}$;
    \item $\Matrix \in \mathbb{C}^{n \times m}$.
  \end{itemize}
  Chiamiamo
  \Definition{eliminazione gaussiana}[gaussiana][eliminazione]
  il seguente algoritmo
  \begin{enumerate}
    \item\label{EliminazioneGaussiana_1} si ponga $l = 0$, $L = \Identity[n]$;
    \item\label{EliminazioneGaussiana_2} si definisca
      $\VarMatrix = (\Matrix_k^j)_{\substack{k = l\\j= l}}^{\substack{k = n - 1}{j = m - 1}}$;
    \item\label{EliminazioneGaussiana_2_1} se possibile, si scambino la riga $l$ di $\Matrix$
      con una riga $r \geq l$ di $\Matrix$ in modo tale che, dopo lo scambio, $\Matrix_l^l \neq 0$;
      se non \`e possibile si vada al punto \ref{EliminazioneGaussiana_9};
    \item\label{EliminazioneGaussiana_3} sia $\GaussElementaryMatrix_{(l,0)}$ la matrice elementare
      di Gauss tale che
      $\GaussElementaryMatrix_{(l,0)}\VarMatrix = \CanonicalBase_1^{(l)}$, dove
      $\CanonicalBase_1^{(l)}$ \`e il $l$-esimo vettore della base
      canonica di $\mathbb{C}^{m - l}$;
    \item\label{EliminazioneGaussiana_4} si ponga
      $(\ElementaryMatrix_{(l)})_k^j =
      \begin{cases}
        1&\text{ se }k = j < l,\\
        (\GaussElementaryMatrix_{(l,0)})_{k - l}^{j - l}&\text{ se }\And{k \geq l}{j \geq l},\\
        0&\text{ altrimenti},
      \end{cases}$
      vale a dire
      \[
        \ElementaryMatrix_{(l)} =
        \lmatrix
        \begin{array}{c|c}
          \Identity &\\
          \hline
          & \GaussElementaryMatrix_{(l,0)}
        \end{array}
        \rmatrix;
      \]
    \item\label{EliminazioneGaussiana_5} si ridefinisca
      $\Matrix = \ElementaryMatrix_{(l)}\Matrix$;
    \item\label{EliminazioneGaussiana_6} si incrementi
      $l$ di $1$;
    \item\label{EliminazioneGaussiana_7} se $l < m$, si torni al
      punto \ref{EliminazioneGaussiana_2};
      $U$.
  \end{enumerate}
\end{Definition}
\begin{listing}
	\insertcode{octave}{"Metodi_numerici_per_sistemi_lineari/EliminazioneGaussiana.m"}
	\caption{Implementazione dell'eliminazione gaussiana in \LanguageName{octave}.}
\end{listing}
\begin{Theorem}
  Con le notazioni della definizione precedente, l'eliminazione di Gauss restituisce la stessa
  matrice triangolare superiore $U$ dell'algoritmo del teorema \ref{th_FattorizzazioneLUP}.
\end{Theorem}
\Proof Segue immediatamente dalle definizioni degli algoritmi. \EndProof
