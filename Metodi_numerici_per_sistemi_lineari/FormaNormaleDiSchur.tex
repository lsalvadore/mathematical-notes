\section{Forma normale di Schur}
\label{MetodiNumericiPerSistemiLineari_FormaNormaleDiSchur}
\begin{Definition}
  Siano
  \begin{itemize}
    \item $\Matrix \in \mathbb{C}^{n \times n}$;
    \item $\UnitaryMatrix \in \mathbb{C}^{n \times n}$ unitaria;
    \item $\SchurNormalForm \in \mathbb{C}^{n \times n}$ triangolare
      superiore;
  \end{itemize}
  tali che $\HermitianTransposed{\UnitaryMatrix} \Matrix \UnitaryMatrix = \SchurNormalForm$.
  Si dice che $\SchurNormalForm$ \`e una
  \Define{forma normale di Schur}[normale di Schur][forma] di $\Matrix$.
\end{Definition}
\begin{Theorem}
  Sia $\Matrix \in \mathbb{C}^{n \times n}$. Il seguente algoritmo restituisce una forma normale
  di Schur $\SchurNormalForm$ di $\Matrix$ e una matrice unitaria $\UnitaryMatrix$ tale che
  $\HermitianTransposed{\UnitaryMatrix} \Matrix \UnitaryMatrix = \SchurNormalForm$:
  \begin{enumerate}
    \item\label{Schur_1} se $n = 1$ l'algoritmo termina restituendo $\SchurNormalForm = \Matrix$
      e $\UnitaryMatrix = \Identity$;
    \item\label{Schur_2} si fissino un autovalore $\Eigenvalue \in \Spectrum{\Matrix}$ e un
      autovettore $\Eigenvector$ di $\Matrix$ relativo ad $\Eigenvalue$;
    \item\label{Schur_3} sia $\HouseholderElementaryMatrix$ una matrice elementare di
      Householder tale che $\HouseholderElementaryMatrix \Eigenvector$
    \item\label{Schur_4} definiamo
      $\VarMatrix = \HermitianTransposed{\HouseholderElementaryMatrix} \Matrix
                    \HouseholderElementaryMatrix$;
    \item\label{Schur_5} applichiamo l'intero algoritmo alla sottomatrice
      $(\VarMatrix_k^j)_{\substack{k = 1, ..., n - 1}{j = 1, ..., n -1}}$ ottenendo
      la matrice triangolare superiore $\VarSchurNormalForm$ e la matrice unitaria
      $\VarUnitaryMatrix$;
    \item\label{Schur_6} definiamo
      \[
        \SchurNormalForm =
        \lmatrix
        \begin{array}{c|c}
          \Eigenvalue & (\VarMatrix_0^j)_{j = 1, ..., n - 1} \VarUnitaryMatrix\\
          \hline
            & \VarSchurNormalForm
        \end{array}
        \rmatrix
      \]
    \item\label{Schur_7} definiamo
      \[
        \UnitaryMatrix =
        \lmatrix
        \begin{array}{c|c}
          1 &\\
          \hline
            & \VarUnitaryMatrix
        \end{array}
        \rmatrix
      \]
    \item\label{Schur_8} l'algoritmo termina restituendo $\SchurNormalForm$ e
      $\VarUnitaryMatrix$.
  \end{enumerate}
\end{Theorem}
\begin{listing}
	\insertcode{octave}{"Metodi_numerici_per_sistemi_lineari/Schur.m"}
	\caption{Implementazione di un algoritmo in \LanguageName{octave} per il calcolo di una
  forma normale di Schur.}
\end{listing}
