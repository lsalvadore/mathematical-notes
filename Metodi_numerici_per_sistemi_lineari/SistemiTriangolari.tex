\section{Sistemi Triangolari.}
\label{MetodiNumericiPerSistemiLineari_SistemiTriangolari}
\begin{Definition}
  Sia il sistema
  $\Matrix X = B$ con
  \begin{itemize}
    \item $n, m \in \NotZero{\mathbb{N}^2}$;
    \item $\Matrix \in \mathbb{C}^{n \times m}$ a gradini;
    \item $X \in \mathbb{C}^m$ incognito;
    \item $B \in \mathbb{C}^n$ noto.
  \end{itemize}
  Si chiama
  \Define{sostituzione all'indietro}[all'indietro][sostituzione]
  l'algoritmo seguente:
  \begin{enumerate}
    \item\label{GradiniSuperiore_0} fissiamo arbitrariamente
      $X \in \mathbb{C}^m$;
    \item\label{GradiniSuperiore_1} poniamo $k = n - 1$;
    \item\label{GradiniSuperiore_2} poniamo $q$ uguale al minimo indice
      $r$ tale che $\Matrix_k^r \neq 0$ e
      $X_q = (T_k^q)^{-1}B_k - \sum_{r = q + 1}^{n - 1} T_k^r X_r$;
    \item\label{GradiniSuperiore_3} decrementiamo $k$ di $1$;
    \item\label{GradiniSuperiore_4} se $k \geq 0$ torniamo al punto
      \ref{GradiniSuperiore_1}, altrimenti l'algoritmo termina.
  \end{enumerate}
\end{Definition}
\begin{listing}
	\insertcode{octave}{"Metodi_numerici_per_sistemi_lineari/SostituzioneAllIndietro.m"}
	\caption{Implementazione della sostituzione all'indietro in \LanguageName{octave}.}
\end{listing}
\begin{Theorem}
  Con le notazioni della definizione precedente, la sostituzione
  all'indietro risolve il sistema $\Matrix X = B$
  L'algoritmo costa $\BigO{n^2}$ operazioni aritmetiche.
\end{Theorem}
\Proof Verifichiamo la correttezza dell'algoritmo per induzione. Il caso $n = 1$
\`e immediato. Per $n > 1$, basta osservare che i primi $n - 1$ cicli
dell'algoritmo sono gli stessi che risolvono il sistema proiettato sulle ultime
$n - 1$ componenti e che l'ultimo ciclo risolve con successo l'equazione
definita dalla prima riga del sistema.
\par Abbiamo, ad ogni ciclo,
\begin{itemize}
  \item $n - 1 - (q + 1) + 1 + 1 = n - q$ somme;
  \item $n - 1 - (q + 1) + 1 = n - q - 1$ moltiplicazioni;
  \item $1$ divisione.
\end{itemize}
\par Ad ogni ciclo, $q$ \`e al pi\`u $k$; quindi in totale le operazioni
sono al pi\`u
\begin{align*}
  \sum_{k = 0}^{n - 1} 2n - 2k
  &= 2n^2 - 2 \frac{(n - 1)n}{2},\\
  &= 2n^2 - n^2 + n,\\
  &= n^2 + n,\\
  &= \BigO{n^2}.\text{ \EndProof}
\end{align*}
