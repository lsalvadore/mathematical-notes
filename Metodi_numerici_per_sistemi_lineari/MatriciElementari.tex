\section{Matrici elementari.}
\label{MetodiNumericiPerLaRisoluzioneDiSistemiLineari_MatriciElementari}
\begin{Definition}
  Chiamiamo
  \Define{matrice elementare}[elementare][matrice]
  una matrice
  $\ElementaryMatrix \in \mathbb{C}^{n \times n}$ ($n \in \NotZero{\mathbb{N}}$)
  tale che
  $\ElementaryMatrix
  = \Identity - \Scalar\Vector\HermitianTransposed{\VarVector}$, con
  \begin{itemize}
    \item $\Scalar \in \mathbb{C}^{n \times n}$;
    \item $\Vector, \VarVector \in \mathbb{C}^{n \times n}$.
  \end{itemize}
\end{Definition}
\begin{Theorem}
  Con le notazioni della definizione precedente, $\ElementaryMatrix$ \`e
  invertibile se e solo se
  $\Scalar\HermitianTransposed{\VarVector}\Vector \neq 1$.
  Quando esistente, $\ElementaryMatrix^{-1}$ \`e una matrice elementare data da
  $\ElementaryMatrix^{-1}
  = \Identity - \VarScalar\Vector\HermitianTransposed{\VarVector}$, con
  $\VarScalar
  = - \frac{\sigma}{\sigma\HermitianTransposed{\VarVector}\Vector - 1}$.
\end{Theorem}
\Proof Supponiamo $\ElementaryMatrix$ invertibile. Abbiamo
\begin{align*}
  \MatrixElement\Vector
  &= \Vector - \Scalar\Vector\HermitianTransposed{\VarVector}\Vector,\\
  &= (1 - \Scalar\HermitianTransposed{\VarVector}\Vector)\Vector.
\end{align*}
Dunque deve essere $1 - \Scalar\HermitianTransposed{\VarVector}\Vector \neq 0$.
\par Supponiamo ora invece al contrario $\ElementaryMatrix$ non invertibile.
Allora esiste $x \in \NotZero{\mathbb{C}^n}$ tale che $\ElementaryMatrix x = 0$,
vale a dire
$x - \Scalar\Vector\HermitianTransposed{\VarVector} x
(1 - \Scalar\Vector\HermitianTransposed{\VarVector}) x
= 0$
e dunque
$\Scalar\Vector\HermitianTransposed{\VarVector} = 1$.
\par La parte restante del teorema si dimostra facilmente per calcolo diretto.
\EndProof
\begin{Lemma}
  \label{MetodiNumericiPerLaRisoluzioneDiSistemiLineari_LemmaRango1}
  Siano
  \begin{itemize}
    \item $n \in \NotZero{\mathbb{N}}$;
    \item $\Matrix \in \mathbb{C}^{n \times n}$;
    \item $\Vector, \VarVector \in \mathbb{C}^n$.
  \end{itemize}
  Se
  $\Matrix = \Vector\HermitianTransposed{\VarVector}$,
  allora
  \begin{itemize}
    \item $\MatrixRank{\Matrix} = 1$;
    \item $\Norm{\Matrix}[2] = \Norm{\Vector}[2] \Norm{\VarVector}[2]$.
  \end{itemize}
\end{Lemma}
\Proof Poich\'e
$\Matrix = \Vector\HermitianTransposed{\VarVector}$,
per ogni $k \in \NotZero{n}$ abbiamo la seguente relazione sulle colonne di
$\Matrix$:
$\Matrix^k = \frac{\VarVector_k}{\VarVector_0} \Matrix^0$.
Dunque $\MatrixRank{\Matrix} = 1$.
\par Siano ora $\UnitaryMatrix$ e $\VarUnitaryMatrix$ matrici unitarie le cui
prime colonne sono
$\UnitaryMatrix^0 = \frac{\Vector}{\Norm{\Vector}}$
e
$\VarUnitaryMatrix^0 = \frac{\VarVector}{\Norm{\VarVector}}$
rispettivamente.
\par Abbiamo
$\Norm{\UnitaryMatrix^{-1}\Matrix\VarUnitaryMatrix}[2]
= \Norm{\Matrix}[2]$
e
$\UnitaryMatrix^{-1}\Matrix\VarUnitaryMatrix$ \`e una matrice interamente
nulla tranne che nell'elemento della prima riga e della prima colonna
che \`e uguale a
$\Norm{\Vector}[2] \Norm{\VarVector}[2]$,
da cui
$\Norm{\Matrix} =
\Norm{\Vector}[2] \Norm{\VarVector}[2]$. \EndProof
\begin{Theorem}
  La risoluzione del sistema
  $\ElementaryMatrix X = B$ con
  \begin{itemize}
    \item $n \in \NotZero{\mathbb{N}}$;
    \item $\ElementaryMatrix \in \mathbb{C}^{n \times n}$ elementare;
    \item $X \in \mathbb{C}^n$ incognito;
    \item $B \in \mathbb{C}^n$ noto;
  \end{itemize}
  pu\`o essere effettuata con
  \begin{itemize}
    \item $3n$ addizioni;
    \item $3n + 1$ moltiplicazioni;
    \item $1$ divisione;
  \end{itemize}
  per un totale di $\BigO{n}$ operazioni aritmetiche.
\end{Theorem}
\begin{Corollary}
  La risoluzione del sistema
  $\Matrix X = B$ con
  \begin{itemize}
    \item $n \in \NotZero{\mathbb{N}}$;
    \item $\VarMatrix \in \mathbb{C}^{n \times n}$ prodotto di matrici
      elementari invertibili;
    \item $\TriangularMatrix \in \mathbb{C}^{n \times n}$ trriangolare
      superiore.
    \item $\Matrix = \VarMatrix\TriangularMatrix$;
    \item $X \in \mathbb{C}^n$ incognito;
    \item $B \in \mathbb{C}^n$ noto;
  \end{itemize}
  costa $\BigO{n^2}$ operazioni aritmetiche.
\end{Corollary}
\Proof Si risolve prima il sistema $\VarMatrix X = B$ ottenendo una soluzione
$Y$ con $\BigO{n}$ operazioni aritmetiche; dunque si risolve il sistema
$\TriangularMatrix X = Y$ con $\BigO{n^2}$ operazioni aritmetiche,
ottenendo la soluzione del sistema originale con un totale di $\BigO{n^2}$
operazioni aritmetiche. \EndProof
\begin{Theorem}
  Siano
  \begin{itemize}
    \item $n \in \NotZero{\mathbb{N}}$;
    \item $\VarVarVector, \VarVarVarVector \in \NotZero{\mathbb{C}^n}$.
  \end{itemize}
  Esiste una matrice elementare non singolare $\ElementaryMatrix$ tale che
  $\ElementaryMatrix\VarVarVector = \VarVarVarVector$.
\end{Theorem}
\Proof Siano
\begin{itemize}
  \item $\Vector = \VarVarVector - \VarVarVarVector$;
  \item $\VarVector \in \mathbb{C}^n$ tale che esso non sia ortogonale n\'e a
    $\VarVarVector$ n\'e a $\VarVarVarVector$
    \footnote{Una tale scelta \`e sempre possibile:
    \begin{itemize}
      \item se $\HermitianTransposed{\VarVarVector}{\VarVarVarVector} = 0$,
        allora si pu\`o scegliere
        $\VarVector = \VarVarVector + \VarVarVarVector$;
      \item se $\HermitianTransposed{\VarVarVector}{\VarVarVarVector} \neq 0$,
        allora si pu\`o scegliere
        $\VarVector = \VarVarVector$.
    \end{itemize}};
  \item $\Scalar = (\HermitianTransposed{\VarVector}\VarVarVector)^{-1}$.
\end{itemize}
\par Poniamo
$\ElementaryMatrix
= \Identity - \Scalar\Vector\HermitianTransposed{\VarVector}$.
\par Abbiamo
\begin{align*}
  \ElementaryMatrix \VarVarVector
  &= \VarVarVector
    - (\HermitianTransposed{\VarVector}\VarVarVector)^{-1}
    (\VarVarVector - \VarVarVarVector)
    \HermitianTransposed{\VarVector}\VarVarVector,\\
  &= \VarVarVector
    - (\HermitianTransposed{\VarVector}\VarVarVector)^{-1}
    \VarVarVector
    \HermitianTransposed{\VarVector}\VarVarVector
    + (\HermitianTransposed{\VarVector}\VarVarVector)^{-1}
    \VarVarVarVector
    \HermitianTransposed{\VarVector}\VarVarVector,\\
  &= \VarVarVarVector.
\end{align*}
\par Inoltre,
\begin{align*}
  \Scalar\HermitianTransposed{\VarVector}\Vector
  &= (\HermitianTransposed{\VarVector}\VarVarVector)^{-1}
    \HermitianTransposed{\VarVector}
    (\VarVarVector - \VarVarVarVector),\\
  &= (\HermitianTransposed{\VarVector}\VarVarVector)^{-1}
    \HermitianTransposed{\VarVector}
    \VarVarVector
    - (\HermitianTransposed{\VarVector}\VarVarVector)^{-1}
    \HermitianTransposed{\VarVector}
    \VarVarVarVector,\\
  &= 1
    - (\HermitianTransposed{\VarVector}\VarVarVector)^{-1}
    \HermitianTransposed{\VarVector}
    \VarVarVarVector,\\
  &\neq 1,
\end{align*}
da cui l'invertibilit\`a di $\ElementaryMatrix$. \EndProof
\begin{Theorem}
  \label{th_FattorizzazioneMatriciElementari}
  Siano
  \begin{itemize}
    \item $n, m \in \NotZero{\mathbb{N}}$;
    \item $\Matrix \in \mathbb{C}^{n \times m}$.
  \end{itemize}
  Il seguente algoritmo costruisce una fattorizzazione
  $\Matrix = A \TriangularMatrix$ dove $A$ \`e prodotto di matrici
  elementari e $\TriangularMatrix$ \`e triangolare superiore:
  \begin{enumerate}
    \item\label{FattorizzazioneMatriciElementari_1} si ponga $l = 0$;
    \item\label{FattorizzazioneMatriciElementari_2} si definisca
      $\VarMatrix = (\Matrix_k^j)_{\substack{k = l\\j= l}}^{\substack{k = n - 1}{j = m - 1}}$;
    \item\label{FattorizzazioneMatriciElementari_3} si fissi una matrice
      elementare non singolare $\ElementaryMatrix_{(l,0)}$ tale che
      $\ElementaryMatrix_{(l,0)}\VarMatrix = \CanonicalBase_1^{(l)}$, dove
      $\CanonicalBase_1^{(l)}$ \`e il $l$-esimo vettore della base
      canonica di $\mathbb{C}^{m - l}$;
    \item\label{FattorizzazioneMatriciElementari_4} si ponga
      $(\ElementaryMatrix_{(l)})_k^j =
      \begin{cases}
        1&\text{ se }k = j < l,\\
        (\ElementaryMatrix_{(l,0)})_{k - l}^{j - l}&\text{ se }\And{k \geq l}{j \geq l},\\
        0&\text{ altrimenti},
      \end{cases}$
      vale a dire
      \[
        \ElementaryMatrix_{(l)} =
        \lmatrix
        \begin{array}{c|c}
          \Identity &\\
          \hline
          & \ElementaryMatrix_{(l,0)}
        \end{array}
        \rmatrix;
      \]
    \item\label{FattorizzazioneMatriciElementari_5} si ridefinisca
      $\Matrix = \ElementaryMatrix_{(l)}\Matrix$;
    \item\label{FattorizzazioneMatriciElementari_6} si incrementi
      $l$ di $1$;
    \item\label{FattorizzazioneMatriciElementari_7} se $l < m$, si torni al
      punto \ref{FattorizzazioneMatriciElementari_2};
    \item\label{FattorizzazioneMatriciElementari_8} si ponga
      $A = \ElementaryMatrix_{(m - 1)}^{-1}\ElementaryMatrix_{(m - 2)}^{-1}
      \cdots
      \ElementaryMatrix_{(1)}^{-1}\ElementaryMatrix_{(0)}^{-1}$;
    \item\label{FattorizzazioneMatriciElementari_9} si ponga
      $\TriangularMatrix = \Matrix$;
    \item\label{FattorizzazioneMatriciElementari_10} l'algoritmo termina restituendo
      $A$ e $\TriangularMatrix$.
  \end{enumerate}
\end{Theorem}
\Proof Si dimostra facilmente per induzione che $\TriangularMatrix$ \`e triangolare.
\par Per dimstrare che $A$ \`e prodotto di matrici elementari \`e sufficiente notare che
ogni $\ElementaryMatrix_{(l)}$ al variare di $l \in m$ \`e elementare. In effetti,
$\ElementaryMatrix_{(l,0)}$ \`e elementare per costruzione: esistono dunque uno scalare
$\Scalar \in \mathbb{C}$ e vettori $\Vector,\VarVector \in \mathbb{C}^{m - l}$  tali che
$\ElementaryMatrix_{(l,0)} = \Identity - \Scalar\Vector\VarVector$.
Per calcolo diretto si verifica che
$\ElementaryMatrix_{(l,0)} = \Identity - \Scalar\tilde{\Vector}\tilde{\VarVector}$, dove
\begin{itemize}
  \item la matrice $\Identity$ \`e stavolta l'identit\`a di $\mathbb{C}^m$;
  \item $\tilde{\Vector} \in \mathbb{C}^m$ \`e tale che
    $\tilde{\Vector}_k =
    \begin{cases}
      0&\text{ se }k < l,\\
      \Vector_{k - l}&\text{ altrimenti};
    \end{cases}$
  \item $\tilde{\VarVector} \in \mathbb{C}^m$ \`e tale che
    $\tilde{\Vector}_k =
    \begin{cases}
      0&\text{ se }k < l,\\
      \VarVector_{k - l}&\text{ altrimenti}.\text{ \EndProof}
    \end{cases}$
\end{itemize}
