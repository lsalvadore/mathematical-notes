\section{Forma di Hessenberg.}
\label{MetodiNumericiPerSistemiLineari_FormaDiHessenberg}
\begin{Definition}
	Siano $\Ring$ un anello, $n, m \in \mathbb{N}$ e $\Matrix \in \Ring^{n \times m}$.
  Diciamo che $\Matrix$ \`e
	\begin{itemize}
		\item \Define{matrice di Hessenberg superiore}[di Hessenberg superiore][matrice] quando
      $\ForAll{(i,j) \in n \times m}{\Implies{i > j + 1}{\Matrix_i^j = 0}}$:
      nel caso quadrato abbiamo
\[
\lmatrix
\begin{array}{cccc}
\Matrix_0^0 & \cdots & \cdots & \Matrix_{n - 1}^{n - 1}\\
\Matrix_1^0 & \Matrix_1^1 & \ddots & \Matrix_{n - 1}^{n - 1}\\
& \ddots & \ddots & \vdots\\
& & \Matrix_{n - 1}^{n - 2} & \Matrix_{n - 1}^{n - 1}
\end{array}
\rmatrix;
\]
		\item \Define{matrice di Hessenberg inferiore}[di Hessenberg inferiore][matrice] quando
      $\ForAll{(i,j) \in n \times m}{\Implies{i < j - 1}{\Matrix_i^j = 0}}$:
      nel caso quadrato abbiamo
\[
\lmatrix
\begin{array}{cccc}
\Matrix_0^0 & \Matrix_0^1 &\\
\vdots & \ddots & \ddots &\\
\vdots & \ddots & \Matrix_{n - 2}^{n - 2} & \Matrix_{n - 2}^{n - 1}\\
\Matrix_{n - 1}^0 & \cdots & \Matrix_{n - 1}^{n - 2} & \Matrix_{n - 1}^{n - 1}\\
\end{array}
\rmatrix.
\]
  \end{itemize}
\end{Definition}
\begin{Definition}
  Siano
  \begin{itemize}
    \item $\Matrix \in \mathbb{C}^{n \times n}$;
    \item $\UnitaryMatrix \in \mathbb{C}^{n \times n}$ unitaria;
    \item $\HessenbergForm \in \mathbb{C}^{n \times n}$ matrice di Hessenberg
      superiore;
  \end{itemize}
  tali che $\HermitianTransposed{\UnitaryMatrix} \Matrix \UnitaryMatrix = \HessenbergForm$.
  Si dice che $\HessenbergForm$ \`e una
  \Define{forma di Hessenberg}[di Hessenberg][forma] di $\Matrix$.
\end{Definition}
\begin{Theorem}
  Siano
  \begin{itemize}
    \item $n, m \in \NotZero{\mathbb{N}}$;
    \item $\Matrix \in \mathbb{C}^{n \times m}$.
  \end{itemize}
  Il seguente algoritmo costruisce una fattorizzazione
  $\Matrix = A \HessenbergForm$ dove $A$ \`e prodotto di matrici
  elementari e $\HessenbergForm$ \`e triangolare superiore:
  \begin{enumerate}
    \item\label{FormaDiHessenberg_1} si ponga $l = 0$;
    \item\label{FormaDiHessenberg_2} si definisca
      $\VarMatrix = (\Matrix_k^j)_{\substack{k = l\\j= l}}^{\substack{k = n - 1}{j = m - 1}}$;
    \item\label{FormaDiHessenberg_3} si fissi una matrice
      elementare di Householder $\HouseholderElementaryMatrix_{(l,0)}$ tale che
      $\HouseholderElementaryMatrix_{(l,0)}\VarMatrix = \CanonicalBase_1^{(l + 1)}$, dove
      $\CanonicalBase_1^{(l + 1)}$ \`e il $(l + 1)$-esimo vettore della base
      canonica di $\mathbb{C}^{m - l - 1}$;
    \item\label{FormaDiHessenberg_4} si ponga
      $(\HouseholderElementaryMatrix_{(l)})_k^j =
      \begin{cases}
        1&\text{ se }k = j < l,\\
        (\HouseholderElementaryMatrix_{(l,0)})_{k - l}^{j - l}&\text{ se }\And{k \geq l}{j \geq l},\\
        0&\text{ altrimenti},
      \end{cases}$
      vale a dire
      \[
        \HouseholderElementaryMatrix_{(l)} =
        \lmatrix
        \begin{array}{c|c}
          \Identity &\\
          \hline
          & \HouseholderElementaryMatrix_{(l,0)}
        \end{array}
        \rmatrix;
      \]
    \item\label{FormaDiHessenberg_5} si ridefinisca
      $\Matrix = \HouseholderElementaryMatrix_{(l)}\Matrix\HouseholderElementaryMatrix_{(l)}$;
    \item\label{FormaDiHessenberg_6} si incrementi
      $l$ di $1$;
    \item\label{FormaDiHessenberg_7} se $l < m - 1$, si torni al
      punto \ref{FormaDiHessenberg_2};
    \item\label{FormaDiHessenberg_8} si ponga
      $A = \HouseholderElementaryMatrix_{(m - 2)}^{-1}\HouseholderElementaryMatrix_{(m - 3)}^{-1}
      \cdots
      \HouseholderElementaryMatrix_{(1)}^{-1}\HouseholderElementaryMatrix_{(0)}^{-1}$;
    \item\label{FormaDiHessenberg_9} si ponga
      $\HessenbergForm = \Matrix$;
    \item\label{FormaDiHessenberg_10} l'algoritmo termina restituendo
      $A$ e $\HessenbergForm$.
  \end{enumerate}
\end{Theorem}
\Proof 
\begin{listing}
	\insertcode{octave}{"Metodi_numerici_per_sistemi_lineari/HessenbergViaHouseholder.m"}
	\caption{Implementazione di un algoritmo in \LanguageName{octave} per il calcolo di una
  forma di Hessenberg tramite matrici elementari di Householder.}
\end{listing}
\begin{Definition}
  Chiamiamo
  \Define{rotazione di Givens}[di Givens][rotazione]
  una matrice unitaria
  $\GivensRotation \in \mathbb{C}^{n \times n}$
  ($n \in \NotZero{\mathbb{N}}$) tale che, per opportuni
  \begin{itemize}
    \item $p \in \NotZero{n}$;
    \item $c, s \in \mathbb{C}$;
  \end{itemize}
  \[
    \GivensRotation_k^j =
    \begin{cases}
      1&\text{ se }\And{k = j}{\And{k \neq p}{k \neq p + 1}},\\
      c&\text{ se }k = j = p,\\
      s&\text{ se }\And{k = p}{j = p + 1},\\
      - \Conjugate{s}&\text{ se }\And{k = p + 1}{j = p},\\
      \Conjugate{c}&\text{ se }k = j = p + 1,\\
      0&\text{ altrimenti},
    \end{cases}
  \]
  vale a dire
  \[
    \GivensRotation =
    \lmatrix
    \begin{array}{cccccccccc}
      1 &&&&&&&\\
      & \ddots &&&&&&\\
      && 1 &&&&&\\
      &&& c & s &&&\\
      &&& - \Conjugate{s} & \Conjugate{c} &&&\\
      &&&&& 1 &&\\
      &&&&&& \ddots &\\
      &&&&&&& 1
    \end{array}
    \rmatrix.
  \]
\end{Definition}
\begin{listing}
	\insertcode{octave}{"Metodi_numerici_per_sistemi_lineari/HessenbergViaGivens.m"}
	\caption{Implementazione di un algoritmo in \LanguageName{octave} per il calcolo di una
  forma di Hessenberg tramite rotazioni di Givens.}
\end{listing}
\begin{listing}
	\insertcode{octave}{"Metodi_numerici_per_sistemi_lineari/BulgeChasing.m"}
	\caption{Implementazione del \English{bulge chasing} in \LanguageName{octave}.}
\end{listing}
