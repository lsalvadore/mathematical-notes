\subsection{Metodo di Richardson.}
\label{MetodiNumericiPerSistemiLineari_MetodoDiRichardson}
\begin{Definition}
  Con le notazioni del teorema \ref{th_MetodiIterativiPerSistemiLineari},
  fissato $\alpha \in \mathbb{C}$
  chiamiamo
  \Define{metodo di Richardson}[di Richardson][metodo]
  il metodo iterativo definito da
  \begin{itemize}
    \item $A = \alpha^{-1} \Identity$;
    \item $B = \alpha^{-1} \Identity - \Matrix$.
  \end{itemize}
\end{Definition}
\begin{Theorem}
  Con le notazioni della definizione precedente abbiamo, per ogni $k \in \mathbb{N}$,
  $X_{k + 1} = X_k - \alpha(\Matrix X_k - C)$.
\end{Theorem}
\Proof Abbiamo
\begin{align*}
  X_{k + 1}
  &= \IterationMatrix X_k + V,\\
  &= A^{-1} B X_k + A^{-1} C,\\
  &= \alpha \Identity (\alpha^{-1} \Identity - \Matrix) X_k
    + \alpha \Identity C,\\
  &= (\Identity - \alpha \Matrix) X_k + \alpha \Identity C,\\
  &= X_k - \alpha(\Matrix X_k - C).
\end{align*}
\begin{Theorem}
  Con le notazioni della definizione precendente, se $\Matrix$ \`e definita
  positiva il metodo di Richardson converge per $\alpha = \Norm{\Matrix}^{-1}$,
  per qualsiasi norma indotta $\Norm{\cdot}$.
\end{Theorem}
\begin{listing}
	\insertcode{octave}{"Metodi_numerici_per_sistemi_lineari/MetodoDiRichardson.m"}
	\caption{Implementazione del metodo di Richardson in linguaggio \LanguageName{octave}.}
\end{listing}
