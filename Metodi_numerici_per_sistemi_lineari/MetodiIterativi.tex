\section{Metodi iterativi.}
\label{MetodiNumericiPerSistemiLineari_MetodiIterativi}
\begin{Theorem}
  \label{th_MetodiIterativiPerSistemiLineari}
  Siano
  \begin{itemize}
    \item $n \in \NotZero{\mathbb{N}}$;
    \item $\Matrix \in \mathbb{C}^{n \times n}$;
    \item $C \in \mathbb{C}^n$;
    \item $A, B \in \mathbb{C}^{n \times n}$ tali che
      \begin{itemize}
        \item $A$ \`e invertibile;
        \item $\Matrix = A - B$;
      \end{itemize}
    \item $X_0 \in \mathbb{C}$.
  \end{itemize}
  Poniamo
  \begin{itemize}
    \item $\IterationMatrix = A^{-1}B$;
    \item $V = A^{-1} C$.
  \end{itemize}
  La successione $(X_k)_{k \in \mathbb{N}}$ definita, per $k > 0$, da
  $X_k = \IterationMatrix X_{k - 1} + V$, quando convergente,
  converge a $\tilde{X}$ soluzione di $\Matrix X = C$.
\end{Theorem}
\Proof Abbiamo
\begin{align*}
  X_k
  &= \IterationMatrix X_{k - 1} + V,\\
  &= A^{-1}B X_{k - 1} + A^{-1} C,
\end{align*}
equivalente a
\[
  A X_k = B X_{k - 1} + C,
\]
e dunque a
\[
  A X_k - B X_{k - 1} = C.
\]
\par Al limite abbiamo
\[
  A \tilde{X} - B \tilde{X} = C,
\]
equivalente a
\[
  \Matrix \tilde{X} = C.\text{ \EndProof}
\]
\begin{Definition}
  Con le notazioni del teorema precedente, definiamo
  \Define{metodo iterativo stazionario}[iterativo stazionario per sistemi lineari][metodo]
  un algoritmo di risoluzione del sistema $\Matrix X = C$ che consiste
  nel calcolare ricorsivamente un numero finito di  elementi della
  successione $(X_k)_{k \in \mathbb{N}}$ sino all'elemento di indice $N$ e
  restituisce $X_N$ come approssimazione di una soluzione.
  Inoltre, il metodo si dice
  \Define{convergente}[di un metodo iterativo stazionario][convergenza]
  a $\tilde{X}$
  se \`e convergente a $\tilde{X}$ la successione
  $(X_k)_{k \in \mathbb{N}}$.
  Infine, chiamiamo
  \Define{matrice di iterazione}[di iterazione][matrice]
  la matrice $\IterationMatrix$.
\end{Definition}
\begin{Theorem}
  Con le notazioni del teorema \ref{th_MetodiIterativiPersistemiLineari},
  fissata una norma di matrice $\Norm{\cdot}$ su
  $\mathbb{C}^{n \times n}$, $\Norm{\IterationMatrix} < 1$ se e solo
  se
  \begin{itemize}
    \item il sistema $\Matrix X = C$ ammette un'unica soluzione $\tilde{X}$;
    \item il metodo iterativo converge a $\tilde{X}$.
  \end{itemize}
\end{Theorem}
\subsection{Metodo di Richardson.}
\label{MetodiNumericiPerSistemiLineari_MetodoDiRichardson}
\begin{Definition}
  Con le notazioni del teorema \ref{th_MetodiIterativiPerSistemiLineari},
  fissato $\alpha \in \mathbb{C}$
  chiamiamo
  \Define{metodo di Richardson}[di Richardson][metodo]
  il metodo iterativo definito da
  \begin{itemize}
    \item $A = \alpha^{-1} \Identity$;
    \item $B = \alpha^{-1} \Identity - \Matrix$.
  \end{itemize}
\end{Definition}
\begin{Theorem}
  Con le notazioni della definizione precedente abbiamo, per ogni $k \in \mathbb{N}$,
  $X_{k + 1} = X_k - \alpha(\Matrix X_k - C)$.
\end{Theorem}
\Proof Abbiamo
\begin{align*}
  X_{k + 1}
  &= \IterationMatrix X_k + V,\\
  &= A^{-1} B X_k + A^{-1} C,\\
  &= \alpha \Identity (\alpha^{-1} \Identity - \Matrix) X_k
    + \alpha \Identity C,\\
  &= (\Identity - \alpha \Matrix) X_k + \alpha \Identity C,\\
  &= X_k - \alpha(\Matrix X_k - C).
\end{align*}
\begin{Theorem}
  Con le notazioni della definizione precendente, se $\Matrix$ \`e definita
  positiva il metodo di Richardson converge per $\alpha = \Norm{\Matrix}^{-1}$,
  per qualsiasi norma indotta $\Norm{\cdot}$.
\end{Theorem}
\begin{listing}
	\insertcode{octave}{"Metodi_numerici_per_sistemi_lineari/MetodoDiRichardson.m"}
	\caption{Implementazione del metodo di Richardson in linguaggio \LanguageName{octave}.}
\end{listing}

\subsection{Metodo di Jacobi.}
\label{MetodiNumericiPerSistemiLineari_MetodoDiJacobi}
\begin{Definition}
  Con le notazioni del teorema \ref{th_MetodiIterativiPerSistemiLineari},
  posto
  \begin{itemize}
    \item $\TriangularMatrix$ la matrice triangolare superiore tale che,
      per $(k,j) \in n \times n$,
      $\TriangularMatrix =
        \begin{cases}
          - \Matrix_k^j\text{ se }k < j,\\
          0\text{ altrimenti}.
        \end{cases}$;
    \item $\DiagonalMatrix$ la matrice triangolare superiore tale che,
      per $(k,j) \in n \times n$,
      $\DiagonalMatrix =
        \begin{cases}
          \Matrix_k^j\text{ se }k = j,\\
          0\text{ altrimenti}.
        \end{cases}$;
    \item $\VarTriangularMatrix$ la matrice triangolare inferiore tale che,
      per $(k,j) \in n \times n$,
      $\VarTriangularMatrix =
        \begin{cases}
          - \Matrix_k^j\text{ se }k > j,\\
          0\text{ altrimenti}.
        \end{cases}$;
  \end{itemize}
  chiamiamo
  \Define{metodo di Jacobi}[di Jacobi][metodo]
  il metodo iterativo definito da
  \begin{itemize}
    \item $A = \DiagonalMatrix$;
    \item $B = \TriangularMatrix + \VarTriangularMatrix$.
  \end{itemize}
\end{Definition}
\begin{listing}
	\insertcode{octave}{"Metodi_numerici_per_sistemi_lineari/MetodoDiJacobi.m"}
	\caption{Implementazione del metodo di Jacobi in linguaggio \LanguageName{octave}.}
\end{listing}
\begin{Theorem}
  Con le notazioni della definizione precedente,
  se \`e verificata una delle proposizioni seguente, allora
  il metodo di Jacobi \`e convergente:
  \begin{itemize}
    \item $\Matrix$ \`e fortemente dominante diagonale;
    \item $\HermitianTransposed{\Matrix}$ \`e fortemente dominante diagonale;
    \item $\Matrix$ \`e irriducibilmente dominante diagonale;
    \item $\HermitianTransposed{\Matrix}$ \`e irriducibilmente dominante diagonale.
  \end{itemize}
\end{Theorem}

\subsection{Metodo di Gauss-Seidel.}
\label{MetodiNumericiPerSistemiLineari_MetodoDiGaussSeidel}
\begin{Definition}
  Con le notazioni del teorema \ref{th_MetodiIterativiPerSistemiLineari},
  posto
  \begin{itemize}
    \item $\TriangularMatrix$ la matrice triangolare superiore tale che,
      per $(k,j) \in n \times n$,
      $\TriangularMatrix =
        \begin{cases}
          - \Matrix_k^j\text{ se }k < j,\\
          0\text{ altrimenti}.
        \end{cases}$;
    \item $\DiagonalMatrix$ la matrice triangolare superiore tale che,
      per $(k,j) \in n \times n$,
      $\DiagonalMatrix =
        \begin{cases}
          \Matrix_k^j\text{ se }k = j,\\
          0\text{ altrimenti}.
        \end{cases}$;
    \item $\VarTriangularMatrix$ la matrice triangolare inferiore tale che,
      per $(k,j) \in n \times n$,
      $\VarTriangularMatrix =
        \begin{cases}
          - \Matrix_k^j\text{ se }k > j,\\
          0\text{ altrimenti}.
        \end{cases}$;
  \end{itemize}
  chiamiamo
  \Define{metodo di Jacobi}[di Jacobi][metodo]
  il metodo iterativo definito da
  \begin{itemize}
    \item $A = \DiagonalMatrix - \TriangularMatrix$;
    \item $B = \VarTriangularMatrix$.
  \end{itemize}
\end{Definition}
\begin{listing}
	\insertcode{octave}{"Metodi_numerici_per_sistemi_lineari/MetodoDiGaussSeidel.m"}
	\caption{Implementazione del metodo di Gauss-Seidel in linguaggio \LanguageName{octave}.}
\end{listing}
\begin{Theorem}
  Con le notazioni della definizione precedente,
  se \`e verificata una delle proposizioni seguente, allora
  il metodo di Jacobi \`e convergente:
  \begin{itemize}
    \item $\Matrix$ \`e fortemente dominante diagonale;
    \item $\HermitianTransposed{\Matrix}$ \`e fortemente dominante diagonale;
    \item $\Matrix$ \`e irriducibilmente dominante diagonale;
    \item $\HermitianTransposed{\Matrix}$ \`e irriducibilmente dominante diagonale.
  \end{itemize}
\end{Theorem}

