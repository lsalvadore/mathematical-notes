\subsection{Metodo di Gauss-Seidel.}
\label{MetodiNumericiPerSistemiLineari_MetodoDiGaussSeidel}
\begin{Definition}
  Con le notazioni del teorema \ref{th_MetodiIterativiPerSistemiLineari},
  posto
  \begin{itemize}
    \item $\TriangularMatrix$ la matrice triangolare superiore tale che,
      per $(k,j) \in n \times n$,
      $\TriangularMatrix =
        \begin{cases}
          - \Matrix_k^j\text{ se }k < j,\\
          0\text{ altrimenti}.
        \end{cases}$;
    \item $\DiagonalMatrix$ la matrice triangolare superiore tale che,
      per $(k,j) \in n \times n$,
      $\DiagonalMatrix =
        \begin{cases}
          \Matrix_k^j\text{ se }k = j,\\
          0\text{ altrimenti}.
        \end{cases}$;
    \item $\VarTriangularMatrix$ la matrice triangolare inferiore tale che,
      per $(k,j) \in n \times n$,
      $\VarTriangularMatrix =
        \begin{cases}
          - \Matrix_k^j\text{ se }k > j,\\
          0\text{ altrimenti}.
        \end{cases}$;
  \end{itemize}
  chiamiamo
  \Define{metodo di Jacobi}[di Jacobi][metodo]
  il metodo iterativo definito da
  \begin{itemize}
    \item $A = \DiagonalMatrix - \TriangularMatrix$;
    \item $B = \VarTriangularMatrix$.
  \end{itemize}
\end{Definition}
\begin{listing}
	\insertcode{octave}{"Metodi_numerici_per_sistemi_lineari/MetodoDiGaussSeidel.m"}
	\caption{Implementazione del metodo di Gauss-Seidel in linguaggio \LanguageName{octave}.}
\end{listing}
\begin{Theorem}
  Con le notazioni della definizione precedente,
  se \`e verificata una delle proposizioni seguente, allora
  il metodo di Jacobi \`e convergente:
  \begin{itemize}
    \item $\Matrix$ \`e fortemente dominante diagonale;
    \item $\HermitianTransposed{\Matrix}$ \`e fortemente dominante diagonale;
    \item $\Matrix$ \`e irriducibilmente dominante diagonale;
    \item $\HermitianTransposed{\Matrix}$ \`e irriducibilmente dominante diagonale.
  \end{itemize}
\end{Theorem}
