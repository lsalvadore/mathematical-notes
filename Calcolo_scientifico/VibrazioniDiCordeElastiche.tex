\subsubsection{Vibrazioni di corde elastiche.}
\label{CalcoloScientifico_VibrazioniDiCordeElastiche}
\par Consideriamo una successione finita di
$N \in \NotZero{\mathbb{N}}$ punti materiali
$(\PointMass_k)_{k \in N}$ di massa $(\Mass_k)_{k \in N}$
e coordinate $(X_k)_{k \in N}$ (nel piano e nello spazio indifferentemente)
tali che
\begin{itemize}
  \item i punti $\PointMass_0$ e $\PointMass_{N - 1}$ siano vincolati
    nella loro posizione al tempo iniziale $t = 0$;
  \item per ogni $k \in N - 1$, $\PointMass_k$ e $\PointMass_{k + 1}$
    sono collegati da una corda elastica di costante elastica
    $\ElasticConstant_k$;
  \item per ogni $k \in \NotZero{(N - 1)}$, $\PointMass_k$ \`e soggetto
    ad una forza d'attrito dinamico con coefficiente $\Friction_k$.
\end{itemize}
\par Sia $k \in \NotZero{(N - 1)}$. Per la seconda legge di Newton, abbiamo
\[
  \Mass_k X_k = - \ElasticConstant X_k - \Friction_k \dot{X_k}.
\]
Inoltre, abbiamo le condizioni al bordo
\[
  \begin{cases}
    \ForAll{t \in \RealNonNegative}{X_0(t) = X_0(0)},\\
    \ForAll{t \in \RealNonNegative}{X_{N - 1}(t) = X_{N - 1}(0)},\\
    \dot{X_0} = \dot{X_{N - 1}} = 0.
  \end{cases}
\]
\par Poniamo ora
\begin{itemize}
  \item $M \in \mathbb{R}^{N \times N}$ uguale alla matrice diagonale i cui
    elementi sono la successione delle masse:
    \[
      M =
      \lmatrix
      \begin{array}{ccc}
        \Mass_0 & &\\
        & \ddots &\\
        & & \Mass_{N - 1}
      \end{array}
      \rmatrix.
    \]
  \item $R \in \mathbb{R}^{N \times N}$ uguale alla matrice diagonale i cui
    elementi sono la successione dei coefficienti di attrito:
    \[
      R =
      \lmatrix
      \begin{array}{ccc}
        \Friction_0 & &\\
        & \ddots &\\
        & & \Friction_{N - 1}
      \end{array}
      \rmatrix.
    \]
  \item $K \in \mathbb{R}^{N \times N}$ uguale alla matrice tridiagonale i
    cui elementi sono
    $K_k^j =
    \begin{cases}
      - \ElasticConstant_k&\text{ se }k = j + 1,\\
      \ElasticConstant_k + \ElasticConstant_{k + 1}&\text{ se }k = j,\\
      - \ElasticConstant_{k + 1}&\text{ se }k = j - 1,
    \end{cases}$
    vale a dire
    \[
      K =
      \lmatrix
      \begin{array}{cccccc}
        \ElasticConstant_0 + \ElasticConstant_1 & - \ElasticConstant_1 &&&&\\
        - \ElasticConstant_0 & \ddots & \ddots &&&\\
        & \ddots & \ddots & \ddots & &\\
        && \ddots & \ddots & - \ElasticConstant\\
        &&& - \ElasticConstant & \ElasticConstant + \ElasticConstant
      \end{array}
      \rmatrix.
    \]
\end{itemize}
