\section{Teorema di Perron-Frobenius.}
\label{CalcoloScientifico_TeoremaDiPerronFrobenius}
\begin{Theorem}
  \TheoremName{Teorema di Perron-Frobenius}[di Perron-Frobenius][teorema]
  Siano
  \begin{itemize}
    \item $n \in \NotZero{\mathbb{N}}$;
    \item $\Matrix \in \mathbb{C}^{n \times n}$ non negativa.
  \end{itemize}
  Allora
  \begin{itemize}
    \item $\SpectralRadius{\Matrix} \in \Spectrum{\Matrix}$;
    \item esiste un autovettore sinistro $\VarEigenvector$ tale che
      $\ForAll{k \in n}{\VarEigenvector \geq 0}$;
    \item esiste un autovettore destro $\Eigenvector$ tale che
      $\ForAll{k \in n}{\Eigenvector \geq 0}$.
  \end{itemize}
  Inoltre,
  \begin{itemize}
    \item se $\Matrix$ \`e irriducibile, allora
      \begin{itemize}
        \item $\SpectralRadius{\Matrix}$ \`e un autovalore semplice;
        \item esiste un autovettore sinistro $\VarEigenvector$ tale che
          $\ForAll{k \in n}{\VarEigenvector > 0}$;
        \item esiste un autovettore destro $\Eigenvector$ tale che
          $\ForAll{k \in n}{\Eigenvector > 0}$;
      \end{itemize}
      \item se $\Matrix$ \`e positiva, allora
        $\SpectralRadius{\Matrix}$
        \`e l'unico autovalore di modulo massimo di $\Matrix$.
  \end{itemize}
\end{Theorem}
