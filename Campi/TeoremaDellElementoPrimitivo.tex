\begin{Theorem}
	Sia $\FieldExtension{\Field_2}{\Field_1}$ un'estensione di campo finita. $\FieldExtension{\Field_2}{\Field_1}$ \`e semplice se e solo se ammette solo un un numero finito di sottoestensioni.
\end{Theorem}
\Proof Supponiamo che $\Field_1$ sia un campo di Galois.
Allora anche $\Field_2$ \`e un campo di Galois e l'estensione
$\FieldExtension{\Field_2}{\Field_1}$ risulta generata da un qualsiasi generatore del gruppo degli elementi invertibili di $\Field_2$.
D'altra parte, un'eventuale sottoestensione di $\FieldExtension{\Field_2}{\Field_1}$ deve avere grado che divide $\FieldDegree{\Field_2}{\Field_1}$ e, poich\'e sono in numero finito sia i divisori di $\FieldDegree{\Field_2}{\Field_1}$ sia i polinomi di $\FieldAdjunction{\Field_1}{x}$ di grado fissato, sono in numero finito anche le sottoestensioni stesse.
\par Supponiamo dunque che $\Field_1$ sia un campo infinito.
\par Sia $\alpha \in \AlgebraicClosure{\Field_1}$ tale che $\Field_2 = \FieldAdjunction{\Field_1}{\alpha}$. Sia $f$ l'applicazione che ad ogni campo $\Field$ tale che $\Field_1 \subseteq \Field \subseteq \Field_2$ associa il polinomio $\MinimalPolynomial{\alpha}[\Field]$, polinomio minimo di $\alpha$ in $\Field$. Per ogni campo $\Field$, $\MinimalPolynomial{\alpha}[\Field]$ deve dividere $\MinimalPolynomial{\alpha}[\Field_1]$, quindi il codominio di $f$ \`e finito.
\par Fissiamo il campo $\Field$ tale che $\Field_1 \subseteq \Field \subseteq \Field_2$. Sia $S$ l'insieme di tutti i coefficienti non nulli del polinomio $f(\Field)$. Abbiamo chiaramente $\FieldAdjunction{\Field_1}{S} \subseteq \Field$. Inoltre $f(\FieldAdjunction{\Field_1}{S}) = f(\Field)$, da cui $\FieldDegree{\Field_2}{\FieldAdjunction{\Field_1{S}}} = \FieldDegree{\Field_2}{\Field}$, da cui $\Field = \FieldAdjunction{\Field_1}{S}$. Pertanto se, $\Field^{(1)}, \Field^{(2)} \in \Dom{f}$ sono tali che $f(\Field^{(1)}) = f(\Field^{(2)}$, allora $\Field^{(1)} = \Field^{(2)}$, vale a dire che $f$ \`e un'applicazione iniettiva: per la finitizza del codominio di $f$, si deduce che \`e finito anche il dominio.
\par Supponiamo invece che $\FieldExtension{\Field_2}{\Field_1}$ ammetta solo un numero finito di sottoestensioni e supponiamo $S \subseteq \Field_2$ tale che $\Field_2 = \FieldAdjunction{\Field_1}{S}$. Procediamo per induzione su $\Cardinality{S}$.
\par Se $\Cardinality{S} = 1$ la tesi \`e immediata. Supponiamo la tesi provata per $\Cardinality{S} = n > 1$. Allora scegliamo $\alpha, \beta \in S$ con $\alpha \neq \beta$. Consideriamo i campi $\FieldAdjunction{\Field_1}{\alpha + t \beta}$ al variare di $t \in \Field_1$; essi
\begin{itemize}
	\item verificano tutti la condizione $\Field_1 \subseteq \FieldAdjunction{\Field_1}{\alpha + t \beta} \subseteq \Field_2$ e dunque sono in numero finito;
	\item sono tutte definite da generatori diversi perch\'e se, per $t_1, t_2 \in \Field_1$, abbiamo $\alpha + t_1 \beta = \alpha + t_2 \beta$, allora abbiamo necessariamente $t_1 = t_2$.
\end{itemize}
\par Dunque, per opportuni $t_1, t_2 \in \Field_1$ con $t_1 \neq t_2$ abbiamo $\FieldAdjunction{\Field_1}{\alpha + t_1 \beta} = \FieldAdjunction{\Field_1}{\alpha + t_2 \beta}$. Ne deduciamo $\beta = (t_1 - t_2)^{-1} (\alpha + t_1 \beta - \alpha - t_2 \beta) \in \FieldAdjunction{\Field_1}{\alpha + t_1 \beta}$ e $\alpha = \alpha + t_1 \beta - t_1 \beta \in \FieldAdjunction{\Field_1}{\alpha + t_1 \beta}$. Dunque $\Field_2 = \FieldAdjunction{\Field_1}{\lbrace \alpha + t_1 \beta} \cup S \SetMin \lbrace{\alpha, \beta \rbrace}$ e la tesi segue per ipotesi induttiva. \EndProof
\subsection{Teorema dell'elemento primitivo.}\label{TeoremaDellElementoPrimitivo}
\begin{Theorem}
	\TheoremName{Teorema dell'elemento primitivo}[dell'elemento primitivo][teorema] Sia $\FieldExtension{\Field_2}{\Field_1}$ un'estensione di campo finita e separabile: l'estensione \`e anche semplice.
\end{Theorem}
\Proof Supponiamo prima che $\Field_2$ sia un campo di Galois. Allora $\Invertible{\Field_2}$ \`e ciclico e quindi, scelto $\zeta \in \Invertible{\Field_2}$, abbiamo $\Field_2 = \FieldAdjunction{\Field_1}{\zeta}$.
\par Supponiamo dunque che $\Field_2$ sia un campo infinito.
\par Sia $S$ un insieme di generatori dell'estensione $\FieldAdjunction{\Field_2}{\Field_1}$: l'insieme $S$ \`e chiaramente finito. Procediamo per induzione su $\Cardinality{S}$.
\par Se $\Cardinality{S} = 1$ la tesi \`e immediata. Supponiamo la tesi provata per $\Cardinality{S} = n > 1$. Allora possiamo scegliere $\beta \in S$ e porre $T = S \SetMin \lbrace \beta \rbrace$. Abbiamo dunque la torre d'estensioni $\Field_1 \subseteq \FieldAdjunction{\Field_1}{T} \subseteq \FieldAdjunction{\Field_1}{S} = \Field_2$, dove, per ipotesi induttiva, $\FieldAdjunction{\Field_1}{T} = \FieldAdjunction{\Field_1}{\alpha}$, per opportuno $\alpha \in \FieldAdjunction{\Field_1}{T}$. Dunque $\Field_2 = \FieldAdjunction{\Field_1}{\alpha,\beta}$.
\par Poich\'e l'estensione \`e separabile, abbiamo $\SeparableDegree{\Field_2}{\Field_1} = \FieldDegree{\Field_2}{\Field_1}$, da cui esiste solo un numero finito di omomorfismi relativi $(\RelativeHomomorphism_k)_{k = 1}^K$, con $K \in \mathbb{N}$, da $\Field_2$ in $\AlgebraicClosure{\Field_1}$.
\par Consideriamo il polinomio $\Polynomial(x) = \prod_{k < j} (\RelativeHomomorphism_k(\alpha) + \RelativeHomomorphism_k(\beta)x - \RelativeHomomorphism_j(\alpha) - \RelativeHomomorphism_j(\beta)x)$. Se fosse $\Polynomial = 0$, allora esisterebbero $k, j \in \mathbb{N}$ con $k \neq j$ tali che $\RelativeHomomorphism_k(\alpha) + \RelativeHomomorphism_k(\beta)x = \RelativeHomomorphism_j(\alpha) + \RelativeHomomorphism_j(\beta)x$, da cui dedurremmo $\RelativeHomomorphism_k(\alpha) = \RelativeHomomorphism_j(\alpha)$ e $\RelativeHomomorphism_k(\beta) = \RelativeHomomorphism_j(\beta)$ e, poich\'e $\alpha$ e $\beta$ generano l'estensione $\FieldExtension{\Field_2}{\Field_1}$, $\RelativeHomomorphism_k = \RelativeHomomorphism_j$, ma questo \`e assurdo. Dunque $\Polynomial \neq 0$ e, poich\'e $\Polynomial$ ammette solo un numero finito di radici mentre $\Field_2$ \`e un campo infinito, esiste $t \in \Field_2$ tale che $\Polynomial(t) \neq 0$.
\par Poniamo $\gamma = \alpha + \beta t$ e proviamo che $\Field_2 = \FieldAdjunction{\Field_1}{\gamma}$. Per costruzione di $t$, abbiamo che $\RelativeHomomorphism_k(\gamma) \neq \RelativeHomomorphism_j(\gamma)$ per ogni $k \neq j$, dunque $\FieldDegree{\FieldAdjunction{\Field_1}{\gamma}}{\Field_1} \geq \FieldDegree{\Field_2}{\Field_1}$, e poich\'e $\FieldAdjunction{\Field_1}{\gamma} \subseteq \Field_2$, vale $\FieldDegree{\FieldAdjunction{\Field_1}{\gamma}}{\Field_1} = \FieldDegree{\Field_2}{\Field_1}$ e dunque $\Field_2 = \FieldAdjunction{\Field_1}{\gamma}$. \EndProof
\begin{Exercice}
	Sia $\Prime \in \mathbb{Z}$. Provare che l'estensione $\FieldExtension{\FieldAdjunction{\GaloisField{\Prime}}{X,Y}}{\FieldAdjunction{\GaloisField{\Prime}}{X^\Prime,Y^\Prime}}$ ammette infinite sottoestensioni.
\end{Exercice}
\Solution Consideriamo il polinomio $\Polynomial = t^\Prime - X^\Prime \in \RingAdjunction{\FieldAdjunction{\GaloisField{\Prime}}{X^\Prime,Y^\Prime}}{t}$: esso si annulla in $X$, che \`e la sua unica radice in $\AlgebraicClosure{\FieldAdjunction{\GaloisField{\Prime}}{X^\Prime,Y^\Prime}}$, quindi abbiamo $\SeparableDegree{\FieldAdjunction{\GaloisField{\Prime}}{X^\Prime,Y^\Prime,X}}{\FieldAdjunction{\GaloisField{\Prime}}{X^\Prime,Y^\Prime}} = 1$, da cui $\FieldDegree{\FieldAdjunction{\GaloisField{\Prime}}{X^\Prime,Y^\Prime,X}}{\FieldAdjunction{\GaloisField{\Prime}}{X^\Prime,Y^\Prime}}$ \`e una potenza di $\Prime$, necessariamente $\Prime$ stesso dato che $\EuclideanDegree \Polynomial = \Prime$.
\par Analogamente si prova che $\SeparableDegree{\FieldAdjunction{\GaloisField{\Prime}}{X^\Prime,Y^\Prime,X,Y}}{\FieldAdjunction{\GaloisField{\Prime}}{X^\Prime,Y^\Prime,X}} = 1$ e $\FieldDegree{\FieldAdjunction{\GaloisField{\Prime}}{X^\Prime,Y^\Prime,X,Y}}{\FieldAdjunction{\GaloisField{\Prime}}{X^\Prime,Y^\Prime,X}} = p$. Dunque $\SeparableDegree{\FieldAdjunction{\GaloisField{\Prime}}{X^\Prime,Y^\Prime,X,Y}}{\FieldAdjunction{\GaloisField{\Prime}}{X^\Prime,Y^\Prime}} = 1$ e $\FieldDegree{\FieldAdjunction{\GaloisField{\Prime}}{X^\Prime,Y^\Prime,X,Y}}{\FieldAdjunction{\GaloisField{\Prime}}{X^\Prime,Y^\Prime}} = \Prime^2$.
\par Consideriamo l'endomorfismo di Frobenius $\Frobenius: \FieldAdjunction{\GaloisField{\Prime}}{X,Y} \rightarrow \FieldAdjunction{\GaloisField{\Prime}}{X^\Prime,Y^\Prime}$. Abbiamo $\FieldAdjunction{\GaloisField{\Prime}}{X^\Prime,Y^\Prime} = \Image{\Frobenius}$. Dunque, per ogni $\alpha \in \FieldAdjunction{\GaloisField{\Prime}}{X^\Prime,Y^\Prime} \SetMin \FieldAdjunction{\GaloisField{\Prime}}{X,Y}$, il polinomio $t^\Prime - \alpha^\prime$ appartiene a $\RingAdjunction{\FieldAdjunction{\GaloisField{\Prime}}{X^\Prime,Y^\Prime}}{t}$ e quindi $\FieldDegree{\FieldAdjunction{\GaloisField{\Prime}{X^\Prime,Y^\Prime, \alpha}}}{\FieldAdjunction{\GaloisField{\Prime}{X^\Prime,Y^\Prime}}} = \Prime \neq \Prime^2$. Quindi $\FieldExtension{\FieldAdjunction{\GaloisField{\Prime}}{X,Y}}{\FieldAdjunction{\GaloisField{\Prime}}{X^\Prime,Y^\Prime}}$ non \`e un'estensione semplice e ammette infinite sottoestensioni. \EndProof
\begin{Theorem}
	Siano $\Field$ un campo, $S \subseteq \AlgebraicClosure{\Field}$ un insieme finito di elementi separabili su $\Field$ e $\beta \in \AlgebraicClosure{\Field}$. L'estensione $\FieldAdjunction{\Field}{S \cup \lbrace \beta \rbrace}$ \`e semplice.
\end{Theorem}
\Proof Se $\Field$ \`e finito allora $\FieldAdjunction{\Field}{S \cup \lbrace \beta \rbrace}$ \`e finito e semplice per la ciclicit\`a del suo gruppo moltiplicativo. Supponiamo dunque $\Field$ infinito.
\par Per il teorema dell'elemento primitivo, esiste $\alpha \in S$ tale che $\FieldAdjunction{\Field}{S} = \FieldAdjunction{\Field}{\alpha}$. Dunque $\FieldAdjunction{\Field}{S \cup \lbrace \beta \rbrace} = \FieldAdjunction{\Field}{\alpha,\beta}$.
\par Abbiamo $\SeparableDegree{\FieldAdjunction{\Field}{\alpha,\beta}}{\Field} < \FieldDegree{\FieldAdjunction{\Field}{\alpha,\beta}}{\Field}$, da cui esiste solo un numero finito di omomorfismi relativi $(\RelativeHomomorphism_k)_{k = 1}^K$, con $K \in \mathbb{N}$, da $\FieldAdjunction{\Field}{\alpha,\beta}$ in $\AlgebraicClosure{\Field}$.
\par Consideriamo il polinomio $\Polynomial(x) = \prod_{k < j} (\RelativeHomomorphism_k(\alpha) + \RelativeHomomorphism_k(\beta)x - \RelativeHomomorphism_j(\alpha) - \RelativeHomomorphism_j(\beta)x)$. Se fosse $\Polynomial = 0$, allora esisterebbero $k, j \in \mathbb{N}$ con $k \neq j$ tali che $\RelativeHomomorphism_k(\alpha) + \RelativeHomomorphism_k(\beta)x = \RelativeHomomorphism_j(\alpha) + \RelativeHomomorphism_j(\beta)x$, da cui dedurremmo $\RelativeHomomorphism_k(\alpha) = \RelativeHomomorphism_j(\alpha)$ e $\RelativeHomomorphism_k(\beta) = \RelativeHomomorphism_j(\beta)$ e, poich\'e $\alpha$ e $\beta$ generano l'estensione $\FieldExtension{\FieldAdjunction{\alpha,\beta}}{\Field}$, $\RelativeHomomorphism_k = \RelativeHomomorphism_j$, ma questo \`e assurdo. Dunque $\Polynomial \neq 0$ e, poich\'e $\Polynomial$ ammette solo un numero finito di radici mentre $\FieldAdjunction{\Field}{\alpha,\beta}$ \`e un campo infinito, esiste $t \in \FieldAdjunction{\Field}{\alpha,\beta}$ tale che $\Polynomial(t) \neq 0$.
\par Poniamo $\gamma = \alpha + \beta t$ e proviamo che $\FieldAdjunction{\Field}{\alpha,\beta} = \FieldAdjunction{\Field}{\gamma}$. Per costruzione di $t$, abbiamo che $\RelativeHomomorphism_k(\gamma) \neq \RelativeHomomorphism_j(\gamma)$ per ogni $k \neq j$, dunque $\SeparableDegree{\FieldAdjunction{\Field}{\gamma}}{\Field_1} \geq \SeparableDegree{\FieldAdjunction{\Field}{\alpha,\beta}}{\Field}$, e poich\'e $\FieldAdjunction{\Field}{\gamma} \subseteq \FieldAdjunction{\Field}{\alpha,\beta}$, vale $\SeparableDegree{\FieldAdjunction{\Field}{\gamma}}{\Field} = \SeparableDegree{\FieldAdjunction{\Field}{\alpha,\beta}}{\Field}$ e dunque $\FieldAdjunction{\Field}{\alpha,\beta} = \FieldAdjunction{\Field}{\gamma}$. \EndProof
\begin{Lemma}
	Sia $\Field$ un campo. Abbiamo $\AlgebraicClosure{\Field} = \FieldComposition{\SeparableAlgebraicClosure{\Field}\PerfectClosure{\Field}}$.
\end{Lemma}
\Proof Abbiamo le seguenti torri d'estensioni:
\begin{itemize}
	\item $\Field \subseteq \SeparableAlgebraicClosure{\Field} \subseteq \FieldComposition{\SeparableAlgebraicClosure{\Field}\PerfectClosure{\Field}} \subseteq \AlgebraicClosure{\Field}$;
	\item $\Field \subseteq \PerfectClosure{\Field} \subseteq \FieldComposition{\SeparableAlgebraicClosure{\Field}\PerfectClosure{\Field}} \subseteq \AlgebraicClosure{\Field}$.
\end{itemize}
\par Per trasferimento per traslazione abbiamo che
\begin{itemize}
	\item $\FieldExtension{\FieldComposition{\SeparableAlgebraicClosure{\Field}}{\PerfectClosure{\Field}}}{\SeparableAlgebraicClosure{\Field}}$ \`e puramente inseparabile;
	\item $\FieldExtension{\FieldComposition{\SeparableAlgebraicClosure{\Field}}{\PerfectClosure{\Field}}}{\PerfectClosure{\Field}}$ \`e separabile;
\end{itemize}
e dunque che $\FieldExtension{\AlgebraicClosure{\Field}}{\FieldComposition{\SeparableAlgebraicClosure{\Field}}{\PerfectClosure{\Field}}}$ \`e simultaneamente separabile e puramente inseparabile, dunque banale. \EndProof
\begin{Theorem}
	Sia $\FieldExtension{\Field_2}{\Field_1}$ una estensione di campi algebrica tale che ogni polinomio in $\RingAdjunction{\Field_1}{x}$ ammetta una radice in $\Field_2$. Allora $\Field_2$ \`e chiusura algebrica di $\Field_1$.
\end{Theorem}
\Proof Per il lemma precedente, basta provare che se $\alpha \in \AlgebraicClosure{\Field}$ \`e separabile o puramente inseparabile, allora $\alpha \in \Field_2$.
\par Se $\alpha$ \`e puramente inseparabile, allora $\alpha \in \Field_2$ perch\'e unica radice del suo polinomio minimo su $\Field_1$.
\par Supponiamo invece $\alpha$ separabile e sia $\Field$ il campo di spezzamento di $\MinimalPolynomial{\alpha}[\Field_1]$ su $\Field_1$. $\FieldExtension{\Field}{\Field_1}$ \`e un'estensione finita e separabile, dunque semplice per il teorema dell'elemento primitivo. Sia $\beta \in \Field$ tale che $\Field = \FieldAdjunction{\Field_1}{\beta}$. Il polinomio minimo $\MinimalPolynomial{\beta}[\Field_1]$ ammette una radice $\gamma \in \Field_2$. D'altra parte, per ragioni di grado delle estensioni, $\FieldAdjunction{\Field_1}{\gamma} = \Field$, dunque $\alpha \in \Field = \FieldAdjunction{\Field_1}{\gamma} \subseteq \Field_2$. \EndProof
