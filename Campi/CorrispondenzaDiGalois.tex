\subsection{Corrispondenza di Galois.}\label{CorrispondenzaDiGalois}
\begin{Definition}
	Sia $\FieldExtension{\Field_2}{\Field_1}$ un'estensione di campi normale e separabile: diciamo che $\FieldExtension{\Field_2}{\Field_1}$ \`e un'\Define{estensione di Galois}[di Galois][estensione]. Chiamiamo inoltre \Define{gruppo di Galois}[di Galois][gruppo] di $\FieldExtension{\Field_2}{\Field_1}$, denotato $\GaloisGroup{\Field_2}{\Field_1}$, il gruppo $\RelativeAutomorphisms{\Field_2}{\Field_1}$, i cui elementi chiamiamo \Define{automorfismi di Galois}[di Galois][automorfismo].
\end{Definition}
\begin{Definition}
	Dato un campo $\Field$, chiamiamo $\GaloisGroup{\AlgebraicClosure{\Field}}{\Field}$ \Define{gruppo assoluto di Galois}[assoluto di Galois][gruppo] di $\Field$, o anche pi\`u brevemente \Define{gruppo assoluto}[assoluto][gruppo] di Galois.
\end{Definition}
\begin{Theorem}
	Sia $\FieldExtension{\Field_2}{\Field_1}$ un'estensione di Galois. Abbiamo $\GroupOrder{\GaloisGroup{\Field_2}{\Field_1}} = \FieldDegree{\Field_2}{\Field_1}$.
\end{Theorem}
\Proof Sia $\RelativeHomomorphism: \Field_2 \rightarrow \AlgebraicClosure{\Field_1}$ un omomorfismo relativo. Per normalit\`a di $\FieldExtension{\Field_2}{\Field_1}$, $\Image{\RelativeHomomorphism} = \Field_2$. \`E dunque possibile associare ad ogni omomorfismo relativo di $\FieldExtension{\Field_2}{\Field_1}$ un automorfismo in $\GaloisGroup{\Field_2}{\Field_1}$ modificandone il codominio e mantenendo lo stesso grafo: la corrispondenza \`e chiaramente biunivoca.
\par Dunque $\GroupOrder{\GaloisGroup{\Field_2}{\Field_1}} = \SeparableDegree{\Field_2}{\Field_1}$ e, poich\'e $\FieldExtension{\Field_2}{\Field_1}$ \`e separabile, $\GroupOrder{\GaloisGroup{\Field_2}{\Field_1}} = \FieldDegree{\Field_2}{\Field_1}$. \EndProof
\begin{Theorem}
	Sia $\Field_1 \subseteq \Field_2 \subseteq \Field_3$ una torre d'estensioni di campo. Se $\FieldExtension{\Field_3}{\Field_1}$ \`e di Galois, allora
	\begin{itemize}
		\item $\FieldExtension{\Field_3}{\Field_2}$ \`e di Galois;
		\item $\FieldExtension{\Field_2}{\Field_1}$ \`e di Galois se e solo se $\FieldExtension{\Field_2}{\Field_1}$ \`e normale;
	\end{itemize}
	Inoltre l'essere estensione di Galois \`e una propriet\`a che si trasferisce per traslazione e per composizione, ma non lungo le torri.
\end{Theorem}
\Proof Tutt le affermazioni seguono direttamente dagli analoghi teoremi sulla normalit\`a e la separabilit\`a. \EndProof
\begin{Theorem}
	Sia $\Field_1 \subseteq \Field_2 \subseteq \Field_3$ una torre d'estensioni di campo con $\FieldExtension{\Field_3}{\Field_1}$ di Galois. Abbiamo
	\begin{itemize}
		\item $\GaloisGroup{\Field_3}{\Field_2} \IsSubgroup \GaloisGroup{\Field_3}{\Field_1}$;
		\item se $\GaloisAutomorphism \in \GaloisGroup{\Field_3}{\Field_1}$, allora $\GaloisAutomorphism\GaloisGroup{\Field_3}{\Field_2}\GaloisAutomorphism^{-1} = \GaloisGroup{\Field_3}{\GaloisAutomorphism{\Field_2}}$;
		\item se $\FieldExtension{\Field_2}{\Field_1}$ \`e di Galois, allora l'applicazione $f: \GaloisGroup{\Field_3}{\Field_1} \rightarrow \GaloisGroup{\Field_2}{\Field_1}$ che a $\GaloisAutomorphism \in \GaloisGroup{\Field_3}{\Field_1}$ associa $\Restricted{\GaloisAutomorphism}{\Field_2} \in \GaloisGroup{\Field_2}{\Field_1}$ \`e surgettiva.
	\end{itemize}
\end{Theorem}
\Proof Un automorfismo di $\Field_3$ che lascia fisso $\Field_2$ a maggior ragione lascia fisso $\Field_1$, dunque $\GaloisGroup{\Field_3}{\Field_2} \IsSubgroup \GaloisGroup{\Field_3}{\Field_1}$.
\par Per ogni automorfismo di Galois $\GaloisAutomorphism \in \GaloisGroup{\Field_3}{\Field_1}$ e ogni $\GaloisAutomorphism_0 \in \GaloisGroup{\Field_3}{\Field_2}$ abbiamo
\begin{itemize}
	\item se $x \in \GaloisAutomorphism(\Field_2)$, allora $(\GaloisAutomorphism \circ \GaloisAutomorphism_0 \circ \GaloisAutomorphism^{-1})(x) = x$;
	\item se $x \in \Field_2$ \`e tale che $(\GaloisAutomorphism \circ \GaloisAutomorphism_0 \circ \GaloisAutomorphism^{-1})(x) = x$, allora $\GaloisAutomorphism_0(\GaloisAutomorphism^{-1}(x)) = \GaloisAutomorphism^{-1}(x)$, da cui $x \in \GaloisAutomorphism{\Field_2}$.
\end{itemize}
Quindi $\GaloisAutomorphism\GaloisGroup{\Field_3}{\Field_2}\GaloisAutomorphism^{-1} = \GaloisGroup{\Field_3}{\GaloisAutomorphism{\Field_2}}$.
\par Ogni automorfismo in $\GaloisGroup{\Field_2}{\Field_1}$ ha lo stesso grafo di un'immersione da $\Field_2$ in $\AlgebraicClosure{\Field_1}$, la quale si pu\`o estendere ad omomorfismo relativo da $\Field_3$ in $\AlgebraicClosure{\Field_1}$, la cui immagine \`e $\Field_3$ per la normalit\`a di $\FieldExtension{\Field_3}{\Field_1}$. \EndProof
\begin{Lemma}
	\TheoremName{Lemma di Artin}[di Artin][lemma]\footnote{Questa denominazione non \`e molto comune.} Siano $\Field$ un campo e $\Group \IsSubgroup \Automorphisms{\Field}$. Abbiamo
	\begin{itemize}
		\item se $\Group$ \`e finito, allora $\FieldExtension{\Field}{\Fixed{\Field}{\Group}}$ \`e un'estensione di Galois finita con $\Group$ per gruppo di Galois;
		\item se $\Group$ non \`e finito e $\FieldExtension{\Field}{\Fixed{\Field}{\Group}}$ \`e algebrica, allora $\FieldExtension{\Field}{\Fixed{\Field}{\Group}}$ \`e un estensione di Galois infinita e $\Group \IsSubgroup \GaloisGroup{\Field}{\Fixed{\Field}{\Group}}$.
	\end{itemize}
\end{Lemma}
\Proof Proviamo che $\FieldExtension{\Field}{\Fixed{\Field}{\Group}}$ \`e
\begin{itemize}
	\item algebrica anche nel caso finito;
	\item di Galois in ogni caso.
\end{itemize}
\par Sia $\alpha \in \Field$. La famiglia $(\GaloisAutomorphism(\alpha))_{\GaloisAutomorphism \in \Group}$ \`e finita e ne indichiamo con $S$ il supporto. Definiamo il polinomio $\Polynomial_{\alpha}(x) = \prod_{s \in S}(x - s)$: esso appartiene a $\RingAdjunction{\Fixed{\Field}{\Group}}{x}$, \`e separabile e ha $\alpha$ per radice, dunque $\alpha$ \`e algebrico e separabile su $\Fixed{\Field}{\Group}$. Inoltre, $\Field$ \`e il campo di spezzamento della famiglia di polinomi $(\Polynomial_\alpha)_{\alpha \in L}$, dunque $\FieldExtension{\Field}{\Fixed{\Field}{\Group}}$ \`e normale. Ne consegue che $\FieldExtension{\Field}{\Fixed{\Field}{\Group}}$ \`e un'estensione di Galois.
\par Per costruzione di $\Fixed{\Field}{\Group}$, abbiamo $\Group \IsSubgroup \GaloisGroup{\Field}{\Fixed{\Field}{\Group}}$.
\par Se $\Group$ non \`e finito, allora a maggior ragione non \`e finito $\GaloisGroup{\Field}{\Fixed{\Field}{\Group}}$ e dunque non lo \`e l'estensione $\FieldExtension{\Field}{\Fixed{\Field}{\Group}}$.
\par Supponiamo $\Group$ finito. Per ogni $\alpha \in \Field$, il polinomio $\Polynomial_\alpha$ prova che $\FieldDegree{\FieldAdjunction{\Fixed{\Field}{\Group}}{\alpha}}{\Fixed{\Field}{\Group}} \leq \GroupOrder{\Group}$: sia $\alpha_0 \in \Field$ tale che $\FieldDegree{\FieldAdjunction{\Fixed{\Field}{\Group}}{\alpha_0}}{\Fixed{\Field}{\Group}} \leq \GroupOrder{\Group} = \max_{\alpha \in \Field} \FieldDegree{\FieldAdjunction{\Fixed{\Field}{\Group}}{\alpha}}{\Fixed{\Field}{\Group}} \leq \GroupOrder{\Group}$.
\par Supponiamo esista $\beta \in \Field$ tale che $\beta \notin \FieldAdjunction{\Fixed{\Field}{\Group}}{\alpha_0}$. Allora, per il teorema dell'elemento primitivo, $\FieldAdjunction{\Fixed{\Field}{\Group}}{\alpha_0,\beta}$, che ha necessarimante grado maggiore di $\FieldAdjunction{\Fixed{\Field}{\Group}}{\alpha_0}$ \`e semplice, ma ci\`o contraddice la definizione di $\alpha_0$. Dunque $\Field = \FieldAdjunction{\Fixed{\Field}{\Group}}{\alpha_0}$.
\par Abbiamo infine $\GroupOrder{\Group} \leq \GroupOrder{\GaloisGroup{\Field}{\Fixed{\Field}{\Group}}} = \FieldDegree{\Field}{\Fixed{\Field}{\Group}} = \FieldDegree{\FieldAdjunction{\Fixed{\Field}{\Group}}{\alpha_0}}{\Fixed{\Field}{\Group}} \leq \GroupOrder{\Group}$. Dunque $\Group$ e $\GaloisGroup{\Field}{\Fixed{\Field}{\Group}}$ hanno lo stesso ordine e pertanto coincidono. \EndProof
\begin{Corollary}
	Sia $\FieldExtension{\Field_2}{\Field_1}$ un'estensione normale. Abbiamo
	\begin{itemize}
		\item $\FieldExtension{\Field_2}{\Fixed{\Field_2}{\RelativeAutomorphisms{\Field_2}{\Field_1}}}$ \`e un'estensione di Galois con gruppo di Galois $\RelativeAutomorphisms{\Field_2}{\Field_1}$;
		\item se $\FieldExtension{\Field_2}{\Field_1}$ \`e di Galois, allora $\Field_1 = \Fixed{\Field_2}{\RelativeAutomorphisms{\Field_2}{\Field_1}}$.
	\end{itemize}
\end{Corollary}
\Proof L'estensione $\FieldExtension{\Field_2}{\Fixed{\Field_2}{\RelativeAutomorphisms{\Field_2}{\Field_1}}}$ \`e normale perch\'e \`e normale $\FieldExtension{\Field_2}{\Field_1}$ e \`e separabile per un teorema precedentemente dimostrato, dunque \`e di Galois. Per il lemma di Artin, $\RelativeAutomorphisms{\Field_2}{\Field_1} \IsSubgroup \GaloisGroup{\Field_2}{\Fixed{\Field_2}{\RelativeAutomorphisms{\Field_2}{\Field_1}}} = \RelativeAutomorphisms{\Field_2}{\Fixed{\Field_2}{\RelativeAutomorphisms{\Field_2}{\Field_1}}} \IsSubgroup \RelativeAutomorphisms{\Field_2}{\Field_1}$. Quindi $\GaloisGroup{\Field_2}{\Fixed{\Field_2}{\RelativeAutomorphisms{\Field_2}{\Field_1}}} = \RelativeAutomorphisms{\Field_1}{\Field_2}$.
\par Se inoltre $\FieldExtension{\Field_2}{\Field_1}$ \`e di Galois, allora \`e separabile e dunque anche $\FieldExtension{\Fixed{\Field_2}{\RelativeAutomorphisms{\Field_2}{\Field_1}}}{\Field_1}$ \`e separabile; ma poich\'e essa \`e anche puramente inseprabile, necessariamente \`e banale. \EndProof
\begin{Theorem}
	\TheoremName{Teorema fondamentale della teoria di Galois (caso finito)}[fondamentale della teoria di Galois (caso finito)][teorema] Sia $\FieldExtension{\Field_2}{\Field_1}$ un'estensione di Galois. Siano le classi
	\begin{itemize}
		\item $G = \lbrace \Group \in \PowerSet{\GaloisGroup{\Field_2}{\Field_1}} | \Group \IsSubgroup \GaloisGroup{\Field_2}{\Field_1} \rbrace$;
		\item $F = \lbrace \Field \in \PowerSet{\Field_2} | \Field_1 \subseteq \Field \subseteq \Field_2\text{ e $\Field$ \`e un campo} \rbrace$.
	\end{itemize}
	Definiamo le applicazioni
	\begin{itemize}
		\item $\phi: G \rightarrow F$ tale che $\phi(\Group) = \Fixed{\Field_2}{\Group}$;
		\item $\psi: F \rightarrow G$ tale che $\psi(\Field) = \GaloisGroup{\Field_2}{\Field}$.
	\end{itemize}
	Abbiamo i seguenti risultati:
	\begin{itemize}
		\item $\phi \circ \psi = \Identity$;
		\item se $\FieldExtension{\Field_2}{\Field_1}$ \`e finita, allora $\psi \circ \phi = \Identity$;
		\item dato $\Field \in F$, $\FieldExtension{\Field}{\Field_1}$ \`e normale se e solo se $\psi(\Field) = \GaloisGroup{\Field_2}{\Field} \IsNormalSubgroup \GaloisGroup{\Field_2}{\Field_1}$;
		\item dato $\Field \in F$, se $\FieldExtension{\Field}{\Field_1}$ normale allora $\Isomorphic{\GaloisGroup{\Field}{\Field_1}}{\Quotient{\GaloisGroup{\Field_2}{\Field_1}}{\GaloisGroup{\Field_2}{\Field}}}$.
	\end{itemize}
\end{Theorem}
\Proof Sia $\Field \in F$. Abbiamo $(\phi \circ \psi)(\Field) = \phi(\GaloisGroup{\Field_2}{\Field}) = \Fixed{\Field_2}{\GaloisGroup{\Field_2}{\Field}}$, da cui, per il corollario precedente, $(\phi \circ \psi)(\Field) = \Field$.
\par Sia $\Group \in G$. Abbiamo $(\psi \circ \phi)(\Group) = \psi(\Fixed{\Field_2}{\Group}) = \GaloisGroup{\Field_2}{\Fixed{\Field_2}{\Group}}$, da cui, per il lemma di Artin, se $\Group$ \`e finito allora $(\psi \circ \phi)(\Group) = \Group$.
\par Ora $\FieldExtension{\Field}{\Field_1}$ \`e normale se e solo se, per ogni omomorfismo relativo $\RelativeHomomorphism$ di $\Field$ rispetto a $\Field_1$, abbiamo $\RelativeHomomorphism(\Field) = \Field$.  Supponiamo $\FieldExtension{\Field}{\Field_1}$ sia effettivamente normale e sia $\GaloisAutomorphism \in \GaloisGroup{\Field_2}{\Field_1}$: allora $\GaloisAutomorphism(\Field) = \Field$. Viceversa, se ogni automorfisma di Galois $\GaloisAutomorphism{\Field_2}{\Field_1}$ \`e tale che $\GaloisAutomorphism(\Field) = \Field$, allora dato che ogni omomorfismo relativo di $\Field$ rispetto a $\Field_1$ si estende ad un automorfismo di $\GaloisGroup{\Field_2}{\Field_1}$, allora $\FieldExtension{\Field}{\Field_1}$ \`e normale. Dunque $\FieldExtension{\Field}{\Field_1}$ \`e normale se e solo se $\ForAll{\GaloisAutomorphism \in \GaloisGroup{\Field_2}{\Field_1}}{\GaloisAutomorphism{\Field} = \Field}$.
\par D'altra parte $\GaloisGroup{\Field_2}{\Field} \IsNormalSubgroup \GaloisGroup{\Field_2}{\Field_1}$ se e solo se, per ogni automorfismo di Galois $\GaloisAutomorphism \in \GaloisGroup{\Field_2}{\Field_1}$, abbiamo $\GaloisAutomorphism\GaloisGroup{\Field_2}{\Field}\GaloisAutomorphism^{-1} = \GaloisGroup{\Field_2}{\GaloisAutomorphism(\Field)} = \GaloisGroup{\Field_2}{\Field}$. Dunque $\GaloisGroup{\Field_2}{\Field} \IsNormalSubgroup \GaloisGroup{\Field_2}{\Field_1}$ se e solo se $\ForAll{\GaloisAutomorphism \in \GaloisGroup{\Field_2}{\Field_1}}{\GaloisAutomorphism\GaloisGroup{\Field_2}{\Field}\GaloisAutomorphism^{-1} = \GaloisGroup{\Field_2}{\GaloisAutomorphism(\Field)} = \GaloisGroup{\Field_2}{\Field}}$.
\par Si conclude, per l'iniettivit\`a di $\psi$ che $\FieldExtension{\Field}{\Field_1}$ \`e normale se e solo se $\GaloisGroup{\Field_2}{\Field} \IsNormalSubgroup \GaloisGroup{\Field_2}{\Field_1}$.
\par Infine l'isomorfismo tra $\GaloisGroup{\Field}{\Field_1}$ e $\Quotient{\GaloisGroup{\Field_2}{\Field_1}}{\GaloisGroup{\Field_2}{\Field}}$ si deduce dall'omomorfismo surgettivo che ad ogni automorfismo di $\GaloisGroup{\Field_2}{\Field_1}$ associa la sua restrizione a $\Field$. \EndProof
\begin{Definition}
	Le corrispondenze definite dal teorema precedente dalle applicazioni $\phi$ e $\psi$ del teorema precedente prendono entrambe il nome di \Define{corrispondenza di Galois}[di Galois][corrispondenza].
\end{Definition}
\par
Possiamo utilizzare la corrispondenza di Galois per fornire una dimostrazione alternativa del teorema dell'elemento primitivo.
\begin{Corollary}
\TheoremName{Teorema dell'elemento primitivo}[dell'elemento primitivo][teorema]
	Sia $\FieldExtension{\Field_2}{\Field_1}$ un'estensione di campi finita e separabile.
	$\FieldExtension{\Field_2}{\Field_1}$ \`e semplice.
\end{Corollary}
\Proof
Consideriamo la chiusura normale $\NormalClosure{\Field_2}$ di $\Field_2$ in $\AlgebraicClosure{\Field_2}$:
$\FieldExtension{\NormalClosure{\Field_2}}{\Field_1}$ \`e separabile e dunque di Galois; inoltre \`e anche finita.
Il gruppo $\GaloisGroup{\NormalClosure{\Field_2}}{\Field_1}$ \`e dunque finito e ammette pertanto solo un numero finito di sottogruppi.
Ne deduciamo che $\FieldExtension{\NormalClosure{\Field_2}}{\Field_1}$ ammette solo un numero finito di estensioni.
Dunque anche $\FieldExtension{\Field_2}{\Field_1}$ ammette solo un numero finito di estensioni.
Ne consegue che $\FieldExtension{\Field_2}{\Field_1}$ \`e semplice.
\EndProof
\begin{Corollary}
	Sia $\FieldExtension{\Field_2}{\Field_1}$ un'estensione finita di Galois e
	siano $\Group_1, \Group_2 \IsSubgroup \GaloisGroup{\Field_2}{\Field_1}$.
	Abbiamo
	\begin{itemize}
		\item
		$\Coimplies
		{\Fixed{\Field_2}{\Group_1}
		\subseteq \Fixed{\Field_2}{\Group_2}}
		{\Group_2 \subseteq \Group_1}$;
		\item
		$\FieldComposition
		{\Fixed{\Field_2}{\Group_1}}
		{\Fixed{\Field_2}{\Group_2}} =
		\Fixed{\Field_2}{\Group_1 \cap \Group_2}$;
		\item
		$\Fixed{\Field_2}{\Group_1} \cap
		\Fixed{\Field_2}{\Group_2} =
		\Fixed{\Field_2}
		{\SpanGroup{\Group_1,\Group_2}}$.
	\end{itemize}
\end{Corollary}
\Proof
Il primo punto segue direttamente dal teorema fondamentale della teoria di Galois.
\par
Per il secondo punto osserviamo che
$\Group_1 \cap \Group_2 \subseteq \Group_1$,
da cui $\Fixed{\Field_2}{\Group_1} \subseteq \Fixed{\Field_2}{\Group_1 \cap \Group_2}$
e analogamente
$\Fixed{\Field_2}{\Group_2} \subseteq \Fixed{\Field_2}{\Group_1 \cap \Group_1}$.
Ne deduciamo
$\FieldComposition{\Fixed{\Field_2}{\Group_1}}{\Fixed{\Field_2}{\Group_2}}
\subseteq \Fixed{\Field_2}{\Group_1 \cap \Group_2}$.
D'altra parte abbiamo
$\GaloisGroup{\Field_2}
{\FieldComposition{\Fixed{\Field_2}{\Group_1}}{\Fixed{\Field_2}{\Group_2}}}
\subseteq
\GaloisGroup{\Field_2}{\Fixed{\Field_2}{\Group_1}} \cap
\GaloisGroup{\Field_2}{\Fixed{\Field_2}{\Group_2}}$,
da cui, tramite la corrispondenza di Galois,
$\FieldComposition{\Fixed{\Field_2}{\Group_1}}{\Fixed{\Field_2}{\Group_2}}
\supseteq
\Fixed{\Field_2}{\Group_1} \cap \Fixed{\Field_2}{\Group_2}
= \Fixed{\Field_2}{\Group_1 \cap \Group_2}$.
\par
Per il terzo punto osserviamo che
$\Group_1 \subseteq \SpanGroup{\Group_1,\Group_2}$,
da cui $\Fixed{\Field_2}{\SpanGroup{\Group_1,\Group_2}} \subseteq \Fixed{\Field_2}{\Group_1}$
e analogamente
$\Fixed{\Field_2}{\SpanGroup{\Group_1,\Group_2}} \subseteq \Fixed{\Field_2}{\Group_2}$.
Ne deduciamo
$\Fixed{\Field_2}{\SpanGroup{\Group_1,\Group_2}}
\subseteq \Fixed{\Field_2}{\Group_1} \cap \Fixed{\Field_2}{\Group_2}$.
D'altra parte ogni elemento di $\Field_2$ che viene lasciato fisso da ogni
automorfismo di $\Group_1$ e di $\Group_2$ viene lasciato fisso anche da
ogni automorfismo di $\SpanGroup{\Group_1,\Group_2}$.
Quindi abbiamo anche
$\Fixed{\Field_2}{\Group_1}
\cap
\Fixed{\Field_2}{\Group_2}
\subseteq
\Fixed{\Field_2}{\SpanGroup{\Group_1}{\Group_2}}$.
\EndProof
\begin{Theorem}
	Con le notazioni del teorema precedente, esistono estensioni di Galois infinite $\FieldExtension{\Field_2}{\Field_1}$ tali che $\phi$ non sia iniettiva; e dunque in particolare tali che $\psi \circ \phi \neq \Identity$.
\end{Theorem}
\Proof
Consideriamo $\Field = \GaloisField{\Prime}$ per qualche $\Prime \in \mathbb{Z}$ primo.
L'estensione $\FieldExtension{\AlgebraicClosure{\Field}}{\Field}$ \`e
\begin{itemize}
	\item normale perch\'e $\AlgebraicClosure{\Field}$ \`e la chiusura algebrica di $\Field$;
	\item separabile perch\'e i campi di Galois sono perfetti.
\end{itemize}
Dunque $\FieldExtension{\AlgebraicClosure{\Field}}{\Field}$ \`e un'estensione di Galois.
Consideriamo l'endomorfismo di Frobenius
$\Frobenius: \AlgebraicClosure{\Field} \rightarrow \AlgebraicClosure{\Field}$.
Essendo $\AlgebraicClosure{\Field}$ algebricamente chiuso di caratteristica $\Prime$, ogni suo elemento ammette una e una sola radice $\Prime$-esima, dunque $\Frobenius$ \`e un automorfismo.
Ora, \`e chiaro che per ogni
$\alpha \in \Field$ abbiamo
$\Frobenius{\alpha} = \alpha^\Prime = \alpha$.
Abbiamo dunque $\SpanGroup{\Frobenius} \IsSubgroup \GaloisGroup{\AlgebraicClosure{\Field}}{\Field}$.
Inoltre, $\alpha \in \AlgebraicClosure{\Field}$ \`e tale che $\Frobenius{\alpha} = \alpha$ se e solo se $\alpha^\Prime = \alpha$ cio\`e se e solo se $\alpha \in \Field$, da cui
$\Fixed{\AlgebraicClosure{\Field}}{\SpanGroup{\Frobenius}} = \Field = \Fixed{\AlgebraicClosure{\Field}}{\GaloisGroup{\AlgebraicClosure{\Field}}{\Field}}$.
\par Supponiamo $\phi$ inettiva.
Allora deve essere $\SpanGroup{\Frobenius} =
\GaloisGroup{\AlgebraicClosure{\Field}}{\Field}$.
\par Consideriamo l'omomorfismo che a $n \in \mathbb{Z}$ associa $\Frobenius^n \in \SpanGroup{\Frobenius}$: esso \`e chiaramente suriettivo.
Inoltre, poich\'e $\AlgebraicClosure{\Field}$ \`e un campo finito e l'equazione $x^{\Prime^n} = x$ ha infinite soluzioni se e solo se $n = 0$, allora l'omomorfismo \`e anche iniettivo.
Dunque $\SpanGroup{\Frobenius}$ \`e isomorfo a $\mathbb{Z}$.
Ora, tutti i sottogruppi di $\mathbb{Z}$ sono finiti,
dunque tutte le estensioni di campo del tipo
$\FieldExtension{\AlgebraicClosure{\Field}}{\Field}$
sono finite per la suriettivit\`a di $\phi$ ed il lemma di Artin.
\par
Tuttavia $\Field' = \GaloisField{\Prime^2}$ \`e un campo tale che $\FieldExtension{\Field'}{\Field}$ \`e finito e dunque $\FieldExtension{\AlgebraicClosure{\Field}}{\Field}$ \`e infinito.
Questo \`e assurdo: dunque
$\SpanGroup{\Frobenius} \neq \GaloisGroup{\AlgebraicClosure{\Field}}{\Field}$
e $\phi$ non \`e un'applicazione iniettiva.
\EndProof
\begin{Theorem}
	Siano date le seguenti torri d'estensione:
	\begin{itemize}
		\item $\Field_1 \subseteq
		\Field_2 \subseteq
		\AlgebraicClosure{\Field_1}$,
		con $\FieldExtension{\Field_2}{\Field_1}$
		finita e di Galois;
		\item $\Field_1 \subseteq
		\Field_3 \subseteq
		\AlgebraicClosure{\Field_1}$.
	\end{itemize}
	Abbiamo $\Isomorphic
	{\GaloisGroup{\FieldComposition{\Field_2}{\Field_3}}
	{\Field_2}}
	{\GaloisGroup{\Field_2}{\Field_2 \cap \Field_3}}$.
\end{Theorem}
\Proof
Sia $\GaloisAutomorphism \in
\GaloisGroup{\FieldComposition{\Field_2}{\Field_3}}{\Field_2}$.
Poich\'e $\FieldExtension{\Field_2}{\Field_1}$ \`e normale,
$\Restricted{\GaloisAutomorphism}{\Field_2}(\Field_2) =
\Field_2$.
Inoltre $\Field_2 \cap \Field_3 \subseteq \Field_2$,
Esiste dunque un'applicazione
$\Homomorphism:
\GaloisGroup{\FieldComposition{\Field_2}{\Field_3}}{\Field_2}
\rightarrow
\GaloisGroup{\Field_2}{\Field_2 \cap \Field_3}$:
\`e immediato verificare che si tratta di un omomorfismo.
\par
$\Homomorphism$ \`e iniettivo perch\'e se due automorfismi di
$\GaloisGroup{\FieldComposition{\Field_2}{\Field_3}}{\Field_2}$
coincidono su $\Field_2$, allora, coincidendo anche su $\Field_3$,
devono coincidere su $\FieldComposition{\Field_2}{\Field_3}$.
\par
Proviamo che $\Homomorphism$ \`e anche suriettivo.
Abbiamo $\Image{\Homomorphism} \subseteq \GaloisGroup{\Field_2}{\Field_2 \cap \Field_3}$,
dunque $\Field_2 \cap \Field_3 \subseteq \Fixed{\Field_2}{\Image{\Homomorphism)}}$.
Inoltre, sia $\alpha \in \Fixed{\Field_2}{\Image{\Homomorphism}}$.
Abbiamo $\alpha \in \Field_2$ perch\'e $\Fixed{\Field_2}{\Image{\Homomorphism}} \subseteq
\Field_2$.
Inoltre, per costruzione, per ogni $\GaloisAutomorphism \in
\GaloisGroup{\FieldComposition{\Field_2}{\Field_3}}{\Field_3}$ abbiamo
$\Homomorphism(\GaloisAutomorphism) (\alpha) = \alpha$, da cui
$\alpha \in \Field_3$.
Ne consegue $\Field_2 \cap \Field_3 = \Fixed{\Field_2}{\Image{\Homomorphism}}$ e quindi,
per la corrispondenza di Galois e a finitezza di $\FieldExtension{\Field_2}{\Field_1}$,
la suriettivit\`a di $\Homomorphism$.
\EndProof
\begin{Theorem}
	Siano $\FieldExtension{\Field_2}{\Field_1}$ e
	$\FieldExtension{\Field_3}{\Field_1}$ estensioni di Galois.
	Allora l'omomorfismo
	$\Homomorphism
	\GaloisGroup{\FieldComposition{\Field_2}{\Field_3}}{\Field_1}
	\rightarrow
	\GaloisGroup{\Field_2}{\Field_1} \times \GaloisGroup{\Field_3}{\Field_1}$,
	che a $\GaloisAutomorphism \in
	\GaloisGroup{\FieldComposition{\Field_2}{\Field_3}}{\Field_1}$
	associa $(\GaloisAutomorphism_1,\GaloisAutomorphism_2 \in
	\GaloisGroup{\Field_2}{\Field_1} \times \GaloisGroup{\Field_3}{\Field_1}$,
	dove $\GaloisAutomorphism_1$ e $\GaloisAutomorphism_2$ hanno lo stesso grafo di
	$\Restricted{\GaloisAutomorphism}{\Field_2}$ e $\Restricted{\GaloisAutomorphism}{\Field_3}$,
	\`e iniettiva.
	Inoltre, se $\Field_2 \cap \Field_3 = \Field_1$, allora \`e anche suriettiva.
	\par
	Infine, se $\FieldExtension{\Field_2}{\Field_1}$ e $\FieldExtension{\Field_3}{\Field_1}$
	sono finite, allora la biettivit\`a dell'applicazione implica
	$\Field_2 \cap \Field_3 = \Field_1$.
\end{Theorem}
\Proof
L'iniettivit\`a segue subito dal fatto che automorfismi di
$\GaloisGroup{\FieldComposition{\Field_2}{\Field_3}}{\Field_1}$
che coincidono su $\Field_2$ e $\Field_3$ coincidono anche su
$\FieldComposition{\Field_2}{\Field_3}$.
\par
La suriettivit\`a nel caso $\Field_2 \cap \Field_3 = \Field_1$
segue dal fatto che
$\Image{\Homomorphism} =
= \lbrace (\GaloisAutomorphism_1, \GaloisAutomorphism_2) \in
\GaloisGroup{\Field_2}{\Field_1} \times \GaloisGroup{\Field_3}{\Field_1} |
\Restricted{\GaloisAutomorphism_1}{\Field_2 \cap \Field_3} =
\Restricted{\GaloisAutomorphism_2}{\Field_2 \cap \Field_3}$.
\par
Supponiamo $\FieldExtension{\Field_2}{\Field_1}$ e
$\FieldExtension{\Field_3}{\Field_1}$ finite.
Per la corrispondenza di Galois, abbiamo che la biettivit\`a di
$\Homomorphism$ equivale all'equicardinalit\`a di
$\GaloisGroup{\FieldComposition{\Field_2}{\Field_3}}{\Field_1}$
e
$\GaloisGroup{\Field_2}{\Field_1} \times \GaloisGroup{\Field_3}{\Field_1}$
dunque a
$\FieldDegree{\FieldComposition{\Field_2}{\Field_3}}{\Field_1} =
\FieldDegree{\Field_2}{\Field_1}
\FieldDegree{\Field_3}{\Field_1}$,
a sua volta equivalente per il teorema precedente a
$\FieldDegree{\Field_3}{\Field_2 \cap \Field_3} =
\FieldDegree{\Field_3}{\Field_1}$ e quindi a
$\Field_2 \cap \Field_3 = \Field_1$.
\EndProof
\begin{Theorem}
	Siano $\Field$ un campo e sia
	$\Polynomial \in \RingAdjunction{\Field}{x}$ separabile.
	Sia $\Set \subseteq \AlgebraicClosure{\Field}$ l'insieme di tutte
	le radici di $\Polynomial$.
	Allora
	\begin{itemize}
		\item l'applicazione
		$\Homomorphism:
		\GaloisGroup{\FieldAdjunction{\Field}{\Set}}{\Field}
		\rightarrow
		\Automorphisms{\Set} \Isomorphic
		\SymmetricGroup{\Cardinality{\Set}}$
		 che a $\GaloisAutomorphism \in
		\GaloisGroup{\FieldAdjunction{\Field}{\Set}}{\Field}$
		 associa
		$\Restricted{\GaloisAutomorphism}{\Set} \in
		\Automorphisms{\Set}$
		\`e un omomorfismo iniettivo;
		\item $\Polynomial$ \`e irriducibile se e solo se
		$\GaloisGroup{\FieldAdjunction{\Field}{\Set}}{\Field}$
		agisce transitivamente su $\Set$.
	\end{itemize}
\end{Theorem}
\Proof
$\Homomorphism$ \`e un omomorfismo perch\'e gli automorfismi di Galois
trasformano radici di polinomi a coefficienti nel campo di base nelle
medesime radici. Inoltre \`e un omomorfismo iniettivo perch\'e se un
automorfismo di Galois lascia fisse tutte le radici di $\Polynomial$,
allora deve essere l'identit\`a/
\par
Affermare che $\Polynomial$ \`e irriducibile equivale ad affermare che \`e
polinomio minimo di ogni sua radice. La transitivit\`a dell'azione di
$\GaloisGroup{\FieldAdjunction{\Field}{\Set}}{\Field}$ segue dunque
dal teorema \ref{th_esistenzaomomorfismirelativi_0}.
\EndProof
\begin{Corollary}
	Sia $\FieldExtension{\Field_2}{\Field_1}$ un'estensione di Galois
	finita.
	Abbiamo $\GaloisGroup{\Field_2}{\Field_1} \IsSubgroup
	\SymmetricGroup{\FieldDegree{\Field_2}{\Field_1}}$.
\end{Corollary}
\Proof Per il teorema dell'elemento primitivo, esiste
$\alpha \in \Field_2$ tale che
$\Field_2 = \FieldAdjunction{\Field_1}{\alpha}$. Dunque $\Field_2$ \`e
il campo di spezzamento  del polinimio minimo $\MinimalPolynomial{\alpha}$
e la tesi segue dal teorema precedente.
\EndProof
\Proof
Segue direttamente dal teorema di Cayley.
\EndProof
\begin{Theorem}
	\TheoremName{Teorema fondamentale dell'algebra} Il campo $\mathbb{C}$ \`e algebricamente chiuso.
\end{Theorem}
\Proof Supponiamo $\Field$ campo tale che $\FieldExtension{\Field}{\mathbb{C}}$ sia algebrica. La chiusura normale di $\Field$ su $\mathbb{R}$ ha grado finito, dunque $\FieldExtension{\NormalClosure{\Field}}{\mathbb{C}}$ ha ancora grado finito. Poich\'e inoltre $\Characteristic{\mathbb{R}} = 0$, $\FieldExtension{\NormalClosure{\Field}}{\mathbb{C}}$ \`e separabile e dunque di Galois.
\par Poich\'e $\FieldDegree{\mathbb{C}}{\mathbb{R}} = 2$ divide $\FieldDegree{\NormalClosure{\Field}}{\mathbb{R}}$, esiste un $2$-Sylow $\Sylow[2]$ sottogruppo di $\GaloisGroup{\NormalClosure{\Field}}{\mathbb{R}}$, a cui corrisponde una sottoestensione $\FieldExtension{\NormalClosure{\Field}}{\VarField}$, di grado uguale all'ordine di $\Sylow[2]$. Dunque $\FieldExtension{\VarField}{\mathbb{R}}$ \`e un estensione di grado dispari. Poich\'e non esistono polinomi irriducibili non lineari di grado dispari su $\mathbb{R}$, necessarimente $\VarField = \mathbb{R}$, da cui $\GaloisGroup{\NormalClosure{\Field}}{\mathbb{C}}$ ha ordine potenza di $2$. Se quest'ultimo ordine \`e almeno $2$, allora $\GaloisGroup{\NormalClosure{\Field}}{\mathbb{C}}$ ammette un sottogruppo di ordine $2$, a cui corrisponde una estensione $\FieldExtension{\VarField'}{\mathbb{C}}$ di grado due. Ma tutti i polinomi di $\RingAdjunction{\mathbb{C}}{x}$ di grado $2$ sono riducibili. Dunque deve essere $\FieldDegree{\NormalClosure{\Field}}{\mathbb{C}} = 1$, da cui $\Field = \mathbb{C}$. \EndProof
