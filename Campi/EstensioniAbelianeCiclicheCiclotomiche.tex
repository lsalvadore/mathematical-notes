\subsection{Estensioni abeliane, cicliche, ciclotomiche.}
\label{EstensioniAbelianeCiclicheCiclotomiche}
\begin{Definition}
	Sia $\FieldExtension{\Field_2}{\Field_1}$
	un'estensione di Galois. L'estensione si dice
	\begin{itemize}
		\item \Define{abeliana}[abeliana][estensione] quando
		$\GaloisGroup{\Field_2}{\Field_1}$ \`e un gruppo abeliano;
		\item \Define{ciclica}[ciclica][estensione] quando
		$\GaloisGroup{\Field_2}{\Field_1}$ \`e un gruppo ciclico.
	\end{itemize}
\end{Definition}
\begin{Theorem}
	Abelianit\`a e ciclicit\`a non sono propriet\`a che si
	trasferiscono lungo le torri.
\end{Theorem}
\Proof
Segue immediatamente dal fatto che la propriet\`a di essere di Galois non
si trasferisce lungo le torri.
\EndProof
\begin{Theorem}
	Sia $\Field_1 \subseteq \Field_2 \subseteq \Field_3$ una torre
	d'estensioni con $\FieldExtension{\Field_2}{\Field_1}$ normale. Se
	\begin{itemize}
		\item $\FieldExtension{\Field_3}{\Field_1}$ \`e abeliana,
		allora $\FieldExtension{\Field_3}{\Field_2}$ e
		$\FieldExtension{\Field_2}{\Field_1}$ sono abeliane;
		\item $\FieldExtension{\Field_3}{\Field_1}$ \`e ciclica,
		allora $\FieldExtension{\Field_3}{\Field_2}$ e
		$\FieldExtension{\Field_2}{\Field_1}$ sono cicliche;
	\end{itemize}
\end{Theorem}
\Proof
$\FieldExtension{\Field_3}{\Field_2}$ e
$\FieldExtension{\Field_2}{\Field_1}$ sono estensioni di Galois: la prima
perch\'e \`e di Galois $\FieldExtension{\Field_3}{\Field_2}$, la seconda
per lo stesso motivo unito alla normalit\`a dell'estensione stessa.
\par
L'abelianit\`a o ciclit\`a di
$\FieldExtension{\Field_3}{\Field_2}$ segue dal fatto che sottogruppi di
gruppi abeliani o ciclici sono abeliani o ciclici rispettivamente.
\par
L'abelianit\`a o ciclit\`a di
$\FieldExtension{\Field_2}{\Field_1}$ segue dal fatto che quozienti di
gruppi abeliani o ciclici sono abeliani o ciclici rispettivamente.
\EndProof
\begin{Theorem}
	Consideriamo le estenioni
	$\FieldExtension{\Field_2}{\Field_1}$ e
	$\FieldExtension{\Field_3}{\Field_1}$, con
	$\FieldExtension{\Field_2}{\Field_1}$ finita.
	Abbiamo
	\begin{itemize}
		\item
		se $\FieldExtension{\Field_2}{\Field_1}$ \`e abeliana,
		allora
		$\FieldExtension{\FieldComposition{\Field_2}{\Field_3}}
		{\Field_3}$ \`e abeliana;
		\item
		se $\FieldExtension{\Field_2}{\Field_1}$ \`e ciclica,
		allora
		$\FieldExtension{\FieldComposition{\Field_2}{\Field_3}}
		{\Field_3}$ \`e ciclica.
	\end{itemize}
\end{Theorem}
\Proof
Segue immediatamente dal fatto che
$\GaloisGroup{\FieldComposition{\Field_2}{\Field_3}}{\Field_3} \Isomorphic
\GaloisGroup{\Field_2}{\Field_2 \cap \Field_3} \IsSubgroup
\GaloisGroup{\Field_2}{\Field_1}$ e dal fatto che sottogruppi di gruppi
abeliani o ciclici sono abeliani o ciclici rispettivamente.
\EndProof
\begin{Theorem}
	Siano le estensioni
	$\FieldExtension{\Field_2}{\Field_1}$ e
	$\FieldExtension{\Field_3}{\Field_1}$.
	Abbiamo,
	\begin{itemize}
		\item
		se
		$\FieldExtension{\Field_2}{\Field_1}$ e
		$\FieldExtension{\Field_3}{\Field_1}$ sono abeliane,
		allora
		$\FieldExtension{\FieldComposition{\Field_2}{\Field_3}}
		{\Field_1}$ \`e abeliana;
		\item
		se
		$\FieldExtension{\Field_2}{\Field_1}$ e
		$\FieldExtension{\Field_3}{\Field_1}$ sono ciclice,
		e
		$\FieldDegree{\Field_2}{\Field_1}$ e
		$\FieldDegree{\Field_3}{\Field_1}$ sono coprimi,
		allora
		$\FieldExtension{\FieldComposition{\Field_2}{\Field_3}}
		{\Field_1}$ \`e ciclica.
	\end{itemize}
\end{Theorem}
\Proof
Abbiamo provato che
$\GaloisGroup{\FieldComposition{\Field_2}{\Field_3}}
{\Field_1}$ si immerge in
$\GaloisGroup{\Field_2}{\Field_1} \times
\GaloisGroup{\Field_3}{\Field_1}$, il quale \`e
\begin{itemize}
	\item un gruppo abeliano se i fattori sono abeliani;
	\item un gruppo ciclico se i fattori sono ciclici e di ordini
	coprimi.
\end{itemize}
\EndProof
\begin{Theorem}
	Sia $\Field$ un campo algebricamente chiuso e
	$\Automorphism \in \Automorphisms{\Field}$.
	Sia $\Field_0$ un'estensione finita di
	$\Fixed{\Field}{\SpanGroup{\Automorphism}}$.
	$\Field_0$ \`e ciclica.
\end{Theorem}
\Proof
Poich\'e $\SpanGroup{\Automorphism}$ \`e un gruppo ciclico,
\`e un gruppo ciclico anche $\SpanGroup{\Automorphism_0}$, dove
$\Automorphism_0$ \`e l'autormofismo di $\Field_0$ con lo stesso
grafo di $\Restricted{\Automorphism}{\Field_0}$.
\par
Abbamo allora $\Field = \Fixed{\Field_0}{\SpanGroup{\Automorphism_0}}$ e,
per il lemma di Artin, $\FieldExtension{\Field_0}{\Field}$ \`e
un'estensione di Galois con gruppo di Galois $\SpanGroup{\Automorphism_0}$
e dunque ciclica.
\EndProof
\begin{Theorem}
	Sia $a \in \mathbb{N}$. $\FieldAdjunction{\mathbb{Q}}{\sqrt{-a}}$
	non si immerge in un'estensione ciclica di $\mathbb{Q}$ di grado
	multiplo di $4$.
\end{Theorem}
\Proof
Sia la torre d'estensione
$\mathbb{Q} \subseteq
\FieldAdjunction{\mathbb{Q}}{\sqrt{-a}} \subseteq
\Field$,
con $\FieldExtension{\Field}{\mathbb{Q}}$ estensione ciclica
di grado $4d$, con $d \in \NotZero{\mathbb{N}}$.
Procediamo per induzione su $d$.
\par
Supponiamo $d = 1$.
$\Field \cap \mathbb{R} \neq \Field$ perch\'e
$\sqrt{-a} \notin \mathbb{R}$.
Inoltre $\Field \cap \mathbb{R} \neq \mathbb{Q}$ perch\'e
\par
Quindi
$\FieldDegree{\Field}{\Field \cap \mathbb{R}} = 2 =
\FieldDegree{\Field}{\FieldAdjunction{\mathbb{Q}}{\sqrt{-a}}}$:
$\Field$ ammette dunque sottocampi distinti di grado $2$ che estendono
$\mathbb{Q}$, in contraddizione con la ciclicit\`a di
$\GaloisField{\Field}{\mathbb{Q}}$ e il teorema di corrispondenza di
Galois.
\par
Sia $d > 1$, all'estensione
$\FieldExtension{\Field}{\FieldAdjunction{\mathbb{Q}}{\sqrt{-a}}}$
corrisponde un sottogruppo ciclico
$\Subgroup \IsSubgroup \GaloisGroup{\Field}{\mathbb{Q}}$ di ordine $2d$.
Sia poi $\Subgroup_0 \IsSubgroup \Subgroup$ di ordine $d$: a $\Subgroup_0$
corrisponde un'estensione di campo $\FieldExtension{\Field}{\Field_0}$,
con
$\FieldAdjunction{\mathbb{Q}}{\sqrt{-a}} \subseteq \Field_0$ e tale che
$\FieldDegree{\Field_0}{\FieldAdjunction{\mathbb{Q}}{\sqrt{-a}}} = 2$.
Abbiamo dunque la torre d'estensioni
$\mathbb{Q} \subseteq \FieldAdjunction{\mathbb{Q}}{\sqrt{-a}} \subseteq
\Field_0$ con
$\FieldDegree{\Field_0}{\mathbb{Q}} = 4$.
L'estensione $\FieldDegree{\Field_0}{\mathbb{Q}}$ \`e anche ciclica
perch\'e il suo gruppo di Galois \`e isomorfo al quoziente
$\Quotient{\GaloisGroup{\Field}{\mathbb{Q}}}
{\GaloisGroup{\Field}{\Field_0}}$.
\par
Ci si riconduce dunque al caso $d = 1$.
\EndProof
