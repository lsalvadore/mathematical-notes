\subsection{Polinomi di \Chebyshev.}
\label{IstituzioniDiAnalisiNumerici_AltriPolinomiOrtogonalSpecifici}
\begin{Definition}
	Chiamiamo \Define{polinomi di \Chebyshev\ di prima specie}[di \Chebyshev\ di prima specie][polinomi] i polinomi $(\FirstChebyshevPolynomial_n)_{n \in \mathbb{N}}$ definiti da
	\[
		\FirstChebyshevPolynomial_{n + 1}(x) =
		\begin{cases}
			\FirstChebyshevPolynomial_0(x) = 1\text{ se }n = - 1,\\
			\FirstChebyshevPolynomial_1(x) = x\text{ se }n = 0,\\
			2x\FirstChebyshevPolynomial_n(x) - \FirstChebyshevPolynomial_{n - 1}(x)\text{ altrimenti}.
		\end{cases}
	\]
\end{Definition}
\begin{Theorem}
	I polinomi di \Chebyshev\ di prima specie $(\FirstChebyshevPolynomial_n)_{n \in \mathbb{N}}$ verificano la relazione
\[
	\ForAll{(n,\theta) \in \mathbb{N} \times \mathbb{R}}{\FirstChebyshevPolynomial_n(\cos{\theta}) = \cos{n \theta}}.
\]
\end{Theorem}
\Proof Procediamo per induzione su $i$.
\par Nei casi $n = 0$ e $n = 1$ la tesi \`e immediata: dimostriamola per $n > 1$. Abbiamo. per qualunque $\theta \in \mathbb{R}$,
\begin{align*}
	\FirstChebyshevPolynomial_{n + 1}(\cos{\theta})
	&= 2 \cos{\theta} \FirstChebyshevPolynomial_n(\cos{\theta}) - \FirstChebyshevPolynomial_{n - 1}(\cos{\theta}),\\
	&= 2 \cos{\theta} \cos{n \theta} - \cos{(n - 1) \theta},\\
	&= 2 \cos{\theta} \cos{n \theta} - \cos{n\theta} \cos{\theta} - \sin{n\theta} \sin{\theta},\\
	&= \cos{\theta} \cos{n \theta} - \sin{n\theta} \sin{\theta},\\
	&= \cos{(n + 1) \theta}.\text{ \EndProof}
\end{align*}
\begin{Corollary}
	Definiamo su $\RingAdjunction{\mathbb{C}}{x}$ il prodotto hermitiano $\HermitianProduct{\Polynomial}{\VarPolynomial} = \int_{-1}^1 (1 - x^2)^{-\frac{1}{2}} \Polynomial(x)\Conjugate{\VarPolynomial(x)}dx$. I polimoni di \Chebyshev\ di prima specie $(\FirstChebyshevPolynomial_i)_{n \in \mathbb{N}}$ verificano la relazione
	\[
		\HermitianProduct{\FirstChebyshevPolynomial_n}{\FirstChebyshevPolynomial_m} =
		\begin{cases}
			0\text{ se }n \neq m,\\
			\pi\text{ se }n = m = 0,\\
			\frac{\pi}{2}\text{ se }n = m \neq 0.
		\end{cases}
	\]
	In particolare i polimoni di \Chebyshev\ di prima specie sono i polinomi ortogonali rispetto al prodotto hermitiano $\HermitianProduct{\cdot}{\cdot}$.
\end{Corollary}
\Proof Abbiamo, per ogni $(n,m) \in \mathbb{N}^2$ e sfruttando il cambio di variabile $x = \cos{\theta}$ per il calcolo dell'integrale,
\begin{align*}
	\HermitianProduct{\FirstChebyshevPolynomial_n}{\FirstChebyshevPolynomial_m}
	&= \int_{-1}^1 (1 - x^2)^{-\frac{1}{2}}\FirstChebyshevPolynomial_n(x)\Conjugate{\FirstChebyshevPolynomial_m(x)} dx,\\
	&= \int_\pi^0 (1 - (\cos{\theta})^2)^{-\frac{1}{2}} \cos{n \theta} \cos{m \theta} (- \sin{\theta}) d\theta,\\
	&= \int_0^\pi \cos{(n \theta)} \cos{(m \theta)} d\theta. 
\end{align*}
\par Abbiamo poi
\begin{itemize}
	\item $\HermitianProduct{\FirstChebyshevPolynomial_0}{\FirstChebyshevPolynomial_0} = \int_0^\pi d\theta = \pi$;
	\item supposto $n = m \neq 0$, $\HermitianProduct{\FirstChebyshevPolynomial_n}{\FirstChebyshevPolynomial_m} = \int_0^\pi (\cos{n \theta})^2 d\theta = \int_0^\pi \frac{\cos{2 n \theta} + 1}{2} d\theta = \left [\frac{\sin{2 n\theta}}{4i} + \frac{\theta}{2} \right ]_0^\pi = \frac{\pi}{2}$;
	\item supposto $n \neq m$, $\HermitianProduct{\FirstChebyshevPolynomial_n}{\FirstChebyshevPolynomial_m} = \int_0^\pi \cos{(n\theta)} \cos{(m\theta)} d\theta = \int_0^\pi \frac{\cos{(n + m)\theta} + \cos{(n - m)\theta}}{2} d\theta = \frac{1}{2} \left [ \frac{\sin{(n + m)\theta}}{n + m} + \frac{\sin{(n - m)\theta}}{n - m} \right ]_0^\pi = 0$. \EndProof
\end{itemize}
\begin{Corollary}
	Dati i polinomi di \Chebyshev\ di prima specie $(\FirstChebyshevPolynomial_n)_{n \in \mathbb{N}}$, abbiamo, per ogni $n \in \mathbb{N}$
	\begin{itemize}
		\item gli zeri di $\FirstChebyshevPolynomial_n$ sono $\left ( \cos{\frac{(2 k - 1)\pi}{2n}} \right )_{k \in n}$;
		\item il massimo di $\FirstChebyshevPolynomial_n$ su $[-1;1]$ \`e $1$ e viene assunto nei punti $\left ( \cos \frac{2k\pi}{n} \right )_{k \in \lfloor \frac{n}{2} \rfloor}$;
		\item il minimo di $\FirstChebyshevPolynomial_n$ su $[-1;1]$ \`e $- 1$ e viene assunto nei punti $\left ( \cos \frac{(2k + 1)\pi}{n} \right )_{k \in \lfloor \frac{n}{2} \rfloor}$;
		\item $\FirstChebyshevPolynomial_n$ ha $n + 1$ estremi locali su $[-1;1]$, alternati tra massimi e minimi.
	\end{itemize}
\end{Corollary}
\Proof Fissiamo $n \in \mathbb{N}$.
\par Cerchiamo gli zeri di $\FirstChebyshevPolynomial_n$ nell'intervallo $[-1;1]$: essi possono essere messi nella forma $\cos{\theta}$ per opportuno $\theta \in \mathbb{R}$. Abbiamo che $\FirstChebyshevPolynomial_n(\cos(\theta)) = \cos(n \theta)$ \`e nullo se e solo se $\Exists{k \in \mathbb{Z}}{n \theta = \frac{\pi}{2} + k\pi}$, equivalente a $\Exists{k \in \mathbb{Z}}{\theta = \frac{(2k + 1)\pi}{2n}}$. Quindi gli zeri in $[-1;1]$ di $\FirstChebyshevPolynomial_n$ sono $\cos \frac{(2k + 1)\pi}{2n}$ con $k \in n$: essi sono in numero $n$ e dunque sono tutte le radici del polinomio $\FirstChebyshevPolynomial_n$.
\par Cerchiamo i punti di massimo e minimo di $\FirstChebyshevPolynomial_n$ nell'intervallo $[-1;1]$: essi possono essere messi nella forma $\cos{\theta}$ per opportuno $\theta \in \mathbb{R}$. Abbiamo
\[
	\FirstChebyshevPolynomial_n(\cos(\theta)) = \cos(n \theta) =
	\begin{cases}
		1\text{ se e solo se }\Exists{k \in \mathbb{Z}}{\theta = \frac{2k\pi}{n}},\\
		- 1\text{ se e solo se }\Exists{k \in \mathbb{Z}}{\theta = \frac{(2k + 1)\pi}{n}}.
	\end{cases}
\]
\par Quindi i massimi di $\FirstChebyshevPolynomial_n$ sono $\cos \frac{2k\pi}{n}$ con $k \in \lfloor \frac{n}{2} \rfloor$ e i minimi $\cos \frac{2(k + 1)\pi}{n}$ con $k \in \lfloor \frac{n - 1}{2} \rfloor$.
\par Infine,
\begin{itemize}
	\item se $n$ \`e pari, poniamo $a = \frac{n}{2}$: abbiamo $\left \lfloor \frac{n}{2} \right \rfloor + 1 + \left \lfloor \frac{n - 1}{2} \right \rfloor  + 1 = a + 1 + \left \lfloor a - \frac{1}{2} \right \rfloor + 1 = a + 1 + a - 1 + 1= 2a + 1 = n + 1$.
	\item se $n$ \`e dispari, poniamo $a = \frac{n - 1}{2}$: abbiamo $\left \lfloor \frac{n}{2} \right \rfloor + 1 + \left \lfloor \frac{n - 1}{2} \right \rfloor  + 1 = \left \lfloor \frac{n - 1 + 1}{2} \right \rfloor + 1 + a + 1 = \left \lfloor a + \frac{1}{2} \right \rfloor + 1 + a + 1 = a + 1 + a + 1 = 2a + 2 = n - 1 + 2 = n + 1$. \EndProof 
\end{itemize}
\begin{Theorem}
	Sia $n \in \NotZero{\mathbb{N}}$ e sia $\FirstChebyshevPolynomial_n$ il polinomio di \Chebyshev di prima specie di grado $n$. Allora,
	\begin{itemize}
		\item $\frac{\FirstChebyshevPolynomial_n}{2^{n - 1}}$ \`e monico;
		\item se $\Polynomial \in \RingAdjunction{C}{x}$ \`e di grado $n$ e monico, allora, considerando $\FirstChebyshevPolynomial_n$ e $\Polynomial$ sul dominio $[-1;1]$, abbiamo $\Norm{\frac{\FirstChebyshevPolynomial_n}{2^{n - 1}}}[\infty] \leq \Norm{\Polynomial}[\infty]$.
	\end{itemize}
\end{Theorem}
\Proof Se $n = 1$, allora $\frac{\FirstChebyshevPolynomial_n}{2^{n - 1}}(x) = \frac{x}{2^{1 - 1}} = x$.
\par Supponiamo ora
\[
	\ForAll{m \in n}{\LeadingCoefficient{\frac{\FirstChebyshevPolynomial_m}{2^{m - 1}}} = 1}.
\]
\par Abbiamo $\LeadingCoefficient{\FirstChebyshevPolynomial_n(x)} = \LeadingCoefficient{2x\FirstChebyshevPolynomial_{n - 1} - \FirstChebyshevPolynomial_{n - 2}} = \LeadingCoefficient{2x\FirstChebyshevPolynomial_{n - 1}} = 2\LeadingCoefficient{\FirstChebyshevPolynomial_{n - 1}}$. Da cui $\LeadingCoefficient{\frac{\FirstChebyshevPolynomial_n}{2^{n - 1}}} = \frac{2 \LeadingCoefficient{\FirstChebyshevPolynomial_{n - 1}}}{2^{n - 1}} = 1$.
\par Quindi, per induzione, $\frac{\FirstChebyshevPolynomial_n}{2^{n - 1}}$ \`e monico.
\par Ora, $\Norm{\frac{\FirstChebyshevPolynomial_n}{2^{n - 1}}}[\infty] = \frac{\Norm{\FirstChebyshevPolynomial_n}[\infty]}{2^{n - 1}} = 2^{1 - n}$.
\par Supponiamo $\Polynomial \in \RingAdjunction{C}{x}$ di grado $n$ e monico tale che $\Norm{\Polynomial}[\infty] \leq \Norm{\frac{\FirstChebyshevPolynomial_n}{2^{n - 1}}}[\infty] = 2^{1 - n}$.
\par Poniamo $\VarPolynomial = \Polynomial - \frac{\FirstChebyshevPolynomial_n}{2^{n - 1}}$. Abbiamo $\PolynomialDegree{\VarPolynomial} \leq n - 1$, quindi o $\VarPolynomial = 0$ o $\PolynomialDegree$ ha al pi\`u $n - 1$ radici distinte.
\par D'altra parte,
\begin{itemize}
	\item sia $M \in [-1;1]$ massimo di $\FirstChebyshevPolynomial_n$: abbiamo $\VarPolynomial(M) = \Polynomial(M) - \frac{\FirstChebyshevPolynomial_n(M)}{2^{n - 1}} = \Polynomial(M) - 2^{1 - n} < 0$;
	\item sia $m \in [-1;1]$ massimo di $\FirstChebyshevPolynomial_n$: abbiamo $\VarPolynomial(m) = \Polynomial(m) - \frac{\FirstChebyshevPolynomial_n(m)}{2^{n - 1}} = \Polynomial(m) + 2^{1 - n} > 0$.
\end{itemize}
\par Per la continuit\`a di $\VarPolynomial$ e poich\'e $\VarPolynomial$ cambia segno $n$ volte, $\VarPolynomial$ deve avere almeno $n$ radici. Dunque, necessariamente, $\VarPolynomial = 0$, a cui $\Polynomial = \frac{\FirstChebyshevPolynomial_n}{2^{n - 1}}$. \EndProof
\begin{Definition}
	Chiamiamo \Define{polinomi di \Chebyshev\ di seconda specie}[di \Chebyshev\ di seconda specie][polinomi] gli unici polinomi $(\SecondChebyshevPolynomial_n)_{n \in \mathbb{N}}$ tali che
	\begin{itemize}
		\item essi siano ortogonali rispetto al prodotto scalare $\ScalarProduct{\Polynomial}{\VarPolynomial} = \int_{-1}^1 (1 - x^2)^{\frac{1}{2}} \Polynomial(x)\VarPolynomial(x)dx$;
		\item $\SecondChebyshevPolynomial_0 = 1$;
		\item $\SecondChebyshevPolynomial_1 = 2x$.
	\end{itemize}
\end{Definition}
