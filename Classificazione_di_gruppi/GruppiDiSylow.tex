\section{Gruppi di Sylow.}\label{GruppiDiSylow}
\begin{Definition}
	Dato $\Prime \in \mathbb{Z}$ primo,
	si chiama \Define{$\Prime$-gruppo} un qualsiasi gruppo
	in cui tutti gli elementi hanno per ordine una potenza di $\Prime$;
	inoltre, dato un gruppo finito $\Group$,
	chiamiamo \Define{$\Prime$-Sylow}
	un $\Prime$-sottogruppo di $\Group$ di ordine
	coprimo col suo indice.
\end{Definition}
\begin{Theorem}
	Sia $\Prime \in \mathbb{Z}$ primo.
	Sia $\Group$ un $\Prime$-gruppo.
	$\GroupCenter{\Group}$ non \`e banale.
\end{Theorem}
\Proof
Per l'equazione delle classi,
$\GroupOrder{\GroupCenter{\Group}} =
\GroupCenter{\Group} +
sum_{x \in R \SetMin \GroupCenter{\Group}}
\frac{\GroupOrder{\Group}}{\Stabilizer{x}}$,
dove l'azione \`e l'azione di coniugio e
$R$ \`e un sistema di rappresentanti delle orbite.
\par
Il membro sinistro e la sommatoria sono entrambi
congruenti a $0$ modulo $\Prime$, dunque lo stesso
deve valere anche per $\GroupCenter{\Group}$, che non
pu\`o dunque essere costituito da un solo elemento.
\EndProof
\begin{Corollary}
	Sia $\Prime \in \mathbb{Z}$ primo.
	Se $\Group$ \`e un $\Prime$-gruppo di ordine
	$\Prime^2$, allora $\Group$ \`e abeliano.
\end{Corollary}
\Proof
Per il teorema precedente, $\GroupCenter{\Group}$
non \`e banale.
\par
Supponiamo $\GroupOrder{\GroupCenter{\Group}} = p$.
Allora esiste $x \in \Group \SetMin \GroupCenter{\Group}$ e
$\GroupCenter{\Group}$ \`e un sottogruppo proprio
di $\Centralizer{x}$, il quale ha dunque cardinalit\`a
$\Prime^2$. Ma allora $\Centralizer{x} = \Group$ e
$x \in \GroupCenter{\Group}$, contro le ipotesi.
Dunque deve essere $\GroupCenter{\Group} = \Group$
e $\Group$ \`e abeliano.
\EndProof
\begin{Theorem}
\TheoremName{Teorema di Cauchy per i gruppi abeliani}[di Cauchy per i gruppi
abeliani][teorema]
	Sia $\Group$ un gruppo abeliano e sia
	$\Prime \in \mathbb{Z}$ primo tale che
	$\Prime | \GroupOrder{\Group}$.
	Esiste $x \in \Group$ tale che
	$\GroupOrder{x} = \Prime$.
\end{Theorem}
\Proof
Procediamo per induzione su $\GroupOrder{\Group}$.
\par
Per caso base, scegliamo $\GroupOrder{\Group} = \Prime$.
$\Group$ \`e allora isomorfo al gruppo ciclico
$\Quotient{\mathbb{Z}}{\SpanIdeal{\Prime}}$,
ove ogni elemento non nullo ha per ordine $\Prime$.
\par
Supponiamo il teorema provato per
$\Prime \leq \GroupOrder{\Group} < n \in \mathbb{Z}$
e proviamolo per $\GroupOrder{\Group} = n$.
Sia $x \in \Group$ tale che $x \neq 1$:
$\Prime$ divide $\GroupOrder{\SpanGroup{x}}$ o
$\GroupIndex{\Group}{\SpanGroup{x}}$.
\par
Se $\Prime | \GroupOrder{\SpanGroup{x}}$, allora
$x^{\frac{\GroupOrder{x}}{\Prime}}$ ha ordine $\Prime$.
\par
Se $\Prime | \GroupIndex{\Group}{\SpanGroup{x}}$,
allora, per ipotesi induttiva, esiste $y \in \Group$ tale che
$\Coset{y}{\SpanGroup{x}} \in \Quotient{\Group}{\SpanGroup{x}}$ abbia
ordine $\Prime$. Ma allora $\Prime | \GroupOrder{y}$ e quindi,
$y^{\frac{\GroupOrder{x}}{\Prime}}$ ha ordine $\Prime$.
\EndProof
\begin{Theorem}
	Sia $\Prime \in \mathbb{Z}$ primo e sia
	$\Group$ un $\Prime$-gruppo finito.
	Sia $n \in \mathbb{N}$ tale che $\Prime^n | \GroupOrder{\Group}$.
	Esiste un sottogruppo normale
	$\NormalSubgroup \IsNormalSubgroup \Group$ tale che
	$\GroupOrder{\NormalSubgroup} = \Prime^n$.
\end{Theorem}
\Proof
Se $n = 1$, la tesi segue immediatamente per applicazione del teorema di Cauchy
per i gruppi abeliani.
\par
Assumiamo la tesi provata per $n$ e proviamola per $n + 1$.
Assumiamo anche $\Prime^{n + 1} | \GroupOrder{\Group}$.
Poich\'e $\GroupCenter{\Group}$ \`e abeliano,
per il teorema di Cauchy esiste un sottogruppo $\NormalSubgroup$ di
$\GroupCenter{\Group}$ di ordine $\Prime$; inoltre abbiamo
$\NormalSubgroup \IsNormalSubgroup \Group$.
Quindi $\Quotient{\Group}{\NormalSubgroup}$ ha ordine
$\frac{\GroupOrder{\Group}}{\Prime}$.
\par
Per ipotesi induttiva, esiste
$\Subgroup \IsSubgroup \Quotient{\Group}{\NormalSubgroup}$
di ordine
$\GroupOrder{\Subgroup} = \Prime^n$.
Per il teorema di corrispondenza di gruppi,
$\bigcup_{X \in \Subgroup} X$ \`e un sottogruppo di $\Group$:
il suo ordine \`e
$\GroupOrder{\Subgroup} \GroupOrder{\NormalSubgroup} =
\Prime^n \Prime = \Prime^{n + 1}$.
\EndProof
\begin{Theorem}
\TheoremName{Primo teorema di Sylow}[primo di Sylow][teorema]
	Sia $\Group$ un gruppo finito.
	Per ogni $\Prime \in \mathbb{Z}$ primo tale che
	$\Prime | \GroupOrder{\Group}$, $\Group$ ammette
	un $\Prime$-Sylow $\Sylow$.
\end{Theorem}
\Proof
Siano $\Prime \in \mathbb{Z}$ primo e $n \in \mathbb{N}$
tale che $\Prime^n | \Cardinality{\Group}$, ma
$\Prime^{n + 1} \NotDivide \Cardinality{\Group}$. Poniamo
$m = \frac{\GroupOrder{\Group}}{\Prime^n}$.
\par
Sia $\Class$ l'insieme di tutti i sottoinsiemi di
$\Group$ di cardinalit\`a $\Prime^n$.
Consideriamo l'azione di moltiplicazione a sinistra
di $\Group$ su $\Class$: per ogni $\Operator \in \Group$
e $X \in \Class$ abbiamo $\Operator \cdot X =
\lbrace y \in \Group | \Exists{x \in X}
{\Operator x = y} \rbrace$.
\par
Ora, per ogni $X \in \Class$ e $x \in \Class$,
abbiamo, per ogni $\Operator \in \Stabilizer{X}$,
$\Operator x \in X$, da cui $\GroupOrder{\Stabilizer{X}} \leq
\Cardinality{X} = \Prime^n$.
\par
La cardinalit\`a di $\Class$ \`e data da
$\binom{\Prime^n m}{\Prime^n}$. Per il teorema binomiale,
$\binom{\Prime^n m}{\Prime^n}$ \`e il coefficiente
che accompagna il monomio $x^{Prime^n}y{\Prime^n m -\Prime^n}$ nel polinomio
$(x + y)^{\Prime^n m} \in \RingAdjunction{\mathbb{Z}{\SpanIdeal{\Prime}}}{x,y}$.
Abbiamo
$(x + y)^{\Prime^n m} =
(x^{\Prime^n} + y^{\Prime^n})^m$,
quindi, modulo $\Prime$, abbiamo
$\binom{\Prime^n m}{\Prime^n} \Congruent
\binom{m}{1} = m$.
Ne deduciamo che esiste $X \in \Class$ tale che
$\Cardinality{\Orbit{X}}$ non sia multipla di $\Prime$;
dunque $\GroupOrder{\Stabilizer{X}}$ contiene tanti fattori
$\Prime$ quanti $\GroupOrder{\Group}$.
Poich\'e abbiamo gi\`a provato che $\GroupOrder{\Stabilizer{X}} \leq
\Prime^n$, $\Stabilizer{X}$ \`e un $\Prime$-Sylow di $\Group$.
\EndProof
\begin{Corollary}
	Siano $\Group$ un gruppo finito,
	$n, \Prime \in \mathbb{Z}$ con
	$\Prime$ primo.
	Se $\Prime^n | \GroupOrder{\Group}$, allora
	$\Group$ ammette un sottogruppo $\Subgroup$
	di ordine $\GroupOrder{\Subgroup} = \Prime^n$.
\end{Corollary}
\Proof
Segue immediatamente dall'esistenza di un $\Prime$-Sylow e
dal fatto che ogni $\Prime$-gruppo finito ammette sottogruppi di ordine
$\Prime^n$ per tutti gli $n \in \mathbb{N}$ tali che $\Prime^n$ divida
l'ordine del $\Prime$-gruppo.
\EndProof
\begin{Corollary}
\TheoremName{Teorema di Cauchy per i gruppi}[di Cauchy per i gruppi][teorema]
	Siano $\Group$ un gruppo finito e
	$\Prime \in \mathbb{Z}$ un primo tale che
	$\Prime | \Cardinality{\Group}$.
	Esiste un elemento $x \in \Group$ di ordine $\Prime$.
\end{Corollary}
\Proof
L'enunciato del teorema coincide col caso $n = 1$ del corollario precedente.
\EndProof
\begin{Corollary}
	Sia $\Prime \in \mathbb{Z}$ primo.
	Sia $\Group$ un $\Prime$-gruppo.
	Esiste $n \in \mathbb{N}$ tale che
	$\GroupOrder{\Group} = \Prime^n$.
\end{Corollary}
\Proof
Se $\Prime' \in \mathbb{Z}$ \`e un primo tale che
$\Prime' | \GroupOrder{\Group}$,
allora per il teorema di Cauchy esiste in $\Group$ un elemento di
ordine $\Prime'$.
Quindi, per definizione di $\Prime$-gruppo,
deve essere $\Prime' = \Prime$.
\EndProof
\begin{Theorem}
\TheoremName{Secondo teorema di Sylow}[secondo di Sylow][teorema]
	Siano $\Group$ un gruppo finito e 
	$\Prime \in \mathbb{Z}$ un primo tale che
	$\Prime | \Cardinality{\Group}$.
	Tutti i $\Prime$-Sylow di $\Group$ sono coniugati tra loro.
\end{Theorem}
\Proof
Sia $\Sylow_1$ un $\Prime$-Sylow e
sia $\Set$ l'insieme di tutti i sottogruppi di $\Group$
coniugati a $\Sylow$.
Consideriamo l'azione di coniugio di $\Group$ su $\Set$.
\par
Poich\'e $\Stabilizer{\Sylow_1} = \Normalizer{\Sylow_1}$,
abbiamo $\Sylow_1 \IsSubgroup \Stabilizer{\Sylow_1}$ e dunque
$\Cardinality{\Set} =
\Cardinality{\Orbit{\Sylow_1}} =
\GroupIndex{\Group}{\Stabilizer{\Sylow_1}}$
\`e coprimo con $\Prime$.
\par
Sia ora $\Sylow_2$ un altro $\Prime$-Sylow e
consideriamo l'azione di coniugio di $\Sylow_2$ su $\Set$.
Abbiamo, per ogni $X \in \Set$,
$\Cardinality{\Orbit{x}} =
\frac{\GroupOrder{\Sylow_2}}{\Stabilizer{\Orbit{X}}} |
\GroupOrder{\Sylow_2}$.
Dunque tutte le orbite per l'azione di $\Sylow_2$ hanno per carinalit\`a
una potenza di $\Prime$. Poich\'e $\Set$ non \`e multiplo di $\Prime$ deve
esistere $X \in \Set$ tale che $\Orbit{X} = \lbrace X \rbrace$.
Pertanto, $\Sylow_2 \IsSubgroup \Normalizer{X}$
\par
Ora, poich\'e $\Sylow_2 \IsSubgroup \Normalizer{X}$, $\Sylow_2 X$ \`e un
$\Prime$-gruppo. Poich\'e $X$ \`e un $\Prime$-Sylow, non solo
$\Sylow_2 X = \Sylow_2$, ma anche $\Sylow_2 X = X$. Dunque $\Sylow_2 = X$
e $\Sylow_2$ risulta coniugato a $\Sylow_1$.
\EndProof
\begin{Theorem}
\TheoremName{Terzo teorema di Sylow}[terzo di Sylow][teorema]
	Sia $\Group$ un gruppo finito e
	$\Prime \in \mathbb{Z}$ un primo tale che
	$\Prime | \Cardinality{\Group}$.
	Sia $n$ il numero di $\Prime$-Sylow di $\Group$.
	Indicato con $\Sylow$ un qualsiasi $\Prime$-Sylow di $\Group$,
	abbiamo
	\begin{itemize}
		\item $n \Congruent 1 (\Modulo \Prime)$;
		\item $n = \Index{\Group}{\Normalizer{\Sylow}}$
	\end{itemize}
\end{Theorem}
\Proof
Riprendiamo le mosse dalla dimostrazione del secondo teorema di
Sylow. Abbiamo provato incidentalmente che $\Sylow_2$ \`e l'unico punto
fisso di dell'azione di $\Sylow_2$ sull'insieme $\Set$ e dunque che tutte
le orbite distinte da $\lbrace \Sylow_2 \rbrace$ hanno cardinalit\`a che
\`e una potenza di $\Prime$. Quindi, in particolare,
$n \Congruent 1 (\Modulo \Prime)$.
\par
Abbiamo gi\`a provato nella dimostrazione del teorema precedente che $n =
\Index{\Group}{\Normalizer{\Sylow}}$.
\EndProof
