\section{Gruppi simmetrici e alternanti.}\label{GruppiSimmetrici}
\begin{Definition}
	Sia $\Class$ una classe.
	Si chiama \Define{gruppo simmetrico}[simmetrico][gruppo]
	di $Class$ il gruppo $\Automorphisms{\Class}$.
	Quando $\Class$ \`e l'insieme di tutt i numeri naturali
	da $1$ a $n \in \mathbb{N}$, allora denotiamo
	$\SymmetricGroup{n}$ il gruppo simmetrico di $\Class$.
	\par
	Gli elementi di $\Automorphisms{\Class}$ si chiamano
	\Define{permutazioni}[permutazione].
	Sia $\Permutation \in \Automorphisms{Class}$ tale che la classe
	$\Class \SetMin \Fixed{\Class}{\Permutation}$ sia finita:
	diciamo che $\Permutation$ \`e \Define{finita}[finita][permutazione]
	si usa indicare $\Permutation$ con la notazione
	$\Permutation[x_1 & ... & x_k][y_1 & ... & y_k]$
	dove $(x_i)_{i = 1}^k$ \`e una famiglia di elementi
	di $\Class$ che comprende
	$\Class \SetMin \Fixed{\Class}{\Permutation}$ e
	$(y_k)_{i = 1}^k = (\Permutation(x_k))_{i = 1}^k$.
\end{Definition}
\begin{Theorem}
	Siano $\Class_1$ e $\Class_2$ due classi.
	Se $\Class_1$ e $\Class_2$ sono equicardinali,
	allora $\Automorphisms{\Class_1}$ e $\Automorphisms{\Class_2}$
	sono gruppi isomorfi.
\end{Theorem}
\Proof
Sia $Map: \Class_1 \rightarrow \Class_2$
un'applicazione biettiva.
Sia $\Isomorphism:
\Automorphisms{\Class_1} \rightarrow
\Automorphisms{\Class_2}$
l'applicazione che a $\Automorphism \in \Automorphisms{\Class_1}$
associa $\Map\Automorphism\Map^{-1} \in \Automorphisms{\Class_2}$.
Segue direttamente dalle definizioni che $\Isomorphism$
\`e un isomorfismo.
\EndProof
\begin{Corollary}
	Sia $\Class$ un insieme finito: $\Isomorphic{\Automorphisms{\Class}}
	\Isomorphic{\SymmetricGroup{\Cardinality{\Class}}}$.
\end{Corollary}
\Proof
Segue immediatamente dal teorema precedente.
\EndProof
\begin{Theorem}
	Per ogni $n \in \mathbb{N}$, abbiamo
	$\GroupOrder{\SymmetricGroup{n}} = \Factorial{n}$.
\end{Theorem}
\Proof
\`E immediato notare che
$\GroupOrder{\SymmetricGroup{0}} = 1$ e
$\GroupOrder{\SymmetricGroup{1}} = 1$.
\par
Supponiamo il teorema provato per $\SymmetricGroup{n}$
e proviamolo per $\SymmetricGroup{n + 1}$.
Consideriamo l'applicazione
$\Map: \SymmetricGroup{n} \times \lbrace 1, ..., n + 1 \rbrace \rightarrow
\SymmetricGroup{n + 1}$
che a $(\Permutation,m) \ in \SymmetricGroup{n} \times \lbrace 1, ..., n + 1 \rbrace$
associa $\Map(\Permutation,m)$ definita da
$\Map(\Permutation,m) =
\begin{cases}
(\Cycle{m & n + 1}\Permutation)(x)\text{ se }x \leq n;\\
m\text{ se }x = n + 1.
\end{cases}$
\`E immediato verificare che l'applicazione \`e biettiva.
Dunque $\GroupOrder{\SymmetricGroup{n + 1}} =
\GroupOrder{\SymmetricGroup{n}} \Cardinality{\lbrace 1, ..., n + 1 \rbrace} =
\Factorial{n}(n + 1) = \Factorial{n + 1}$.
\EndProof
\begin{Theorem}
\TheoremName{Teorema di Cayley}[di Cayley][teorema]
	Sia $\Group$ un gruppo. Esiste
	$\Subgroup \IsSubgroup \Automorphisms{\Group}$,
	dove $\Automorphisms{Group}$ \`e il gruppo delle
	permutazioni del sostegno di $\Group$ (e non il gruppo
	degli automorfismi del gruppo $\Group$),
	tale che $\Isomorphic{\Group}{\Subgroup}$.
\end{Theorem}
\Proof
\`E sufficiente prendere come $\Subgroup$
l'immagine dell'azione di traslazione a sinistra di
$\Group$ su se stesso: l'azione \`e fedele, da cui l'isomorfismo.
\EndProof
\begin{Definition}
	Siano $\Class$ una classe e $\Permutation \in \Automorphisms{\Class}$.
	Diciamo che $\Permutation$ \`e un \Define{ciclo} quando l'azione di
	$\SpanGroup{\Permutation}$ su $\Class$ prevede un'unica orbita di
	cardinalit\`a maggiore o uguale di $2$.
	Se tale orbita \`e finita, cio\`e se
	$\Permutation =
	\Permutation[x_1 & ... & x_{k - 1} & x_k][x_2 & ... & x_k & x_1]$,
	allora $\Permutation$ si chiama \Define{ciclo finito}[finito][ciclo] o
	\Define{$k$-ciclo}{$k$-ciclo}[ciclo] e si
	denota in maniera pi\`u compatta
	$\Cycle{x_1 & ... & x_k}$.
	Un $2$-ciclo si chiama pi\`u comunemente
	\Define{trasposizione}.
	La \Define{lunghezza}[di un ciclo][lunghezza] di un ciclo \`e la cardinalit\`a
	della sua unica orbita.
\end{Definition}
\begin{Theorem}
	Sia $\Permutation$ un ciclo finito.
	$\Permutation$ \`e prodotto di trasposizioni.
\end{Theorem}
\Proof
Sia $\Permutation = \Cycle{x_1 & ... & x_k}$.
Abbiamo
$\Permutation = \prod_{i = 1}^{k - 1} \Cycle{x_i & x_{i + 1}}$.
\EndProof
\begin{Definition}
	Siano $\Permutation_1$ e $\Permutation_2$
	due cicli di $\Automorphisms{\Class}$.
	Se $\Class \SetMin \Fixed{\Class}{\Permutation_1} \cap
	\Class \SetMin \Fixed{\Class}{\Permutation_2}$ si dice
	che $\Permutation_1$ e $\Permutation_2$ sono
	\Define{cicli disgiunti}[disgiunti cicli].
\end{Definition}
\begin{Theorem}
	Ogni famiglia di cicli disgiunti commuta.
\end{Theorem}
\Proof
Segue immediatamente dalle definizioni.
\EndProof
\begin{Theorem}
	Siano $\Class$ una classe e $\Permutation \in \Automorphisms{\Class}$.
	$\Permutation$ pu\\`o essere scritto come prodotto\footnote{Eventualmente infinito, nel qual caso il prodotto `e l'unica permutazione che prolunga tutti i fattori.} di cicli disgiunti
	in modo unico a meno dell'ordine dei fattori e di moltiplicazione per $1$-cicli;
	pi\`u precisamente, ogni gruppo simmetrico \`e generato dai suoi cicli.
\end{Theorem}
\Proof
Consideriamo l'azione di $\SpanGroup{\Permutation}$ su $\Class$.
Costruiamo la famiglia di permutazioni
$(\Permutation_i)_{i \in \Quotient{\Class}{\SpanGroup{\Permutation}}}$
tale che $\Permutation_i(x) =
\begin{cases}
\Permutation(x)\text{, se }x \in i,\\
x\text{, altrimenti}.
\end{cases}$
\`E immediato verificare che
$\prod_{i \in \Quotient{\Class}{\SpanGroup{\Permutation}}} \Permutation_i =
\Permutation$ e che tutti i $\Permutation_i$ sono cicli disgiunti.
D'altra parte, sia
$(\Permutation_i)_{i \in I}$ una famiglia di permutazioni disgiunte
tali che $\prod_{i \in I} \Permutation_i = \Permutation$.
Sia $x \in \Class$. Per ipotesi, $x$ \`e un punto fisso.
di $\Permutation$ o esiste un'unica permutazione $\Permutation_i$,
con $i \in I$,
che trasforma $x$ in un altro punto $\Permutation_i(x) = \Permutation(x)$.
Inoltre, poich\'e $\Permutation_i$ \`e un ciclo, abbiamo
$\ForAll{n \in \mathbb{N}}{\Permutation_i^n(x) = \Permutation^n(x)}$.
Ma questo equivale a dire che $\Permutation_i$ \`e una delle
permutazioni della scomposizione precedentemente costruita.
Poich\'e quanto appena affermato vale per tutti gli $x \in \Class$,
tutte le permutazioni della scomposizione $\prod_{i \in I} \Permutation_i$
sono presenti nella scomposizione precedentemente costruita; e poich\'e
nessun prodotto di una sottofamiglia dei fattori della scomposizione precedente
ha $\Permutation$ per prodotto, le due scomposizioni coincidono necessariamente.
\EndProof
\begin{Corollary}
	Siano $\Class$ una classe finita e $\Permutation \in \Automorphisms{\Class}$.
	$\Permutation$ pu\`o essere scritto come prodotto di trasposizioni;
	pi\`u precisamente, ogni gruppo simmetrico finito \`e generato dalle sue
	trasposizioni.
\end{Corollary}
\Proof
Segue direttamente dal teorema precedente e dal fatto
che ogni ciclo pu\`o essere scritto come prodotto di
trasposizioni.
\EndProof
\begin{Theorem}
	Siano $\Class$ una classe e
	$\Permutation, \Permutation_1 \in \Automorphisms{Class}$.
	Se $\Permutation$ \`e un ciclo, allora il suo coniugato
	$\Permutation_1\Permutation\Permutation_1^{-1}$ \`e un ciclo
	di medesima lunghezza.
\end{Theorem}
\Proof
Sia $x \in \Class$.
Si verifica direttamente che
$x \in \Fixed{\Class}{\SpanGroup{\Permutation_1\Permutation\Permutation_1^{-1}}}$
se e solo se
$\Permutation_1^{-1}(x) \in \Fixed{\Class}{\SpanGroup{\Permutation}}$.
\par
Proviamo che
$\Class \SetMin
\Fixed{\Class}{\SpanGroup{\Permutation_1\Permutation\Permutation_1^{-1}}}$ costituisce
un'unica orbita per l'azione di
$\SpanGroup{\Permutation_1\Permutation\Permutation_1^{-1}}$.
Siano $x, y \in
\Class \SetMin
\Fixed{\Class}{\SpanGroup{\Permutation_1\Permutation\Permutation_1^{-1}}}$.
Poich\'e $\Permutation$ \`e un ciclo e per quanto appena provato riguardo ai punti fissi
di $\Permutation_1\Permutation\Permutation_1^{-1}$, esiste $n \in \mathbb{Z}$ tale che
$\Permutation_1^{-1}(y) = \Permutation^n\Permutation_1^{-1}(x)$, da cui
$y = \Permutation_1\Permutation^n\Permutation_1^{-1}(x)$ e quindi
$\Class \SetMin
\Fixed{\Class}{\SpanGroup{\Permutation_1\Permutation\Permutation_1^{-1}}}$ costituisce
l'unica orbita per l'azione di
$\SpanGroup{\Permutation_1\Permutation\Permutation_1^{-1}}$.
\par
L'uguaglianza della lunghezza dei due cicli segue dall'equicardinalit\`a
delle classi dei punti fissi.
\EndProof
\begin{Corollary}
	Sia $\Class$ una classe e sia
	$\Permutation \in \Automorphisms{\Class}$.
	Tutte le permutazioni coniugate a $\Permutation$
	si scompogono in cicli di ugual numero e uguali lunghezze.
\end{Corollary}
\Proof
Segue immediatamente dal teorema precedente,
considerando che la permutazione coniugata a $\Permutation$ tramite
una qualsiasi permutazione $\Permutation_1$ \`e il prodotto dei
coniugati dei fattori della scomposizione in cicli disgiunti di $\Permutation$
per mezzo della stessa permutazione $\Permutation_1$.
\EndProof
\begin{Definition}
	Sia $\Class$ un ordinamento totale
	e sia $\Permutation \in \Automorphisms{\Class}$.
	Definiamo \Define{inversione} una qualsiasi coppia
	$(x,y) \in \Class^2$ tale che
	$x < y$ e $\Permutation(x) > \Permutation(y)$.
\end{Definition}
\begin{Definition}
	Sia $\Class$ un ordinamento totale finito.
	Definiamo \Define{segno}[di una permutazione][segno] o
	\Define{segnatura}[di una permutazione][segnatura] o
	\Define{parit\`a}[di una permutationze][parit\`a]
	l'applicazione $\PermutationSign: \Automorphisms{\Class} \rightarrow
	\Invertible{\Quotient{\mathbb{Z}}{\SpanIdeal{3}}}$ che a
	$\Permutation \in \Automorphisms{\Class}$
	associa $\PermutationSign(\Permutation) =
	\begin{cases}
		1\text{, se il numero di inversioni di }\Permutation\text{ \`e pari};\\
		-1\text{, se il numero di inversioni di }\Permutation\text{ \`e dispari}.
	\end{cases}$
	Inoltre diciamo che una permutazione $\Permutation$ \`e
	\Define{pari}[pari][permutazione]
	quando $\PermutationSign{\Permutation} = 1$ e
	\Define{dispari}[dispari][permutazione]
	quando $\PermutationSign{\Permutation} = -1$.
\end{Definition}
\begin{Theorem}
	Sia $\Class$ un ordinamento totale finito
	e siano $\Permutation, \Transposition \in \Automorphisms{\Class}$.
	Se $\Transposition$ \`e una trasposizione, allora
	$\PermutationSign{\Transposition\Permutation} =
	- \PermutationSign{\Permutation}$.
\end{Theorem}
\Proof
Sia $\Transposition = \Cycle{i & j}$.
Tutte le inversioni di $\Permutation$ che non coinvolgono n\'e $i$ n\'e $j$
sussistono in $\Transposition\Permutation$.
Inoltre sussitono tutte le inversioni che coinvolgono $i$ e un elemento maggiore di $j$
e tutte le inversioni che coinvolgono $j$ e un elemento minore di $i$.
Restono solo da valutare tutte le inversioni che coinvolgono $i$ o $j$ con un elemento
$x \in [i,j]$.
\par
Supponiamo che le inversioni di $i$ con elementi di $[i,j]$ nella permutazione
$\Permutation$ siano $n_i$: nella permutazione $\Transposition\Permutation$
gli elementi di $[i,j]$ che costituiscono inversione con $i$ sono tutti e soli quelli
che non la costituiscono in $\Permutation$ e sono diversi da $i$; sono dunque in numero
$j - i - n_i$.
Analogamente, se $n_j$ \`e il numero di inversioni di $j$ con elementi di $[i,j]$,
allora le inversioni di $\Transposition\Permutation$ che coinvolgono $j$ e un elemento
di $[i,j]$ sono $j - i - n_j$.
La differenza tra il numero di inversioni di $\Permutation$ e quello di $\Transposition\Permutation$
\`e dunque $n_i + n_j - (j - i - n_i) - (j - i - n_j) \pm 1 =
2n_i + 2n_j - 2j + 2i \pm 1$ (il pi\`u o il meno dipende da se la coppia $(i,j)$
sia o meno un'inversione in $Permutation$), che \`e un numero dispari, da cui la tesi.
\EndProof
\begin{Corollary}
	Sia $\Class$ un ordinamento totale finito
	e sia $\Permutation \in \Automorphisms{\Class}$.
	$\Permutation$ \`e
	\begin{itemize}
		\item
		pari se e solo se si scrive come prodotto di un numero pari di trasposizioni;
		\item
		dispari se e solo se si scrive come prodotto di un numero dispari di trasposizioni.
	\end{itemize}
	In particolare, la parit\`a del numero di fattori in ogni scrittura di $\Permutation$
	come prodotto di trasposizione \`e la stessa.
\end{Corollary}
\Proof
Segue dal fatto che $\Permutation$ si scrive come prodotto di trasposizioni
e che ogni trasposizione \`e dispari.
\EndProof
\begin{Corollary}
	Sia $\Class$ un ordinamento totale finito.
	L'applicazione $\PermutationSign: \Automorphisms{\Class} \rightarrow
	\Invertible{\Quotient{\mathbb{Z}}{\SpanIdeal{3}}}$ \`e un omomorfismo.
\end{Corollary}
\Proof
Segue direttamente dal corollario precedente considerando
le permutazioni come prodotti di trasposizioni.
\EndProof
\begin{Definition}
	Chiamiamo $n$-esimo
	\Define{gruppo alternante}[alternante][gruppo]
	il nucleo dell'applicazione
	$\PermutationSign: \SymmetricGroup{n} \rightarrow
	\Invertible{\Quotient{\mathbb{Z}}{\SpanIdeal{3}}}$.
\end{Definition}
\begin{Theorem}
	Per ogni $n \in \mathbb{N}$,
	il gruppo alternante $\AlternatingGroup{n}$ \`e un
	sottogruppo normale di $\SymmetricGroup{n}$ di indice $2$.
\end{Theorem}
\Proof
La normalit\`a di $\AlternatingGroup{n}$ segue dal
fatto che il nucleo di un omomorfismo \`e sempre un sottogruppo
normale del suo dominio.
\par
Siano $\Permutation_1, \Permutation_2 \in \SymmetricGroup{n} \SetMin \AlternatingGroup{n}$.
$\Permutation_1$ e $\Permutation_2$ sono permutazioni dispari,
pertanto $\Permutation_1\Permutation_2^{-1} \in \AlternatingGroup{n}$, da cui
$\SymmetricGroup{n} \SetMin \AlternatingGroup{n}$ \`e l'unica classe laterale di
$\AlternatingGroup{n}$ oltre a $\AlternatingGroup{n}$ stesso.
\EndProof
