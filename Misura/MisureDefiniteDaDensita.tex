\section{Misure definite da densit\`a.}
\label{Misura_MisureDefiniteDaDensita}
\begin{Definition}
	Siano dati uno spazio misurabile $(\MeasurableSpace,\SigmaAlgebra)$ munito
  della misura $\Measure$ e un'applicazione misurabile
  $\DensityFunction: \MeasurableSpace \rightarrow [0,+\infty]$. Chiamiamo
  $\VarMeasure: \SigmaAlgebra \rightarrow [0, +\infty]$ tale che
  $\VarMeasure(A) = \int_A f d\Measure$
  \Define{misura definita dalla densit\`a $\DensityFunction$ rispetto a
  $\Measure$}:
  esprimiamo ci\`o anche con la notazione
  $\VarMeasure = \DensityDefinedProbability{\DensityFunction}{\Measure}$
\end{Definition}
\begin{Theorem}
	Una misura definita da densit\`a \`e effettivamente una misura.
\end{Theorem}
\begin{Theorem}
	Siano $\VarMeasure$ e $\Measure$ due misure su
  $(\MeasurableSpace,\SigmaAlgebra)$ e supponiamo esistano
  $\DensityFunction$ e $\VarDensityFunction$
  ($\DensityFunction, \VarDensityFunction:
  \MeasurableSpace \rightarrow [0,+\infty]$) applicazioni misurabili tali che
  $\VarMeasure
    = \DensityDefinedProbability{\DensityFunction}{\Measure}
    = \DensityDefinedProbability{\VarDensityFunction}{\VarMeasure}$, allora
  $\DensityFunction$ e $\VarDensityFunction$ sono equivalenti.
\end{Theorem}
\begin{Definition}
	Con le notazioni del teorema precedene, diciamo che $\DensityFunction$ \`e
  \Define{una versione della densit\`a di $\VarMeasure$ rispetto a $\Measure$} e
  denotiamo $\DensityFunction = \DensityVersion{\VarMeasure}{\Measure}$.
\end{Definition}
\par Osserviamo che la simbologia introdotta dalla definizione precedente \`e un
abuso di notazione:
$\DensityVersion{\VarMeasure}{\Measure}$ \`e in realt\`a una classe di
equivalenza di funzioni e non una sola funzione e sarebbe dunque pi\`u giusto
scrivere
$\DensityFunction \in \DensityVersion{\VarMeasure}{\Measure}$.
