\section{Definizioni e teoremi di base.}
\label{Misura_DefinizioniETeoremiDiBase}
\begin{Definition}
	Un'\Define{algebra di parti}[di parti][algebra]\footnote{Il termine in uso
  solitamente \`e
  \Define{algebra di insiemi}[di insiemi][algebra], ma preferiamo il termine
  ``algebra di parti'' perch\'e consente facili estensioni al caso in cui le
  parti in questione siano parti di una classe, e non solo di un insieme, come
  \`e invece il caso per l'usuale definizione di algebra di insiemi.}
  $\AlgebraOfParts$ su una classe $\MeasureSpace$ \`e una parte di
  $\PowerSet{\MeasureSpace}$ tale che
  \begin{itemize}
    \item $\emptyset \in \AlgebraOfParts$;
    \item $\AlgebraOfParts$ sia chiuso per passaggio al complementare;
    \item $\AlgebraOfParts$ sia chiuso per unione finita.
  \end{itemize}
\end{Definition}
\begin{Theorem}
  Un'algebra $\AlgebraOfParts$ su $\MeasureSpace$ \`e chiusa per
  intersezione finita.
\end{Theorem}
\Proof Siano $\MeasurablePart, \VarMeasurablePart \in \AlgebraOfParts$.
Abbiamo
$\MeasurablePart \cap \VarMeasurablePart
= \Complement{\Complement{\MeasurablePart} \cup \Complement{\VarMeasurablePart}}
\in \AlgebraOfParts$. \EndProof
\begin{Definition}
  Una
  \Define{$\sigma$-algebra} su una classe $\MeasureSpace$ \`e un'algebra
  di parti su $\MeasureSpace$ chiusa per unione numerabile.
\end{Definition}
\begin{Theorem}
  Una $\sigma$-algebra $\SigmaAlgebra$ su $\MeasureSpace$ \`e chiusa per
  intersezione numerabile.
\end{Theorem}
\Proof Sia
$(\MeasurablePart_n)_{n \in \mathbb{N}} \in \SigmaAlgebra^\mathbb{N}$.
Abbiamo
$\bigcap_{n \in \mathbb{N}} \MeasurablePart_n
= \Complement{\bigcup_{n \in \mathbb{N}} \Complement{\MeasurablePart_n}}
\in \SigmaAlgebra$. \EndProof
\begin{Theorem}
  Sia $(\AlgebraOfParts_i)_{i \in I}$ una famiglia di algebre di parti
  sulla stessa classe $\MeasureSpace$.
  $\bigcap_{i \in I} \AlgebraOfParts$ \`e un'algebra di parti su
  $\MeasureSpace$.
\end{Theorem}
\Proof Segue immediatamente dalle definizioni. \EndProof
\begin{Corollary}
  Sia $(\SigmaAlgebra_i)_{i \in I}$ una famiglia di $\sigma$-algebre sulla
  stessa classe $\MeasureSpace$.
  $\bigcap_{i \in I} \SigmaAlgebra_i$ \`e una $\sigma$-algebra su
  $\MeasureSpace$.
\end{Corollary}
\Proof Segue immediatamente dalle definizioni. \EndProof
\begin{Definition}
  Sia $\MeasureSpace$ una classe e sia $S \subseteq \PowerSet{\MeasureSpace}$.
  Chiamiamo
  \begin{itemize}
    \item \Define{algebra di parti generata}[di parti generata][algebra]
      da $S$ in $\MeasureSpace$ la pi\`u piccola algebra di parti
      $\AlgebraOfParts$ di $\MeasureSpace$ tale che
      $S \subseteq \AlgebraOfParts$;
    \item \Define{$\sigma$-algebra generata}[generata][$\sigma$-algebra]
      da $S$ in $\MeasureSpace$ la pi\`u piccola $\sigma$-algebra
      $\SigmaAlgebra$ di $\MeasureSpace$ tale che
      $S \subseteq \SigmaAlgebra$, denotata
      $\GeneratedSigmaAlgebra{S}$.
  \end{itemize}
\end{Definition}
\begin{Definition}
  Sia $(\TopologicalSpace,\Topology)$ uno spazio topologico. Chiamiamo
  $\GeneratedSigmaAlgebra{\Topology}$
  \Define{$\sigma$-algebra di Borel}[di Borel][$\sigma$-algebra]
  e chiamiamo gli elementi di
  $\GeneratedSigmaAlgebra{\Topology}$
  \Define{boreliani}[boreliano].
\end{Definition}
\begin{Definition}
	Uno \Define{spazio misurabile}[misurabile][spazio]
  $(\MeasureSpace,\SigmaAlgebra)$
  \`e una coppia in cui
  \begin{itemize}
    \item $\MeasureSpace$ \`e una classe;
    \item $\SigmaAlgebra$ una $\sigma$-algebra su $\MeasureSpace$.
  \end{itemize}
\end{Definition}
\begin{Definition}
  Sia $(\TopologicalSpace,\Topology)$ uno spazio topologico. Chiamiamo
  $(\TopologicalSpace,\GeneratedSigmaAlgebra{\Topology})$
  \Define{spazio boreliano}[boreliano][spazio].
\end{Definition}
\begin{Definition}
	Siano
  \begin{itemize}
    \item $(\MeasurableSpace,\SigmaAlgebra)$ e
      $(\VarMeasurableSpace,\VarSigmaAlgebra)$ spazi misurabili;
    \item un'applicazione $\MeasurableApplication: \MeasurableSpace \rightarrow
      \VarMeasurableSpace$.
  \end{itemize}
  Se $\ForAll{\VarMeasurablePart \in \VarSigmaAlgebra}
  {\MeasurableApplication^{-1}(\VarMeasurablePart) \in \SigmaAlgebra}$, allora
  si dice che $\MeasurableApplication$ \`e
  \Define{misurabile}[misurabile][applicazione]
  Se inoltre $\MeasurableSpace$ e $\VarMeasurableSpace$ sono spazi boreliani,
  allora
  $\MeasurableApplication$ si dice
  \Define{boreliana}[boreliana][applicazione].
\end{Definition}
\begin{Theorem}
  Con le notazioni del teorema precedente,
  se $\MeasurableApplication$ \`e costante, allora \`e misurabile.
\end{Theorem}
\Proof Sia $c \in \Image{\MeasurableApplication}$.
\par Sia $\VarMeasurablePart \in \VarSigmaAlgebra$.
\par Ora,
\begin{itemize}
  \item se $c \notin \VarMeasurablePart$, allora
    $\MeasurableApplication^{-1} (\VarMeasurablePart)
    = \emptyset \in \SigmaAlgebra$;
  \item se $c \in \VarMeasurablePart$, allora
    $\MeasurableApplication^{-1} (\VarMeasurablePart)
    = \MeasurableSpace \in \SigmaAlgebra$. \EndProof
\end{itemize}
\begin{Theorem}
  Siano
  \begin{itemize}
    \item $(\MeasureSpace_0,\SigmaAlgebra_0)$,
      $(\MeasureSpace_1,\SigmaAlgebra_1)$
      e $(\MeasureSpace_2,\SigmaAlgebra_2)$
      spazi misurabili;
    \item $\MeasurableApplication_0:
      \MeasureSpace_0 \rightarrow \MeasureSpace_1$
      un'applicazione misurabile;
    \item $\MeasurableApplication_1:
      \MeasureSpace_1 \rightarrow \MeasureSpace_2$
      un'applicazione misurabile.
  \end{itemize}
  L'applicazione
  $\MeasurableApplication
  = \MeasurableApplication_1 \circ \MeasurableApplication_0$ \`e
  misurabile.
\end{Theorem}
\Proof Sia $\MeasurablePart \in \SigmaAlgebra_2$. Per la misurabilit\`a
di $\MeasurableApplication_1$, $\MeasurableApplication_1^{-1}(\MeasurablePart)$
\`e misurabile. Per la misurabilit\`a di $\MeasurableApplication_0$,
$\MeasurableApplication^{-1}(\MeasurablePart) =
\MeasurableApplication_0^{-1} (\MeasurableApplication_1^{-1} (\MeasurablePart))$
\`e misurabile. \EndProof
\begin{Corollary}
  Le applicazioni misurabili costituiscono una cateogria i cui oggetti sono gli
  spazi misurabili. Denoteremo tale categoria $\CatMeasurableSpace$.
\end{Corollary}
\Proof Segue direttamente dal teorema precedente. \EndProof
\begin{Definition}
  Data una classe $\MeasureSpace$ e
  $\SigmaAlgebra  \subseteq \PowerSet{\MeasureSpace}$,
  un'applicazione
  $\Measure: \SigmaAlgebra \rightarrow \mathbb{R}$
  si dice
  \begin{itemize}
    \item \Define{additiva}[additiva][applicazione] se
      \begin{itemize}
        \item $\SigmaAlgebra$ \`e un'algebra di parti su
          $\MeasureSpace$;
        \item per ogni coppia
            $(\MeasurablePart,\VarMeasurablePart) \in \SigmaAlgebra^2$
            di parti disgiunte abbiamo
            $\Measure(\MeasurablePart \cup \VarMeasurablePart) =
            \Measure(\MeasurablePart) + \Measure(\VarMeasurablePart)$;
      \end{itemize}
    \item \Define{$\sigma$-additiva}[$\sigma$-additiva][applicazione] se
      \begin{itemize}
        \item $\SigmaAlgebra$ \`e una $\sigma$-algebra su
          $\MeasureSpace$;
        \item per ogni famiglia
          $(\MeasurablePart_n)_{n \in \mathbb{N}} \in \SigmaAlgebra^\mathbb{N}$
          di parti disgiunte due a due abbiamo
          $\Measure \left ( \bigcup_{n \in \mathbb{N}} \MeasurablePart_n\right )
            = \sum_{n \in \mathbb{N}} \Measure(\MeasurablePart_n)$.
      \end{itemize}
  \end{itemize}
  Nell'ipotesi che
  \begin{itemize}
    \item $\SigmaAlgebra$ sia una $\sigma$-algebra;
    \item $\Measure$ sia $\sigma$-additiva;
    \item $\Cod{\Measure} = [0,+\infty]$;
  \end{itemize}
  definiamo
  \begin{itemize}
    \item \Define{misura} sullo spazio misurabile
      $(\MeasureSpace,\SigmaAlgebra)$
      l'applicazione $\Measure$;
    \item \Define{spazio di misura}[di misura][spazio] la terna
      $(\MeasureSpace,\SigmaAlgebra,\Measure)$.
  \end{itemize}
  Inoltre, dato $\MeasurablePart \in \SigmaAlgebra$, diciamo che
  \begin{itemize}
    \item $\MeasurablePart$ contiene \Define{quasi tutti}[tutti][quasi]
      gli elementi di $\MeasureSpace$ se
      $\Measure(\MeasurablePart) = \Measure(\MeasureSpace)$;
    \item $\MeasurablePart$ contiene \Define{quasi nessun}[nessuno][quasi]
      elemento di $\MeasureSpace$ se
      $\Measure(\MeasurablePart) = 0$; $\MeasurablePart$ si dice in tal caso
      \Define{trascurabile}[trascurabile][parte].
  \end{itemize}
\end{Definition}
\begin{Definition}
  Sia $(\MeasurableSpace,\SigmaAlgebra)$ uno spazio di Borel. Una misura
  $\Measure$ definita su uno spazio misurabile $(\MeasureSpace,\SigmaAlgebra)$
  si chiama
  \Define{misura di Borel}[di Borel][misura].
\end{Definition}
\begin{Theorem}
  Dati
  \begin{itemize}
    \item una classe $\MeasureSpace$,
    \item $\SigmaAlgebra  \subseteq \PowerSet{\MeasureSpace}$,
    \item un'applicazione
          $\Measure: \SigmaAlgebra \rightarrow \mathbb{R}$
          additiva;
  \end{itemize}
  abbiamo, se $\emptyset \in \SigmaAlgebra$,
  $\Measure(\emptyset) = 0$
\end{Theorem}
\Proof Abbiamo
\begin{align*}
  \Measure(\emptyset)
  &= \Measure(\emptyset \cup \emptyset),\\
  &= \Measure(\emptyset) + \Measure(\emptyset),
\end{align*}
da cui $\Measure(\emptyset) = 0$. \EndProof
\begin{Theorem}
	Dati uno spazio misurabile
  $(\MeasureSpace,\SigmaAlgebra)$ e una funzione
  $\Measure: \SigmaAlgebra \rightarrow [0; +\infty]$ sono equivalenti i
  seguenti fatti:
	\begin{itemize}
		\item $\Measure$ \`e una misura;
		\item se
      $(\MeasurablePart_n)_{n \in \mathbb{N}} \in \SigmaAlgebra^\mathbb{N}$
      \`e una successione crescente, allora
      $\lim_{n \rightarrow \infty} \Measure(\MeasurablePart_n)
      = \Measure(\bigcup_{n \in \mathbb{N}} \MeasurablePart_n)$;
		\item se
      $(\MeasurablePart_n)_{n \in \mathbb{N}} \in \SigmaAlgebra^\mathbb{N}$
      \`e una successione decrescente e
      $\Exists{n \in \mathbb{N}}{\MeasurablePart_n < + \infty}$, allora
      $\lim_{n \rightarrow \infty} \Measure(\MeasurablePart_n)
      = \Measure(\bigcap_{n \in \mathbb{N}} \MeasurablePart_n)$.
	\end{itemize}
\end{Theorem}
\begin{Theorem}
  Siano
  \begin{itemize}
    \item $(\MeasureSpace,\SigmaAlgebra,\Measure)$ uno spazio di misura;
    \item $(\MeasurableSpace,\VarSigmaAlgebra)$ uno spazio misurabile;
    \item $\MeasurableApplication: \MeasureSpace \rightarrow \MeasurableSpace$
      un'applicazione misurabile.
  \end{itemize}
  L'applicazione
  $\VarMeasure: \VarSigmaAlgebra \rightarrow [0;+\infty]$
  che a $\VarMeasurablePart \in \VarSigmaAlgebra$ associa
  $(\Measure \circ \MeasurableApplication^{-1}) (\VarMeasurablePart)$ \`e
  una misura sullo spazio misurabile $(\MeasurableSpace,\VarSigmaAlgebra)$.
\end{Theorem}
\Proof Sia
$(\VarMeasurablePart_n)_{n \in \mathbb{N}} \in \VarSigmaAlgebra^\mathbb{N}$.
Abbiamo
\begin{align*}
  \VarMeasure \left ( \bigcup_{n \in \mathbb{N}} \VarMeasurablePart_n \right )
  &= \Measure \left (
    \MeasurableApplication^{-1} \left (
    \bigcup_{n \in \mathbb{N}} \VarMeasurablePart_n \right ) \right ),\\
  &= \Measure \left (
    \bigcup_{n \in \mathbb{N}}
      \MeasurableApplication^{-1} (\VarMeasurablePart_n) \right ),\\
  &= \sum_{n \in \mathbb{N}} \Measure (
      \MeasurableApplication^{-1} (\VarMeasurablePart_n) ),\\
  &= \sum_{n \in \mathbb{N}} \VarMeasure (\VarMeasurablePart_n).\text{ \EndProof}
\end{align*}
\begin{Definition}
  Con le notazioni del teorema precedente, chiamiamo $\VarMeasure$
  \Define{misura immagine}[immagine][misura] di $\Measure$ tramite
  $\MeasurableApplication$.
  Inoltre, se $\phi(x)$ \`e una formula che definisce una parte di
  $\MeasurableSpace$ in $\VarSigmaAlgebra$ secondo lo schema di comprensione,
  allora denotiamo $\Measure (\phi(\MeasurableApplication))$ la quantit\`a
  $\VarMeasure(\lbrace x \in \MeasurableSpace | \phi(x) \rbrace)$.
\end{Definition}
