\section{Misura di Lebesgue.}
\label{Misura_MisuraDiLebesgue}
\begin{Theorem}
  Le parti misurabili secondo Lebesgue costituiscono una $\sigma$-algebra.
\end{Theorem}
\begin{Definition}
  La $\sigma$-algebra di tutti le parti misurabili secondo Lebesgue, d'ora in
  poi denotata $\LebesgueSigmaAlgebra$, si chiama
  \Define{$\sigma$-algebra di Lebesgue}[di Lebesgue][$\sigma$-algebra].
\end{Definition}
\begin{Definition}
  Una misura si dice \Define{completa}[completa][misura] se ogni parte di una
  parte trascurabile \`e misurabile.
\end{Definition}
\begin{Theorem}
  La misura di Lebesgue coindice con la misura di Borel su tutti gli elementi
  della $\sigma$-algebra di Borel.
\end{Theorem}
\begin{Theorem}
  La misura di Lebesgue \`e il completamento della misura di Borel.
\end{Theorem}
