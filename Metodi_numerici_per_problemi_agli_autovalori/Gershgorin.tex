\section{Teoremi di Gershgorin.}
\label{MetodiNumericiPerProblemiAgliAutovalori_Gershgorin}
\begin{Definition}
	Data una matrice
  $\Matrix \in \mathbb{C}{n \times n}$,
  con
  $n \in \NotZero{\mathbb{N}}$, definiamo
  \Define{$k$-esimo cerchio di Gershgorin}[di Gershgorin][cerchio],
  con
  $k \in n$, il cerchio
  \[
    \GershgorinDisk{k}[\Matrix]
    = \lbrace z \in \mathbb{C} |
      \AbsoluteValue{z - \Matrix_k^k} \leq
      \sum_{j = 0,\\j \neq i}^{n - 1} \AbsoluteValue{\Matrix_k^j}.
  \]
\end{Definition}
\begin{Theorem}
	\TheoremName{Primo teorema di Gershgorin}[primo di Gershgorin][teorema]
  Data una matrice
  $\Matrix \in \mathbb{C}^{n \times n}$
  ($n \in \NotZero{\mathbb{N}}$),
  abbiamo
  $\Spectrum{\Matrix} \subseteq \bigcup_{k \in n} \GershgorinDisk{k}$.
\end{Theorem}
\Proof Fissato $\Eigenvalue \in \Spectrum{\Matrix}$, sia
$\Eigenvector$ un autovettore relativo a $\Eigenvalue$.
Sia $K \in n$ tale che $\ForAll{k \in n}{x_K \geq x_k}$.
\par Da
$\Matrix \Eigenvector = \Eigenvalue \Eigenvector$
deduciamo, per ogni $k \in n$,
$\sum_{j = 1}^n \Matrix_K^j \Eigenvector_j = \Eigenvalue \Eigenvector_K$,
da cui
$\sum_{j = 0,\\j \neq K}^{n - 1} \Matrix_K^j \Eigenvector_j
= \Eigenvalue \Eigenvector_K - \Matrix_K^K \Eigenvector_K$.
Poich\'e certamente $\Eigenvector_K \neq 0$, deduciamo ancora
$\sum_{j = 0,\\j \neq K}^{n - 1}
\Matrix_K^j \frac{\Eigenvector_j}{\Eigenvector_K}
= \Eigenvalue - \Matrix_K^K$.
La tesi segue dal fatto che il modulo del primo membro \`e maggiorato da
$\sum_{j = 0,\\j \neq K}^{n - 1} \AbsoluteValue{\Matrix_K^j}$
che \`e proprio il raggio del $K$-esimo cerchio di Gershgorin. \EndProof
\begin{Theorem}
	\TheoremName{Secondo teorema di Gershgorin}[secondo di Gershgorin][teorema]
  Data una matrice $\Matrix \in \mathbb{C}{n \times n}$, con
  $n \in \NotZero{\mathbb{N}}$,
  se esiste $K \in \mathbb{N}$ tale che
  \[
    \left ( \bigcup_{k = 0}^K \GershgorinDisk{k} \right ) \cap \left ( \bigcup_{k = K}^{n - 1} \GershgorinDisk{k} \right ) = \emptyset,
  \]
  allora
  \[
    \Cardinality{\Spectrum{\Matrix} \cap \left ( \bigcup_{k = 0}^K \GershgorinDisk{k} \right )} = K.
  \]
\end{Theorem}
\Proof Sia $\DiagonalMatrix$ la matrice diagonale tale che per ogni
$k \in n$ si abbia
$\DiagonalMatrix_k^k = \Matrix_k^k$.
Definiamo l'applicazione
$\phi: [0,1] \rightarrow \mathbb{C}{n \times n}$ tale che
$\phi(t) = (1 - t)\DiagonalMatrix + t\Matrix$ e osserviamo che,
poich\'e gli autovalori di una matrice dipendono con continuit\`a dai
coefficienti della matrice stessa, gli autovalori di $\phi(t)$ dipendono con
continuit\`a dal parametro $t$.
\par I primi $K$ cerchi di Gershgorin della matrice $\phi(0)$ contengono
effettivamente esattamente $K$ autovalori.
Per definizione di $\phi$, per ogni $k \in n$ e ogni $t \in [0,1]$, il
$k$-esimo cerchio di Gershgorin ha lo stesso centro e raggio pi\`u piccolo
del $k$-esimo centro di Gershgorin di $\Matrix$.
Pertanto, per ogni $t \in [0,1]$, i primi $K$ cerchi di Gershgorin di
$\phi(t)$ sono disgiunti dai restanti cerchi: dovendo gli autovalori di
$\phi(t)$ muoversi con continuit\`a al variare di $t$ rimanendo sempre
nell'unione dei cerchi di Gershgorin, essi non possono lasciare l'unione
dei primi $K$ cerchi per entrare nei cerchi restanti o viceversa.
Quindi il numero di autovalori nei primi $K$ cerchi resta costantemente $K$.
\EndProof
\begin{Theorem}
	\TheoremName{Terzo teorema di Gershgorin}[terzo di Gershgorin][teorema]
  Data una matrice $\Matrix \in \mathbb{C}{n \times n}$, se
	\begin{itemize}
		\item $\Matrix$ \`e irriducibile;
		\item $\Eigenvalue$ \`e un autovalore di $\Matrix$;
		\item $\Implies{\Eigenvalue \in \GershgorinDisk{k}}{\Boundary\GershgorinDisk{k}}$;
	\end{itemize}
	allora $\Eigenvalue \in \bigcap_{k = 1}^n \Boundary{\GershgorinDisk{k}}$.
\end{Theorem}
\Proof Riprendiamo la dimostrazione del primo teorema di Gershgorin.
Abbiamo visto che
\begin{align*}
  \sum_{\substack{j = 0\\j \neq K}}^{n - 1} \AbsoluteValue{\Matrix_K^j}
  &\geq \sum_{\substack{j = 0\\j \neq K}}^{n - 1} \AbsoluteValue{\Matrix_K^j}
    \frac{\AbsoluteValue{\Eigenvector_j}}{\AbsoluteValue{\Eigenvector_K}},\\
  &\geq \AbsoluteValue{\sum_{\substack{j = 0\\j \neq K}}^{n - 1} \Matrix_K^j
    \frac{\Eigenvector_j}{\Eigenvector_K}},\\
  &= \AbsoluteValue{\Eigenvalue - \Matrix_K^K}.
\end{align*}
Dunque $\Eigenvalue \in \GershgorinDisk{K}$.
Allora $\Eigenvalue \in \Boundary\GershgorinDisk{K}$ e abbiamo
$\sum_{j = 1,\\j \neq K}^{n - 1} \AbsoluteValue{\Matrix_K^j}
= \sum_{j = 1,\\j \neq K}^{n - 1} \AbsoluteValue{\Matrix_K^j}
\frac{\AbsoluteValue{\Eigenvector_j}}{\AbsoluteValue{\Eigenvector_K}}$.
Dunque, per ogni $j \in n - \SetMin \lbrace K \rbrace$, abbiamo
$\Implies{\Matrix_K^j \neq 0}{\Eigenvector_j = \Eigenvector_K}$.
\par Assumiamo $J \in n$ tale che $\Matrix_K^J = 0$. Poich\'e $\Matrix$ \`e
irriducibile, il grafo associato a $\Matrix$ \`e fortemente connesso, il
ch\'e implica che esiste una successione finita $(a_r)_{r \in m}$
($m \in \mathbb{N}$) di elementi di $n$ tale che
	\begin{itemize}
		\item $a_0 = J$;
		\item $a_m = K$;
		\item per ogni $r < m$, $\Matrix_{a_r}^{a_{r + 1}} \neq 0$.
	\end{itemize}
Ne deduciamo, per ogni $r < m$,
$\Eigenvector_{a_r} = \Eigenvector_{a_{r + 1}}$ e dunque
$\Eigenvector_J = \Eigenvector_K$. Dunque $\Eigenvector$ ha tutte le
componenti uguali e $\Eigenvalue$ appartiene al bordo di tutti i cerchi di
Gershgorin. \EndProof
