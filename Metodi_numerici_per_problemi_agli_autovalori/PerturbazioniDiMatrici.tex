\section{Perturbazioni di matrici.}
\label{CalcoloScientifico_PerturbazioniDiMatrici}
\begin{Theorem}
  Sia $n \in \NotZero{\mathbb{N}}$.
  Per ogni $k \in n$, l'applicazione
  $\Eigenvalue_k: \mathbb{C}^{n \times n} \rightarrow \mathbb{C}$
  che a $\Matrix \in \mathbb{C}^{n \times n}$ associa il $k$-esimo
  elemento della successione crescente degli $n$ autovalori di $\Matrix$
  (contati con molteplicit\`a algebrica) \`e continua.
\end{Theorem}
\begin{Theorem}
  Con le notazioni del teorema precedente, fissato $k \in n$,
  il numero di condizionamento $\ConditionNumber$ di $\Eigenvalue_k$ in
  $\Matrix \in \mathbb{N}$ verifica, al primo ordine, la relazione
  \[
    \ConditionNumber \leq
    \frac{\Norm{\VarEigenvector}\Norm{\Eigenvector}}{
      \AbsoluteValue{\HermitianTransposed{\VarEigenvector}\Eigenvector}},
  \]
  dove $\Eigenvector$ e $\VarEigenvector$ sono autovettori destro e sinistro
  rispettivamente relativi a $\Eigenvalue = \Eigenvalue_k(\Matrix)$.
\end{Theorem}
\Proof Abbiamo $\Matrix \Eigenvector = \Eigenvalue \Eigenvector$.
Supponendo $\Matrix$ perturbata da $\Perturbation{\Matrix}$, anche
$\Eigenvector$ e $\Eigenvalue$ risultano perturbati da
$\Perturbation{\Eigenvector}$ e
$\Perturbation{\Eigenvalue}$ rispettivamente. Abbiamo
\[
  (\Matrix + \Perturbation{\Matrix})(\Eigenvector + \Perturbation{\Eigenvector})
  = (\Eigenvalue + \Perturbation{\Eigenvalue})
    (\Eigenvector + \Perturbation{\Eigenvector}).
\]
Sviluppando e conservando solo i termini al primo ordine,
\[
  \Matrix\Eigenvector
  + \Matrix \Perturbation{\Eigenvector}
  + \Perturbation{\Matrix}\Eigenvector
  = \Eigenvalue\Eigenvector
  + \Eigenvalue\Perturbation{\Eigenvector}
  + \Perturbation{\Eigenvalue}\Eigenvector,
\]
equivalente a
\[
  \Matrix \Perturbation{\Eigenvector}
  + \Perturbation{\Matrix}\Eigenvector
  = \Eigenvalue\Perturbation{\Eigenvector}
  + \Perturbation{\Eigenvalue}\Eigenvector.
\]
\par Moltiplichiamo a sinistra per $\HermitianTransposed{\VarEigenvector}$:
\[
  \HermitianTransposed{\VarEigenvector} \Matrix \Perturbation{\Eigenvector}
  + \HermitianTransposed{\VarEigenvector} \Perturbation{\Matrix}\Eigenvector
  = \HermitianTransposed{\VarEigenvector} \Eigenvalue\Perturbation{\Eigenvector}
  +\HermitianTransposed{\VarEigenvector} \Perturbation{\Eigenvalue}\Eigenvector.
\]
equivalente a 
\[
  \HermitianTransposed{\VarEigenvector} \Perturbation{\Matrix}\Eigenvector
  =\HermitianTransposed{\VarEigenvector} \Perturbation{\Eigenvalue}\Eigenvector.
\]
Utilizzando la disuguaglianza di Cauchy-Schwarz abbiamo
\begin{align*}
  \AbsoluteValue{\Perturbation{\Eigenvalue}}
    \AbsoluteValue{\HermitianTransposed{\VarEigenvector}\Vector}
  &= \AbsoluteValue{\HermitianTransposed{\VarEigenvector}
      \Perturbation{\Eigenvalue}\Eigenvector},\\
  &= \AbsoluteValue{\HermitianTransposed{\VarEigenvector}
      \Perturbation{\Matrix}\Eigenvector},\\
  &\leq \Norm{\VarEigenvector}
      \Norm{\Perturbation{\Matrix}\Eigenvector},\\
  &\leq \Norm{\VarEigenvector}
      \Norm{\Perturbation{\Matrix}}\Norm{\Eigenvector},
\end{align*}
equivalente a
$\frac{\AbsoluteValue{\Perturbation{\Eigenvalue}}}{
  \Norm{\Perturbation{\Matrix}}}
\leq \frac{\Norm{\VarEigenvector}\Norm{\Eigenvector}}{
  \AbsoluteValue{\HermitianTransposed{\VarEigenvector}{\Eigenvector}}}$,
da cui la tesi. \EndProof
\begin{Corollary}
  Con le notazioni del teorema precedente, se $\Matrix$ \`e normale, allora
$\ConditionNumber \leq 1$.
\end{Corollary}
\Proof Se $\Matrix$ \`e normale, allora autovettori sinistri e destri
coincidono. Scegliendo un autovettore $\Eigenvector$ relativo a $\Eigenvalue$.
$\ConditionNumber \leq
\frac{\Norm{\Eigenvector}\Norm{\Eigenvector}}{
  \AbsoluteValue{\HermitianTransposed{\Eigenvector}\Eigenvector}} = 1$.
\EndProof
\begin{Theorem}
  \TheoremName{Teorema di Hirsch}[di Hirsch][teorema]
  Siano
  \begin{itemize}
    \item $n \in \NotZero{\mathbb{N}}$;
    \item $\Norm{\cdot}$ una norma su $\mathbb{C}^n$;
    \item $\Matrix \in \mathbb{C}^{n \times n}$;
    \item $\Eigenvalue \in \Spectrum{\Matrix}$.
  \end{itemize}
  Abbiamo $\AbsoluteValue{\Eigenvalue} \leq \Norm{\Matrix}$.
\end{Theorem}
\Proof Sia $\Eigenvector$ autovettore relativo a $\Eigenvalue$.
$\Norm{\Matrix}
= \max_{\Vector \neq 0} \frac{\Matrix \Vector}{\Norm{\Vector}} 
\geq \frac{\Matrix\Eigenvector}{\Eigenvector}
\geq \frac{\Eigenvalue\Eigenvector}{\Eigenvector}
= \AbsoluteValue{\Eigenvalue}$. \EndProof
\begin{Definition}
  Fissato $n \in \NotZero{\mathbb{N}}$, una norma su $\mathbb{C}^n$ si dice
  \Define{assoluta}[assoluta][norma]
  quando
  \[
    \ForAll{(z_k)_{k \in n} \in \mathbb{C}}{
      \Norm{(z_k)_{k \in n}} = \Norm{(\AbsoluteValue{(z_k)})_{k \in n}}}.
  \]
\end{Definition}
\begin{Theorem}
  Tutte le $p$-norme sono assolute.
\end{Theorem}
\Proof Segue direttamente dalle definizioni. \EndProof
\begin{Lemma}
  Siano
  \begin{itemize}
    \item $n \in \NotZero{\mathbb{N}}$;
    \item $\DiagonalMatrix \in \mathbb{C}^{n \times n}$ diagonale;
    \item $\Norm{\cdot}[p]$ una $p$-norma su $\mathbb{C}^n$.
  \end{itemize}
  Abbiamo
  $\Norm{\DiagonalMatrix}[p]
  = \max_{k \in n} \AbsoluteValue{\DiagonalMatrix_k^k}$.
\end{Lemma}
\Proof Denotata con $(\CanonicalBase_k)_{k \in n}$ la base canonica
di $\mathbb{C}^n$, abbiamo, per ogni $k \in n$,
$\DiagonalMatrix \CanonicalBase_k =
\DiagonalMatrix_k^k \CanonicalBase_k$.
\par Abbiamo, per ogni $k \in n$,
\begin{align*}
  \Norm{\DiagonalMatrix}[p]
  &= \max_{\Vector \in \NotZero{\mathbb{C}}}
      \frac{\Norm{\DiagonalMatrix \Vector}[p]}{\Norm{\Vector}[p]},\\
  &\geq \frac{\Norm{\DiagonalMatrix \CanonicalBase_k}[p]}
          {\Norm{\CanonicalBase_k}[p]},\\
  &= \AbsoluteValue{\DiagonalMatrix_k^k}.
\end{align*}
Dunque
$\Norm{\DiagonalMatrix}[p]
\geq \max_{k \in n} \AbsoluteValue{\DiagonalMatrix_k^k}$.
\par D'altra parte, per un qualunque vettore unitario $\Vector$, abbiamo
\begin{align*}
  \Norm{\DiagonalMatrix \Vector}[p]
  &= \left ( \sum_{k \in n}
    \AbsoluteValue{\DiagonalMatrix_k^k \Vector_k}^p \right )^{\frac{1}{p}},\\
  &\leq \max_{k \in n} \DiagonalMatrix_k^k \left ( \sum_{k \in n}
    \AbsoluteValue{\Vector_k}^p \right )^{\frac{1}{p}},\\
  &= \max_{k \in n} \DiagonalMatrix_k^k \Norm{\Vector}[p],\\
  &= \max_{k \in n} \DiagonalMatrix_k^k.\text{ \EndProof}
\end{align*}
\begin{Theorem}
  \TheoremName{Teorema di Bauer-Fike}[di Bauer-Fike][teorema]
  Siano
  \begin{itemize}
    \item $n \in \NotZero{\mathbb{N}}$;
    \item $\Norm{\cdot}[p]$ una $p$-norma su $\mathbb{C}^{n \times n}$;
    \item $\Matrix \in \mathbb{C}^{n \times n}$ diagonalizzabile;
    \item $V, \DiagonalMatrix \in \mathbb{C}^{n \times n}$ tali che
      $V^{-1} \Matrix V = \DiagonalMatrix$;
    \item $\Perturbation{\Matrix} \in \mathbb{C}^{n \times n}$;
    \item $\VarEigenvalue \in \Spectrum(\Matrix + \Perturbation{\Matrix})$.
  \end{itemize}
  Abbiamo
  \[
    \min_{\Eigenvalue \in \Spectrum{\Matrix}}
    \AbsoluteValue{\VarEigenvalue - \Eigenvalue}
    \leq \VarMatrixConditionNumber \Norm{\Perturbation{\Matrix}}[p],
  \]
  dove $\VarMatrixConditionNumber$ \`e il numero di condizionamento della matrice
  $V$.
\end{Theorem}
\Proof Se $\VarEigenvalue \in \Spectrum{\Matrix}$, allora 
$\min_{\Eigenvalue \in \Spectrum{\Matrix}}
\AbsoluteValue{\VarEigenvalue - \Eigenvalue} = 0$ e la tesi \`e immediata.
Supponiamo dunque nel seguito, $\VarEigenvalue \notin \Spectrum{\Matrix}$.
\par Sia $\VarEigenvector$ autovettore di $\Matrix + \Perturbation{\Matrix}$
relativo a $\VarEigenvalue$:
\[
  (\Matrix + \Perturbation{\Matrix} - \VarEigenvalue\Identity) \VarEigenvector
  = 0.
\]
\par Effettuando un cambio di base tramite la matrice $V$,
\[
  (V^{-1}\Matrix V
  + V^{-1} \Perturbation{\Matrix} V
  - \VarEigenvalue V^{-1} \Identity V) V \VarEigenvector = 0,
\]
equivalente a
\[
  (\DiagonalMatrix
  + V^{-1} \Perturbation{\Matrix} V
  - \VarEigenvalue \Identity) V \VarEigenvector = 0.
\]
Poich\'e
$\VarEigenvalue \notin \Spectrum{\Matrix} = \Spectrum{\DiagonalMatrix}$,
$\DiagonalMatrix - \VarEigenvalue\Identity$ \`e invertibile. Moltiplichiamo a
sinistra l'equazione precedente per
$(\DiagonalMatrix - \VarEigenvalue\Identity)^{-1}$:
\[
  (\Identity
  + (\DiagonalMatrix - \VarEigenvalue\Identity)^{-1}
    V^{-1} \Perturbation{\Matrix} V) V \VarEigenvector = 0.
\]
Dunque
$-1 \in \Spectrum{
(\DiagonalMatrix - \VarEigenvalue\Identity)^{-1} V^{-1}
\Perturbation{\Matrix} V}$.
\par Per il teorema di Hirsch,
\begin{align*}
  1
  &\leq \Norm{\DiagonalMatrix - \VarEigenvalue\Identity)^{-1} V^{-1}
  \Perturbation{\Matrix} V}[p],\\
  &\leq \Norm{(\DiagonalMatrix - \VarEigenvalue\Identity)^{-1}}[p]
  \Norm{V^{-1}}[p] \Norm{\Perturbation{\Matrix}}[p] \Norm{V}[p],\\
  &\leq \Norm{(\DiagonalMatrix - \VarEigenvalue\Identity)^{-1}}[p]
  \VarMatrixConditionNumber \Norm{\Perturbation{\Matrix}}[p].
\end{align*}
\par D'altra parte, $(\DiagonalMatrix - \VarEigenvalue \Identity)^{-1}$ \`e
diagonale, dunque, per il lemma precedente,
$\Norm{(\DiagonalMatrix - \VarEigenvalue\Identity)^{-1}}[p]
= \max_{\Eigenvalue \in \Spectrum{\Matrix}} \frac{1}{
  \AbsoluteValue{\VarEigenvalue - \Eigenvalue}}
= \frac{1}{\min_{\Eigenvalue \in \Spectrum{\Matrix}}
  \AbsoluteValue{\VarEigenvalue - \Eigenvalue}}$.
E dunque infine
\[
  \min_{\Eigenvalue \in \Spectrum{\Matrix}} 
  \AbsoluteValue{\VarEigenvalue - \Eigenvalue}
  \leq \VarMatrixConditionNumber \Norm{\Perturbation{\Matrix}}[p].\text{ \EndProof}
\]
\begin{Corollary}
  Con le notazioni del teorema precedente, se $\Matrix$ \`e normale, allora
  \[
    \min_{\Eigenvalue \in \Spectrum{\Matrix}}
    \AbsoluteValue{\VarEigenvalue - \Eigenvalue}
    \leq \Norm{\Perturbation{\Matrix}}[2].
  \]
\end{Corollary}
\Proof Segue direttamente dal teorema precedente e dal fatto che $\Matrix$ \`e
diagonalizzabile per mezzo di matrici unitarie. \EndProof
\begin{Theorem}
  L'errore all'indietro $\BackwardError$ rispetto alla norma $\Norm{\cdot}[2]$
  di un'applicazione che a una matrice
  $\Matrix \in \mathbb{C}^{n \times n}$ ($n \in \mathbb{N}$) associa una coppia
  $(\Eigenvalue,\Eigenvector) \in \mathbb{C} \times (\NotZero{\mathbb{C}^n})$
  approssimazione di una coppia
  $(\VarEigenvalue,\VarEigenvector) \in
    \mathbb{C} \times (\NotZero{\mathbb{C}^n})$
  tale che $\Matrix\VarEigenvector = \VarEigenvalue\VarEigenvector$
  \`e
  \[
    \BackwardError
    = \frac{\Norm{\Matrix\Eigenvector - \Eigenvalue\Eigenvector}[2]}
      {\Norm{\Eigenvector}[2]}
  \]
\end{Theorem}
\Proof Sia $\Perturbation{\Matrix} \in \mathbb{C}^{n \times n}$.
Supponiamo
\[
  (\Matrix + \Perturbation{\Matrix})\Eigenvector = \Eigenvalue\Eigenvector;
\]
ci\`o \`e equivalente a
\[
  \Perturbation{\Matrix}\Eigenvector
  = \Eigenvalue\Eigenvector - \Matrix\Eigenvector.
\]
Passando alle norme, abbiamo
\[
  \Norm{\Perturbation{\Matrix}\Eigenvector}[2]
  = \Norm{\Eigenvalue\Eigenvector - \Matrix\Eigenvector}[2],
\]
da cui
\begin{align*}
  \frac{\Norm{\Matrix\Eigenvector - \Eigenvalue\Eigenvector}[2]}
    {\Norm{\Eigenvector}[2]}
  &= \frac{\Norm{\Perturbation{\Matrix}\Eigenvector}[2]}
    {\Norm{\Eigenvector}[2]},
  \leq \Norm{\Perturbation{\Matrix}}[2].
\end{align*}
Dunque
$\BackwardError
\geq \frac{\Norm{\Matrix\Eigenvector - \Eigenvalue\Eigenvector}[2]}
{\Norm{\Eigenvector}[2]}$.
\par D'altra parte, poniamo
\begin{itemize}
  \item $\Vector = \Matrix\Eigenvector - \Eigenvalue\Eigenvector$;
  \item $\Perturbation{\Matrix}
    = - \frac{\Vector\HermitianTransposed{\Eigenvector}}
        {\Norm{\Eigenvector}[2]^2}$.
\end{itemize}
\par Abbiamo
\begin{align*}
  (\Matrix + \Perturbation{\Matrix})\Eigenvector
  &= \Matrix\Eigenvector
    - \frac{\Vector\HermitianTransposed{\Eigenvector}}
      {\Norm{\Eigenvector}[2]^2} \Eigenvector,\\
  &= \Matrix\Eigenvector - \Vector,\\
  &= \Matrix\Eigenvector - \Matrix\Eigenvector + \Eigenvalue\Eigenvector,\\
  &= \Eigenvalue\Eigenvector.
\end{align*}
Per il lemma
\ref{MetodiNumericiPerLaRisoluzioneDiSistemiLineari_LemmaRango1},
abbiamo inoltre
\begin{align*}
  \Norm{\Perturbation{\Matrix}}[2]
  &= \Norm{- \frac{\Vector}{\Norm{\Eigenvector}[2]^2}}[2] \Norm{\Eigenvector}[2],\\
  &= \frac{\Norm{\Vector}[2]}{\Norm{\Eigenvector}[2]^2} \Norm{\Eigenvector}[2],\\
  &= \frac{\Norm{\Vector}[2]}{\Norm{\Eigenvector}[2]},\\
  &= \frac{\Norm{\Matrix\Eigenvector - \Eigenvalue\Eigenvector}[2]}
    {\Norm{\Eigenvector}[2]}.\text{ \EndProof}
\end{align*}
