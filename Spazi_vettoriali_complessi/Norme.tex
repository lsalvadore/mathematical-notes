\section{Norme}
\label{SpaziVettorialiComplessi_Norme}
\begin{Theorem}
  Siano
  \begin{itemize}
    \item $n \in \NotZero{\mathbb{N}}$;
    \item $p \in \mathbb{R}$ tale che $p \geq 1$.
  \end{itemize}
  L'applicazione $\Norm{\cdot}[p]: \mathbb{C}^n \rightarrow \mathbb{R}$
  definita per ogni $x \in \NormedSpace$ da
  \[
    \Norm{x}[p] = \sqrt[p]{\sum_{k \in n} \AbsoluteValue{x_k}^p}
  \]
  \`e una norma.
\end{Theorem}
\Proof Tutte le propriet\`a che definiscono una norma si verificano
immediatamente per $\Norm{\cdot}[p]$ ad eccezione della disuguaglianza
triangolare: procediamo duqnue a dimostrare la validit\`a anche di questa
disuguaglianza.
\par 

\begin{Theorem}
  Siano
  \begin{itemize}
    \item $n \in \NotZero{\mathbb{N}}$;
    \item $p \in \mathbb{R}$ tale che $p \geq 1$.
  \end{itemize}
  L'applicazione $\Norm{\cdot}[\infty]: \mathbb{C}^n \rightarrow \mathbb{R}$
  definita per ogni $x \in \NormedSpace$ da
  \[
    \Norm{x}[\infty] = \max_{k \in n} \AbsoluteValue{x_k}
  \]
  \`e una norma.
\end{Theorem}
\begin{Theorem}
  Sia $\Norm{\cdot}$ una norma su un $\mathbb{C}$-spazio vettoriale
  $\NormedSpace$ di dimensione finita $n \in \mathbb{N}$.
  $\Norm{\cdot}$ \`e un'applicazione continua rispetto alla topologia
  di $\NormedSpace$ indotta da $\Norm{\cdot}[1]$.
\end{Theorem}
\Proof Fissiamo $\epsilon > 0$.
\par Per la disuguaglianza triangolare abbiamo
\begin{align*}
  &\Norm{x} - \Norm{y} \leq \Norm{x - y},\\
  &\Norm{y} - \Norm{x} \leq \Norm{y - x} = \Norm{x - y},
\end{align*}
da cui
\[
  \AbsoluteValue{\Norm{x} - \Norm{y}} \leq \Norm{x - y}.
\]
\par Abbiamo inoltre, denotati $(e_i)_{i \in n}$ i vettori della base canonica
di $\mathbb{C}^n$,
\begin{align*}
  \Norm{x - y}
  &= \Norm{\sum_{i \in n} (x_i - y_i) e_i},\\
  &\leq \sum_{i \in n} \Norm{(x_i - y_i) e_i},\\
  &= \sum_{i \in n} \AbsoluteValue{x_i - y_i} \Norm{e_i},\\
  &\leq \sum_{i \in n} \AbsoluteValue{x_i - y_i} \max_{i \in n} \Norm{e_i},\\
  &\leq \Norm{x - y}[1] \max_{i \in n} \Norm{e_i}.
\end{align*}
\par Poniamo ora
$\delta = \frac{\epsilon}{\max_{i \in n} \Norm{e_i}}$ e supponiamo
$\Norm{x - y}[1] < \delta$. Abbiamo infine
\begin{align*}
  \AbsoluteValue{\Norm{x} - \Norm{y}}
  &\leq \Norm{x - y},\\
  &\leq \Norm{x - y}[1] \max_{i \in n} \Norm{e_i},\\
  &= \Norm{x - y}[1] \frac{\epsilon}{\delta},\\
  &< \epsilon.\text{ \EndProof}
\end{align*}
\begin{Theorem}
  \TheoremName{Teorema di equivalenza delle norme}[di equivalenza delle norme][teorema]
  Fissato $n \in \NotZero{\mathbb{N}}$, siano $\Norm{\cdot}$ e $\Norm{\cdot}'$ due norme
  su $\mathbb{C}^n$. Valgono i seguenti fatti:
  \begin{itemize}
    \item esistono costanti $m, M \in \RealNonNegative$ tali che, per ogni
      $x \in \mathbb{C}^n$,
      \[
        m\Norm{x}' \leq \Norm{x} \leq M\Norm{x}';
      \]
    \item $\Norm{\cdot}$ e $\Norm{\cdot}'$ sono norme equivalenti.
  \end{itemize}
\end{Theorem}
\Proof Assumiamo in un primo momento $\Norm{\cdot}' = \Norm{\cdot}[1]$.
\par Se $x = 0$, allora la tesi \`e immediata.
\par Supponiamo $x \neq 0$ e poniamo
$y = \frac{x}{\Norm{x}[1]}$. Abbiamo $y \in \NSphere{n}$, dove $\NSphere{n}$ \`e
la sfera unitaria rispetto alla norma $\Norm{\cdot}[1]$.
\par L'insieme $\NSphere{n}$ \`e limitato per definizione e chiuso in quanto
controimmagine di $1$ rispetto alla norma $\Norm{\cdot}[1]$, che \`e
un'applicazione continua: $\NSphere{n}$ \`e dunque compatto.
Ne consegue, per il teorema precedente e per il teorema di Weierstrass,
 che la norma $\Norm{\cdot}$ ammette minimo $m \in \RealNonNegative$ e massimo
$M \in \RealNonNegative$ su $\NSphere{n}$.
\par Abbiamo allora
\[
  m \leq \Norm{y} \leq M,
\]
da cui,
\[
  m\Norm{x}[1] \leq \Norm{x} \leq M\Norm{x}[1].
\]
\par Ora, per $\Norm{\cdot}'$ arbitraria, basta osservare che per opportune
costanti $m', M' \in \mathbb{R}$ abbiamo 
\[
  m'\Norm{x}[1] \leq \Norm{x}' \leq M'\Norm{x}[1],
\]
da cui, dividendo per $m'$ e combinando con la coppia di diseguaglianze
precedente,
\[
  \Norm{x} \leq M\Norm{x}[1] \leq \frac{M}{m'}\Norm{x}';
\]
analogamente, dividendo per $M'$ e combinando,
\[
  \frac{m}{M'} \Norm{x}' \leq m\Norm{x}[1] \leq \Norm{x};
\]
e quindi
\[
  \frac{m}{M'} \Norm{x}' \leq m\Norm{x}[1] \leq \Norm{x}
  \leq M\Norm{x}[1] \leq \frac{M}{m'}\Norm{x}'.
\]
\par Sia $\Open$ un aperto della topologia indotta da $\Norm{\cdot}$.
Per ogni $x \in \Open$ esiste $r > 0$ tale che
\[
  \OpenBall{x}{r} = \lbrace y \in \mathbb{C}^n | \Norm{x - y} < r \rbrace
    \subseteq \Open.
\]
\par Per quanto gi\`a provato, esiste $M \in \RealNonNegative$ tale che
$\Norm{x - y} \leq M\Norm{x - y}'$.
\par Poniamo inoltre
\[
  \OpenBall{x}{r}' = \lbrace y \in \mathbb{C}^n |
    \Norm{x - y}' < \frac{r}{M} \rbrace.
\]
\par Per ogni $y \in \OpenBall{x}{r}'$, abbiamo
\begin{align*}
  \Norm{x - y}
  &\leq M\Norm{x - y}',\\
  &< M \frac{r}{M},\\
  &= r.
\end{align*}
\par Ne deduciamo
$\OpenBall{x}{r}' \subseteq \OpenBall{x}{r} \subseteq \Open$.
Dunque $\Open$ \`e aperto rispetto alla topologia indotta da $\Norm{\cdot}'$.
Quindi la topologia indotta da $\Norm{\cdot}'$ \`e pi\`u fine rispetto alla
topologia indotta da $\Norm{\cdot}$. Analogamente si prova che la topologia
indotta da $\Norm{\cdot}$ \`e pi\`u fine di quella indotta da $\Norm{\cdot}'$.
E quindi infini le due topologie sono equivalenti. \EndProof
\begin{Corollary}
  Fissati $p, q \in \NotZero{\mathbb{N}}$, siano
  \begin{itemize}
    \item $\Norm{\cdot}[a]$ e $\Norm{\cdot}[b]$ norme su $\mathbb{C}^p$;
    \item $\Norm{\cdot}[A]$ e $\Norm{\cdot}[B]$ norme su $\mathbb{C}^q$;
    \item $f: \mathbb{C}^p \rightarrow \mathbb{C}^q$ un'applicazione
      lipschitziana rispetto alle distanze indotte da $\Norm{\cdot}[a]$ e
      $\Norm{\cdot}[A]$.
  \end{itemize}
  $f$ \`e lipschitziana anche rispetto alle distanze indotta da
  $\Norm{\cdot}[b]$ e $\Norm{\cdot}[B]$.
\end{Corollary}
\Proof Sia $L > 0$ tale che, che per ogni $x, y \in \mathbb{C}^p$ abbiamo
$\Norm{f(y) - f(x)}[A] \leq L \Norm{y - x}[a]$.
\par Per il teorema di equivalenza delle norme,
\begin{itemize}
  \item esiste $M$ tale che, per ogni
    $x, y \in \mathbb{C}^q$, $\Norm{y - x}[a] \leq M \Norm{y - x}[b]$;
  \item esiste $m$ tale che, per ogni
    $X, Y \in \mathbb{C}^p$, $m \Norm{Y - X}[B] \leq \Norm{Y - X}[A]$.
\end{itemize}
\par Ne deduciamo
\begin{align*}
  \Norm{f(y) - f(x)}[B]
  &\leq m^{-1} \Norm{f(y) - f(x)}[A],\\
  &\leq m^{-1} L \Norm{y - x}[a],\\
  &\leq m^{-1} L M \Norm{y - x}[b]. \text{ \EndProof}
\end{align*}
