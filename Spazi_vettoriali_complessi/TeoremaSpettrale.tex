\section{Teorema spettrale}
\label{SpaziVettorialiComplessi_TeoremaSpettrale}
\begin{Theorem}
  Sia $\LinearOperator$ un operatore lineare su uno spazio prehilbertiano
  complesso $\PrehilbertSpace$ con prodotto hermitiano
  $\HermitianProduct{\cdot}{\cdot}$.
  Abbiamo
  \[
    \Coimplies{\Eigenvalue \in \Spectrum{\LinearOperator}}
    {\Conjugate{\Eigenvalue} \in \Spectrum{\Adjoint{\LinearOperator}}}.
  \]
\end{Theorem}
\Proof $\Eigenvalue \in \mathbb{C}$ non \`e un autovalore di $\LinearOperator$
se e solo se $\LinearOperator - \Eigenvalue\Identity$ \`e invertibile,
vale a dire
se e solo se esiste un altro operatore $\VarLinearOperator$ tale che
\[
  \VarLinearOperator(\LinearOperator - \Eigenvalue\Identity) =
  (\LinearOperator - \Eigenvalue\Identity)\VarLinearOperator =
  \Identity;
\]
ci\`o equivale a
\[
  (\Adjoint{\LinearOperator} - \Conjugate{\Eigenvalue}\Identity)
    \Adjoint{\VarLinearOperator} =
  \Adjoint{\VarLinearOperator}
    (\Adjoint{\LinearOperator} - \Conjugate{\Eigenvalue}\Identity) =
  \Identity;
\]
e dunque a
$\Conjugate{\Eigenvalue} \notin \Spectrum{\Adjoint{\LinearOperator}}$.
\EndProof
\begin{Theorem}
  Sia $\NormalOperator$ un operatore normale su uno spazio prehilbertiano
  complesso $\PrehilbertSpace$ con prodotto hermitiano
  $\HermitianProduct{\cdot}{\cdot}$.
  Sia inoltre $\Eigenvalue \in \mathbb{C}$ un autovalore di $\NormalOperator$
  relativo all'autovettore $\Eigenvector \in \PrehilbertSpace$.
  Allora $\Adjoint{\NormalOperator} \Eigenvector
  = \Conjugate{\Eigenvalue} \Eigenvector$.
\end{Theorem}
\Proof Abbiamo
\begin{align*}
\Norm{\Adjoint{\NormalOperator}\Eigenvector-\Conjugate{\Eigenvalue}\Eigenvector}
  &= \HermitianProduct{\Adjoint{\NormalOperator}\Eigenvector}
        {\Adjoint{\NormalOperator}\Eigenvector}
    - \HermitianProduct{\Adjoint{\NormalOperator}\Eigenvector}
        {\Conjugate{\Eigenvalue}\Eigenvector}
    - \HermitianProduct{\Conjugate{\Eigenvalue}\Eigenvector}
        {\Adjoint{\NormalOperator}\Eigenvector}
    + \HermitianProduct{\Conjugate{\Eigenvalue}\Eigenvector}
        {\Conjugate{\Eigenvalue}\Eigenvector},\\
  &= \HermitianProduct{\NormalOperator\Adjoint{\NormalOperator}\Eigenvector}
        {\Eigenvector}
    - \HermitianProduct{\Adjoint{\NormalOperator}\NormalOperator\Eigenvector}
        {\Eigenvector}
    - \HermitianProduct{\Eigenvalue\Eigenvector}
        {\Eigenvalue\Eigenvector}
    + \HermitianProduct{\Eigenvalue\Eigenvector}
        {\Eigenvalue\Eigenvector},\\
  &= \HermitianProduct{\NormalOperator\Adjoint{\NormalOperator}\Eigenvector
        - \Adjoint{\NormalOperator}\NormalOperator\Eigenvector}
        {\Eigenvector},\\
  &= 0.\text{ \EndProof}
\end{align*}
\begin{Corollary}
  Sia $\LinearOperator$ un operatore lineare su uno spazio prehilbertiano
  complesso $\PrehilbertSpace$ con prodotto hermitiano
  $\HermitianProduct{\cdot}{\cdot}$. Tutti gli autovalori di
  $\HermitianOperator$ sono reali.
\end{Corollary}
\Proof $\HermitianOperator$ \`e hermitiano dunque normale.
\par Quindi un autovettore ha per autovalore sia $\Eigenvalue$ che
$\Conjugate{\Eigenvalue}$ e per l'unicit\`a dell'autovettore deve essere
$\Eigenvalue = \Conjugate{\Eigenvalue} \in \mathbb{R}$. \EndProof
\begin{Corollary}
  Sia $\NormalOperator$ un operatore normale su uno spazio prehilbertiano
  complesso $\PrehilbertSpace$ con prodotto hermitiano
  $\HermitianProduct{\cdot}{\cdot}$.
  Gli autospazi di $\NormalOperator$ sono fra loro ortogonali.
\end{Corollary}
\Proof Siano $\Eigenvalue, \VarEigenvalue \in \Spectrum{\NormalOperator}$,
con $\Eigenvalue \neq \VarEigenvalue$.
Siano inoltre $\Eigenvector$ e $\VarEigenvector$ autovettori di
$\NormalOperator$ relativi a $\Eigenvalue$ e $\VarEigenvalue$
rispettivamente.
\par Supponiamo, senza perdita di generalit\`a, che il prodotto interno sia
un prodotto hermitiano.
\par Abbiamo
\begin{align*}
  \Eigenvalue \HermitianProduct{\Eigenvector}{\VarEigenvector}
  &= \HermitianProduct{\Eigenvalue\Eigenvector}{\VarEigenvector},\\
  &= \HermitianProduct{\NormalOperator\Eigenvector}{\VarEigenvector},\\
  &= \HermitianProduct{\Eigenvector}{\Adjoint{\NormalOperator}\VarEigenvector},\\
  &= \HermitianProduct{\Eigenvector}{\Conjugate{\VarEigenvalue}\VarEigenvector},\\
  &= \VarEigenvalue\HermitianProduct{\Eigenvector}{\VarEigenvector}.
\end{align*}
\par Da cui necessariamente
$\HermitianProduct{\Eigenvector}{\VarEigenvector}$. \EndProof
\begin{Theorem}
  Sia $\NormalOperator$ un operatore normale su uno spazio prehilbertiano
  complesso $\PrehilbertSpace$ con prodotto hermitiano
  $\HermitianProduct{\cdot}{\cdot}$.
  Gli autospazi di $\NormalOperator$ sono tra loro ortogonali.
\end{Theorem}
\Proof Siano $\Eigenvalue, \VarEigenvalue \in \Spectrum{\NormalOperator}$ e
$\Eigenvector, \VarEigenvector \in \PrehilbertSpace$ tali che
$\NormalOperator\Eigenvector = \Eigenvalue\Eigenvector$
e
$\NormalOperator\VarEigenvector = \VarEigenvalue\VarEigenvector$.
\par Abbiamo
\begin{align*}
  \Eigenvalue\HermitianProduct{\Eigenvector}{\VarEigenvector}
  &= \HermitianProduct{\Eigenvalue\Eigenvector}{\VarEigenvector},\\
  &= \HermitianProduct{\NormalOperator\Eigenvector}{\VarEigenvector},\\
  &=\HermitianProduct{\Eigenvector}{\Adjoint{\NormalOperator}\VarEigenvector},\\
 &=\HermitianProduct{\Eigenvector}{\Conjugate{\VarEigenvalue}\VarEigenvector},\\
  &= \VarEigenvalue\HermitianProduct{\Eigenvector}{\VarEigenvector}.
\end{align*}
\par Dunque
$\Or{\Eigenvalue = \VarEigenvalue}
{\HermitianProduct{\Eigenvector}{\VarEigenvector}}$. \EndProof
\begin{Theorem}
  Sia $\NormalOperator$ un operatore normale su uno spazio prehilbertiano
  complesso $\PrehilbertSpace$ con prodotto hermitiano
  $\HermitianProduct{\cdot}{\cdot}$.
  Sia inoltre $\Eigenvector$ autovettore di $\NormalOperator$:
  lo spazio $\OrthogonalComplement{\LinearSpan{\Eigenvector}}$ \`e 
  invariante rispetto a $\NormalOperator$.
\end{Theorem}
\Proof Sia $\Vector \in \OrthogonalComplement{\LinearSpan{\Eigenvector}}$ e
sia $\Eigenvalue \in \Spectrum{\NormalOperator}$ l'autovalore associato a
$\Eigenvector$.
Abbiamo
\begin{align*}
  \HermitianProduct{\NormalOperator\Vector}{\Eigenvector}
  &= \HermitianProduct{\NormalOperator\Vector}{\Eigenvector},\\
  &= \HermitianProduct{\Vector}{\Adjoint{\NormalOperator}\Eigenvector},\\
  &= \HermitianProduct{\Vector}{\Conjugate{\Eigenvalue}\Eigenvector},\\
  &= \HermitianProduct{\Eigenvalue\Vector}{\Eigenvector},\\
  &= \Eigenvalue\HermitianProduct{\Vector}{\Eigenvector},\\
  &= 0.\text{ \EndProof}
\end{align*}
\begin{Corollary}
  \TheoremName{Teorema spettrale}[spettrale][teorema] Sia $\NormalOperator$ un
  operatore normale su uno spazio prehilbertiano complesso $\PrehilbertSpace$
  con prodotto hermitiano $\HermitianProduct{\cdot}{\cdot}$.
  Esiste una base $(\Eigenvector_k)_{k \in n}$ di $\PrehilbertSpace$ costituita
  da autovettori ortonormali di $\NormalOperator$.
  Inoltre,
  \begin{itemize}
    \item la matrice che rappresenta $\NormalOperator$ \`e diagonale, con tutti
      gli autovalori di $\NormalOperator$ sulla diagonale presenti tante volte
      quanta \`e la molteplicit\`a geometrica di ciascuno di essi;
    \item per ogni autovalore di $\NormalOperator$, la molteplicit\`a algebrica
      e la molteplicit\`a geometrica coincidono.
  \end{itemize}
\end{Corollary}
\Proof Procediamo per induzione sulla dimensione di $\PrehilbertSpace$.
Se $\LinearDimension{\PrehilbertSpace} = 1$ il teorema \`e immediatamente
dimostrato.
\par Supponiamo il teorema dimostrato per
$\LinearDimension{\PrehilbertSpace} = N$
e proviamolo per
$\LinearDimension{\PrehilbertSpace} = N + 1$.
\par Sia $\Eigenvalue \in \Spectrum{\NormalOperator}$ e
$\Eigenvector \in \PrehilbertSpace$
autovettore associato a $\Eigenvalue$. Per il teorema precedente,
$\NormalOperator$ si riduce su $\LinearSpan{\Eigenvector}$ e
$\OrthogonalComplement{\LinearSpan{\Eigenvector}}$.
Ma per ipotesi induttiva,
$\OrthogonalComplement{\LinearSpan{\Eigenvector}}$
ammette una base di autovettori ortonormali di
$\Restricted{\NormalOperator}{\OrthogonalComplement{\LinearSpan{\Eigenvector}}}$
e dunque di $\NormalOperator$. Estendiamo tale base col vettore
$\frac{\Eigenvector}{\Norm{\Eigenvector}}$ e, se necessario, ortogonalizziamo
l'insieme dei vettori della base che sono autovettori di $\NormalOperator$
con autovalore $\Eigenvalue$: la base cos\`i ottenuta \`e ortonormale.
\par Ne segue immediatamente la diagonalit\`a della matrice che rappresenta
$\NormalOperator$ rispetto ad una base s\`i costituita con gli autovalori
presenti sulla diagonale tante volte quanta \`e la loro molteplicit\`a
geometrica. Infine, l'uguaglianza delle molteplicit\`a geometriche e algebriche
segue dal calcolo diretto del polinomio caratteristico a partire dalla suddetta
matrice diagonale. \EndProof
\begin{Definition}
  Con le notazioni del teorema precedente diciamo che
  $(\Eigenvector_k)_{k \in n}$ \`e una
  \Define{base spettrale}[spettrale][base].
\end{Definition}
