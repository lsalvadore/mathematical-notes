\section{Applicazioni trasposte e applicazione aggiunte.}
\label{GeometriaLineare_ApplicazioniTrasposteEApplicazioniAggiunte}
\begin{Definition}
	Sia
  $\Linear: \LinearSpace \rightarrow \VarLinearSpace$
  un'applicazione lineare tra i $\Field$-spazi vettoriali
  $\LinearSpace$ e $\VarLinearSpace$.
  Chiamiamo
  \Define{applicazione trasposta}[lineare trasposta][applicazione] di
  $\Linear$ l'applicazione
  $\Transposed{\Linear}: \Dual{\VarLinearSpace} \rightarrow \Dual{\LinearSpace}$
  che a $\VarFunctional \in \Dual{\VarLinearSpace}$ associa
  $\VarFunctional \circ \Linear \in \Dual{\LinearSpace}$.
\end{Definition}
\begin{Theorem}
	Siano $\LinearSpace$ e $\VarLinearSpace$ $\Field$-spazi vettoriali finiti
  di dimensioni
  $\Dimension{\LinearSpace} = n$
  e
  $\Dimension{\VarLinearSpace} = m$.
  Fissisamo basi
  $(\LinearBase_j)_{j \in n}$
  e
  $(\VarLinearBase_i)_{i \in m}$
  di $\LinearSpace$ e $\VarLinearSpace$ rispettivamente.
  Se $A$ \`e la matrice che rappresenta $\Linear$ rispetto alle basi fissate e
  $B$ la matrice che rappresenta $\Transposed{\Linear}$ rispetto alle basi
  duali, allora $B = \Transposed{A}$.
\end{Theorem}
\Proof Fissiamo $i \in m$. Abbiamo
  $\Transposed{\Linear}(\VarLinearBase^i) = \VarLinearBase^i \circ \Linear$.
\begin{Theorem}
	Siano
	\begin{itemize}
		\item $\LinearSpace$ e $\VarLinearSpace$ $\Field$-spazi vettoriali su cui
          sono definiti prodotti denotati entrambi
          $\InnerProduct{\cdot}{\cdot}$;
		\item $\LinearOperator: \LinearSpace \rightarrow \VarLinearSpace$ lineare.
	\end{itemize}
	Esiste una e una sola applicazione lineare
  $\Adjoint{\LinearOperator}: \VarLinearSpace \rightarrow \LinearSpace$ tale che
  $\ForAll{(x,y) \in \LinearSpace \times \VarLinearSpace}
    {\InnerProduct{\LinearOperator x}{y}
      = \InnerProduct{x}{\Adjoint{\LinearOperator} y}}$.
\end{Theorem}
\Proof Se l'applicazione $\Adjoint{\Linear}$ esiste,
allora essa \`e necessarimente l'applicazione che a $y \in \VarLinearSpace$
associa il vettore di $\LinearSpace$ che, per il teorema di rappresentazione di
Riesz, rappresenta il funzionale su $\LinearSpace$ dato da
$x \mapsto \InnerProduct{\LinearOperator x}{y}$; e dunque, se esiste, essa \`e
unica. Occora allora provare solo la linearit\`a di $\Adjoint{\Linear}$.
\par Siano $y_0, y_1 \in \VarLinearSpace$.
Abbiamo, per ogni $x \in \LinearSpace$,
$\InnerProduct{\LinearOperator x}{y_0 + y_1}
= \InnerProduct{x}{\Adjoint{\LinearOperator}(y_0 + y_1)}$, ma anche
\begin{align*}
  \InnerProduct{\LinearOperator x}{y_0 + y_1}
  &= \InnerProduct{\LinearOperator x}{y_0}
    + \InnerProduct{\LinearOperator x}{y_1},\\
  &= \InnerProduct{x}{\Adjoint{\LinearOperator}y_0}
    + \InnerProduct{x}{\Adjoint{\LinearOperator}y_1},\\
  &= \InnerProduct{x}{\Adjoint{\LinearOperator}y_0
    + \Adjoint{\LinearOperator}y_1}.
\end{align*}
\par Quindi, per il teorema di rappresentazione di Riesz,
$\Adjoint{\LinearOperator}(y_0 + y_1)
= \Adjoint{\LinearOperator}y_0 + \Adjoint{\LinearOperator}y_1$.
\par Siano ora $y \in \VarLinearSpace$ e $\Scalar \in \Field$.
Abbiamo, per ogni $x \in \LinearSpace$,
$\InnerProduct{\LinearOperator x}{\Scalar y}
= \InnerProduct{x}{\Adjoint{\LinearOperator} \Scalar y}$,
ma anche
\begin{align*}
  \InnerProduct{\LinearOperator x}{\Scalar y}
  &= \Conjugate{\Scalar} \InnerProduct{\LinearOperator x}{y},\\
  &= \Conjugate{\Scalar}\InnerProduct{x}{\Adjoint{\LinearOperator}y},\\
  &= \InnerProduct{x}{\Scalar \Adjoint{\LinearOperator} y}.
\end{align*}
\par Quindi, per il teorema di rappresentazione di Riesz,
$\Adjoint{\LinearOperator} \Scalar y = \Scalar \Adjoint{\LinearOperator} y$.
\par $\Adjoint{\LinearOperator}$ \`e pertanto un'applicazione lineare. \EndProof
\begin{Theorem}\label{GeometriaLineare_Teorema_TraspostaAggiunta}
	Siano
	\begin{itemize}
		\item $\LinearSpace$ e $\VarLinearSpace$ $\Field$-spazi vettoriali su cui
          sono definiti prodotti denotati entrambi $\InnerProduct{\cdot}{\cdot}$
		\item $\LinearOperator: \LinearSpace \rightarrow \VarLinearSpace$ lineare;
		\item per ogni $\Vector \in \LinearSpace$,
          $F_{\LinearSpace}: \LinearSpace \rightarrow \Dual{\LinearSpace}$
          che a $\Vector \in \LinearSpace$ associa
          $\Functional \in \Dual{\LinearSpace}$ definito da
          $\Functional: x \mapsto \InnerProduct{x}{\Vector}$;
		\item per ogni $\VarVector \in \LinearSpace$,
      $G_{\VarLinearSpace}: \VarLinearSpace \rightarrow \Dual{\VarLinearSpace}$
          che a $\VarVector \in \VarLinearSpace$ associa
          $\VarFunctional \in \Dual{\VarLinearSpace}$ definito da
          $\VarFunctional: x \mapsto \InnerProduct{x}{\VarVector}$.
	\end{itemize}
	Abbiamo
  $F_{\LinearSpace} \circ \Adjoint{\LinearOperator}
    = \Transposed{\LinearOperator} \circ G_{\LinearSpace}$,
  \Cfr\ figura \label{GeometriaLineare_Figura_TraspostaAggiunta}.
\end{Theorem}
\begin{figure}
	\center
	\begin{tikzcd}[column sep=6em, row sep=6em]
		{\VarLinearSpace} \arrow{r}{\Adjoint{\LinearOperator}} \arrow{d}{G_{\LinearSpace}} & {\LinearSpace} \arrow{d}{F_{\LinearSpace}}\\
		{\Dual{\VarLinearSpace}} \arrow{r}{\Transposed{\LinearOperator}} & {\Dual{\LinearSpace}}
	\end{tikzcd}
	\caption{Enunciato del teorema \ref{GeometriaLineare_Teorema_TraspostaAggiunta}, diagramma 1.}
	\label{GeometriaLineare_Figura_TraspostaAggiunta}
\end{figure}
\Proof Fissiamo $\VarVector \in \VarLinearSpace$ e sia
$\Vector \in \LinearSpace$.
Abbiamo
\begin{itemize}
	\item $(F_{\LinearSpace} \circ \Adjoint{\LinearOperator})(\VarVector)(\Vector)
        = \InnerProduct{\Vector}{\Adjoint{\LinearOperator} \VarVector}
        = \InnerProduct{\LinearOperator \Vector}{\VarVector}$;
	\item $(\Transposed{\LinearOperator} \circ  G_{\VarLinearSpace})
          (\VarVector)(\Vector)
        = G_{\VarLinearSpace}(\VarVector)(\LinearOperator \Vector)
        = \InnerProduct{\LinearOperator \Vector}{\VarVector}$. \EndProof
\end{itemize}
\begin{Definition}
  Sia $\PrehilbertSpace$ uno spazio prehilbertiano con prodotto interno
  $\InnerProduct{\cdot}{\cdot}$.
  Sia inoltre
  $\HermitianOperator: \PrehilbertSpace \rightarrow \PrehilbertSpace$
  un operatore lineare autoaggiunto.
  $\HermitianOperator$ si dice
  \begin{itemize}
    \item \Define{hermitiano}[hermitiano][operatore] quando lo spazio
          prehilbertiano \`e complesso;
    \item \Define{simmetrico}[simmetrico][operatore] quando lo spazio
          prehilbertiano \`e reale.
    \item \Define{normale}[normale][operatore] quando
          $\HermitianOperator\Adjoint{\HermitianOperator}
            = \Adjoint{\HermitianOperator}\HermitianOperator$.
  \end{itemize}
\end{Definition}
\begin{Theorem}
  Sia $\PrehilbertSpace$ uno spazio prehilbertiano con prodotto interno
  $\InnerProduct{\cdot}{\cdot}$.
  Sia inoltre
  $\HermitianOperator: \PrehilbertSpace \rightarrow \PrehilbertSpace$
  un operatore lineare hermitiano.
  $\HermitianOperator$ \`e normale.
\end{Theorem}
\Proof Segue direttamente dalle definizioni. \EndProof
