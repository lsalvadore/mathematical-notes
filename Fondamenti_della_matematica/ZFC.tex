\subsection{Teoria di Zermelo e Fraenkel (ZF).}\label{ZFC}
\par La teoria di Zermelo e Fraenkel \`e la teoria degli insiemi di riferimento per ragioni storiche e didattiche, sebbene la nascita della teoria delle categorie abbia contribuito a rendere pi\`u popolare quella di von Neumann, Bernays e G\"odel. Si tratta di una teoria del primo ordine con uguaglianza che prevede solo il concetto di insieme, e non quello di classe: \`e sempre possibile parlare di classi, ma ci\`o deve essere fatto utilizzando complicate perifrasi il cui senso \`e ``tali termini non possono costituire un insieme'' (e dunque costituiscono una classe in NBG).
\par Come vedremo, molti assiomi sono comuni con le teorie precedenti, eventualmente restringendone la portata agli insiemi: per maggior chiarezza, riporteremo comunque esplicitamente tutti gli assiomi e le definizioni necessarie.
\begin{Definition}
	Ogni termine della teoria si chiama \Define{insieme}. Quando vale la relazione $\Set_1 \in \Set_2$ diciamo che l'insieme $\Set_1$ \Define{appartiene}[appartenenza] all'insieme $\Set_2$, che $\Set_1$ \`e \Define{elemento} di $\Set_2$ e che $\Set_2$ \Define{contiene}[contenimento] $\Set_1$. Denotiamo la relazione $\Not{\Set_1 \in \Set_2}$ col simbolo $\notin$.
\end{Definition}
\begin{Definition}
	Quando $(\Set_1 \in \Set_2) \Rightarrow (\Set_1 \in \Set_3)$ diciamo che $\Set_2$ \`e un \Define{sottoinsieme} di $\Set_3$, denotato $\Set_2 \subseteq \Set_3$ e che $\Set_3$ \`e un \Define{soprainsieme} di $\Set_2$, denotato $\Set_3 \supseteq \Set_2$; se $\Set_2 \neq \Set_3$, allora le notazioni diventano rispettivamente $\Set_2 \subset \Set_3$ e $\Set_3 \supset \Set_2$.
\end{Definition}
\begin{Axiom}
	\TheoremName{Assioma di estensionalit\`a (ZF)}[di estensionalit\`a (ZF)][assioma] Abbiamo $\Set_1 = \Set_2$ se e solo se $\Set_1 \subseteq \Set_2$ e $\Set_2 \subseteq \Set_1$.
\end{Axiom}
\par Per l'assioma di estensionalit\`a, dato un numero finito di insiemi, se esiste un insieme che li contiene tutti senza contenere nient'altro, allora esso \`e unico: lo denoteremo elencandone gli elementi tra parentesi graffe.
\begin{Axiom}
	\TheoremName{Schema di comprensione degli insiemi}[di comprensione degli insiemi][schema] Se $\phi(x)$ \`e una formula del primo ordine, allora per ogni insieme $\Set_0$ esiste un insieme $\Set$ tale che $x \in \Set \Leftrightarrow \And{x \in \Set_0}{\phi(x)}$.
\end{Axiom}
\par Dati una formula $\phi(x)$ e un insieme $\Set_0$ come nell'assioma precedente, denotiamo l'insieme $\Set$ con $\lbrace x \in \Set_0 | \phi(x) \rbrace$.
\begin{Definition}
	Se l'insieme $\Set$ verifica la relazione $(\forall x)(x \notin \Set)$ allora diciamo che $\Set$ \`e \Define{vuoto}[vuoto][insieme].
\end{Definition}
\begin{Axiom}
	Esiste un insieme vuoto $\emptyset$.
\end{Axiom}
\begin{Theorem}
	Sia $\Set_1$ un insieme vuoto e $\Set_2$ un insieme qualsiasi: abbiamo $\Set_1 \subseteq \Set_2$.
\end{Theorem}
\Proof Sia $x \in \Set_1$: poich\'e, per definizione di insieme vuoto, $x \notin \Set_1$, ne deduciamo $x \in \Set_2$. \EndProof
\begin{Corollary}
	$\emptyset$ \`e l'unico insieme vuoto.
\end{Corollary}
\Proof Se $\Set$ \`e un altro insieme vuoto, allora $\emptyset \subseteq \Set$ e $\Set \subseteq \emptyset$, da cui la tesi tramite l'assioma di estensionalit\`a. \EndProof
\begin{Definition}
	Se $\Set_1$ e $\Set_2$ sono due insiemi tali che non esista $x$ per cui  $x \in \Set_1$ e $x \in \Set_2$, allora si dice che $\Set_1$ e $\Set_2$ sono \Define{disgiunti}[disgiunti][insiemi].
\end{Definition}
\begin{Axiom}
	\TheoremName{Assioma di regolarit\`a (o di fondazione)}[di regolarit\`a (o di fondazione)][assioma] Per ogni insieme $\Set_0$ esiste un elemento $\Set_1 \in \Set_0$ tale che $\Set_0$ e $\Set_1$ sono disgiunti.
\end{Axiom}
\begin{Axiom}
	\TheoremName{Assioma dell'accoppiamento}[della coppia][assioma] Dati due insiemi $\Set_1$ e $\Set_2$ esiste l'insieme $\lbrace \Set_1, \Set_2 \rbrace$.
\end{Axiom}
\begin{Theorem}
	Se $\Set$ \`e un insieme, esiste l'insieme $\lbrace \Set \rbrace$.
\end{Theorem}
\Proof $\lbrace \Set \rbrace$ si ottiene dall'assioma di accoppiamento prendendo $\Set_1 = \Set_2 = \Set$. \EndProof
\begin{Definition}
	Dati due insiemi $\Set_1$, $\Set_2$, la \Define{coppia} $(\Set_1,\Set_2)$ \`e l'insieme $\lbrace \lbrace \Set_1 \rbrace, \lbrace \Set_1, \Set_2 \rbrace \rbrace$. Per qualsiasi numero naturale\footnote{Non abbiamo ancora definito formalmente il concetto di numero naturale: ci affidiamo qui al concetto intuitivo metamatematico di ``numero usato per contare''.} $n$ maggiore\footnote{Non avendo definito l'insieme dei numeri naturali, non abbiamo definito neanche un loro ordinamento, n\'e l'operazione di sottrazione di $1$ utilizzata poco pi\`u avanti: persistiamo nell'appellarci all'intuizione metamatematica.} di $2$, la \Define{$n$-upla} $(\Set_1, ..., \Set_{n - 1}, \Set_n)$ \`e definita come la coppia $((\Set_1, ..., \Set_{n - 1}), \Set_n)$. Una \Define{terna} \`e una $3$-upla, una \Define{quadrupla} una $4$-upla, una \Define{quintupla} una $5$-upla e cos\`i via. Gli elementi che appaiono in una $n$-upla si chiamano \Define{componenti}[componente].
\end{Definition}
\begin{Definition}
	Chiamiamo \Define{insieme funzionale}[funzionale][insieme] un insieme $\Function$ costituito da sole coppie tale che $\And{(x,y) \in \Function}{(x,z) \in \Function} \Rightarrow y = z$.
\end{Definition}
\begin{Axiom}
	\TheoremName{Assioma di rimpiazzamento} Se $\Function$ \`e un insieme funzionale e gli argomenti di $\Function$ costituiscono un insieme, allora anche le immagini di $\Function$ costituiscono un insieme, chiamato \Define{immagine}[immagine][insieme] di $\Function$ e denotato $\Image{\Function}$.
\end{Axiom}
\begin{Axiom}
	\TheoremName{Assioma dell'insieme delle parti}[dell'insieme delle parti][assioma] Dato un insieme $\Set$, esiste l'insieme $\lbrace x | x \subseteq \Set \rbrace$, d'ora in poi denotato $\PowerSet{\Set}$.
\end{Axiom}
\begin{Axiom}
	\TheoremName{Assioma dell'unione}[dell'unione][assioma] Dato un insieme $\Set$, esiste l'insieme $\lbrace x | (\exists y)(\And{y \in \Set}{x \in y}) \rbrace$, d'ora in poi denotato $\bigcup \Set$. Se $\Set$ ammette solo un numero finito di elementi $\Set_1, ..., \Set_n$ , allora denotiamo $\bigcup \Set$ anche con $\Set_1 \cup ... \cup \Set_n$.
\end{Axiom}
\begin{Definition}
	Un insieme $\Set$ si dice \Define{induttivo}[induttivo][insieme] quando
	\begin{itemize}
		\item $\emptyset \in \Set$;
		\item $x \in \Set \Rightarrow x \cup \lbrace x \rbrace \in \Set$.
	\end{itemize}
\end{Definition}
\begin{Axiom}
	\TheoremName{Assioma dell'infinito}[dell'infinito][assioma] Esiste un insieme induttivo.
\end{Axiom}
\par Presentiamo un ultimo assioma. Esso \`e oggetto di controversie: assumerlo vero porta a volte a risultati desiderabili, altre volte a risultati che appaiono controintuitivi. Per questo motivo, si evita il pi\`u possibile di ricorrere ad esso in una dimostrazione e, quando il suo uso \`e veramente inevitabile, ci\`o viene sottolineato.
\par Il seguente assioma viene spesso indicato con la sigla AC e, quando si suppone vero, alla sigla ZF che indica la teoria di Zermelo e Fraenkel si aggiunge una C ottenendo la sigla ZFC.
\begin{Axiom}
	\TheoremName{Assioma della scelta}[della scelta][assioma] Dato un insieme $\Set$ tale che $\emptyset \notin \Set$, esiste un insieme funzionale $\Function$ tale che
	\begin{itemize}
		\item gli argomenti di $\Function$ costituiscono l'insieme $\Set$;
		\item se $(x,y) \in \Function$, allora $x \in y$.
	\end{itemize}
\end{Axiom}
