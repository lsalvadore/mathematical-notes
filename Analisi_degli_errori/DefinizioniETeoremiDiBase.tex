\section{Definizioni e teoremi di base.}
\label{AnalisiDegliErrori_DefinizioniETeoremiDiBase}
\begin{Definition}
  Sia $\Distance$ una distanza su uno spazio metrico $\MetricSpace$.
	Sia $x \in \MetricSpace$ e supponiamo di voler approssimare $x$ con un altro
  punto $\tilde{x} \in \MetricSpace$.
  Chiamiamo la quantit\`a $\AbsoluteError = \Distance{\tilde{x}}{x}$
  \Define{errore assoluto}[assoluto][errore]
  o anche semplicemente
  \Define{errore},
  o ancora
  \Define{perturbazione}.
	\par Se la distanza $\Distance$ \`e indotta da una norma $\Norm{\cdot}$
  e $\Norm{x} \neq 0$, allora definiamo
  \Define{errore relativo}[relativo][errore]
  o
  \Define{perturbazione relativa}[relativa][perturbazione]
  la quantit\`a
  $\RelativeError = \frac{\Norm{\tilde{x} - x}}{\Norm{x}}$.
  \par Nel caso particolare in cui
  $\MetricSpace = \mathbb{R}$, si definiscono altres\`i
  \NIDefine{errore assoluto} e \NIDefine{errore relativo} le quantit\`a
  $\AbsoluteError = \tilde{x} - x$
  e
  $\RelativeError = \frac{\tilde{x} - x}{x}$
  rispettivamente.
\end{Definition}
\begin{Definition}
  Chiamiamo
  \Define{errore al primo ordine}[al primo ordine][errore]
  l'approssimazione di un errore $\Error$ ottenuta trascurando,
  in un'espressione di $\Error$ in dipendenza di altri errori,
  tutti gli errori di grado maggiore di $1$.
\end{Definition}
\par L'analisi al primo ordine \`e molto utile nella pratica perch\'e, nelle
applicazioni, si suppongono sempre gli errori molto piccoli, e dunque i prodotti
di errori sono ancora pi\`u piccoli, impercettibili dagli strumenti di
approssimazione, siano essi strumenti di misura, di calcolo o altri.
\begin{Definition}
  Siano
  \begin{itemize}
    \item $n \in \mathbb{N}$;
    \item $f: \mathbb{R}^n \rightarrow \mathbb{R}$;
    \item $X \in \MetricSpace$;
    \item $Y = f(X)$.
  \end{itemize}
  Fissata una perturbazione $\Perturbation{X} \in \mathbb{R}^n$ di $X$,
  chiamiamo
  \Define{errore in avanti}[in avanti][errore]
  la perturbazione $\Perturbation{Y}$ di $Y$ unica soluzione dell'equazione
  $f(X + \Perturbation{X}) = Y + \Perturbation{Y}$.
  Fissata una perturbazione $\Perturbation{Y} \in \mathbb{R}$ di $Y$,
  chiamiamo
  \Define{errore all'indietro}[all'indietro][errore]
  l'estremo inferiore della classe di tutte le perturbazioni
  $\Perturbation{X}$ soluzioni dell'equazione
  $f(X + \Perturbation{X}) = Y + \Perturbation{Y}$.
\end{Definition}
\begin{Definition}
  Siano
  \begin{itemize}
    \item $n \in \NotZero{\mathbb{N}}$;
    \item $x, \tilde{x} \in \mathbb{R}^n$;
    \item $f: \mathbb{R}^n \rightarrow \mathbb{R}$;
    \item $\tilde{f}: \mathbb{R}^n \rightarrow \mathbb{R}$.
  \end{itemize}
  Supponiamo $f(x) \neq 0$.
  Chiamiamo
  \begin{itemize}
    \item \Define{errore inerente}[inerente][errore]
      l'errore relativo $\InherentError$ di $f(x)$ approssimato con
      $f(\tilde{x})$:
      \[
        \InherentError = \frac{f(\tilde{x}) - f(x)}{f(x)};
      \]
    \item \Define{errore algoritmico}[algoritmico][errore]
      l'errore relativo $\AlgorithmicError$ di $f(x)$ approssimato con
      $\tilde{f(x)}$, restituito come dato in uscita da un algoritmo
      fissato per il calcolo della funzione $f$ passando in ingresso $x$:
      \[
        \AlgorithmicError = \frac{\tilde{f(x)} - f(x)}{f(x)};
      \]
    \item \Define{errore analitico}[analitico][errore]
      l'errore relativo $\AnalyticalError$ di $f(x)$ approssimato con
      $\tilde{f}(x)$:
      \[
        \AnalyticalError = \frac{\tilde{f}(x) - f(x)}{f(x)};
      \]
    \item \Define{errore totale}[totale][errore]
      l'errore relativo $\TotalError$ di $f(x)$ approssimato con
      $\overline{\tilde{f}(\tilde{x})}$, restituito come dato in uscita da un
      algoritmo fissato per il calcolo della funzione $\tilde{f}$ passando in 
      ingresso $\tilde{x}$:
      \[
        \TotalError = \frac{\overline{\tilde{f}(\tilde{x})} - f(x)}{f(x)}.
      \]
  \end{itemize}
\end{Definition}
\begin{Theorem}
  Supponendo
  \begin{itemize}
    \item $n \in \NotZero{\mathbb{N}}$
    \item $x, \tilde{x} \in \mathbb{R}^n$;
    \item $f: \mathbb{R}^n \rightarrow \mathbb{R}$;
    \item $\tilde{f}: \mathbb{R}^n \rightarrow \mathbb{R}$.
  \end{itemize}
  abbiamo, al primo ordine,
  \[
    \TotalError = \InherentError + \AlgorithmicError + \AnalyticalError,
  \]
  dove
  \begin{itemize}
    \item $\TotalError$ 
  \end{itemize}
\end{Theorem}
\Proof Abbiamo, al primo ordine,
\begin{align*}
  \TotalError
  &= \frac{\overline{\tilde{f}(\tilde{x})} - f(x)}{f(x)},\\
  &= \frac{\overline{\tilde{f}(\tilde{x})}}{f(x)} - 1,\\
  &= \frac{\overline{\tilde{f}(\tilde{x})}}{\tilde{f}(\tilde{x})}
      \frac{\tilde{f}(\tilde{x})}{f(\tilde{x})}
      \frac{f(\tilde{x})}{f(x)} - 1,\\
  &= (\AnalyticalError + 1)
      (\AlgorithmicError + 1)
      (\InherentError + 1) - 1,\\
  &= \AnalyticalError + \AlgorithmicError + \InherentError.\text{ \EndProof}
\end{align*}
